\documentclass[14pt]{extbook}
\usepackage{multicol, enumerate, enumitem, hyperref, color, soul, setspace, parskip, fancyhdr} %General Packages
\usepackage{amssymb, amsthm, amsmath, latexsym, units, mathtools} %Math Packages
\everymath{\displaystyle} %All math in Display Style
% Packages with additional options
\usepackage[headsep=0.5cm,headheight=12pt, left=1 in,right= 1 in,top= 1 in,bottom= 1 in]{geometry}
\usepackage[usenames,dvipsnames]{xcolor}
\usepackage{dashrule}  % Package to use the command below to create lines between items
\newcommand{\litem}[1]{\item#1\hspace*{-1cm}\rule{\textwidth}{0.4pt}}
\pagestyle{fancy}
\lhead{Progress Quiz 4}
\chead{}
\rhead{Version C}
\lfoot{5346-5907}
\cfoot{}
\rfoot{Summer C 2021}
\begin{document}

\begin{enumerate}
\litem{
Choose the equation of the function graphed below.
\begin{center}
    \includegraphics[width=0.5\textwidth]{../Figures/radicalGraphToEquationC.png}
\end{center}
\begin{enumerate}[label=\Alph*.]
\item \( f(x) = - \sqrt[3]{x - 14} - 5 \)
\item \( f(x) = - \sqrt[3]{x + 14} - 5 \)
\item \( f(x) = \sqrt[3]{x - 14} - 5 \)
\item \( f(x) = \sqrt[3]{x + 14} - 5 \)
\item \( \text{None of the above} \)

\end{enumerate} }
\litem{
Choose the graph of the equation below.\[ f(x) = - \sqrt[3]{x + 10} - 3 \]\begin{enumerate}[label=\Alph*.]
\begin{multicols}{2}\item \includegraphics[width = 0.3\textwidth]{../Figures/radicalEquationToGraphCopyAC.png}\item \includegraphics[width = 0.3\textwidth]{../Figures/radicalEquationToGraphCopyBC.png}\item \includegraphics[width = 0.3\textwidth]{../Figures/radicalEquationToGraphCopyCC.png}\item \includegraphics[width = 0.3\textwidth]{../Figures/radicalEquationToGraphCopyDC.png}\end{multicols}\item None of the above.
\end{enumerate} }
\litem{
Choose the graph of the equation below.\[ f(x) = \sqrt[3]{x - 12} + 5 \]\begin{enumerate}[label=\Alph*.]
\begin{multicols}{2}\item \includegraphics[width = 0.3\textwidth]{../Figures/radicalEquationToGraphAC.png}\item \includegraphics[width = 0.3\textwidth]{../Figures/radicalEquationToGraphBC.png}\item \includegraphics[width = 0.3\textwidth]{../Figures/radicalEquationToGraphCC.png}\item \includegraphics[width = 0.3\textwidth]{../Figures/radicalEquationToGraphDC.png}\end{multicols}\item None of the above.
\end{enumerate} }
\litem{
Solve the radical equation below. Then, choose the interval(s) that the solution(s) belongs to.\[ \sqrt{72 x^2 + 28} - \sqrt{-95 x} = 0 \]\begin{enumerate}[label=\Alph*.]
\item \( x \in [-1.08,-0.68] \)
\item \( x \in [-0.52,0.09] \)
\item \( x_1 \in [0.42, 0.48] \text{ and } x_2 \in [0.4,1.8] \)
\item \( x_1 \in [-1.08, -0.68] \text{ and } x_2 \in [-1,-0.4] \)
\item \( \text{All solutions lead to invalid or complex values in the equation.} \)

\end{enumerate} }
\litem{
Solve the radical equation below. Then, choose the interval(s) that the solution(s) belongs to.\[ \sqrt{-6 x - 7} - \sqrt{-3 x + 3} = 0 \]\begin{enumerate}[label=\Alph*.]
\item \( x_1 \in [-1.28, -0.57] \text{ and } x_2 \in [-1,6] \)
\item \( \text{All solutions lead to invalid or complex values in the equation.} \)
\item \( x \in [-3.86,-3.31] \)
\item \( x \in [-1.63,-1.32] \)
\item \( x_1 \in [-3.86, -3.31] \text{ and } x_2 \in [-4.17,-0.17] \)

\end{enumerate} }
\litem{
Choose the equation of the function graphed below.
\begin{center}
    \includegraphics[width=0.5\textwidth]{../Figures/radicalGraphToEquationCopyC.png}
\end{center}
\begin{enumerate}[label=\Alph*.]
\item \( f(x) = \sqrt[3]{x + 14} - 7 \)
\item \( f(x) = - \sqrt[3]{x - 14} - 7 \)
\item \( f(x) = \sqrt[3]{x - 14} - 7 \)
\item \( f(x) = - \sqrt[3]{x + 14} - 7 \)
\item \( \text{None of the above} \)

\end{enumerate} }
\litem{
Solve the radical equation below. Then, choose the interval(s) that the solution(s) belongs to.\[ \sqrt{36 x^2 - 42} - \sqrt{-6 x} = 0 \]\begin{enumerate}[label=\Alph*.]
\item \( \text{All solutions lead to invalid or complex values in the equation.} \)
\item \( x_1 \in [-1, 4] \text{ and } x_2 \in [1.08,1.41] \)
\item \( x \in [-1,4] \)
\item \( x \in [-5.17,0.83] \)
\item \( x_1 \in [-5.17, 0.83] \text{ and } x_2 \in [0.87,1.09] \)

\end{enumerate} }
\litem{
Solve the radical equation below. Then, choose the interval(s) that the solution(s) belongs to.\[ \sqrt{3 x + 9} - \sqrt{-8 x - 9} = 0 \]\begin{enumerate}[label=\Alph*.]
\item \( x \in [-0.5,2.1] \)
\item \( x \in [-2.7,-0.1] \)
\item \( x_1 \in [-5.4, -2.6] \text{ and } x_2 \in [-1.3,1.2] \)
\item \( x_1 \in [-5.4, -2.6] \text{ and } x_2 \in [-3.6,-1.5] \)
\item \( \text{All solutions lead to invalid or complex values in the equation.} \)

\end{enumerate} }
\litem{
What is the domain of the function below?\[ f(x) = \sqrt[6]{-6 x - 4} \]\begin{enumerate}[label=\Alph*.]
\item \( (-\infty, \infty) \)
\item \( [a, \infty), \text{where } a \in [-1.87, -0.88] \)
\item \( [a, \infty), \text{where } a \in [-1.11, -0.03] \)
\item \( (-\infty, a], \text{where } a \in [-2, -0.91] \)
\item \( (-\infty, a], \text{ where } a \in [-1.32, -0.62] \)

\end{enumerate} }
\litem{
What is the domain of the function below?\[ f(x) = \sqrt[4]{-6 x + 7} \]\begin{enumerate}[label=\Alph*.]
\item \( [a, \infty), \text{where } a \in [-0.42, 1.1] \)
\item \( (-\infty, a], \text{ where } a \in [0.98, 1.2] \)
\item \( (-\infty, a], \text{where } a \in [0.83, 0.88] \)
\item \( (-\infty, \infty) \)
\item \( [a, \infty), \text{where } a \in [0.94, 3.35] \)

\end{enumerate} }
\end{enumerate}

\end{document}