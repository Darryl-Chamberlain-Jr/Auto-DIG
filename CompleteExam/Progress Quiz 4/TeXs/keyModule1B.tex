\documentclass{extbook}[14pt]
\usepackage{multicol, enumerate, enumitem, hyperref, color, soul, setspace, parskip, fancyhdr, amssymb, amsthm, amsmath, latexsym, units, mathtools}
\everymath{\displaystyle}
\usepackage[headsep=0.5cm,headheight=0cm, left=1 in,right= 1 in,top= 1 in,bottom= 1 in]{geometry}
\usepackage{dashrule}  % Package to use the command below to create lines between items
\newcommand{\litem}[1]{\item #1

\rule{\textwidth}{0.4pt}}
\pagestyle{fancy}
\lhead{}
\chead{Answer Key for Progress Quiz 4 Version B}
\rhead{}
\lfoot{5346-5907}
\cfoot{}
\rfoot{Summer C 2021}
\begin{document}
\textbf{This key should allow you to understand why you choose the option you did (beyond just getting a question right or wrong). \href{https://xronos.clas.ufl.edu/mac1105spring2020/courseDescriptionAndMisc/Exams/LearningFromResults}{More instructions on how to use this key can be found here}.}

\textbf{If you have a suggestion to make the keys better, \href{https://forms.gle/CZkbZmPbC9XALEE88}{please fill out the short survey here}.}

\textit{Note: This key is auto-generated and may contain issues and/or errors. The keys are reviewed after each exam to ensure grading is done accurately. If there are issues (like duplicate options), they are noted in the offline gradebook. The keys are a work-in-progress to give students as many resources to improve as possible.}

\rule{\textwidth}{0.4pt}

\begin{enumerate}\litem{
Simplify the expression below into the form $a+bi$. Then, choose the intervals that $a$ and $b$ belong to.
\[ (-6 + 10 i)(5 + 8 i) \]The solution is \( -110 + 2 i \), which is option B.\begin{enumerate}[label=\Alph*.]
\item \( a \in [-31, -24] \text{ and } b \in [77, 82] \)

 $-30 + 80 i$, which corresponds to just multiplying the real terms to get the real part of the solution and the coefficients in the complex terms to get the complex part.
\item \( a \in [-113, -107] \text{ and } b \in [1, 8] \)

* $-110 + 2 i$, which is the correct option.
\item \( a \in [45, 54] \text{ and } b \in [93, 99] \)

 $50 + 98 i$, which corresponds to adding a minus sign in the second term.
\item \( a \in [-113, -107] \text{ and } b \in [-6, 1] \)

 $-110 - 2 i$, which corresponds to adding a minus sign in both terms.
\item \( a \in [45, 54] \text{ and } b \in [-98, -93] \)

 $50 - 98 i$, which corresponds to adding a minus sign in the first term.
\end{enumerate}

\textbf{General Comment:} You can treat $i$ as a variable and distribute. Just remember that $i^2=-1$, so you can continue to reduce after you distribute.
}
\litem{
Simplify the expression below and choose the interval the simplification is contained within.
\[ 20 - 3^2 + 14 \div 9 * 10 \div 17 \]The solution is \( 11.915 \), which is option C.\begin{enumerate}[label=\Alph*.]
\item \( [29.61, 30.26] \)

 29.915, which corresponds to an Order of Operations error: multiplying by negative before squaring. For example: $(-3)^2 \neq -3^2$
\item \( [28.47, 29.04] \)

 29.009, which corresponds to two Order of Operations errors.
\item \( [11.8, 12.05] \)

* 11.915, this is the correct option
\item \( [10.14, 11.46] \)

 11.009, which corresponds to an Order of Operations error: not reading left-to-right for multiplication/division.
\item \( \text{None of the above} \)

 You may have gotten this by making an unanticipated error. If you got a value that is not any of the others, please let the coordinator know so they can help you figure out what happened.
\end{enumerate}

\textbf{General Comment:} While you may remember (or were taught) PEMDAS is done in order, it is actually done as P/E/MD/AS. When we are at MD or AS, we read left to right.
}
\litem{
Choose the \textbf{smallest} set of Complex numbers that the number below belongs to.
\[ \frac{-11}{0}+\sqrt{221} i \]The solution is \( \text{Not a Complex Number} \), which is option D.\begin{enumerate}[label=\Alph*.]
\item \( \text{Rational} \)

These are numbers that can be written as fraction of Integers (e.g., -2/3 + 5)
\item \( \text{Nonreal Complex} \)

This is a Complex number $(a+bi)$ that is not Real (has $i$ as part of the number).
\item \( \text{Pure Imaginary} \)

This is a Complex number $(a+bi)$ that \textbf{only} has an imaginary part like $2i$.
\item \( \text{Not a Complex Number} \)

* This is the correct option!
\item \( \text{Irrational} \)

These cannot be written as a fraction of Integers. Remember: $\pi$ is not an Integer!
\end{enumerate}

\textbf{General Comment:} Be sure to simplify $i^2 = -1$. This may remove the imaginary portion for your number. If you are having trouble, you may want to look at the \textit{Subgroups of the Real Numbers} section.
}
\litem{
Choose the \textbf{smallest} set of Real numbers that the number below belongs to.
\[ \sqrt{\frac{33856}{529}} \]The solution is \( \text{Whole} \), which is option A.\begin{enumerate}[label=\Alph*.]
\item \( \text{Whole} \)

* This is the correct option!
\item \( \text{Not a Real number} \)

These are Nonreal Complex numbers \textbf{OR} things that are not numbers (e.g., dividing by 0).
\item \( \text{Integer} \)

These are the negative and positive counting numbers (..., -3, -2, -1, 0, 1, 2, 3, ...)
\item \( \text{Irrational} \)

These cannot be written as a fraction of Integers.
\item \( \text{Rational} \)

These are numbers that can be written as fraction of Integers (e.g., -2/3)
\end{enumerate}

\textbf{General Comment:} First, you \textbf{NEED} to simplify the expression. This question simplifies to $184$. 
 
 Be sure you look at the simplified fraction and not just the decimal expansion. Numbers such as 13, 17, and 19 provide \textbf{long but repeating/terminating decimal expansions!} 
 
 The only ways to *not* be a Real number are: dividing by 0 or taking the square root of a negative number. 
 
 Irrational numbers are more than just square root of 3: adding or subtracting values from square root of 3 is also irrational.
}
\litem{
Simplify the expression below and choose the interval the simplification is contained within.
\[ 16 - 6 \div 3 * 10 - (5 * 14) \]The solution is \( -74.000 \), which is option B.\begin{enumerate}[label=\Alph*.]
\item \( [-55.2, -51.2] \)

 -54.200, which corresponds to an Order of Operations error: not reading left-to-right for multiplication/division.
\item \( [-79, -73] \)

* -74.000, which is the correct option.
\item \( [83.8, 88.8] \)

 85.800, which corresponds to not distributing addition and subtraction correctly.
\item \( [-135, -117] \)

 -126.000, which corresponds to not distributing a negative correctly.
\item \( \text{None of the above} \)

 You may have gotten this by making an unanticipated error. If you got a value that is not any of the others, please let the coordinator know so they can help you figure out what happened.
\end{enumerate}

\textbf{General Comment:} While you may remember (or were taught) PEMDAS is done in order, it is actually done as P/E/MD/AS. When we are at MD or AS, we read left to right.
}
\litem{
Choose the \textbf{smallest} set of Real numbers that the number below belongs to.
\[ \sqrt{\frac{-1950}{15}} \]The solution is \( \text{Not a Real number} \), which is option E.\begin{enumerate}[label=\Alph*.]
\item \( \text{Whole} \)

These are the counting numbers with 0 (0, 1, 2, 3, ...)
\item \( \text{Irrational} \)

These cannot be written as a fraction of Integers.
\item \( \text{Rational} \)

These are numbers that can be written as fraction of Integers (e.g., -2/3)
\item \( \text{Integer} \)

These are the negative and positive counting numbers (..., -3, -2, -1, 0, 1, 2, 3, ...)
\item \( \text{Not a Real number} \)

* This is the correct option!
\end{enumerate}

\textbf{General Comment:} First, you \textbf{NEED} to simplify the expression. This question simplifies to $\sqrt{130} i$. 
 
 Be sure you look at the simplified fraction and not just the decimal expansion. Numbers such as 13, 17, and 19 provide \textbf{long but repeating/terminating decimal expansions!} 
 
 The only ways to *not* be a Real number are: dividing by 0 or taking the square root of a negative number. 
 
 Irrational numbers are more than just square root of 3: adding or subtracting values from square root of 3 is also irrational.
}
\litem{
Simplify the expression below into the form $a+bi$. Then, choose the intervals that $a$ and $b$ belong to.
\[ \frac{72 + 33 i}{-2 - 4 i} \]The solution is \( -13.80  + 11.10 i \), which is option B.\begin{enumerate}[label=\Alph*.]
\item \( a \in [-1.5, 1] \text{ and } b \in [-19, -16.5] \)

 $-0.60  - 17.70 i$, which corresponds to forgetting to multiply the conjugate by the numerator and not computing the conjugate correctly.
\item \( a \in [-14.5, -13] \text{ and } b \in [10, 12] \)

* $-13.80  + 11.10 i$, which is the correct option.
\item \( a \in [-14.5, -13] \text{ and } b \in [221.5, 223] \)

 $-13.80  + 222.00 i$, which corresponds to forgetting to multiply the conjugate by the numerator.
\item \( a \in [-277.5, -275.5] \text{ and } b \in [10, 12] \)

 $-276.00  + 11.10 i$, which corresponds to forgetting to multiply the conjugate by the numerator and using a plus instead of a minus in the denominator.
\item \( a \in [-37, -35] \text{ and } b \in [-9, -7.5] \)

 $-36.00  - 8.25 i$, which corresponds to just dividing the first term by the first term and the second by the second.
\end{enumerate}

\textbf{General Comment:} Multiply the numerator and denominator by the *conjugate* of the denominator, then simplify. For example, if we have $2+3i$, the conjugate is $2-3i$.
}
\litem{
Simplify the expression below into the form $a+bi$. Then, choose the intervals that $a$ and $b$ belong to.
\[ \frac{-45 - 66 i}{3 - i} \]The solution is \( -6.90  - 24.30 i \), which is option D.\begin{enumerate}[label=\Alph*.]
\item \( a \in [-17, -14.5] \text{ and } b \in [64.5, 66.5] \)

 $-15.00  + 66.00 i$, which corresponds to just dividing the first term by the first term and the second by the second.
\item \( a \in [-20.5, -19] \text{ and } b \in [-17, -14] \)

 $-20.10  - 15.30 i$, which corresponds to forgetting to multiply the conjugate by the numerator and not computing the conjugate correctly.
\item \( a \in [-69.5, -68.5] \text{ and } b \in [-24.5, -23] \)

 $-69.00  - 24.30 i$, which corresponds to forgetting to multiply the conjugate by the numerator and using a plus instead of a minus in the denominator.
\item \( a \in [-8.5, -5.5] \text{ and } b \in [-24.5, -23] \)

* $-6.90  - 24.30 i$, which is the correct option.
\item \( a \in [-8.5, -5.5] \text{ and } b \in [-244, -242.5] \)

 $-6.90  - 243.00 i$, which corresponds to forgetting to multiply the conjugate by the numerator.
\end{enumerate}

\textbf{General Comment:} Multiply the numerator and denominator by the *conjugate* of the denominator, then simplify. For example, if we have $2+3i$, the conjugate is $2-3i$.
}
\litem{
Simplify the expression below into the form $a+bi$. Then, choose the intervals that $a$ and $b$ belong to.
\[ (-5 - 9 i)(-7 - 10 i) \]The solution is \( -55 + 113 i \), which is option B.\begin{enumerate}[label=\Alph*.]
\item \( a \in [117, 133] \text{ and } b \in [-18, -10] \)

 $125 - 13 i$, which corresponds to adding a minus sign in the first term.
\item \( a \in [-55, -54] \text{ and } b \in [107, 117] \)

* $-55 + 113 i$, which is the correct option.
\item \( a \in [-55, -54] \text{ and } b \in [-115, -107] \)

 $-55 - 113 i$, which corresponds to adding a minus sign in both terms.
\item \( a \in [33, 41] \text{ and } b \in [84, 91] \)

 $35 + 90 i$, which corresponds to just multiplying the real terms to get the real part of the solution and the coefficients in the complex terms to get the complex part.
\item \( a \in [117, 133] \text{ and } b \in [12, 16] \)

 $125 + 13 i$, which corresponds to adding a minus sign in the second term.
\end{enumerate}

\textbf{General Comment:} You can treat $i$ as a variable and distribute. Just remember that $i^2=-1$, so you can continue to reduce after you distribute.
}
\litem{
Choose the \textbf{smallest} set of Complex numbers that the number below belongs to.
\[ \sqrt{\frac{-1664}{8}}+\sqrt{0}i \]The solution is \( \text{Pure Imaginary} \), which is option D.\begin{enumerate}[label=\Alph*.]
\item \( \text{Rational} \)

These are numbers that can be written as fraction of Integers (e.g., -2/3 + 5)
\item \( \text{Nonreal Complex} \)

This is a Complex number $(a+bi)$ that is not Real (has $i$ as part of the number).
\item \( \text{Not a Complex Number} \)

This is not a number. The only non-Complex number we know is dividing by 0 as this is not a number!
\item \( \text{Pure Imaginary} \)

* This is the correct option!
\item \( \text{Irrational} \)

These cannot be written as a fraction of Integers. Remember: $\pi$ is not an Integer!
\end{enumerate}

\textbf{General Comment:} Be sure to simplify $i^2 = -1$. This may remove the imaginary portion for your number. If you are having trouble, you may want to look at the \textit{Subgroups of the Real Numbers} section.
}
\end{enumerate}

\end{document}