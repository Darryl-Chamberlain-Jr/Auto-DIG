\documentclass{extbook}[14pt]
\usepackage{multicol, enumerate, enumitem, hyperref, color, soul, setspace, parskip, fancyhdr, amssymb, amsthm, amsmath, bbm, latexsym, units, mathtools}
\everymath{\displaystyle}
\usepackage[headsep=0.5cm,headheight=0cm, left=1 in,right= 1 in,top= 1 in,bottom= 1 in]{geometry}
\usepackage{dashrule}  % Package to use the command below to create lines between items
\newcommand{\litem}[1]{\item #1

\rule{\textwidth}{0.4pt}}
\pagestyle{fancy}
\lhead{}
\chead{Answer Key for Progress Quiz 4 Version B}
\rhead{}
\lfoot{9187-5854}
\cfoot{}
\rfoot{Spring 2021}
\begin{document}
\textbf{This key should allow you to understand why you choose the option you did (beyond just getting a question right or wrong). \href{https://xronos.clas.ufl.edu/mac1105spring2020/courseDescriptionAndMisc/Exams/LearningFromResults}{More instructions on how to use this key can be found here}.}

\textbf{If you have a suggestion to make the keys better, \href{https://forms.gle/CZkbZmPbC9XALEE88}{please fill out the short survey here}.}

\textit{Note: This key is auto-generated and may contain issues and/or errors. The keys are reviewed after each exam to ensure grading is done accurately. If there are issues (like duplicate options), they are noted in the offline gradebook. The keys are a work-in-progress to give students as many resources to improve as possible.}

\rule{\textwidth}{0.4pt}

\begin{enumerate}\litem{
Choose the \textbf{smallest} set of Complex numbers that the number below belongs to.
\[ \sqrt{\frac{1547}{13}}+10i^2 \]The solution is \( \text{Irrational} \), which is option C.\begin{enumerate}[label=\Alph*.]
\item \( \text{Rational} \)

These are numbers that can be written as fraction of Integers (e.g., -2/3 + 5)
\item \( \text{Nonreal Complex} \)

This is a Complex number $(a+bi)$ that is not Real (has $i$ as part of the number).
\item \( \text{Irrational} \)

* This is the correct option!
\item \( \text{Pure Imaginary} \)

This is a Complex number $(a+bi)$ that \textbf{only} has an imaginary part like $2i$.
\item \( \text{Not a Complex Number} \)

This is not a number. The only non-Complex number we know is dividing by 0 as this is not a number!
\end{enumerate}

\textbf{General Comment:} Be sure to simplify $i^2 = -1$. This may remove the imaginary portion for your number. If you are having trouble, you may want to look at the \textit{Subgroups of the Real Numbers} section.
}
\litem{
Choose the \textbf{smallest} set of Complex numbers that the number below belongs to.
\[ \sqrt{\frac{0}{14}}+\sqrt{10}i \]The solution is \( \text{Pure Imaginary} \), which is option E.\begin{enumerate}[label=\Alph*.]
\item \( \text{Rational} \)

These are numbers that can be written as fraction of Integers (e.g., -2/3 + 5)
\item \( \text{Not a Complex Number} \)

This is not a number. The only non-Complex number we know is dividing by 0 as this is not a number!
\item \( \text{Nonreal Complex} \)

This is a Complex number $(a+bi)$ that is not Real (has $i$ as part of the number).
\item \( \text{Irrational} \)

These cannot be written as a fraction of Integers. Remember: $\pi$ is not an Integer!
\item \( \text{Pure Imaginary} \)

* This is the correct option!
\end{enumerate}

\textbf{General Comment:} Be sure to simplify $i^2 = -1$. This may remove the imaginary portion for your number. If you are having trouble, you may want to look at the \textit{Subgroups of the Real Numbers} section.
}
\litem{
Simplify the expression below and choose the interval the simplification is contained within.
\[ 4 - 8^2 + 15 \div 5 * 10 \div 9 \]The solution is \( -56.667 \), which is option B.\begin{enumerate}[label=\Alph*.]
\item \( [66.4, 68.6] \)

 68.033, which corresponds to two Order of Operations errors.
\item \( [-57.8, -56.4] \)

* -56.667, this is the correct option
\item \( [-60, -58.8] \)

 -59.967, which corresponds to an Order of Operations error: not reading left-to-right for multiplication/division.
\item \( [69.1, 73.9] \)

 71.333, which corresponds to an Order of Operations error: multiplying by negative before squaring. For example: $(-3)^2 \neq -3^2$
\item \( \text{None of the above} \)

 You may have gotten this by making an unanticipated error. If you got a value that is not any of the others, please let the coordinator know so they can help you figure out what happened.
\end{enumerate}

\textbf{General Comment:} While you may remember (or were taught) PEMDAS is done in order, it is actually done as P/E/MD/AS. When we are at MD or AS, we read left to right.
}
\litem{
Simplify the expression below and choose the interval the simplification is contained within.
\[ 8 - 20^2 + 13 \div 18 * 6 \div 1 \]The solution is \( -387.667 \), which is option D.\begin{enumerate}[label=\Alph*.]
\item \( [-392.1, -390.4] \)

 -391.880, which corresponds to an Order of Operations error: not reading left-to-right for multiplication/division.
\item \( [404.2, 408.6] \)

 408.120, which corresponds to two Order of Operations errors.
\item \( [411.3, 413.3] \)

 412.333, which corresponds to an Order of Operations error: multiplying by negative before squaring. For example: $(-3)^2 \neq -3^2$
\item \( [-390.7, -387.2] \)

* -387.667, this is the correct option
\item \( \text{None of the above} \)

 You may have gotten this by making an unanticipated error. If you got a value that is not any of the others, please let the coordinator know so they can help you figure out what happened.
\end{enumerate}

\textbf{General Comment:} While you may remember (or were taught) PEMDAS is done in order, it is actually done as P/E/MD/AS. When we are at MD or AS, we read left to right.
}
\litem{
Simplify the expression below into the form $a+bi$. Then, choose the intervals that $a$ and $b$ belong to.
\[ (3 - 6 i)(5 + 4 i) \]The solution is \( 39 - 18 i \), which is option B.\begin{enumerate}[label=\Alph*.]
\item \( a \in [8, 16] \text{ and } b \in [-24, -23] \)

 $15 - 24 i$, which corresponds to just multiplying the real terms to get the real part of the solution and the coefficients in the complex terms to get the complex part.
\item \( a \in [34, 46] \text{ and } b \in [-22, -17] \)

* $39 - 18 i$, which is the correct option.
\item \( a \in [-9, -4] \text{ and } b \in [-42, -41] \)

 $-9 - 42 i$, which corresponds to adding a minus sign in the second term.
\item \( a \in [34, 46] \text{ and } b \in [17, 20] \)

 $39 + 18 i$, which corresponds to adding a minus sign in both terms.
\item \( a \in [-9, -4] \text{ and } b \in [42, 43] \)

 $-9 + 42 i$, which corresponds to adding a minus sign in the first term.
\end{enumerate}

\textbf{General Comment:} You can treat $i$ as a variable and distribute. Just remember that $i^2=-1$, so you can continue to reduce after you distribute.
}
\litem{
Choose the \textbf{smallest} set of Real numbers that the number below belongs to.
\[ \sqrt{\frac{190969}{529}} \]The solution is \( \text{Whole} \), which is option D.\begin{enumerate}[label=\Alph*.]
\item \( \text{Integer} \)

These are the negative and positive counting numbers (..., -3, -2, -1, 0, 1, 2, 3, ...)
\item \( \text{Rational} \)

These are numbers that can be written as fraction of Integers (e.g., -2/3)
\item \( \text{Not a Real number} \)

These are Nonreal Complex numbers \textbf{OR} things that are not numbers (e.g., dividing by 0).
\item \( \text{Whole} \)

* This is the correct option!
\item \( \text{Irrational} \)

These cannot be written as a fraction of Integers.
\end{enumerate}

\textbf{General Comment:} First, you \textbf{NEED} to simplify the expression. This question simplifies to $437$. 
 
 Be sure you look at the simplified fraction and not just the decimal expansion. Numbers such as 13, 17, and 19 provide \textbf{long but repeating/terminating decimal expansions!} 
 
 The only ways to *not* be a Real number are: dividing by 0 or taking the square root of a negative number. 
 
 Irrational numbers are more than just square root of 3: adding or subtracting values from square root of 3 is also irrational.
}
\litem{
Simplify the expression below into the form $a+bi$. Then, choose the intervals that $a$ and $b$ belong to.
\[ \frac{-54 + 55 i}{2 - 7 i} \]The solution is \( -9.30  - 5.06 i \), which is option C.\begin{enumerate}[label=\Alph*.]
\item \( a \in [-10.5, -9] \text{ and } b \in [-269, -267.5] \)

 $-9.30  - 268.00 i$, which corresponds to forgetting to multiply the conjugate by the numerator.
\item \( a \in [-28.5, -25.5] \text{ and } b \in [-8.5, -7] \)

 $-27.00  - 7.86 i$, which corresponds to just dividing the first term by the first term and the second by the second.
\item \( a \in [-10.5, -9] \text{ and } b \in [-6.5, -3.5] \)

* $-9.30  - 5.06 i$, which is the correct option.
\item \( a \in [-493.5, -491.5] \text{ and } b \in [-6.5, -3.5] \)

 $-493.00  - 5.06 i$, which corresponds to forgetting to multiply the conjugate by the numerator and using a plus instead of a minus in the denominator.
\item \( a \in [2.5, 6] \text{ and } b \in [8, 10] \)

 $5.23  + 9.21 i$, which corresponds to forgetting to multiply the conjugate by the numerator and not computing the conjugate correctly.
\end{enumerate}

\textbf{General Comment:} Multiply the numerator and denominator by the *conjugate* of the denominator, then simplify. For example, if we have $2+3i$, the conjugate is $2-3i$.
}
\litem{
Choose the \textbf{smallest} set of Real numbers that the number below belongs to.
\[ -\sqrt{\frac{48400}{100}} \]The solution is \( \text{Integer} \), which is option A.\begin{enumerate}[label=\Alph*.]
\item \( \text{Integer} \)

* This is the correct option!
\item \( \text{Rational} \)

These are numbers that can be written as fraction of Integers (e.g., -2/3)
\item \( \text{Not a Real number} \)

These are Nonreal Complex numbers \textbf{OR} things that are not numbers (e.g., dividing by 0).
\item \( \text{Whole} \)

These are the counting numbers with 0 (0, 1, 2, 3, ...)
\item \( \text{Irrational} \)

These cannot be written as a fraction of Integers.
\end{enumerate}

\textbf{General Comment:} First, you \textbf{NEED} to simplify the expression. This question simplifies to $-220$. 
 
 Be sure you look at the simplified fraction and not just the decimal expansion. Numbers such as 13, 17, and 19 provide \textbf{long but repeating/terminating decimal expansions!} 
 
 The only ways to *not* be a Real number are: dividing by 0 or taking the square root of a negative number. 
 
 Irrational numbers are more than just square root of 3: adding or subtracting values from square root of 3 is also irrational.
}
\litem{
Simplify the expression below into the form $a+bi$. Then, choose the intervals that $a$ and $b$ belong to.
\[ \frac{36 - 33 i}{-2 + i} \]The solution is \( -21.00  + 6.00 i \), which is option E.\begin{enumerate}[label=\Alph*.]
\item \( a \in [-22, -19.5] \text{ and } b \in [29.5, 30.5] \)

 $-21.00  + 30.00 i$, which corresponds to forgetting to multiply the conjugate by the numerator.
\item \( a \in [-19.5, -17] \text{ and } b \in [-34.5, -31] \)

 $-18.00  - 33.00 i$, which corresponds to just dividing the first term by the first term and the second by the second.
\item \( a \in [-8.5, -7] \text{ and } b \in [19, 21] \)

 $-7.80  + 20.40 i$, which corresponds to forgetting to multiply the conjugate by the numerator and not computing the conjugate correctly.
\item \( a \in [-105.5, -104] \text{ and } b \in [4.5, 7] \)

 $-105.00  + 6.00 i$, which corresponds to forgetting to multiply the conjugate by the numerator and using a plus instead of a minus in the denominator.
\item \( a \in [-22, -19.5] \text{ and } b \in [4.5, 7] \)

* $-21.00  + 6.00 i$, which is the correct option.
\end{enumerate}

\textbf{General Comment:} Multiply the numerator and denominator by the *conjugate* of the denominator, then simplify. For example, if we have $2+3i$, the conjugate is $2-3i$.
}
\litem{
Simplify the expression below into the form $a+bi$. Then, choose the intervals that $a$ and $b$ belong to.
\[ (4 - 7 i)(-10 + 3 i) \]The solution is \( -19 + 82 i \), which is option C.\begin{enumerate}[label=\Alph*.]
\item \( a \in [-41, -36] \text{ and } b \in [-22, -20] \)

 $-40 - 21 i$, which corresponds to just multiplying the real terms to get the real part of the solution and the coefficients in the complex terms to get the complex part.
\item \( a \in [-19, -16] \text{ and } b \in [-87, -74] \)

 $-19 - 82 i$, which corresponds to adding a minus sign in both terms.
\item \( a \in [-19, -16] \text{ and } b \in [72, 84] \)

* $-19 + 82 i$, which is the correct option.
\item \( a \in [-65, -57] \text{ and } b \in [58, 60] \)

 $-61 + 58 i$, which corresponds to adding a minus sign in the second term.
\item \( a \in [-65, -57] \text{ and } b \in [-59, -52] \)

 $-61 - 58 i$, which corresponds to adding a minus sign in the first term.
\end{enumerate}

\textbf{General Comment:} You can treat $i$ as a variable and distribute. Just remember that $i^2=-1$, so you can continue to reduce after you distribute.
}
\end{enumerate}

\end{document}