\documentclass[14pt]{extbook}
\usepackage{multicol, enumerate, enumitem, hyperref, color, soul, setspace, parskip, fancyhdr} %General Packages
\usepackage{amssymb, amsthm, amsmath, bbm, latexsym, units, mathtools} %Math Packages
\everymath{\displaystyle} %All math in Display Style
% Packages with additional options
\usepackage[headsep=0.5cm,headheight=12pt, left=1 in,right= 1 in,top= 1 in,bottom= 1 in]{geometry}
\usepackage[usenames,dvipsnames]{xcolor}
\usepackage{dashrule}  % Package to use the command below to create lines between items
\newcommand{\litem}[1]{\item#1\hspace*{-1cm}\rule{\textwidth}{0.4pt}}
\pagestyle{fancy}
\lhead{Progress Quiz 4}
\chead{}
\rhead{Version A}
\lfoot{9187-5854}
\cfoot{}
\rfoot{Spring 2021}
\begin{document}

\begin{enumerate}
\litem{
Choose the interval below that $f$ composed with $g$ at $x=-1$ is in.\[ f(x) = x^{3} +3 x^{2} +x -4 \text{ and } g(x) = -4x^{3} -4 x^{2} -4 x -3 \]\begin{enumerate}[label=\Alph*.]
\item \( (f \circ g)(-1) \in [7, 14] \)
\item \( (f \circ g)(-1) \in [0, 3] \)
\item \( (f \circ g)(-1) \in [69, 79] \)
\item \( (f \circ g)(-1) \in [76, 88] \)
\item \( \text{It is not possible to compose the two functions.} \)

\end{enumerate} }
\litem{
Find the inverse of the function below (if it exists). Then, evaluate the inverse at $x = 10$ and choose the interval that $f^{-1}(10)$ belongs to.\[ f(x) = 4 x^2 + 2 \]\begin{enumerate}[label=\Alph*.]
\item \( f^{-1}(10) \in [4.75, 5.55] \)
\item \( f^{-1}(10) \in [-0.04, 1.42] \)
\item \( f^{-1}(10) \in [2.38, 4.3] \)
\item \( f^{-1}(10) \in [1.59, 2.01] \)
\item \( \text{ The function is not invertible for all Real numbers. } \)

\end{enumerate} }
\litem{
Find the inverse of the function below (if it exists). Then, evaluate the inverse at $x = -10$ and choose the interval the $f^{-1}(-10)$ belongs to.\[ f(x) = \sqrt[3]{4 x - 3} \]\begin{enumerate}[label=\Alph*.]
\item \( f^{-1}(-10) \in [248.25, 250.25] \)
\item \( f^{-1}(-10) \in [250.75, 252.75] \)
\item \( f^{-1}(-10) \in [-249.25, -247.25] \)
\item \( f^{-1}(-10) \in [-252.75, -249.75] \)
\item \( \text{ The function is not invertible for all Real numbers. } \)

\end{enumerate} }
\litem{
Choose the interval below that $f$ composed with $g$ at $x=1$ is in.\[ f(x) = 3x^{3} +2 x^{2} -2 x \text{ and } g(x) = -3x^{3} +2 x^{2} +3 x \]\begin{enumerate}[label=\Alph*.]
\item \( (f \circ g)(1) \in [23, 26] \)
\item \( (f \circ g)(1) \in [28, 32] \)
\item \( (f \circ g)(1) \in [-54, -50] \)
\item \( (f \circ g)(1) \in [-65, -56] \)
\item \( \text{It is not possible to compose the two functions.} \)

\end{enumerate} }
\litem{
Determine whether the function below is 1-1.\[ f(x) = \sqrt{6 x - 20} \]\begin{enumerate}[label=\Alph*.]
\item \( \text{No, because the domain of the function is not $(-\infty, \infty)$.} \)
\item \( \text{No, because there is an $x$-value that goes to 2 different $y$-values.} \)
\item \( \text{No, because there is a $y$-value that goes to 2 different $x$-values.} \)
\item \( \text{Yes, the function is 1-1.} \)
\item \( \text{No, because the range of the function is not $(-\infty, \infty)$.} \)

\end{enumerate} }
\litem{
Find the inverse of the function below. Then, evaluate the inverse at $x = 8$ and choose the interval that $f^{-1}(8)$ belongs to.\[ f(x) = \ln{(x-3)}+2 \]\begin{enumerate}[label=\Alph*.]
\item \( f^{-1}(8) \in [397.43, 401.43] \)
\item \( f^{-1}(8) \in [22027.47, 22031.47] \)
\item \( f^{-1}(8) \in [144.41, 151.41] \)
\item \( f^{-1}(8) \in [59876.14, 59877.14] \)
\item \( f^{-1}(8) \in [403.43, 408.43] \)

\end{enumerate} }
\litem{
Find the inverse of the function below. Then, evaluate the inverse at $x = 6$ and choose the interval that $f^{-1}(6)$ belongs to.\[ f(x) = e^{x+4}-2 \]\begin{enumerate}[label=\Alph*.]
\item \( f^{-1}(6) \in [-2.99, -1.9] \)
\item \( f^{-1}(6) \in [5.92, 6.27] \)
\item \( f^{-1}(6) \in [0.1, 1.54] \)
\item \( f^{-1}(6) \in [-1.36, -0.9] \)
\item \( f^{-1}(6) \in [-0.78, -0.48] \)

\end{enumerate} }
\litem{
Add the following functions, then choose the domain of the resulting function from the list below.\[ f(x) = \sqrt{4x-17}  \text{ and } g(x) = 9x + 7 \]\begin{enumerate}[label=\Alph*.]
\item \( \text{ The domain is all Real numbers except } x = a, \text{ where } a \in [1.8, 6.8] \)
\item \( \text{ The domain is all Real numbers less than or equal to } x = a, \text{ where } a \in [-1.5, 5.5] \)
\item \( \text{ The domain is all Real numbers greater than or equal to } x = a, \text{ where } a \in [-4.75, 6.25] \)
\item \( \text{ The domain is all Real numbers except } x = a \text{ and } x = b, \text{ where } a \in [-9.83, -1.83] \text{ and } b \in [1.2, 9.2] \)
\item \( \text{ The domain is all Real numbers. } \)

\end{enumerate} }
\litem{
Determine whether the function below is 1-1.\[ f(x) = \sqrt{-6 x + 29} \]\begin{enumerate}[label=\Alph*.]
\item \( \text{Yes, the function is 1-1.} \)
\item \( \text{No, because there is an $x$-value that goes to 2 different $y$-values.} \)
\item \( \text{No, because the domain of the function is not $(-\infty, \infty)$.} \)
\item \( \text{No, because there is a $y$-value that goes to 2 different $x$-values.} \)
\item \( \text{No, because the range of the function is not $(-\infty, \infty)$.} \)

\end{enumerate} }
\litem{
Add the following functions, then choose the domain of the resulting function from the list below.\[ f(x) = 7x^{3} +9 x^{2} +5 x + 7 \text{ and } g(x) = 7x^{4} +9 x^{3} +4 x + 2 \]\begin{enumerate}[label=\Alph*.]
\item \( \text{ The domain is all Real numbers except } x = a, \text{ where } a \in [3.75, 8.75] \)
\item \( \text{ The domain is all Real numbers less than or equal to } x = a, \text{ where } a \in [-4, 0] \)
\item \( \text{ The domain is all Real numbers greater than or equal to } x = a, \text{ where } a \in [-6.67, -1.67] \)
\item \( \text{ The domain is all Real numbers except } x = a \text{ and } x = b, \text{ where } a \in [2.33, 7.33] \text{ and } b \in [-0.4, 8.6] \)
\item \( \text{ The domain is all Real numbers. } \)

\end{enumerate} }
\end{enumerate}

\end{document}