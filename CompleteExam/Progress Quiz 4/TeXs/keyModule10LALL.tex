\documentclass{extbook}[14pt]
\usepackage{multicol, enumerate, enumitem, hyperref, color, soul, setspace, parskip, fancyhdr, amssymb, amsthm, amsmath, latexsym, units, mathtools}
\everymath{\displaystyle}
\usepackage[headsep=0.5cm,headheight=0cm, left=1 in,right= 1 in,top= 1 in,bottom= 1 in]{geometry}
\usepackage{dashrule}  % Package to use the command below to create lines between items
\newcommand{\litem}[1]{\item #1

\rule{\textwidth}{0.4pt}}
\pagestyle{fancy}
\lhead{}
\chead{Answer Key for Progress Quiz 4 Version ALL}
\rhead{}
\lfoot{5346-5907}
\cfoot{}
\rfoot{Summer C 2021}
\begin{document}
\textbf{This key should allow you to understand why you choose the option you did (beyond just getting a question right or wrong). \href{https://xronos.clas.ufl.edu/mac1105spring2020/courseDescriptionAndMisc/Exams/LearningFromResults}{More instructions on how to use this key can be found here}.}

\textbf{If you have a suggestion to make the keys better, \href{https://forms.gle/CZkbZmPbC9XALEE88}{please fill out the short survey here}.}

\textit{Note: This key is auto-generated and may contain issues and/or errors. The keys are reviewed after each exam to ensure grading is done accurately. If there are issues (like duplicate options), they are noted in the offline gradebook. The keys are a work-in-progress to give students as many resources to improve as possible.}

\rule{\textwidth}{0.4pt}

\begin{enumerate}\litem{
Factor the polynomial below completely, knowing that $x + 3$ is a factor. Then, choose the intervals the zeros of the polynomial belong to, where $z_1 \leq z_2 \leq z_3 \leq z_4$. \textit{To make the problem easier, all zeros are between -5 and 5.}
\[ f(x) = 16x^{4} +16 x^{3} -105 x^{2} -9 x + 54 \]The solution is \( [-3, -0.75, 0.75, 2] \), which is option C.\begin{enumerate}[label=\Alph*.]
\item \( z_1 \in [-2.2, 0.1], \text{   }  z_2 \in [-1.19, 0.23], z_3 \in [0.4, 0.94], \text{   and   } z_4 \in [2.36, 3.36] \)

 Distractor 1: Corresponds to negatives of all zeros.
\item \( z_1 \in [-3.3, -2.7], \text{   }  z_2 \in [-2.54, -1.67], z_3 \in [0.02, 0.66], \text{   and   } z_4 \in [2.36, 3.36] \)

 Distractor 4: Corresponds to moving factors from one rational to another.
\item \( z_1 \in [-3.3, -2.7], \text{   }  z_2 \in [-1.19, 0.23], z_3 \in [0.4, 0.94], \text{   and   } z_4 \in [1.68, 2.69] \)

* This is the solution!
\item \( z_1 \in [-3.3, -2.7], \text{   }  z_2 \in [-1.63, -0.84], z_3 \in [1.24, 1.45], \text{   and   } z_4 \in [1.68, 2.69] \)

 Distractor 2: Corresponds to inversing rational roots.
\item \( z_1 \in [-2.2, 0.1], \text{   }  z_2 \in [-1.63, -0.84], z_3 \in [1.24, 1.45], \text{   and   } z_4 \in [2.36, 3.36] \)

 Distractor 3: Corresponds to negatives of all zeros AND inversing rational roots.
\end{enumerate}

\textbf{General Comment:} Remember to try the middle-most integers first as these normally are the zeros. Also, once you get it to a quadratic, you can use your other factoring techniques to finish factoring.
}
\litem{
Perform the division below. Then, find the intervals that correspond to the quotient in the form $ax^2+bx+c$ and remainder $r$.
\[ \frac{15x^{3} -97 x^{2} +168 x -77}{x -4} \]The solution is \( 15x^{2} -37 x + 20 + \frac{3}{x -4} \), which is option C.\begin{enumerate}[label=\Alph*.]
\item \( a \in [57, 68], \text{   } b \in [-337, -332], \text{   } c \in [1515, 1519], \text{   and   } r \in [-6142, -6136]. \)

 You divided by the opposite of the factor AND multiplied the first factor rather than just bringing it down.
\item \( a \in [13, 16], \text{   } b \in [-163, -155], \text{   } c \in [796, 797], \text{   and   } r \in [-3269, -3257]. \)

 You divided by the opposite of the factor.
\item \( a \in [13, 16], \text{   } b \in [-37, -35], \text{   } c \in [18, 23], \text{   and   } r \in [3, 4]. \)

* This is the solution!
\item \( a \in [13, 16], \text{   } b \in [-53, -47], \text{   } c \in [10, 17], \text{   and   } r \in [-45, -37]. \)

 You multiplied by the synthetic number and subtracted rather than adding during synthetic division.
\item \( a \in [57, 68], \text{   } b \in [142, 144], \text{   } c \in [737, 743], \text{   and   } r \in [2883, 2885]. \)

 You multiplied by the synthetic number rather than bringing the first factor down.
\end{enumerate}

\textbf{General Comment:} Be sure to synthetically divide by the zero of the denominator!
}
\litem{
Factor the polynomial below completely. Then, choose the intervals the zeros of the polynomial belong to, where $z_1 \leq z_2 \leq z_3$. \textit{To make the problem easier, all zeros are between -5 and 5.}
\[ f(x) = 25x^{3} +75 x^{2} +56 x + 12 \]The solution is \( [-2, -0.6, -0.4] \), which is option E.\begin{enumerate}[label=\Alph*.]
\item \( z_1 \in [-0.2, 0.2], \text{   }  z_2 \in [0.9, 2.6], \text{   and   } z_3 \in [2.78, 3.85] \)

 Distractor 4: Corresponds to moving factors from one rational to another.
\item \( z_1 \in [1.49, 1.78], \text{   }  z_2 \in [0.9, 2.6], \text{   and   } z_3 \in [2.36, 2.61] \)

 Distractor 3: Corresponds to negatives of all zeros AND inversing rational roots.
\item \( z_1 \in [-2.59, -2.42], \text{   }  z_2 \in [-3.5, -1.7], \text{   and   } z_3 \in [-2.5, -1.58] \)

 Distractor 2: Corresponds to inversing rational roots.
\item \( z_1 \in [0.33, 0.79], \text{   }  z_2 \in [0, 1.3], \text{   and   } z_3 \in [1.92, 2.46] \)

 Distractor 1: Corresponds to negatives of all zeros.
\item \( z_1 \in [-2.08, -1.76], \text{   }  z_2 \in [-0.8, 0.1], \text{   and   } z_3 \in [-0.43, -0.12] \)

* This is the solution!
\end{enumerate}

\textbf{General Comment:} Remember to try the middle-most integers first as these normally are the zeros. Also, once you get it to a quadratic, you can use your other factoring techniques to finish factoring.
}
\litem{
What are the \textit{possible Integer} roots of the polynomial below?
\[ f(x) = 6x^{4} +3 x^{3} +2 x^{2} +2 x + 4 \]The solution is \( \pm 1,\pm 2,\pm 4 \), which is option D.\begin{enumerate}[label=\Alph*.]
\item \( \text{ All combinations of: }\frac{\pm 1,\pm 2,\pm 4}{\pm 1,\pm 2,\pm 3,\pm 6} \)

This would have been the solution \textbf{if asked for the possible Rational roots}!
\item \( \text{ All combinations of: }\frac{\pm 1,\pm 2,\pm 3,\pm 6}{\pm 1,\pm 2,\pm 4} \)

 Distractor 3: Corresponds to the plus or minus of the inverse quotient (an/a0) of the factors. 
\item \( \pm 1,\pm 2,\pm 3,\pm 6 \)

 Distractor 1: Corresponds to the plus or minus factors of a1 only.
\item \( \pm 1,\pm 2,\pm 4 \)

* This is the solution \textbf{since we asked for the possible Integer roots}!
\item \( \text{There is no formula or theorem that tells us all possible Integer roots.} \)

 Distractor 4: Corresponds to not recognizing Integers as a subset of Rationals.
\end{enumerate}

\textbf{General Comment:} We have a way to find the possible Rational roots. The possible Integer roots are the Integers in this list.
}
\litem{
Factor the polynomial below completely. Then, choose the intervals the zeros of the polynomial belong to, where $z_1 \leq z_2 \leq z_3$. \textit{To make the problem easier, all zeros are between -5 and 5.}
\[ f(x) = 15x^{3} +56 x^{2} -105 x -50 \]The solution is \( [-5, -0.4, 1.67] \), which is option A.\begin{enumerate}[label=\Alph*.]
\item \( z_1 \in [-6, -4], \text{   }  z_2 \in [-0.81, -0.19], \text{   and   } z_3 \in [0.8, 1.8] \)

* This is the solution!
\item \( z_1 \in [-1.67, -0.67], \text{   }  z_2 \in [0.19, 0.57], \text{   and   } z_3 \in [4.9, 5.8] \)

 Distractor 1: Corresponds to negatives of all zeros.
\item \( z_1 \in [-6, -4], \text{   }  z_2 \in [-2.86, -2.03], \text{   and   } z_3 \in [-0.1, 1.3] \)

 Distractor 2: Corresponds to inversing rational roots.
\item \( z_1 \in [-1.6, 1.4], \text{   }  z_2 \in [2.21, 2.72], \text{   and   } z_3 \in [4.9, 5.8] \)

 Distractor 3: Corresponds to negatives of all zeros AND inversing rational roots.
\item \( z_1 \in [-6, -4], \text{   }  z_2 \in [-0.06, 0.29], \text{   and   } z_3 \in [4.9, 5.8] \)

 Distractor 4: Corresponds to moving factors from one rational to another.
\end{enumerate}

\textbf{General Comment:} Remember to try the middle-most integers first as these normally are the zeros. Also, once you get it to a quadratic, you can use your other factoring techniques to finish factoring.
}
\litem{
Perform the division below. Then, find the intervals that correspond to the quotient in the form $ax^2+bx+c$ and remainder $r$.
\[ \frac{4x^{3} -48 x + 66}{x + 4} \]The solution is \( 4x^{2} -16 x + 16 + \frac{2}{x + 4} \), which is option D.\begin{enumerate}[label=\Alph*.]
\item \( a \in [-20, -15], b \in [61, 73], c \in [-308, -302], \text{ and } r \in [1278, 1285]. \)

 You multipled by the synthetic number rather than bringing the first factor down.
\item \( a \in [4, 7], b \in [16, 17], c \in [14, 22], \text{ and } r \in [129, 137]. \)

 You divided by the opposite of the factor.
\item \( a \in [-20, -15], b \in [-68, -57], c \in [-308, -302], \text{ and } r \in [-1157, -1149]. \)

 You divided by the opposite of the factor AND multipled the first factor rather than just bringing it down.
\item \( a \in [4, 7], b \in [-16, -10], c \in [14, 22], \text{ and } r \in [2, 5]. \)

* This is the solution!
\item \( a \in [4, 7], b \in [-27, -17], c \in [50, 55], \text{ and } r \in [-197, -190]. \)

 You multipled by the synthetic number and subtracted rather than adding during synthetic division.
\end{enumerate}

\textbf{General Comment:} Be sure to synthetically divide by the zero of the denominator! Also, make sure to include 0 placeholders for missing terms.
}
\litem{
Factor the polynomial below completely, knowing that $x + 2$ is a factor. Then, choose the intervals the zeros of the polynomial belong to, where $z_1 \leq z_2 \leq z_3 \leq z_4$. \textit{To make the problem easier, all zeros are between -5 and 5.}
\[ f(x) = 12x^{4} -43 x^{3} -21 x^{2} +166 x -120 \]The solution is \( [-2, 1.25, 1.333, 3] \), which is option E.\begin{enumerate}[label=\Alph*.]
\item \( z_1 \in [-3.47, -2.46], \text{   }  z_2 \in [-1.71, -1], z_3 \in [-1.56, -1.13], \text{   and   } z_4 \in [1.5, 2.6] \)

 Distractor 1: Corresponds to negatives of all zeros.
\item \( z_1 \in [-2.22, -1.47], \text{   }  z_2 \in [0.11, 1.16], z_3 \in [0.44, 1.05], \text{   and   } z_4 \in [2.7, 3.2] \)

 Distractor 2: Corresponds to inversing rational roots.
\item \( z_1 \in [-4.53, -3.35], \text{   }  z_2 \in [-3.93, -2.74], z_3 \in [-0.56, -0.3], \text{   and   } z_4 \in [1.5, 2.6] \)

 Distractor 4: Corresponds to moving factors from one rational to another.
\item \( z_1 \in [-3.47, -2.46], \text{   }  z_2 \in [-1.16, -0.7], z_3 \in [-0.93, -0.52], \text{   and   } z_4 \in [1.5, 2.6] \)

 Distractor 3: Corresponds to negatives of all zeros AND inversing rational roots.
\item \( z_1 \in [-2.22, -1.47], \text{   }  z_2 \in [1.05, 1.59], z_3 \in [1.32, 1.66], \text{   and   } z_4 \in [2.7, 3.2] \)

* This is the solution!
\end{enumerate}

\textbf{General Comment:} Remember to try the middle-most integers first as these normally are the zeros. Also, once you get it to a quadratic, you can use your other factoring techniques to finish factoring.
}
\litem{
What are the \textit{possible Rational} roots of the polynomial below?
\[ f(x) = 4x^{3} +3 x^{2} +7 x + 6 \]The solution is \( \text{ All combinations of: }\frac{\pm 1,\pm 2,\pm 3,\pm 6}{\pm 1,\pm 2,\pm 4} \), which is option D.\begin{enumerate}[label=\Alph*.]
\item \( \pm 1,\pm 2,\pm 4 \)

 Distractor 1: Corresponds to the plus or minus factors of a1 only.
\item \( \text{ All combinations of: }\frac{\pm 1,\pm 2,\pm 4}{\pm 1,\pm 2,\pm 3,\pm 6} \)

 Distractor 3: Corresponds to the plus or minus of the inverse quotient (an/a0) of the factors. 
\item \( \pm 1,\pm 2,\pm 3,\pm 6 \)

This would have been the solution \textbf{if asked for the possible Integer roots}!
\item \( \text{ All combinations of: }\frac{\pm 1,\pm 2,\pm 3,\pm 6}{\pm 1,\pm 2,\pm 4} \)

* This is the solution \textbf{since we asked for the possible Rational roots}!
\item \( \text{ There is no formula or theorem that tells us all possible Rational roots.} \)

 Distractor 4: Corresponds to not recalling the theorem for rational roots of a polynomial.
\end{enumerate}

\textbf{General Comment:} We have a way to find the possible Rational roots. The possible Integer roots are the Integers in this list.
}
\litem{
Perform the division below. Then, find the intervals that correspond to the quotient in the form $ax^2+bx+c$ and remainder $r$.
\[ \frac{6x^{3} +4 x^{2} -34 x + 28}{x + 3} \]The solution is \( 6x^{2} -14 x + 8 + \frac{4}{x + 3} \), which is option C.\begin{enumerate}[label=\Alph*.]
\item \( a \in [-20, -9], \text{   } b \in [56, 65], \text{   } c \in [-213, -206], \text{   and   } r \in [649, 653]. \)

 You multiplied by the synthetic number rather than bringing the first factor down.
\item \( a \in [6, 10], \text{   } b \in [-20, -19], \text{   } c \in [43, 47], \text{   and   } r \in [-158, -148]. \)

 You multiplied by the synthetic number and subtracted rather than adding during synthetic division.
\item \( a \in [6, 10], \text{   } b \in [-17, -8], \text{   } c \in [2, 9], \text{   and   } r \in [-1, 6]. \)

* This is the solution!
\item \( a \in [6, 10], \text{   } b \in [18, 26], \text{   } c \in [32, 36], \text{   and   } r \in [119, 127]. \)

 You divided by the opposite of the factor.
\item \( a \in [-20, -9], \text{   } b \in [-51, -46], \text{   } c \in [-186, -177], \text{   and   } r \in [-526, -517]. \)

 You divided by the opposite of the factor AND multiplied the first factor rather than just bringing it down.
\end{enumerate}

\textbf{General Comment:} Be sure to synthetically divide by the zero of the denominator!
}
\litem{
Perform the division below. Then, find the intervals that correspond to the quotient in the form $ax^2+bx+c$ and remainder $r$.
\[ \frac{20x^{3} +63 x^{2} -24}{x + 3} \]The solution is \( 20x^{2} +3 x -9 + \frac{3}{x + 3} \), which is option B.\begin{enumerate}[label=\Alph*.]
\item \( a \in [-67, -57], b \in [239, 245], c \in [-730, -721], \text{ and } r \in [2163, 2165]. \)

 You multipled by the synthetic number rather than bringing the first factor down.
\item \( a \in [20, 26], b \in [3, 7], c \in [-10, -7], \text{ and } r \in [0, 11]. \)

* This is the solution!
\item \( a \in [20, 26], b \in [-17, -14], c \in [62, 70], \text{ and } r \in [-303, -294]. \)

 You multipled by the synthetic number and subtracted rather than adding during synthetic division.
\item \( a \in [-67, -57], b \in [-119, -114], c \in [-354, -349], \text{ and } r \in [-1085, -1073]. \)

 You divided by the opposite of the factor AND multipled the first factor rather than just bringing it down.
\item \( a \in [20, 26], b \in [119, 127], c \in [364, 371], \text{ and } r \in [1080, 1086]. \)

 You divided by the opposite of the factor.
\end{enumerate}

\textbf{General Comment:} Be sure to synthetically divide by the zero of the denominator! Also, make sure to include 0 placeholders for missing terms.
}
\litem{
Factor the polynomial below completely, knowing that $x -2$ is a factor. Then, choose the intervals the zeros of the polynomial belong to, where $z_1 \leq z_2 \leq z_3 \leq z_4$. \textit{To make the problem easier, all zeros are between -5 and 5.}
\[ f(x) = 20x^{4} -143 x^{3} +212 x^{2} +33 x -90 \]The solution is \( [-0.6, 0.75, 2, 5] \), which is option C.\begin{enumerate}[label=\Alph*.]
\item \( z_1 \in [-2, -1.6], \text{   }  z_2 \in [0.98, 1.85], z_3 \in [1.93, 2.08], \text{   and   } z_4 \in [4.45, 5.68] \)

 Distractor 2: Corresponds to inversing rational roots.
\item \( z_1 \in [-5.3, -3.4], \text{   }  z_2 \in [-2.67, -1.65], z_3 \in [-1.14, -0.62], \text{   and   } z_4 \in [-0.5, 1.45] \)

 Distractor 1: Corresponds to negatives of all zeros.
\item \( z_1 \in [-1.5, 0.4], \text{   }  z_2 \in [0.49, 1.03], z_3 \in [1.93, 2.08], \text{   and   } z_4 \in [4.45, 5.68] \)

* This is the solution!
\item \( z_1 \in [-5.3, -3.4], \text{   }  z_2 \in [-2.67, -1.65], z_3 \in [-0.68, 0.13], \text{   and   } z_4 \in [2.83, 3.55] \)

 Distractor 4: Corresponds to moving factors from one rational to another.
\item \( z_1 \in [-5.3, -3.4], \text{   }  z_2 \in [-2.67, -1.65], z_3 \in [-1.51, -1.1], \text{   and   } z_4 \in [0.65, 2.43] \)

 Distractor 3: Corresponds to negatives of all zeros AND inversing rational roots.
\end{enumerate}

\textbf{General Comment:} Remember to try the middle-most integers first as these normally are the zeros. Also, once you get it to a quadratic, you can use your other factoring techniques to finish factoring.
}
\litem{
Perform the division below. Then, find the intervals that correspond to the quotient in the form $ax^2+bx+c$ and remainder $r$.
\[ \frac{10x^{3} -64 x^{2} +74 x -25}{x -5} \]The solution is \( 10x^{2} -14 x + 4 + \frac{-5}{x -5} \), which is option A.\begin{enumerate}[label=\Alph*.]
\item \( a \in [9, 14], \text{   } b \in [-15, -6], \text{   } c \in [2, 5], \text{   and   } r \in [-9, -1]. \)

* This is the solution!
\item \( a \in [9, 14], \text{   } b \in [-119, -108], \text{   } c \in [642, 645], \text{   and   } r \in [-3249, -3242]. \)

 You divided by the opposite of the factor.
\item \( a \in [43, 53], \text{   } b \in [-320, -306], \text{   } c \in [1641, 1645], \text{   and   } r \in [-8246, -8238]. \)

 You divided by the opposite of the factor AND multiplied the first factor rather than just bringing it down.
\item \( a \in [43, 53], \text{   } b \in [180, 194], \text{   } c \in [1000, 1007], \text{   and   } r \in [4994, 4996]. \)

 You multiplied by the synthetic number rather than bringing the first factor down.
\item \( a \in [9, 14], \text{   } b \in [-32, -17], \text{   } c \in [-24, -21], \text{   and   } r \in [-119, -110]. \)

 You multiplied by the synthetic number and subtracted rather than adding during synthetic division.
\end{enumerate}

\textbf{General Comment:} Be sure to synthetically divide by the zero of the denominator!
}
\litem{
Factor the polynomial below completely. Then, choose the intervals the zeros of the polynomial belong to, where $z_1 \leq z_2 \leq z_3$. \textit{To make the problem easier, all zeros are between -5 and 5.}
\[ f(x) = 16x^{3} +40 x^{2} +x -30 \]The solution is \( [-2, -1.25, 0.75] \), which is option E.\begin{enumerate}[label=\Alph*.]
\item \( z_1 \in [-0.65, -0.05], \text{   }  z_2 \in [1.8, 2.3], \text{   and   } z_3 \in [4.98, 5.07] \)

 Distractor 4: Corresponds to moving factors from one rational to another.
\item \( z_1 \in [-2.18, -1.53], \text{   }  z_2 \in [-0.83, -0.62], \text{   and   } z_3 \in [1.25, 1.67] \)

 Distractor 2: Corresponds to inversing rational roots.
\item \( z_1 \in [-1.67, -1.19], \text{   }  z_2 \in [0.74, 1.09], \text{   and   } z_3 \in [1.9, 2.6] \)

 Distractor 3: Corresponds to negatives of all zeros AND inversing rational roots.
\item \( z_1 \in [-0.91, -0.62], \text{   }  z_2 \in [1.04, 1.44], \text{   and   } z_3 \in [1.9, 2.6] \)

 Distractor 1: Corresponds to negatives of all zeros.
\item \( z_1 \in [-2.18, -1.53], \text{   }  z_2 \in [-1.34, -1.21], \text{   and   } z_3 \in [0.49, 0.83] \)

* This is the solution!
\end{enumerate}

\textbf{General Comment:} Remember to try the middle-most integers first as these normally are the zeros. Also, once you get it to a quadratic, you can use your other factoring techniques to finish factoring.
}
\litem{
What are the \textit{possible Integer} roots of the polynomial below?
\[ f(x) = 7x^{3} +3 x^{2} +4 x + 4 \]The solution is \( \pm 1,\pm 2,\pm 4 \), which is option A.\begin{enumerate}[label=\Alph*.]
\item \( \pm 1,\pm 2,\pm 4 \)

* This is the solution \textbf{since we asked for the possible Integer roots}!
\item \( \text{ All combinations of: }\frac{\pm 1,\pm 2,\pm 4}{\pm 1,\pm 7} \)

This would have been the solution \textbf{if asked for the possible Rational roots}!
\item \( \text{ All combinations of: }\frac{\pm 1,\pm 7}{\pm 1,\pm 2,\pm 4} \)

 Distractor 3: Corresponds to the plus or minus of the inverse quotient (an/a0) of the factors. 
\item \( \pm 1,\pm 7 \)

 Distractor 1: Corresponds to the plus or minus factors of a1 only.
\item \( \text{There is no formula or theorem that tells us all possible Integer roots.} \)

 Distractor 4: Corresponds to not recognizing Integers as a subset of Rationals.
\end{enumerate}

\textbf{General Comment:} We have a way to find the possible Rational roots. The possible Integer roots are the Integers in this list.
}
\litem{
Factor the polynomial below completely. Then, choose the intervals the zeros of the polynomial belong to, where $z_1 \leq z_2 \leq z_3$. \textit{To make the problem easier, all zeros are between -5 and 5.}
\[ f(x) = 15x^{3} +71 x^{2} +32 x -48 \]The solution is \( [-4, -1.33, 0.6] \), which is option E.\begin{enumerate}[label=\Alph*.]
\item \( z_1 \in [-0.31, 0], \text{   }  z_2 \in [3.4, 5.2], \text{   and   } z_3 \in [2.8, 5.1] \)

 Distractor 4: Corresponds to moving factors from one rational to another.
\item \( z_1 \in [-1.74, -1.25], \text{   }  z_2 \in [0, 1.2], \text{   and   } z_3 \in [2.8, 5.1] \)

 Distractor 3: Corresponds to negatives of all zeros AND inversing rational roots.
\item \( z_1 \in [-4.13, -3.86], \text{   }  z_2 \in [-1.2, -0.2], \text{   and   } z_3 \in [1.4, 2.1] \)

 Distractor 2: Corresponds to inversing rational roots.
\item \( z_1 \in [-0.81, -0.46], \text{   }  z_2 \in [0.9, 2.3], \text{   and   } z_3 \in [2.8, 5.1] \)

 Distractor 1: Corresponds to negatives of all zeros.
\item \( z_1 \in [-4.13, -3.86], \text{   }  z_2 \in [-2.6, -1.1], \text{   and   } z_3 \in [0.3, 1.1] \)

* This is the solution!
\end{enumerate}

\textbf{General Comment:} Remember to try the middle-most integers first as these normally are the zeros. Also, once you get it to a quadratic, you can use your other factoring techniques to finish factoring.
}
\litem{
Perform the division below. Then, find the intervals that correspond to the quotient in the form $ax^2+bx+c$ and remainder $r$.
\[ \frac{20x^{3} +105 x^{2} -122}{x + 5} \]The solution is \( 20x^{2} +5 x -25 + \frac{3}{x + 5} \), which is option C.\begin{enumerate}[label=\Alph*.]
\item \( a \in [-103, -96], b \in [598, 607], c \in [-3033, -3021], \text{ and } r \in [15001, 15007]. \)

 You multipled by the synthetic number rather than bringing the first factor down.
\item \( a \in [17, 23], b \in [200, 209], c \in [1024, 1033], \text{ and } r \in [4999, 5011]. \)

 You divided by the opposite of the factor.
\item \( a \in [17, 23], b \in [-2, 9], c \in [-25, -22], \text{ and } r \in [-2, 10]. \)

* This is the solution!
\item \( a \in [-103, -96], b \in [-400, -393], c \in [-1977, -1970], \text{ and } r \in [-10001, -9989]. \)

 You divided by the opposite of the factor AND multipled the first factor rather than just bringing it down.
\item \( a \in [17, 23], b \in [-15, -14], c \in [88, 95], \text{ and } r \in [-665, -656]. \)

 You multipled by the synthetic number and subtracted rather than adding during synthetic division.
\end{enumerate}

\textbf{General Comment:} Be sure to synthetically divide by the zero of the denominator! Also, make sure to include 0 placeholders for missing terms.
}
\litem{
Factor the polynomial below completely, knowing that $x -4$ is a factor. Then, choose the intervals the zeros of the polynomial belong to, where $z_1 \leq z_2 \leq z_3 \leq z_4$. \textit{To make the problem easier, all zeros are between -5 and 5.}
\[ f(x) = 8x^{4} +14 x^{3} -163 x^{2} -129 x + 180 \]The solution is \( [-5, -1.5, 0.75, 4] \), which is option C.\begin{enumerate}[label=\Alph*.]
\item \( z_1 \in [-4.9, -3.6], \text{   }  z_2 \in [-3.06, -2.98], z_3 \in [0.27, 0.47], \text{   and   } z_4 \in [4.4, 6.6] \)

 Distractor 4: Corresponds to moving factors from one rational to another.
\item \( z_1 \in [-4.9, -3.6], \text{   }  z_2 \in [-1.49, -1.28], z_3 \in [0.66, 0.7], \text{   and   } z_4 \in [4.4, 6.6] \)

 Distractor 3: Corresponds to negatives of all zeros AND inversing rational roots.
\item \( z_1 \in [-5.6, -4.8], \text{   }  z_2 \in [-1.51, -1.49], z_3 \in [0.7, 0.79], \text{   and   } z_4 \in [2.9, 4.2] \)

* This is the solution!
\item \( z_1 \in [-5.6, -4.8], \text{   }  z_2 \in [-0.68, -0.65], z_3 \in [1.25, 1.37], \text{   and   } z_4 \in [2.9, 4.2] \)

 Distractor 2: Corresponds to inversing rational roots.
\item \( z_1 \in [-4.9, -3.6], \text{   }  z_2 \in [-0.78, -0.72], z_3 \in [1.49, 1.51], \text{   and   } z_4 \in [4.4, 6.6] \)

 Distractor 1: Corresponds to negatives of all zeros.
\end{enumerate}

\textbf{General Comment:} Remember to try the middle-most integers first as these normally are the zeros. Also, once you get it to a quadratic, you can use your other factoring techniques to finish factoring.
}
\litem{
What are the \textit{possible Integer} roots of the polynomial below?
\[ f(x) = 3x^{4} +4 x^{3} +6 x^{2} +3 x + 5 \]The solution is \( \pm 1,\pm 5 \), which is option A.\begin{enumerate}[label=\Alph*.]
\item \( \pm 1,\pm 5 \)

* This is the solution \textbf{since we asked for the possible Integer roots}!
\item \( \text{ All combinations of: }\frac{\pm 1,\pm 5}{\pm 1,\pm 3} \)

This would have been the solution \textbf{if asked for the possible Rational roots}!
\item \( \text{ All combinations of: }\frac{\pm 1,\pm 3}{\pm 1,\pm 5} \)

 Distractor 3: Corresponds to the plus or minus of the inverse quotient (an/a0) of the factors. 
\item \( \pm 1,\pm 3 \)

 Distractor 1: Corresponds to the plus or minus factors of a1 only.
\item \( \text{There is no formula or theorem that tells us all possible Integer roots.} \)

 Distractor 4: Corresponds to not recognizing Integers as a subset of Rationals.
\end{enumerate}

\textbf{General Comment:} We have a way to find the possible Rational roots. The possible Integer roots are the Integers in this list.
}
\litem{
Perform the division below. Then, find the intervals that correspond to the quotient in the form $ax^2+bx+c$ and remainder $r$.
\[ \frac{15x^{3} +70 x^{2} +105 x + 53}{x + 2} \]The solution is \( 15x^{2} +40 x + 25 + \frac{3}{x + 2} \), which is option C.\begin{enumerate}[label=\Alph*.]
\item \( a \in [-30, -29], \text{   } b \in [128, 137], \text{   } c \in [-163, -151], \text{   and   } r \in [358, 366]. \)

 You multiplied by the synthetic number rather than bringing the first factor down.
\item \( a \in [14, 17], \text{   } b \in [97, 101], \text{   } c \in [298, 307], \text{   and   } r \in [663, 668]. \)

 You divided by the opposite of the factor.
\item \( a \in [14, 17], \text{   } b \in [39, 45], \text{   } c \in [23, 27], \text{   and   } r \in [3, 4]. \)

* This is the solution!
\item \( a \in [14, 17], \text{   } b \in [20, 29], \text{   } c \in [30, 33], \text{   and   } r \in [-37, -31]. \)

 You multiplied by the synthetic number and subtracted rather than adding during synthetic division.
\item \( a \in [-30, -29], \text{   } b \in [9, 11], \text{   } c \in [123, 128], \text{   and   } r \in [296, 309]. \)

 You divided by the opposite of the factor AND multiplied the first factor rather than just bringing it down.
\end{enumerate}

\textbf{General Comment:} Be sure to synthetically divide by the zero of the denominator!
}
\litem{
Perform the division below. Then, find the intervals that correspond to the quotient in the form $ax^2+bx+c$ and remainder $r$.
\[ \frac{6x^{3} +26 x^{2} -29}{x + 4} \]The solution is \( 6x^{2} +2 x -8 + \frac{3}{x + 4} \), which is option A.\begin{enumerate}[label=\Alph*.]
\item \( a \in [3, 10], b \in [2, 4], c \in [-11, -5], \text{ and } r \in [-6, 5]. \)

* This is the solution!
\item \( a \in [-27, -20], b \in [117, 124], c \in [-488, -487], \text{ and } r \in [1917, 1927]. \)

 You multipled by the synthetic number rather than bringing the first factor down.
\item \( a \in [3, 10], b \in [-9, 1], c \in [18, 21], \text{ and } r \in [-129, -128]. \)

 You multipled by the synthetic number and subtracted rather than adding during synthetic division.
\item \( a \in [-27, -20], b \in [-73, -64], c \in [-280, -274], \text{ and } r \in [-1151, -1146]. \)

 You divided by the opposite of the factor AND multipled the first factor rather than just bringing it down.
\item \( a \in [3, 10], b \in [47, 52], c \in [194, 202], \text{ and } r \in [769, 776]. \)

 You divided by the opposite of the factor.
\end{enumerate}

\textbf{General Comment:} Be sure to synthetically divide by the zero of the denominator! Also, make sure to include 0 placeholders for missing terms.
}
\litem{
Factor the polynomial below completely, knowing that $x + 5$ is a factor. Then, choose the intervals the zeros of the polynomial belong to, where $z_1 \leq z_2 \leq z_3 \leq z_4$. \textit{To make the problem easier, all zeros are between -5 and 5.}
\[ f(x) = 15x^{4} +139 x^{3} +383 x^{2} +333 x + 90 \]The solution is \( [-5, -3, -0.667, -0.6] \), which is option A.\begin{enumerate}[label=\Alph*.]
\item \( z_1 \in [-5.01, -4.54], \text{   }  z_2 \in [-3.73, -2.62], z_3 \in [-1.6, 1.2], \text{   and   } z_4 \in [-1.19, 0.12] \)

* This is the solution!
\item \( z_1 \in [0.37, 0.91], \text{   }  z_2 \in [0.35, 0.89], z_3 \in [2.4, 3.6], \text{   and   } z_4 \in [3.93, 5.1] \)

 Distractor 1: Corresponds to negatives of all zeros.
\item \( z_1 \in [-0.16, 0.36], \text{   }  z_2 \in [1.98, 3.21], z_3 \in [2.4, 3.6], \text{   and   } z_4 \in [3.93, 5.1] \)

 Distractor 4: Corresponds to moving factors from one rational to another.
\item \( z_1 \in [-5.01, -4.54], \text{   }  z_2 \in [-3.73, -2.62], z_3 \in [-4, -0.9], \text{   and   } z_4 \in [-2.54, -1.2] \)

 Distractor 2: Corresponds to inversing rational roots.
\item \( z_1 \in [1.45, 1.87], \text{   }  z_2 \in [1.62, 1.94], z_3 \in [2.4, 3.6], \text{   and   } z_4 \in [3.93, 5.1] \)

 Distractor 3: Corresponds to negatives of all zeros AND inversing rational roots.
\end{enumerate}

\textbf{General Comment:} Remember to try the middle-most integers first as these normally are the zeros. Also, once you get it to a quadratic, you can use your other factoring techniques to finish factoring.
}
\litem{
Perform the division below. Then, find the intervals that correspond to the quotient in the form $ax^2+bx+c$ and remainder $r$.
\[ \frac{6x^{3} -1 x^{2} -20 x + 14}{x + 2} \]The solution is \( 6x^{2} -13 x + 6 + \frac{2}{x + 2} \), which is option B.\begin{enumerate}[label=\Alph*.]
\item \( a \in [6, 9], \text{   } b \in [-20.2, -15.3], \text{   } c \in [35, 40], \text{   and   } r \in [-99, -93]. \)

 You multiplied by the synthetic number and subtracted rather than adding during synthetic division.
\item \( a \in [6, 9], \text{   } b \in [-13.5, -8.7], \text{   } c \in [5, 13], \text{   and   } r \in [-3, 4]. \)

* This is the solution!
\item \( a \in [-17, -11], \text{   } b \in [-30.2, -22.4], \text{   } c \in [-73, -68], \text{   and   } r \in [-128, -122]. \)

 You divided by the opposite of the factor AND multiplied the first factor rather than just bringing it down.
\item \( a \in [6, 9], \text{   } b \in [9.7, 12.2], \text{   } c \in [0, 3], \text{   and   } r \in [12, 22]. \)

 You divided by the opposite of the factor.
\item \( a \in [-17, -11], \text{   } b \in [22.7, 24.5], \text{   } c \in [-69, -65], \text{   and   } r \in [142, 152]. \)

 You multiplied by the synthetic number rather than bringing the first factor down.
\end{enumerate}

\textbf{General Comment:} Be sure to synthetically divide by the zero of the denominator!
}
\litem{
Factor the polynomial below completely. Then, choose the intervals the zeros of the polynomial belong to, where $z_1 \leq z_2 \leq z_3$. \textit{To make the problem easier, all zeros are between -5 and 5.}
\[ f(x) = 8x^{3} +38 x^{2} +15 x -36 \]The solution is \( [-4, -1.5, 0.75] \), which is option A.\begin{enumerate}[label=\Alph*.]
\item \( z_1 \in [-4.08, -3.9], \text{   }  z_2 \in [-1.8, -1.23], \text{   and   } z_3 \in [0.1, 0.9] \)

* This is the solution!
\item \( z_1 \in [-1.37, -1.16], \text{   }  z_2 \in [0.22, 0.87], \text{   and   } z_3 \in [2.3, 4.9] \)

 Distractor 3: Corresponds to negatives of all zeros AND inversing rational roots.
\item \( z_1 \in [-0.64, -0.36], \text{   }  z_2 \in [2.91, 3.23], \text{   and   } z_3 \in [2.3, 4.9] \)

 Distractor 4: Corresponds to moving factors from one rational to another.
\item \( z_1 \in [-0.86, -0.55], \text{   }  z_2 \in [1.36, 1.85], \text{   and   } z_3 \in [2.3, 4.9] \)

 Distractor 1: Corresponds to negatives of all zeros.
\item \( z_1 \in [-4.08, -3.9], \text{   }  z_2 \in [-1.2, -0.15], \text{   and   } z_3 \in [1, 2.4] \)

 Distractor 2: Corresponds to inversing rational roots.
\end{enumerate}

\textbf{General Comment:} Remember to try the middle-most integers first as these normally are the zeros. Also, once you get it to a quadratic, you can use your other factoring techniques to finish factoring.
}
\litem{
What are the \textit{possible Integer} roots of the polynomial below?
\[ f(x) = 4x^{3} +7 x^{2} +5 x + 5 \]The solution is \( \pm 1,\pm 5 \), which is option B.\begin{enumerate}[label=\Alph*.]
\item \( \text{ All combinations of: }\frac{\pm 1,\pm 2,\pm 4}{\pm 1,\pm 5} \)

 Distractor 3: Corresponds to the plus or minus of the inverse quotient (an/a0) of the factors. 
\item \( \pm 1,\pm 5 \)

* This is the solution \textbf{since we asked for the possible Integer roots}!
\item \( \pm 1,\pm 2,\pm 4 \)

 Distractor 1: Corresponds to the plus or minus factors of a1 only.
\item \( \text{ All combinations of: }\frac{\pm 1,\pm 5}{\pm 1,\pm 2,\pm 4} \)

This would have been the solution \textbf{if asked for the possible Rational roots}!
\item \( \text{There is no formula or theorem that tells us all possible Integer roots.} \)

 Distractor 4: Corresponds to not recognizing Integers as a subset of Rationals.
\end{enumerate}

\textbf{General Comment:} We have a way to find the possible Rational roots. The possible Integer roots are the Integers in this list.
}
\litem{
Factor the polynomial below completely. Then, choose the intervals the zeros of the polynomial belong to, where $z_1 \leq z_2 \leq z_3$. \textit{To make the problem easier, all zeros are between -5 and 5.}
\[ f(x) = 8x^{3} -22 x^{2} -65 x + 100 \]The solution is \( [-2.5, 1.25, 4] \), which is option C.\begin{enumerate}[label=\Alph*.]
\item \( z_1 \in [-4, -3], \text{   }  z_2 \in [-0.96, -0.67], \text{   and   } z_3 \in [0.1, 2.1] \)

 Distractor 3: Corresponds to negatives of all zeros AND inversing rational roots.
\item \( z_1 \in [-4, -3], \text{   }  z_2 \in [-0.74, -0.29], \text{   and   } z_3 \in [4.9, 5.1] \)

 Distractor 4: Corresponds to moving factors from one rational to another.
\item \( z_1 \in [-3.5, -1.5], \text{   }  z_2 \in [1.21, 1.28], \text{   and   } z_3 \in [3.8, 4.2] \)

* This is the solution!
\item \( z_1 \in [-4, -3], \text{   }  z_2 \in [-1.31, -1.08], \text{   and   } z_3 \in [2.1, 3] \)

 Distractor 1: Corresponds to negatives of all zeros.
\item \( z_1 \in [-2.4, 2.6], \text{   }  z_2 \in [0.79, 0.87], \text{   and   } z_3 \in [3.8, 4.2] \)

 Distractor 2: Corresponds to inversing rational roots.
\end{enumerate}

\textbf{General Comment:} Remember to try the middle-most integers first as these normally are the zeros. Also, once you get it to a quadratic, you can use your other factoring techniques to finish factoring.
}
\litem{
Perform the division below. Then, find the intervals that correspond to the quotient in the form $ax^2+bx+c$ and remainder $r$.
\[ \frac{12x^{3} -36 x + 29}{x + 2} \]The solution is \( 12x^{2} -24 x + 12 + \frac{5}{x + 2} \), which is option A.\begin{enumerate}[label=\Alph*.]
\item \( a \in [12, 15], b \in [-26, -18], c \in [10, 14], \text{ and } r \in [5, 7]. \)

* This is the solution!
\item \( a \in [-25, -16], b \in [-48, -47], c \in [-135, -129], \text{ and } r \in [-240, -232]. \)

 You divided by the opposite of the factor AND multipled the first factor rather than just bringing it down.
\item \( a \in [-25, -16], b \in [40, 54], c \in [-135, -129], \text{ and } r \in [293, 294]. \)

 You multipled by the synthetic number rather than bringing the first factor down.
\item \( a \in [12, 15], b \in [-42, -32], c \in [67, 77], \text{ and } r \in [-188, -182]. \)

 You multipled by the synthetic number and subtracted rather than adding during synthetic division.
\item \( a \in [12, 15], b \in [21, 29], c \in [10, 14], \text{ and } r \in [47, 55]. \)

 You divided by the opposite of the factor.
\end{enumerate}

\textbf{General Comment:} Be sure to synthetically divide by the zero of the denominator! Also, make sure to include 0 placeholders for missing terms.
}
\litem{
Factor the polynomial below completely, knowing that $x -5$ is a factor. Then, choose the intervals the zeros of the polynomial belong to, where $z_1 \leq z_2 \leq z_3 \leq z_4$. \textit{To make the problem easier, all zeros are between -5 and 5.}
\[ f(x) = 6x^{4} -19 x^{3} -81 x^{2} +90 x + 200 \]The solution is \( [-2.5, -1.333, 2, 5] \), which is option A.\begin{enumerate}[label=\Alph*.]
\item \( z_1 \in [-3.1, -1.7], \text{   }  z_2 \in [-1.43, -1.12], z_3 \in [1.75, 2.23], \text{   and   } z_4 \in [4.3, 6.3] \)

* This is the solution!
\item \( z_1 \in [-1.3, -0.4], \text{   }  z_2 \in [-0.6, 0.2], z_3 \in [1.75, 2.23], \text{   and   } z_4 \in [4.3, 6.3] \)

 Distractor 2: Corresponds to inversing rational roots.
\item \( z_1 \in [-6.8, -4.8], \text{   }  z_2 \in [-2.63, -1.81], z_3 \in [0.33, 0.66], \text{   and   } z_4 \in [0.2, 2] \)

 Distractor 3: Corresponds to negatives of all zeros AND inversing rational roots.
\item \( z_1 \in [-6.8, -4.8], \text{   }  z_2 \in [-2.63, -1.81], z_3 \in [0.58, 1.18], \text{   and   } z_4 \in [4.3, 6.3] \)

 Distractor 4: Corresponds to moving factors from one rational to another.
\item \( z_1 \in [-6.8, -4.8], \text{   }  z_2 \in [-2.63, -1.81], z_3 \in [1.07, 1.78], \text{   and   } z_4 \in [1.4, 3.4] \)

 Distractor 1: Corresponds to negatives of all zeros.
\end{enumerate}

\textbf{General Comment:} Remember to try the middle-most integers first as these normally are the zeros. Also, once you get it to a quadratic, you can use your other factoring techniques to finish factoring.
}
\litem{
What are the \textit{possible Rational} roots of the polynomial below?
\[ f(x) = 2x^{2} +4 x + 4 \]The solution is \( \text{ All combinations of: }\frac{\pm 1,\pm 2,\pm 4}{\pm 1,\pm 2} \), which is option A.\begin{enumerate}[label=\Alph*.]
\item \( \text{ All combinations of: }\frac{\pm 1,\pm 2,\pm 4}{\pm 1,\pm 2} \)

* This is the solution \textbf{since we asked for the possible Rational roots}!
\item \( \text{ All combinations of: }\frac{\pm 1,\pm 2}{\pm 1,\pm 2,\pm 4} \)

 Distractor 3: Corresponds to the plus or minus of the inverse quotient (an/a0) of the factors. 
\item \( \pm 1,\pm 2 \)

 Distractor 1: Corresponds to the plus or minus factors of a1 only.
\item \( \pm 1,\pm 2,\pm 4 \)

This would have been the solution \textbf{if asked for the possible Integer roots}!
\item \( \text{ There is no formula or theorem that tells us all possible Rational roots.} \)

 Distractor 4: Corresponds to not recalling the theorem for rational roots of a polynomial.
\end{enumerate}

\textbf{General Comment:} We have a way to find the possible Rational roots. The possible Integer roots are the Integers in this list.
}
\litem{
Perform the division below. Then, find the intervals that correspond to the quotient in the form $ax^2+bx+c$ and remainder $r$.
\[ \frac{12x^{3} +17 x^{2} -24 x -18}{x + 2} \]The solution is \( 12x^{2} -7 x -10 + \frac{2}{x + 2} \), which is option A.\begin{enumerate}[label=\Alph*.]
\item \( a \in [8, 16], \text{   } b \in [-8, -5], \text{   } c \in [-11, -5], \text{   and   } r \in [-2, 10]. \)

* This is the solution!
\item \( a \in [8, 16], \text{   } b \in [38, 42], \text{   } c \in [56, 61], \text{   and   } r \in [96, 102]. \)

 You divided by the opposite of the factor.
\item \( a \in [-26, -19], \text{   } b \in [61, 70], \text{   } c \in [-160, -152], \text{   and   } r \in [288, 293]. \)

 You multiplied by the synthetic number rather than bringing the first factor down.
\item \( a \in [8, 16], \text{   } b \in [-22, -17], \text{   } c \in [29, 34], \text{   and   } r \in [-121, -116]. \)

 You multiplied by the synthetic number and subtracted rather than adding during synthetic division.
\item \( a \in [-26, -19], \text{   } b \in [-35, -28], \text{   } c \in [-87, -83], \text{   and   } r \in [-192, -189]. \)

 You divided by the opposite of the factor AND multiplied the first factor rather than just bringing it down.
\end{enumerate}

\textbf{General Comment:} Be sure to synthetically divide by the zero of the denominator!
}
\litem{
Perform the division below. Then, find the intervals that correspond to the quotient in the form $ax^2+bx+c$ and remainder $r$.
\[ \frac{6x^{3} +35 x^{2} -127}{x + 5} \]The solution is \( 6x^{2} +5 x -25 + \frac{-2}{x + 5} \), which is option A.\begin{enumerate}[label=\Alph*.]
\item \( a \in [6, 10], b \in [4, 6], c \in [-32, -21], \text{ and } r \in [-2, -1]. \)

* This is the solution!
\item \( a \in [6, 10], b \in [65, 67], c \in [316, 329], \text{ and } r \in [1496, 1502]. \)

 You divided by the opposite of the factor.
\item \( a \in [6, 10], b \in [-5, 1], c \in [5, 9], \text{ and } r \in [-163, -160]. \)

 You multipled by the synthetic number and subtracted rather than adding during synthetic division.
\item \( a \in [-30, -28], b \in [183, 190], c \in [-926, -922], \text{ and } r \in [4497, 4503]. \)

 You multipled by the synthetic number rather than bringing the first factor down.
\item \( a \in [-30, -28], b \in [-117, -107], c \in [-578, -574], \text{ and } r \in [-3002, -3001]. \)

 You divided by the opposite of the factor AND multipled the first factor rather than just bringing it down.
\end{enumerate}

\textbf{General Comment:} Be sure to synthetically divide by the zero of the denominator! Also, make sure to include 0 placeholders for missing terms.
}
\end{enumerate}

\end{document}