\documentclass{extbook}[14pt]
\usepackage{multicol, enumerate, enumitem, hyperref, color, soul, setspace, parskip, fancyhdr, amssymb, amsthm, amsmath, bbm, latexsym, units, mathtools}
\everymath{\displaystyle}
\usepackage[headsep=0.5cm,headheight=0cm, left=1 in,right= 1 in,top= 1 in,bottom= 1 in]{geometry}
\usepackage{dashrule}  % Package to use the command below to create lines between items
\newcommand{\litem}[1]{\item #1

\rule{\textwidth}{0.4pt}}
\pagestyle{fancy}
\lhead{}
\chead{Answer Key for Progress Quiz 4 Version B}
\rhead{}
\lfoot{4378-7085}
\cfoot{}
\rfoot{Fall 2020}
\begin{document}
\textbf{This key should allow you to understand why you choose the option you did (beyond just getting a question right or wrong). \href{https://xronos.clas.ufl.edu/mac1105spring2020/courseDescriptionAndMisc/Exams/LearningFromResults}{More instructions on how to use this key can be found here}.}

\textbf{If you have a suggestion to make the keys better, \href{https://forms.gle/CZkbZmPbC9XALEE88}{please fill out the short survey here}.}

\textit{Note: This key is auto-generated and may contain issues and/or errors. The keys are reviewed after each exam to ensure grading is done accurately. If there are issues (like duplicate options), they are noted in the offline gradebook. The keys are a work-in-progress to give students as many resources to improve as possible.}

\rule{\textwidth}{0.4pt}

\begin{enumerate}\litem{
Using an interval or intervals, describe all the $x$-values within or including a distance of the given values.
\[ \text{ More than } 6 \text{ units from the number } 1. \]
The solution is \( (-\infty, -5) \cup (7, \infty) \), which is option A.\begin{enumerate}[label=\Alph*.]
\item \( (-\infty, -5) \cup (7, \infty) \)

This describes the values more than 6 from 1
\item \( (-\infty, -5] \cup [7, \infty) \)

This describes the values no less than 6 from 1
\item \( [-5, 7] \)

This describes the values no more than 6 from 1
\item \( (-5, 7) \)

This describes the values less than 6 from 1
\item \( \text{None of the above} \)

You likely thought the values in the interval were not correct.
\end{enumerate}

\textbf{General Comment:} When thinking about this language, it helps to draw a number line and try points.
}
\litem{
Solve the linear inequality below. Then, choose the constant and interval combination that describes the solution set.
\[ \frac{4}{9} - \frac{9}{6} x \geq \frac{3}{3} x + \frac{6}{2} \]
The solution is \( (-\infty, -1.022] \), which is option A.\begin{enumerate}[label=\Alph*.]
\item \( (-\infty, a], \text{ where } a \in [-3.02, -0.02] \)

* $(-\infty, -1.022]$, which is the correct option.
\item \( [a, \infty), \text{ where } a \in [-3.02, 0.98] \)

 $[-1.022, \infty)$, which corresponds to switching the direction of the interval. You likely did this if you did not flip the inequality when dividing by a negative!
\item \( (-\infty, a], \text{ where } a \in [0.02, 5.02] \)

 $(-\infty, 1.022]$, which corresponds to negating the endpoint of the solution.
\item \( [a, \infty), \text{ where } a \in [1.02, 4.02] \)

 $[1.022, \infty)$, which corresponds to switching the direction of the interval AND negating the endpoint. You likely did this if you did not flip the inequality when dividing by a negative as well as not moving values over to a side properly.
\item \( \text{None of the above}. \)

You may have chosen this if you thought the inequality did not match the ends of the intervals.
\end{enumerate}

\textbf{General Comment:} Remember that less/greater than or equal to includes the endpoint, while less/greater do not. Also, remember that you need to flip the inequality when you multiply or divide by a negative.
}
\litem{
Solve the linear inequality below. Then, choose the constant and interval combination that describes the solution set.
\[ -7 + 7 x > 9 x \text{ or } -3 + 6 x < 8 x \]
The solution is \( (-\infty, -3.5) \text{ or } (-1.5, \infty) \), which is option D.\begin{enumerate}[label=\Alph*.]
\item \( (-\infty, a] \cup [b, \infty), \text{ where } a \in [1.5, 7.5] \text{ and } b \in [1.5, 8.5] \)

Corresponds to including the endpoints AND negating.
\item \( (-\infty, a] \cup [b, \infty), \text{ where } a \in [-3.5, -2.5] \text{ and } b \in [-5.5, 0.5] \)

Corresponds to including the endpoints (when they should be excluded).
\item \( (-\infty, a) \cup (b, \infty), \text{ where } a \in [0.5, 3.5] \text{ and } b \in [2.5, 10.5] \)

Corresponds to inverting the inequality and negating the solution.
\item \( (-\infty, a) \cup (b, \infty), \text{ where } a \in [-5.5, -1.5] \text{ and } b \in [-3.5, 1.5] \)

 * Correct option.
\item \( (-\infty, \infty) \)

Corresponds to the variable canceling, which does not happen in this instance.
\end{enumerate}

\textbf{General Comment:} When multiplying or dividing by a negative, flip the sign.
}
\litem{
Solve the linear inequality below. Then, choose the constant and interval combination that describes the solution set.
\[ -7x + 7 \geq 10x -7 \]
The solution is \( (-\infty, 0.824] \), which is option C.\begin{enumerate}[label=\Alph*.]
\item \( (-\infty, a], \text{ where } a \in [-1.33, -0.51] \)

 $(-\infty, -0.824]$, which corresponds to negating the endpoint of the solution.
\item \( [a, \infty), \text{ where } a \in [-1.94, -0.76] \)

 $[-0.824, \infty)$, which corresponds to switching the direction of the interval AND negating the endpoint. You likely did this if you did not flip the inequality when dividing by a negative as well as not moving values over to a side properly.
\item \( (-\infty, a], \text{ where } a \in [-0.13, 1.4] \)

* $(-\infty, 0.824]$, which is the correct option.
\item \( [a, \infty), \text{ where } a \in [0.24, 1.48] \)

 $[0.824, \infty)$, which corresponds to switching the direction of the interval. You likely did this if you did not flip the inequality when dividing by a negative!
\item \( \text{None of the above}. \)

You may have chosen this if you thought the inequality did not match the ends of the intervals.
\end{enumerate}

\textbf{General Comment:} Remember that less/greater than or equal to includes the endpoint, while less/greater do not. Also, remember that you need to flip the inequality when you multiply or divide by a negative.
}
\litem{
Solve the linear inequality below. Then, choose the constant and interval combination that describes the solution set.
\[ -7 + 5 x > 7 x \text{ or } 7 + 5 x < 8 x \]
The solution is \( (-\infty, -3.5) \text{ or } (2.333, \infty) \), which is option B.\begin{enumerate}[label=\Alph*.]
\item \( (-\infty, a] \cup [b, \infty), \text{ where } a \in [-4, -2.4] \text{ and } b \in [1.33, 3.33] \)

Corresponds to including the endpoints (when they should be excluded).
\item \( (-\infty, a) \cup (b, \infty), \text{ where } a \in [-4.48, -2.89] \text{ and } b \in [1.2, 3.2] \)

 * Correct option.
\item \( (-\infty, a] \cup [b, \infty), \text{ where } a \in [-3, 0.8] \text{ and } b \in [2.5, 6.5] \)

Corresponds to including the endpoints AND negating.
\item \( (-\infty, a) \cup (b, \infty), \text{ where } a \in [-2.81, -1.53] \text{ and } b \in [2.6, 5.3] \)

Corresponds to inverting the inequality and negating the solution.
\item \( (-\infty, \infty) \)

Corresponds to the variable canceling, which does not happen in this instance.
\end{enumerate}

\textbf{General Comment:} When multiplying or dividing by a negative, flip the sign.
}
\litem{
Solve the linear inequality below. Then, choose the constant and interval combination that describes the solution set.
\[ \frac{3}{4} + \frac{7}{5} x > \frac{8}{6} x + \frac{10}{9} \]
The solution is \( (5.417, \infty) \), which is option B.\begin{enumerate}[label=\Alph*.]
\item \( (-\infty, a), \text{ where } a \in [3.42, 8.42] \)

 $(-\infty, 5.417)$, which corresponds to switching the direction of the interval. You likely did this if you did not flip the inequality when dividing by a negative!
\item \( (a, \infty), \text{ where } a \in [4.42, 6.42] \)

* $(5.417, \infty)$, which is the correct option.
\item \( (-\infty, a), \text{ where } a \in [-8.42, -4.42] \)

 $(-\infty, -5.417)$, which corresponds to switching the direction of the interval AND negating the endpoint. You likely did this if you did not flip the inequality when dividing by a negative as well as not moving values over to a side properly.
\item \( (a, \infty), \text{ where } a \in [-6.42, -4.42] \)

 $(-5.417, \infty)$, which corresponds to negating the endpoint of the solution.
\item \( \text{None of the above}. \)

You may have chosen this if you thought the inequality did not match the ends of the intervals.
\end{enumerate}

\textbf{General Comment:} Remember that less/greater than or equal to includes the endpoint, while less/greater do not. Also, remember that you need to flip the inequality when you multiply or divide by a negative.
}
\litem{
Solve the linear inequality below. Then, choose the constant and interval combination that describes the solution set.
\[ 5 + 4 x \leq \frac{22 x + 5}{4} < 8 + 5 x \]
The solution is \( [2.50, 13.50) \), which is option B.\begin{enumerate}[label=\Alph*.]
\item \( (-\infty, a] \cup (b, \infty), \text{ where } a \in [-0.5, 4.5] \text{ and } b \in [7.5, 16.5] \)

$(-\infty, 2.50] \cup (13.50, \infty)$, which corresponds to displaying the and-inequality as an or-inequality.
\item \( [a, b), \text{ where } a \in [1.5, 6.5] \text{ and } b \in [12.5, 14.5] \)

$[2.50, 13.50)$, which is the correct option.
\item \( (-\infty, a) \cup [b, \infty), \text{ where } a \in [-1.5, 4.5] \text{ and } b \in [11.5, 16.5] \)

$(-\infty, 2.50) \cup [13.50, \infty)$, which corresponds to displaying the and-inequality as an or-inequality AND flipping the inequality.
\item \( (a, b], \text{ where } a \in [1.5, 3.5] \text{ and } b \in [9.5, 16.5] \)

$(2.50, 13.50]$, which corresponds to flipping the inequality.
\item \( \text{None of the above.} \)


\end{enumerate}

\textbf{General Comment:} To solve, you will need to break up the compound inequality into two inequalities. Be sure to keep track of the inequality! It may be best to draw a number line and graph your solution.
}
\litem{
Solve the linear inequality below. Then, choose the constant and interval combination that describes the solution set.
\[ -3x -4 \leq 6x + 9 \]
The solution is \( [-1.444, \infty) \), which is option A.\begin{enumerate}[label=\Alph*.]
\item \( [a, \infty), \text{ where } a \in [-2.28, -0.92] \)

* $[-1.444, \infty)$, which is the correct option.
\item \( (-\infty, a], \text{ where } a \in [0.44, 3.44] \)

 $(-\infty, 1.444]$, which corresponds to switching the direction of the interval AND negating the endpoint. You likely did this if you did not flip the inequality when dividing by a negative as well as not moving values over to a side properly.
\item \( (-\infty, a], \text{ where } a \in [-8.44, 0.56] \)

 $(-\infty, -1.444]$, which corresponds to switching the direction of the interval. You likely did this if you did not flip the inequality when dividing by a negative!
\item \( [a, \infty), \text{ where } a \in [0.5, 1.88] \)

 $[1.444, \infty)$, which corresponds to negating the endpoint of the solution.
\item \( \text{None of the above}. \)

You may have chosen this if you thought the inequality did not match the ends of the intervals.
\end{enumerate}

\textbf{General Comment:} Remember that less/greater than or equal to includes the endpoint, while less/greater do not. Also, remember that you need to flip the inequality when you multiply or divide by a negative.
}
\litem{
Solve the linear inequality below. Then, choose the constant and interval combination that describes the solution set.
\[ 9 + 3 x \leq \frac{68 x + 5}{9} < 7 + 7 x \]
The solution is \( \text{None of the above.} \), which is option E.\begin{enumerate}[label=\Alph*.]
\item \( [a, b), \text{ where } a \in [-2.1, -1.2] \text{ and } b \in [-13.6, -7.6] \)

$[-1.85, -11.60)$, which is the correct interval but negatives of the actual endpoints.
\item \( (a, b], \text{ where } a \in [-2.85, 0.15] \text{ and } b \in [-12.6, -10.6] \)

$(-1.85, -11.60]$, which corresponds to flipping the inequality and getting negatives of the actual endpoints.
\item \( (-\infty, a] \cup (b, \infty), \text{ where } a \in [-2.85, 0.15] \text{ and } b \in [-14.6, -10.6] \)

$(-\infty, -1.85] \cup (-11.60, \infty)$, which corresponds to displaying the and-inequality as an or-inequality and getting negatives of the actual endpoints.
\item \( (-\infty, a) \cup [b, \infty), \text{ where } a \in [-4.85, -0.85] \text{ and } b \in [-11.6, -9.6] \)

$(-\infty, -1.85) \cup [-11.60, \infty)$, which corresponds to displaying the and-inequality as an or-inequality AND flipping the inequality AND getting negatives of the actual endpoints.
\item \( \text{None of the above.} \)

* This is correct as the answer should be $[1.85, 11.60)$.
\end{enumerate}

\textbf{General Comment:} To solve, you will need to break up the compound inequality into two inequalities. Be sure to keep track of the inequality! It may be best to draw a number line and graph your solution.
}
\litem{
Using an interval or intervals, describe all the $x$-values within or including a distance of the given values.
\[ \text{ Less than } 5 \text{ units from the number } -6. \]
The solution is \( (-11, -1) \), which is option D.\begin{enumerate}[label=\Alph*.]
\item \( [-11, -1] \)

This describes the values no more than 5 from -6
\item \( (-\infty, -11) \cup (-1, \infty) \)

This describes the values more than 5 from -6
\item \( (-\infty, -11] \cup [-1, \infty) \)

This describes the values no less than 5 from -6
\item \( (-11, -1) \)

This describes the values less than 5 from -6
\item \( \text{None of the above} \)

You likely thought the values in the interval were not correct.
\end{enumerate}

\textbf{General Comment:} When thinking about this language, it helps to draw a number line and try points.
}
\end{enumerate}

\end{document}