\documentclass{extbook}[14pt]
\usepackage{multicol, enumerate, enumitem, hyperref, color, soul, setspace, parskip, fancyhdr, amssymb, amsthm, amsmath, latexsym, units, mathtools}
\everymath{\displaystyle}
\usepackage[headsep=0.5cm,headheight=0cm, left=1 in,right= 1 in,top= 1 in,bottom= 1 in]{geometry}
\usepackage{dashrule}  % Package to use the command below to create lines between items
\newcommand{\litem}[1]{\item #1

\rule{\textwidth}{0.4pt}}
\pagestyle{fancy}
\lhead{}
\chead{Answer Key for Progress Quiz 4 Version B}
\rhead{}
\lfoot{5346-5907}
\cfoot{}
\rfoot{Summer C 2021}
\begin{document}
\textbf{This key should allow you to understand why you choose the option you did (beyond just getting a question right or wrong). \href{https://xronos.clas.ufl.edu/mac1105spring2020/courseDescriptionAndMisc/Exams/LearningFromResults}{More instructions on how to use this key can be found here}.}

\textbf{If you have a suggestion to make the keys better, \href{https://forms.gle/CZkbZmPbC9XALEE88}{please fill out the short survey here}.}

\textit{Note: This key is auto-generated and may contain issues and/or errors. The keys are reviewed after each exam to ensure grading is done accurately. If there are issues (like duplicate options), they are noted in the offline gradebook. The keys are a work-in-progress to give students as many resources to improve as possible.}

\rule{\textwidth}{0.4pt}

\begin{enumerate}\litem{
Solve the linear inequality below. Then, choose the constant and interval combination that describes the solution set.
\[ -8 + 9 x < \frac{30 x - 8}{3} \leq -9 + 6 x \]The solution is \( (-5.33, -1.58] \), which is option A.\begin{enumerate}[label=\Alph*.]
\item \( (a, b], \text{ where } a \in [-7.5, -3.75] \text{ and } b \in [-6, 1.5] \)

* $(-5.33, -1.58]$, which is the correct option.
\item \( (-\infty, a] \cup (b, \infty), \text{ where } a \in [-9.75, -1.5] \text{ and } b \in [-8.25, -0.75] \)

$(-\infty, -5.33] \cup (-1.58, \infty)$, which corresponds to displaying the and-inequality as an or-inequality AND flipping the inequality.
\item \( [a, b), \text{ where } a \in [-6.75, -4.5] \text{ and } b \in [-3, 0] \)

$[-5.33, -1.58)$, which corresponds to flipping the inequality.
\item \( (-\infty, a) \cup [b, \infty), \text{ where } a \in [-6, -3.75] \text{ and } b \in [-4.5, -0.75] \)

$(-\infty, -5.33) \cup [-1.58, \infty)$, which corresponds to displaying the and-inequality as an or-inequality.
\item \( \text{None of the above.} \)


\end{enumerate}

\textbf{General Comment:} To solve, you will need to break up the compound inequality into two inequalities. Be sure to keep track of the inequality! It may be best to draw a number line and graph your solution.
}
\litem{
Solve the linear inequality below. Then, choose the constant and interval combination that describes the solution set.
\[ \frac{9}{2} - \frac{5}{4} x \geq \frac{-4}{8} x - \frac{5}{9} \]The solution is \( (-\infty, 6.741] \), which is option A.\begin{enumerate}[label=\Alph*.]
\item \( (-\infty, a], \text{ where } a \in [6, 8.25] \)

* $(-\infty, 6.741]$, which is the correct option.
\item \( (-\infty, a], \text{ where } a \in [-10.5, -3.75] \)

 $(-\infty, -6.741]$, which corresponds to negating the endpoint of the solution.
\item \( [a, \infty), \text{ where } a \in [6, 8.25] \)

 $[6.741, \infty)$, which corresponds to switching the direction of the interval. You likely did this if you did not flip the inequality when dividing by a negative!
\item \( [a, \infty), \text{ where } a \in [-8.25, -5.25] \)

 $[-6.741, \infty)$, which corresponds to switching the direction of the interval AND negating the endpoint. You likely did this if you did not flip the inequality when dividing by a negative as well as not moving values over to a side properly.
\item \( \text{None of the above}. \)

You may have chosen this if you thought the inequality did not match the ends of the intervals.
\end{enumerate}

\textbf{General Comment:} Remember that less/greater than or equal to includes the endpoint, while less/greater do not. Also, remember that you need to flip the inequality when you multiply or divide by a negative.
}
\litem{
Solve the linear inequality below. Then, choose the constant and interval combination that describes the solution set.
\[ \frac{7}{6} - \frac{5}{7} x < \frac{7}{9} x - \frac{3}{5} \]The solution is \( (1.184, \infty) \), which is option C.\begin{enumerate}[label=\Alph*.]
\item \( (-\infty, a), \text{ where } a \in [-4.5, 0.75] \)

 $(-\infty, -1.184)$, which corresponds to switching the direction of the interval AND negating the endpoint. You likely did this if you did not flip the inequality when dividing by a negative as well as not moving values over to a side properly.
\item \( (-\infty, a), \text{ where } a \in [0.75, 3] \)

 $(-\infty, 1.184)$, which corresponds to switching the direction of the interval. You likely did this if you did not flip the inequality when dividing by a negative!
\item \( (a, \infty), \text{ where } a \in [0, 3.75] \)

* $(1.184, \infty)$, which is the correct option.
\item \( (a, \infty), \text{ where } a \in [-3, -0.75] \)

 $(-1.184, \infty)$, which corresponds to negating the endpoint of the solution.
\item \( \text{None of the above}. \)

You may have chosen this if you thought the inequality did not match the ends of the intervals.
\end{enumerate}

\textbf{General Comment:} Remember that less/greater than or equal to includes the endpoint, while less/greater do not. Also, remember that you need to flip the inequality when you multiply or divide by a negative.
}
\litem{
Using an interval or intervals, describe all the $x$-values within or including a distance of the given values.
\[ \text{ No less than } 8 \text{ units from the number } -4. \]The solution is \( (-\infty, -12] \cup [4, \infty) \), which is option C.\begin{enumerate}[label=\Alph*.]
\item \( (-12, 4) \)

This describes the values less than 8 from -4
\item \( (-\infty, -12) \cup (4, \infty) \)

This describes the values more than 8 from -4
\item \( (-\infty, -12] \cup [4, \infty) \)

This describes the values no less than 8 from -4
\item \( [-12, 4] \)

This describes the values no more than 8 from -4
\item \( \text{None of the above} \)

You likely thought the values in the interval were not correct.
\end{enumerate}

\textbf{General Comment:} When thinking about this language, it helps to draw a number line and try points.
}
\litem{
Solve the linear inequality below. Then, choose the constant and interval combination that describes the solution set.
\[ -9 + 3 x > 5 x \text{ or } 3 + 9 x < 12 x \]The solution is \( (-\infty, -4.5) \text{ or } (1.0, \infty) \), which is option C.\begin{enumerate}[label=\Alph*.]
\item \( (-\infty, a] \cup [b, \infty), \text{ where } a \in [-5.25, -3] \text{ and } b \in [-6, 3] \)

Corresponds to including the endpoints (when they should be excluded).
\item \( (-\infty, a] \cup [b, \infty), \text{ where } a \in [-1.5, 3.75] \text{ and } b \in [3.75, 7.5] \)

Corresponds to including the endpoints AND negating.
\item \( (-\infty, a) \cup (b, \infty), \text{ where } a \in [-5.62, -3.82] \text{ and } b \in [-1.5, 1.5] \)

 * Correct option.
\item \( (-\infty, a) \cup (b, \infty), \text{ where } a \in [-3.9, -0.6] \text{ and } b \in [2.25, 6] \)

Corresponds to inverting the inequality and negating the solution.
\item \( (-\infty, \infty) \)

Corresponds to the variable canceling, which does not happen in this instance.
\end{enumerate}

\textbf{General Comment:} When multiplying or dividing by a negative, flip the sign.
}
\litem{
Using an interval or intervals, describe all the $x$-values within or including a distance of the given values.
\[ \text{ No more than } 9 \text{ units from the number } 1. \]The solution is \( [-8, 10] \), which is option C.\begin{enumerate}[label=\Alph*.]
\item \( (-\infty, -8) \cup (10, \infty) \)

This describes the values more than 9 from 1
\item \( (-\infty, -8] \cup [10, \infty) \)

This describes the values no less than 9 from 1
\item \( [-8, 10] \)

This describes the values no more than 9 from 1
\item \( (-8, 10) \)

This describes the values less than 9 from 1
\item \( \text{None of the above} \)

You likely thought the values in the interval were not correct.
\end{enumerate}

\textbf{General Comment:} When thinking about this language, it helps to draw a number line and try points.
}
\litem{
Solve the linear inequality below. Then, choose the constant and interval combination that describes the solution set.
\[ -8 - 4 x \leq \frac{-21 x + 7}{6} < -9 - 9 x \]The solution is \( \text{None of the above.} \), which is option E.\begin{enumerate}[label=\Alph*.]
\item \( [a, b), \text{ where } a \in [17.25, 21] \text{ and } b \in [0.67, 5.7] \)

$[18.33, 1.85)$, which is the correct interval but negatives of the actual endpoints.
\item \( (-\infty, a) \cup [b, \infty), \text{ where } a \in [14.25, 21.75] \text{ and } b \in [0, 4.5] \)

$(-\infty, 18.33) \cup [1.85, \infty)$, which corresponds to displaying the and-inequality as an or-inequality AND flipping the inequality AND getting negatives of the actual endpoints.
\item \( (a, b], \text{ where } a \in [14.25, 19.5] \text{ and } b \in [0, 3] \)

$(18.33, 1.85]$, which corresponds to flipping the inequality and getting negatives of the actual endpoints.
\item \( (-\infty, a] \cup (b, \infty), \text{ where } a \in [17.25, 19.5] \text{ and } b \in [0, 4.5] \)

$(-\infty, 18.33] \cup (1.85, \infty)$, which corresponds to displaying the and-inequality as an or-inequality and getting negatives of the actual endpoints.
\item \( \text{None of the above.} \)

* This is correct as the answer should be $[-18.33, -1.85)$.
\end{enumerate}

\textbf{General Comment:} To solve, you will need to break up the compound inequality into two inequalities. Be sure to keep track of the inequality! It may be best to draw a number line and graph your solution.
}
\litem{
Solve the linear inequality below. Then, choose the constant and interval combination that describes the solution set.
\[ 4x -4 < 5x + 3 \]The solution is \( (-7.0, \infty) \), which is option A.\begin{enumerate}[label=\Alph*.]
\item \( (a, \infty), \text{ where } a \in [-7, -1] \)

* $(-7.0, \infty)$, which is the correct option.
\item \( (-\infty, a), \text{ where } a \in [5, 10] \)

 $(-\infty, 7.0)$, which corresponds to switching the direction of the interval AND negating the endpoint. You likely did this if you did not flip the inequality when dividing by a negative as well as not moving values over to a side properly.
\item \( (-\infty, a), \text{ where } a \in [-7, -3] \)

 $(-\infty, -7.0)$, which corresponds to switching the direction of the interval. You likely did this if you did not flip the inequality when dividing by a negative!
\item \( (a, \infty), \text{ where } a \in [4, 10] \)

 $(7.0, \infty)$, which corresponds to negating the endpoint of the solution.
\item \( \text{None of the above}. \)

You may have chosen this if you thought the inequality did not match the ends of the intervals.
\end{enumerate}

\textbf{General Comment:} Remember that less/greater than or equal to includes the endpoint, while less/greater do not. Also, remember that you need to flip the inequality when you multiply or divide by a negative.
}
\litem{
Solve the linear inequality below. Then, choose the constant and interval combination that describes the solution set.
\[ 7x + 8 \leq 10x + 7 \]The solution is \( [0.333, \infty) \), which is option D.\begin{enumerate}[label=\Alph*.]
\item \( (-\infty, a], \text{ where } a \in [-0.6, -0.1] \)

 $(-\infty, -0.333]$, which corresponds to switching the direction of the interval AND negating the endpoint. You likely did this if you did not flip the inequality when dividing by a negative as well as not moving values over to a side properly.
\item \( [a, \infty), \text{ where } a \in [-0.75, 0.22] \)

 $[-0.333, \infty)$, which corresponds to negating the endpoint of the solution.
\item \( (-\infty, a], \text{ where } a \in [-0.3, 1.5] \)

 $(-\infty, 0.333]$, which corresponds to switching the direction of the interval. You likely did this if you did not flip the inequality when dividing by a negative!
\item \( [a, \infty), \text{ where } a \in [-0.14, 0.86] \)

* $[0.333, \infty)$, which is the correct option.
\item \( \text{None of the above}. \)

You may have chosen this if you thought the inequality did not match the ends of the intervals.
\end{enumerate}

\textbf{General Comment:} Remember that less/greater than or equal to includes the endpoint, while less/greater do not. Also, remember that you need to flip the inequality when you multiply or divide by a negative.
}
\litem{
Solve the linear inequality below. Then, choose the constant and interval combination that describes the solution set.
\[ -3 + 9 x > 11 x \text{ or } 8 + 6 x < 7 x \]The solution is \( (-\infty, -1.5) \text{ or } (8.0, \infty) \), which is option D.\begin{enumerate}[label=\Alph*.]
\item \( (-\infty, a] \cup [b, \infty), \text{ where } a \in [-6, -0.75] \text{ and } b \in [3, 10.5] \)

Corresponds to including the endpoints (when they should be excluded).
\item \( (-\infty, a] \cup [b, \infty), \text{ where } a \in [-10.5, -3] \text{ and } b \in [-1.5, 3] \)

Corresponds to including the endpoints AND negating.
\item \( (-\infty, a) \cup (b, \infty), \text{ where } a \in [-9, -6.75] \text{ and } b \in [0, 2.25] \)

Corresponds to inverting the inequality and negating the solution.
\item \( (-\infty, a) \cup (b, \infty), \text{ where } a \in [-6, -0.75] \text{ and } b \in [3.75, 9.75] \)

 * Correct option.
\item \( (-\infty, \infty) \)

Corresponds to the variable canceling, which does not happen in this instance.
\end{enumerate}

\textbf{General Comment:} When multiplying or dividing by a negative, flip the sign.
}
\end{enumerate}

\end{document}