\documentclass{extbook}[14pt]
\usepackage{multicol, enumerate, enumitem, hyperref, color, soul, setspace, parskip, fancyhdr, amssymb, amsthm, amsmath, bbm, latexsym, units, mathtools}
\everymath{\displaystyle}
\usepackage[headsep=0.5cm,headheight=0cm, left=1 in,right= 1 in,top= 1 in,bottom= 1 in]{geometry}
\usepackage{dashrule}  % Package to use the command below to create lines between items
\newcommand{\litem}[1]{\item #1

\rule{\textwidth}{0.4pt}}
\pagestyle{fancy}
\lhead{}
\chead{Answer Key for Progress Quiz 4 Version B}
\rhead{}
\lfoot{9187-5854}
\cfoot{}
\rfoot{Spring 2021}
\begin{document}
\textbf{This key should allow you to understand why you choose the option you did (beyond just getting a question right or wrong). \href{https://xronos.clas.ufl.edu/mac1105spring2020/courseDescriptionAndMisc/Exams/LearningFromResults}{More instructions on how to use this key can be found here}.}

\textbf{If you have a suggestion to make the keys better, \href{https://forms.gle/CZkbZmPbC9XALEE88}{please fill out the short survey here}.}

\textit{Note: This key is auto-generated and may contain issues and/or errors. The keys are reviewed after each exam to ensure grading is done accurately. If there are issues (like duplicate options), they are noted in the offline gradebook. The keys are a work-in-progress to give students as many resources to improve as possible.}

\rule{\textwidth}{0.4pt}

\begin{enumerate}\litem{
Solve the linear inequality below. Then, choose the constant and interval combination that describes the solution set.
\[ -9 - 5 x \leq \frac{25 x + 5}{5} < 3 + 4 x \]The solution is \( \text{None of the above.} \), which is option E.\begin{enumerate}[label=\Alph*.]
\item \( (-\infty, a) \cup [b, \infty), \text{ where } a \in [0, 4] \text{ and } b \in [-3, 1] \)

$(-\infty, 1.00) \cup [-2.00, \infty)$, which corresponds to displaying the and-inequality as an or-inequality AND flipping the inequality AND getting negatives of the actual endpoints.
\item \( (a, b], \text{ where } a \in [0.1, 2.8] \text{ and } b \in [-4, -1] \)

$(1.00, -2.00]$, which corresponds to flipping the inequality and getting negatives of the actual endpoints.
\item \( [a, b), \text{ where } a \in [0, 7] \text{ and } b \in [-2, 1] \)

$[1.00, -2.00)$, which is the correct interval but negatives of the actual endpoints.
\item \( (-\infty, a] \cup (b, \infty), \text{ where } a \in [0.6, 1.6] \text{ and } b \in [-3.4, -0.1] \)

$(-\infty, 1.00] \cup (-2.00, \infty)$, which corresponds to displaying the and-inequality as an or-inequality and getting negatives of the actual endpoints.
\item \( \text{None of the above.} \)

* This is correct as the answer should be $[-1.00, 2.00)$.
\end{enumerate}

\textbf{General Comment:} To solve, you will need to break up the compound inequality into two inequalities. Be sure to keep track of the inequality! It may be best to draw a number line and graph your solution.
}
\litem{
Solve the linear inequality below. Then, choose the constant and interval combination that describes the solution set.
\[ \frac{7}{8} - \frac{8}{4} x > \frac{-5}{3} x - \frac{6}{9} \]The solution is \( (-\infty, 4.625) \), which is option C.\begin{enumerate}[label=\Alph*.]
\item \( (-\infty, a), \text{ where } a \in [-5.62, -2.62] \)

 $(-\infty, -4.625)$, which corresponds to negating the endpoint of the solution.
\item \( (a, \infty), \text{ where } a \in [2.62, 5.62] \)

 $(4.625, \infty)$, which corresponds to switching the direction of the interval. You likely did this if you did not flip the inequality when dividing by a negative!
\item \( (-\infty, a), \text{ where } a \in [3.62, 9.62] \)

* $(-\infty, 4.625)$, which is the correct option.
\item \( (a, \infty), \text{ where } a \in [-5.62, -1.62] \)

 $(-4.625, \infty)$, which corresponds to switching the direction of the interval AND negating the endpoint. You likely did this if you did not flip the inequality when dividing by a negative as well as not moving values over to a side properly.
\item \( \text{None of the above}. \)

You may have chosen this if you thought the inequality did not match the ends of the intervals.
\end{enumerate}

\textbf{General Comment:} Remember that less/greater than or equal to includes the endpoint, while less/greater do not. Also, remember that you need to flip the inequality when you multiply or divide by a negative.
}
\litem{
Solve the linear inequality below. Then, choose the constant and interval combination that describes the solution set.
\[ -4 + 8 x > 11 x \text{ or } 9 + 7 x < 8 x \]The solution is \( (-\infty, -1.333) \text{ or } (9.0, \infty) \), which is option D.\begin{enumerate}[label=\Alph*.]
\item \( (-\infty, a] \cup [b, \infty), \text{ where } a \in [-13, -6] \text{ and } b \in [-3.67, 2.33] \)

Corresponds to including the endpoints AND negating.
\item \( (-\infty, a) \cup (b, \infty), \text{ where } a \in [-9, -6] \text{ and } b \in [0.33, 2.33] \)

Corresponds to inverting the inequality and negating the solution.
\item \( (-\infty, a] \cup [b, \infty), \text{ where } a \in [-5.33, -0.33] \text{ and } b \in [8, 10] \)

Corresponds to including the endpoints (when they should be excluded).
\item \( (-\infty, a) \cup (b, \infty), \text{ where } a \in [-2.33, 0.67] \text{ and } b \in [7, 13] \)

 * Correct option.
\item \( (-\infty, \infty) \)

Corresponds to the variable canceling, which does not happen in this instance.
\end{enumerate}

\textbf{General Comment:} When multiplying or dividing by a negative, flip the sign.
}
\litem{
Solve the linear inequality below. Then, choose the constant and interval combination that describes the solution set.
\[ -9x + 7 \geq -4x + 10 \]The solution is \( (-\infty, -0.6] \), which is option C.\begin{enumerate}[label=\Alph*.]
\item \( (-\infty, a], \text{ where } a \in [-0.33, 0.62] \)

 $(-\infty, 0.6]$, which corresponds to negating the endpoint of the solution.
\item \( [a, \infty), \text{ where } a \in [-3, 0.2] \)

 $[-0.6, \infty)$, which corresponds to switching the direction of the interval. You likely did this if you did not flip the inequality when dividing by a negative!
\item \( (-\infty, a], \text{ where } a \in [-0.96, 0.24] \)

* $(-\infty, -0.6]$, which is the correct option.
\item \( [a, \infty), \text{ where } a \in [0.1, 1.1] \)

 $[0.6, \infty)$, which corresponds to switching the direction of the interval AND negating the endpoint. You likely did this if you did not flip the inequality when dividing by a negative as well as not moving values over to a side properly.
\item \( \text{None of the above}. \)

You may have chosen this if you thought the inequality did not match the ends of the intervals.
\end{enumerate}

\textbf{General Comment:} Remember that less/greater than or equal to includes the endpoint, while less/greater do not. Also, remember that you need to flip the inequality when you multiply or divide by a negative.
}
\litem{
Using an interval or intervals, describe all the $x$-values within or including a distance of the given values.
\[ \text{ More than } 6 \text{ units from the number } -9. \]The solution is \( (-\infty, -15) \cup (-3, \infty) \), which is option A.\begin{enumerate}[label=\Alph*.]
\item \( (-\infty, -15) \cup (-3, \infty) \)

This describes the values more than 6 from -9
\item \( [-15, -3] \)

This describes the values no more than 6 from -9
\item \( (-15, -3) \)

This describes the values less than 6 from -9
\item \( (-\infty, -15] \cup [-3, \infty) \)

This describes the values no less than 6 from -9
\item \( \text{None of the above} \)

You likely thought the values in the interval were not correct.
\end{enumerate}

\textbf{General Comment:} When thinking about this language, it helps to draw a number line and try points.
}
\litem{
Solve the linear inequality below. Then, choose the constant and interval combination that describes the solution set.
\[ \frac{7}{4} - \frac{3}{6} x \geq \frac{5}{2} x - \frac{4}{3} \]The solution is \( (-\infty, 1.028] \), which is option D.\begin{enumerate}[label=\Alph*.]
\item \( (-\infty, a], \text{ where } a \in [-4.03, -0.03] \)

 $(-\infty, -1.028]$, which corresponds to negating the endpoint of the solution.
\item \( [a, \infty), \text{ where } a \in [0.9, 2.8] \)

 $[1.028, \infty)$, which corresponds to switching the direction of the interval. You likely did this if you did not flip the inequality when dividing by a negative!
\item \( [a, \infty), \text{ where } a \in [-2.4, 0.3] \)

 $[-1.028, \infty)$, which corresponds to switching the direction of the interval AND negating the endpoint. You likely did this if you did not flip the inequality when dividing by a negative as well as not moving values over to a side properly.
\item \( (-\infty, a], \text{ where } a \in [0.03, 5.03] \)

* $(-\infty, 1.028]$, which is the correct option.
\item \( \text{None of the above}. \)

You may have chosen this if you thought the inequality did not match the ends of the intervals.
\end{enumerate}

\textbf{General Comment:} Remember that less/greater than or equal to includes the endpoint, while less/greater do not. Also, remember that you need to flip the inequality when you multiply or divide by a negative.
}
\litem{
Using an interval or intervals, describe all the $x$-values within or including a distance of the given values.
\[ \text{ No more than } 3 \text{ units from the number } 4. \]The solution is \( [1, 7] \), which is option C.\begin{enumerate}[label=\Alph*.]
\item \( (1, 7) \)

This describes the values less than 3 from 4
\item \( (-\infty, 1] \cup [7, \infty) \)

This describes the values no less than 3 from 4
\item \( [1, 7] \)

This describes the values no more than 3 from 4
\item \( (-\infty, 1) \cup (7, \infty) \)

This describes the values more than 3 from 4
\item \( \text{None of the above} \)

You likely thought the values in the interval were not correct.
\end{enumerate}

\textbf{General Comment:} When thinking about this language, it helps to draw a number line and try points.
}
\litem{
Solve the linear inequality below. Then, choose the constant and interval combination that describes the solution set.
\[ -9 + 8 x < \frac{75 x + 9}{9} \leq -8 + 6 x \]The solution is \( (-30.00, -3.86] \), which is option A.\begin{enumerate}[label=\Alph*.]
\item \( (a, b], \text{ where } a \in [-31, -28] \text{ and } b \in [-4.86, -1.86] \)

* $(-30.00, -3.86]$, which is the correct option.
\item \( (-\infty, a] \cup (b, \infty), \text{ where } a \in [-35, -29] \text{ and } b \in [-6.86, 1.14] \)

$(-\infty, -30.00] \cup (-3.86, \infty)$, which corresponds to displaying the and-inequality as an or-inequality AND flipping the inequality.
\item \( [a, b), \text{ where } a \in [-31, -27] \text{ and } b \in [-6.86, 0.14] \)

$[-30.00, -3.86)$, which corresponds to flipping the inequality.
\item \( (-\infty, a) \cup [b, \infty), \text{ where } a \in [-33, -29] \text{ and } b \in [-5.86, -0.86] \)

$(-\infty, -30.00) \cup [-3.86, \infty)$, which corresponds to displaying the and-inequality as an or-inequality.
\item \( \text{None of the above.} \)


\end{enumerate}

\textbf{General Comment:} To solve, you will need to break up the compound inequality into two inequalities. Be sure to keep track of the inequality! It may be best to draw a number line and graph your solution.
}
\litem{
Solve the linear inequality below. Then, choose the constant and interval combination that describes the solution set.
\[ 9 + 4 x > 6 x \text{ or } 9 + 7 x < 8 x \]The solution is \( (-\infty, 4.5) \text{ or } (9.0, \infty) \), which is option B.\begin{enumerate}[label=\Alph*.]
\item \( (-\infty, a] \cup [b, \infty), \text{ where } a \in [3.5, 6.5] \text{ and } b \in [7, 10] \)

Corresponds to including the endpoints (when they should be excluded).
\item \( (-\infty, a) \cup (b, \infty), \text{ where } a \in [3.5, 5.5] \text{ and } b \in [5, 10] \)

 * Correct option.
\item \( (-\infty, a] \cup [b, \infty), \text{ where } a \in [-9, -7] \text{ and } b \in [-5.5, -3.5] \)

Corresponds to including the endpoints AND negating.
\item \( (-\infty, a) \cup (b, \infty), \text{ where } a \in [-10, -7] \text{ and } b \in [-8.5, -1.5] \)

Corresponds to inverting the inequality and negating the solution.
\item \( (-\infty, \infty) \)

Corresponds to the variable canceling, which does not happen in this instance.
\end{enumerate}

\textbf{General Comment:} When multiplying or dividing by a negative, flip the sign.
}
\litem{
Solve the linear inequality below. Then, choose the constant and interval combination that describes the solution set.
\[ -8x + 3 < -4x -10 \]The solution is \( (3.25, \infty) \), which is option C.\begin{enumerate}[label=\Alph*.]
\item \( (a, \infty), \text{ where } a \in [-3.25, -0.25] \)

 $(-3.25, \infty)$, which corresponds to negating the endpoint of the solution.
\item \( (-\infty, a), \text{ where } a \in [-2.75, 10.25] \)

 $(-\infty, 3.25)$, which corresponds to switching the direction of the interval. You likely did this if you did not flip the inequality when dividing by a negative!
\item \( (a, \infty), \text{ where } a \in [2.25, 7.25] \)

* $(3.25, \infty)$, which is the correct option.
\item \( (-\infty, a), \text{ where } a \in [-3.25, 1.75] \)

 $(-\infty, -3.25)$, which corresponds to switching the direction of the interval AND negating the endpoint. You likely did this if you did not flip the inequality when dividing by a negative as well as not moving values over to a side properly.
\item \( \text{None of the above}. \)

You may have chosen this if you thought the inequality did not match the ends of the intervals.
\end{enumerate}

\textbf{General Comment:} Remember that less/greater than or equal to includes the endpoint, while less/greater do not. Also, remember that you need to flip the inequality when you multiply or divide by a negative.
}
\end{enumerate}

\end{document}