\documentclass{extbook}[14pt]
\usepackage{multicol, enumerate, enumitem, hyperref, color, soul, setspace, parskip, fancyhdr, amssymb, amsthm, amsmath, latexsym, units, mathtools}
\everymath{\displaystyle}
\usepackage[headsep=0.5cm,headheight=0cm, left=1 in,right= 1 in,top= 1 in,bottom= 1 in]{geometry}
\usepackage{dashrule}  % Package to use the command below to create lines between items
\newcommand{\litem}[1]{\item #1

\rule{\textwidth}{0.4pt}}
\pagestyle{fancy}
\lhead{}
\chead{Answer Key for Progress Quiz 4 Version ALL}
\rhead{}
\lfoot{5346-5907}
\cfoot{}
\rfoot{Summer C 2021}
\begin{document}
\textbf{This key should allow you to understand why you choose the option you did (beyond just getting a question right or wrong). \href{https://xronos.clas.ufl.edu/mac1105spring2020/courseDescriptionAndMisc/Exams/LearningFromResults}{More instructions on how to use this key can be found here}.}

\textbf{If you have a suggestion to make the keys better, \href{https://forms.gle/CZkbZmPbC9XALEE88}{please fill out the short survey here}.}

\textit{Note: This key is auto-generated and may contain issues and/or errors. The keys are reviewed after each exam to ensure grading is done accurately. If there are issues (like duplicate options), they are noted in the offline gradebook. The keys are a work-in-progress to give students as many resources to improve as possible.}

\rule{\textwidth}{0.4pt}

\begin{enumerate}\litem{
Solve the linear inequality below. Then, choose the constant and interval combination that describes the solution set.
\[ -4 - 7 x \leq \frac{-60 x + 3}{9} < 9 - 8 x \]The solution is \( \text{None of the above.} \), which is option E.\begin{enumerate}[label=\Alph*.]
\item \( (-\infty, a] \cup (b, \infty), \text{ where } a \in [10.5, 18.75] \text{ and } b \in [-7.5, -6] \)

$(-\infty, 13.00] \cup (-6.50, \infty)$, which corresponds to displaying the and-inequality as an or-inequality and getting negatives of the actual endpoints.
\item \( (-\infty, a) \cup [b, \infty), \text{ where } a \in [9, 15] \text{ and } b \in [-7.5, -4.5] \)

$(-\infty, 13.00) \cup [-6.50, \infty)$, which corresponds to displaying the and-inequality as an or-inequality AND flipping the inequality AND getting negatives of the actual endpoints.
\item \( (a, b], \text{ where } a \in [10.5, 13.5] \text{ and } b \in [-7.5, -6] \)

$(13.00, -6.50]$, which corresponds to flipping the inequality and getting negatives of the actual endpoints.
\item \( [a, b), \text{ where } a \in [8.25, 14.25] \text{ and } b \in [-8.25, -3.75] \)

$[13.00, -6.50)$, which is the correct interval but negatives of the actual endpoints.
\item \( \text{None of the above.} \)

* This is correct as the answer should be $[-13.00, 6.50)$.
\end{enumerate}

\textbf{General Comment:} To solve, you will need to break up the compound inequality into two inequalities. Be sure to keep track of the inequality! It may be best to draw a number line and graph your solution.
}
\litem{
Solve the linear inequality below. Then, choose the constant and interval combination that describes the solution set.
\[ \frac{3}{2} - \frac{7}{4} x \geq \frac{6}{6} x - \frac{9}{7} \]The solution is \( (-\infty, 1.013] \), which is option B.\begin{enumerate}[label=\Alph*.]
\item \( [a, \infty), \text{ where } a \in [-2.25, 0] \)

 $[-1.013, \infty)$, which corresponds to switching the direction of the interval AND negating the endpoint. You likely did this if you did not flip the inequality when dividing by a negative as well as not moving values over to a side properly.
\item \( (-\infty, a], \text{ where } a \in [0, 3.75] \)

* $(-\infty, 1.013]$, which is the correct option.
\item \( [a, \infty), \text{ where } a \in [-0.75, 2.25] \)

 $[1.013, \infty)$, which corresponds to switching the direction of the interval. You likely did this if you did not flip the inequality when dividing by a negative!
\item \( (-\infty, a], \text{ where } a \in [-2.25, 0] \)

 $(-\infty, -1.013]$, which corresponds to negating the endpoint of the solution.
\item \( \text{None of the above}. \)

You may have chosen this if you thought the inequality did not match the ends of the intervals.
\end{enumerate}

\textbf{General Comment:} Remember that less/greater than or equal to includes the endpoint, while less/greater do not. Also, remember that you need to flip the inequality when you multiply or divide by a negative.
}
\litem{
Solve the linear inequality below. Then, choose the constant and interval combination that describes the solution set.
\[ \frac{-10}{5} - \frac{4}{9} x > \frac{-3}{3} x - \frac{8}{7} \]The solution is \( (1.543, \infty) \), which is option B.\begin{enumerate}[label=\Alph*.]
\item \( (-\infty, a), \text{ where } a \in [0, 4.5] \)

 $(-\infty, 1.543)$, which corresponds to switching the direction of the interval. You likely did this if you did not flip the inequality when dividing by a negative!
\item \( (a, \infty), \text{ where } a \in [0.75, 3.75] \)

* $(1.543, \infty)$, which is the correct option.
\item \( (-\infty, a), \text{ where } a \in [-2.25, 0.75] \)

 $(-\infty, -1.543)$, which corresponds to switching the direction of the interval AND negating the endpoint. You likely did this if you did not flip the inequality when dividing by a negative as well as not moving values over to a side properly.
\item \( (a, \infty), \text{ where } a \in [-5.25, 0] \)

 $(-1.543, \infty)$, which corresponds to negating the endpoint of the solution.
\item \( \text{None of the above}. \)

You may have chosen this if you thought the inequality did not match the ends of the intervals.
\end{enumerate}

\textbf{General Comment:} Remember that less/greater than or equal to includes the endpoint, while less/greater do not. Also, remember that you need to flip the inequality when you multiply or divide by a negative.
}
\litem{
Using an interval or intervals, describe all the $x$-values within or including a distance of the given values.
\[ \text{ No more than } 8 \text{ units from the number } 5. \]The solution is \( [-3, 13] \), which is option D.\begin{enumerate}[label=\Alph*.]
\item \( (-\infty, -3) \cup (13, \infty) \)

This describes the values more than 8 from 5
\item \( (-\infty, -3] \cup [13, \infty) \)

This describes the values no less than 8 from 5
\item \( (-3, 13) \)

This describes the values less than 8 from 5
\item \( [-3, 13] \)

This describes the values no more than 8 from 5
\item \( \text{None of the above} \)

You likely thought the values in the interval were not correct.
\end{enumerate}

\textbf{General Comment:} When thinking about this language, it helps to draw a number line and try points.
}
\litem{
Solve the linear inequality below. Then, choose the constant and interval combination that describes the solution set.
\[ 5 + 6 x > 8 x \text{ or } 9 + 8 x < 9 x \]The solution is \( (-\infty, 2.5) \text{ or } (9.0, \infty) \), which is option C.\begin{enumerate}[label=\Alph*.]
\item \( (-\infty, a] \cup [b, \infty), \text{ where } a \in [-0.75, 6] \text{ and } b \in [6, 9.75] \)

Corresponds to including the endpoints (when they should be excluded).
\item \( (-\infty, a) \cup (b, \infty), \text{ where } a \in [-10.5, -8.25] \text{ and } b \in [-4.5, 0] \)

Corresponds to inverting the inequality and negating the solution.
\item \( (-\infty, a) \cup (b, \infty), \text{ where } a \in [-0.75, 3] \text{ and } b \in [4.5, 9.75] \)

 * Correct option.
\item \( (-\infty, a] \cup [b, \infty), \text{ where } a \in [-10.5, -6] \text{ and } b \in [-5.25, -1.5] \)

Corresponds to including the endpoints AND negating.
\item \( (-\infty, \infty) \)

Corresponds to the variable canceling, which does not happen in this instance.
\end{enumerate}

\textbf{General Comment:} When multiplying or dividing by a negative, flip the sign.
}
\litem{
Using an interval or intervals, describe all the $x$-values within or including a distance of the given values.
\[ \text{ No less than } 6 \text{ units from the number } -7. \]The solution is \( (-\infty, -13] \cup [-1, \infty) \), which is option D.\begin{enumerate}[label=\Alph*.]
\item \( (-\infty, -13) \cup (-1, \infty) \)

This describes the values more than 6 from -7
\item \( (-13, -1) \)

This describes the values less than 6 from -7
\item \( [-13, -1] \)

This describes the values no more than 6 from -7
\item \( (-\infty, -13] \cup [-1, \infty) \)

This describes the values no less than 6 from -7
\item \( \text{None of the above} \)

You likely thought the values in the interval were not correct.
\end{enumerate}

\textbf{General Comment:} When thinking about this language, it helps to draw a number line and try points.
}
\litem{
Solve the linear inequality below. Then, choose the constant and interval combination that describes the solution set.
\[ -6 + 3 x \leq \frac{12 x - 6}{3} < 9 - 4 x \]The solution is \( [-4.00, 1.38) \), which is option A.\begin{enumerate}[label=\Alph*.]
\item \( [a, b), \text{ where } a \in [-10.5, 3.75] \text{ and } b \in [0, 5.25] \)

$[-4.00, 1.38)$, which is the correct option.
\item \( (a, b], \text{ where } a \in [-6.75, -2.25] \text{ and } b \in [0.75, 4.5] \)

$(-4.00, 1.38]$, which corresponds to flipping the inequality.
\item \( (-\infty, a] \cup (b, \infty), \text{ where } a \in [-7.5, 2.25] \text{ and } b \in [0, 5.25] \)

$(-\infty, -4.00] \cup (1.38, \infty)$, which corresponds to displaying the and-inequality as an or-inequality.
\item \( (-\infty, a) \cup [b, \infty), \text{ where } a \in [-5.25, -1.5] \text{ and } b \in [0.97, 1.95] \)

$(-\infty, -4.00) \cup [1.38, \infty)$, which corresponds to displaying the and-inequality as an or-inequality AND flipping the inequality.
\item \( \text{None of the above.} \)


\end{enumerate}

\textbf{General Comment:} To solve, you will need to break up the compound inequality into two inequalities. Be sure to keep track of the inequality! It may be best to draw a number line and graph your solution.
}
\litem{
Solve the linear inequality below. Then, choose the constant and interval combination that describes the solution set.
\[ 5x -4 \leq 10x -6 \]The solution is \( [0.4, \infty) \), which is option B.\begin{enumerate}[label=\Alph*.]
\item \( (-\infty, a], \text{ where } a \in [-0.66, 0.16] \)

 $(-\infty, -0.4]$, which corresponds to switching the direction of the interval AND negating the endpoint. You likely did this if you did not flip the inequality when dividing by a negative as well as not moving values over to a side properly.
\item \( [a, \infty), \text{ where } a \in [0.16, 1.52] \)

* $[0.4, \infty)$, which is the correct option.
\item \( [a, \infty), \text{ where } a \in [-1.1, -0.18] \)

 $[-0.4, \infty)$, which corresponds to negating the endpoint of the solution.
\item \( (-\infty, a], \text{ where } a \in [0.27, 2.08] \)

 $(-\infty, 0.4]$, which corresponds to switching the direction of the interval. You likely did this if you did not flip the inequality when dividing by a negative!
\item \( \text{None of the above}. \)

You may have chosen this if you thought the inequality did not match the ends of the intervals.
\end{enumerate}

\textbf{General Comment:} Remember that less/greater than or equal to includes the endpoint, while less/greater do not. Also, remember that you need to flip the inequality when you multiply or divide by a negative.
}
\litem{
Solve the linear inequality below. Then, choose the constant and interval combination that describes the solution set.
\[ 8x + 8 \geq 10x -10 \]The solution is \( (-\infty, 9.0] \), which is option C.\begin{enumerate}[label=\Alph*.]
\item \( (-\infty, a], \text{ where } a \in [-14, -3] \)

 $(-\infty, -9.0]$, which corresponds to negating the endpoint of the solution.
\item \( [a, \infty), \text{ where } a \in [-9, -4] \)

 $[-9.0, \infty)$, which corresponds to switching the direction of the interval AND negating the endpoint. You likely did this if you did not flip the inequality when dividing by a negative as well as not moving values over to a side properly.
\item \( (-\infty, a], \text{ where } a \in [9, 10] \)

* $(-\infty, 9.0]$, which is the correct option.
\item \( [a, \infty), \text{ where } a \in [8, 10] \)

 $[9.0, \infty)$, which corresponds to switching the direction of the interval. You likely did this if you did not flip the inequality when dividing by a negative!
\item \( \text{None of the above}. \)

You may have chosen this if you thought the inequality did not match the ends of the intervals.
\end{enumerate}

\textbf{General Comment:} Remember that less/greater than or equal to includes the endpoint, while less/greater do not. Also, remember that you need to flip the inequality when you multiply or divide by a negative.
}
\litem{
Solve the linear inequality below. Then, choose the constant and interval combination that describes the solution set.
\[ -6 + 9 x > 12 x \text{ or } 6 + 5 x < 7 x \]The solution is \( (-\infty, -2.0) \text{ or } (3.0, \infty) \), which is option B.\begin{enumerate}[label=\Alph*.]
\item \( (-\infty, a] \cup [b, \infty), \text{ where } a \in [-3.6, -2.62] \text{ and } b \in [1.3, 2.5] \)

Corresponds to including the endpoints AND negating.
\item \( (-\infty, a) \cup (b, \infty), \text{ where } a \in [-2.56, -1.27] \text{ and } b \in [2.85, 3.9] \)

 * Correct option.
\item \( (-\infty, a] \cup [b, \infty), \text{ where } a \in [-2.7, -1.57] \text{ and } b \in [2.39, 4.02] \)

Corresponds to including the endpoints (when they should be excluded).
\item \( (-\infty, a) \cup (b, \infty), \text{ where } a \in [-3.4, -2.6] \text{ and } b \in [0.53, 2.55] \)

Corresponds to inverting the inequality and negating the solution.
\item \( (-\infty, \infty) \)

Corresponds to the variable canceling, which does not happen in this instance.
\end{enumerate}

\textbf{General Comment:} When multiplying or dividing by a negative, flip the sign.
}
\litem{
Solve the linear inequality below. Then, choose the constant and interval combination that describes the solution set.
\[ -8 + 9 x < \frac{30 x - 8}{3} \leq -9 + 6 x \]The solution is \( (-5.33, -1.58] \), which is option A.\begin{enumerate}[label=\Alph*.]
\item \( (a, b], \text{ where } a \in [-7.5, -3.75] \text{ and } b \in [-6, 1.5] \)

* $(-5.33, -1.58]$, which is the correct option.
\item \( (-\infty, a] \cup (b, \infty), \text{ where } a \in [-9.75, -1.5] \text{ and } b \in [-8.25, -0.75] \)

$(-\infty, -5.33] \cup (-1.58, \infty)$, which corresponds to displaying the and-inequality as an or-inequality AND flipping the inequality.
\item \( [a, b), \text{ where } a \in [-6.75, -4.5] \text{ and } b \in [-3, 0] \)

$[-5.33, -1.58)$, which corresponds to flipping the inequality.
\item \( (-\infty, a) \cup [b, \infty), \text{ where } a \in [-6, -3.75] \text{ and } b \in [-4.5, -0.75] \)

$(-\infty, -5.33) \cup [-1.58, \infty)$, which corresponds to displaying the and-inequality as an or-inequality.
\item \( \text{None of the above.} \)


\end{enumerate}

\textbf{General Comment:} To solve, you will need to break up the compound inequality into two inequalities. Be sure to keep track of the inequality! It may be best to draw a number line and graph your solution.
}
\litem{
Solve the linear inequality below. Then, choose the constant and interval combination that describes the solution set.
\[ \frac{9}{2} - \frac{5}{4} x \geq \frac{-4}{8} x - \frac{5}{9} \]The solution is \( (-\infty, 6.741] \), which is option A.\begin{enumerate}[label=\Alph*.]
\item \( (-\infty, a], \text{ where } a \in [6, 8.25] \)

* $(-\infty, 6.741]$, which is the correct option.
\item \( (-\infty, a], \text{ where } a \in [-10.5, -3.75] \)

 $(-\infty, -6.741]$, which corresponds to negating the endpoint of the solution.
\item \( [a, \infty), \text{ where } a \in [6, 8.25] \)

 $[6.741, \infty)$, which corresponds to switching the direction of the interval. You likely did this if you did not flip the inequality when dividing by a negative!
\item \( [a, \infty), \text{ where } a \in [-8.25, -5.25] \)

 $[-6.741, \infty)$, which corresponds to switching the direction of the interval AND negating the endpoint. You likely did this if you did not flip the inequality when dividing by a negative as well as not moving values over to a side properly.
\item \( \text{None of the above}. \)

You may have chosen this if you thought the inequality did not match the ends of the intervals.
\end{enumerate}

\textbf{General Comment:} Remember that less/greater than or equal to includes the endpoint, while less/greater do not. Also, remember that you need to flip the inequality when you multiply or divide by a negative.
}
\litem{
Solve the linear inequality below. Then, choose the constant and interval combination that describes the solution set.
\[ \frac{7}{6} - \frac{5}{7} x < \frac{7}{9} x - \frac{3}{5} \]The solution is \( (1.184, \infty) \), which is option C.\begin{enumerate}[label=\Alph*.]
\item \( (-\infty, a), \text{ where } a \in [-4.5, 0.75] \)

 $(-\infty, -1.184)$, which corresponds to switching the direction of the interval AND negating the endpoint. You likely did this if you did not flip the inequality when dividing by a negative as well as not moving values over to a side properly.
\item \( (-\infty, a), \text{ where } a \in [0.75, 3] \)

 $(-\infty, 1.184)$, which corresponds to switching the direction of the interval. You likely did this if you did not flip the inequality when dividing by a negative!
\item \( (a, \infty), \text{ where } a \in [0, 3.75] \)

* $(1.184, \infty)$, which is the correct option.
\item \( (a, \infty), \text{ where } a \in [-3, -0.75] \)

 $(-1.184, \infty)$, which corresponds to negating the endpoint of the solution.
\item \( \text{None of the above}. \)

You may have chosen this if you thought the inequality did not match the ends of the intervals.
\end{enumerate}

\textbf{General Comment:} Remember that less/greater than or equal to includes the endpoint, while less/greater do not. Also, remember that you need to flip the inequality when you multiply or divide by a negative.
}
\litem{
Using an interval or intervals, describe all the $x$-values within or including a distance of the given values.
\[ \text{ No less than } 8 \text{ units from the number } -4. \]The solution is \( (-\infty, -12] \cup [4, \infty) \), which is option C.\begin{enumerate}[label=\Alph*.]
\item \( (-12, 4) \)

This describes the values less than 8 from -4
\item \( (-\infty, -12) \cup (4, \infty) \)

This describes the values more than 8 from -4
\item \( (-\infty, -12] \cup [4, \infty) \)

This describes the values no less than 8 from -4
\item \( [-12, 4] \)

This describes the values no more than 8 from -4
\item \( \text{None of the above} \)

You likely thought the values in the interval were not correct.
\end{enumerate}

\textbf{General Comment:} When thinking about this language, it helps to draw a number line and try points.
}
\litem{
Solve the linear inequality below. Then, choose the constant and interval combination that describes the solution set.
\[ -9 + 3 x > 5 x \text{ or } 3 + 9 x < 12 x \]The solution is \( (-\infty, -4.5) \text{ or } (1.0, \infty) \), which is option C.\begin{enumerate}[label=\Alph*.]
\item \( (-\infty, a] \cup [b, \infty), \text{ where } a \in [-5.25, -3] \text{ and } b \in [-6, 3] \)

Corresponds to including the endpoints (when they should be excluded).
\item \( (-\infty, a] \cup [b, \infty), \text{ where } a \in [-1.5, 3.75] \text{ and } b \in [3.75, 7.5] \)

Corresponds to including the endpoints AND negating.
\item \( (-\infty, a) \cup (b, \infty), \text{ where } a \in [-5.62, -3.82] \text{ and } b \in [-1.5, 1.5] \)

 * Correct option.
\item \( (-\infty, a) \cup (b, \infty), \text{ where } a \in [-3.9, -0.6] \text{ and } b \in [2.25, 6] \)

Corresponds to inverting the inequality and negating the solution.
\item \( (-\infty, \infty) \)

Corresponds to the variable canceling, which does not happen in this instance.
\end{enumerate}

\textbf{General Comment:} When multiplying or dividing by a negative, flip the sign.
}
\litem{
Using an interval or intervals, describe all the $x$-values within or including a distance of the given values.
\[ \text{ No more than } 9 \text{ units from the number } 1. \]The solution is \( [-8, 10] \), which is option C.\begin{enumerate}[label=\Alph*.]
\item \( (-\infty, -8) \cup (10, \infty) \)

This describes the values more than 9 from 1
\item \( (-\infty, -8] \cup [10, \infty) \)

This describes the values no less than 9 from 1
\item \( [-8, 10] \)

This describes the values no more than 9 from 1
\item \( (-8, 10) \)

This describes the values less than 9 from 1
\item \( \text{None of the above} \)

You likely thought the values in the interval were not correct.
\end{enumerate}

\textbf{General Comment:} When thinking about this language, it helps to draw a number line and try points.
}
\litem{
Solve the linear inequality below. Then, choose the constant and interval combination that describes the solution set.
\[ -8 - 4 x \leq \frac{-21 x + 7}{6} < -9 - 9 x \]The solution is \( \text{None of the above.} \), which is option E.\begin{enumerate}[label=\Alph*.]
\item \( [a, b), \text{ where } a \in [17.25, 21] \text{ and } b \in [0.67, 5.7] \)

$[18.33, 1.85)$, which is the correct interval but negatives of the actual endpoints.
\item \( (-\infty, a) \cup [b, \infty), \text{ where } a \in [14.25, 21.75] \text{ and } b \in [0, 4.5] \)

$(-\infty, 18.33) \cup [1.85, \infty)$, which corresponds to displaying the and-inequality as an or-inequality AND flipping the inequality AND getting negatives of the actual endpoints.
\item \( (a, b], \text{ where } a \in [14.25, 19.5] \text{ and } b \in [0, 3] \)

$(18.33, 1.85]$, which corresponds to flipping the inequality and getting negatives of the actual endpoints.
\item \( (-\infty, a] \cup (b, \infty), \text{ where } a \in [17.25, 19.5] \text{ and } b \in [0, 4.5] \)

$(-\infty, 18.33] \cup (1.85, \infty)$, which corresponds to displaying the and-inequality as an or-inequality and getting negatives of the actual endpoints.
\item \( \text{None of the above.} \)

* This is correct as the answer should be $[-18.33, -1.85)$.
\end{enumerate}

\textbf{General Comment:} To solve, you will need to break up the compound inequality into two inequalities. Be sure to keep track of the inequality! It may be best to draw a number line and graph your solution.
}
\litem{
Solve the linear inequality below. Then, choose the constant and interval combination that describes the solution set.
\[ 4x -4 < 5x + 3 \]The solution is \( (-7.0, \infty) \), which is option A.\begin{enumerate}[label=\Alph*.]
\item \( (a, \infty), \text{ where } a \in [-7, -1] \)

* $(-7.0, \infty)$, which is the correct option.
\item \( (-\infty, a), \text{ where } a \in [5, 10] \)

 $(-\infty, 7.0)$, which corresponds to switching the direction of the interval AND negating the endpoint. You likely did this if you did not flip the inequality when dividing by a negative as well as not moving values over to a side properly.
\item \( (-\infty, a), \text{ where } a \in [-7, -3] \)

 $(-\infty, -7.0)$, which corresponds to switching the direction of the interval. You likely did this if you did not flip the inequality when dividing by a negative!
\item \( (a, \infty), \text{ where } a \in [4, 10] \)

 $(7.0, \infty)$, which corresponds to negating the endpoint of the solution.
\item \( \text{None of the above}. \)

You may have chosen this if you thought the inequality did not match the ends of the intervals.
\end{enumerate}

\textbf{General Comment:} Remember that less/greater than or equal to includes the endpoint, while less/greater do not. Also, remember that you need to flip the inequality when you multiply or divide by a negative.
}
\litem{
Solve the linear inequality below. Then, choose the constant and interval combination that describes the solution set.
\[ 7x + 8 \leq 10x + 7 \]The solution is \( [0.333, \infty) \), which is option D.\begin{enumerate}[label=\Alph*.]
\item \( (-\infty, a], \text{ where } a \in [-0.6, -0.1] \)

 $(-\infty, -0.333]$, which corresponds to switching the direction of the interval AND negating the endpoint. You likely did this if you did not flip the inequality when dividing by a negative as well as not moving values over to a side properly.
\item \( [a, \infty), \text{ where } a \in [-0.75, 0.22] \)

 $[-0.333, \infty)$, which corresponds to negating the endpoint of the solution.
\item \( (-\infty, a], \text{ where } a \in [-0.3, 1.5] \)

 $(-\infty, 0.333]$, which corresponds to switching the direction of the interval. You likely did this if you did not flip the inequality when dividing by a negative!
\item \( [a, \infty), \text{ where } a \in [-0.14, 0.86] \)

* $[0.333, \infty)$, which is the correct option.
\item \( \text{None of the above}. \)

You may have chosen this if you thought the inequality did not match the ends of the intervals.
\end{enumerate}

\textbf{General Comment:} Remember that less/greater than or equal to includes the endpoint, while less/greater do not. Also, remember that you need to flip the inequality when you multiply or divide by a negative.
}
\litem{
Solve the linear inequality below. Then, choose the constant and interval combination that describes the solution set.
\[ -3 + 9 x > 11 x \text{ or } 8 + 6 x < 7 x \]The solution is \( (-\infty, -1.5) \text{ or } (8.0, \infty) \), which is option D.\begin{enumerate}[label=\Alph*.]
\item \( (-\infty, a] \cup [b, \infty), \text{ where } a \in [-6, -0.75] \text{ and } b \in [3, 10.5] \)

Corresponds to including the endpoints (when they should be excluded).
\item \( (-\infty, a] \cup [b, \infty), \text{ where } a \in [-10.5, -3] \text{ and } b \in [-1.5, 3] \)

Corresponds to including the endpoints AND negating.
\item \( (-\infty, a) \cup (b, \infty), \text{ where } a \in [-9, -6.75] \text{ and } b \in [0, 2.25] \)

Corresponds to inverting the inequality and negating the solution.
\item \( (-\infty, a) \cup (b, \infty), \text{ where } a \in [-6, -0.75] \text{ and } b \in [3.75, 9.75] \)

 * Correct option.
\item \( (-\infty, \infty) \)

Corresponds to the variable canceling, which does not happen in this instance.
\end{enumerate}

\textbf{General Comment:} When multiplying or dividing by a negative, flip the sign.
}
\litem{
Solve the linear inequality below. Then, choose the constant and interval combination that describes the solution set.
\[ -8 + 5 x \leq \frac{20 x - 4}{3} < 7 + 3 x \]The solution is \( \text{None of the above.} \), which is option E.\begin{enumerate}[label=\Alph*.]
\item \( (a, b], \text{ where } a \in [0.75, 9] \text{ and } b \in [-3.75, -1.5] \)

$(4.00, -2.27]$, which corresponds to flipping the inequality and getting negatives of the actual endpoints.
\item \( [a, b), \text{ where } a \in [-2.25, 6.75] \text{ and } b \in [-5.25, -0.75] \)

$[4.00, -2.27)$, which is the correct interval but negatives of the actual endpoints.
\item \( (-\infty, a] \cup (b, \infty), \text{ where } a \in [0.75, 6] \text{ and } b \in [-5.25, -0.75] \)

$(-\infty, 4.00] \cup (-2.27, \infty)$, which corresponds to displaying the and-inequality as an or-inequality and getting negatives of the actual endpoints.
\item \( (-\infty, a) \cup [b, \infty), \text{ where } a \in [3, 4.5] \text{ and } b \in [-8.25, -0.75] \)

$(-\infty, 4.00) \cup [-2.27, \infty)$, which corresponds to displaying the and-inequality as an or-inequality AND flipping the inequality AND getting negatives of the actual endpoints.
\item \( \text{None of the above.} \)

* This is correct as the answer should be $[-4.00, 2.27)$.
\end{enumerate}

\textbf{General Comment:} To solve, you will need to break up the compound inequality into two inequalities. Be sure to keep track of the inequality! It may be best to draw a number line and graph your solution.
}
\litem{
Solve the linear inequality below. Then, choose the constant and interval combination that describes the solution set.
\[ \frac{8}{7} - \frac{10}{9} x \geq \frac{-8}{5} x - \frac{10}{3} \]The solution is \( [-9.156, \infty) \), which is option C.\begin{enumerate}[label=\Alph*.]
\item \( (-\infty, a], \text{ where } a \in [7.5, 9.75] \)

 $(-\infty, 9.156]$, which corresponds to switching the direction of the interval AND negating the endpoint. You likely did this if you did not flip the inequality when dividing by a negative as well as not moving values over to a side properly.
\item \( [a, \infty), \text{ where } a \in [7.5, 15] \)

 $[9.156, \infty)$, which corresponds to negating the endpoint of the solution.
\item \( [a, \infty), \text{ where } a \in [-12, -8.25] \)

* $[-9.156, \infty)$, which is the correct option.
\item \( (-\infty, a], \text{ where } a \in [-12, -4.5] \)

 $(-\infty, -9.156]$, which corresponds to switching the direction of the interval. You likely did this if you did not flip the inequality when dividing by a negative!
\item \( \text{None of the above}. \)

You may have chosen this if you thought the inequality did not match the ends of the intervals.
\end{enumerate}

\textbf{General Comment:} Remember that less/greater than or equal to includes the endpoint, while less/greater do not. Also, remember that you need to flip the inequality when you multiply or divide by a negative.
}
\litem{
Solve the linear inequality below. Then, choose the constant and interval combination that describes the solution set.
\[ \frac{9}{3} - \frac{5}{9} x \geq \frac{3}{8} x + \frac{4}{4} \]The solution is \( (-\infty, 2.149] \), which is option A.\begin{enumerate}[label=\Alph*.]
\item \( (-\infty, a], \text{ where } a \in [1.5, 6] \)

* $(-\infty, 2.149]$, which is the correct option.
\item \( [a, \infty), \text{ where } a \in [-1.5, 5.25] \)

 $[2.149, \infty)$, which corresponds to switching the direction of the interval. You likely did this if you did not flip the inequality when dividing by a negative!
\item \( [a, \infty), \text{ where } a \in [-7.5, -1.5] \)

 $[-2.149, \infty)$, which corresponds to switching the direction of the interval AND negating the endpoint. You likely did this if you did not flip the inequality when dividing by a negative as well as not moving values over to a side properly.
\item \( (-\infty, a], \text{ where } a \in [-5.25, 0] \)

 $(-\infty, -2.149]$, which corresponds to negating the endpoint of the solution.
\item \( \text{None of the above}. \)

You may have chosen this if you thought the inequality did not match the ends of the intervals.
\end{enumerate}

\textbf{General Comment:} Remember that less/greater than or equal to includes the endpoint, while less/greater do not. Also, remember that you need to flip the inequality when you multiply or divide by a negative.
}
\litem{
Using an interval or intervals, describe all the $x$-values within or including a distance of the given values.
\[ \text{ No more than } 3 \text{ units from the number } -3. \]The solution is \( [-6, 0] \), which is option C.\begin{enumerate}[label=\Alph*.]
\item \( (-\infty, -6) \cup (0, \infty) \)

This describes the values more than 3 from -3
\item \( (-\infty, -6] \cup [0, \infty) \)

This describes the values no less than 3 from -3
\item \( [-6, 0] \)

This describes the values no more than 3 from -3
\item \( (-6, 0) \)

This describes the values less than 3 from -3
\item \( \text{None of the above} \)

You likely thought the values in the interval were not correct.
\end{enumerate}

\textbf{General Comment:} When thinking about this language, it helps to draw a number line and try points.
}
\litem{
Solve the linear inequality below. Then, choose the constant and interval combination that describes the solution set.
\[ -8 + 7 x > 9 x \text{ or } -5 + 8 x < 11 x \]The solution is \( (-\infty, -4.0) \text{ or } (-1.667, \infty) \), which is option D.\begin{enumerate}[label=\Alph*.]
\item \( (-\infty, a] \cup [b, \infty), \text{ where } a \in [-5.25, -3.75] \text{ and } b \in [-4.5, 0] \)

Corresponds to including the endpoints (when they should be excluded).
\item \( (-\infty, a] \cup [b, \infty), \text{ where } a \in [0, 2.25] \text{ and } b \in [0.75, 6.75] \)

Corresponds to including the endpoints AND negating.
\item \( (-\infty, a) \cup (b, \infty), \text{ where } a \in [-2.25, 4.5] \text{ and } b \in [-0.75, 9.75] \)

Corresponds to inverting the inequality and negating the solution.
\item \( (-\infty, a) \cup (b, \infty), \text{ where } a \in [-7.5, 0] \text{ and } b \in [-6.75, 2.25] \)

 * Correct option.
\item \( (-\infty, \infty) \)

Corresponds to the variable canceling, which does not happen in this instance.
\end{enumerate}

\textbf{General Comment:} When multiplying or dividing by a negative, flip the sign.
}
\litem{
Using an interval or intervals, describe all the $x$-values within or including a distance of the given values.
\[ \text{ Less than } 9 \text{ units from the number } 4. \]The solution is \( (-5, 13) \), which is option C.\begin{enumerate}[label=\Alph*.]
\item \( (-\infty, -5] \cup [13, \infty) \)

This describes the values no less than 9 from 4
\item \( (-\infty, -5) \cup (13, \infty) \)

This describes the values more than 9 from 4
\item \( (-5, 13) \)

This describes the values less than 9 from 4
\item \( [-5, 13] \)

This describes the values no more than 9 from 4
\item \( \text{None of the above} \)

You likely thought the values in the interval were not correct.
\end{enumerate}

\textbf{General Comment:} When thinking about this language, it helps to draw a number line and try points.
}
\litem{
Solve the linear inequality below. Then, choose the constant and interval combination that describes the solution set.
\[ 8 - 4 x \leq \frac{23 x - 6}{7} < 9 + 3 x \]The solution is \( [1.22, 34.50) \), which is option D.\begin{enumerate}[label=\Alph*.]
\item \( (-\infty, a] \cup (b, \infty), \text{ where } a \in [0, 1.5] \text{ and } b \in [33, 40.5] \)

$(-\infty, 1.22] \cup (34.50, \infty)$, which corresponds to displaying the and-inequality as an or-inequality.
\item \( (-\infty, a) \cup [b, \infty), \text{ where } a \in [0, 4.5] \text{ and } b \in [33, 39] \)

$(-\infty, 1.22) \cup [34.50, \infty)$, which corresponds to displaying the and-inequality as an or-inequality AND flipping the inequality.
\item \( (a, b], \text{ where } a \in [0, 3.75] \text{ and } b \in [32.25, 35.25] \)

$(1.22, 34.50]$, which corresponds to flipping the inequality.
\item \( [a, b), \text{ where } a \in [0.97, 2.77] \text{ and } b \in [33.75, 36.75] \)

$[1.22, 34.50)$, which is the correct option.
\item \( \text{None of the above.} \)


\end{enumerate}

\textbf{General Comment:} To solve, you will need to break up the compound inequality into two inequalities. Be sure to keep track of the inequality! It may be best to draw a number line and graph your solution.
}
\litem{
Solve the linear inequality below. Then, choose the constant and interval combination that describes the solution set.
\[ -3x -8 \leq 9x + 7 \]The solution is \( [-1.25, \infty) \), which is option B.\begin{enumerate}[label=\Alph*.]
\item \( (-\infty, a], \text{ where } a \in [-2.25, 0.75] \)

 $(-\infty, -1.25]$, which corresponds to switching the direction of the interval. You likely did this if you did not flip the inequality when dividing by a negative!
\item \( [a, \infty), \text{ where } a \in [-2.6, 0.3] \)

* $[-1.25, \infty)$, which is the correct option.
\item \( (-\infty, a], \text{ where } a \in [0.25, 2.25] \)

 $(-\infty, 1.25]$, which corresponds to switching the direction of the interval AND negating the endpoint. You likely did this if you did not flip the inequality when dividing by a negative as well as not moving values over to a side properly.
\item \( [a, \infty), \text{ where } a \in [-0.1, 1.5] \)

 $[1.25, \infty)$, which corresponds to negating the endpoint of the solution.
\item \( \text{None of the above}. \)

You may have chosen this if you thought the inequality did not match the ends of the intervals.
\end{enumerate}

\textbf{General Comment:} Remember that less/greater than or equal to includes the endpoint, while less/greater do not. Also, remember that you need to flip the inequality when you multiply or divide by a negative.
}
\litem{
Solve the linear inequality below. Then, choose the constant and interval combination that describes the solution set.
\[ -4x + 3 > 3x -5 \]The solution is \( (-\infty, 1.143) \), which is option B.\begin{enumerate}[label=\Alph*.]
\item \( (a, \infty), \text{ where } a \in [-6.2, 1.1] \)

 $(-1.143, \infty)$, which corresponds to switching the direction of the interval AND negating the endpoint. You likely did this if you did not flip the inequality when dividing by a negative as well as not moving values over to a side properly.
\item \( (-\infty, a), \text{ where } a \in [0.14, 3.14] \)

* $(-\infty, 1.143)$, which is the correct option.
\item \( (a, \infty), \text{ where } a \in [-0.7, 3.1] \)

 $(1.143, \infty)$, which corresponds to switching the direction of the interval. You likely did this if you did not flip the inequality when dividing by a negative!
\item \( (-\infty, a), \text{ where } a \in [-9.14, -0.14] \)

 $(-\infty, -1.143)$, which corresponds to negating the endpoint of the solution.
\item \( \text{None of the above}. \)

You may have chosen this if you thought the inequality did not match the ends of the intervals.
\end{enumerate}

\textbf{General Comment:} Remember that less/greater than or equal to includes the endpoint, while less/greater do not. Also, remember that you need to flip the inequality when you multiply or divide by a negative.
}
\litem{
Solve the linear inequality below. Then, choose the constant and interval combination that describes the solution set.
\[ -4 + 9 x > 11 x \text{ or } 5 + 7 x < 8 x \]The solution is \( (-\infty, -2.0) \text{ or } (5.0, \infty) \), which is option C.\begin{enumerate}[label=\Alph*.]
\item \( (-\infty, a) \cup (b, \infty), \text{ where } a \in [-7.35, -3.3] \text{ and } b \in [-0.75, 4.5] \)

Corresponds to inverting the inequality and negating the solution.
\item \( (-\infty, a] \cup [b, \infty), \text{ where } a \in [-3.75, 4.5] \text{ and } b \in [4.5, 9] \)

Corresponds to including the endpoints (when they should be excluded).
\item \( (-\infty, a) \cup (b, \infty), \text{ where } a \in [-2.92, -0.45] \text{ and } b \in [4.5, 6] \)

 * Correct option.
\item \( (-\infty, a] \cup [b, \infty), \text{ where } a \in [-11.25, -3.75] \text{ and } b \in [-0.75, 3] \)

Corresponds to including the endpoints AND negating.
\item \( (-\infty, \infty) \)

Corresponds to the variable canceling, which does not happen in this instance.
\end{enumerate}

\textbf{General Comment:} When multiplying or dividing by a negative, flip the sign.
}
\end{enumerate}

\end{document}