\documentclass{extbook}[14pt]
\usepackage{multicol, enumerate, enumitem, hyperref, color, soul, setspace, parskip, fancyhdr, amssymb, amsthm, amsmath, bbm, latexsym, units, mathtools}
\everymath{\displaystyle}
\usepackage[headsep=0.5cm,headheight=0cm, left=1 in,right= 1 in,top= 1 in,bottom= 1 in]{geometry}
\usepackage{dashrule}  % Package to use the command below to create lines between items
\newcommand{\litem}[1]{\item #1

\rule{\textwidth}{0.4pt}}
\pagestyle{fancy}
\lhead{}
\chead{Answer Key for Progress Quiz 4 Version A}
\rhead{}
\lfoot{4378-7085}
\cfoot{}
\rfoot{Fall 2020}
\begin{document}
\textbf{This key should allow you to understand why you choose the option you did (beyond just getting a question right or wrong). \href{https://xronos.clas.ufl.edu/mac1105spring2020/courseDescriptionAndMisc/Exams/LearningFromResults}{More instructions on how to use this key can be found here}.}

\textbf{If you have a suggestion to make the keys better, \href{https://forms.gle/CZkbZmPbC9XALEE88}{please fill out the short survey here}.}

\textit{Note: This key is auto-generated and may contain issues and/or errors. The keys are reviewed after each exam to ensure grading is done accurately. If there are issues (like duplicate options), they are noted in the offline gradebook. The keys are a work-in-progress to give students as many resources to improve as possible.}

\rule{\textwidth}{0.4pt}

\begin{enumerate}\litem{
Solve the equation for $x$ and choose the interval that contains the solution (if it exists).
\[ \log_{4}{(-4x+7)}+4 = 2 \]
The solution is \( x = 1.734 \), which is option C.\begin{enumerate}[label=\Alph*.]
\item \( x \in [-9.9, -4.4] \)

$x = -5.750$, which corresponds to reversing the base and exponent when converting and reversing the value with $x$.
\item \( x \in [-3.8, -0.6] \)

$x = -2.250$, which corresponds to ignoring the vertical shift when converting to exponential form.
\item \( x \in [-0.4, 2.3] \)

* $x = 1.734$, which is the correct option.
\item \( x \in [-3.8, -0.6] \)

$x = -2.250$, which corresponds to reversing the base and exponent when converting.
\item \( \text{There is no Real solution to the equation.} \)

Corresponds to believing a negative coefficient within the log equation means there is no Real solution.
\end{enumerate}

\textbf{General Comment:} \textbf{General Comments:} First, get the equation in the form $\log_b{(cx+d)} = a$. Then, convert to $b^a = cx+d$ and solve.
}
\litem{
Which of the following intervals describes the Range of the function below?
\[ f(x) = e^{x+5}-3 \]
The solution is \( (-3, \infty) \), which is option B.\begin{enumerate}[label=\Alph*.]
\item \( [a, \infty), a \in [-4, -2] \)

$[-3, \infty)$, which corresponds to including the endpoint.
\item \( (a, \infty), a \in [-4, -2] \)

* $(-3, \infty)$, which is the correct option.
\item \( (-\infty, a], a \in [2, 8] \)

$(-\infty, 3]$, which corresponds to using the negative vertical shift AND flipping the Range interval AND including the endpoint.
\item \( (-\infty, a), a \in [2, 8] \)

$(-\infty, 3)$, which corresponds to using the negative vertical shift AND flipping the Range interval.
\item \( (-\infty, \infty) \)

This corresponds to confusing range of an exponential function with the domain of an exponential function.
\end{enumerate}

\textbf{General Comment:} \textbf{General Comments}: Domain of a basic exponential function is $(-\infty, \infty)$ while the Range is $(0, \infty)$. We can shift these intervals [and even flip when $a<0$!] to find the new Domain/Range.
}
\litem{
Which of the following intervals describes the Domain of the function below?
\[ f(x) = -\log_2{(x+7)}-2 \]
The solution is \( (-7, \infty) \), which is option B.\begin{enumerate}[label=\Alph*.]
\item \( [a, \infty), a \in [-4, 1] \)

$[-2, \infty)$, which corresponds to using the vertical shift when shifting the Domain AND including the endpoint.
\item \( (a, \infty), a \in [-9, -6] \)

* $(-7, \infty)$, which is the correct option.
\item \( (-\infty, a], a \in [-1, 5] \)

$(-\infty, 2]$, which corresponds to using the negative vertical shift AND including the endpoint AND flipping the domain.
\item \( (-\infty, a), a \in [3, 12] \)

$(-\infty, 7)$, which corresponds to flipping the Domain. Remember: the general for is $a*\log(x-h)+k$, \textbf{where $a$ does not affect the domain}.
\item \( (-\infty, \infty) \)

This corresponds to thinking of the range of the log function (or the domain of the exponential function).
\end{enumerate}

\textbf{General Comment:} \textbf{General Comments}: The domain of a basic logarithmic function is $(0, \infty)$ and the Range is $(-\infty, \infty)$. We can use shifts when finding the Domain, but the Range will always be all Real numbers.
}
\litem{
 Solve the equation for $x$ and choose the interval that contains $x$ (if it exists).
\[  5 = \ln{\sqrt[7]{\frac{21}{e^{9x}}}} \]
The solution is \( x = -3.551 \), which is option C.\begin{enumerate}[label=\Alph*.]
\item \( x \in [-1.69, -1.16] \)

$x = -1.590$, which corresponds to thinking you need to take the natural log of on the left before reducing.
\item \( x \in [-0.88, -0.76] \)

$x = -0.773$, which corresponds to treating any root as a square root.
\item \( x \in [-3.98, -3.29] \)

* $x = -3.551$, which is the correct option.
\item \( \text{There is no Real solution to the equation.} \)

This corresponds to believing you cannot solve the equation.
\item \( \text{None of the above.} \)

This corresponds to making an unexpected error.
\end{enumerate}

\textbf{General Comment:} \textbf{General Comments}: After using the properties of logarithmic functions to break up the right-hand side, use $\ln(e) = 1$ to reduce the question to a linear function to solve. You can put $\ln(21)$ into a calculator if you are having trouble.
}
\litem{
Which of the following intervals describes the Range of the function below?
\[ f(x) = \log_2{(x-8)}+5 \]
The solution is \( (\infty, \infty) \), which is option E.\begin{enumerate}[label=\Alph*.]
\item \( (-\infty, a), a \in [-6.1, -3.7] \)

$(-\infty, -5)$, which corresponds to using the using the negative of vertical shift on $(0, \infty)$.
\item \( [a, \infty), a \in [-10.8, -7.8] \)

$[-8, \infty)$, which corresponds to using the negative of the horizontal shift AND including the endpoint.
\item \( (-\infty, a), a \in [3.7, 6.8] \)

$(-\infty, 5)$, which corresponds to using the vertical shift while the Range is $(-\infty, \infty)$.
\item \( [a, \infty), a \in [5.2, 11] \)

$[5, \infty)$, which corresponds to using the flipped Domain AND including the endpoint.
\item \( (-\infty, \infty) \)

*This is the correct option.
\end{enumerate}

\textbf{General Comment:} \textbf{General Comments}: The domain of a basic logarithmic function is $(0, \infty)$ and the Range is $(-\infty, \infty)$. We can use shifts when finding the Domain, but the Range will always be all Real numbers.
}
\litem{
Solve the equation for $x$ and choose the interval that contains the solution (if it exists).
\[ \log_{4}{(3x+7)}+4 = 2 \]
The solution is \( x = -2.312 \), which is option A.\begin{enumerate}[label=\Alph*.]
\item \( x \in [-4.1, 0.1] \)

* $x = -2.312$, which is the correct option.
\item \( x \in [1, 7] \)

$x = 3.000$, which corresponds to reversing the base and exponent when converting.
\item \( x \in [1, 7] \)

$x = 3.000$, which corresponds to ignoring the vertical shift when converting to exponential form.
\item \( x \in [6.2, 7.8] \)

$x = 7.667$, which corresponds to reversing the base and exponent when converting and reversing the value with $x$.
\item \( \text{There is no Real solution to the equation.} \)

Corresponds to believing a negative coefficient within the log equation means there is no Real solution.
\end{enumerate}

\textbf{General Comment:} \textbf{General Comments:} First, get the equation in the form $\log_b{(cx+d)} = a$. Then, convert to $b^a = cx+d$ and solve.
}
\litem{
Solve the equation for $x$ and choose the interval that contains the solution (if it exists).
\[ 5^{2x-4} = \left(\frac{1}{49}\right)^{4x+5} \]
The solution is \( x = -0.693 \), which is option D.\begin{enumerate}[label=\Alph*.]
\item \( x \in [-0.3, 0.8] \)

$x = 0.479$, which corresponds to distributing the $\ln(base)$ to the first term of the exponent only.
\item \( x \in [3.7, 8] \)

$x = 6.511$, which corresponds to distributing the $\ln(base)$ to the second term of the exponent only.
\item \( x \in [-5.1, -3] \)

$x = -4.500$, which corresponds to solving the numerators as equal while ignoring the bases are different.
\item \( x \in [-2.7, 0.4] \)

* $x = -0.693$, which is the correct option.
\item \( \text{There is no Real solution to the equation.} \)

This corresponds to believing there is no solution since the bases are not powers of each other.
\end{enumerate}

\textbf{General Comment:} \textbf{General Comments:} This question was written so that the bases could not be written the same. You will need to take the log of both sides.
}
\litem{
Which of the following intervals describes the Domain of the function below?
\[ f(x) = e^{x+1}-5 \]
The solution is \( (-\infty, \infty) \), which is option E.\begin{enumerate}[label=\Alph*.]
\item \( [a, \infty), a \in [4, 6] \)

$[5, \infty)$, which corresponds to using the negative vertical shift AND flipping the Range interval AND including the endpoint.
\item \( (-\infty, a), a \in [-10, 0] \)

$(-\infty, -5)$, which corresponds to using the correct vertical shift *if we wanted the Range*.
\item \( (-\infty, a], a \in [-10, 0] \)

$(-\infty, -5]$, which corresponds to using the correct vertical shift *if we wanted the Range* AND including the endpoint.
\item \( (a, \infty), a \in [4, 6] \)

$(5, \infty)$, which corresponds to using the negative vertical shift AND flipping the Range interval.
\item \( (-\infty, \infty) \)

* This is the correct option.
\end{enumerate}

\textbf{General Comment:} \textbf{General Comments}: Domain of a basic exponential function is $(-\infty, \infty)$ while the Range is $(0, \infty)$. We can shift these intervals [and even flip when $a<0$!] to find the new Domain/Range.
}
\litem{
 Solve the equation for $x$ and choose the interval that contains $x$ (if it exists).
\[  15 = \ln{\sqrt[4]{\frac{11}{e^{8x}}}} \]
The solution is \( x = -7.2, \text{ which does not fit in any of the interval options.} \), which is option E.\begin{enumerate}[label=\Alph*.]
\item \( x \in [-1.8, -1] \)

$x = -1.654$, which corresponds to thinking you need to take the natural log of the left side before reducing.
\item \( x \in [7.1, 7.7] \)

$x = 7.200$, which is the negative of the correct solution.
\item \( x \in [-5, -2.8] \)

$x = -3.450$, which corresponds to treating any root as a square root.
\item \( \text{There is no Real solution to the equation.} \)

This corresponds to believing you cannot solve the equation.
\item \( \text{None of the above.} \)

*$x = -7.200$ is the correct solution and does not fit in any of the other intervals.
\end{enumerate}

\textbf{General Comment:} \textbf{General Comments}: After using the properties of logarithmic functions to break up the right-hand side, use $\ln(e) = 1$ to reduce the question to a linear function to solve. You can put $\ln(11)$ into a calculator if you are having trouble.
}
\litem{
Solve the equation for $x$ and choose the interval that contains the solution (if it exists).
\[ 4^{3x-5} = 9^{2x+2} \]
The solution is \( x = -48.080 \), which is option C.\begin{enumerate}[label=\Alph*.]
\item \( x \in [6, 8] \)

$x = 7.000$, which corresponds to solving the numerators as equal while ignoring the bases are different.
\item \( x \in [9.33, 12.33] \)

$x = 11.326$, which corresponds to distributing the $\ln(base)$ to the second term of the exponent only.
\item \( x \in [-48.08, -46.08] \)

* $x = -48.080$, which is the correct option.
\item \( x \in [-29.72, -28.72] \)

$x = -29.716$, which corresponds to distributing the $\ln(base)$ to the first term of the exponent only.
\item \( \text{There is no Real solution to the equation.} \)

This corresponds to believing there is no solution since the bases are not powers of each other.
\end{enumerate}

\textbf{General Comment:} \textbf{General Comments:} This question was written so that the bases could not be written the same. You will need to take the log of both sides.
}
\end{enumerate}

\end{document}