\documentclass{extbook}[14pt]
\usepackage{multicol, enumerate, enumitem, hyperref, color, soul, setspace, parskip, fancyhdr, amssymb, amsthm, amsmath, bbm, latexsym, units, mathtools}
\everymath{\displaystyle}
\usepackage[headsep=0.5cm,headheight=0cm, left=1 in,right= 1 in,top= 1 in,bottom= 1 in]{geometry}
\usepackage{dashrule}  % Package to use the command below to create lines between items
\newcommand{\litem}[1]{\item #1

\rule{\textwidth}{0.4pt}}
\pagestyle{fancy}
\lhead{}
\chead{Answer Key for Progress Quiz 4 Version A}
\rhead{}
\lfoot{9187-5854}
\cfoot{}
\rfoot{Spring 2021}
\begin{document}
\textbf{This key should allow you to understand why you choose the option you did (beyond just getting a question right or wrong). \href{https://xronos.clas.ufl.edu/mac1105spring2020/courseDescriptionAndMisc/Exams/LearningFromResults}{More instructions on how to use this key can be found here}.}

\textbf{If you have a suggestion to make the keys better, \href{https://forms.gle/CZkbZmPbC9XALEE88}{please fill out the short survey here}.}

\textit{Note: This key is auto-generated and may contain issues and/or errors. The keys are reviewed after each exam to ensure grading is done accurately. If there are issues (like duplicate options), they are noted in the offline gradebook. The keys are a work-in-progress to give students as many resources to improve as possible.}

\rule{\textwidth}{0.4pt}

\begin{enumerate}\litem{
Which of the following intervals describes the Range of the function below?
\[ f(x) = \log_2{(x+8)}+5 \]The solution is \( (\infty, \infty) \), which is option E.\begin{enumerate}[label=\Alph*.]
\item \( (-\infty, a), a \in [-6.4, -2.8] \)

$(-\infty, -5)$, which corresponds to using the using the negative of vertical shift on $(0, \infty)$.
\item \( [a, \infty), a \in [7, 8.8] \)

$[8, \infty)$, which corresponds to using the negative of the horizontal shift AND including the endpoint.
\item \( (-\infty, a), a \in [2.9, 7.4] \)

$(-\infty, 5)$, which corresponds to using the vertical shift while the Range is $(-\infty, \infty)$.
\item \( [a, \infty), a \in [-8.6, -6.9] \)

$[5, \infty)$, which corresponds to using the flipped Domain AND including the endpoint.
\item \( (-\infty, \infty) \)

*This is the correct option.
\end{enumerate}

\textbf{General Comment:} \textbf{General Comments}: The domain of a basic logarithmic function is $(0, \infty)$ and the Range is $(-\infty, \infty)$. We can use shifts when finding the Domain, but the Range will always be all Real numbers.
}
\litem{
Solve the equation for $x$ and choose the interval that contains the solution (if it exists).
\[ 4^{3x-3} = 27^{2x+2} \]The solution is \( x = -4.419 \), which is option D.\begin{enumerate}[label=\Alph*.]
\item \( x \in [10.5, 11.9] \)

$x = 10.751$, which corresponds to distributing the $\ln(base)$ to the second term of the exponent only.
\item \( x \in [-3.4, -1.2] \)

$x = -2.055$, which corresponds to distributing the $\ln(base)$ to the first term of the exponent only.
\item \( x \in [4.6, 6.9] \)

$x = 5.000$, which corresponds to solving the numerators as equal while ignoring the bases are different.
\item \( x \in [-5.9, -2.5] \)

* $x = -4.419$, which is the correct option.
\item \( \text{There is no Real solution to the equation.} \)

This corresponds to believing there is no solution since the bases are not powers of each other.
\end{enumerate}

\textbf{General Comment:} \textbf{General Comments:} This question was written so that the bases could not be written the same. You will need to take the log of both sides.
}
\litem{
Which of the following intervals describes the Range of the function below?
\[ f(x) = -e^{x+3}+2 \]The solution is \( (-\infty, 2) \), which is option B.\begin{enumerate}[label=\Alph*.]
\item \( (a, \infty), a \in [-4, -1] \)

$(-2, \infty)$, which corresponds to using the negative vertical shift AND flipping the Range interval.
\item \( (-\infty, a), a \in [0, 4] \)

* $(-\infty, 2)$, which is the correct option.
\item \( (-\infty, a], a \in [0, 4] \)

$(-\infty, 2]$, which corresponds to including the endpoint.
\item \( [a, \infty), a \in [-4, -1] \)

$[-2, \infty)$, which corresponds to using the negative vertical shift AND flipping the Range interval AND including the endpoint.
\item \( (-\infty, \infty) \)

This corresponds to confusing range of an exponential function with the domain of an exponential function.
\end{enumerate}

\textbf{General Comment:} \textbf{General Comments}: Domain of a basic exponential function is $(-\infty, \infty)$ while the Range is $(0, \infty)$. We can shift these intervals [and even flip when $a<0$!] to find the new Domain/Range.
}
\litem{
Solve the equation for $x$ and choose the interval that contains the solution (if it exists).
\[ \log_{4}{(2x+6)}+6 = 2 \]The solution is \( x = -2.998 \), which is option B.\begin{enumerate}[label=\Alph*.]
\item \( x \in [5, 10] \)

$x = 5.000$, which corresponds to ignoring the vertical shift when converting to exponential form.
\item \( x \in [-9, 3] \)

* $x = -2.998$, which is the correct option.
\item \( x \in [130, 133] \)

$x = 131.000$, which corresponds to reversing the base and exponent when converting and reversing the value with $x$.
\item \( x \in [123, 128] \)

$x = 125.000$, which corresponds to reversing the base and exponent when converting.
\item \( \text{There is no Real solution to the equation.} \)

Corresponds to believing a negative coefficient within the log equation means there is no Real solution.
\end{enumerate}

\textbf{General Comment:} \textbf{General Comments:} First, get the equation in the form $\log_b{(cx+d)} = a$. Then, convert to $b^a = cx+d$ and solve.
}
\litem{
Which of the following intervals describes the Range of the function below?
\[ f(x) = -e^{x+7}-2 \]The solution is \( (-\infty, -2) \), which is option C.\begin{enumerate}[label=\Alph*.]
\item \( [a, \infty), a \in [1.9, 6.7] \)

$[2, \infty)$, which corresponds to using the negative vertical shift AND flipping the Range interval AND including the endpoint.
\item \( (a, \infty), a \in [1.9, 6.7] \)

$(2, \infty)$, which corresponds to using the negative vertical shift AND flipping the Range interval.
\item \( (-\infty, a), a \in [-2.9, -0.8] \)

* $(-\infty, -2)$, which is the correct option.
\item \( (-\infty, a], a \in [-2.9, -0.8] \)

$(-\infty, -2]$, which corresponds to including the endpoint.
\item \( (-\infty, \infty) \)

This corresponds to confusing range of an exponential function with the domain of an exponential function.
\end{enumerate}

\textbf{General Comment:} \textbf{General Comments}: Domain of a basic exponential function is $(-\infty, \infty)$ while the Range is $(0, \infty)$. We can shift these intervals [and even flip when $a<0$!] to find the new Domain/Range.
}
\litem{
 Solve the equation for $x$ and choose the interval that contains $x$ (if it exists).
\[  17 = \ln{\sqrt[5]{\frac{15}{e^{8x}}}} \]The solution is \( x = -10.286 \), which is option B.\begin{enumerate}[label=\Alph*.]
\item \( x \in [-2.7, 0] \)

$x = -2.109$, which corresponds to thinking you need to take the natural log of on the left before reducing.
\item \( x \in [-10.5, -9.6] \)

* $x = -10.286$, which is the correct option.
\item \( x \in [-5.3, -2.9] \)

$x = -3.911$, which corresponds to treating any root as a square root.
\item \( \text{There is no Real solution to the equation.} \)

This corresponds to believing you cannot solve the equation.
\item \( \text{None of the above.} \)

This corresponds to making an unexpected error.
\end{enumerate}

\textbf{General Comment:} \textbf{General Comments}: After using the properties of logarithmic functions to break up the right-hand side, use $\ln(e) = 1$ to reduce the question to a linear function to solve. You can put $\ln(15)$ into a calculator if you are having trouble.
}
\litem{
Solve the equation for $x$ and choose the interval that contains the solution (if it exists).
\[ \log_{5}{(2x+6)}+5 = 2 \]The solution is \( x = -2.996 \), which is option A.\begin{enumerate}[label=\Alph*.]
\item \( x \in [-5, -2] \)

* $x = -2.996$, which is the correct option.
\item \( x \in [-128.5, -120.5] \)

$x = -124.500$, which corresponds to reversing the base and exponent when converting.
\item \( x \in [7.5, 13.5] \)

$x = 9.500$, which corresponds to ignoring the vertical shift when converting to exponential form.
\item \( x \in [-121.5, -109.5] \)

$x = -118.500$, which corresponds to reversing the base and exponent when converting and reversing the value with $x$.
\item \( \text{There is no Real solution to the equation.} \)

Corresponds to believing a negative coefficient within the log equation means there is no Real solution.
\end{enumerate}

\textbf{General Comment:} \textbf{General Comments:} First, get the equation in the form $\log_b{(cx+d)} = a$. Then, convert to $b^a = cx+d$ and solve.
}
\litem{
Which of the following intervals describes the Range of the function below?
\[ f(x) = \log_2{(x+7)}-4 \]The solution is \( (\infty, \infty) \), which is option E.\begin{enumerate}[label=\Alph*.]
\item \( (-\infty, a), a \in [-5.04, -3.6] \)

$(-\infty, -4)$, which corresponds to using the vertical shift while the Range is $(-\infty, \infty)$.
\item \( [a, \infty), a \in [5.64, 7.65] \)

$[7, \infty)$, which corresponds to using the negative of the horizontal shift AND including the endpoint.
\item \( (-\infty, a), a \in [2.7, 4.33] \)

$(-\infty, 4)$, which corresponds to using the using the negative of vertical shift on $(0, \infty)$.
\item \( [a, \infty), a \in [-7.79, -6.84] \)

$[-4, \infty)$, which corresponds to using the flipped Domain AND including the endpoint.
\item \( (-\infty, \infty) \)

*This is the correct option.
\end{enumerate}

\textbf{General Comment:} \textbf{General Comments}: The domain of a basic logarithmic function is $(0, \infty)$ and the Range is $(-\infty, \infty)$. We can use shifts when finding the Domain, but the Range will always be all Real numbers.
}
\litem{
Solve the equation for $x$ and choose the interval that contains the solution (if it exists).
\[ 4^{-3x+3} = \left(\frac{1}{9}\right)^{-2x-4} \]The solution is \( x = -0.541 \), which is option C.\begin{enumerate}[label=\Alph*.]
\item \( x \in [-0.4, 3.2] \)

$x = 0.818$, which corresponds to distributing the $\ln(base)$ to the first term of the exponent only.
\item \( x \in [5.5, 8.2] \)

$x = 7.000$, which corresponds to solving the numerators as equal while ignoring the bases are different.
\item \( x \in [-1.4, 0.3] \)

* $x = -0.541$, which is the correct option.
\item \( x \in [-5.7, -4.4] \)

$x = -4.630$, which corresponds to distributing the $\ln(base)$ to the second term of the exponent only.
\item \( \text{There is no Real solution to the equation.} \)

This corresponds to believing there is no solution since the bases are not powers of each other.
\end{enumerate}

\textbf{General Comment:} \textbf{General Comments:} This question was written so that the bases could not be written the same. You will need to take the log of both sides.
}
\litem{
 Solve the equation for $x$ and choose the interval that contains $x$ (if it exists).
\[  17 = \sqrt[4]{\frac{16}{e^{3x}}} \]The solution is \( x = -2.853 \), which is option C.\begin{enumerate}[label=\Alph*.]
\item \( x \in [-27.59, -22.59] \)

$x = -23.591$, which corresponds to thinking you don't need to take the natural log of both sides before reducing, as if the equation already had a natural log on the right side.
\item \( x \in [-1.96, 2.04] \)

$x = -0.965$, which corresponds to treating any root as a square root.
\item \( x \in [-3.85, -1.85] \)

* $x = -2.853$, which is the correct option.
\item \( \text{There is no Real solution to the equation.} \)

This corresponds to believing you cannot solve the equation.
\item \( \text{None of the above.} \)

This corresponds to making an unexpected error.
\end{enumerate}

\textbf{General Comment:} \textbf{General Comments}: After using the properties of logarithmic functions to break up the right-hand side, use $\ln(e) = 1$ to reduce the question to a linear function to solve. You can put $\ln(16)$ into a calculator if you are having trouble.
}
\end{enumerate}

\end{document}