\documentclass{extbook}[14pt]
\usepackage{multicol, enumerate, enumitem, hyperref, color, soul, setspace, parskip, fancyhdr, amssymb, amsthm, amsmath, latexsym, units, mathtools}
\everymath{\displaystyle}
\usepackage[headsep=0.5cm,headheight=0cm, left=1 in,right= 1 in,top= 1 in,bottom= 1 in]{geometry}
\usepackage{dashrule}  % Package to use the command below to create lines between items
\newcommand{\litem}[1]{\item #1

\rule{\textwidth}{0.4pt}}
\pagestyle{fancy}
\lhead{}
\chead{Answer Key for Progress Quiz 4 Version A}
\rhead{}
\lfoot{5346-5907}
\cfoot{}
\rfoot{Summer C 2021}
\begin{document}
\textbf{This key should allow you to understand why you choose the option you did (beyond just getting a question right or wrong). \href{https://xronos.clas.ufl.edu/mac1105spring2020/courseDescriptionAndMisc/Exams/LearningFromResults}{More instructions on how to use this key can be found here}.}

\textbf{If you have a suggestion to make the keys better, \href{https://forms.gle/CZkbZmPbC9XALEE88}{please fill out the short survey here}.}

\textit{Note: This key is auto-generated and may contain issues and/or errors. The keys are reviewed after each exam to ensure grading is done accurately. If there are issues (like duplicate options), they are noted in the offline gradebook. The keys are a work-in-progress to give students as many resources to improve as possible.}

\rule{\textwidth}{0.4pt}

\begin{enumerate}\litem{
Solve the equation for $x$ and choose the interval that contains the solution (if it exists).
\[ \log_{4}{(2x+5)}+4 = 3 \]The solution is \( x = -2.375 \), which is option C.\begin{enumerate}[label=\Alph*.]
\item \( x \in [-2.34, -1.45] \)

$x = -2.000$, which corresponds to reversing the base and exponent when converting.
\item \( x \in [29.26, 29.94] \)

$x = 29.500$, which corresponds to ignoring the vertical shift when converting to exponential form.
\item \( x \in [-2.97, -2.09] \)

* $x = -2.375$, which is the correct option.
\item \( x \in [2.93, 3.12] \)

$x = 3.000$, which corresponds to reversing the base and exponent when converting and reversing the value with $x$.
\item \( \text{There is no Real solution to the equation.} \)

Corresponds to believing a negative coefficient within the log equation means there is no Real solution.
\end{enumerate}

\textbf{General Comment:} \textbf{General Comments:} First, get the equation in the form $\log_b{(cx+d)} = a$. Then, convert to $b^a = cx+d$ and solve.
}
\litem{
 Solve the equation for $x$ and choose the interval that contains $x$ (if it exists).
\[  11 = \ln{\sqrt[3]{\frac{19}{e^{8x}}}} \]The solution is \( x = -3.757, \text{ which does not fit in any of the interval options.} \), which is option E.\begin{enumerate}[label=\Alph*.]
\item \( x \in [-1.6, -1] \)

$x = -1.267$, which corresponds to thinking you need to take the natural log of the left side before reducing.
\item \( x \in [-3.7, -2.2] \)

$x = -2.382$, which corresponds to treating any root as a square root.
\item \( x \in [3.5, 4.7] \)

$x = 3.757$, which is the negative of the correct solution.
\item \( \text{There is no Real solution to the equation.} \)

This corresponds to believing you cannot solve the equation.
\item \( \text{None of the above.} \)

*$x = -3.757$ is the correct solution and does not fit in any of the other intervals.
\end{enumerate}

\textbf{General Comment:} \textbf{General Comments}: After using the properties of logarithmic functions to break up the right-hand side, use $\ln(e) = 1$ to reduce the question to a linear function to solve. You can put $\ln(19)$ into a calculator if you are having trouble.
}
\litem{
Which of the following intervals describes the Range of the function below?
\[ f(x) = \log_2{(x+7)}+7 \]The solution is \( (\infty, \infty) \), which is option E.\begin{enumerate}[label=\Alph*.]
\item \( (-\infty, a), a \in [-7, -5] \)

$(-\infty, -7)$, which corresponds to using the using the negative of vertical shift on $(0, \infty)$.
\item \( [a, \infty), a \in [6, 8] \)

$[7, \infty)$, which corresponds to using the negative of the horizontal shift AND including the endpoint.
\item \( (-\infty, a), a \in [6, 8] \)

$(-\infty, 7)$, which corresponds to using the vertical shift while the Range is $(-\infty, \infty)$.
\item \( [a, \infty), a \in [-7, -5] \)

$[7, \infty)$, which corresponds to using the flipped Domain AND including the endpoint.
\item \( (-\infty, \infty) \)

*This is the correct option.
\end{enumerate}

\textbf{General Comment:} \textbf{General Comments}: The domain of a basic logarithmic function is $(0, \infty)$ and the Range is $(-\infty, \infty)$. We can use shifts when finding the Domain, but the Range will always be all Real numbers.
}
\litem{
Which of the following intervals describes the Range of the function below?
\[ f(x) = -e^{x-6}+2 \]The solution is \( (-\infty, 2) \), which is option D.\begin{enumerate}[label=\Alph*.]
\item \( (a, \infty), a \in [-3, 0] \)

$(-2, \infty)$, which corresponds to using the negative vertical shift AND flipping the Range interval.
\item \( (-\infty, a], a \in [2, 5] \)

$(-\infty, 2]$, which corresponds to including the endpoint.
\item \( [a, \infty), a \in [-3, 0] \)

$[-2, \infty)$, which corresponds to using the negative vertical shift AND flipping the Range interval AND including the endpoint.
\item \( (-\infty, a), a \in [2, 5] \)

* $(-\infty, 2)$, which is the correct option.
\item \( (-\infty, \infty) \)

This corresponds to confusing range of an exponential function with the domain of an exponential function.
\end{enumerate}

\textbf{General Comment:} \textbf{General Comments}: Domain of a basic exponential function is $(-\infty, \infty)$ while the Range is $(0, \infty)$. We can shift these intervals [and even flip when $a<0$!] to find the new Domain/Range.
}
\litem{
 Solve the equation for $x$ and choose the interval that contains $x$ (if it exists).
\[  5 = \ln{\sqrt[7]{\frac{30}{e^{6x}}}} \]The solution is \( x = -5.266, \text{ which does not fit in any of the interval options.} \), which is option E.\begin{enumerate}[label=\Alph*.]
\item \( x \in [4.27, 8.27] \)

$x = 5.266$, which is the negative of the correct solution.
\item \( x \in [-3.44, -1.44] \)

$x = -2.445$, which corresponds to thinking you need to take the natural log of the left side before reducing.
\item \( x \in [-2.1, 1.9] \)

$x = -1.100$, which corresponds to treating any root as a square root.
\item \( \text{There is no Real solution to the equation.} \)

This corresponds to believing you cannot solve the equation.
\item \( \text{None of the above.} \)

*$x = -5.266$ is the correct solution and does not fit in any of the other intervals.
\end{enumerate}

\textbf{General Comment:} \textbf{General Comments}: After using the properties of logarithmic functions to break up the right-hand side, use $\ln(e) = 1$ to reduce the question to a linear function to solve. You can put $\ln(30)$ into a calculator if you are having trouble.
}
\litem{
Which of the following intervals describes the Domain of the function below?
\[ f(x) = e^{x+4}-3 \]The solution is \( (-\infty, \infty) \), which is option E.\begin{enumerate}[label=\Alph*.]
\item \( [a, \infty), a \in [2, 7] \)

$[3, \infty)$, which corresponds to using the negative vertical shift AND flipping the Range interval AND including the endpoint.
\item \( (-\infty, a], a \in [-11, -2] \)

$(-\infty, -3]$, which corresponds to using the correct vertical shift *if we wanted the Range* AND including the endpoint.
\item \( (-\infty, a), a \in [-11, -2] \)

$(-\infty, -3)$, which corresponds to using the correct vertical shift *if we wanted the Range*.
\item \( (a, \infty), a \in [2, 7] \)

$(3, \infty)$, which corresponds to using the negative vertical shift AND flipping the Range interval.
\item \( (-\infty, \infty) \)

* This is the correct option.
\end{enumerate}

\textbf{General Comment:} \textbf{General Comments}: Domain of a basic exponential function is $(-\infty, \infty)$ while the Range is $(0, \infty)$. We can shift these intervals [and even flip when $a<0$!] to find the new Domain/Range.
}
\litem{
Solve the equation for $x$ and choose the interval that contains the solution (if it exists).
\[ 3^{-3x-3} = \left(\frac{1}{343}\right)^{-2x-5} \]The solution is \( x = -2.170 \), which is option C.\begin{enumerate}[label=\Alph*.]
\item \( x \in [1, 3] \)

$x = 2.000$, which corresponds to solving the numerators as equal while ignoring the bases are different.
\item \( x \in [-33.48, -29.48] \)

$x = -32.484$, which corresponds to distributing the $\ln(base)$ to the second term of the exponent only.
\item \( x \in [-2.17, -0.17] \)

* $x = -2.170$, which is the correct option.
\item \( x \in [-1.87, 1.13] \)

$x = 0.134$, which corresponds to distributing the $\ln(base)$ to the first term of the exponent only.
\item \( \text{There is no Real solution to the equation.} \)

This corresponds to believing there is no solution since the bases are not powers of each other.
\end{enumerate}

\textbf{General Comment:} \textbf{General Comments:} This question was written so that the bases could not be written the same. You will need to take the log of both sides.
}
\litem{
Which of the following intervals describes the Range of the function below?
\[ f(x) = -\log_2{(x+1)}+4 \]The solution is \( (\infty, \infty) \), which is option E.\begin{enumerate}[label=\Alph*.]
\item \( (-\infty, a), a \in [-5.5, -1.1] \)

$(-\infty, -4)$, which corresponds to using the using the negative of vertical shift on $(0, \infty)$.
\item \( [a, \infty), a \in [-1.1, -0.3] \)

$[4, \infty)$, which corresponds to using the flipped Domain AND including the endpoint.
\item \( (-\infty, a), a \in [3.7, 8.4] \)

$(-\infty, 4)$, which corresponds to using the vertical shift while the Range is $(-\infty, \infty)$.
\item \( [a, \infty), a \in [0.7, 1.3] \)

$[1, \infty)$, which corresponds to using the negative of the horizontal shift AND including the endpoint.
\item \( (-\infty, \infty) \)

*This is the correct option.
\end{enumerate}

\textbf{General Comment:} \textbf{General Comments}: The domain of a basic logarithmic function is $(0, \infty)$ and the Range is $(-\infty, \infty)$. We can use shifts when finding the Domain, but the Range will always be all Real numbers.
}
\litem{
Solve the equation for $x$ and choose the interval that contains the solution (if it exists).
\[ 4^{-4x+5} = 27^{-2x-2} \]The solution is \( x = -12.922 \), which is option A.\begin{enumerate}[label=\Alph*.]
\item \( x \in [-13.92, -10.92] \)

* $x = -12.922$, which is the correct option.
\item \( x \in [-6.69, -3.69] \)

$x = -6.689$, which corresponds to distributing the $\ln(base)$ to the first term of the exponent only.
\item \( x \in [0.5, 4.5] \)

$x = 3.500$, which corresponds to solving the numerators as equal while ignoring the bases are different.
\item \( x \in [4.76, 8.76] \)

$x = 6.762$, which corresponds to distributing the $\ln(base)$ to the second term of the exponent only.
\item \( \text{There is no Real solution to the equation.} \)

This corresponds to believing there is no solution since the bases are not powers of each other.
\end{enumerate}

\textbf{General Comment:} \textbf{General Comments:} This question was written so that the bases could not be written the same. You will need to take the log of both sides.
}
\litem{
Solve the equation for $x$ and choose the interval that contains the solution (if it exists).
\[ \log_{4}{(2x+5)}+6 = 3 \]The solution is \( x = -2.492 \), which is option C.\begin{enumerate}[label=\Alph*.]
\item \( x \in [25.5, 31.5] \)

$x = 29.500$, which corresponds to ignoring the vertical shift when converting to exponential form.
\item \( x \in [42, 46] \)

$x = 43.000$, which corresponds to reversing the base and exponent when converting and reversing the value with $x$.
\item \( x \in [-3.49, 3.51] \)

* $x = -2.492$, which is the correct option.
\item \( x \in [37, 41] \)

$x = 38.000$, which corresponds to reversing the base and exponent when converting.
\item \( \text{There is no Real solution to the equation.} \)

Corresponds to believing a negative coefficient within the log equation means there is no Real solution.
\end{enumerate}

\textbf{General Comment:} \textbf{General Comments:} First, get the equation in the form $\log_b{(cx+d)} = a$. Then, convert to $b^a = cx+d$ and solve.
}
\end{enumerate}

\end{document}