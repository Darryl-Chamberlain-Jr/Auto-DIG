\documentclass{extbook}[14pt]
\usepackage{multicol, enumerate, enumitem, hyperref, color, soul, setspace, parskip, fancyhdr, amssymb, amsthm, amsmath, bbm, latexsym, units, mathtools}
\everymath{\displaystyle}
\usepackage[headsep=0.5cm,headheight=0cm, left=1 in,right= 1 in,top= 1 in,bottom= 1 in]{geometry}
\usepackage{dashrule}  % Package to use the command below to create lines between items
\newcommand{\litem}[1]{\item #1

\rule{\textwidth}{0.4pt}}
\pagestyle{fancy}
\lhead{}
\chead{Answer Key for Progress Quiz 4 Version B}
\rhead{}
\lfoot{4378-7085}
\cfoot{}
\rfoot{Fall 2020}
\begin{document}
\textbf{This key should allow you to understand why you choose the option you did (beyond just getting a question right or wrong). \href{https://xronos.clas.ufl.edu/mac1105spring2020/courseDescriptionAndMisc/Exams/LearningFromResults}{More instructions on how to use this key can be found here}.}

\textbf{If you have a suggestion to make the keys better, \href{https://forms.gle/CZkbZmPbC9XALEE88}{please fill out the short survey here}.}

\textit{Note: This key is auto-generated and may contain issues and/or errors. The keys are reviewed after each exam to ensure grading is done accurately. If there are issues (like duplicate options), they are noted in the offline gradebook. The keys are a work-in-progress to give students as many resources to improve as possible.}

\rule{\textwidth}{0.4pt}

\begin{enumerate}\litem{
Which of the following intervals describes the Range of the function below?
\[ f(x) = -\log_2{(x+2)}+4 \]
The solution is \( (\infty, \infty) \), which is option E.\begin{enumerate}[label=\Alph*.]
\item \( (-\infty, a), a \in [-6, -2.5] \)

$(-\infty, -4)$, which corresponds to using the using the negative of vertical shift on $(0, \infty)$.
\item \( [a, \infty), a \in [-2.5, -0.3] \)

$[4, \infty)$, which corresponds to using the flipped Domain AND including the endpoint.
\item \( (-\infty, a), a \in [3.7, 7.3] \)

$(-\infty, 4)$, which corresponds to using the vertical shift while the Range is $(-\infty, \infty)$.
\item \( [a, \infty), a \in [0, 3.8] \)

$[2, \infty)$, which corresponds to using the negative of the horizontal shift AND including the endpoint.
\item \( (-\infty, \infty) \)

*This is the correct option.
\end{enumerate}

\textbf{General Comment:} \textbf{General Comments}: The domain of a basic logarithmic function is $(0, \infty)$ and the Range is $(-\infty, \infty)$. We can use shifts when finding the Domain, but the Range will always be all Real numbers.
}
\litem{
 Solve the equation for $x$ and choose the interval that contains $x$ (if it exists).
\[  14 = \sqrt[3]{\frac{8}{e^{8x}}} \]
The solution is \( x = -0.73 \), which is option B.\begin{enumerate}[label=\Alph*.]
\item \( x \in [-0.68, -0.39] \)

$x = -0.400$, which corresponds to treating any root as a square root.
\item \( x \in [-1.64, -0.72] \)

* $x = -0.730$, which is the correct option.
\item \( x \in [-6.69, -5.16] \)

$x = -5.510$, which corresponds to thinking you don't need to take the natural log of both sides before reducing, as if the equation already had a natural log on the right side.
\item \( \text{There is no Real solution to the equation.} \)

This corresponds to believing you cannot solve the equation.
\item \( \text{None of the above.} \)

This corresponds to making an unexpected error.
\end{enumerate}

\textbf{General Comment:} \textbf{General Comments}: After using the properties of logarithmic functions to break up the right-hand side, use $\ln(e) = 1$ to reduce the question to a linear function to solve. You can put $\ln(8)$ into a calculator if you are having trouble.
}
\litem{
Which of the following intervals describes the Domain of the function below?
\[ f(x) = -e^{x+5}-8 \]
The solution is \( (-\infty, \infty) \), which is option E.\begin{enumerate}[label=\Alph*.]
\item \( (a, \infty), a \in [5, 11] \)

$(8, \infty)$, which corresponds to using the negative vertical shift AND flipping the Range interval.
\item \( (-\infty, a], a \in [-9, -2] \)

$(-\infty, -8]$, which corresponds to using the correct vertical shift *if we wanted the Range* AND including the endpoint.
\item \( (-\infty, a), a \in [-9, -2] \)

$(-\infty, -8)$, which corresponds to using the correct vertical shift *if we wanted the Range*.
\item \( [a, \infty), a \in [5, 11] \)

$[8, \infty)$, which corresponds to using the negative vertical shift AND flipping the Range interval AND including the endpoint.
\item \( (-\infty, \infty) \)

* This is the correct option.
\end{enumerate}

\textbf{General Comment:} \textbf{General Comments}: Domain of a basic exponential function is $(-\infty, \infty)$ while the Range is $(0, \infty)$. We can shift these intervals [and even flip when $a<0$!] to find the new Domain/Range.
}
\litem{
Solve the equation for $x$ and choose the interval that contains the solution (if it exists).
\[ 4^{-2x-2} = \left(\frac{1}{25}\right)^{3x-5} \]
The solution is \( x = 2.741 \), which is option B.\begin{enumerate}[label=\Alph*.]
\item \( x \in [-3.9, -2.8] \)

$x = -3.773$, which corresponds to distributing the $\ln(base)$ to the second term of the exponent only.
\item \( x \in [1.6, 3.1] \)

* $x = 2.741$, which is the correct option.
\item \( x \in [-0.9, 0.1] \)

$x = -0.436$, which corresponds to distributing the $\ln(base)$ to the first term of the exponent only.
\item \( x \in [-0.3, 0.9] \)

$x = 0.600$, which corresponds to solving the numerators as equal while ignoring the bases are different.
\item \( \text{There is no Real solution to the equation.} \)

This corresponds to believing there is no solution since the bases are not powers of each other.
\end{enumerate}

\textbf{General Comment:} \textbf{General Comments:} This question was written so that the bases could not be written the same. You will need to take the log of both sides.
}
\litem{
Solve the equation for $x$ and choose the interval that contains the solution (if it exists).
\[ \log_{2}{(-2x+7)}+4 = 3 \]
The solution is \( x = 3.250 \), which is option B.\begin{enumerate}[label=\Alph*.]
\item \( x \in [2.91, 3.15] \)

$x = 3.000$, which corresponds to reversing the base and exponent when converting.
\item \( x \in [3.09, 3.26] \)

* $x = 3.250$, which is the correct option.
\item \( x \in [-4.21, -3.82] \)

$x = -4.000$, which corresponds to reversing the base and exponent when converting and reversing the value with $x$.
\item \( x \in [-0.63, -0.45] \)

$x = -0.500$, which corresponds to ignoring the vertical shift when converting to exponential form.
\item \( \text{There is no Real solution to the equation.} \)

Corresponds to believing a negative coefficient within the log equation means there is no Real solution.
\end{enumerate}

\textbf{General Comment:} \textbf{General Comments:} First, get the equation in the form $\log_b{(cx+d)} = a$. Then, convert to $b^a = cx+d$ and solve.
}
\litem{
Solve the equation for $x$ and choose the interval that contains the solution (if it exists).
\[ \log_{5}{(-3x+5)}+6 = 3 \]
The solution is \( x = 1.664 \), which is option B.\begin{enumerate}[label=\Alph*.]
\item \( x \in [77.5, 80.8] \)

$x = 79.333$, which corresponds to reversing the base and exponent when converting and reversing the value with $x$.
\item \( x \in [0.5, 4.9] \)

* $x = 1.664$, which is the correct option.
\item \( x \in [-41.3, -37.6] \)

$x = -40.000$, which corresponds to ignoring the vertical shift when converting to exponential form.
\item \( x \in [82.2, 83.4] \)

$x = 82.667$, which corresponds to reversing the base and exponent when converting.
\item \( \text{There is no Real solution to the equation.} \)

Corresponds to believing a negative coefficient within the log equation means there is no Real solution.
\end{enumerate}

\textbf{General Comment:} \textbf{General Comments:} First, get the equation in the form $\log_b{(cx+d)} = a$. Then, convert to $b^a = cx+d$ and solve.
}
\litem{
Which of the following intervals describes the Range of the function below?
\[ f(x) = \log_2{(x+4)}-8 \]
The solution is \( (\infty, \infty) \), which is option E.\begin{enumerate}[label=\Alph*.]
\item \( [a, \infty), a \in [-0.6, 5.6] \)

$[4, \infty)$, which corresponds to using the negative of the horizontal shift AND including the endpoint.
\item \( (-\infty, a), a \in [7.9, 8.7] \)

$(-\infty, 8)$, which corresponds to using the using the negative of vertical shift on $(0, \infty)$.
\item \( (-\infty, a), a \in [-9.2, -6.4] \)

$(-\infty, -8)$, which corresponds to using the vertical shift while the Range is $(-\infty, \infty)$.
\item \( [a, \infty), a \in [-4.9, 0.3] \)

$[-8, \infty)$, which corresponds to using the flipped Domain AND including the endpoint.
\item \( (-\infty, \infty) \)

*This is the correct option.
\end{enumerate}

\textbf{General Comment:} \textbf{General Comments}: The domain of a basic logarithmic function is $(0, \infty)$ and the Range is $(-\infty, \infty)$. We can use shifts when finding the Domain, but the Range will always be all Real numbers.
}
\litem{
Solve the equation for $x$ and choose the interval that contains the solution (if it exists).
\[ 2^{-3x+3} = 25^{-2x+5} \]
The solution is \( x = 3.216 \), which is option D.\begin{enumerate}[label=\Alph*.]
\item \( x \in [-14.6, -13.7] \)

$x = -14.015$, which corresponds to distributing the $\ln(base)$ to the second term of the exponent only.
\item \( x \in [-0.4, 1.7] \)

$x = 0.459$, which corresponds to distributing the $\ln(base)$ to the first term of the exponent only.
\item \( x \in [-2.6, -1.8] \)

$x = -2.000$, which corresponds to solving the numerators as equal while ignoring the bases are different.
\item \( x \in [3.1, 3.8] \)

* $x = 3.216$, which is the correct option.
\item \( \text{There is no Real solution to the equation.} \)

This corresponds to believing there is no solution since the bases are not powers of each other.
\end{enumerate}

\textbf{General Comment:} \textbf{General Comments:} This question was written so that the bases could not be written the same. You will need to take the log of both sides.
}
\litem{
 Solve the equation for $x$ and choose the interval that contains $x$ (if it exists).
\[  8 = \ln{\sqrt[3]{\frac{21}{e^{4x}}}} \]
The solution is \( x = -5.239 \), which is option A.\begin{enumerate}[label=\Alph*.]
\item \( x \in [-5.48, -4.83] \)

* $x = -5.239$, which is the correct option.
\item \( x \in [-2.42, -1.76] \)

$x = -2.321$, which corresponds to thinking you need to take the natural log of on the left before reducing.
\item \( x \in [-3.87, -2.81] \)

$x = -3.239$, which corresponds to treating any root as a square root.
\item \( \text{There is no Real solution to the equation.} \)

This corresponds to believing you cannot solve the equation.
\item \( \text{None of the above.} \)

This corresponds to making an unexpected error.
\end{enumerate}

\textbf{General Comment:} \textbf{General Comments}: After using the properties of logarithmic functions to break up the right-hand side, use $\ln(e) = 1$ to reduce the question to a linear function to solve. You can put $\ln(21)$ into a calculator if you are having trouble.
}
\litem{
Which of the following intervals describes the Domain of the function below?
\[ f(x) = e^{x-5}-8 \]
The solution is \( (-\infty, \infty) \), which is option E.\begin{enumerate}[label=\Alph*.]
\item \( (-\infty, a], a \in [-8, -7] \)

$(-\infty, -8]$, which corresponds to using the correct vertical shift *if we wanted the Range* AND including the endpoint.
\item \( (-\infty, a), a \in [-8, -7] \)

$(-\infty, -8)$, which corresponds to using the correct vertical shift *if we wanted the Range*.
\item \( (a, \infty), a \in [7, 11] \)

$(8, \infty)$, which corresponds to using the negative vertical shift AND flipping the Range interval.
\item \( [a, \infty), a \in [7, 11] \)

$[8, \infty)$, which corresponds to using the negative vertical shift AND flipping the Range interval AND including the endpoint.
\item \( (-\infty, \infty) \)

* This is the correct option.
\end{enumerate}

\textbf{General Comment:} \textbf{General Comments}: Domain of a basic exponential function is $(-\infty, \infty)$ while the Range is $(0, \infty)$. We can shift these intervals [and even flip when $a<0$!] to find the new Domain/Range.
}
\end{enumerate}

\end{document}