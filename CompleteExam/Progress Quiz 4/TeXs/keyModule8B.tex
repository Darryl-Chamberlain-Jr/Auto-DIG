\documentclass{extbook}[14pt]
\usepackage{multicol, enumerate, enumitem, hyperref, color, soul, setspace, parskip, fancyhdr, amssymb, amsthm, amsmath, latexsym, units, mathtools}
\everymath{\displaystyle}
\usepackage[headsep=0.5cm,headheight=0cm, left=1 in,right= 1 in,top= 1 in,bottom= 1 in]{geometry}
\usepackage{dashrule}  % Package to use the command below to create lines between items
\newcommand{\litem}[1]{\item #1

\rule{\textwidth}{0.4pt}}
\pagestyle{fancy}
\lhead{}
\chead{Answer Key for Progress Quiz 4 Version B}
\rhead{}
\lfoot{5346-5907}
\cfoot{}
\rfoot{Summer C 2021}
\begin{document}
\textbf{This key should allow you to understand why you choose the option you did (beyond just getting a question right or wrong). \href{https://xronos.clas.ufl.edu/mac1105spring2020/courseDescriptionAndMisc/Exams/LearningFromResults}{More instructions on how to use this key can be found here}.}

\textbf{If you have a suggestion to make the keys better, \href{https://forms.gle/CZkbZmPbC9XALEE88}{please fill out the short survey here}.}

\textit{Note: This key is auto-generated and may contain issues and/or errors. The keys are reviewed after each exam to ensure grading is done accurately. If there are issues (like duplicate options), they are noted in the offline gradebook. The keys are a work-in-progress to give students as many resources to improve as possible.}

\rule{\textwidth}{0.4pt}

\begin{enumerate}\litem{
Solve the equation for $x$ and choose the interval that contains the solution (if it exists).
\[ \log_{4}{(3x+7)}+6 = 3 \]The solution is \( x = -2.328 \), which is option C.\begin{enumerate}[label=\Alph*.]
\item \( x \in [28.33, 32.33] \)

$x = 29.333$, which corresponds to reversing the base and exponent when converting and reversing the value with $x$.
\item \( x \in [22.67, 27.67] \)

$x = 24.667$, which corresponds to reversing the base and exponent when converting.
\item \( x \in [-5.33, -0.33] \)

* $x = -2.328$, which is the correct option.
\item \( x \in [19, 23] \)

$x = 19.000$, which corresponds to ignoring the vertical shift when converting to exponential form.
\item \( \text{There is no Real solution to the equation.} \)

Corresponds to believing a negative coefficient within the log equation means there is no Real solution.
\end{enumerate}

\textbf{General Comment:} \textbf{General Comments:} First, get the equation in the form $\log_b{(cx+d)} = a$. Then, convert to $b^a = cx+d$ and solve.
}
\litem{
 Solve the equation for $x$ and choose the interval that contains $x$ (if it exists).
\[  9 = \ln{\sqrt[3]{\frac{21}{e^{9x}}}} \]The solution is \( x = -2.662 \), which is option B.\begin{enumerate}[label=\Alph*.]
\item \( x \in [-2.05, -1.13] \)

$x = -1.662$, which corresponds to treating any root as a square root.
\item \( x \in [-2.79, -2.34] \)

* $x = -2.662$, which is the correct option.
\item \( x \in [-1.6, -0.88] \)

$x = -1.071$, which corresponds to thinking you need to take the natural log of on the left before reducing.
\item \( \text{There is no Real solution to the equation.} \)

This corresponds to believing you cannot solve the equation.
\item \( \text{None of the above.} \)

This corresponds to making an unexpected error.
\end{enumerate}

\textbf{General Comment:} \textbf{General Comments}: After using the properties of logarithmic functions to break up the right-hand side, use $\ln(e) = 1$ to reduce the question to a linear function to solve. You can put $\ln(21)$ into a calculator if you are having trouble.
}
\litem{
Which of the following intervals describes the Range of the function below?
\[ f(x) = -\log_2{(x+5)}+1 \]The solution is \( (\infty, \infty) \), which is option E.\begin{enumerate}[label=\Alph*.]
\item \( (-\infty, a), a \in [-1.1, 0.5] \)

$(-\infty, -1)$, which corresponds to using the using the negative of vertical shift on $(0, \infty)$.
\item \( [a, \infty), a \in [-5.1, -1.3] \)

$[1, \infty)$, which corresponds to using the flipped Domain AND including the endpoint.
\item \( (-\infty, a), a \in [0.6, 1.9] \)

$(-\infty, 1)$, which corresponds to using the vertical shift while the Range is $(-\infty, \infty)$.
\item \( [a, \infty), a \in [3, 5.2] \)

$[5, \infty)$, which corresponds to using the negative of the horizontal shift AND including the endpoint.
\item \( (-\infty, \infty) \)

*This is the correct option.
\end{enumerate}

\textbf{General Comment:} \textbf{General Comments}: The domain of a basic logarithmic function is $(0, \infty)$ and the Range is $(-\infty, \infty)$. We can use shifts when finding the Domain, but the Range will always be all Real numbers.
}
\litem{
Which of the following intervals describes the Domain of the function below?
\[ f(x) = -e^{x+6}-9 \]The solution is \( (-\infty, \infty) \), which is option E.\begin{enumerate}[label=\Alph*.]
\item \( (-\infty, a), a \in [-15, -5] \)

$(-\infty, -9)$, which corresponds to using the correct vertical shift *if we wanted the Range*.
\item \( [a, \infty), a \in [6, 12] \)

$[9, \infty)$, which corresponds to using the negative vertical shift AND flipping the Range interval AND including the endpoint.
\item \( (-\infty, a], a \in [-15, -5] \)

$(-\infty, -9]$, which corresponds to using the correct vertical shift *if we wanted the Range* AND including the endpoint.
\item \( (a, \infty), a \in [6, 12] \)

$(9, \infty)$, which corresponds to using the negative vertical shift AND flipping the Range interval.
\item \( (-\infty, \infty) \)

* This is the correct option.
\end{enumerate}

\textbf{General Comment:} \textbf{General Comments}: Domain of a basic exponential function is $(-\infty, \infty)$ while the Range is $(0, \infty)$. We can shift these intervals [and even flip when $a<0$!] to find the new Domain/Range.
}
\litem{
 Solve the equation for $x$ and choose the interval that contains $x$ (if it exists).
\[  16 = \sqrt[5]{\frac{25}{e^{9x}}} \]The solution is \( x = -1.183, \text{ which does not fit in any of the interval options.} \), which is option E.\begin{enumerate}[label=\Alph*.]
\item \( x \in [0.5, 1.6] \)

$x = 1.183$, which is the negative of the correct solution.
\item \( x \in [-0.4, 0] \)

$x = -0.258$, which corresponds to treating any root as a square root.
\item \( x \in [-10.8, -7.4] \)

$x = -9.247$, which corresponds to thinking you don't need to take the natural log of both sides before reducing, as if the right side already has a natural log.
\item \( \text{There is no Real solution to the equation.} \)

This corresponds to believing you cannot solve the equation.
\item \( \text{None of the above.} \)

* $x = -1.183$ is the correct solution and does not fit in any of the other intervals.
\end{enumerate}

\textbf{General Comment:} \textbf{General Comments}: After using the properties of logarithmic functions to break up the right-hand side, use $\ln(e) = 1$ to reduce the question to a linear function to solve. You can put $\ln(25)$ into a calculator if you are having trouble.
}
\litem{
Which of the following intervals describes the Domain of the function below?
\[ f(x) = e^{x-2}+6 \]The solution is \( (-\infty, \infty) \), which is option E.\begin{enumerate}[label=\Alph*.]
\item \( (-\infty, a], a \in [3, 7] \)

$(-\infty, 6]$, which corresponds to using the correct vertical shift *if we wanted the Range* AND including the endpoint.
\item \( (-\infty, a), a \in [3, 7] \)

$(-\infty, 6)$, which corresponds to using the correct vertical shift *if we wanted the Range*.
\item \( [a, \infty), a \in [-9, 4] \)

$[-6, \infty)$, which corresponds to using the negative vertical shift AND flipping the Range interval AND including the endpoint.
\item \( (a, \infty), a \in [-9, 4] \)

$(-6, \infty)$, which corresponds to using the negative vertical shift AND flipping the Range interval.
\item \( (-\infty, \infty) \)

* This is the correct option.
\end{enumerate}

\textbf{General Comment:} \textbf{General Comments}: Domain of a basic exponential function is $(-\infty, \infty)$ while the Range is $(0, \infty)$. We can shift these intervals [and even flip when $a<0$!] to find the new Domain/Range.
}
\litem{
Solve the equation for $x$ and choose the interval that contains the solution (if it exists).
\[ 5^{5x-2} = 216^{3x-4} \]The solution is \( x = 2.263 \), which is option C.\begin{enumerate}[label=\Alph*.]
\item \( x \in [-0.75, 1.25] \)

$x = 0.248$, which corresponds to distributing the $\ln(base)$ to the first term of the exponent only.
\item \( x \in [-2, 0] \)

$x = -1.000$, which corresponds to solving the numerators as equal while ignoring the bases are different.
\item \( x \in [2.26, 4.26] \)

* $x = 2.263$, which is the correct option.
\item \( x \in [-13.14, -5.14] \)

$x = -9.141$, which corresponds to distributing the $\ln(base)$ to the second term of the exponent only.
\item \( \text{There is no Real solution to the equation.} \)

This corresponds to believing there is no solution since the bases are not powers of each other.
\end{enumerate}

\textbf{General Comment:} \textbf{General Comments:} This question was written so that the bases could not be written the same. You will need to take the log of both sides.
}
\litem{
Which of the following intervals describes the Domain of the function below?
\[ f(x) = -\log_2{(x-6)}-6 \]The solution is \( (6, \infty) \), which is option C.\begin{enumerate}[label=\Alph*.]
\item \( (-\infty, a], a \in [3, 7] \)

$(-\infty, 6]$, which corresponds to using the negative vertical shift AND including the endpoint AND flipping the domain.
\item \( [a, \infty), a \in [-7, -5] \)

$[-6, \infty)$, which corresponds to using the vertical shift when shifting the Domain AND including the endpoint.
\item \( (a, \infty), a \in [3, 7] \)

* $(6, \infty)$, which is the correct option.
\item \( (-\infty, a), a \in [-7, -5] \)

$(-\infty, -6)$, which corresponds to flipping the Domain. Remember: the general for is $a*\log(x-h)+k$, \textbf{where $a$ does not affect the domain}.
\item \( (-\infty, \infty) \)

This corresponds to thinking of the range of the log function (or the domain of the exponential function).
\end{enumerate}

\textbf{General Comment:} \textbf{General Comments}: The domain of a basic logarithmic function is $(0, \infty)$ and the Range is $(-\infty, \infty)$. We can use shifts when finding the Domain, but the Range will always be all Real numbers.
}
\litem{
Solve the equation for $x$ and choose the interval that contains the solution (if it exists).
\[ 3^{-3x-5} = \left(\frac{1}{343}\right)^{-2x+5} \]The solution is \( x = 1.583 \), which is option D.\begin{enumerate}[label=\Alph*.]
\item \( x \in [-11, -8] \)

$x = -10.000$, which corresponds to solving the numerators as equal while ignoring the bases are different.
\item \( x \in [-0.67, 1.33] \)

$x = -0.668$, which corresponds to distributing the $\ln(base)$ to the first term of the exponent only.
\item \( x \in [17.7, 26.7] \)

$x = 23.696$, which corresponds to distributing the $\ln(base)$ to the second term of the exponent only.
\item \( x \in [0.58, 3.58] \)

* $x = 1.583$, which is the correct option.
\item \( \text{There is no Real solution to the equation.} \)

This corresponds to believing there is no solution since the bases are not powers of each other.
\end{enumerate}

\textbf{General Comment:} \textbf{General Comments:} This question was written so that the bases could not be written the same. You will need to take the log of both sides.
}
\litem{
Solve the equation for $x$ and choose the interval that contains the solution (if it exists).
\[ \log_{3}{(4x+5)}+6 = 3 \]The solution is \( x = -1.241 \), which is option A.\begin{enumerate}[label=\Alph*.]
\item \( x \in [-3.24, 4.76] \)

* $x = -1.241$, which is the correct option.
\item \( x \in [2.5, 8.5] \)

$x = 5.500$, which corresponds to ignoring the vertical shift when converting to exponential form.
\item \( x \in [-8, -7] \)

$x = -8.000$, which corresponds to reversing the base and exponent when converting.
\item \( x \in [-7.5, -2.5] \)

$x = -5.500$, which corresponds to reversing the base and exponent when converting and reversing the value with $x$.
\item \( \text{There is no Real solution to the equation.} \)

Corresponds to believing a negative coefficient within the log equation means there is no Real solution.
\end{enumerate}

\textbf{General Comment:} \textbf{General Comments:} First, get the equation in the form $\log_b{(cx+d)} = a$. Then, convert to $b^a = cx+d$ and solve.
}
\end{enumerate}

\end{document}