\documentclass[14pt]{extbook}
\usepackage{multicol, enumerate, enumitem, hyperref, color, soul, setspace, parskip, fancyhdr} %General Packages
\usepackage{amssymb, amsthm, amsmath, latexsym, units, mathtools} %Math Packages
\everymath{\displaystyle} %All math in Display Style
% Packages with additional options
\usepackage[headsep=0.5cm,headheight=12pt, left=1 in,right= 1 in,top= 1 in,bottom= 1 in]{geometry}
\usepackage[usenames,dvipsnames]{xcolor}
\usepackage{dashrule}  % Package to use the command below to create lines between items
\newcommand{\litem}[1]{\item#1\hspace*{-1cm}\rule{\textwidth}{0.4pt}}
\pagestyle{fancy}
\lhead{Progress Quiz 4}
\chead{}
\rhead{Version C}
\lfoot{5346-5907}
\cfoot{}
\rfoot{Summer C 2021}
\begin{document}

\begin{enumerate}
\litem{
Determine whether the function below is 1-1.\[ f(x) = (5 x - 20)^3 \]\begin{enumerate}[label=\Alph*.]
\item \( \text{No, because the domain of the function is not $(-\infty, \infty)$.} \)
\item \( \text{No, because the range of the function is not $(-\infty, \infty)$.} \)
\item \( \text{Yes, the function is 1-1.} \)
\item \( \text{No, because there is an $x$-value that goes to 2 different $y$-values.} \)
\item \( \text{No, because there is a $y$-value that goes to 2 different $x$-values.} \)

\end{enumerate} }
\litem{
Find the inverse of the function below. Then, evaluate the inverse at $x = 8$ and choose the interval that $f^-1(8)$ belongs to.\[ f(x) = e^{x-4}-4 \]\begin{enumerate}[label=\Alph*.]
\item \( f^{-1}(8) \in [-1.52, -0.52] \)
\item \( f^{-1}(8) \in [6.48, 8.48] \)
\item \( f^{-1}(8) \in [-4.61, -1.61] \)
\item \( f^{-1}(8) \in [-1.52, -0.52] \)
\item \( f^{-1}(8) \in [-4.61, -1.61] \)

\end{enumerate} }
\litem{
Subtract the following functions, then choose the domain of the resulting function from the list below.\[ f(x) = 7x^{2} +7 x + 9 \text{ and } g(x) = 2x + 8 \]\begin{enumerate}[label=\Alph*.]
\item \( \text{ The domain is all Real numbers except } x = a, \text{ where } a \in [-9.17, -0.17] \)
\item \( \text{ The domain is all Real numbers less than or equal to } x = a, \text{ where } a \in [0.4, 7.4] \)
\item \( \text{ The domain is all Real numbers greater than or equal to } x = a, \text{ where } a \in [-0.33, 9.67] \)
\item \( \text{ The domain is all Real numbers except } x = a \text{ and } x = b, \text{ where } a \in [-4.6, 7.4] \text{ and } b \in [-4.75, 0.25] \)
\item \( \text{ The domain is all Real numbers. } \)

\end{enumerate} }
\litem{
Find the inverse of the function below (if it exists). Then, evaluate the inverse at $x = 11$ and choose the interval that $f^-1(11)$ belongs to.\[ f(x) = \sqrt[3]{3 x - 2} \]\begin{enumerate}[label=\Alph*.]
\item \( f^{-1}(11) \in [-443.95, -442.28] \)
\item \( f^{-1}(11) \in [441.04, 444.2] \)
\item \( f^{-1}(11) \in [444.01, 444.69] \)
\item \( f^{-1}(11) \in [-445.64, -444.03] \)
\item \( \text{ The function is not invertible for all Real numbers. } \)

\end{enumerate} }
\litem{
Find the inverse of the function below. Then, evaluate the inverse at $x = 10$ and choose the interval that $f^-1(10)$ belongs to.\[ f(x) = \ln{(x-5)}+3 \]\begin{enumerate}[label=\Alph*.]
\item \( f^{-1}(10) \in [3269019.37, 3269022.37] \)
\item \( f^{-1}(10) \in [1084.63, 1094.63] \)
\item \( f^{-1}(10) \in [442415.39, 442424.39] \)
\item \( f^{-1}(10) \in [147.41, 155.41] \)
\item \( f^{-1}(10) \in [1098.63, 1103.63] \)

\end{enumerate} }
\litem{
Determine whether the function below is 1-1.\[ f(x) = (4 x - 29)^3 \]\begin{enumerate}[label=\Alph*.]
\item \( \text{Yes, the function is 1-1.} \)
\item \( \text{No, because there is a $y$-value that goes to 2 different $x$-values.} \)
\item \( \text{No, because the range of the function is not $(-\infty, \infty)$.} \)
\item \( \text{No, because the domain of the function is not $(-\infty, \infty)$.} \)
\item \( \text{No, because there is an $x$-value that goes to 2 different $y$-values.} \)

\end{enumerate} }
\litem{
Find the inverse of the function below (if it exists). Then, evaluate the inverse at $x = -10$ and choose the interval that $f^-1(-10)$ belongs to.\[ f(x) = 3 x^2 - 4 \]\begin{enumerate}[label=\Alph*.]
\item \( f^{-1}(-10) \in [4.25, 4.48] \)
\item \( f^{-1}(-10) \in [2.74, 3.91] \)
\item \( f^{-1}(-10) \in [2.01, 2.17] \)
\item \( f^{-1}(-10) \in [0.95, 2.12] \)
\item \( \text{ The function is not invertible for all Real numbers. } \)

\end{enumerate} }
\litem{
Multiply the following functions, then choose the domain of the resulting function from the list below.\[ f(x) = 8x + 9 \text{ and } g(x) = 2x + 9 \]\begin{enumerate}[label=\Alph*.]
\item \( \text{ The domain is all Real numbers greater than or equal to } x = a, \text{ where } a \in [3.25, 11.25] \)
\item \( \text{ The domain is all Real numbers except } x = a, \text{ where } a \in [-8.33, -3.33] \)
\item \( \text{ The domain is all Real numbers less than or equal to } x = a, \text{ where } a \in [1.83, 3.83] \)
\item \( \text{ The domain is all Real numbers except } x = a \text{ and } x = b, \text{ where } a \in [-6.6, -0.6] \text{ and } b \in [4.2, 13.2] \)
\item \( \text{ The domain is all Real numbers. } \)

\end{enumerate} }
\litem{
Choose the interval below that $f$ composed with $g$ at $x=1$ is in.\[ f(x) = 2x^{3} -1 x^{2} +2 x -3 \text{ and } g(x) = x^{3} + x^{2} +3 x -4 \]\begin{enumerate}[label=\Alph*.]
\item \( (f \circ g)(1) \in [-0.8, 2] \)
\item \( (f \circ g)(1) \in [-7.6, -5.6] \)
\item \( (f \circ g)(1) \in [-4.3, -3.3] \)
\item \( (f \circ g)(1) \in [-11.7, -9.2] \)
\item \( \text{It is not possible to compose the two functions.} \)

\end{enumerate} }
\litem{
Choose the interval below that $f$ composed with $g$ at $x=1$ is in.\[ f(x) = 2x^{3} +2 x^{2} -2 x \text{ and } g(x) = -2x^{3} -1 x^{2} +x + 1 \]\begin{enumerate}[label=\Alph*.]
\item \( (f \circ g)(1) \in [-28, -25] \)
\item \( (f \circ g)(1) \in [2, 8] \)
\item \( (f \circ g)(1) \in [-10, 1] \)
\item \( (f \circ g)(1) \in [-19, -16] \)
\item \( \text{It is not possible to compose the two functions.} \)

\end{enumerate} }
\end{enumerate}

\end{document}