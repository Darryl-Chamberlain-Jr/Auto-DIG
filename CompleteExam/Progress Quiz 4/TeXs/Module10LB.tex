\documentclass[14pt]{extbook}
\usepackage{multicol, enumerate, enumitem, hyperref, color, soul, setspace, parskip, fancyhdr} %General Packages
\usepackage{amssymb, amsthm, amsmath, bbm, latexsym, units, mathtools} %Math Packages
\everymath{\displaystyle} %All math in Display Style
% Packages with additional options
\usepackage[headsep=0.5cm,headheight=12pt, left=1 in,right= 1 in,top= 1 in,bottom= 1 in]{geometry}
\usepackage[usenames,dvipsnames]{xcolor}
\usepackage{dashrule}  % Package to use the command below to create lines between items
\newcommand{\litem}[1]{\item#1\hspace*{-1cm}\rule{\textwidth}{0.4pt}}
\pagestyle{fancy}
\lhead{Progress Quiz 4}
\chead{}
\rhead{Version B}
\lfoot{9187-5854}
\cfoot{}
\rfoot{Spring 2021}
\begin{document}

\begin{enumerate}
\litem{
Factor the polynomial below completely. Then, choose the intervals the zeros of the polynomial belong to, where $z_1 \leq z_2 \leq z_3$. \textit{To make the problem easier, all zeros are between -5 and 5.}\[ f(x) = 15x^{3} +29 x^{2} -8 x -12 \]\begin{enumerate}[label=\Alph*.]
\item \( z_1 \in [-1.99, -0.93], \text{   }  z_2 \in [1.5, 1.9], \text{   and   } z_3 \in [1.73, 2.12] \)
\item \( z_1 \in [-2.22, -1.58], \text{   }  z_2 \in [-1.3, -0.2], \text{   and   } z_3 \in [0.38, 0.79] \)
\item \( z_1 \in [-2.22, -1.58], \text{   }  z_2 \in [-2.4, -0.7], \text{   and   } z_3 \in [1.34, 1.69] \)
\item \( z_1 \in [-1.14, -0.57], \text{   }  z_2 \in [0.4, 0.7], \text{   and   } z_3 \in [1.73, 2.12] \)
\item \( z_1 \in [-0.38, -0.12], \text{   }  z_2 \in [1.9, 2.8], \text{   and   } z_3 \in [2.34, 3.32] \)

\end{enumerate} }
\litem{
Perform the division below. Then, find the intervals that correspond to the quotient in the form $ax^2+bx+c$ and remainder $r$.\[ \frac{9x^{3} +33 x^{2} -32 x -84}{x + 4} \]\begin{enumerate}[label=\Alph*.]
\item \( a \in [9, 11], \text{   } b \in [69, 73], \text{   } c \in [241, 248], \text{   and   } r \in [885, 895]. \)
\item \( a \in [9, 11], \text{   } b \in [-13, -9], \text{   } c \in [27, 29], \text{   and   } r \in [-226, -213]. \)
\item \( a \in [-40, -31], \text{   } b \in [-115, -107], \text{   } c \in [-478, -473], \text{   and   } r \in [-1990, -1981]. \)
\item \( a \in [-40, -31], \text{   } b \in [172, 178], \text{   } c \in [-743, -736], \text{   and   } r \in [2871, 2877]. \)
\item \( a \in [9, 11], \text{   } b \in [-3, 4], \text{   } c \in [-22, -18], \text{   and   } r \in [-7, -1]. \)

\end{enumerate} }
\litem{
What are the \textit{possible Rational} roots of the polynomial below?\[ f(x) = 3x^{2} +4 x + 5 \]\begin{enumerate}[label=\Alph*.]
\item \( \pm 1,\pm 5 \)
\item \( \pm 1,\pm 3 \)
\item \( \text{ All combinations of: }\frac{\pm 1,\pm 5}{\pm 1,\pm 3} \)
\item \( \text{ All combinations of: }\frac{\pm 1,\pm 3}{\pm 1,\pm 5} \)
\item \( \text{ There is no formula or theorem that tells us all possible Rational roots.} \)

\end{enumerate} }
\litem{
Factor the polynomial below completely. Then, choose the intervals the zeros of the polynomial belong to, where $z_1 \leq z_2 \leq z_3$. \textit{To make the problem easier, all zeros are between -5 and 5.}\[ f(x) = 15x^{3} -59 x^{2} -10 x + 24 \]\begin{enumerate}[label=\Alph*.]
\item \( z_1 \in [-1.14, 0.07], \text{   }  z_2 \in [0.17, 0.63], \text{   and   } z_3 \in [3.67, 4.06] \)
\item \( z_1 \in [-4.12, -3.78], \text{   }  z_2 \in [-2, -1.4], \text{   and   } z_3 \in [1.48, 1.6] \)
\item \( z_1 \in [-4.12, -3.78], \text{   }  z_2 \in [-0.58, -0.1], \text{   and   } z_3 \in [1.82, 2.21] \)
\item \( z_1 \in [-1.66, -1.07], \text{   }  z_2 \in [1.16, 1.88], \text{   and   } z_3 \in [3.67, 4.06] \)
\item \( z_1 \in [-4.12, -3.78], \text{   }  z_2 \in [-0.9, -0.39], \text{   and   } z_3 \in [-0.04, 0.82] \)

\end{enumerate} }
\litem{
Perform the division below. Then, find the intervals that correspond to the quotient in the form $ax^2+bx+c$ and remainder $r$.\[ \frac{15x^{3} +62 x^{2} -36}{x + 4} \]\begin{enumerate}[label=\Alph*.]
\item \( a \in [13, 23], b \in [-13, -8], c \in [65, 70], \text{ and } r \in [-361, -360]. \)
\item \( a \in [-65, -55], b \in [298, 305], c \in [-1214, -1206], \text{ and } r \in [4794, 4797]. \)
\item \( a \in [13, 23], b \in [120, 125], c \in [487, 489], \text{ and } r \in [1916, 1918]. \)
\item \( a \in [13, 23], b \in [0, 7], c \in [-16, -2], \text{ and } r \in [-7, 0]. \)
\item \( a \in [-65, -55], b \in [-181, -177], c \in [-712, -710], \text{ and } r \in [-2886, -2876]. \)

\end{enumerate} }
\litem{
Perform the division below. Then, find the intervals that correspond to the quotient in the form $ax^2+bx+c$ and remainder $r$.\[ \frac{15x^{3} +25 x^{2} -20 x -18}{x + 2} \]\begin{enumerate}[label=\Alph*.]
\item \( a \in [-36, -24], \text{   } b \in [-38, -30], \text{   } c \in [-91, -88], \text{   and   } r \in [-198, -194]. \)
\item \( a \in [-36, -24], \text{   } b \in [85, 88], \text{   } c \in [-192, -185], \text{   and   } r \in [361, 363]. \)
\item \( a \in [15, 19], \text{   } b \in [-25, -15], \text{   } c \in [38, 45], \text{   and   } r \in [-145, -132]. \)
\item \( a \in [15, 19], \text{   } b \in [51, 57], \text{   } c \in [85, 91], \text{   and   } r \in [159, 165]. \)
\item \( a \in [15, 19], \text{   } b \in [-9, -2], \text{   } c \in [-15, -6], \text{   and   } r \in [-2, 5]. \)

\end{enumerate} }
\litem{
What are the \textit{possible Rational} roots of the polynomial below?\[ f(x) = 4x^{4} +7 x^{3} +2 x^{2} +3 x + 2 \]\begin{enumerate}[label=\Alph*.]
\item \( \pm 1,\pm 2 \)
\item \( \text{ All combinations of: }\frac{\pm 1,\pm 2,\pm 4}{\pm 1,\pm 2} \)
\item \( \text{ All combinations of: }\frac{\pm 1,\pm 2}{\pm 1,\pm 2,\pm 4} \)
\item \( \pm 1,\pm 2,\pm 4 \)
\item \( \text{ There is no formula or theorem that tells us all possible Rational roots.} \)

\end{enumerate} }
\litem{
Factor the polynomial below completely, knowing that $x-3$ is a factor. Then, choose the intervals the zeros of the polynomial belong to, where $z_1 \leq z_2 \leq z_3 \leq z_4$. \textit{To make the problem easier, all zeros are between -5 and 5.}\[ f(x) = 9x^{4} +27 x^{3} -61 x^{2} -243 x -180 \]\begin{enumerate}[label=\Alph*.]
\item \( z_1 \in [-5, -2], \text{   }  z_2 \in [1.28, 1.34], z_3 \in [1.25, 2.2], \text{   and   } z_4 \in [2, 3.1] \)
\item \( z_1 \in [-5, -2], \text{   }  z_2 \in [0.53, 0.58], z_3 \in [2.1, 3.67], \text{   and   } z_4 \in [3.7, 4.4] \)
\item \( z_1 \in [-5, -2], \text{   }  z_2 \in [-1.67, -1.62], z_3 \in [-1.85, -0.77], \text{   and   } z_4 \in [2, 3.1] \)
\item \( z_1 \in [-5, -2], \text{   }  z_2 \in [0.6, 0.65], z_3 \in [-0.08, 1.5], \text{   and   } z_4 \in [2, 3.1] \)
\item \( z_1 \in [-5, -2], \text{   }  z_2 \in [-0.76, -0.75], z_3 \in [-0.84, -0.43], \text{   and   } z_4 \in [2, 3.1] \)

\end{enumerate} }
\litem{
Factor the polynomial below completely, knowing that $x+3$ is a factor. Then, choose the intervals the zeros of the polynomial belong to, where $z_1 \leq z_2 \leq z_3 \leq z_4$. \textit{To make the problem easier, all zeros are between -5 and 5.}\[ f(x) = 15x^{4} -13 x^{3} -155 x^{2} +117 x + 180 \]\begin{enumerate}[label=\Alph*.]
\item \( z_1 \in [-4, 0], \text{   }  z_2 \in [-1.3, -1.22], z_3 \in [0.58, 0.63], \text{   and   } z_4 \in [1, 7] \)
\item \( z_1 \in [-4, 0], \text{   }  z_2 \in [-0.62, -0.45], z_3 \in [1.11, 1.26], \text{   and   } z_4 \in [1, 7] \)
\item \( z_1 \in [-5, -4], \text{   }  z_2 \in [-3.13, -2.79], z_3 \in [0.04, 0.31], \text{   and   } z_4 \in [1, 7] \)
\item \( z_1 \in [-4, 0], \text{   }  z_2 \in [-1.77, -1.51], z_3 \in [0.76, 0.97], \text{   and   } z_4 \in [1, 7] \)
\item \( z_1 \in [-4, 0], \text{   }  z_2 \in [-1.12, -0.75], z_3 \in [1.44, 1.72], \text{   and   } z_4 \in [1, 7] \)

\end{enumerate} }
\litem{
Perform the division below. Then, find the intervals that correspond to the quotient in the form $ax^2+bx+c$ and remainder $r$.\[ \frac{4x^{3} -49 x -65}{x -4} \]\begin{enumerate}[label=\Alph*.]
\item \( a \in [3, 9], b \in [15, 17.3], c \in [12, 22], \text{ and } r \in [-5, -3]. \)
\item \( a \in [10, 24], b \in [-65.5, -61.7], c \in [206, 213], \text{ and } r \in [-897, -884]. \)
\item \( a \in [3, 9], b \in [-16.7, -14.7], c \in [12, 22], \text{ and } r \in [-126, -123]. \)
\item \( a \in [10, 24], b \in [62.6, 66.2], c \in [206, 213], \text{ and } r \in [759, 766]. \)
\item \( a \in [3, 9], b \in [11, 13], c \in [-17, -5], \text{ and } r \in [-105, -102]. \)

\end{enumerate} }
\end{enumerate}

\end{document}