\documentclass{extbook}[14pt]
\usepackage{multicol, enumerate, enumitem, hyperref, color, soul, setspace, parskip, fancyhdr, amssymb, amsthm, amsmath, latexsym, units, mathtools}
\everymath{\displaystyle}
\usepackage[headsep=0.5cm,headheight=0cm, left=1 in,right= 1 in,top= 1 in,bottom= 1 in]{geometry}
\usepackage{dashrule}  % Package to use the command below to create lines between items
\newcommand{\litem}[1]{\item #1

\rule{\textwidth}{0.4pt}}
\pagestyle{fancy}
\lhead{}
\chead{Answer Key for Progress Quiz 4 Version B}
\rhead{}
\lfoot{5346-5907}
\cfoot{}
\rfoot{Summer C 2021}
\begin{document}
\textbf{This key should allow you to understand why you choose the option you did (beyond just getting a question right or wrong). \href{https://xronos.clas.ufl.edu/mac1105spring2020/courseDescriptionAndMisc/Exams/LearningFromResults}{More instructions on how to use this key can be found here}.}

\textbf{If you have a suggestion to make the keys better, \href{https://forms.gle/CZkbZmPbC9XALEE88}{please fill out the short survey here}.}

\textit{Note: This key is auto-generated and may contain issues and/or errors. The keys are reviewed after each exam to ensure grading is done accurately. If there are issues (like duplicate options), they are noted in the offline gradebook. The keys are a work-in-progress to give students as many resources to improve as possible.}

\rule{\textwidth}{0.4pt}

\begin{enumerate}\litem{
Factor the polynomial below completely, knowing that $x -2$ is a factor. Then, choose the intervals the zeros of the polynomial belong to, where $z_1 \leq z_2 \leq z_3 \leq z_4$. \textit{To make the problem easier, all zeros are between -5 and 5.}
\[ f(x) = 20x^{4} -143 x^{3} +212 x^{2} +33 x -90 \]The solution is \( [-0.6, 0.75, 2, 5] \), which is option C.\begin{enumerate}[label=\Alph*.]
\item \( z_1 \in [-2, -1.6], \text{   }  z_2 \in [0.98, 1.85], z_3 \in [1.93, 2.08], \text{   and   } z_4 \in [4.45, 5.68] \)

 Distractor 2: Corresponds to inversing rational roots.
\item \( z_1 \in [-5.3, -3.4], \text{   }  z_2 \in [-2.67, -1.65], z_3 \in [-1.14, -0.62], \text{   and   } z_4 \in [-0.5, 1.45] \)

 Distractor 1: Corresponds to negatives of all zeros.
\item \( z_1 \in [-1.5, 0.4], \text{   }  z_2 \in [0.49, 1.03], z_3 \in [1.93, 2.08], \text{   and   } z_4 \in [4.45, 5.68] \)

* This is the solution!
\item \( z_1 \in [-5.3, -3.4], \text{   }  z_2 \in [-2.67, -1.65], z_3 \in [-0.68, 0.13], \text{   and   } z_4 \in [2.83, 3.55] \)

 Distractor 4: Corresponds to moving factors from one rational to another.
\item \( z_1 \in [-5.3, -3.4], \text{   }  z_2 \in [-2.67, -1.65], z_3 \in [-1.51, -1.1], \text{   and   } z_4 \in [0.65, 2.43] \)

 Distractor 3: Corresponds to negatives of all zeros AND inversing rational roots.
\end{enumerate}

\textbf{General Comment:} Remember to try the middle-most integers first as these normally are the zeros. Also, once you get it to a quadratic, you can use your other factoring techniques to finish factoring.
}
\litem{
Perform the division below. Then, find the intervals that correspond to the quotient in the form $ax^2+bx+c$ and remainder $r$.
\[ \frac{10x^{3} -64 x^{2} +74 x -25}{x -5} \]The solution is \( 10x^{2} -14 x + 4 + \frac{-5}{x -5} \), which is option A.\begin{enumerate}[label=\Alph*.]
\item \( a \in [9, 14], \text{   } b \in [-15, -6], \text{   } c \in [2, 5], \text{   and   } r \in [-9, -1]. \)

* This is the solution!
\item \( a \in [9, 14], \text{   } b \in [-119, -108], \text{   } c \in [642, 645], \text{   and   } r \in [-3249, -3242]. \)

 You divided by the opposite of the factor.
\item \( a \in [43, 53], \text{   } b \in [-320, -306], \text{   } c \in [1641, 1645], \text{   and   } r \in [-8246, -8238]. \)

 You divided by the opposite of the factor AND multiplied the first factor rather than just bringing it down.
\item \( a \in [43, 53], \text{   } b \in [180, 194], \text{   } c \in [1000, 1007], \text{   and   } r \in [4994, 4996]. \)

 You multiplied by the synthetic number rather than bringing the first factor down.
\item \( a \in [9, 14], \text{   } b \in [-32, -17], \text{   } c \in [-24, -21], \text{   and   } r \in [-119, -110]. \)

 You multiplied by the synthetic number and subtracted rather than adding during synthetic division.
\end{enumerate}

\textbf{General Comment:} Be sure to synthetically divide by the zero of the denominator!
}
\litem{
Factor the polynomial below completely. Then, choose the intervals the zeros of the polynomial belong to, where $z_1 \leq z_2 \leq z_3$. \textit{To make the problem easier, all zeros are between -5 and 5.}
\[ f(x) = 16x^{3} +40 x^{2} +x -30 \]The solution is \( [-2, -1.25, 0.75] \), which is option E.\begin{enumerate}[label=\Alph*.]
\item \( z_1 \in [-0.65, -0.05], \text{   }  z_2 \in [1.8, 2.3], \text{   and   } z_3 \in [4.98, 5.07] \)

 Distractor 4: Corresponds to moving factors from one rational to another.
\item \( z_1 \in [-2.18, -1.53], \text{   }  z_2 \in [-0.83, -0.62], \text{   and   } z_3 \in [1.25, 1.67] \)

 Distractor 2: Corresponds to inversing rational roots.
\item \( z_1 \in [-1.67, -1.19], \text{   }  z_2 \in [0.74, 1.09], \text{   and   } z_3 \in [1.9, 2.6] \)

 Distractor 3: Corresponds to negatives of all zeros AND inversing rational roots.
\item \( z_1 \in [-0.91, -0.62], \text{   }  z_2 \in [1.04, 1.44], \text{   and   } z_3 \in [1.9, 2.6] \)

 Distractor 1: Corresponds to negatives of all zeros.
\item \( z_1 \in [-2.18, -1.53], \text{   }  z_2 \in [-1.34, -1.21], \text{   and   } z_3 \in [0.49, 0.83] \)

* This is the solution!
\end{enumerate}

\textbf{General Comment:} Remember to try the middle-most integers first as these normally are the zeros. Also, once you get it to a quadratic, you can use your other factoring techniques to finish factoring.
}
\litem{
What are the \textit{possible Integer} roots of the polynomial below?
\[ f(x) = 7x^{3} +3 x^{2} +4 x + 4 \]The solution is \( \pm 1,\pm 2,\pm 4 \), which is option A.\begin{enumerate}[label=\Alph*.]
\item \( \pm 1,\pm 2,\pm 4 \)

* This is the solution \textbf{since we asked for the possible Integer roots}!
\item \( \text{ All combinations of: }\frac{\pm 1,\pm 2,\pm 4}{\pm 1,\pm 7} \)

This would have been the solution \textbf{if asked for the possible Rational roots}!
\item \( \text{ All combinations of: }\frac{\pm 1,\pm 7}{\pm 1,\pm 2,\pm 4} \)

 Distractor 3: Corresponds to the plus or minus of the inverse quotient (an/a0) of the factors. 
\item \( \pm 1,\pm 7 \)

 Distractor 1: Corresponds to the plus or minus factors of a1 only.
\item \( \text{There is no formula or theorem that tells us all possible Integer roots.} \)

 Distractor 4: Corresponds to not recognizing Integers as a subset of Rationals.
\end{enumerate}

\textbf{General Comment:} We have a way to find the possible Rational roots. The possible Integer roots are the Integers in this list.
}
\litem{
Factor the polynomial below completely. Then, choose the intervals the zeros of the polynomial belong to, where $z_1 \leq z_2 \leq z_3$. \textit{To make the problem easier, all zeros are between -5 and 5.}
\[ f(x) = 15x^{3} +71 x^{2} +32 x -48 \]The solution is \( [-4, -1.33, 0.6] \), which is option E.\begin{enumerate}[label=\Alph*.]
\item \( z_1 \in [-0.31, 0], \text{   }  z_2 \in [3.4, 5.2], \text{   and   } z_3 \in [2.8, 5.1] \)

 Distractor 4: Corresponds to moving factors from one rational to another.
\item \( z_1 \in [-1.74, -1.25], \text{   }  z_2 \in [0, 1.2], \text{   and   } z_3 \in [2.8, 5.1] \)

 Distractor 3: Corresponds to negatives of all zeros AND inversing rational roots.
\item \( z_1 \in [-4.13, -3.86], \text{   }  z_2 \in [-1.2, -0.2], \text{   and   } z_3 \in [1.4, 2.1] \)

 Distractor 2: Corresponds to inversing rational roots.
\item \( z_1 \in [-0.81, -0.46], \text{   }  z_2 \in [0.9, 2.3], \text{   and   } z_3 \in [2.8, 5.1] \)

 Distractor 1: Corresponds to negatives of all zeros.
\item \( z_1 \in [-4.13, -3.86], \text{   }  z_2 \in [-2.6, -1.1], \text{   and   } z_3 \in [0.3, 1.1] \)

* This is the solution!
\end{enumerate}

\textbf{General Comment:} Remember to try the middle-most integers first as these normally are the zeros. Also, once you get it to a quadratic, you can use your other factoring techniques to finish factoring.
}
\litem{
Perform the division below. Then, find the intervals that correspond to the quotient in the form $ax^2+bx+c$ and remainder $r$.
\[ \frac{20x^{3} +105 x^{2} -122}{x + 5} \]The solution is \( 20x^{2} +5 x -25 + \frac{3}{x + 5} \), which is option C.\begin{enumerate}[label=\Alph*.]
\item \( a \in [-103, -96], b \in [598, 607], c \in [-3033, -3021], \text{ and } r \in [15001, 15007]. \)

 You multipled by the synthetic number rather than bringing the first factor down.
\item \( a \in [17, 23], b \in [200, 209], c \in [1024, 1033], \text{ and } r \in [4999, 5011]. \)

 You divided by the opposite of the factor.
\item \( a \in [17, 23], b \in [-2, 9], c \in [-25, -22], \text{ and } r \in [-2, 10]. \)

* This is the solution!
\item \( a \in [-103, -96], b \in [-400, -393], c \in [-1977, -1970], \text{ and } r \in [-10001, -9989]. \)

 You divided by the opposite of the factor AND multipled the first factor rather than just bringing it down.
\item \( a \in [17, 23], b \in [-15, -14], c \in [88, 95], \text{ and } r \in [-665, -656]. \)

 You multipled by the synthetic number and subtracted rather than adding during synthetic division.
\end{enumerate}

\textbf{General Comment:} Be sure to synthetically divide by the zero of the denominator! Also, make sure to include 0 placeholders for missing terms.
}
\litem{
Factor the polynomial below completely, knowing that $x -4$ is a factor. Then, choose the intervals the zeros of the polynomial belong to, where $z_1 \leq z_2 \leq z_3 \leq z_4$. \textit{To make the problem easier, all zeros are between -5 and 5.}
\[ f(x) = 8x^{4} +14 x^{3} -163 x^{2} -129 x + 180 \]The solution is \( [-5, -1.5, 0.75, 4] \), which is option C.\begin{enumerate}[label=\Alph*.]
\item \( z_1 \in [-4.9, -3.6], \text{   }  z_2 \in [-3.06, -2.98], z_3 \in [0.27, 0.47], \text{   and   } z_4 \in [4.4, 6.6] \)

 Distractor 4: Corresponds to moving factors from one rational to another.
\item \( z_1 \in [-4.9, -3.6], \text{   }  z_2 \in [-1.49, -1.28], z_3 \in [0.66, 0.7], \text{   and   } z_4 \in [4.4, 6.6] \)

 Distractor 3: Corresponds to negatives of all zeros AND inversing rational roots.
\item \( z_1 \in [-5.6, -4.8], \text{   }  z_2 \in [-1.51, -1.49], z_3 \in [0.7, 0.79], \text{   and   } z_4 \in [2.9, 4.2] \)

* This is the solution!
\item \( z_1 \in [-5.6, -4.8], \text{   }  z_2 \in [-0.68, -0.65], z_3 \in [1.25, 1.37], \text{   and   } z_4 \in [2.9, 4.2] \)

 Distractor 2: Corresponds to inversing rational roots.
\item \( z_1 \in [-4.9, -3.6], \text{   }  z_2 \in [-0.78, -0.72], z_3 \in [1.49, 1.51], \text{   and   } z_4 \in [4.4, 6.6] \)

 Distractor 1: Corresponds to negatives of all zeros.
\end{enumerate}

\textbf{General Comment:} Remember to try the middle-most integers first as these normally are the zeros. Also, once you get it to a quadratic, you can use your other factoring techniques to finish factoring.
}
\litem{
What are the \textit{possible Integer} roots of the polynomial below?
\[ f(x) = 3x^{4} +4 x^{3} +6 x^{2} +3 x + 5 \]The solution is \( \pm 1,\pm 5 \), which is option A.\begin{enumerate}[label=\Alph*.]
\item \( \pm 1,\pm 5 \)

* This is the solution \textbf{since we asked for the possible Integer roots}!
\item \( \text{ All combinations of: }\frac{\pm 1,\pm 5}{\pm 1,\pm 3} \)

This would have been the solution \textbf{if asked for the possible Rational roots}!
\item \( \text{ All combinations of: }\frac{\pm 1,\pm 3}{\pm 1,\pm 5} \)

 Distractor 3: Corresponds to the plus or minus of the inverse quotient (an/a0) of the factors. 
\item \( \pm 1,\pm 3 \)

 Distractor 1: Corresponds to the plus or minus factors of a1 only.
\item \( \text{There is no formula or theorem that tells us all possible Integer roots.} \)

 Distractor 4: Corresponds to not recognizing Integers as a subset of Rationals.
\end{enumerate}

\textbf{General Comment:} We have a way to find the possible Rational roots. The possible Integer roots are the Integers in this list.
}
\litem{
Perform the division below. Then, find the intervals that correspond to the quotient in the form $ax^2+bx+c$ and remainder $r$.
\[ \frac{15x^{3} +70 x^{2} +105 x + 53}{x + 2} \]The solution is \( 15x^{2} +40 x + 25 + \frac{3}{x + 2} \), which is option C.\begin{enumerate}[label=\Alph*.]
\item \( a \in [-30, -29], \text{   } b \in [128, 137], \text{   } c \in [-163, -151], \text{   and   } r \in [358, 366]. \)

 You multiplied by the synthetic number rather than bringing the first factor down.
\item \( a \in [14, 17], \text{   } b \in [97, 101], \text{   } c \in [298, 307], \text{   and   } r \in [663, 668]. \)

 You divided by the opposite of the factor.
\item \( a \in [14, 17], \text{   } b \in [39, 45], \text{   } c \in [23, 27], \text{   and   } r \in [3, 4]. \)

* This is the solution!
\item \( a \in [14, 17], \text{   } b \in [20, 29], \text{   } c \in [30, 33], \text{   and   } r \in [-37, -31]. \)

 You multiplied by the synthetic number and subtracted rather than adding during synthetic division.
\item \( a \in [-30, -29], \text{   } b \in [9, 11], \text{   } c \in [123, 128], \text{   and   } r \in [296, 309]. \)

 You divided by the opposite of the factor AND multiplied the first factor rather than just bringing it down.
\end{enumerate}

\textbf{General Comment:} Be sure to synthetically divide by the zero of the denominator!
}
\litem{
Perform the division below. Then, find the intervals that correspond to the quotient in the form $ax^2+bx+c$ and remainder $r$.
\[ \frac{6x^{3} +26 x^{2} -29}{x + 4} \]The solution is \( 6x^{2} +2 x -8 + \frac{3}{x + 4} \), which is option A.\begin{enumerate}[label=\Alph*.]
\item \( a \in [3, 10], b \in [2, 4], c \in [-11, -5], \text{ and } r \in [-6, 5]. \)

* This is the solution!
\item \( a \in [-27, -20], b \in [117, 124], c \in [-488, -487], \text{ and } r \in [1917, 1927]. \)

 You multipled by the synthetic number rather than bringing the first factor down.
\item \( a \in [3, 10], b \in [-9, 1], c \in [18, 21], \text{ and } r \in [-129, -128]. \)

 You multipled by the synthetic number and subtracted rather than adding during synthetic division.
\item \( a \in [-27, -20], b \in [-73, -64], c \in [-280, -274], \text{ and } r \in [-1151, -1146]. \)

 You divided by the opposite of the factor AND multipled the first factor rather than just bringing it down.
\item \( a \in [3, 10], b \in [47, 52], c \in [194, 202], \text{ and } r \in [769, 776]. \)

 You divided by the opposite of the factor.
\end{enumerate}

\textbf{General Comment:} Be sure to synthetically divide by the zero of the denominator! Also, make sure to include 0 placeholders for missing terms.
}
\end{enumerate}

\end{document}