\documentclass{extbook}[14pt]
\usepackage{multicol, enumerate, enumitem, hyperref, color, soul, setspace, parskip, fancyhdr, amssymb, amsthm, amsmath, latexsym, units, mathtools}
\everymath{\displaystyle}
\usepackage[headsep=0.5cm,headheight=0cm, left=1 in,right= 1 in,top= 1 in,bottom= 1 in]{geometry}
\usepackage{dashrule}  % Package to use the command below to create lines between items
\newcommand{\litem}[1]{\item #1

\rule{\textwidth}{0.4pt}}
\pagestyle{fancy}
\lhead{}
\chead{Answer Key for Progress Quiz 4 Version A}
\rhead{}
\lfoot{5346-5907}
\cfoot{}
\rfoot{Summer C 2021}
\begin{document}
\textbf{This key should allow you to understand why you choose the option you did (beyond just getting a question right or wrong). \href{https://xronos.clas.ufl.edu/mac1105spring2020/courseDescriptionAndMisc/Exams/LearningFromResults}{More instructions on how to use this key can be found here}.}

\textbf{If you have a suggestion to make the keys better, \href{https://forms.gle/CZkbZmPbC9XALEE88}{please fill out the short survey here}.}

\textit{Note: This key is auto-generated and may contain issues and/or errors. The keys are reviewed after each exam to ensure grading is done accurately. If there are issues (like duplicate options), they are noted in the offline gradebook. The keys are a work-in-progress to give students as many resources to improve as possible.}

\rule{\textwidth}{0.4pt}

\begin{enumerate}\litem{
Simplify the expression below into the form $a+bi$. Then, choose the intervals that $a$ and $b$ belong to.
\[ (3 - 6 i)(-2 + 8 i) \]The solution is \( 42 + 36 i \), which is option C.\begin{enumerate}[label=\Alph*.]
\item \( a \in [-56, -49] \text{ and } b \in [12, 15] \)

 $-54 + 12 i$, which corresponds to adding a minus sign in the first term.
\item \( a \in [41, 46] \text{ and } b \in [-39, -34] \)

 $42 - 36 i$, which corresponds to adding a minus sign in both terms.
\item \( a \in [41, 46] \text{ and } b \in [33, 44] \)

* $42 + 36 i$, which is the correct option.
\item \( a \in [-9, -5] \text{ and } b \in [-54, -45] \)

 $-6 - 48 i$, which corresponds to just multiplying the real terms to get the real part of the solution and the coefficients in the complex terms to get the complex part.
\item \( a \in [-56, -49] \text{ and } b \in [-13, -7] \)

 $-54 - 12 i$, which corresponds to adding a minus sign in the second term.
\end{enumerate}

\textbf{General Comment:} You can treat $i$ as a variable and distribute. Just remember that $i^2=-1$, so you can continue to reduce after you distribute.
}
\litem{
Simplify the expression below and choose the interval the simplification is contained within.
\[ 12 - 6^2 + 3 \div 15 * 2 \div 4 \]The solution is \( -23.900 \), which is option A.\begin{enumerate}[label=\Alph*.]
\item \( [-23.94, -23.85] \)

* -23.900, this is the correct option
\item \( [47.97, 48.03] \)

 48.025, which corresponds to two Order of Operations errors.
\item \( [48.03, 48.14] \)

 48.100, which corresponds to an Order of Operations error: multiplying by negative before squaring. For example: $(-3)^2 \neq -3^2$
\item \( [-24.03, -23.94] \)

 -23.975, which corresponds to an Order of Operations error: not reading left-to-right for multiplication/division.
\item \( \text{None of the above} \)

 You may have gotten this by making an unanticipated error. If you got a value that is not any of the others, please let the coordinator know so they can help you figure out what happened.
\end{enumerate}

\textbf{General Comment:} While you may remember (or were taught) PEMDAS is done in order, it is actually done as P/E/MD/AS. When we are at MD or AS, we read left to right.
}
\litem{
Choose the \textbf{smallest} set of Complex numbers that the number below belongs to.
\[ \sqrt{\frac{-2244}{12}} i+\sqrt{165}i \]The solution is \( \text{Nonreal Complex} \), which is option D.\begin{enumerate}[label=\Alph*.]
\item \( \text{Not a Complex Number} \)

This is not a number. The only non-Complex number we know is dividing by 0 as this is not a number!
\item \( \text{Pure Imaginary} \)

This is a Complex number $(a+bi)$ that \textbf{only} has an imaginary part like $2i$.
\item \( \text{Rational} \)

These are numbers that can be written as fraction of Integers (e.g., -2/3 + 5)
\item \( \text{Nonreal Complex} \)

* This is the correct option!
\item \( \text{Irrational} \)

These cannot be written as a fraction of Integers. Remember: $\pi$ is not an Integer!
\end{enumerate}

\textbf{General Comment:} Be sure to simplify $i^2 = -1$. This may remove the imaginary portion for your number. If you are having trouble, you may want to look at the \textit{Subgroups of the Real Numbers} section.
}
\litem{
Choose the \textbf{smallest} set of Real numbers that the number below belongs to.
\[ \sqrt{\frac{38025}{225}} \]The solution is \( \text{Whole} \), which is option C.\begin{enumerate}[label=\Alph*.]
\item \( \text{Not a Real number} \)

These are Nonreal Complex numbers \textbf{OR} things that are not numbers (e.g., dividing by 0).
\item \( \text{Integer} \)

These are the negative and positive counting numbers (..., -3, -2, -1, 0, 1, 2, 3, ...)
\item \( \text{Whole} \)

* This is the correct option!
\item \( \text{Rational} \)

These are numbers that can be written as fraction of Integers (e.g., -2/3)
\item \( \text{Irrational} \)

These cannot be written as a fraction of Integers.
\end{enumerate}

\textbf{General Comment:} First, you \textbf{NEED} to simplify the expression. This question simplifies to $195$. 
 
 Be sure you look at the simplified fraction and not just the decimal expansion. Numbers such as 13, 17, and 19 provide \textbf{long but repeating/terminating decimal expansions!} 
 
 The only ways to *not* be a Real number are: dividing by 0 or taking the square root of a negative number. 
 
 Irrational numbers are more than just square root of 3: adding or subtracting values from square root of 3 is also irrational.
}
\litem{
Simplify the expression below and choose the interval the simplification is contained within.
\[ 14 - 10^2 + 13 \div 20 * 8 \div 5 \]The solution is \( -84.960 \), which is option D.\begin{enumerate}[label=\Alph*.]
\item \( [-86.15, -85.68] \)

 -85.984, which corresponds to an Order of Operations error: not reading left-to-right for multiplication/division.
\item \( [114.67, 115.06] \)

 115.040, which corresponds to an Order of Operations error: multiplying by negative before squaring. For example: $(-3)^2 \neq -3^2$
\item \( [113.95, 114.92] \)

 114.016, which corresponds to two Order of Operations errors.
\item \( [-85.12, -83.54] \)

* -84.960, this is the correct option
\item \( \text{None of the above} \)

 You may have gotten this by making an unanticipated error. If you got a value that is not any of the others, please let the coordinator know so they can help you figure out what happened.
\end{enumerate}

\textbf{General Comment:} While you may remember (or were taught) PEMDAS is done in order, it is actually done as P/E/MD/AS. When we are at MD or AS, we read left to right.
}
\litem{
Choose the \textbf{smallest} set of Real numbers that the number below belongs to.
\[ \sqrt{\frac{15876}{36}} \]The solution is \( \text{Whole} \), which is option C.\begin{enumerate}[label=\Alph*.]
\item \( \text{Irrational} \)

These cannot be written as a fraction of Integers.
\item \( \text{Not a Real number} \)

These are Nonreal Complex numbers \textbf{OR} things that are not numbers (e.g., dividing by 0).
\item \( \text{Whole} \)

* This is the correct option!
\item \( \text{Integer} \)

These are the negative and positive counting numbers (..., -3, -2, -1, 0, 1, 2, 3, ...)
\item \( \text{Rational} \)

These are numbers that can be written as fraction of Integers (e.g., -2/3)
\end{enumerate}

\textbf{General Comment:} First, you \textbf{NEED} to simplify the expression. This question simplifies to $126$. 
 
 Be sure you look at the simplified fraction and not just the decimal expansion. Numbers such as 13, 17, and 19 provide \textbf{long but repeating/terminating decimal expansions!} 
 
 The only ways to *not* be a Real number are: dividing by 0 or taking the square root of a negative number. 
 
 Irrational numbers are more than just square root of 3: adding or subtracting values from square root of 3 is also irrational.
}
\litem{
Simplify the expression below into the form $a+bi$. Then, choose the intervals that $a$ and $b$ belong to.
\[ \frac{54 + 55 i}{-2 - 3 i} \]The solution is \( -21.00  + 4.00 i \), which is option B.\begin{enumerate}[label=\Alph*.]
\item \( a \in [-273.5, -271.5] \text{ and } b \in [3, 4.5] \)

 $-273.00  + 4.00 i$, which corresponds to forgetting to multiply the conjugate by the numerator and using a plus instead of a minus in the denominator.
\item \( a \in [-22, -20] \text{ and } b \in [3, 4.5] \)

* $-21.00  + 4.00 i$, which is the correct option.
\item \( a \in [-22, -20] \text{ and } b \in [51, 52.5] \)

 $-21.00  + 52.00 i$, which corresponds to forgetting to multiply the conjugate by the numerator.
\item \( a \in [-27.5, -26] \text{ and } b \in [-18.5, -17] \)

 $-27.00  - 18.33 i$, which corresponds to just dividing the first term by the first term and the second by the second.
\item \( a \in [3.5, 6] \text{ and } b \in [-21.5, -20.5] \)

 $4.38  - 20.92 i$, which corresponds to forgetting to multiply the conjugate by the numerator and not computing the conjugate correctly.
\end{enumerate}

\textbf{General Comment:} Multiply the numerator and denominator by the *conjugate* of the denominator, then simplify. For example, if we have $2+3i$, the conjugate is $2-3i$.
}
\litem{
Simplify the expression below into the form $a+bi$. Then, choose the intervals that $a$ and $b$ belong to.
\[ \frac{72 + 55 i}{7 - 3 i} \]The solution is \( 5.84  + 10.36 i \), which is option C.\begin{enumerate}[label=\Alph*.]
\item \( a \in [11, 12] \text{ and } b \in [2.5, 3.5] \)

 $11.53  + 2.91 i$, which corresponds to forgetting to multiply the conjugate by the numerator and not computing the conjugate correctly.
\item \( a \in [338.5, 340.5] \text{ and } b \in [8.5, 12.5] \)

 $339.00  + 10.36 i$, which corresponds to forgetting to multiply the conjugate by the numerator and using a plus instead of a minus in the denominator.
\item \( a \in [5.5, 6.5] \text{ and } b \in [8.5, 12.5] \)

* $5.84  + 10.36 i$, which is the correct option.
\item \( a \in [5.5, 6.5] \text{ and } b \in [600, 603] \)

 $5.84  + 601.00 i$, which corresponds to forgetting to multiply the conjugate by the numerator.
\item \( a \in [10, 10.5] \text{ and } b \in [-19.5, -17] \)

 $10.29  - 18.33 i$, which corresponds to just dividing the first term by the first term and the second by the second.
\end{enumerate}

\textbf{General Comment:} Multiply the numerator and denominator by the *conjugate* of the denominator, then simplify. For example, if we have $2+3i$, the conjugate is $2-3i$.
}
\litem{
Simplify the expression below into the form $a+bi$. Then, choose the intervals that $a$ and $b$ belong to.
\[ (-2 + 9 i)(-5 + 6 i) \]The solution is \( -44 - 57 i \), which is option B.\begin{enumerate}[label=\Alph*.]
\item \( a \in [58, 66] \text{ and } b \in [-40, -32] \)

 $64 - 33 i$, which corresponds to adding a minus sign in the second term.
\item \( a \in [-47, -43] \text{ and } b \in [-58, -56] \)

* $-44 - 57 i$, which is the correct option.
\item \( a \in [10, 12] \text{ and } b \in [49, 55] \)

 $10 + 54 i$, which corresponds to just multiplying the real terms to get the real part of the solution and the coefficients in the complex terms to get the complex part.
\item \( a \in [-47, -43] \text{ and } b \in [55, 60] \)

 $-44 + 57 i$, which corresponds to adding a minus sign in both terms.
\item \( a \in [58, 66] \text{ and } b \in [32, 37] \)

 $64 + 33 i$, which corresponds to adding a minus sign in the first term.
\end{enumerate}

\textbf{General Comment:} You can treat $i$ as a variable and distribute. Just remember that $i^2=-1$, so you can continue to reduce after you distribute.
}
\litem{
Choose the \textbf{smallest} set of Complex numbers that the number below belongs to.
\[ \frac{-18}{2}+\sqrt{-36}i \]The solution is \( \text{Rational} \), which is option C.\begin{enumerate}[label=\Alph*.]
\item \( \text{Irrational} \)

These cannot be written as a fraction of Integers. Remember: $\pi$ is not an Integer!
\item \( \text{Pure Imaginary} \)

This is a Complex number $(a+bi)$ that \textbf{only} has an imaginary part like $2i$.
\item \( \text{Rational} \)

* This is the correct option!
\item \( \text{Not a Complex Number} \)

This is not a number. The only non-Complex number we know is dividing by 0 as this is not a number!
\item \( \text{Nonreal Complex} \)

This is a Complex number $(a+bi)$ that is not Real (has $i$ as part of the number).
\end{enumerate}

\textbf{General Comment:} Be sure to simplify $i^2 = -1$. This may remove the imaginary portion for your number. If you are having trouble, you may want to look at the \textit{Subgroups of the Real Numbers} section.
}
\end{enumerate}

\end{document}