\documentclass{extbook}[14pt]
\usepackage{multicol, enumerate, enumitem, hyperref, color, soul, setspace, parskip, fancyhdr, amssymb, amsthm, amsmath, latexsym, units, mathtools}
\everymath{\displaystyle}
\usepackage[headsep=0.5cm,headheight=0cm, left=1 in,right= 1 in,top= 1 in,bottom= 1 in]{geometry}
\usepackage{dashrule}  % Package to use the command below to create lines between items
\newcommand{\litem}[1]{\item #1

\rule{\textwidth}{0.4pt}}
\pagestyle{fancy}
\lhead{}
\chead{Answer Key for Progress Quiz 4 Version C}
\rhead{}
\lfoot{5346-5907}
\cfoot{}
\rfoot{Summer C 2021}
\begin{document}
\textbf{This key should allow you to understand why you choose the option you did (beyond just getting a question right or wrong). \href{https://xronos.clas.ufl.edu/mac1105spring2020/courseDescriptionAndMisc/Exams/LearningFromResults}{More instructions on how to use this key can be found here}.}

\textbf{If you have a suggestion to make the keys better, \href{https://forms.gle/CZkbZmPbC9XALEE88}{please fill out the short survey here}.}

\textit{Note: This key is auto-generated and may contain issues and/or errors. The keys are reviewed after each exam to ensure grading is done accurately. If there are issues (like duplicate options), they are noted in the offline gradebook. The keys are a work-in-progress to give students as many resources to improve as possible.}

\rule{\textwidth}{0.4pt}

\begin{enumerate}\litem{
Solve the equation for $x$ and choose the interval that contains the solution (if it exists).
\[ \log_{2}{(-2x+8)}+4 = 3 \]The solution is \( x = 3.750 \), which is option A.\begin{enumerate}[label=\Alph*.]
\item \( x \in [3.7, 3.78] \)

* $x = 3.750$, which is the correct option.
\item \( x \in [-4.59, -4.19] \)

$x = -4.500$, which corresponds to reversing the base and exponent when converting and reversing the value with $x$.
\item \( x \in [-0.12, 0.17] \)

$x = -0.000$, which corresponds to ignoring the vertical shift when converting to exponential form.
\item \( x \in [3.36, 3.51] \)

$x = 3.500$, which corresponds to reversing the base and exponent when converting.
\item \( \text{There is no Real solution to the equation.} \)

Corresponds to believing a negative coefficient within the log equation means there is no Real solution.
\end{enumerate}

\textbf{General Comment:} \textbf{General Comments:} First, get the equation in the form $\log_b{(cx+d)} = a$. Then, convert to $b^a = cx+d$ and solve.
}
\litem{
 Solve the equation for $x$ and choose the interval that contains $x$ (if it exists).
\[  25 = \ln{\sqrt[5]{\frac{9}{e^{8x}}}} \]The solution is \( x = -15.35, \text{ which does not fit in any of the interval options.} \), which is option E.\begin{enumerate}[label=\Alph*.]
\item \( x \in [-5.98, -2.98] \)

$x = -5.975$, which corresponds to treating any root as a square root.
\item \( x \in [13.35, 17.35] \)

$x = 15.350$, which is the negative of the correct solution.
\item \( x \in [-3.29, -1.29] \)

$x = -2.286$, which corresponds to thinking you need to take the natural log of the left side before reducing.
\item \( \text{There is no Real solution to the equation.} \)

This corresponds to believing you cannot solve the equation.
\item \( \text{None of the above.} \)

*$x = -15.350$ is the correct solution and does not fit in any of the other intervals.
\end{enumerate}

\textbf{General Comment:} \textbf{General Comments}: After using the properties of logarithmic functions to break up the right-hand side, use $\ln(e) = 1$ to reduce the question to a linear function to solve. You can put $\ln(9)$ into a calculator if you are having trouble.
}
\litem{
Which of the following intervals describes the Range of the function below?
\[ f(x) = -\log_2{(x-2)}+6 \]The solution is \( (\infty, \infty) \), which is option E.\begin{enumerate}[label=\Alph*.]
\item \( (-\infty, a), a \in [4.6, 6.6] \)

$(-\infty, 6)$, which corresponds to using the vertical shift while the Range is $(-\infty, \infty)$.
\item \( [a, \infty), a \in [0.9, 2.9] \)

$[6, \infty)$, which corresponds to using the flipped Domain AND including the endpoint.
\item \( [a, \infty), a \in [-4.7, -1.7] \)

$[-2, \infty)$, which corresponds to using the negative of the horizontal shift AND including the endpoint.
\item \( (-\infty, a), a \in [-9.9, -3.8] \)

$(-\infty, -6)$, which corresponds to using the using the negative of vertical shift on $(0, \infty)$.
\item \( (-\infty, \infty) \)

*This is the correct option.
\end{enumerate}

\textbf{General Comment:} \textbf{General Comments}: The domain of a basic logarithmic function is $(0, \infty)$ and the Range is $(-\infty, \infty)$. We can use shifts when finding the Domain, but the Range will always be all Real numbers.
}
\litem{
Which of the following intervals describes the Range of the function below?
\[ f(x) = -e^{x+2}-2 \]The solution is \( (-\infty, -2) \), which is option A.\begin{enumerate}[label=\Alph*.]
\item \( (-\infty, a), a \in [-7, 0] \)

* $(-\infty, -2)$, which is the correct option.
\item \( (-\infty, a], a \in [-7, 0] \)

$(-\infty, -2]$, which corresponds to including the endpoint.
\item \( [a, \infty), a \in [0, 5] \)

$[2, \infty)$, which corresponds to using the negative vertical shift AND flipping the Range interval AND including the endpoint.
\item \( (a, \infty), a \in [0, 5] \)

$(2, \infty)$, which corresponds to using the negative vertical shift AND flipping the Range interval.
\item \( (-\infty, \infty) \)

This corresponds to confusing range of an exponential function with the domain of an exponential function.
\end{enumerate}

\textbf{General Comment:} \textbf{General Comments}: Domain of a basic exponential function is $(-\infty, \infty)$ while the Range is $(0, \infty)$. We can shift these intervals [and even flip when $a<0$!] to find the new Domain/Range.
}
\litem{
 Solve the equation for $x$ and choose the interval that contains $x$ (if it exists).
\[  12 = \sqrt[3]{\frac{12}{e^{3x}}} \]The solution is \( x = -1.657 \), which is option A.\begin{enumerate}[label=\Alph*.]
\item \( x \in [-1.86, -1.15] \)

* $x = -1.657$, which is the correct option.
\item \( x \in [-13.34, -12.38] \)

$x = -12.828$, which corresponds to thinking you don't need to take the natural log of both sides before reducing, as if the equation already had a natural log on the right side.
\item \( x \in [-1.4, -0.25] \)

$x = -0.828$, which corresponds to treating any root as a square root.
\item \( \text{There is no Real solution to the equation.} \)

This corresponds to believing you cannot solve the equation.
\item \( \text{None of the above.} \)

This corresponds to making an unexpected error.
\end{enumerate}

\textbf{General Comment:} \textbf{General Comments}: After using the properties of logarithmic functions to break up the right-hand side, use $\ln(e) = 1$ to reduce the question to a linear function to solve. You can put $\ln(12)$ into a calculator if you are having trouble.
}
\litem{
Which of the following intervals describes the Domain of the function below?
\[ f(x) = -e^{x-2}+7 \]The solution is \( (-\infty, \infty) \), which is option E.\begin{enumerate}[label=\Alph*.]
\item \( (-\infty, a], a \in [1, 8] \)

$(-\infty, 7]$, which corresponds to using the correct vertical shift *if we wanted the Range* AND including the endpoint.
\item \( (-\infty, a), a \in [1, 8] \)

$(-\infty, 7)$, which corresponds to using the correct vertical shift *if we wanted the Range*.
\item \( (a, \infty), a \in [-7, -3] \)

$(-7, \infty)$, which corresponds to using the negative vertical shift AND flipping the Range interval.
\item \( [a, \infty), a \in [-7, -3] \)

$[-7, \infty)$, which corresponds to using the negative vertical shift AND flipping the Range interval AND including the endpoint.
\item \( (-\infty, \infty) \)

* This is the correct option.
\end{enumerate}

\textbf{General Comment:} \textbf{General Comments}: Domain of a basic exponential function is $(-\infty, \infty)$ while the Range is $(0, \infty)$. We can shift these intervals [and even flip when $a<0$!] to find the new Domain/Range.
}
\litem{
Solve the equation for $x$ and choose the interval that contains the solution (if it exists).
\[ 5^{2x+5} = \left(\frac{1}{9}\right)^{3x+4} \]The solution is \( x = -1.716 \), which is option D.\begin{enumerate}[label=\Alph*.]
\item \( x \in [0.9, 2.6] \)

$x = 1.000$, which corresponds to solving the numerators as equal while ignoring the bases are different.
\item \( x \in [16.4, 18] \)

$x = 16.836$, which corresponds to distributing the $\ln(base)$ to the second term of the exponent only.
\item \( x \in [-1.1, 0.2] \)

$x = -0.102$, which corresponds to distributing the $\ln(base)$ to the first term of the exponent only.
\item \( x \in [-3.3, -1.3] \)

* $x = -1.716$, which is the correct option.
\item \( \text{There is no Real solution to the equation.} \)

This corresponds to believing there is no solution since the bases are not powers of each other.
\end{enumerate}

\textbf{General Comment:} \textbf{General Comments:} This question was written so that the bases could not be written the same. You will need to take the log of both sides.
}
\litem{
Which of the following intervals describes the Domain of the function below?
\[ f(x) = -\log_2{(x-8)}+3 \]The solution is \( (8, \infty) \), which is option C.\begin{enumerate}[label=\Alph*.]
\item \( (-\infty, a), a \in [-10, -7.3] \)

$(-\infty, -8)$, which corresponds to flipping the Domain. Remember: the general for is $a*\log(x-h)+k$, \textbf{where $a$ does not affect the domain}.
\item \( (-\infty, a], a \in [-5.4, 0.1] \)

$(-\infty, -3]$, which corresponds to using the negative vertical shift AND including the endpoint AND flipping the domain.
\item \( (a, \infty), a \in [7.7, 11.6] \)

* $(8, \infty)$, which is the correct option.
\item \( [a, \infty), a \in [-0.2, 3.5] \)

$[3, \infty)$, which corresponds to using the vertical shift when shifting the Domain AND including the endpoint.
\item \( (-\infty, \infty) \)

This corresponds to thinking of the range of the log function (or the domain of the exponential function).
\end{enumerate}

\textbf{General Comment:} \textbf{General Comments}: The domain of a basic logarithmic function is $(0, \infty)$ and the Range is $(-\infty, \infty)$. We can use shifts when finding the Domain, but the Range will always be all Real numbers.
}
\litem{
Solve the equation for $x$ and choose the interval that contains the solution (if it exists).
\[ 2^{2x-2} = \left(\frac{1}{343}\right)^{3x-5} \]The solution is \( x = 1.618 \), which is option C.\begin{enumerate}[label=\Alph*.]
\item \( x \in [2.9, 3.2] \)

$x = 3.000$, which corresponds to solving the numerators as equal while ignoring the bases are different.
\item \( x \in [-31.8, -30.1] \)

$x = -30.575$, which corresponds to distributing the $\ln(base)$ to the second term of the exponent only.
\item \( x \in [0.1, 1.7] \)

* $x = 1.618$, which is the correct option.
\item \( x \in [-0.6, 0.9] \)

$x = -0.159$, which corresponds to distributing the $\ln(base)$ to the first term of the exponent only.
\item \( \text{There is no Real solution to the equation.} \)

This corresponds to believing there is no solution since the bases are not powers of each other.
\end{enumerate}

\textbf{General Comment:} \textbf{General Comments:} This question was written so that the bases could not be written the same. You will need to take the log of both sides.
}
\litem{
Solve the equation for $x$ and choose the interval that contains the solution (if it exists).
\[ \log_{2}{(-2x+6)}+4 = 2 \]The solution is \( x = 2.875 \), which is option B.\begin{enumerate}[label=\Alph*.]
\item \( x \in [0.59, 1.84] \)

$x = 1.000$, which corresponds to reversing the base and exponent when converting.
\item \( x \in [1.64, 4.82] \)

* $x = 2.875$, which is the correct option.
\item \( x \in [0.59, 1.84] \)

$x = 1.000$, which corresponds to ignoring the vertical shift when converting to exponential form.
\item \( x \in [-6.08, -3.89] \)

$x = -5.000$, which corresponds to reversing the base and exponent when converting and reversing the value with $x$.
\item \( \text{There is no Real solution to the equation.} \)

Corresponds to believing a negative coefficient within the log equation means there is no Real solution.
\end{enumerate}

\textbf{General Comment:} \textbf{General Comments:} First, get the equation in the form $\log_b{(cx+d)} = a$. Then, convert to $b^a = cx+d$ and solve.
}
\end{enumerate}

\end{document}