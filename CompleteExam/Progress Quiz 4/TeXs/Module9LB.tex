\documentclass[14pt]{extbook}
\usepackage{multicol, enumerate, enumitem, hyperref, color, soul, setspace, parskip, fancyhdr} %General Packages
\usepackage{amssymb, amsthm, amsmath, bbm, latexsym, units, mathtools} %Math Packages
\everymath{\displaystyle} %All math in Display Style
% Packages with additional options
\usepackage[headsep=0.5cm,headheight=12pt, left=1 in,right= 1 in,top= 1 in,bottom= 1 in]{geometry}
\usepackage[usenames,dvipsnames]{xcolor}
\usepackage{dashrule}  % Package to use the command below to create lines between items
\newcommand{\litem}[1]{\item#1\hspace*{-1cm}\rule{\textwidth}{0.4pt}}
\pagestyle{fancy}
\lhead{Progress Quiz 4}
\chead{}
\rhead{Version B}
\lfoot{9187-5854}
\cfoot{}
\rfoot{Spring 2021}
\begin{document}

\begin{enumerate}
\litem{
Choose the interval below that $f$ composed with $g$ at $x=1$ is in.\[ f(x) = -3x^{3} + x^{2} +4 x \text{ and } g(x) = x^{3} -4 x^{2} +3 x \]\begin{enumerate}[label=\Alph*.]
\item \( (f \circ g)(1) \in [-10.36, -9.09] \)
\item \( (f \circ g)(1) \in [-0.68, 1.48] \)
\item \( (f \circ g)(1) \in [-6.93, -4.27] \)
\item \( (f \circ g)(1) \in [-2.19, -1.23] \)
\item \( \text{It is not possible to compose the two functions.} \)

\end{enumerate} }
\litem{
Find the inverse of the function below (if it exists). Then, evaluate the inverse at $x = 10$ and choose the interval that $f^{-1}(10)$ belongs to.\[ f(x) = 3 x^2 + 2 \]\begin{enumerate}[label=\Alph*.]
\item \( f^{-1}(10) \in [5.53, 5.96] \)
\item \( f^{-1}(10) \in [2.49, 2.74] \)
\item \( f^{-1}(10) \in [1.91, 2.26] \)
\item \( f^{-1}(10) \in [0.81, 1.76] \)
\item \( \text{ The function is not invertible for all Real numbers. } \)

\end{enumerate} }
\litem{
Find the inverse of the function below (if it exists). Then, evaluate the inverse at $x = 14$ and choose the interval the $f^{-1}(14)$ belongs to.\[ f(x) = \sqrt[3]{4 x - 5} \]\begin{enumerate}[label=\Alph*.]
\item \( f^{-1}(14) \in [678.75, 685.75] \)
\item \( f^{-1}(14) \in [-688.25, -685.25] \)
\item \( f^{-1}(14) \in [-684.75, -681.75] \)
\item \( f^{-1}(14) \in [686.25, 689.25] \)
\item \( \text{ The function is not invertible for all Real numbers. } \)

\end{enumerate} }
\litem{
Choose the interval below that $f$ composed with $g$ at $x=-1$ is in.\[ f(x) = 4x^{3} +2 x^{2} -4 x + 1 \text{ and } g(x) = -x^{3} -3 x^{2} -3 x \]\begin{enumerate}[label=\Alph*.]
\item \( (f \circ g)(-1) \in [-63, -58] \)
\item \( (f \circ g)(-1) \in [-1, 8] \)
\item \( (f \circ g)(-1) \in [-9, -5] \)
\item \( (f \circ g)(-1) \in [-55, -50] \)
\item \( \text{It is not possible to compose the two functions.} \)

\end{enumerate} }
\litem{
Determine whether the function below is 1-1.\[ f(x) = 12 x^2 - 42 x - 132 \]\begin{enumerate}[label=\Alph*.]
\item \( \text{No, because there is a $y$-value that goes to 2 different $x$-values.} \)
\item \( \text{No, because the domain of the function is not $(-\infty, \infty)$.} \)
\item \( \text{Yes, the function is 1-1.} \)
\item \( \text{No, because the range of the function is not $(-\infty, \infty)$.} \)
\item \( \text{No, because there is an $x$-value that goes to 2 different $y$-values.} \)

\end{enumerate} }
\litem{
Find the inverse of the function below. Then, evaluate the inverse at $x = 10$ and choose the interval that $f^{-1}(10)$ belongs to.\[ f(x) = e^{x+4}-5 \]\begin{enumerate}[label=\Alph*.]
\item \( f^{-1}(10) \in [-2.48, -2.36] \)
\item \( f^{-1}(10) \in [6.48, 6.81] \)
\item \( f^{-1}(10) \in [-3.33, -2.99] \)
\item \( f^{-1}(10) \in [-1.51, -1.22] \)
\item \( f^{-1}(10) \in [-3.43, -3.32] \)

\end{enumerate} }
\litem{
Find the inverse of the function below. Then, evaluate the inverse at $x = 8$ and choose the interval that $f^{-1}(8)$ belongs to.\[ f(x) = e^{x+2}-4 \]\begin{enumerate}[label=\Alph*.]
\item \( f^{-1}(8) \in [-2.56, -2.18] \)
\item \( f^{-1}(8) \in [-0.28, 0.99] \)
\item \( f^{-1}(8) \in [3.2, 6.23] \)
\item \( f^{-1}(8) \in [-4.9, -2.33] \)
\item \( f^{-1}(8) \in [-2.12, -0.75] \)

\end{enumerate} }
\litem{
Multiply the following functions, then choose the domain of the resulting function from the list below.\[ f(x) = \sqrt{5x-25}  \text{ and } g(x) = 9x^{3} +8 x^{2} +7 x + 8 \]\begin{enumerate}[label=\Alph*.]
\item \( \text{ The domain is all Real numbers except } x = a, \text{ where } a \in [4.25, 9.25] \)
\item \( \text{ The domain is all Real numbers less than or equal to } x = a, \text{ where } a \in [-5.33, 3.67] \)
\item \( \text{ The domain is all Real numbers greater than or equal to } x = a, \text{ where } a \in [-1, 9] \)
\item \( \text{ The domain is all Real numbers except } x = a \text{ and } x = b, \text{ where } a \in [-8.83, -1.83] \text{ and } b \in [-7.25, -3.25] \)
\item \( \text{ The domain is all Real numbers. } \)

\end{enumerate} }
\litem{
Determine whether the function below is 1-1.\[ f(x) = \sqrt{-4 x - 15} \]\begin{enumerate}[label=\Alph*.]
\item \( \text{No, because there is an $x$-value that goes to 2 different $y$-values.} \)
\item \( \text{No, because there is a $y$-value that goes to 2 different $x$-values.} \)
\item \( \text{No, because the domain of the function is not $(-\infty, \infty)$.} \)
\item \( \text{No, because the range of the function is not $(-\infty, \infty)$.} \)
\item \( \text{Yes, the function is 1-1.} \)

\end{enumerate} }
\litem{
Multiply the following functions, then choose the domain of the resulting function from the list below.\[ f(x) = 8x^{2} +5 x + 4 \text{ and } g(x) = \sqrt{6x-27}  \]\begin{enumerate}[label=\Alph*.]
\item \( \text{ The domain is all Real numbers greater than or equal to } x = a, \text{ where } a \in [4.5, 6.5] \)
\item \( \text{ The domain is all Real numbers except } x = a, \text{ where } a \in [4.4, 9.4] \)
\item \( \text{ The domain is all Real numbers less than or equal to } x = a, \text{ where } a \in [-10.75, 1.25] \)
\item \( \text{ The domain is all Real numbers except } x = a \text{ and } x = b, \text{ where } a \in [-9.17, 1.83] \text{ and } b \in [-6.2, 5.8] \)
\item \( \text{ The domain is all Real numbers. } \)

\end{enumerate} }
\end{enumerate}

\end{document}