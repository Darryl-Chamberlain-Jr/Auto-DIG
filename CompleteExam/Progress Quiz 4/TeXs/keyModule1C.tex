\documentclass{extbook}[14pt]
\usepackage{multicol, enumerate, enumitem, hyperref, color, soul, setspace, parskip, fancyhdr, amssymb, amsthm, amsmath, latexsym, units, mathtools}
\everymath{\displaystyle}
\usepackage[headsep=0.5cm,headheight=0cm, left=1 in,right= 1 in,top= 1 in,bottom= 1 in]{geometry}
\usepackage{dashrule}  % Package to use the command below to create lines between items
\newcommand{\litem}[1]{\item #1

\rule{\textwidth}{0.4pt}}
\pagestyle{fancy}
\lhead{}
\chead{Answer Key for Progress Quiz 4 Version C}
\rhead{}
\lfoot{5346-5907}
\cfoot{}
\rfoot{Summer C 2021}
\begin{document}
\textbf{This key should allow you to understand why you choose the option you did (beyond just getting a question right or wrong). \href{https://xronos.clas.ufl.edu/mac1105spring2020/courseDescriptionAndMisc/Exams/LearningFromResults}{More instructions on how to use this key can be found here}.}

\textbf{If you have a suggestion to make the keys better, \href{https://forms.gle/CZkbZmPbC9XALEE88}{please fill out the short survey here}.}

\textit{Note: This key is auto-generated and may contain issues and/or errors. The keys are reviewed after each exam to ensure grading is done accurately. If there are issues (like duplicate options), they are noted in the offline gradebook. The keys are a work-in-progress to give students as many resources to improve as possible.}

\rule{\textwidth}{0.4pt}

\begin{enumerate}\litem{
Simplify the expression below into the form $a+bi$. Then, choose the intervals that $a$ and $b$ belong to.
\[ (-9 + 6 i)(-3 - 7 i) \]The solution is \( 69 + 45 i \), which is option B.\begin{enumerate}[label=\Alph*.]
\item \( a \in [27, 36] \text{ and } b \in [-42.6, -39.8] \)

 $27 - 42 i$, which corresponds to just multiplying the real terms to get the real part of the solution and the coefficients in the complex terms to get the complex part.
\item \( a \in [63, 72] \text{ and } b \in [42.5, 48.1] \)

* $69 + 45 i$, which is the correct option.
\item \( a \in [63, 72] \text{ and } b \in [-45.8, -44.8] \)

 $69 - 45 i$, which corresponds to adding a minus sign in both terms.
\item \( a \in [-15, -12] \text{ and } b \in [77.3, 81.3] \)

 $-15 + 81 i$, which corresponds to adding a minus sign in the first term.
\item \( a \in [-15, -12] \text{ and } b \in [-81.8, -80.1] \)

 $-15 - 81 i$, which corresponds to adding a minus sign in the second term.
\end{enumerate}

\textbf{General Comment:} You can treat $i$ as a variable and distribute. Just remember that $i^2=-1$, so you can continue to reduce after you distribute.
}
\litem{
Simplify the expression below and choose the interval the simplification is contained within.
\[ 2 - 14^2 + 15 \div 1 * 13 \div 11 \]The solution is \( -176.273 \), which is option C.\begin{enumerate}[label=\Alph*.]
\item \( [-193.9, -191.9] \)

 -193.895, which corresponds to an Order of Operations error: not reading left-to-right for multiplication/division.
\item \( [196.1, 205.1] \)

 198.105, which corresponds to two Order of Operations errors.
\item \( [-182.27, -169.27] \)

* -176.273, this is the correct option
\item \( [214.73, 220.73] \)

 215.727, which corresponds to an Order of Operations error: multiplying by negative before squaring. For example: $(-3)^2 \neq -3^2$
\item \( \text{None of the above} \)

 You may have gotten this by making an unanticipated error. If you got a value that is not any of the others, please let the coordinator know so they can help you figure out what happened.
\end{enumerate}

\textbf{General Comment:} While you may remember (or were taught) PEMDAS is done in order, it is actually done as P/E/MD/AS. When we are at MD or AS, we read left to right.
}
\litem{
Choose the \textbf{smallest} set of Complex numbers that the number below belongs to.
\[ \frac{0}{-12 \pi}+\sqrt{6}i \]The solution is \( \text{Pure Imaginary} \), which is option B.\begin{enumerate}[label=\Alph*.]
\item \( \text{Rational} \)

These are numbers that can be written as fraction of Integers (e.g., -2/3 + 5)
\item \( \text{Pure Imaginary} \)

* This is the correct option!
\item \( \text{Not a Complex Number} \)

This is not a number. The only non-Complex number we know is dividing by 0 as this is not a number!
\item \( \text{Nonreal Complex} \)

This is a Complex number $(a+bi)$ that is not Real (has $i$ as part of the number).
\item \( \text{Irrational} \)

These cannot be written as a fraction of Integers. Remember: $\pi$ is not an Integer!
\end{enumerate}

\textbf{General Comment:} Be sure to simplify $i^2 = -1$. This may remove the imaginary portion for your number. If you are having trouble, you may want to look at the \textit{Subgroups of the Real Numbers} section.
}
\litem{
Choose the \textbf{smallest} set of Real numbers that the number below belongs to.
\[ \sqrt{\frac{64}{121}} \]The solution is \( \text{Rational} \), which is option E.\begin{enumerate}[label=\Alph*.]
\item \( \text{Whole} \)

These are the counting numbers with 0 (0, 1, 2, 3, ...)
\item \( \text{Irrational} \)

These cannot be written as a fraction of Integers.
\item \( \text{Not a Real number} \)

These are Nonreal Complex numbers \textbf{OR} things that are not numbers (e.g., dividing by 0).
\item \( \text{Integer} \)

These are the negative and positive counting numbers (..., -3, -2, -1, 0, 1, 2, 3, ...)
\item \( \text{Rational} \)

* This is the correct option!
\end{enumerate}

\textbf{General Comment:} First, you \textbf{NEED} to simplify the expression. This question simplifies to $\frac{8}{11}$. 
 
 Be sure you look at the simplified fraction and not just the decimal expansion. Numbers such as 13, 17, and 19 provide \textbf{long but repeating/terminating decimal expansions!} 
 
 The only ways to *not* be a Real number are: dividing by 0 or taking the square root of a negative number. 
 
 Irrational numbers are more than just square root of 3: adding or subtracting values from square root of 3 is also irrational.
}
\litem{
Simplify the expression below and choose the interval the simplification is contained within.
\[ 14 - 9^2 + 3 \div 7 * 8 \div 4 \]The solution is \( -66.143 \), which is option B.\begin{enumerate}[label=\Alph*.]
\item \( [95.76, 96.27] \)

 95.857, which corresponds to an Order of Operations error: multiplying by negative before squaring. For example: $(-3)^2 \neq -3^2$
\item \( [-66.87, -65.79] \)

* -66.143, this is the correct option
\item \( [94.97, 95.85] \)

 95.013, which corresponds to two Order of Operations errors.
\item \( [-67.26, -66.98] \)

 -66.987, which corresponds to an Order of Operations error: not reading left-to-right for multiplication/division.
\item \( \text{None of the above} \)

 You may have gotten this by making an unanticipated error. If you got a value that is not any of the others, please let the coordinator know so they can help you figure out what happened.
\end{enumerate}

\textbf{General Comment:} While you may remember (or were taught) PEMDAS is done in order, it is actually done as P/E/MD/AS. When we are at MD or AS, we read left to right.
}
\litem{
Choose the \textbf{smallest} set of Real numbers that the number below belongs to.
\[ -\sqrt{\frac{484}{49}} \]The solution is \( \text{Rational} \), which is option A.\begin{enumerate}[label=\Alph*.]
\item \( \text{Rational} \)

* This is the correct option!
\item \( \text{Whole} \)

These are the counting numbers with 0 (0, 1, 2, 3, ...)
\item \( \text{Irrational} \)

These cannot be written as a fraction of Integers.
\item \( \text{Integer} \)

These are the negative and positive counting numbers (..., -3, -2, -1, 0, 1, 2, 3, ...)
\item \( \text{Not a Real number} \)

These are Nonreal Complex numbers \textbf{OR} things that are not numbers (e.g., dividing by 0).
\end{enumerate}

\textbf{General Comment:} First, you \textbf{NEED} to simplify the expression. This question simplifies to $-\frac{22}{7}$. 
 
 Be sure you look at the simplified fraction and not just the decimal expansion. Numbers such as 13, 17, and 19 provide \textbf{long but repeating/terminating decimal expansions!} 
 
 The only ways to *not* be a Real number are: dividing by 0 or taking the square root of a negative number. 
 
 Irrational numbers are more than just square root of 3: adding or subtracting values from square root of 3 is also irrational.
}
\litem{
Simplify the expression below into the form $a+bi$. Then, choose the intervals that $a$ and $b$ belong to.
\[ \frac{-18 - 55 i}{1 - 6 i} \]The solution is \( 8.43  - 4.41 i \), which is option C.\begin{enumerate}[label=\Alph*.]
\item \( a \in [-18.5, -17.5] \text{ and } b \in [8, 9.5] \)

 $-18.00  + 9.17 i$, which corresponds to just dividing the first term by the first term and the second by the second.
\item \( a \in [311, 313] \text{ and } b \in [-5.5, -3.5] \)

 $312.00  - 4.41 i$, which corresponds to forgetting to multiply the conjugate by the numerator and using a plus instead of a minus in the denominator.
\item \( a \in [7.5, 9.5] \text{ and } b \in [-5.5, -3.5] \)

* $8.43  - 4.41 i$, which is the correct option.
\item \( a \in [-9.5, -8.5] \text{ and } b \in [0.5, 2.5] \)

 $-9.41  + 1.43 i$, which corresponds to forgetting to multiply the conjugate by the numerator and not computing the conjugate correctly.
\item \( a \in [7.5, 9.5] \text{ and } b \in [-163.5, -162] \)

 $8.43  - 163.00 i$, which corresponds to forgetting to multiply the conjugate by the numerator.
\end{enumerate}

\textbf{General Comment:} Multiply the numerator and denominator by the *conjugate* of the denominator, then simplify. For example, if we have $2+3i$, the conjugate is $2-3i$.
}
\litem{
Simplify the expression below into the form $a+bi$. Then, choose the intervals that $a$ and $b$ belong to.
\[ \frac{45 - 88 i}{-3 + 4 i} \]The solution is \( -19.48  + 3.36 i \), which is option A.\begin{enumerate}[label=\Alph*.]
\item \( a \in [-20.5, -18] \text{ and } b \in [1.5, 5] \)

* $-19.48  + 3.36 i$, which is the correct option.
\item \( a \in [-16, -14.5] \text{ and } b \in [-22.5, -21.5] \)

 $-15.00  - 22.00 i$, which corresponds to just dividing the first term by the first term and the second by the second.
\item \( a \in [-487.5, -486] \text{ and } b \in [1.5, 5] \)

 $-487.00  + 3.36 i$, which corresponds to forgetting to multiply the conjugate by the numerator and using a plus instead of a minus in the denominator.
\item \( a \in [7.5, 10] \text{ and } b \in [17, 18] \)

 $8.68  + 17.76 i$, which corresponds to forgetting to multiply the conjugate by the numerator and not computing the conjugate correctly.
\item \( a \in [-20.5, -18] \text{ and } b \in [83.5, 85] \)

 $-19.48  + 84.00 i$, which corresponds to forgetting to multiply the conjugate by the numerator.
\end{enumerate}

\textbf{General Comment:} Multiply the numerator and denominator by the *conjugate* of the denominator, then simplify. For example, if we have $2+3i$, the conjugate is $2-3i$.
}
\litem{
Simplify the expression below into the form $a+bi$. Then, choose the intervals that $a$ and $b$ belong to.
\[ (4 + 6 i)(3 + 10 i) \]The solution is \( -48 + 58 i \), which is option E.\begin{enumerate}[label=\Alph*.]
\item \( a \in [69, 79] \text{ and } b \in [22, 29] \)

 $72 + 22 i$, which corresponds to adding a minus sign in the first term.
\item \( a \in [-49, -46] \text{ and } b \in [-59, -56] \)

 $-48 - 58 i$, which corresponds to adding a minus sign in both terms.
\item \( a \in [11, 18] \text{ and } b \in [60, 68] \)

 $12 + 60 i$, which corresponds to just multiplying the real terms to get the real part of the solution and the coefficients in the complex terms to get the complex part.
\item \( a \in [69, 79] \text{ and } b \in [-25, -16] \)

 $72 - 22 i$, which corresponds to adding a minus sign in the second term.
\item \( a \in [-49, -46] \text{ and } b \in [58, 59] \)

* $-48 + 58 i$, which is the correct option.
\end{enumerate}

\textbf{General Comment:} You can treat $i$ as a variable and distribute. Just remember that $i^2=-1$, so you can continue to reduce after you distribute.
}
\litem{
Choose the \textbf{smallest} set of Complex numbers that the number below belongs to.
\[ \frac{\sqrt{165}}{8}+\sqrt{-10}i \]The solution is \( \text{Irrational} \), which is option A.\begin{enumerate}[label=\Alph*.]
\item \( \text{Irrational} \)

* This is the correct option!
\item \( \text{Rational} \)

These are numbers that can be written as fraction of Integers (e.g., -2/3 + 5)
\item \( \text{Not a Complex Number} \)

This is not a number. The only non-Complex number we know is dividing by 0 as this is not a number!
\item \( \text{Nonreal Complex} \)

This is a Complex number $(a+bi)$ that is not Real (has $i$ as part of the number).
\item \( \text{Pure Imaginary} \)

This is a Complex number $(a+bi)$ that \textbf{only} has an imaginary part like $2i$.
\end{enumerate}

\textbf{General Comment:} Be sure to simplify $i^2 = -1$. This may remove the imaginary portion for your number. If you are having trouble, you may want to look at the \textit{Subgroups of the Real Numbers} section.
}
\end{enumerate}

\end{document}