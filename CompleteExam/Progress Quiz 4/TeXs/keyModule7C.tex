\documentclass{extbook}[14pt]
\usepackage{multicol, enumerate, enumitem, hyperref, color, soul, setspace, parskip, fancyhdr, amssymb, amsthm, amsmath, latexsym, units, mathtools}
\everymath{\displaystyle}
\usepackage[headsep=0.5cm,headheight=0cm, left=1 in,right= 1 in,top= 1 in,bottom= 1 in]{geometry}
\usepackage{dashrule}  % Package to use the command below to create lines between items
\newcommand{\litem}[1]{\item #1

\rule{\textwidth}{0.4pt}}
\pagestyle{fancy}
\lhead{}
\chead{Answer Key for Progress Quiz 4 Version C}
\rhead{}
\lfoot{5346-5907}
\cfoot{}
\rfoot{Summer C 2021}
\begin{document}
\textbf{This key should allow you to understand why you choose the option you did (beyond just getting a question right or wrong). \href{https://xronos.clas.ufl.edu/mac1105spring2020/courseDescriptionAndMisc/Exams/LearningFromResults}{More instructions on how to use this key can be found here}.}

\textbf{If you have a suggestion to make the keys better, \href{https://forms.gle/CZkbZmPbC9XALEE88}{please fill out the short survey here}.}

\textit{Note: This key is auto-generated and may contain issues and/or errors. The keys are reviewed after each exam to ensure grading is done accurately. If there are issues (like duplicate options), they are noted in the offline gradebook. The keys are a work-in-progress to give students as many resources to improve as possible.}

\rule{\textwidth}{0.4pt}

\begin{enumerate}\litem{
Choose the graph of the equation below.
\[ f(x) = \frac{1}{x - 1} + 1 \]The solution is the graph below, which is option D.
    \begin{center}
        \includegraphics[width=0.3\textwidth]{../Figures/rationalEquationToGraphDC.png}
    \end{center}\begin{enumerate}[label=\Alph*.]
\begin{multicols}{2}
\item \includegraphics[width = 0.3\textwidth]{../Figures/rationalEquationToGraphAC.png}
\item \includegraphics[width = 0.3\textwidth]{../Figures/rationalEquationToGraphBC.png}
\item \includegraphics[width = 0.3\textwidth]{../Figures/rationalEquationToGraphCC.png}
\item \includegraphics[width = 0.3\textwidth]{../Figures/rationalEquationToGraphDC.png}
\end{multicols}\item None of the above.\end{enumerate}
\textbf{General Comment:} Remember that the general form of a basic rational equation is $ f(x) = \frac{a}{(x-h)^n} + k$, where $a$ is the leading coefficient (and in this case, we assume is either $1$ or $-1$), $n$ is the degree (in this case, either $1$ or $2$), and $(h, k)$ is the intersection of the asymptotes.
}
\litem{
Choose the equation of the function graphed below.

\begin{center}
    \includegraphics[width=0.5\textwidth]{../Figures/rationalGraphToEquationCopyC.png}
\end{center}


The solution is \( f(x) = \frac{-1}{x + 3} + 2 \), which is option B.\begin{enumerate}[label=\Alph*.]
\item \( f(x) = \frac{1}{x - 3} + 2 \)

Corresponds to using the general form $f(x) = \frac{a}{x+h}+k$ and the opposite leading coefficient.
\item \( f(x) = \frac{-1}{x + 3} + 2 \)

This is the correct option.
\item \( f(x) = \frac{-1}{(x + 3)^2} + 2 \)

Corresponds to thinking the graph was a shifted version of $\frac{1}{x^2}$.
\item \( f(x) = \frac{1}{(x - 3)^2} + 2 \)

Corresponds to thinking the graph was a shifted version of $\frac{1}{x^2}$, using the general form $f(x) = \frac{a}{x+h}+k$, and the opposite leading coefficient.
\item \( \text{None of the above} \)

This corresponds to believing the vertex of the graph was not correct.
\end{enumerate}

\textbf{General Comment:} Remember that the general form of a basic rational equation is $ f(x) = \frac{a}{(x-h)^n} + k$, where $a$ is the leading coefficient (and in this case, we assume is either $1$ or $-1$), $n$ is the degree (in this case, either $1$ or $2$), and $(h, k)$ is the intersection of the asymptotes.
}
\litem{
Choose the equation of the function graphed below.

\begin{center}
    \includegraphics[width=0.5\textwidth]{../Figures/rationalGraphToEquationC.png}
\end{center}


The solution is \( \text{None of the above as it should be } f(x) = \frac{1}{(x - 1)^2} - 3 \), which is option E.\begin{enumerate}[label=\Alph*.]
\item \( f(x) = \frac{-1}{x - 1} + 4 \)

Corresponds to thinking the graph was a shifted version of $\frac{1}{x}$, using the general form $f(x) = \frac{a}{(x-h)^2}+k$, the opposite leading coefficient, AND not noticing the $y$-value was wrong.
\item \( f(x) = \frac{-1}{(x - 1)^2} + 4 \)

Corresponds to using the general form $f(x) = \frac{a}{(x-h)^2}+k$, the opposite leading coefficient, AND not noticing the $y$-value was wrong.
\item \( f(x) = \frac{1}{(x + 1)^2} + 4 \)

The $x$- and $y$-value of the equation does not match the graph.
\item \( f(x) = \frac{1}{x + 1} + 4 \)

Corresponds to thinking the graph was a shifted version of $\frac{1}{x}$ AND not noticing the $y$-value was wrong.
\item \( \text{None of the above} \)

None of the equation options were the correct equation.
\end{enumerate}

\textbf{General Comment:} Remember that the general form of a basic rational equation is $ f(x) = \frac{a}{(x-h)^n} + k$, where $a$ is the leading coefficient (and in this case, we assume is either $1$ or $-1$), $n$ is the degree (in this case, either $1$ or $2$), and $(h, k)$ is the intersection of the asymptotes.
}
\litem{
Solve the rational equation below. Then, choose the interval(s) that the solution(s) belongs to.
\[ \frac{-39}{-78x -39} + 1 = \frac{-39}{-78x -39} \]The solution is \( \text{all solutions are invalid or lead to complex values in the equation.} \), which is option C.\begin{enumerate}[label=\Alph*.]
\item \( x_1 \in [-0.6, 0.2] \text{ and } x_2 \in [-0.3,2.1] \)

$x = -0.500 \text{ and } x = 0.500$, which corresponds to getting the correct solution and believing there should be a second solution to the equation.
\item \( x \in [-0.5,0.5] \)

$x = -0.500$, which corresponds to not checking if this value leads to dividing by 0 in the original equation and thus is not a valid solution.
\item \( \text{All solutions lead to invalid or complex values in the equation.} \)

*$x = -0.500$ leads to dividing by 0 in the original equation and thus is not a valid solution, which is the correct option.
\item \( x \in [0.3,1.4] \)

$x = 0.500$, which corresponds to not distributing the factor $-78x -39$ correctly when trying to eliminate the fraction.
\item \( x_1 \in [-0.6, 0.2] \text{ and } x_2 \in [-0.6,-0.2] \)

$x = -0.500 \text{ and } x = -0.500$, which corresponds to getting the correct solution and believing there should be a second solution to the equation.
\end{enumerate}

\textbf{General Comment:} Distractors are different based on the number of solutions. Remember that after solving, we need to make sure our solution does not make the original equation divide by zero!
}
\litem{
Solve the rational equation below. Then, choose the interval(s) that the solution(s) belongs to.
\[ \frac{6x}{6x + 6} + \frac{-4x^{2}}{-12x^{2} -36 x -24} = \frac{5}{-2x -4} \]The solution is \( \text{There are two solutions: } x = -2.674 \text{ and } x = -0.701 \), which is option B.\begin{enumerate}[label=\Alph*.]
\item \( x_1 \in [-3.46, -2.65] \text{ and } x_2 \in [-1.06,-0.71] \)


\item \( x_1 \in [-3.46, -2.65] \text{ and } x_2 \in [-0.77,-0.49] \)

* $x = -2.674 \text{ and } x = -0.701$, which is the correct option.
\item \( x \in [-2.32,-0.88] \)


\item \( \text{All solutions lead to invalid or complex values in the equation.} \)


\item \( x \in [-1.35,0.49] \)


\end{enumerate}

\textbf{General Comment:} Distractors are different based on the number of solutions. Remember that after solving, we need to make sure our solution does not make the original equation divide by zero!
}
\litem{
Determine the domain of the function below.
\[ f(x) = \frac{3}{12x^{2} +29 x + 15} \]The solution is \( \text{All Real numbers except } x = -1.667 \text{ and } x = -0.750. \), which is option C.\begin{enumerate}[label=\Alph*.]
\item \( \text{All Real numbers except } x = a \text{ and } x = b, \text{ where } a \in [-15.6, -13.6] \text{ and } b \in [-13.9, -11.7] \)

All Real numbers except $x = -15.000$ and $x = -12.000$, which corresponds to not factoring the denominator correctly.
\item \( \text{All Real numbers except } x = a, \text{ where } a \in [-1.7, -1.5] \)

All Real numbers except $x = -1.667$, which corresponds to removing only 1 value from the denominator.
\item \( \text{All Real numbers except } x = a \text{ and } x = b, \text{ where } a \in [-1.7, -1.5] \text{ and } b \in [-0.8, -0.2] \)

All Real numbers except $x = -1.667$ and $x = -0.750$, which is the correct option.
\item \( \text{All Real numbers.} \)

This corresponds to thinking the denominator has complex roots or that rational functions have a domain of all Real numbers.
\item \( \text{All Real numbers except } x = a, \text{ where } a \in [-15.6, -13.6] \)

All Real numbers except $x = -15.000$, which corresponds to removing a distractor value from the denominator.
\end{enumerate}

\textbf{General Comment:} Recall that dividing by zero is not a real number. Therefore the domain is all real numbers \textbf{except} those that make the denominator 0.
}
\litem{
Solve the rational equation below. Then, choose the interval(s) that the solution(s) belongs to.
\[ \frac{84}{98x -98} + 1 = \frac{84}{98x -98} \]The solution is \( \text{all solutions are invalid or lead to complex values in the equation.} \), which is option B.\begin{enumerate}[label=\Alph*.]
\item \( x_1 \in [-1.3, 0.3] \text{ and } x_2 \in [0,3] \)

$x = -1.000 \text{ and } x = 1.000$, which corresponds to getting the correct solution and believing there should be a second solution to the equation.
\item \( \text{All solutions lead to invalid or complex values in the equation.} \)

*$x = 1.000$ leads to dividing by 0 in the original equation and thus is not a valid solution, which is the correct option.
\item \( x_1 \in [0.4, 1.8] \text{ and } x_2 \in [0,3] \)

$x = 1.000 \text{ and } x = 1.000$, which corresponds to getting the correct solution and believing there should be a second solution to the equation.
\item \( x \in [1.0,2.0] \)

$x = 1.000$, which corresponds to not checking if this value leads to dividing by 0 in the original equation and thus is not a valid solution.
\item \( x \in [-1.3,0.3] \)

$x = -1.000$, which corresponds to not distributing the factor $98x -98$ correctly when trying to eliminate the fraction.
\end{enumerate}

\textbf{General Comment:} Distractors are different based on the number of solutions. Remember that after solving, we need to make sure our solution does not make the original equation divide by zero!
}
\litem{
Solve the rational equation below. Then, choose the interval(s) that the solution(s) belongs to.
\[ \frac{4x}{-5x + 6} + \frac{-4x^{2}}{-15x^{2} +28 x -12} = \frac{3}{3x -2} \]The solution is \( \text{There are two solutions: } x = -2.000 \text{ and } x = 1.125 \), which is option C.\begin{enumerate}[label=\Alph*.]
\item \( x \in [0.97,1.79] \)


\item \( x_1 \in [-2.15, -1.77] \text{ and } x_2 \in [1.17,1.24] \)


\item \( x_1 \in [-2.15, -1.77] \text{ and } x_2 \in [1.04,1.13] \)

* $x = -2.000 \text{ and } x = 1.125$, which is the correct option.
\item \( \text{All solutions lead to invalid or complex values in the equation.} \)


\item \( x \in [0.33,0.72] \)


\end{enumerate}

\textbf{General Comment:} Distractors are different based on the number of solutions. Remember that after solving, we need to make sure our solution does not make the original equation divide by zero!
}
\litem{
Choose the graph of the equation below.
\[ f(x) = \frac{1}{x + 2} - 3 \]The solution is the graph below, which is option B.
    \begin{center}
        \includegraphics[width=0.3\textwidth]{../Figures/rationalEquationToGraphCopyBC.png}
    \end{center}\begin{enumerate}[label=\Alph*.]
\begin{multicols}{2}
\item \includegraphics[width = 0.3\textwidth]{../Figures/rationalEquationToGraphCopyAC.png}
\item \includegraphics[width = 0.3\textwidth]{../Figures/rationalEquationToGraphCopyBC.png}
\item \includegraphics[width = 0.3\textwidth]{../Figures/rationalEquationToGraphCopyCC.png}
\item \includegraphics[width = 0.3\textwidth]{../Figures/rationalEquationToGraphCopyDC.png}
\end{multicols}\item None of the above.\end{enumerate}
\textbf{General Comment:} Remember that the general form of a basic rational equation is $ f(x) = \frac{a}{(x-h)^n} + k$, where $a$ is the leading coefficient (and in this case, we assume is either $1$ or $-1$), $n$ is the degree (in this case, either $1$ or $2$), and $(h, k)$ is the intersection of the asymptotes.
}
\litem{
Determine the domain of the function below.
\[ f(x) = \frac{4}{30x^{2} -49 x + 20} \]The solution is \( \text{All Real numbers except } x = 0.800 \text{ and } x = 0.833. \), which is option B.\begin{enumerate}[label=\Alph*.]
\item \( \text{All Real numbers except } x = a, \text{ where } a \in [19.95, 20] \)

All Real numbers except $x = 20.000$, which corresponds to removing a distractor value from the denominator.
\item \( \text{All Real numbers except } x = a \text{ and } x = b, \text{ where } a \in [0.78, 0.83] \text{ and } b \in [0.82, 0.84] \)

All Real numbers except $x = 0.800$ and $x = 0.833$, which is the correct option.
\item \( \text{All Real numbers.} \)

This corresponds to thinking the denominator has complex roots or that rational functions have a domain of all Real numbers.
\item \( \text{All Real numbers except } x = a \text{ and } x = b, \text{ where } a \in [19.95, 20] \text{ and } b \in [29.98, 30.02] \)

All Real numbers except $x = 20.000$ and $x = 30.000$, which corresponds to not factoring the denominator correctly.
\item \( \text{All Real numbers except } x = a, \text{ where } a \in [0.78, 0.83] \)

All Real numbers except $x = 0.800$, which corresponds to removing only 1 value from the denominator.
\end{enumerate}

\textbf{General Comment:} Recall that dividing by zero is not a real number. Therefore the domain is all real numbers \textbf{except} those that make the denominator 0.
}
\end{enumerate}

\end{document}