\documentclass{extbook}[14pt]
\usepackage{multicol, enumerate, enumitem, hyperref, color, soul, setspace, parskip, fancyhdr, amssymb, amsthm, amsmath, bbm, latexsym, units, mathtools}
\everymath{\displaystyle}
\usepackage[headsep=0.5cm,headheight=0cm, left=1 in,right= 1 in,top= 1 in,bottom= 1 in]{geometry}
\usepackage{dashrule}  % Package to use the command below to create lines between items
\newcommand{\litem}[1]{\item #1

\rule{\textwidth}{0.4pt}}
\pagestyle{fancy}
\lhead{}
\chead{Answer Key for Progress Quiz 4 Version C}
\rhead{}
\lfoot{9187-5854}
\cfoot{}
\rfoot{Spring 2021}
\begin{document}
\textbf{This key should allow you to understand why you choose the option you did (beyond just getting a question right or wrong). \href{https://xronos.clas.ufl.edu/mac1105spring2020/courseDescriptionAndMisc/Exams/LearningFromResults}{More instructions on how to use this key can be found here}.}

\textbf{If you have a suggestion to make the keys better, \href{https://forms.gle/CZkbZmPbC9XALEE88}{please fill out the short survey here}.}

\textit{Note: This key is auto-generated and may contain issues and/or errors. The keys are reviewed after each exam to ensure grading is done accurately. If there are issues (like duplicate options), they are noted in the offline gradebook. The keys are a work-in-progress to give students as many resources to improve as possible.}

\rule{\textwidth}{0.4pt}

\begin{enumerate}\litem{
Determine the domain of the function below.
\[ f(x) = \frac{5}{18x^{2} +33 x + 12} \]The solution is \( \text{All Real numbers except } x = -1.333 \text{ and } x = -0.500. \), which is option C.\begin{enumerate}[label=\Alph*.]
\item \( \text{All Real numbers.} \)

This corresponds to thinking the denominator has complex roots or that rational functions have a domain of all Real numbers.
\item \( \text{All Real numbers except } x = a \text{ and } x = b, \text{ where } a \in [-18.5, -17.4] \text{ and } b \in [-13.5, -11.5] \)

All Real numbers except $x = -18.000$ and $x = -12.000$, which corresponds to not factoring the denominator correctly.
\item \( \text{All Real numbers except } x = a \text{ and } x = b, \text{ where } a \in [-2.2, -0.6] \text{ and } b \in [-0.7, 0] \)

All Real numbers except $x = -1.333$ and $x = -0.500$, which is the correct option.
\item \( \text{All Real numbers except } x = a, \text{ where } a \in [-18.5, -17.4] \)

All Real numbers except $x = -18.000$, which corresponds to removing a distractor value from the denominator.
\item \( \text{All Real numbers except } x = a, \text{ where } a \in [-2.2, -0.6] \)

All Real numbers except $x = -1.333$, which corresponds to removing only 1 value from the denominator.
\end{enumerate}

\textbf{General Comment:} Recall that dividing by zero is not a real number. Therefore the domain is all real numbers \textbf{except} those that make the denominator 0.
}
\litem{
Choose the equation of the function graphed below.

\begin{center}
    \includegraphics[width=0.5\textwidth]{../Figures/rationalGraphToEquationCopyC.png}
\end{center}


The solution is \( \text{None of the above as it should be } f(x) = \frac{1}{x - 3} - 3 \), which is option E.\begin{enumerate}[label=\Alph*.]
\item \( f(x) = \frac{1}{(x - 3)^2} - 5 \)

Corresponds to thinking the graph was a shifted version of $\frac{1}{x^2}$ not noticing the $y$-value was wrong.
\item \( f(x) = \frac{-1}{(x + 3)^2} - 5 \)

Corresponds to thinking the graph was a shifted version of $\frac{1}{x^2}$, using the general form $f(x) = \frac{a}{x+h}+k$, the opposite leading coefficient, AND not noticing the $y$-value was wrong.
\item \( f(x) = \frac{-1}{x + 3} - 5 \)

Corresponds to using the general form $f(x) = \frac{a}{x+h}+k$, the opposite leading coefficient AND not noticing the $y$-value was wrong.
\item \( f(x) = \frac{1}{x - 3} - 5 \)

The $y$-value of the equation does not match the graph.
\item \( \text{None of the above} \)

None of the equation options were the correct equation.
\end{enumerate}

\textbf{General Comment:} Remember that the general form of a basic rational equation is $ f(x) = \frac{a}{(x-h)^n} + k$, where $a$ is the leading coefficient (and in this case, we assume is either $1$ or $-1$), $n$ is the degree (in this case, either $1$ or $2$), and $(h, k)$ is the intersection of the asymptotes.
}
\litem{
Solve the rational equation below. Then, choose the interval(s) that the solution(s) belongs to.
\[ \frac{-5}{6x + 4} + 9 = \frac{-7}{-36x -24} \]The solution is \( x = -0.552 \), which is option A.\begin{enumerate}[label=\Alph*.]
\item \( x \in [-1.55,0.45] \)

* $x = -0.552$, which is the correct option.
\item \( x_1 \in [-0.95, -0.7] \text{ and } x_2 \in [-1.3,-0.3] \)

$x = -0.704 \text{ and } x = -0.552$, which corresponds to getting the correct solution and believing there should be a second solution to the equation.
\item \( x_1 \in [-0.56, -0.46] \text{ and } x_2 \in [0.3,1] \)

$x = -0.552 \text{ and } x = 0.781$, which corresponds to getting the correct solution and believing there should be a second solution to the equation.
\item \( x \in [0.72,0.9] \)

$x = 0.781$, which corresponds to not distributing the factor $6x + 4$ correctly when trying to eliminate the fraction.
\item \( \text{All solutions lead to invalid or complex values in the equation.} \)

This corresponds to thinking $x = -0.552$ leads to dividing by zero in the original equation, which it does not.
\end{enumerate}

\textbf{General Comment:} Distractors are different based on the number of solutions. Remember that after solving, we need to make sure our solution does not make the original equation divide by zero!
}
\litem{
Determine the domain of the function below.
\[ f(x) = \frac{6}{30x^{2} -50 x + 20} \]The solution is \( \text{All Real numbers except } x = 0.667 \text{ and } x = 1.000. \), which is option B.\begin{enumerate}[label=\Alph*.]
\item \( \text{All Real numbers except } x = a, \text{ where } a \in [23.84, 24.07] \)

All Real numbers except $x = 24.000$, which corresponds to removing a distractor value from the denominator.
\item \( \text{All Real numbers except } x = a \text{ and } x = b, \text{ where } a \in [0.43, 0.98] \text{ and } b \in [0.99, 1.54] \)

All Real numbers except $x = 0.667$ and $x = 1.000$, which is the correct option.
\item \( \text{All Real numbers except } x = a, \text{ where } a \in [0.43, 0.98] \)

All Real numbers except $x = 0.667$, which corresponds to removing only 1 value from the denominator.
\item \( \text{All Real numbers except } x = a \text{ and } x = b, \text{ where } a \in [23.84, 24.07] \text{ and } b \in [24.95, 25.17] \)

All Real numbers except $x = 24.000$ and $x = 25.000$, which corresponds to not factoring the denominator correctly.
\item \( \text{All Real numbers.} \)

This corresponds to thinking the denominator has complex roots or that rational functions have a domain of all Real numbers.
\end{enumerate}

\textbf{General Comment:} Recall that dividing by zero is not a real number. Therefore the domain is all real numbers \textbf{except} those that make the denominator 0.
}
\litem{
Choose the graph of the equation below.
\[ f(x) = \frac{1}{x + 3} + 1 \]The solution is the graph below, which is option E.
\begin{center}
    \includegraphics[width=0.3\textwidth]{../Figures/rationalEquationToGraphCopyEC.png}
\end{center}\begin{enumerate}[label=\Alph*.]
\begin{multicols}{2}
\item \includegraphics[width = 0.3\textwidth]{../Figures/rationalEquationToGraphCopyAC.png}
\item \includegraphics[width = 0.3\textwidth]{../Figures/rationalEquationToGraphCopyBC.png}
\item \includegraphics[width = 0.3\textwidth]{../Figures/rationalEquationToGraphCopyCC.png}
\item \includegraphics[width = 0.3\textwidth]{../Figures/rationalEquationToGraphCopyDC.png}
\end{multicols}\item None of the above.\end{enumerate}
\textbf{General Comment:} Remember that the general form of a basic rational equation is $ f(x) = \frac{a}{(x-h)^n} + k$, where $a$ is the leading coefficient (and in this case, we assume is either $1$ or $-1$), $n$ is the degree (in this case, either $1$ or $2$), and $(h, k)$ is the intersection of the asymptotes.
}
\litem{
Solve the rational equation below. Then, choose the interval(s) that the solution(s) belongs to.
\[ \frac{2}{2x + 2} + -3 = \frac{-7}{-10x -10} \]The solution is \( x = -0.900 \), which is option D.\begin{enumerate}[label=\Alph*.]
\item \( x_1 \in [-1.9, 0.1] \text{ and } x_2 \in [0.3,1] \)

$x = -0.900 \text{ and } x = 0.500$, which corresponds to getting the correct solution and believing there should be a second solution to the equation.
\item \( \text{All solutions lead to invalid or complex values in the equation.} \)

This corresponds to thinking $x = -0.900$ leads to dividing by zero in the original equation, which it does not.
\item \( x \in [0.1,3.1] \)

$x = 1.100$, which corresponds to not distributing the factor $2x + 2$ correctly when trying to eliminate the fraction.
\item \( x \in [-0.9,1.1] \)

* $x = -0.900$, which is the correct option.
\item \( x_1 \in [-1.9, 0.1] \text{ and } x_2 \in [0.9,1.8] \)

$x = -0.900 \text{ and } x = 1.100$, which corresponds to getting the correct solution and believing there should be a second solution to the equation.
\end{enumerate}

\textbf{General Comment:} Distractors are different based on the number of solutions. Remember that after solving, we need to make sure our solution does not make the original equation divide by zero!
}
\litem{
Solve the rational equation below. Then, choose the interval(s) that the solution(s) belongs to.
\[ \frac{-3x}{-7x + 6} + \frac{-7x^{2}}{42x^{2} -50 x + 12} = \frac{4}{-6x + 2} \]The solution is \( \text{There are two solutions: } x = 0.784 \text{ and } x = -2.784 \), which is option A.\begin{enumerate}[label=\Alph*.]
\item \( x_1 \in [0.65, 0.83] \text{ and } x_2 \in [-5.4,-0.2] \)

* $x = 0.784 \text{ and } x = -2.784$, which is the correct option.
\item \( \text{All solutions lead to invalid or complex values in the equation.} \)


\item \( x \in [0.28,0.73] \)


\item \( x \in [-2.87,-2.56] \)


\item \( x_1 \in [0.65, 0.83] \text{ and } x_2 \in [-0.8,4.4] \)


\end{enumerate}

\textbf{General Comment:} Distractors are different based on the number of solutions. Remember that after solving, we need to make sure our solution does not make the original equation divide by zero!
}
\litem{
Choose the equation of the function graphed below.

\begin{center}
    \includegraphics[width=0.5\textwidth]{../Figures/rationalGraphToEquationC.png}
\end{center}


The solution is \( f(x) = \frac{1}{x - 2} - 2 \), which is option C.\begin{enumerate}[label=\Alph*.]
\item \( f(x) = \frac{-1}{(x + 2)^2} - 2 \)

Corresponds to thinking the graph was a shifted version of $\frac{1}{x^2}$, using the general form $f(x) = \frac{a}{x+h}+k$, and the opposite leading coefficient.
\item \( f(x) = \frac{-1}{x + 2} - 2 \)

Corresponds to using the general form $f(x) = \frac{a}{x+h}+k$ and the opposite leading coefficient.
\item \( f(x) = \frac{1}{x - 2} - 2 \)

This is the correct option.
\item \( f(x) = \frac{1}{(x - 2)^2} - 2 \)

Corresponds to thinking the graph was a shifted version of $\frac{1}{x^2}$.
\item \( \text{None of the above} \)

This corresponds to believing the vertex of the graph was not correct.
\end{enumerate}

\textbf{General Comment:} Remember that the general form of a basic rational equation is $ f(x) = \frac{a}{(x-h)^n} + k$, where $a$ is the leading coefficient (and in this case, we assume is either $1$ or $-1$), $n$ is the degree (in this case, either $1$ or $2$), and $(h, k)$ is the intersection of the asymptotes.
}
\litem{
Choose the graph of the equation below.
\[ f(x) = \frac{1}{(x - 1)^2} - 3 \]The solution is the graph below, which is option B.
\begin{center}
    \includegraphics[width=0.3\textwidth]{../Figures/rationalEquationToGraphBC.png}
\end{center}\begin{enumerate}[label=\Alph*.]
\begin{multicols}{2}
\item \includegraphics[width = 0.3\textwidth]{../Figures/rationalEquationToGraphAC.png}
\item \includegraphics[width = 0.3\textwidth]{../Figures/rationalEquationToGraphBC.png}
\item \includegraphics[width = 0.3\textwidth]{../Figures/rationalEquationToGraphCC.png}
\item \includegraphics[width = 0.3\textwidth]{../Figures/rationalEquationToGraphDC.png}
\end{multicols}\item None of the above.\end{enumerate}
\textbf{General Comment:} Remember that the general form of a basic rational equation is $ f(x) = \frac{a}{(x-h)^n} + k$, where $a$ is the leading coefficient (and in this case, we assume is either $1$ or $-1$), $n$ is the degree (in this case, either $1$ or $2$), and $(h, k)$ is the intersection of the asymptotes.
}
\litem{
Solve the rational equation below. Then, choose the interval(s) that the solution(s) belongs to.
\[ \frac{-2x}{6x + 6} + \frac{-3x^{2}}{-42x^{2} -60 x -18} = \frac{7}{-7x -3} \]The solution is \( \text{There are two solutions: } x = -0.912 \text{ and } x = 4.185 \), which is option A.\begin{enumerate}[label=\Alph*.]
\item \( x_1 \in [-1.06, -0.52] \text{ and } x_2 \in [3.18,10.18] \)

* $x = -0.912 \text{ and } x = 4.185$, which is the correct option.
\item \( x \in [4.09,4.55] \)


\item \( \text{All solutions lead to invalid or complex values in the equation.} \)


\item \( x_1 \in [-1.06, -0.52] \text{ and } x_2 \in [-4,4] \)


\item \( x \in [-0.59,-0.39] \)


\end{enumerate}

\textbf{General Comment:} Distractors are different based on the number of solutions. Remember that after solving, we need to make sure our solution does not make the original equation divide by zero!
}
\end{enumerate}

\end{document}