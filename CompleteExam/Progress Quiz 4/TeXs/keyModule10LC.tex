\documentclass{extbook}[14pt]
\usepackage{multicol, enumerate, enumitem, hyperref, color, soul, setspace, parskip, fancyhdr, amssymb, amsthm, amsmath, latexsym, units, mathtools}
\everymath{\displaystyle}
\usepackage[headsep=0.5cm,headheight=0cm, left=1 in,right= 1 in,top= 1 in,bottom= 1 in]{geometry}
\usepackage{dashrule}  % Package to use the command below to create lines between items
\newcommand{\litem}[1]{\item #1

\rule{\textwidth}{0.4pt}}
\pagestyle{fancy}
\lhead{}
\chead{Answer Key for Progress Quiz 4 Version C}
\rhead{}
\lfoot{5346-5907}
\cfoot{}
\rfoot{Summer C 2021}
\begin{document}
\textbf{This key should allow you to understand why you choose the option you did (beyond just getting a question right or wrong). \href{https://xronos.clas.ufl.edu/mac1105spring2020/courseDescriptionAndMisc/Exams/LearningFromResults}{More instructions on how to use this key can be found here}.}

\textbf{If you have a suggestion to make the keys better, \href{https://forms.gle/CZkbZmPbC9XALEE88}{please fill out the short survey here}.}

\textit{Note: This key is auto-generated and may contain issues and/or errors. The keys are reviewed after each exam to ensure grading is done accurately. If there are issues (like duplicate options), they are noted in the offline gradebook. The keys are a work-in-progress to give students as many resources to improve as possible.}

\rule{\textwidth}{0.4pt}

\begin{enumerate}\litem{
Factor the polynomial below completely, knowing that $x + 5$ is a factor. Then, choose the intervals the zeros of the polynomial belong to, where $z_1 \leq z_2 \leq z_3 \leq z_4$. \textit{To make the problem easier, all zeros are between -5 and 5.}
\[ f(x) = 15x^{4} +139 x^{3} +383 x^{2} +333 x + 90 \]The solution is \( [-5, -3, -0.667, -0.6] \), which is option A.\begin{enumerate}[label=\Alph*.]
\item \( z_1 \in [-5.01, -4.54], \text{   }  z_2 \in [-3.73, -2.62], z_3 \in [-1.6, 1.2], \text{   and   } z_4 \in [-1.19, 0.12] \)

* This is the solution!
\item \( z_1 \in [0.37, 0.91], \text{   }  z_2 \in [0.35, 0.89], z_3 \in [2.4, 3.6], \text{   and   } z_4 \in [3.93, 5.1] \)

 Distractor 1: Corresponds to negatives of all zeros.
\item \( z_1 \in [-0.16, 0.36], \text{   }  z_2 \in [1.98, 3.21], z_3 \in [2.4, 3.6], \text{   and   } z_4 \in [3.93, 5.1] \)

 Distractor 4: Corresponds to moving factors from one rational to another.
\item \( z_1 \in [-5.01, -4.54], \text{   }  z_2 \in [-3.73, -2.62], z_3 \in [-4, -0.9], \text{   and   } z_4 \in [-2.54, -1.2] \)

 Distractor 2: Corresponds to inversing rational roots.
\item \( z_1 \in [1.45, 1.87], \text{   }  z_2 \in [1.62, 1.94], z_3 \in [2.4, 3.6], \text{   and   } z_4 \in [3.93, 5.1] \)

 Distractor 3: Corresponds to negatives of all zeros AND inversing rational roots.
\end{enumerate}

\textbf{General Comment:} Remember to try the middle-most integers first as these normally are the zeros. Also, once you get it to a quadratic, you can use your other factoring techniques to finish factoring.
}
\litem{
Perform the division below. Then, find the intervals that correspond to the quotient in the form $ax^2+bx+c$ and remainder $r$.
\[ \frac{6x^{3} -1 x^{2} -20 x + 14}{x + 2} \]The solution is \( 6x^{2} -13 x + 6 + \frac{2}{x + 2} \), which is option B.\begin{enumerate}[label=\Alph*.]
\item \( a \in [6, 9], \text{   } b \in [-20.2, -15.3], \text{   } c \in [35, 40], \text{   and   } r \in [-99, -93]. \)

 You multiplied by the synthetic number and subtracted rather than adding during synthetic division.
\item \( a \in [6, 9], \text{   } b \in [-13.5, -8.7], \text{   } c \in [5, 13], \text{   and   } r \in [-3, 4]. \)

* This is the solution!
\item \( a \in [-17, -11], \text{   } b \in [-30.2, -22.4], \text{   } c \in [-73, -68], \text{   and   } r \in [-128, -122]. \)

 You divided by the opposite of the factor AND multiplied the first factor rather than just bringing it down.
\item \( a \in [6, 9], \text{   } b \in [9.7, 12.2], \text{   } c \in [0, 3], \text{   and   } r \in [12, 22]. \)

 You divided by the opposite of the factor.
\item \( a \in [-17, -11], \text{   } b \in [22.7, 24.5], \text{   } c \in [-69, -65], \text{   and   } r \in [142, 152]. \)

 You multiplied by the synthetic number rather than bringing the first factor down.
\end{enumerate}

\textbf{General Comment:} Be sure to synthetically divide by the zero of the denominator!
}
\litem{
Factor the polynomial below completely. Then, choose the intervals the zeros of the polynomial belong to, where $z_1 \leq z_2 \leq z_3$. \textit{To make the problem easier, all zeros are between -5 and 5.}
\[ f(x) = 8x^{3} +38 x^{2} +15 x -36 \]The solution is \( [-4, -1.5, 0.75] \), which is option A.\begin{enumerate}[label=\Alph*.]
\item \( z_1 \in [-4.08, -3.9], \text{   }  z_2 \in [-1.8, -1.23], \text{   and   } z_3 \in [0.1, 0.9] \)

* This is the solution!
\item \( z_1 \in [-1.37, -1.16], \text{   }  z_2 \in [0.22, 0.87], \text{   and   } z_3 \in [2.3, 4.9] \)

 Distractor 3: Corresponds to negatives of all zeros AND inversing rational roots.
\item \( z_1 \in [-0.64, -0.36], \text{   }  z_2 \in [2.91, 3.23], \text{   and   } z_3 \in [2.3, 4.9] \)

 Distractor 4: Corresponds to moving factors from one rational to another.
\item \( z_1 \in [-0.86, -0.55], \text{   }  z_2 \in [1.36, 1.85], \text{   and   } z_3 \in [2.3, 4.9] \)

 Distractor 1: Corresponds to negatives of all zeros.
\item \( z_1 \in [-4.08, -3.9], \text{   }  z_2 \in [-1.2, -0.15], \text{   and   } z_3 \in [1, 2.4] \)

 Distractor 2: Corresponds to inversing rational roots.
\end{enumerate}

\textbf{General Comment:} Remember to try the middle-most integers first as these normally are the zeros. Also, once you get it to a quadratic, you can use your other factoring techniques to finish factoring.
}
\litem{
What are the \textit{possible Integer} roots of the polynomial below?
\[ f(x) = 4x^{3} +7 x^{2} +5 x + 5 \]The solution is \( \pm 1,\pm 5 \), which is option B.\begin{enumerate}[label=\Alph*.]
\item \( \text{ All combinations of: }\frac{\pm 1,\pm 2,\pm 4}{\pm 1,\pm 5} \)

 Distractor 3: Corresponds to the plus or minus of the inverse quotient (an/a0) of the factors. 
\item \( \pm 1,\pm 5 \)

* This is the solution \textbf{since we asked for the possible Integer roots}!
\item \( \pm 1,\pm 2,\pm 4 \)

 Distractor 1: Corresponds to the plus or minus factors of a1 only.
\item \( \text{ All combinations of: }\frac{\pm 1,\pm 5}{\pm 1,\pm 2,\pm 4} \)

This would have been the solution \textbf{if asked for the possible Rational roots}!
\item \( \text{There is no formula or theorem that tells us all possible Integer roots.} \)

 Distractor 4: Corresponds to not recognizing Integers as a subset of Rationals.
\end{enumerate}

\textbf{General Comment:} We have a way to find the possible Rational roots. The possible Integer roots are the Integers in this list.
}
\litem{
Factor the polynomial below completely. Then, choose the intervals the zeros of the polynomial belong to, where $z_1 \leq z_2 \leq z_3$. \textit{To make the problem easier, all zeros are between -5 and 5.}
\[ f(x) = 8x^{3} -22 x^{2} -65 x + 100 \]The solution is \( [-2.5, 1.25, 4] \), which is option C.\begin{enumerate}[label=\Alph*.]
\item \( z_1 \in [-4, -3], \text{   }  z_2 \in [-0.96, -0.67], \text{   and   } z_3 \in [0.1, 2.1] \)

 Distractor 3: Corresponds to negatives of all zeros AND inversing rational roots.
\item \( z_1 \in [-4, -3], \text{   }  z_2 \in [-0.74, -0.29], \text{   and   } z_3 \in [4.9, 5.1] \)

 Distractor 4: Corresponds to moving factors from one rational to another.
\item \( z_1 \in [-3.5, -1.5], \text{   }  z_2 \in [1.21, 1.28], \text{   and   } z_3 \in [3.8, 4.2] \)

* This is the solution!
\item \( z_1 \in [-4, -3], \text{   }  z_2 \in [-1.31, -1.08], \text{   and   } z_3 \in [2.1, 3] \)

 Distractor 1: Corresponds to negatives of all zeros.
\item \( z_1 \in [-2.4, 2.6], \text{   }  z_2 \in [0.79, 0.87], \text{   and   } z_3 \in [3.8, 4.2] \)

 Distractor 2: Corresponds to inversing rational roots.
\end{enumerate}

\textbf{General Comment:} Remember to try the middle-most integers first as these normally are the zeros. Also, once you get it to a quadratic, you can use your other factoring techniques to finish factoring.
}
\litem{
Perform the division below. Then, find the intervals that correspond to the quotient in the form $ax^2+bx+c$ and remainder $r$.
\[ \frac{12x^{3} -36 x + 29}{x + 2} \]The solution is \( 12x^{2} -24 x + 12 + \frac{5}{x + 2} \), which is option A.\begin{enumerate}[label=\Alph*.]
\item \( a \in [12, 15], b \in [-26, -18], c \in [10, 14], \text{ and } r \in [5, 7]. \)

* This is the solution!
\item \( a \in [-25, -16], b \in [-48, -47], c \in [-135, -129], \text{ and } r \in [-240, -232]. \)

 You divided by the opposite of the factor AND multipled the first factor rather than just bringing it down.
\item \( a \in [-25, -16], b \in [40, 54], c \in [-135, -129], \text{ and } r \in [293, 294]. \)

 You multipled by the synthetic number rather than bringing the first factor down.
\item \( a \in [12, 15], b \in [-42, -32], c \in [67, 77], \text{ and } r \in [-188, -182]. \)

 You multipled by the synthetic number and subtracted rather than adding during synthetic division.
\item \( a \in [12, 15], b \in [21, 29], c \in [10, 14], \text{ and } r \in [47, 55]. \)

 You divided by the opposite of the factor.
\end{enumerate}

\textbf{General Comment:} Be sure to synthetically divide by the zero of the denominator! Also, make sure to include 0 placeholders for missing terms.
}
\litem{
Factor the polynomial below completely, knowing that $x -5$ is a factor. Then, choose the intervals the zeros of the polynomial belong to, where $z_1 \leq z_2 \leq z_3 \leq z_4$. \textit{To make the problem easier, all zeros are between -5 and 5.}
\[ f(x) = 6x^{4} -19 x^{3} -81 x^{2} +90 x + 200 \]The solution is \( [-2.5, -1.333, 2, 5] \), which is option A.\begin{enumerate}[label=\Alph*.]
\item \( z_1 \in [-3.1, -1.7], \text{   }  z_2 \in [-1.43, -1.12], z_3 \in [1.75, 2.23], \text{   and   } z_4 \in [4.3, 6.3] \)

* This is the solution!
\item \( z_1 \in [-1.3, -0.4], \text{   }  z_2 \in [-0.6, 0.2], z_3 \in [1.75, 2.23], \text{   and   } z_4 \in [4.3, 6.3] \)

 Distractor 2: Corresponds to inversing rational roots.
\item \( z_1 \in [-6.8, -4.8], \text{   }  z_2 \in [-2.63, -1.81], z_3 \in [0.33, 0.66], \text{   and   } z_4 \in [0.2, 2] \)

 Distractor 3: Corresponds to negatives of all zeros AND inversing rational roots.
\item \( z_1 \in [-6.8, -4.8], \text{   }  z_2 \in [-2.63, -1.81], z_3 \in [0.58, 1.18], \text{   and   } z_4 \in [4.3, 6.3] \)

 Distractor 4: Corresponds to moving factors from one rational to another.
\item \( z_1 \in [-6.8, -4.8], \text{   }  z_2 \in [-2.63, -1.81], z_3 \in [1.07, 1.78], \text{   and   } z_4 \in [1.4, 3.4] \)

 Distractor 1: Corresponds to negatives of all zeros.
\end{enumerate}

\textbf{General Comment:} Remember to try the middle-most integers first as these normally are the zeros. Also, once you get it to a quadratic, you can use your other factoring techniques to finish factoring.
}
\litem{
What are the \textit{possible Rational} roots of the polynomial below?
\[ f(x) = 2x^{2} +4 x + 4 \]The solution is \( \text{ All combinations of: }\frac{\pm 1,\pm 2,\pm 4}{\pm 1,\pm 2} \), which is option A.\begin{enumerate}[label=\Alph*.]
\item \( \text{ All combinations of: }\frac{\pm 1,\pm 2,\pm 4}{\pm 1,\pm 2} \)

* This is the solution \textbf{since we asked for the possible Rational roots}!
\item \( \text{ All combinations of: }\frac{\pm 1,\pm 2}{\pm 1,\pm 2,\pm 4} \)

 Distractor 3: Corresponds to the plus or minus of the inverse quotient (an/a0) of the factors. 
\item \( \pm 1,\pm 2 \)

 Distractor 1: Corresponds to the plus or minus factors of a1 only.
\item \( \pm 1,\pm 2,\pm 4 \)

This would have been the solution \textbf{if asked for the possible Integer roots}!
\item \( \text{ There is no formula or theorem that tells us all possible Rational roots.} \)

 Distractor 4: Corresponds to not recalling the theorem for rational roots of a polynomial.
\end{enumerate}

\textbf{General Comment:} We have a way to find the possible Rational roots. The possible Integer roots are the Integers in this list.
}
\litem{
Perform the division below. Then, find the intervals that correspond to the quotient in the form $ax^2+bx+c$ and remainder $r$.
\[ \frac{12x^{3} +17 x^{2} -24 x -18}{x + 2} \]The solution is \( 12x^{2} -7 x -10 + \frac{2}{x + 2} \), which is option A.\begin{enumerate}[label=\Alph*.]
\item \( a \in [8, 16], \text{   } b \in [-8, -5], \text{   } c \in [-11, -5], \text{   and   } r \in [-2, 10]. \)

* This is the solution!
\item \( a \in [8, 16], \text{   } b \in [38, 42], \text{   } c \in [56, 61], \text{   and   } r \in [96, 102]. \)

 You divided by the opposite of the factor.
\item \( a \in [-26, -19], \text{   } b \in [61, 70], \text{   } c \in [-160, -152], \text{   and   } r \in [288, 293]. \)

 You multiplied by the synthetic number rather than bringing the first factor down.
\item \( a \in [8, 16], \text{   } b \in [-22, -17], \text{   } c \in [29, 34], \text{   and   } r \in [-121, -116]. \)

 You multiplied by the synthetic number and subtracted rather than adding during synthetic division.
\item \( a \in [-26, -19], \text{   } b \in [-35, -28], \text{   } c \in [-87, -83], \text{   and   } r \in [-192, -189]. \)

 You divided by the opposite of the factor AND multiplied the first factor rather than just bringing it down.
\end{enumerate}

\textbf{General Comment:} Be sure to synthetically divide by the zero of the denominator!
}
\litem{
Perform the division below. Then, find the intervals that correspond to the quotient in the form $ax^2+bx+c$ and remainder $r$.
\[ \frac{6x^{3} +35 x^{2} -127}{x + 5} \]The solution is \( 6x^{2} +5 x -25 + \frac{-2}{x + 5} \), which is option A.\begin{enumerate}[label=\Alph*.]
\item \( a \in [6, 10], b \in [4, 6], c \in [-32, -21], \text{ and } r \in [-2, -1]. \)

* This is the solution!
\item \( a \in [6, 10], b \in [65, 67], c \in [316, 329], \text{ and } r \in [1496, 1502]. \)

 You divided by the opposite of the factor.
\item \( a \in [6, 10], b \in [-5, 1], c \in [5, 9], \text{ and } r \in [-163, -160]. \)

 You multipled by the synthetic number and subtracted rather than adding during synthetic division.
\item \( a \in [-30, -28], b \in [183, 190], c \in [-926, -922], \text{ and } r \in [4497, 4503]. \)

 You multipled by the synthetic number rather than bringing the first factor down.
\item \( a \in [-30, -28], b \in [-117, -107], c \in [-578, -574], \text{ and } r \in [-3002, -3001]. \)

 You divided by the opposite of the factor AND multipled the first factor rather than just bringing it down.
\end{enumerate}

\textbf{General Comment:} Be sure to synthetically divide by the zero of the denominator! Also, make sure to include 0 placeholders for missing terms.
}
\end{enumerate}

\end{document}