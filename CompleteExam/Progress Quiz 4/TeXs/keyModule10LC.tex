\documentclass{extbook}[14pt]
\usepackage{multicol, enumerate, enumitem, hyperref, color, soul, setspace, parskip, fancyhdr, amssymb, amsthm, amsmath, bbm, latexsym, units, mathtools}
\everymath{\displaystyle}
\usepackage[headsep=0.5cm,headheight=0cm, left=1 in,right= 1 in,top= 1 in,bottom= 1 in]{geometry}
\usepackage{dashrule}  % Package to use the command below to create lines between items
\newcommand{\litem}[1]{\item #1

\rule{\textwidth}{0.4pt}}
\pagestyle{fancy}
\lhead{}
\chead{Answer Key for Progress Quiz 4 Version C}
\rhead{}
\lfoot{9187-5854}
\cfoot{}
\rfoot{Spring 2021}
\begin{document}
\textbf{This key should allow you to understand why you choose the option you did (beyond just getting a question right or wrong). \href{https://xronos.clas.ufl.edu/mac1105spring2020/courseDescriptionAndMisc/Exams/LearningFromResults}{More instructions on how to use this key can be found here}.}

\textbf{If you have a suggestion to make the keys better, \href{https://forms.gle/CZkbZmPbC9XALEE88}{please fill out the short survey here}.}

\textit{Note: This key is auto-generated and may contain issues and/or errors. The keys are reviewed after each exam to ensure grading is done accurately. If there are issues (like duplicate options), they are noted in the offline gradebook. The keys are a work-in-progress to give students as many resources to improve as possible.}

\rule{\textwidth}{0.4pt}

\begin{enumerate}\litem{
Factor the polynomial below completely. Then, choose the intervals the zeros of the polynomial belong to, where $z_1 \leq z_2 \leq z_3$. \textit{To make the problem easier, all zeros are between -5 and 5.}
\[ f(x) = 6x^{3} + x^{2} -20 x -12 \]The solution is \( [-1.5, -0.6666666666666666, 2] \), which is option D.\begin{enumerate}[label=\Alph*.]
\item \( z_1 \in [-2.2, -1.8], \text{   }  z_2 \in [0.07, 0.47], \text{   and   } z_3 \in [2.8, 3.8] \)

 Distractor 4: Corresponds to moving factors from one rational to another.
\item \( z_1 \in [-1.6, -0.4], \text{   }  z_2 \in [-1.31, -0.3], \text{   and   } z_3 \in [1.6, 2.3] \)

 Distractor 2: Corresponds to inversing rational roots.
\item \( z_1 \in [-2.2, -1.8], \text{   }  z_2 \in [0.49, 0.9], \text{   and   } z_3 \in [1, 1.9] \)

 Distractor 1: Corresponds to negatives of all zeros.
\item \( z_1 \in [-1.6, -0.4], \text{   }  z_2 \in [-1.31, -0.3], \text{   and   } z_3 \in [1.6, 2.3] \)

* This is the solution!
\item \( z_1 \in [-2.2, -1.8], \text{   }  z_2 \in [0.49, 0.9], \text{   and   } z_3 \in [1, 1.9] \)

 Distractor 3: Corresponds to negatives of all zeros AND inversing rational roots.
\end{enumerate}

\textbf{General Comment:} Remember to try the middle-most integers first as these normally are the zeros. Also, once you get it to a quadratic, you can use your other factoring techniques to finish factoring.
}
\litem{
Perform the division below. Then, find the intervals that correspond to the quotient in the form $ax^2+bx+c$ and remainder $r$.
\[ \frac{9x^{3} +27 x^{2} +8 x -24}{x + 2} \]The solution is \( 9x^{2} +9 x -10 + \frac{-4}{x + 2} \), which is option D.\begin{enumerate}[label=\Alph*.]
\item \( a \in [8, 12], \text{   } b \in [-3, 5], \text{   } c \in [8, 11], \text{   and   } r \in [-49, -46]. \)

 You multiplied by the synthetic number and subtracted rather than adding during synthetic division.
\item \( a \in [-20, -11], \text{   } b \in [58, 70], \text{   } c \in [-119, -115], \text{   and   } r \in [211, 215]. \)

 You multiplied by the synthetic number rather than bringing the first factor down.
\item \( a \in [8, 12], \text{   } b \in [41, 50], \text{   } c \in [94, 101], \text{   and   } r \in [172, 179]. \)

 You divided by the opposite of the factor.
\item \( a \in [8, 12], \text{   } b \in [8, 12], \text{   } c \in [-10, -9], \text{   and   } r \in [-6, 0]. \)

* This is the solution!
\item \( a \in [-20, -11], \text{   } b \in [-14, -7], \text{   } c \in [-10, -9], \text{   and   } r \in [-46, -43]. \)

 You divided by the opposite of the factor AND multiplied the first factor rather than just bringing it down.
\end{enumerate}

\textbf{General Comment:} Be sure to synthetically divide by the zero of the denominator!
}
\litem{
What are the \textit{possible Rational} roots of the polynomial below?
\[ f(x) = 5x^{4} +4 x^{3} +3 x^{2} +2 x + 7 \]The solution is \( \text{ All combinations of: }\frac{\pm 1,\pm 7}{\pm 1,\pm 5} \), which is option B.\begin{enumerate}[label=\Alph*.]
\item \( \pm 1,\pm 7 \)

This would have been the solution \textbf{if asked for the possible Integer roots}!
\item \( \text{ All combinations of: }\frac{\pm 1,\pm 7}{\pm 1,\pm 5} \)

* This is the solution \textbf{since we asked for the possible Rational roots}!
\item \( \pm 1,\pm 5 \)

 Distractor 1: Corresponds to the plus or minus factors of a1 only.
\item \( \text{ All combinations of: }\frac{\pm 1,\pm 5}{\pm 1,\pm 7} \)

 Distractor 3: Corresponds to the plus or minus of the inverse quotient (an/a0) of the factors. 
\item \( \text{ There is no formula or theorem that tells us all possible Rational roots.} \)

 Distractor 4: Corresponds to not recalling the theorem for rational roots of a polynomial.
\end{enumerate}

\textbf{General Comment:} We have a way to find the possible Rational roots. The possible Integer roots are the Integers in this list.
}
\litem{
Factor the polynomial below completely. Then, choose the intervals the zeros of the polynomial belong to, where $z_1 \leq z_2 \leq z_3$. \textit{To make the problem easier, all zeros are between -5 and 5.}
\[ f(x) = 12x^{3} +53 x^{2} -45 x -50 \]The solution is \( [-5, -0.6666666666666666, 1.25] \), which is option E.\begin{enumerate}[label=\Alph*.]
\item \( z_1 \in [-1.32, -1.24], \text{   }  z_2 \in [0.55, 1.16], \text{   and   } z_3 \in [4.1, 6.1] \)

 Distractor 1: Corresponds to negatives of all zeros.
\item \( z_1 \in [-0.87, -0.32], \text{   }  z_2 \in [1.26, 1.75], \text{   and   } z_3 \in [4.1, 6.1] \)

 Distractor 3: Corresponds to negatives of all zeros AND inversing rational roots.
\item \( z_1 \in [-5.27, -4.64], \text{   }  z_2 \in [0.07, 0.66], \text{   and   } z_3 \in [4.1, 6.1] \)

 Distractor 4: Corresponds to moving factors from one rational to another.
\item \( z_1 \in [-5.27, -4.64], \text{   }  z_2 \in [-1.52, -1.4], \text{   and   } z_3 \in [0.4, 1.1] \)

 Distractor 2: Corresponds to inversing rational roots.
\item \( z_1 \in [-5.27, -4.64], \text{   }  z_2 \in [-1.1, -0.47], \text{   and   } z_3 \in [0.9, 1.7] \)

* This is the solution!
\end{enumerate}

\textbf{General Comment:} Remember to try the middle-most integers first as these normally are the zeros. Also, once you get it to a quadratic, you can use your other factoring techniques to finish factoring.
}
\litem{
Perform the division below. Then, find the intervals that correspond to the quotient in the form $ax^2+bx+c$ and remainder $r$.
\[ \frac{9x^{3} -27 x -22}{x -2} \]The solution is \( 9x^{2} +18 x + 9 + \frac{-4}{x -2} \), which is option E.\begin{enumerate}[label=\Alph*.]
\item \( a \in [6, 13], b \in [-18, -10], c \in [5, 11], \text{ and } r \in [-43, -34]. \)

 You divided by the opposite of the factor.
\item \( a \in [17, 22], b \in [32, 43], c \in [45, 46], \text{ and } r \in [64, 72]. \)

 You multipled by the synthetic number rather than bringing the first factor down.
\item \( a \in [6, 13], b \in [6, 12], c \in [-18, -17], \text{ and } r \in [-43, -34]. \)

 You multipled by the synthetic number and subtracted rather than adding during synthetic division.
\item \( a \in [17, 22], b \in [-42, -33], c \in [45, 46], \text{ and } r \in [-113, -108]. \)

 You divided by the opposite of the factor AND multipled the first factor rather than just bringing it down.
\item \( a \in [6, 13], b \in [15, 19], c \in [5, 11], \text{ and } r \in [-6, 1]. \)

* This is the solution!
\end{enumerate}

\textbf{General Comment:} Be sure to synthetically divide by the zero of the denominator! Also, make sure to include 0 placeholders for missing terms.
}
\litem{
Perform the division below. Then, find the intervals that correspond to the quotient in the form $ax^2+bx+c$ and remainder $r$.
\[ \frac{4x^{3} +34 x^{2} +80 x + 48}{x + 5} \]The solution is \( 4x^{2} +14 x + 10 + \frac{-2}{x + 5} \), which is option B.\begin{enumerate}[label=\Alph*.]
\item \( a \in [-23, -19], \text{   } b \in [131.5, 135.1], \text{   } c \in [-593, -583], \text{   and   } r \in [2997, 3001]. \)

 You multiplied by the synthetic number rather than bringing the first factor down.
\item \( a \in [1, 7], \text{   } b \in [11.2, 15.1], \text{   } c \in [4, 15], \text{   and   } r \in [-2, 1]. \)

* This is the solution!
\item \( a \in [1, 7], \text{   } b \in [51, 55.2], \text{   } c \in [346, 353], \text{   and   } r \in [1794, 1801]. \)

 You divided by the opposite of the factor.
\item \( a \in [-23, -19], \text{   } b \in [-66.4, -64.7], \text{   } c \in [-250, -246], \text{   and   } r \in [-1203, -1201]. \)

 You divided by the opposite of the factor AND multiplied the first factor rather than just bringing it down.
\item \( a \in [1, 7], \text{   } b \in [7.1, 11.1], \text{   } c \in [17, 23], \text{   and   } r \in [-77, -63]. \)

 You multiplied by the synthetic number and subtracted rather than adding during synthetic division.
\end{enumerate}

\textbf{General Comment:} Be sure to synthetically divide by the zero of the denominator!
}
\litem{
What are the \textit{possible Rational} roots of the polynomial below?
\[ f(x) = 2x^{4} +4 x^{3} +6 x^{2} +4 x + 6 \]The solution is \( \text{ All combinations of: }\frac{\pm 1,\pm 2,\pm 3,\pm 6}{\pm 1,\pm 2} \), which is option C.\begin{enumerate}[label=\Alph*.]
\item \( \text{ All combinations of: }\frac{\pm 1,\pm 2}{\pm 1,\pm 2,\pm 3,\pm 6} \)

 Distractor 3: Corresponds to the plus or minus of the inverse quotient (an/a0) of the factors. 
\item \( \pm 1,\pm 2,\pm 3,\pm 6 \)

This would have been the solution \textbf{if asked for the possible Integer roots}!
\item \( \text{ All combinations of: }\frac{\pm 1,\pm 2,\pm 3,\pm 6}{\pm 1,\pm 2} \)

* This is the solution \textbf{since we asked for the possible Rational roots}!
\item \( \pm 1,\pm 2 \)

 Distractor 1: Corresponds to the plus or minus factors of a1 only.
\item \( \text{ There is no formula or theorem that tells us all possible Rational roots.} \)

 Distractor 4: Corresponds to not recalling the theorem for rational roots of a polynomial.
\end{enumerate}

\textbf{General Comment:} We have a way to find the possible Rational roots. The possible Integer roots are the Integers in this list.
}
\litem{
Factor the polynomial below completely, knowing that $x+3$ is a factor. Then, choose the intervals the zeros of the polynomial belong to, where $z_1 \leq z_2 \leq z_3 \leq z_4$. \textit{To make the problem easier, all zeros are between -5 and 5.}
\[ f(x) = 6x^{4} -7 x^{3} -43 x^{2} +84 x -36 \]The solution is \( [-3, 0.6666666666666666, 1.5, 2] \), which is option D.\begin{enumerate}[label=\Alph*.]
\item \( z_1 \in [-3.15, -2.72], \text{   }  z_2 \in [-2.12, -1.91], z_3 \in [-0.4, -0.11], \text{   and   } z_4 \in [2.3, 4.1] \)

 Distractor 4: Corresponds to moving factors from one rational to another.
\item \( z_1 \in [-2.21, -1.48], \text{   }  z_2 \in [-1.99, -1.44], z_3 \in [-0.76, -0.63], \text{   and   } z_4 \in [2.3, 4.1] \)

 Distractor 3: Corresponds to negatives of all zeros AND inversing rational roots.
\item \( z_1 \in [-2.21, -1.48], \text{   }  z_2 \in [-1.99, -1.44], z_3 \in [-0.76, -0.63], \text{   and   } z_4 \in [2.3, 4.1] \)

 Distractor 1: Corresponds to negatives of all zeros.
\item \( z_1 \in [-3.15, -2.72], \text{   }  z_2 \in [0.04, 0.94], z_3 \in [1.03, 1.69], \text{   and   } z_4 \in [1.1, 2.1] \)

* This is the solution!
\item \( z_1 \in [-3.15, -2.72], \text{   }  z_2 \in [0.04, 0.94], z_3 \in [1.03, 1.69], \text{   and   } z_4 \in [1.1, 2.1] \)

 Distractor 2: Corresponds to inversing rational roots.
\end{enumerate}

\textbf{General Comment:} Remember to try the middle-most integers first as these normally are the zeros. Also, once you get it to a quadratic, you can use your other factoring techniques to finish factoring.
}
\litem{
Factor the polynomial below completely, knowing that $x+5$ is a factor. Then, choose the intervals the zeros of the polynomial belong to, where $z_1 \leq z_2 \leq z_3 \leq z_4$. \textit{To make the problem easier, all zeros are between -5 and 5.}
\[ f(x) = 12x^{4} +119 x^{3} +390 x^{2} +525 x + 250 \]The solution is \( [-5, -2, -1.6666666666666667, -1.25] \), which is option B.\begin{enumerate}[label=\Alph*.]
\item \( z_1 \in [1.1, 1.3], \text{   }  z_2 \in [1.38, 1.76], z_3 \in [1.6, 3.4], \text{   and   } z_4 \in [3.4, 5.5] \)

 Distractor 1: Corresponds to negatives of all zeros.
\item \( z_1 \in [-5.12, -4.81], \text{   }  z_2 \in [-2.07, -1.82], z_3 \in [-1.7, -1.1], \text{   and   } z_4 \in [-2.3, -0.7] \)

* This is the solution!
\item \( z_1 \in [0.36, 0.55], \text{   }  z_2 \in [1.82, 2.42], z_3 \in [3.7, 6.2], \text{   and   } z_4 \in [3.4, 5.5] \)

 Distractor 4: Corresponds to moving factors from one rational to another.
\item \( z_1 \in [-5.12, -4.81], \text{   }  z_2 \in [-2.07, -1.82], z_3 \in [-1.6, -0.1], \text{   and   } z_4 \in [-0.8, 0.5] \)

 Distractor 2: Corresponds to inversing rational roots.
\item \( z_1 \in [0.5, 0.89], \text{   }  z_2 \in [0.52, 0.87], z_3 \in [1.6, 3.4], \text{   and   } z_4 \in [3.4, 5.5] \)

 Distractor 3: Corresponds to negatives of all zeros AND inversing rational roots.
\end{enumerate}

\textbf{General Comment:} Remember to try the middle-most integers first as these normally are the zeros. Also, once you get it to a quadratic, you can use your other factoring techniques to finish factoring.
}
\litem{
Perform the division below. Then, find the intervals that correspond to the quotient in the form $ax^2+bx+c$ and remainder $r$.
\[ \frac{20x^{3} +65 x^{2} -48}{x + 3} \]The solution is \( 20x^{2} +5 x -15 + \frac{-3}{x + 3} \), which is option C.\begin{enumerate}[label=\Alph*.]
\item \( a \in [-60, -59], b \in [242, 246], c \in [-735, -734], \text{ and } r \in [2155, 2161]. \)

 You multipled by the synthetic number rather than bringing the first factor down.
\item \( a \in [-60, -59], b \in [-115, -110], c \in [-346, -344], \text{ and } r \in [-1087, -1081]. \)

 You divided by the opposite of the factor AND multipled the first factor rather than just bringing it down.
\item \( a \in [14, 22], b \in [2, 6], c \in [-20, -11], \text{ and } r \in [-9, -1]. \)

* This is the solution!
\item \( a \in [14, 22], b \in [120, 128], c \in [369, 376], \text{ and } r \in [1075, 1086]. \)

 You divided by the opposite of the factor.
\item \( a \in [14, 22], b \in [-20, -11], c \in [59, 66], \text{ and } r \in [-288, -282]. \)

 You multipled by the synthetic number and subtracted rather than adding during synthetic division.
\end{enumerate}

\textbf{General Comment:} Be sure to synthetically divide by the zero of the denominator! Also, make sure to include 0 placeholders for missing terms.
}
\end{enumerate}

\end{document}