\documentclass{extbook}[14pt]
\usepackage{multicol, enumerate, enumitem, hyperref, color, soul, setspace, parskip, fancyhdr, amssymb, amsthm, amsmath, bbm, latexsym, units, mathtools}
\everymath{\displaystyle}
\usepackage[headsep=0.5cm,headheight=0cm, left=1 in,right= 1 in,top= 1 in,bottom= 1 in]{geometry}
\usepackage{dashrule}  % Package to use the command below to create lines between items
\newcommand{\litem}[1]{\item #1

\rule{\textwidth}{0.4pt}}
\pagestyle{fancy}
\lhead{}
\chead{Answer Key for Progress Quiz 4 Version A}
\rhead{}
\lfoot{8448-1521}
\cfoot{}
\rfoot{Fall 2020}
\begin{document}
\textbf{This key should allow you to understand why you choose the option you did (beyond just getting a question right or wrong). \href{https://xronos.clas.ufl.edu/mac1105spring2020/courseDescriptionAndMisc/Exams/LearningFromResults}{More instructions on how to use this key can be found here}.}

\textbf{If you have a suggestion to make the keys better, \href{https://forms.gle/CZkbZmPbC9XALEE88}{please fill out the short survey here}.}

\textit{Note: This key is auto-generated and may contain issues and/or errors. The keys are reviewed after each exam to ensure grading is done accurately. If there are issues (like duplicate options), they are noted in the offline gradebook. The keys are a work-in-progress to give students as many resources to improve as possible.}

\rule{\textwidth}{0.4pt}

\begin{enumerate}\litem{
Choose the equation of the function graphed below.

\begin{center}
    \includegraphics[width=0.5\textwidth]{../Figures/rationalGraphToEquationA.png}
\end{center}



The solution is \( f(x) = \frac{1}{x - 1} + 2 \), which is option D.\begin{enumerate}[label=\Alph*.]
\item \( f(x) = \frac{1}{(x - 1)^2} + 2 \)

Corresponds to thinking the graph was a shifted version of $\frac{1}{x^2}$.
\item \( f(x) = \frac{-1}{(x + 1)^2} + 2 \)

Corresponds to thinking the graph was a shifted version of $\frac{1}{x^2}$, using the general form $f(x) = \frac{a}{x+h}+k$, and the opposite leading coefficient.
\item \( f(x) = \frac{-1}{x + 1} + 2 \)

Corresponds to using the general form $f(x) = \frac{a}{x+h}+k$ and the opposite leading coefficient.
\item \( f(x) = \frac{1}{x - 1} + 2 \)

This is the correct option.
\item \( \text{None of the above} \)

This corresponds to believing the vertex of the graph was not correct.
\end{enumerate}

\textbf{General Comment:} Remember that the general form of a basic rational equation is $ f(x) = \frac{a}{(x-h)^n} + k$, where $a$ is the leading coefficient (and in this case, we assume is either $1$ or $-1$), $n$ is the degree (in this case, either $1$ or $2$), and $(h, k)$ is the intersection of the asymptotes.
}
\litem{
Solve the rational equation below. Then, choose the interval(s) that the solution(s) belongs to.
\[ \frac{-63}{28x + 14} + 1 = \frac{-63}{28x + 14} \]
The solution is \( \text{all solutions are invalid or lead to complex values in the equation.} \), which is option A.\begin{enumerate}[label=\Alph*.]
\item \( \text{All solutions lead to invalid or complex values in the equation.} \)

*$x = -0.500$ leads to dividing by 0 in the original equation and thus is not a valid solution, which is the correct option.
\item \( x \in [-0.5,0.5] \)

$x = -0.500$, which corresponds to not checking if this value leads to dividing by 0 in the original equation and thus is not a valid solution.
\item \( x \in [0.1,0.6] \)

$x = 0.500$, which corresponds to not distributing the factor $28x + 14$ correctly when trying to eliminate the fraction.
\item \( x_1 \in [-1, -0.3] \text{ and } x_2 \in [0.2,1.1] \)

$x = -0.500 \text{ and } x = 0.500$, which corresponds to getting the correct solution and believing there should be a second solution to the equation.
\item \( x_1 \in [-1, -0.3] \text{ and } x_2 \in [-1.2,0.2] \)

$x = -0.500 \text{ and } x = -0.500$, which corresponds to getting the correct solution and believing there should be a second solution to the equation.
\end{enumerate}

\textbf{General Comment:} Distractors are different based on the number of solutions. Remember that after solving, we need to make sure our solution does not make the original equation divide by zero!
}
\litem{
Choose the graph of the equation below.
\[ f(x) = \frac{-1}{x + 2} + 2 \]
The solution is the graph below, which is option E.
\begin{center}
    \includegraphics[width=0.3\textwidth]{../Figures/rationalEquationToGraphEA.png}
\end{center}\begin{enumerate}[label=\Alph*.]
\begin{multicols}{2}
\item \includegraphics[width = 0.3\textwidth]{../Figures/rationalEquationToGraphAA.png}
\item \includegraphics[width = 0.3\textwidth]{../Figures/rationalEquationToGraphBA.png}
\item \includegraphics[width = 0.3\textwidth]{../Figures/rationalEquationToGraphCA.png}
\item \includegraphics[width = 0.3\textwidth]{../Figures/rationalEquationToGraphDA.png}
\end{multicols}\item None of the above.\end{enumerate}
\textbf{General Comment:} Remember that the general form of a basic rational equation is $ f(x) = \frac{a}{(x-h)^n} + k$, where $a$ is the leading coefficient (and in this case, we assume is either $1$ or $-1$), $n$ is the degree (in this case, either $1$ or $2$), and $(h, k)$ is the intersection of the asymptotes.
}
\litem{
Solve the rational equation below. Then, choose the interval(s) that the solution(s) belongs to.
\[ \frac{7x}{4x + 3} + \frac{-3x^{2}}{8x^{2} +26 x + 15} = \frac{-2}{2x + 5} \]
The solution is \( \text{There are two solutions: } x = -3.764 \text{ and } x = -0.145 \), which is option E.\begin{enumerate}[label=\Alph*.]
\item \( x \in [-0.75,2.24] \)


\item \( x_1 \in [-4.13, -2.89] \text{ and } x_2 \in [-1.61,-0.48] \)


\item \( \text{All solutions lead to invalid or complex values in the equation.} \)


\item \( x \in [-3.66,-1.02] \)


\item \( x_1 \in [-4.13, -2.89] \text{ and } x_2 \in [-0.35,0.12] \)

* $x = -3.764 \text{ and } x = -0.145$, which is the correct option.
\end{enumerate}

\textbf{General Comment:} Distractors are different based on the number of solutions. Remember that after solving, we need to make sure our solution does not make the original equation divide by zero!
}
\litem{
Solve the rational equation below. Then, choose the interval(s) that the solution(s) belongs to.
\[ \frac{56}{56x + 32} + 1 = \frac{56}{56x + 32} \]
The solution is \( \text{all solutions are invalid or lead to complex values in the equation.} \), which is option D.\begin{enumerate}[label=\Alph*.]
\item \( x_1 \in [-1.57, 0.43] \text{ and } x_2 \in [-0.5,1] \)

$x = -0.571 \text{ and } x = 0.571$, which corresponds to getting the correct solution and believing there should be a second solution to the equation.
\item \( x \in [-0.57,0.43] \)

$x = -0.571$, which corresponds to not checking if this value leads to dividing by 0 in the original equation and thus is not a valid solution.
\item \( x \in [0.57,2.57] \)

$x = 0.571$, which corresponds to not distributing the factor $56x + 32$ correctly when trying to eliminate the fraction.
\item \( \text{All solutions lead to invalid or complex values in the equation.} \)

*$x = -0.571$ leads to dividing by 0 in the original equation and thus is not a valid solution, which is the correct option.
\item \( x_1 \in [-1.57, 0.43] \text{ and } x_2 \in [-1.5,0.4] \)

$x = -0.571 \text{ and } x = -0.571$, which corresponds to getting the correct solution and believing there should be a second solution to the equation.
\end{enumerate}

\textbf{General Comment:} Distractors are different based on the number of solutions. Remember that after solving, we need to make sure our solution does not make the original equation divide by zero!
}
\litem{
Determine the domain of the function below.
\[ f(x) = \frac{6}{20x^{2} +31 x + 12} \]
The solution is \( \text{All Real numbers except } x = -0.800 \text{ and } x = -0.750. \), which is option D.\begin{enumerate}[label=\Alph*.]
\item \( \text{All Real numbers except } x = a, \text{ where } a \in [-0.82, -0.79] \)

All Real numbers except $x = -0.800$, which corresponds to removing only 1 value from the denominator.
\item \( \text{All Real numbers except } x = a \text{ and } x = b, \text{ where } a \in [-20.05, -19.95] \text{ and } b \in [-12.05, -11.96] \)

All Real numbers except $x = -20.000$ and $x = -12.000$, which corresponds to not factoring the denominator correctly.
\item \( \text{All Real numbers.} \)

This corresponds to thinking the denominator has complex roots or that rational functions have a domain of all Real numbers.
\item \( \text{All Real numbers except } x = a \text{ and } x = b, \text{ where } a \in [-0.82, -0.79] \text{ and } b \in [-0.76, -0.73] \)

All Real numbers except $x = -0.800$ and $x = -0.750$, which is the correct option.
\item \( \text{All Real numbers except } x = a, \text{ where } a \in [-20.05, -19.95] \)

All Real numbers except $x = -20.000$, which corresponds to removing a distractor value from the denominator.
\end{enumerate}

\textbf{General Comment:} Recall that dividing by zero is not a real number. Therefore the domain is all real numbers \textbf{except} those that make the denominator 0.
}
\litem{
Choose the graph of the equation below.
\[ f(x) = \frac{1}{x + 1} + 2 \]
The solution is the graph below, which is option B.
\begin{center}
    \includegraphics[width=0.3\textwidth]{../Figures/rationalEquationToGraphCopyBA.png}
\end{center}\begin{enumerate}[label=\Alph*.]
\begin{multicols}{2}
\item \includegraphics[width = 0.3\textwidth]{../Figures/rationalEquationToGraphCopyAA.png}
\item \includegraphics[width = 0.3\textwidth]{../Figures/rationalEquationToGraphCopyBA.png}
\item \includegraphics[width = 0.3\textwidth]{../Figures/rationalEquationToGraphCopyCA.png}
\item \includegraphics[width = 0.3\textwidth]{../Figures/rationalEquationToGraphCopyDA.png}
\end{multicols}\item None of the above.\end{enumerate}
\textbf{General Comment:} Remember that the general form of a basic rational equation is $ f(x) = \frac{a}{(x-h)^n} + k$, where $a$ is the leading coefficient (and in this case, we assume is either $1$ or $-1$), $n$ is the degree (in this case, either $1$ or $2$), and $(h, k)$ is the intersection of the asymptotes.
}
\litem{
Solve the rational equation below. Then, choose the interval(s) that the solution(s) belongs to.
\[ \frac{-2x}{-4x -4} + \frac{-7x^{2}}{-8x^{2} -36 x -28} = \frac{-5}{2x + 7} \]
The solution is \( \text{There are two solutions: } x = -2.301 \text{ and } x = -0.790 \), which is option C.\begin{enumerate}[label=\Alph*.]
\item \( x_1 \in [-3.29, -1.73] \text{ and } x_2 \in [-1.04,-0.9] \)


\item \( \text{All solutions lead to invalid or complex values in the equation.} \)


\item \( x_1 \in [-3.29, -1.73] \text{ and } x_2 \in [-0.91,-0.75] \)

* $x = -2.301 \text{ and } x = -0.790$, which is the correct option.
\item \( x \in [-2.09,0.72] \)


\item \( x \in [-3.97,-3.05] \)


\end{enumerate}

\textbf{General Comment:} Distractors are different based on the number of solutions. Remember that after solving, we need to make sure our solution does not make the original equation divide by zero!
}
\litem{
Choose the equation of the function graphed below.

\begin{center}
    \includegraphics[width=0.5\textwidth]{../Figures/rationalGraphToEquationCopyA.png}
\end{center}



The solution is \( f(x) = \frac{-1}{(x + 3)^2} + 2 \), which is option A.\begin{enumerate}[label=\Alph*.]
\item \( f(x) = \frac{-1}{(x + 3)^2} + 2 \)

This is the correct option.
\item \( f(x) = \frac{-1}{x + 3} + 2 \)

Corresponds to thinking the graph was a shifted version of $\frac{1}{x}$.
\item \( f(x) = \frac{1}{(x - 3)^2} + 2 \)

Corresponds to using the general form $f(x) = \frac{a}{(x+h)^2}+k$ and the opposite leading coefficient.
\item \( f(x) = \frac{1}{x - 3} + 2 \)

Corresponds to thinking the graph was a shifted version of $\frac{1}{x}$, using the general form $f(x) = \frac{a}{(x+h)^2}+k$, and the opposite leading coefficient.
\item \( \text{None of the above} \)

This corresponds to believing the vertex of the graph was not correct.
\end{enumerate}

\textbf{General Comment:} Remember that the general form of a basic rational equation is $ f(x) = \frac{a}{(x-h)^n} + k$, where $a$ is the leading coefficient (and in this case, we assume is either $1$ or $-1$), $n$ is the degree (in this case, either $1$ or $2$), and $(h, k)$ is the intersection of the asymptotes.
}
\litem{
Determine the domain of the function below.
\[ f(x) = \frac{4}{20x^{2} +54 x + 36} \]
The solution is \( \text{All Real numbers except } x = -1.500 \text{ and } x = -1.200. \), which is option C.\begin{enumerate}[label=\Alph*.]
\item \( \text{All Real numbers except } x = a \text{ and } x = b, \text{ where } a \in [-30.02, -29.55] \text{ and } b \in [-24.02, -23.73] \)

All Real numbers except $x = -30.000$ and $x = -24.000$, which corresponds to not factoring the denominator correctly.
\item \( \text{All Real numbers except } x = a, \text{ where } a \in [-1.58, -1.45] \)

All Real numbers except $x = -1.500$, which corresponds to removing only 1 value from the denominator.
\item \( \text{All Real numbers except } x = a \text{ and } x = b, \text{ where } a \in [-1.58, -1.45] \text{ and } b \in [-1.39, -1.11] \)

All Real numbers except $x = -1.500$ and $x = -1.200$, which is the correct option.
\item \( \text{All Real numbers except } x = a, \text{ where } a \in [-30.02, -29.55] \)

All Real numbers except $x = -30.000$, which corresponds to removing a distractor value from the denominator.
\item \( \text{All Real numbers.} \)

This corresponds to thinking the denominator has complex roots or that rational functions have a domain of all Real numbers.
\end{enumerate}

\textbf{General Comment:} Recall that dividing by zero is not a real number. Therefore the domain is all real numbers \textbf{except} those that make the denominator 0.
}
\end{enumerate}

\end{document}