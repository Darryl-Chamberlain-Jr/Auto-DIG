\documentclass[14pt]{extbook}
\usepackage{multicol, enumerate, enumitem, hyperref, color, soul, setspace, parskip, fancyhdr} %General Packages
\usepackage{amssymb, amsthm, amsmath, latexsym, units, mathtools} %Math Packages
\everymath{\displaystyle} %All math in Display Style
% Packages with additional options
\usepackage[headsep=0.5cm,headheight=12pt, left=1 in,right= 1 in,top= 1 in,bottom= 1 in]{geometry}
\usepackage[usenames,dvipsnames]{xcolor}
\usepackage{dashrule}  % Package to use the command below to create lines between items
\newcommand{\litem}[1]{\item#1\hspace*{-1cm}\rule{\textwidth}{0.4pt}}
\pagestyle{fancy}
\lhead{Progress Quiz 4}
\chead{}
\rhead{Version C}
\lfoot{5346-5907}
\cfoot{}
\rfoot{Summer C 2021}
\begin{document}

\begin{enumerate}
\litem{
Factor the polynomial below completely, knowing that $x + 5$ is a factor. Then, choose the intervals the zeros of the polynomial belong to, where $z_1 \leq z_2 \leq z_3 \leq z_4$. \textit{To make the problem easier, all zeros are between -5 and 5.}\[ f(x) = 15x^{4} +139 x^{3} +383 x^{2} +333 x + 90 \]\begin{enumerate}[label=\Alph*.]
\item \( z_1 \in [-5.01, -4.54], \text{   }  z_2 \in [-3.73, -2.62], z_3 \in [-1.6, 1.2], \text{   and   } z_4 \in [-1.19, 0.12] \)
\item \( z_1 \in [0.37, 0.91], \text{   }  z_2 \in [0.35, 0.89], z_3 \in [2.4, 3.6], \text{   and   } z_4 \in [3.93, 5.1] \)
\item \( z_1 \in [-0.16, 0.36], \text{   }  z_2 \in [1.98, 3.21], z_3 \in [2.4, 3.6], \text{   and   } z_4 \in [3.93, 5.1] \)
\item \( z_1 \in [-5.01, -4.54], \text{   }  z_2 \in [-3.73, -2.62], z_3 \in [-4, -0.9], \text{   and   } z_4 \in [-2.54, -1.2] \)
\item \( z_1 \in [1.45, 1.87], \text{   }  z_2 \in [1.62, 1.94], z_3 \in [2.4, 3.6], \text{   and   } z_4 \in [3.93, 5.1] \)

\end{enumerate} }
\litem{
Perform the division below. Then, find the intervals that correspond to the quotient in the form $ax^2+bx+c$ and remainder $r$.\[ \frac{6x^{3} -1 x^{2} -20 x + 14}{x + 2} \]\begin{enumerate}[label=\Alph*.]
\item \( a \in [6, 9], \text{   } b \in [-20.2, -15.3], \text{   } c \in [35, 40], \text{   and   } r \in [-99, -93]. \)
\item \( a \in [6, 9], \text{   } b \in [-13.5, -8.7], \text{   } c \in [5, 13], \text{   and   } r \in [-3, 4]. \)
\item \( a \in [-17, -11], \text{   } b \in [-30.2, -22.4], \text{   } c \in [-73, -68], \text{   and   } r \in [-128, -122]. \)
\item \( a \in [6, 9], \text{   } b \in [9.7, 12.2], \text{   } c \in [0, 3], \text{   and   } r \in [12, 22]. \)
\item \( a \in [-17, -11], \text{   } b \in [22.7, 24.5], \text{   } c \in [-69, -65], \text{   and   } r \in [142, 152]. \)

\end{enumerate} }
\litem{
Factor the polynomial below completely. Then, choose the intervals the zeros of the polynomial belong to, where $z_1 \leq z_2 \leq z_3$. \textit{To make the problem easier, all zeros are between -5 and 5.}\[ f(x) = 8x^{3} +38 x^{2} +15 x -36 \]\begin{enumerate}[label=\Alph*.]
\item \( z_1 \in [-4.08, -3.9], \text{   }  z_2 \in [-1.8, -1.23], \text{   and   } z_3 \in [0.1, 0.9] \)
\item \( z_1 \in [-1.37, -1.16], \text{   }  z_2 \in [0.22, 0.87], \text{   and   } z_3 \in [2.3, 4.9] \)
\item \( z_1 \in [-0.64, -0.36], \text{   }  z_2 \in [2.91, 3.23], \text{   and   } z_3 \in [2.3, 4.9] \)
\item \( z_1 \in [-0.86, -0.55], \text{   }  z_2 \in [1.36, 1.85], \text{   and   } z_3 \in [2.3, 4.9] \)
\item \( z_1 \in [-4.08, -3.9], \text{   }  z_2 \in [-1.2, -0.15], \text{   and   } z_3 \in [1, 2.4] \)

\end{enumerate} }
\litem{
What are the \textit{possible Integer} roots of the polynomial below?\[ f(x) = 4x^{3} +7 x^{2} +5 x + 5 \]\begin{enumerate}[label=\Alph*.]
\item \( \text{ All combinations of: }\frac{\pm 1,\pm 2,\pm 4}{\pm 1,\pm 5} \)
\item \( \pm 1,\pm 5 \)
\item \( \pm 1,\pm 2,\pm 4 \)
\item \( \text{ All combinations of: }\frac{\pm 1,\pm 5}{\pm 1,\pm 2,\pm 4} \)
\item \( \text{There is no formula or theorem that tells us all possible Integer roots.} \)

\end{enumerate} }
\litem{
Factor the polynomial below completely. Then, choose the intervals the zeros of the polynomial belong to, where $z_1 \leq z_2 \leq z_3$. \textit{To make the problem easier, all zeros are between -5 and 5.}\[ f(x) = 8x^{3} -22 x^{2} -65 x + 100 \]\begin{enumerate}[label=\Alph*.]
\item \( z_1 \in [-4, -3], \text{   }  z_2 \in [-0.96, -0.67], \text{   and   } z_3 \in [0.1, 2.1] \)
\item \( z_1 \in [-4, -3], \text{   }  z_2 \in [-0.74, -0.29], \text{   and   } z_3 \in [4.9, 5.1] \)
\item \( z_1 \in [-3.5, -1.5], \text{   }  z_2 \in [1.21, 1.28], \text{   and   } z_3 \in [3.8, 4.2] \)
\item \( z_1 \in [-4, -3], \text{   }  z_2 \in [-1.31, -1.08], \text{   and   } z_3 \in [2.1, 3] \)
\item \( z_1 \in [-2.4, 2.6], \text{   }  z_2 \in [0.79, 0.87], \text{   and   } z_3 \in [3.8, 4.2] \)

\end{enumerate} }
\litem{
Perform the division below. Then, find the intervals that correspond to the quotient in the form $ax^2+bx+c$ and remainder $r$.\[ \frac{12x^{3} -36 x + 29}{x + 2} \]\begin{enumerate}[label=\Alph*.]
\item \( a \in [12, 15], b \in [-26, -18], c \in [10, 14], \text{ and } r \in [5, 7]. \)
\item \( a \in [-25, -16], b \in [-48, -47], c \in [-135, -129], \text{ and } r \in [-240, -232]. \)
\item \( a \in [-25, -16], b \in [40, 54], c \in [-135, -129], \text{ and } r \in [293, 294]. \)
\item \( a \in [12, 15], b \in [-42, -32], c \in [67, 77], \text{ and } r \in [-188, -182]. \)
\item \( a \in [12, 15], b \in [21, 29], c \in [10, 14], \text{ and } r \in [47, 55]. \)

\end{enumerate} }
\litem{
Factor the polynomial below completely, knowing that $x -5$ is a factor. Then, choose the intervals the zeros of the polynomial belong to, where $z_1 \leq z_2 \leq z_3 \leq z_4$. \textit{To make the problem easier, all zeros are between -5 and 5.}\[ f(x) = 6x^{4} -19 x^{3} -81 x^{2} +90 x + 200 \]\begin{enumerate}[label=\Alph*.]
\item \( z_1 \in [-3.1, -1.7], \text{   }  z_2 \in [-1.43, -1.12], z_3 \in [1.75, 2.23], \text{   and   } z_4 \in [4.3, 6.3] \)
\item \( z_1 \in [-1.3, -0.4], \text{   }  z_2 \in [-0.6, 0.2], z_3 \in [1.75, 2.23], \text{   and   } z_4 \in [4.3, 6.3] \)
\item \( z_1 \in [-6.8, -4.8], \text{   }  z_2 \in [-2.63, -1.81], z_3 \in [0.33, 0.66], \text{   and   } z_4 \in [0.2, 2] \)
\item \( z_1 \in [-6.8, -4.8], \text{   }  z_2 \in [-2.63, -1.81], z_3 \in [0.58, 1.18], \text{   and   } z_4 \in [4.3, 6.3] \)
\item \( z_1 \in [-6.8, -4.8], \text{   }  z_2 \in [-2.63, -1.81], z_3 \in [1.07, 1.78], \text{   and   } z_4 \in [1.4, 3.4] \)

\end{enumerate} }
\litem{
What are the \textit{possible Rational} roots of the polynomial below?\[ f(x) = 2x^{2} +4 x + 4 \]\begin{enumerate}[label=\Alph*.]
\item \( \text{ All combinations of: }\frac{\pm 1,\pm 2,\pm 4}{\pm 1,\pm 2} \)
\item \( \text{ All combinations of: }\frac{\pm 1,\pm 2}{\pm 1,\pm 2,\pm 4} \)
\item \( \pm 1,\pm 2 \)
\item \( \pm 1,\pm 2,\pm 4 \)
\item \( \text{ There is no formula or theorem that tells us all possible Rational roots.} \)

\end{enumerate} }
\litem{
Perform the division below. Then, find the intervals that correspond to the quotient in the form $ax^2+bx+c$ and remainder $r$.\[ \frac{12x^{3} +17 x^{2} -24 x -18}{x + 2} \]\begin{enumerate}[label=\Alph*.]
\item \( a \in [8, 16], \text{   } b \in [-8, -5], \text{   } c \in [-11, -5], \text{   and   } r \in [-2, 10]. \)
\item \( a \in [8, 16], \text{   } b \in [38, 42], \text{   } c \in [56, 61], \text{   and   } r \in [96, 102]. \)
\item \( a \in [-26, -19], \text{   } b \in [61, 70], \text{   } c \in [-160, -152], \text{   and   } r \in [288, 293]. \)
\item \( a \in [8, 16], \text{   } b \in [-22, -17], \text{   } c \in [29, 34], \text{   and   } r \in [-121, -116]. \)
\item \( a \in [-26, -19], \text{   } b \in [-35, -28], \text{   } c \in [-87, -83], \text{   and   } r \in [-192, -189]. \)

\end{enumerate} }
\litem{
Perform the division below. Then, find the intervals that correspond to the quotient in the form $ax^2+bx+c$ and remainder $r$.\[ \frac{6x^{3} +35 x^{2} -127}{x + 5} \]\begin{enumerate}[label=\Alph*.]
\item \( a \in [6, 10], b \in [4, 6], c \in [-32, -21], \text{ and } r \in [-2, -1]. \)
\item \( a \in [6, 10], b \in [65, 67], c \in [316, 329], \text{ and } r \in [1496, 1502]. \)
\item \( a \in [6, 10], b \in [-5, 1], c \in [5, 9], \text{ and } r \in [-163, -160]. \)
\item \( a \in [-30, -28], b \in [183, 190], c \in [-926, -922], \text{ and } r \in [4497, 4503]. \)
\item \( a \in [-30, -28], b \in [-117, -107], c \in [-578, -574], \text{ and } r \in [-3002, -3001]. \)

\end{enumerate} }
\end{enumerate}

\end{document}