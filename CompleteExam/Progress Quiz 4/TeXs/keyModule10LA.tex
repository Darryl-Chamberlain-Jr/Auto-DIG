\documentclass{extbook}[14pt]
\usepackage{multicol, enumerate, enumitem, hyperref, color, soul, setspace, parskip, fancyhdr, amssymb, amsthm, amsmath, bbm, latexsym, units, mathtools}
\everymath{\displaystyle}
\usepackage[headsep=0.5cm,headheight=0cm, left=1 in,right= 1 in,top= 1 in,bottom= 1 in]{geometry}
\usepackage{dashrule}  % Package to use the command below to create lines between items
\newcommand{\litem}[1]{\item #1

\rule{\textwidth}{0.4pt}}
\pagestyle{fancy}
\lhead{}
\chead{Answer Key for Progress Quiz 4 Version A}
\rhead{}
\lfoot{9187-5854}
\cfoot{}
\rfoot{Spring 2021}
\begin{document}
\textbf{This key should allow you to understand why you choose the option you did (beyond just getting a question right or wrong). \href{https://xronos.clas.ufl.edu/mac1105spring2020/courseDescriptionAndMisc/Exams/LearningFromResults}{More instructions on how to use this key can be found here}.}

\textbf{If you have a suggestion to make the keys better, \href{https://forms.gle/CZkbZmPbC9XALEE88}{please fill out the short survey here}.}

\textit{Note: This key is auto-generated and may contain issues and/or errors. The keys are reviewed after each exam to ensure grading is done accurately. If there are issues (like duplicate options), they are noted in the offline gradebook. The keys are a work-in-progress to give students as many resources to improve as possible.}

\rule{\textwidth}{0.4pt}

\begin{enumerate}\litem{
Factor the polynomial below completely. Then, choose the intervals the zeros of the polynomial belong to, where $z_1 \leq z_2 \leq z_3$. \textit{To make the problem easier, all zeros are between -5 and 5.}
\[ f(x) = 20x^{3} -39 x^{2} -14 x + 24 \]The solution is \( [-0.8, 0.75, 2] \), which is option B.\begin{enumerate}[label=\Alph*.]
\item \( z_1 \in [-1.65, -1.12], \text{   }  z_2 \in [0.87, 1.97], \text{   and   } z_3 \in [1.94, 2.35] \)

 Distractor 2: Corresponds to inversing rational roots.
\item \( z_1 \in [-1.21, -0.43], \text{   }  z_2 \in [0.73, 1.22], \text{   and   } z_3 \in [1.94, 2.35] \)

* This is the solution!
\item \( z_1 \in [-2.02, -1.74], \text{   }  z_2 \in [-1.61, -1.28], \text{   and   } z_3 \in [1.19, 1.28] \)

 Distractor 3: Corresponds to negatives of all zeros AND inversing rational roots.
\item \( z_1 \in [-2.02, -1.74], \text{   }  z_2 \in [-1.13, -0.62], \text{   and   } z_3 \in [0.38, 0.87] \)

 Distractor 1: Corresponds to negatives of all zeros.
\item \( z_1 \in [-3.47, -2.95], \text{   }  z_2 \in [-2.19, -1.88], \text{   and   } z_3 \in [-0.01, 0.26] \)

 Distractor 4: Corresponds to moving factors from one rational to another.
\end{enumerate}

\textbf{General Comment:} Remember to try the middle-most integers first as these normally are the zeros. Also, once you get it to a quadratic, you can use your other factoring techniques to finish factoring.
}
\litem{
Perform the division below. Then, find the intervals that correspond to the quotient in the form $ax^2+bx+c$ and remainder $r$.
\[ \frac{6x^{3} +30 x^{2} +48 x + 29}{x + 2} \]The solution is \( 6x^{2} +18 x + 12 + \frac{5}{x + 2} \), which is option E.\begin{enumerate}[label=\Alph*.]
\item \( a \in [-13, -7], \text{   } b \in [52, 56], \text{   } c \in [-61, -52], \text{   and   } r \in [148, 154]. \)

 You multiplied by the synthetic number rather than bringing the first factor down.
\item \( a \in [-13, -7], \text{   } b \in [0, 8], \text{   } c \in [60, 62], \text{   and   } r \in [148, 154]. \)

 You divided by the opposite of the factor AND multiplied the first factor rather than just bringing it down.
\item \( a \in [4, 13], \text{   } b \in [8, 15], \text{   } c \in [12, 18], \text{   and   } r \in [-7, -5]. \)

 You multiplied by the synthetic number and subtracted rather than adding during synthetic division.
\item \( a \in [4, 13], \text{   } b \in [40, 46], \text{   } c \in [130, 142], \text{   and   } r \in [290, 298]. \)

 You divided by the opposite of the factor.
\item \( a \in [4, 13], \text{   } b \in [17, 19], \text{   } c \in [12, 18], \text{   and   } r \in [3, 9]. \)

* This is the solution!
\end{enumerate}

\textbf{General Comment:} Be sure to synthetically divide by the zero of the denominator!
}
\litem{
What are the \textit{possible Integer} roots of the polynomial below?
\[ f(x) = 5x^{3} +4 x^{2} +2 x + 2 \]The solution is \( \pm 1,\pm 2 \), which is option B.\begin{enumerate}[label=\Alph*.]
\item \( \text{ All combinations of: }\frac{\pm 1,\pm 5}{\pm 1,\pm 2} \)

 Distractor 3: Corresponds to the plus or minus of the inverse quotient (an/a0) of the factors. 
\item \( \pm 1,\pm 2 \)

* This is the solution \textbf{since we asked for the possible Integer roots}!
\item \( \text{ All combinations of: }\frac{\pm 1,\pm 2}{\pm 1,\pm 5} \)

This would have been the solution \textbf{if asked for the possible Rational roots}!
\item \( \pm 1,\pm 5 \)

 Distractor 1: Corresponds to the plus or minus factors of a1 only.
\item \( \text{There is no formula or theorem that tells us all possible Integer roots.} \)

 Distractor 4: Corresponds to not recognizing Integers as a subset of Rationals.
\end{enumerate}

\textbf{General Comment:} We have a way to find the possible Rational roots. The possible Integer roots are the Integers in this list.
}
\litem{
Factor the polynomial below completely. Then, choose the intervals the zeros of the polynomial belong to, where $z_1 \leq z_2 \leq z_3$. \textit{To make the problem easier, all zeros are between -5 and 5.}
\[ f(x) = 15x^{3} -8 x^{2} -36 x -16 \]The solution is \( [-0.8, -0.6666666666666666, 2] \), which is option D.\begin{enumerate}[label=\Alph*.]
\item \( z_1 \in [-2.44, -1.88], \text{   }  z_2 \in [1.24, 1.38], \text{   and   } z_3 \in [0.94, 1.69] \)

 Distractor 3: Corresponds to negatives of all zeros AND inversing rational roots.
\item \( z_1 \in [-2.44, -1.88], \text{   }  z_2 \in [0.16, 0.43], \text{   and   } z_3 \in [1.74, 2.21] \)

 Distractor 4: Corresponds to moving factors from one rational to another.
\item \( z_1 \in [-2.44, -1.88], \text{   }  z_2 \in [0.45, 1.12], \text{   and   } z_3 \in [0.58, 1.23] \)

 Distractor 1: Corresponds to negatives of all zeros.
\item \( z_1 \in [-0.86, -0.75], \text{   }  z_2 \in [-0.97, -0.08], \text{   and   } z_3 \in [1.74, 2.21] \)

* This is the solution!
\item \( z_1 \in [-1.58, -1.32], \text{   }  z_2 \in [-1.27, -0.89], \text{   and   } z_3 \in [1.74, 2.21] \)

 Distractor 2: Corresponds to inversing rational roots.
\end{enumerate}

\textbf{General Comment:} Remember to try the middle-most integers first as these normally are the zeros. Also, once you get it to a quadratic, you can use your other factoring techniques to finish factoring.
}
\litem{
Perform the division below. Then, find the intervals that correspond to the quotient in the form $ax^2+bx+c$ and remainder $r$.
\[ \frac{4x^{3} +12 x^{2} -20}{x + 2} \]The solution is \( 4x^{2} +4 x -8 + \frac{-4}{x + 2} \), which is option C.\begin{enumerate}[label=\Alph*.]
\item \( a \in [-9, -4], b \in [-5, -2], c \in [-15, -7], \text{ and } r \in [-37, -30]. \)

 You divided by the opposite of the factor AND multipled the first factor rather than just bringing it down.
\item \( a \in [-9, -4], b \in [26, 29], c \in [-59, -55], \text{ and } r \in [90, 93]. \)

 You multipled by the synthetic number rather than bringing the first factor down.
\item \( a \in [1, 8], b \in [3, 7], c \in [-15, -7], \text{ and } r \in [-4, 3]. \)

* This is the solution!
\item \( a \in [1, 8], b \in [16, 23], c \in [40, 41], \text{ and } r \in [52, 65]. \)

 You divided by the opposite of the factor.
\item \( a \in [1, 8], b \in [-2, 1], c \in [0, 3], \text{ and } r \in [-26, -16]. \)

 You multipled by the synthetic number and subtracted rather than adding during synthetic division.
\end{enumerate}

\textbf{General Comment:} Be sure to synthetically divide by the zero of the denominator! Also, make sure to include 0 placeholders for missing terms.
}
\litem{
Perform the division below. Then, find the intervals that correspond to the quotient in the form $ax^2+bx+c$ and remainder $r$.
\[ \frac{8x^{3} +38 x^{2} -16 x -35}{x + 5} \]The solution is \( 8x^{2} -2 x -6 + \frac{-5}{x + 5} \), which is option E.\begin{enumerate}[label=\Alph*.]
\item \( a \in [3, 10], \text{   } b \in [-10, -6], \text{   } c \in [40, 46], \text{   and   } r \in [-299, -298]. \)

 You multiplied by the synthetic number and subtracted rather than adding during synthetic division.
\item \( a \in [-42, -39], \text{   } b \in [237, 243], \text{   } c \in [-1206, -1200], \text{   and   } r \in [5994, 5997]. \)

 You multiplied by the synthetic number rather than bringing the first factor down.
\item \( a \in [-42, -39], \text{   } b \in [-165, -160], \text{   } c \in [-831, -823], \text{   and   } r \in [-4170, -4164]. \)

 You divided by the opposite of the factor AND multiplied the first factor rather than just bringing it down.
\item \( a \in [3, 10], \text{   } b \in [77, 79], \text{   } c \in [369, 377], \text{   and   } r \in [1831, 1838]. \)

 You divided by the opposite of the factor.
\item \( a \in [3, 10], \text{   } b \in [-2, 6], \text{   } c \in [-8, -2], \text{   and   } r \in [-8, -3]. \)

* This is the solution!
\end{enumerate}

\textbf{General Comment:} Be sure to synthetically divide by the zero of the denominator!
}
\litem{
What are the \textit{possible Rational} roots of the polynomial below?
\[ f(x) = 4x^{2} +3 x + 3 \]The solution is \( \text{ All combinations of: }\frac{\pm 1,\pm 3}{\pm 1,\pm 2,\pm 4} \), which is option B.\begin{enumerate}[label=\Alph*.]
\item \( \text{ All combinations of: }\frac{\pm 1,\pm 2,\pm 4}{\pm 1,\pm 3} \)

 Distractor 3: Corresponds to the plus or minus of the inverse quotient (an/a0) of the factors. 
\item \( \text{ All combinations of: }\frac{\pm 1,\pm 3}{\pm 1,\pm 2,\pm 4} \)

* This is the solution \textbf{since we asked for the possible Rational roots}!
\item \( \pm 1,\pm 2,\pm 4 \)

 Distractor 1: Corresponds to the plus or minus factors of a1 only.
\item \( \pm 1,\pm 3 \)

This would have been the solution \textbf{if asked for the possible Integer roots}!
\item \( \text{ There is no formula or theorem that tells us all possible Rational roots.} \)

 Distractor 4: Corresponds to not recalling the theorem for rational roots of a polynomial.
\end{enumerate}

\textbf{General Comment:} We have a way to find the possible Rational roots. The possible Integer roots are the Integers in this list.
}
\litem{
Factor the polynomial below completely, knowing that $x-4$ is a factor. Then, choose the intervals the zeros of the polynomial belong to, where $z_1 \leq z_2 \leq z_3 \leq z_4$. \textit{To make the problem easier, all zeros are between -5 and 5.}
\[ f(x) = 8x^{4} -90 x^{3} +331 x^{2} -441 x + 180 \]The solution is \( [0.75, 1.5, 4, 5] \), which is option A.\begin{enumerate}[label=\Alph*.]
\item \( z_1 \in [0.69, 1], \text{   }  z_2 \in [1.43, 2.16], z_3 \in [3.7, 4.09], \text{   and   } z_4 \in [4.67, 5.01] \)

* This is the solution!
\item \( z_1 \in [-5.16, -4.81], \text{   }  z_2 \in [-4.36, -3.69], z_3 \in [-3.01, -2.4], \text{   and   } z_4 \in [-0.4, 0.06] \)

 Distractor 4: Corresponds to moving factors from one rational to another.
\item \( z_1 \in [-5.16, -4.81], \text{   }  z_2 \in [-4.36, -3.69], z_3 \in [-1.49, -1.21], \text{   and   } z_4 \in [-0.74, -0.63] \)

 Distractor 3: Corresponds to negatives of all zeros AND inversing rational roots.
\item \( z_1 \in [-5.16, -4.81], \text{   }  z_2 \in [-4.36, -3.69], z_3 \in [-1.84, -1.39], \text{   and   } z_4 \in [-1.1, -0.74] \)

 Distractor 1: Corresponds to negatives of all zeros.
\item \( z_1 \in [0.62, 0.69], \text{   }  z_2 \in [0.74, 1.48], z_3 \in [3.7, 4.09], \text{   and   } z_4 \in [4.67, 5.01] \)

 Distractor 2: Corresponds to inversing rational roots.
\end{enumerate}

\textbf{General Comment:} Remember to try the middle-most integers first as these normally are the zeros. Also, once you get it to a quadratic, you can use your other factoring techniques to finish factoring.
}
\litem{
Factor the polynomial below completely, knowing that $x-4$ is a factor. Then, choose the intervals the zeros of the polynomial belong to, where $z_1 \leq z_2 \leq z_3 \leq z_4$. \textit{To make the problem easier, all zeros are between -5 and 5.}
\[ f(x) = 9x^{4} -72 x^{3} +188 x^{2} -192 x + 64 \]The solution is \( [0.6666666666666666, 1.3333333333333333, 2, 4] \), which is option E.\begin{enumerate}[label=\Alph*.]
\item \( z_1 \in [-4.2, -3.96], \text{   }  z_2 \in [-2.06, -1.99], z_3 \in [-2.06, -1.88], \text{   and   } z_4 \in [-0.45, -0.4] \)

 Distractor 4: Corresponds to moving factors from one rational to another.
\item \( z_1 \in [-4.2, -3.96], \text{   }  z_2 \in [-2.06, -1.99], z_3 \in [-1.54, -1.39], \text{   and   } z_4 \in [-0.76, -0.7] \)

 Distractor 3: Corresponds to negatives of all zeros AND inversing rational roots.
\item \( z_1 \in [-4.2, -3.96], \text{   }  z_2 \in [-2.06, -1.99], z_3 \in [-1.37, -1.31], \text{   and   } z_4 \in [-0.68, -0.6] \)

 Distractor 1: Corresponds to negatives of all zeros.
\item \( z_1 \in [0.72, 0.84], \text{   }  z_2 \in [1.49, 1.8], z_3 \in [1.9, 2.04], \text{   and   } z_4 \in [3.95, 4.05] \)

 Distractor 2: Corresponds to inversing rational roots.
\item \( z_1 \in [0.51, 0.74], \text{   }  z_2 \in [1.21, 1.41], z_3 \in [1.9, 2.04], \text{   and   } z_4 \in [3.95, 4.05] \)

* This is the solution!
\end{enumerate}

\textbf{General Comment:} Remember to try the middle-most integers first as these normally are the zeros. Also, once you get it to a quadratic, you can use your other factoring techniques to finish factoring.
}
\litem{
Perform the division below. Then, find the intervals that correspond to the quotient in the form $ax^2+bx+c$ and remainder $r$.
\[ \frac{4x^{3} -12 x + 4}{x + 2} \]The solution is \( 4x^{2} -8 x + 4 + \frac{-4}{x + 2} \), which is option C.\begin{enumerate}[label=\Alph*.]
\item \( a \in [-9, -2], b \in [15.4, 17.6], c \in [-45, -42], \text{ and } r \in [92, 93]. \)

 You multipled by the synthetic number rather than bringing the first factor down.
\item \( a \in [4, 6], b \in [-12.8, -9.8], c \in [23, 30], \text{ and } r \in [-68, -65]. \)

 You multipled by the synthetic number and subtracted rather than adding during synthetic division.
\item \( a \in [4, 6], b \in [-11.5, -5.8], c \in [4, 5], \text{ and } r \in [-5, 1]. \)

* This is the solution!
\item \( a \in [4, 6], b \in [6.7, 12.3], c \in [4, 5], \text{ and } r \in [10, 16]. \)

 You divided by the opposite of the factor.
\item \( a \in [-9, -2], b \in [-17.3, -12.6], c \in [-45, -42], \text{ and } r \in [-84, -81]. \)

 You divided by the opposite of the factor AND multipled the first factor rather than just bringing it down.
\end{enumerate}

\textbf{General Comment:} Be sure to synthetically divide by the zero of the denominator! Also, make sure to include 0 placeholders for missing terms.
}
\end{enumerate}

\end{document}