\documentclass{extbook}[14pt]
\usepackage{multicol, enumerate, enumitem, hyperref, color, soul, setspace, parskip, fancyhdr, amssymb, amsthm, amsmath, latexsym, units, mathtools}
\everymath{\displaystyle}
\usepackage[headsep=0.5cm,headheight=0cm, left=1 in,right= 1 in,top= 1 in,bottom= 1 in]{geometry}
\usepackage{dashrule}  % Package to use the command below to create lines between items
\newcommand{\litem}[1]{\item #1

\rule{\textwidth}{0.4pt}}
\pagestyle{fancy}
\lhead{}
\chead{Answer Key for Progress Quiz 4 Version A}
\rhead{}
\lfoot{5346-5907}
\cfoot{}
\rfoot{Summer C 2021}
\begin{document}
\textbf{This key should allow you to understand why you choose the option you did (beyond just getting a question right or wrong). \href{https://xronos.clas.ufl.edu/mac1105spring2020/courseDescriptionAndMisc/Exams/LearningFromResults}{More instructions on how to use this key can be found here}.}

\textbf{If you have a suggestion to make the keys better, \href{https://forms.gle/CZkbZmPbC9XALEE88}{please fill out the short survey here}.}

\textit{Note: This key is auto-generated and may contain issues and/or errors. The keys are reviewed after each exam to ensure grading is done accurately. If there are issues (like duplicate options), they are noted in the offline gradebook. The keys are a work-in-progress to give students as many resources to improve as possible.}

\rule{\textwidth}{0.4pt}

\begin{enumerate}\litem{
Factor the polynomial below completely, knowing that $x + 3$ is a factor. Then, choose the intervals the zeros of the polynomial belong to, where $z_1 \leq z_2 \leq z_3 \leq z_4$. \textit{To make the problem easier, all zeros are between -5 and 5.}
\[ f(x) = 16x^{4} +16 x^{3} -105 x^{2} -9 x + 54 \]The solution is \( [-3, -0.75, 0.75, 2] \), which is option C.\begin{enumerate}[label=\Alph*.]
\item \( z_1 \in [-2.2, 0.1], \text{   }  z_2 \in [-1.19, 0.23], z_3 \in [0.4, 0.94], \text{   and   } z_4 \in [2.36, 3.36] \)

 Distractor 1: Corresponds to negatives of all zeros.
\item \( z_1 \in [-3.3, -2.7], \text{   }  z_2 \in [-2.54, -1.67], z_3 \in [0.02, 0.66], \text{   and   } z_4 \in [2.36, 3.36] \)

 Distractor 4: Corresponds to moving factors from one rational to another.
\item \( z_1 \in [-3.3, -2.7], \text{   }  z_2 \in [-1.19, 0.23], z_3 \in [0.4, 0.94], \text{   and   } z_4 \in [1.68, 2.69] \)

* This is the solution!
\item \( z_1 \in [-3.3, -2.7], \text{   }  z_2 \in [-1.63, -0.84], z_3 \in [1.24, 1.45], \text{   and   } z_4 \in [1.68, 2.69] \)

 Distractor 2: Corresponds to inversing rational roots.
\item \( z_1 \in [-2.2, 0.1], \text{   }  z_2 \in [-1.63, -0.84], z_3 \in [1.24, 1.45], \text{   and   } z_4 \in [2.36, 3.36] \)

 Distractor 3: Corresponds to negatives of all zeros AND inversing rational roots.
\end{enumerate}

\textbf{General Comment:} Remember to try the middle-most integers first as these normally are the zeros. Also, once you get it to a quadratic, you can use your other factoring techniques to finish factoring.
}
\litem{
Perform the division below. Then, find the intervals that correspond to the quotient in the form $ax^2+bx+c$ and remainder $r$.
\[ \frac{15x^{3} -97 x^{2} +168 x -77}{x -4} \]The solution is \( 15x^{2} -37 x + 20 + \frac{3}{x -4} \), which is option C.\begin{enumerate}[label=\Alph*.]
\item \( a \in [57, 68], \text{   } b \in [-337, -332], \text{   } c \in [1515, 1519], \text{   and   } r \in [-6142, -6136]. \)

 You divided by the opposite of the factor AND multiplied the first factor rather than just bringing it down.
\item \( a \in [13, 16], \text{   } b \in [-163, -155], \text{   } c \in [796, 797], \text{   and   } r \in [-3269, -3257]. \)

 You divided by the opposite of the factor.
\item \( a \in [13, 16], \text{   } b \in [-37, -35], \text{   } c \in [18, 23], \text{   and   } r \in [3, 4]. \)

* This is the solution!
\item \( a \in [13, 16], \text{   } b \in [-53, -47], \text{   } c \in [10, 17], \text{   and   } r \in [-45, -37]. \)

 You multiplied by the synthetic number and subtracted rather than adding during synthetic division.
\item \( a \in [57, 68], \text{   } b \in [142, 144], \text{   } c \in [737, 743], \text{   and   } r \in [2883, 2885]. \)

 You multiplied by the synthetic number rather than bringing the first factor down.
\end{enumerate}

\textbf{General Comment:} Be sure to synthetically divide by the zero of the denominator!
}
\litem{
Factor the polynomial below completely. Then, choose the intervals the zeros of the polynomial belong to, where $z_1 \leq z_2 \leq z_3$. \textit{To make the problem easier, all zeros are between -5 and 5.}
\[ f(x) = 25x^{3} +75 x^{2} +56 x + 12 \]The solution is \( [-2, -0.6, -0.4] \), which is option E.\begin{enumerate}[label=\Alph*.]
\item \( z_1 \in [-0.2, 0.2], \text{   }  z_2 \in [0.9, 2.6], \text{   and   } z_3 \in [2.78, 3.85] \)

 Distractor 4: Corresponds to moving factors from one rational to another.
\item \( z_1 \in [1.49, 1.78], \text{   }  z_2 \in [0.9, 2.6], \text{   and   } z_3 \in [2.36, 2.61] \)

 Distractor 3: Corresponds to negatives of all zeros AND inversing rational roots.
\item \( z_1 \in [-2.59, -2.42], \text{   }  z_2 \in [-3.5, -1.7], \text{   and   } z_3 \in [-2.5, -1.58] \)

 Distractor 2: Corresponds to inversing rational roots.
\item \( z_1 \in [0.33, 0.79], \text{   }  z_2 \in [0, 1.3], \text{   and   } z_3 \in [1.92, 2.46] \)

 Distractor 1: Corresponds to negatives of all zeros.
\item \( z_1 \in [-2.08, -1.76], \text{   }  z_2 \in [-0.8, 0.1], \text{   and   } z_3 \in [-0.43, -0.12] \)

* This is the solution!
\end{enumerate}

\textbf{General Comment:} Remember to try the middle-most integers first as these normally are the zeros. Also, once you get it to a quadratic, you can use your other factoring techniques to finish factoring.
}
\litem{
What are the \textit{possible Integer} roots of the polynomial below?
\[ f(x) = 6x^{4} +3 x^{3} +2 x^{2} +2 x + 4 \]The solution is \( \pm 1,\pm 2,\pm 4 \), which is option D.\begin{enumerate}[label=\Alph*.]
\item \( \text{ All combinations of: }\frac{\pm 1,\pm 2,\pm 4}{\pm 1,\pm 2,\pm 3,\pm 6} \)

This would have been the solution \textbf{if asked for the possible Rational roots}!
\item \( \text{ All combinations of: }\frac{\pm 1,\pm 2,\pm 3,\pm 6}{\pm 1,\pm 2,\pm 4} \)

 Distractor 3: Corresponds to the plus or minus of the inverse quotient (an/a0) of the factors. 
\item \( \pm 1,\pm 2,\pm 3,\pm 6 \)

 Distractor 1: Corresponds to the plus or minus factors of a1 only.
\item \( \pm 1,\pm 2,\pm 4 \)

* This is the solution \textbf{since we asked for the possible Integer roots}!
\item \( \text{There is no formula or theorem that tells us all possible Integer roots.} \)

 Distractor 4: Corresponds to not recognizing Integers as a subset of Rationals.
\end{enumerate}

\textbf{General Comment:} We have a way to find the possible Rational roots. The possible Integer roots are the Integers in this list.
}
\litem{
Factor the polynomial below completely. Then, choose the intervals the zeros of the polynomial belong to, where $z_1 \leq z_2 \leq z_3$. \textit{To make the problem easier, all zeros are between -5 and 5.}
\[ f(x) = 15x^{3} +56 x^{2} -105 x -50 \]The solution is \( [-5, -0.4, 1.67] \), which is option A.\begin{enumerate}[label=\Alph*.]
\item \( z_1 \in [-6, -4], \text{   }  z_2 \in [-0.81, -0.19], \text{   and   } z_3 \in [0.8, 1.8] \)

* This is the solution!
\item \( z_1 \in [-1.67, -0.67], \text{   }  z_2 \in [0.19, 0.57], \text{   and   } z_3 \in [4.9, 5.8] \)

 Distractor 1: Corresponds to negatives of all zeros.
\item \( z_1 \in [-6, -4], \text{   }  z_2 \in [-2.86, -2.03], \text{   and   } z_3 \in [-0.1, 1.3] \)

 Distractor 2: Corresponds to inversing rational roots.
\item \( z_1 \in [-1.6, 1.4], \text{   }  z_2 \in [2.21, 2.72], \text{   and   } z_3 \in [4.9, 5.8] \)

 Distractor 3: Corresponds to negatives of all zeros AND inversing rational roots.
\item \( z_1 \in [-6, -4], \text{   }  z_2 \in [-0.06, 0.29], \text{   and   } z_3 \in [4.9, 5.8] \)

 Distractor 4: Corresponds to moving factors from one rational to another.
\end{enumerate}

\textbf{General Comment:} Remember to try the middle-most integers first as these normally are the zeros. Also, once you get it to a quadratic, you can use your other factoring techniques to finish factoring.
}
\litem{
Perform the division below. Then, find the intervals that correspond to the quotient in the form $ax^2+bx+c$ and remainder $r$.
\[ \frac{4x^{3} -48 x + 66}{x + 4} \]The solution is \( 4x^{2} -16 x + 16 + \frac{2}{x + 4} \), which is option D.\begin{enumerate}[label=\Alph*.]
\item \( a \in [-20, -15], b \in [61, 73], c \in [-308, -302], \text{ and } r \in [1278, 1285]. \)

 You multipled by the synthetic number rather than bringing the first factor down.
\item \( a \in [4, 7], b \in [16, 17], c \in [14, 22], \text{ and } r \in [129, 137]. \)

 You divided by the opposite of the factor.
\item \( a \in [-20, -15], b \in [-68, -57], c \in [-308, -302], \text{ and } r \in [-1157, -1149]. \)

 You divided by the opposite of the factor AND multipled the first factor rather than just bringing it down.
\item \( a \in [4, 7], b \in [-16, -10], c \in [14, 22], \text{ and } r \in [2, 5]. \)

* This is the solution!
\item \( a \in [4, 7], b \in [-27, -17], c \in [50, 55], \text{ and } r \in [-197, -190]. \)

 You multipled by the synthetic number and subtracted rather than adding during synthetic division.
\end{enumerate}

\textbf{General Comment:} Be sure to synthetically divide by the zero of the denominator! Also, make sure to include 0 placeholders for missing terms.
}
\litem{
Factor the polynomial below completely, knowing that $x + 2$ is a factor. Then, choose the intervals the zeros of the polynomial belong to, where $z_1 \leq z_2 \leq z_3 \leq z_4$. \textit{To make the problem easier, all zeros are between -5 and 5.}
\[ f(x) = 12x^{4} -43 x^{3} -21 x^{2} +166 x -120 \]The solution is \( [-2, 1.25, 1.333, 3] \), which is option E.\begin{enumerate}[label=\Alph*.]
\item \( z_1 \in [-3.47, -2.46], \text{   }  z_2 \in [-1.71, -1], z_3 \in [-1.56, -1.13], \text{   and   } z_4 \in [1.5, 2.6] \)

 Distractor 1: Corresponds to negatives of all zeros.
\item \( z_1 \in [-2.22, -1.47], \text{   }  z_2 \in [0.11, 1.16], z_3 \in [0.44, 1.05], \text{   and   } z_4 \in [2.7, 3.2] \)

 Distractor 2: Corresponds to inversing rational roots.
\item \( z_1 \in [-4.53, -3.35], \text{   }  z_2 \in [-3.93, -2.74], z_3 \in [-0.56, -0.3], \text{   and   } z_4 \in [1.5, 2.6] \)

 Distractor 4: Corresponds to moving factors from one rational to another.
\item \( z_1 \in [-3.47, -2.46], \text{   }  z_2 \in [-1.16, -0.7], z_3 \in [-0.93, -0.52], \text{   and   } z_4 \in [1.5, 2.6] \)

 Distractor 3: Corresponds to negatives of all zeros AND inversing rational roots.
\item \( z_1 \in [-2.22, -1.47], \text{   }  z_2 \in [1.05, 1.59], z_3 \in [1.32, 1.66], \text{   and   } z_4 \in [2.7, 3.2] \)

* This is the solution!
\end{enumerate}

\textbf{General Comment:} Remember to try the middle-most integers first as these normally are the zeros. Also, once you get it to a quadratic, you can use your other factoring techniques to finish factoring.
}
\litem{
What are the \textit{possible Rational} roots of the polynomial below?
\[ f(x) = 4x^{3} +3 x^{2} +7 x + 6 \]The solution is \( \text{ All combinations of: }\frac{\pm 1,\pm 2,\pm 3,\pm 6}{\pm 1,\pm 2,\pm 4} \), which is option D.\begin{enumerate}[label=\Alph*.]
\item \( \pm 1,\pm 2,\pm 4 \)

 Distractor 1: Corresponds to the plus or minus factors of a1 only.
\item \( \text{ All combinations of: }\frac{\pm 1,\pm 2,\pm 4}{\pm 1,\pm 2,\pm 3,\pm 6} \)

 Distractor 3: Corresponds to the plus or minus of the inverse quotient (an/a0) of the factors. 
\item \( \pm 1,\pm 2,\pm 3,\pm 6 \)

This would have been the solution \textbf{if asked for the possible Integer roots}!
\item \( \text{ All combinations of: }\frac{\pm 1,\pm 2,\pm 3,\pm 6}{\pm 1,\pm 2,\pm 4} \)

* This is the solution \textbf{since we asked for the possible Rational roots}!
\item \( \text{ There is no formula or theorem that tells us all possible Rational roots.} \)

 Distractor 4: Corresponds to not recalling the theorem for rational roots of a polynomial.
\end{enumerate}

\textbf{General Comment:} We have a way to find the possible Rational roots. The possible Integer roots are the Integers in this list.
}
\litem{
Perform the division below. Then, find the intervals that correspond to the quotient in the form $ax^2+bx+c$ and remainder $r$.
\[ \frac{6x^{3} +4 x^{2} -34 x + 28}{x + 3} \]The solution is \( 6x^{2} -14 x + 8 + \frac{4}{x + 3} \), which is option C.\begin{enumerate}[label=\Alph*.]
\item \( a \in [-20, -9], \text{   } b \in [56, 65], \text{   } c \in [-213, -206], \text{   and   } r \in [649, 653]. \)

 You multiplied by the synthetic number rather than bringing the first factor down.
\item \( a \in [6, 10], \text{   } b \in [-20, -19], \text{   } c \in [43, 47], \text{   and   } r \in [-158, -148]. \)

 You multiplied by the synthetic number and subtracted rather than adding during synthetic division.
\item \( a \in [6, 10], \text{   } b \in [-17, -8], \text{   } c \in [2, 9], \text{   and   } r \in [-1, 6]. \)

* This is the solution!
\item \( a \in [6, 10], \text{   } b \in [18, 26], \text{   } c \in [32, 36], \text{   and   } r \in [119, 127]. \)

 You divided by the opposite of the factor.
\item \( a \in [-20, -9], \text{   } b \in [-51, -46], \text{   } c \in [-186, -177], \text{   and   } r \in [-526, -517]. \)

 You divided by the opposite of the factor AND multiplied the first factor rather than just bringing it down.
\end{enumerate}

\textbf{General Comment:} Be sure to synthetically divide by the zero of the denominator!
}
\litem{
Perform the division below. Then, find the intervals that correspond to the quotient in the form $ax^2+bx+c$ and remainder $r$.
\[ \frac{20x^{3} +63 x^{2} -24}{x + 3} \]The solution is \( 20x^{2} +3 x -9 + \frac{3}{x + 3} \), which is option B.\begin{enumerate}[label=\Alph*.]
\item \( a \in [-67, -57], b \in [239, 245], c \in [-730, -721], \text{ and } r \in [2163, 2165]. \)

 You multipled by the synthetic number rather than bringing the first factor down.
\item \( a \in [20, 26], b \in [3, 7], c \in [-10, -7], \text{ and } r \in [0, 11]. \)

* This is the solution!
\item \( a \in [20, 26], b \in [-17, -14], c \in [62, 70], \text{ and } r \in [-303, -294]. \)

 You multipled by the synthetic number and subtracted rather than adding during synthetic division.
\item \( a \in [-67, -57], b \in [-119, -114], c \in [-354, -349], \text{ and } r \in [-1085, -1073]. \)

 You divided by the opposite of the factor AND multipled the first factor rather than just bringing it down.
\item \( a \in [20, 26], b \in [119, 127], c \in [364, 371], \text{ and } r \in [1080, 1086]. \)

 You divided by the opposite of the factor.
\end{enumerate}

\textbf{General Comment:} Be sure to synthetically divide by the zero of the denominator! Also, make sure to include 0 placeholders for missing terms.
}
\end{enumerate}

\end{document}