\documentclass{extbook}[14pt]
\usepackage{multicol, enumerate, enumitem, hyperref, color, soul, setspace, parskip, fancyhdr, amssymb, amsthm, amsmath, latexsym, units, mathtools}
\everymath{\displaystyle}
\usepackage[headsep=0.5cm,headheight=0cm, left=1 in,right= 1 in,top= 1 in,bottom= 1 in]{geometry}
\usepackage{dashrule}  % Package to use the command below to create lines between items
\newcommand{\litem}[1]{\item #1

\rule{\textwidth}{0.4pt}}
\pagestyle{fancy}
\lhead{}
\chead{Answer Key for Progress Quiz 4 Version ALL}
\rhead{}
\lfoot{5346-5907}
\cfoot{}
\rfoot{Summer C 2021}
\begin{document}
\textbf{This key should allow you to understand why you choose the option you did (beyond just getting a question right or wrong). \href{https://xronos.clas.ufl.edu/mac1105spring2020/courseDescriptionAndMisc/Exams/LearningFromResults}{More instructions on how to use this key can be found here}.}

\textbf{If you have a suggestion to make the keys better, \href{https://forms.gle/CZkbZmPbC9XALEE88}{please fill out the short survey here}.}

\textit{Note: This key is auto-generated and may contain issues and/or errors. The keys are reviewed after each exam to ensure grading is done accurately. If there are issues (like duplicate options), they are noted in the offline gradebook. The keys are a work-in-progress to give students as many resources to improve as possible.}

\rule{\textwidth}{0.4pt}

\begin{enumerate}\litem{
Determine whether the function below is 1-1.
\[ f(x) = (3 x - 21)^3 \]The solution is \( \text{yes} \), which is option B.\begin{enumerate}[label=\Alph*.]
\item \( \text{No, because there is a $y$-value that goes to 2 different $x$-values.} \)

Corresponds to the Horizontal Line test, which this function passes.
\item \( \text{Yes, the function is 1-1.} \)

* This is the solution.
\item \( \text{No, because there is an $x$-value that goes to 2 different $y$-values.} \)

Corresponds to the Vertical Line test, which checks if an expression is a function.
\item \( \text{No, because the range of the function is not $(-\infty, \infty)$.} \)

Corresponds to believing 1-1 means the range is all Real numbers.
\item \( \text{No, because the domain of the function is not $(-\infty, \infty)$.} \)

Corresponds to believing 1-1 means the domain is all Real numbers.
\end{enumerate}

\textbf{General Comment:} There are only two valid options: The function is 1-1 OR No because there is a $y$-value that goes to 2 different $x$-values.
}
\litem{
Find the inverse of the function below. Then, evaluate the inverse at $x = 8$ and choose the interval that $f^-1(8)$ belongs to.
\[ f(x) = e^{x+3}-4 \]The solution is \( f^{-1}(8) = -0.515 \), which is option D.\begin{enumerate}[label=\Alph*.]
\item \( f^{-1}(8) \in [-2, -1.42] \)

 This solution corresponds to distractor 4.
\item \( f^{-1}(8) \in [-2.82, -2.48] \)

 This solution corresponds to distractor 2.
\item \( f^{-1}(8) \in [5.43, 5.73] \)

 This solution corresponds to distractor 1.
\item \( f^{-1}(8) \in [-0.6, -0.48] \)

 This is the solution.
\item \( f^{-1}(8) \in [-2.56, -2.07] \)

 This solution corresponds to distractor 3.
\end{enumerate}

\textbf{General Comment:} Natural log and exponential functions always have an inverse. Once you switch the $x$ and $y$, use the conversion $ e^y = x \leftrightarrow y=\ln(x)$.
}
\litem{
Add the following functions, then choose the domain of the resulting function from the list below.
\[ f(x) = \sqrt{4x-25}  \text{ and } g(x) = 3x^{2} + 5 \]The solution is \( \text{ The domain is all Real numbers greater than or equal to} x = 6.25. \), which is option A.\begin{enumerate}[label=\Alph*.]
\item \( \text{ The domain is all Real numbers greater than or equal to } x = a, \text{ where } a \in [4.25, 9.25] \)


\item \( \text{ The domain is all Real numbers except } x = a, \text{ where } a \in [0.25, 7.25] \)


\item \( \text{ The domain is all Real numbers less than or equal to } x = a, \text{ where } a \in [-8.8, -0.8] \)


\item \( \text{ The domain is all Real numbers except } x = a \text{ and } x = b, \text{ where } a \in [-6.4, -1.4] \text{ and } b \in [3.33, 10.33] \)


\item \( \text{ The domain is all Real numbers. } \)


\end{enumerate}

\textbf{General Comment:} The new domain is the intersection of the previous domains.
}
\litem{
Find the inverse of the function below (if it exists). Then, evaluate the inverse at $x = -15$ and choose the interval that $f^-1(-15)$ belongs to.
\[ f(x) = 5 x^2 - 3 \]The solution is \( \text{ The function is not invertible for all Real numbers. } \), which is option E.\begin{enumerate}[label=\Alph*.]
\item \( f^{-1}(-15) \in [2.29, 3.07] \)

 Distractor 3: This corresponds to finding the (nonexistent) inverse and dividing by a negative.
\item \( f^{-1}(-15) \in [4.54, 5.58] \)

 Distractor 4: This corresponds to both distractors 2 and 3.
\item \( f^{-1}(-15) \in [0.83, 1.66] \)

 Distractor 1: This corresponds to trying to find the inverse even though the function is not 1-1. 
\item \( f^{-1}(-15) \in [1.78, 1.96] \)

 Distractor 2: This corresponds to finding the (nonexistent) inverse and not subtracting by the vertical shift.
\item \( \text{ The function is not invertible for all Real numbers. } \)

* This is the correct option.
\end{enumerate}

\textbf{General Comment:} Be sure you check that the function is 1-1 before trying to find the inverse!
}
\litem{
Find the inverse of the function below. Then, evaluate the inverse at $x = 6$ and choose the interval that $f^-1(6)$ belongs to.
\[ f(x) = \ln{(x-4)}+3 \]The solution is \( f^{-1}(6) = 24.086 \), which is option D.\begin{enumerate}[label=\Alph*.]
\item \( f^{-1}(6) \in [9.39, 11.39] \)

 This solution corresponds to distractor 4.
\item \( f^{-1}(6) \in [22028.47, 22031.47] \)

 This solution corresponds to distractor 2.
\item \( f^{-1}(6) \in [11.09, 17.09] \)

 This solution corresponds to distractor 3.
\item \( f^{-1}(6) \in [23.09, 30.09] \)

 This is the solution.
\item \( f^{-1}(6) \in [8102.08, 8111.08] \)

 This solution corresponds to distractor 1.
\end{enumerate}

\textbf{General Comment:} Natural log and exponential functions always have an inverse. Once you switch the $x$ and $y$, use the conversion $ e^y = x \leftrightarrow y=\ln(x)$.
}
\litem{
Determine whether the function below is 1-1.
\[ f(x) = (5 x - 31)^3 \]The solution is \( \text{yes} \), which is option A.\begin{enumerate}[label=\Alph*.]
\item \( \text{Yes, the function is 1-1.} \)

* This is the solution.
\item \( \text{No, because there is an $x$-value that goes to 2 different $y$-values.} \)

Corresponds to the Vertical Line test, which checks if an expression is a function.
\item \( \text{No, because there is a $y$-value that goes to 2 different $x$-values.} \)

Corresponds to the Horizontal Line test, which this function passes.
\item \( \text{No, because the range of the function is not $(-\infty, \infty)$.} \)

Corresponds to believing 1-1 means the range is all Real numbers.
\item \( \text{No, because the domain of the function is not $(-\infty, \infty)$.} \)

Corresponds to believing 1-1 means the domain is all Real numbers.
\end{enumerate}

\textbf{General Comment:} There are only two valid options: The function is 1-1 OR No because there is a $y$-value that goes to 2 different $x$-values.
}
\litem{
Find the inverse of the function below (if it exists). Then, evaluate the inverse at $x = 10$ and choose the interval that $f^-1(10)$ belongs to.
\[ f(x) = 4 x^2 - 5 \]The solution is \( \text{ The function is not invertible for all Real numbers. } \), which is option E.\begin{enumerate}[label=\Alph*.]
\item \( f^{-1}(10) \in [0.75, 1.29] \)

 Distractor 2: This corresponds to finding the (nonexistent) inverse and not subtracting by the vertical shift.
\item \( f^{-1}(10) \in [1.22, 2.45] \)

 Distractor 1: This corresponds to trying to find the inverse even though the function is not 1-1. 
\item \( f^{-1}(10) \in [2.14, 4.12] \)

 Distractor 3: This corresponds to finding the (nonexistent) inverse and dividing by a negative.
\item \( f^{-1}(10) \in [6.06, 7.1] \)

 Distractor 4: This corresponds to both distractors 2 and 3.
\item \( \text{ The function is not invertible for all Real numbers. } \)

* This is the correct option.
\end{enumerate}

\textbf{General Comment:} Be sure you check that the function is 1-1 before trying to find the inverse!
}
\litem{
Subtract the following functions, then choose the domain of the resulting function from the list below.
\[ f(x) = \sqrt{5x+26}  \text{ and } g(x) = 8x^{2} +6 x \]The solution is \( \text{ The domain is all Real numbers greater than or equal to} x = -5.2. \), which is option A.\begin{enumerate}[label=\Alph*.]
\item \( \text{ The domain is all Real numbers greater than or equal to } x = a, \text{ where } a \in [-6.2, -3.2] \)


\item \( \text{ The domain is all Real numbers except } x = a, \text{ where } a \in [1.17, 9.17] \)


\item \( \text{ The domain is all Real numbers less than or equal to } x = a, \text{ where } a \in [-1.8, 3.2] \)


\item \( \text{ The domain is all Real numbers except } x = a \text{ and } x = b, \text{ where } a \in [5.33, 6.33] \text{ and } b \in [-8.6, -5.6] \)


\item \( \text{ The domain is all Real numbers. } \)


\end{enumerate}

\textbf{General Comment:} The new domain is the intersection of the previous domains.
}
\litem{
Choose the interval below that $f$ composed with $g$ at $x=-1$ is in.
\[ f(x) = -x^{3} -3 x^{2} -2 x -4 \text{ and } g(x) = x^{3} +4 x^{2} +3 x -3 \]The solution is \( 2.0 \), which is option A.\begin{enumerate}[label=\Alph*.]
\item \( (f \circ g)(-1) \in [2, 3] \)

* This is the correct solution
\item \( (f \circ g)(-1) \in [-7, -4] \)

 Distractor 2: Corresponds to being slightly off from the solution.
\item \( (f \circ g)(-1) \in [-20, -13] \)

 Distractor 1: Corresponds to reversing the composition.
\item \( (f \circ g)(-1) \in [-30, -17] \)

 Distractor 3: Corresponds to being slightly off from the solution.
\item \( \text{It is not possible to compose the two functions.} \)


\end{enumerate}

\textbf{General Comment:} $f$ composed with $g$ at $x$ means $f(g(x))$. The order matters!
}
\litem{
Choose the interval below that $f$ composed with $g$ at $x=1$ is in.
\[ f(x) = -x^{3} +3 x^{2} +4 x -1 \text{ and } g(x) = -x^{3} +2 x^{2} -x + 2 \]The solution is \( 11.0 \), which is option C.\begin{enumerate}[label=\Alph*.]
\item \( (f \circ g)(1) \in [-83, -72] \)

 Distractor 1: Corresponds to reversing the composition.
\item \( (f \circ g)(1) \in [-90, -82] \)

 Distractor 3: Corresponds to being slightly off from the solution.
\item \( (f \circ g)(1) \in [10, 18] \)

* This is the correct solution
\item \( (f \circ g)(1) \in [-4, 6] \)

 Distractor 2: Corresponds to being slightly off from the solution.
\item \( \text{It is not possible to compose the two functions.} \)


\end{enumerate}

\textbf{General Comment:} $f$ composed with $g$ at $x$ means $f(g(x))$. The order matters!
}
\litem{
Determine whether the function below is 1-1.
\[ f(x) = -24 x^2 + 4 x + 580 \]The solution is \( \text{no} \), which is option A.\begin{enumerate}[label=\Alph*.]
\item \( \text{No, because there is a $y$-value that goes to 2 different $x$-values.} \)

* This is the solution.
\item \( \text{No, because there is an $x$-value that goes to 2 different $y$-values.} \)

Corresponds to the Vertical Line test, which checks if an expression is a function.
\item \( \text{Yes, the function is 1-1.} \)

Corresponds to believing the function passes the Horizontal Line test.
\item \( \text{No, because the range of the function is not $(-\infty, \infty)$.} \)

Corresponds to believing 1-1 means the range is all Real numbers.
\item \( \text{No, because the domain of the function is not $(-\infty, \infty)$.} \)

Corresponds to believing 1-1 means the domain is all Real numbers.
\end{enumerate}

\textbf{General Comment:} There are only two valid options: The function is 1-1 OR No because there is a $y$-value that goes to 2 different $x$-values.
}
\litem{
Find the inverse of the function below. Then, evaluate the inverse at $x = 7$ and choose the interval that $f^-1(7)$ belongs to.
\[ f(x) = e^{x-4}-2 \]The solution is \( f^{-1}(7) = 6.197 \), which is option E.\begin{enumerate}[label=\Alph*.]
\item \( f^{-1}(7) \in [-1.91, -1.76] \)

 This solution corresponds to distractor 1.
\item \( f^{-1}(7) \in [-0.95, -0.87] \)

 This solution corresponds to distractor 4.
\item \( f^{-1}(7) \in [-0.02, 0.51] \)

 This solution corresponds to distractor 3.
\item \( f^{-1}(7) \in [-0.52, -0.32] \)

 This solution corresponds to distractor 2.
\item \( f^{-1}(7) \in [5.63, 6.64] \)

 This is the solution.
\end{enumerate}

\textbf{General Comment:} Natural log and exponential functions always have an inverse. Once you switch the $x$ and $y$, use the conversion $ e^y = x \leftrightarrow y=\ln(x)$.
}
\litem{
Multiply the following functions, then choose the domain of the resulting function from the list below.
\[ f(x) = 2x^{2} +7 x + 8 \text{ and } g(x) = \sqrt{4x-17}  \]The solution is \( \text{ The domain is all Real numbers greater than or equal to} x = 4.25. \), which is option A.\begin{enumerate}[label=\Alph*.]
\item \( \text{ The domain is all Real numbers greater than or equal to } x = a, \text{ where } a \in [3.25, 7.25] \)


\item \( \text{ The domain is all Real numbers except } x = a, \text{ where } a \in [-5.75, -1.75] \)


\item \( \text{ The domain is all Real numbers less than or equal to } x = a, \text{ where } a \in [-1.5, 9.5] \)


\item \( \text{ The domain is all Real numbers except } x = a \text{ and } x = b, \text{ where } a \in [3.6, 10.6] \text{ and } b \in [-0.6, 7.4] \)


\item \( \text{ The domain is all Real numbers. } \)


\end{enumerate}

\textbf{General Comment:} The new domain is the intersection of the previous domains.
}
\litem{
Find the inverse of the function below (if it exists). Then, evaluate the inverse at $x = -15$ and choose the interval that $f^-1(-15)$ belongs to.
\[ f(x) = \sqrt[3]{3 x - 2} \]The solution is \( -1124.3333333333333 \), which is option C.\begin{enumerate}[label=\Alph*.]
\item \( f^{-1}(-15) \in [1122.93, 1124.98] \)

 This solution corresponds to distractor 2.
\item \( f^{-1}(-15) \in [-1126.09, -1125.21] \)

 Distractor 1: This corresponds to 
\item \( f^{-1}(-15) \in [-1124.91, -1123.2] \)

* This is the correct solution.
\item \( f^{-1}(-15) \in [1124.54, 1127.79] \)

 This solution corresponds to distractor 3.
\item \( \text{ The function is not invertible for all Real numbers. } \)

 This solution corresponds to distractor 4.
\end{enumerate}

\textbf{General Comment:} Be sure you check that the function is 1-1 before trying to find the inverse!
}
\litem{
Find the inverse of the function below. Then, evaluate the inverse at $x = 6$ and choose the interval that $f^-1(6)$ belongs to.
\[ f(x) = e^{x+4}+2 \]The solution is \( f^{-1}(6) = -2.614 \), which is option E.\begin{enumerate}[label=\Alph*.]
\item \( f^{-1}(6) \in [4.15, 5.18] \)

 This solution corresponds to distractor 4.
\item \( f^{-1}(6) \in [3.83, 4.12] \)

 This solution corresponds to distractor 2.
\item \( f^{-1}(6) \in [5.31, 5.63] \)

 This solution corresponds to distractor 1.
\item \( f^{-1}(6) \in [2.13, 2.72] \)

 This solution corresponds to distractor 3.
\item \( f^{-1}(6) \in [-2.81, -2.57] \)

 This is the solution.
\end{enumerate}

\textbf{General Comment:} Natural log and exponential functions always have an inverse. Once you switch the $x$ and $y$, use the conversion $ e^y = x \leftrightarrow y=\ln(x)$.
}
\litem{
Determine whether the function below is 1-1.
\[ f(x) = (6 x - 35)^3 \]The solution is \( \text{yes} \), which is option A.\begin{enumerate}[label=\Alph*.]
\item \( \text{Yes, the function is 1-1.} \)

* This is the solution.
\item \( \text{No, because there is an $x$-value that goes to 2 different $y$-values.} \)

Corresponds to the Vertical Line test, which checks if an expression is a function.
\item \( \text{No, because there is a $y$-value that goes to 2 different $x$-values.} \)

Corresponds to the Horizontal Line test, which this function passes.
\item \( \text{No, because the range of the function is not $(-\infty, \infty)$.} \)

Corresponds to believing 1-1 means the range is all Real numbers.
\item \( \text{No, because the domain of the function is not $(-\infty, \infty)$.} \)

Corresponds to believing 1-1 means the domain is all Real numbers.
\end{enumerate}

\textbf{General Comment:} There are only two valid options: The function is 1-1 OR No because there is a $y$-value that goes to 2 different $x$-values.
}
\litem{
Find the inverse of the function below (if it exists). Then, evaluate the inverse at $x = -13$ and choose the interval that $f^-1(-13)$ belongs to.
\[ f(x) = \sqrt[3]{3 x - 4} \]The solution is \( -731.0 \), which is option A.\begin{enumerate}[label=\Alph*.]
\item \( f^{-1}(-13) \in [-731.4, -729] \)

* This is the correct solution.
\item \( f^{-1}(-13) \in [729.1, 731.8] \)

 This solution corresponds to distractor 2.
\item \( f^{-1}(-13) \in [-735.7, -733.5] \)

 Distractor 1: This corresponds to 
\item \( f^{-1}(-13) \in [731.6, 734.4] \)

 This solution corresponds to distractor 3.
\item \( \text{ The function is not invertible for all Real numbers. } \)

 This solution corresponds to distractor 4.
\end{enumerate}

\textbf{General Comment:} Be sure you check that the function is 1-1 before trying to find the inverse!
}
\litem{
Multiply the following functions, then choose the domain of the resulting function from the list below.
\[ f(x) = \sqrt{5x+21}  \text{ and } g(x) = 8x + 8 \]The solution is \( \text{ The domain is all Real numbers greater than or equal to} x = -4.2. \), which is option B.\begin{enumerate}[label=\Alph*.]
\item \( \text{ The domain is all Real numbers except } x = a, \text{ where } a \in [-4.25, -2.25] \)


\item \( \text{ The domain is all Real numbers greater than or equal to } x = a, \text{ where } a \in [-10.2, -3.2] \)


\item \( \text{ The domain is all Real numbers less than or equal to } x = a, \text{ where } a \in [-6.5, -2.5] \)


\item \( \text{ The domain is all Real numbers except } x = a \text{ and } x = b, \text{ where } a \in [-7.75, -0.75] \text{ and } b \in [0.25, 8.25] \)


\item \( \text{ The domain is all Real numbers. } \)


\end{enumerate}

\textbf{General Comment:} The new domain is the intersection of the previous domains.
}
\litem{
Choose the interval below that $f$ composed with $g$ at $x=-1$ is in.
\[ f(x) = -x^{3} -2 x^{2} -4 x \text{ and } g(x) = x^{3} -1 x^{2} -4 x \]The solution is \( -24.0 \), which is option D.\begin{enumerate}[label=\Alph*.]
\item \( (f \circ g)(-1) \in [-6, 3] \)

 Distractor 3: Corresponds to being slightly off from the solution.
\item \( (f \circ g)(-1) \in [3, 14] \)

 Distractor 1: Corresponds to reversing the composition.
\item \( (f \circ g)(-1) \in [-21, -16] \)

 Distractor 2: Corresponds to being slightly off from the solution.
\item \( (f \circ g)(-1) \in [-30, -23] \)

* This is the correct solution
\item \( \text{It is not possible to compose the two functions.} \)


\end{enumerate}

\textbf{General Comment:} $f$ composed with $g$ at $x$ means $f(g(x))$. The order matters!
}
\litem{
Choose the interval below that $f$ composed with $g$ at $x=-1$ is in.
\[ f(x) = -x^{3} +3 x^{2} +4 x -3 \text{ and } g(x) = 3x^{3} +4 x^{2} -2 x -4 \]The solution is \( -3.0 \), which is option D.\begin{enumerate}[label=\Alph*.]
\item \( (f \circ g)(-1) \in [-39, -33] \)

 Distractor 3: Corresponds to being slightly off from the solution.
\item \( (f \circ g)(-1) \in [4, 7] \)

 Distractor 2: Corresponds to being slightly off from the solution.
\item \( (f \circ g)(-1) \in [-44, -41] \)

 Distractor 1: Corresponds to reversing the composition.
\item \( (f \circ g)(-1) \in [-3, 0] \)

* This is the correct solution
\item \( \text{It is not possible to compose the two functions.} \)


\end{enumerate}

\textbf{General Comment:} $f$ composed with $g$ at $x$ means $f(g(x))$. The order matters!
}
\litem{
Determine whether the function below is 1-1.
\[ f(x) = (5 x - 20)^3 \]The solution is \( \text{yes} \), which is option C.\begin{enumerate}[label=\Alph*.]
\item \( \text{No, because the domain of the function is not $(-\infty, \infty)$.} \)

Corresponds to believing 1-1 means the domain is all Real numbers.
\item \( \text{No, because the range of the function is not $(-\infty, \infty)$.} \)

Corresponds to believing 1-1 means the range is all Real numbers.
\item \( \text{Yes, the function is 1-1.} \)

* This is the solution.
\item \( \text{No, because there is an $x$-value that goes to 2 different $y$-values.} \)

Corresponds to the Vertical Line test, which checks if an expression is a function.
\item \( \text{No, because there is a $y$-value that goes to 2 different $x$-values.} \)

Corresponds to the Horizontal Line test, which this function passes.
\end{enumerate}

\textbf{General Comment:} There are only two valid options: The function is 1-1 OR No because there is a $y$-value that goes to 2 different $x$-values.
}
\litem{
Find the inverse of the function below. Then, evaluate the inverse at $x = 8$ and choose the interval that $f^-1(8)$ belongs to.
\[ f(x) = e^{x-4}-4 \]The solution is \( f^{-1}(8) = 6.485 \), which is option B.\begin{enumerate}[label=\Alph*.]
\item \( f^{-1}(8) \in [-1.52, -0.52] \)

 This solution corresponds to distractor 1.
\item \( f^{-1}(8) \in [6.48, 8.48] \)

 This is the solution.
\item \( f^{-1}(8) \in [-4.61, -1.61] \)

 This solution corresponds to distractor 4.
\item \( f^{-1}(8) \in [-1.52, -0.52] \)

 This solution corresponds to distractor 3.
\item \( f^{-1}(8) \in [-4.61, -1.61] \)

 This solution corresponds to distractor 2.
\end{enumerate}

\textbf{General Comment:} Natural log and exponential functions always have an inverse. Once you switch the $x$ and $y$, use the conversion $ e^y = x \leftrightarrow y=\ln(x)$.
}
\litem{
Subtract the following functions, then choose the domain of the resulting function from the list below.
\[ f(x) = 7x^{2} +7 x + 9 \text{ and } g(x) = 2x + 8 \]The solution is \( (-\infty, \infty) \), which is option E.\begin{enumerate}[label=\Alph*.]
\item \( \text{ The domain is all Real numbers except } x = a, \text{ where } a \in [-9.17, -0.17] \)


\item \( \text{ The domain is all Real numbers less than or equal to } x = a, \text{ where } a \in [0.4, 7.4] \)


\item \( \text{ The domain is all Real numbers greater than or equal to } x = a, \text{ where } a \in [-0.33, 9.67] \)


\item \( \text{ The domain is all Real numbers except } x = a \text{ and } x = b, \text{ where } a \in [-4.6, 7.4] \text{ and } b \in [-4.75, 0.25] \)


\item \( \text{ The domain is all Real numbers. } \)


\end{enumerate}

\textbf{General Comment:} The new domain is the intersection of the previous domains.
}
\litem{
Find the inverse of the function below (if it exists). Then, evaluate the inverse at $x = 11$ and choose the interval that $f^-1(11)$ belongs to.
\[ f(x) = \sqrt[3]{3 x - 2} \]The solution is \( 444.3333333333333 \), which is option C.\begin{enumerate}[label=\Alph*.]
\item \( f^{-1}(11) \in [-443.95, -442.28] \)

 This solution corresponds to distractor 3.
\item \( f^{-1}(11) \in [441.04, 444.2] \)

 Distractor 1: This corresponds to 
\item \( f^{-1}(11) \in [444.01, 444.69] \)

* This is the correct solution.
\item \( f^{-1}(11) \in [-445.64, -444.03] \)

 This solution corresponds to distractor 2.
\item \( \text{ The function is not invertible for all Real numbers. } \)

 This solution corresponds to distractor 4.
\end{enumerate}

\textbf{General Comment:} Be sure you check that the function is 1-1 before trying to find the inverse!
}
\litem{
Find the inverse of the function below. Then, evaluate the inverse at $x = 10$ and choose the interval that $f^-1(10)$ belongs to.
\[ f(x) = \ln{(x-5)}+3 \]The solution is \( f^{-1}(10) = 1101.633 \), which is option E.\begin{enumerate}[label=\Alph*.]
\item \( f^{-1}(10) \in [3269019.37, 3269022.37] \)

 This solution corresponds to distractor 2.
\item \( f^{-1}(10) \in [1084.63, 1094.63] \)

 This solution corresponds to distractor 3.
\item \( f^{-1}(10) \in [442415.39, 442424.39] \)

 This solution corresponds to distractor 1.
\item \( f^{-1}(10) \in [147.41, 155.41] \)

 This solution corresponds to distractor 4.
\item \( f^{-1}(10) \in [1098.63, 1103.63] \)

 This is the solution.
\end{enumerate}

\textbf{General Comment:} Natural log and exponential functions always have an inverse. Once you switch the $x$ and $y$, use the conversion $ e^y = x \leftrightarrow y=\ln(x)$.
}
\litem{
Determine whether the function below is 1-1.
\[ f(x) = (4 x - 29)^3 \]The solution is \( \text{yes} \), which is option A.\begin{enumerate}[label=\Alph*.]
\item \( \text{Yes, the function is 1-1.} \)

* This is the solution.
\item \( \text{No, because there is a $y$-value that goes to 2 different $x$-values.} \)

Corresponds to the Horizontal Line test, which this function passes.
\item \( \text{No, because the range of the function is not $(-\infty, \infty)$.} \)

Corresponds to believing 1-1 means the range is all Real numbers.
\item \( \text{No, because the domain of the function is not $(-\infty, \infty)$.} \)

Corresponds to believing 1-1 means the domain is all Real numbers.
\item \( \text{No, because there is an $x$-value that goes to 2 different $y$-values.} \)

Corresponds to the Vertical Line test, which checks if an expression is a function.
\end{enumerate}

\textbf{General Comment:} There are only two valid options: The function is 1-1 OR No because there is a $y$-value that goes to 2 different $x$-values.
}
\litem{
Find the inverse of the function below (if it exists). Then, evaluate the inverse at $x = -10$ and choose the interval that $f^-1(-10)$ belongs to.
\[ f(x) = 3 x^2 - 4 \]The solution is \( \text{ The function is not invertible for all Real numbers. } \), which is option E.\begin{enumerate}[label=\Alph*.]
\item \( f^{-1}(-10) \in [4.25, 4.48] \)

 Distractor 4: This corresponds to both distractors 2 and 3.
\item \( f^{-1}(-10) \in [2.74, 3.91] \)

 Distractor 3: This corresponds to finding the (nonexistent) inverse and dividing by a negative.
\item \( f^{-1}(-10) \in [2.01, 2.17] \)

 Distractor 2: This corresponds to finding the (nonexistent) inverse and not subtracting by the vertical shift.
\item \( f^{-1}(-10) \in [0.95, 2.12] \)

 Distractor 1: This corresponds to trying to find the inverse even though the function is not 1-1. 
\item \( \text{ The function is not invertible for all Real numbers. } \)

* This is the correct option.
\end{enumerate}

\textbf{General Comment:} Be sure you check that the function is 1-1 before trying to find the inverse!
}
\litem{
Multiply the following functions, then choose the domain of the resulting function from the list below.
\[ f(x) = 8x + 9 \text{ and } g(x) = 2x + 9 \]The solution is \( (-\infty, \infty) \), which is option E.\begin{enumerate}[label=\Alph*.]
\item \( \text{ The domain is all Real numbers greater than or equal to } x = a, \text{ where } a \in [3.25, 11.25] \)


\item \( \text{ The domain is all Real numbers except } x = a, \text{ where } a \in [-8.33, -3.33] \)


\item \( \text{ The domain is all Real numbers less than or equal to } x = a, \text{ where } a \in [1.83, 3.83] \)


\item \( \text{ The domain is all Real numbers except } x = a \text{ and } x = b, \text{ where } a \in [-6.6, -0.6] \text{ and } b \in [4.2, 13.2] \)


\item \( \text{ The domain is all Real numbers. } \)


\end{enumerate}

\textbf{General Comment:} The new domain is the intersection of the previous domains.
}
\litem{
Choose the interval below that $f$ composed with $g$ at $x=1$ is in.
\[ f(x) = 2x^{3} -1 x^{2} +2 x -3 \text{ and } g(x) = x^{3} + x^{2} +3 x -4 \]The solution is \( 0.0 \), which is option A.\begin{enumerate}[label=\Alph*.]
\item \( (f \circ g)(1) \in [-0.8, 2] \)

* This is the correct solution
\item \( (f \circ g)(1) \in [-7.6, -5.6] \)

 Distractor 2: Corresponds to being slightly off from the solution.
\item \( (f \circ g)(1) \in [-4.3, -3.3] \)

 Distractor 1: Corresponds to reversing the composition.
\item \( (f \circ g)(1) \in [-11.7, -9.2] \)

 Distractor 3: Corresponds to being slightly off from the solution.
\item \( \text{It is not possible to compose the two functions.} \)


\end{enumerate}

\textbf{General Comment:} $f$ composed with $g$ at $x$ means $f(g(x))$. The order matters!
}
\litem{
Choose the interval below that $f$ composed with $g$ at $x=1$ is in.
\[ f(x) = 2x^{3} +2 x^{2} -2 x \text{ and } g(x) = -2x^{3} -1 x^{2} +x + 1 \]The solution is \( 2.0 \), which is option B.\begin{enumerate}[label=\Alph*.]
\item \( (f \circ g)(1) \in [-28, -25] \)

 Distractor 3: Corresponds to being slightly off from the solution.
\item \( (f \circ g)(1) \in [2, 8] \)

* This is the correct solution
\item \( (f \circ g)(1) \in [-10, 1] \)

 Distractor 2: Corresponds to being slightly off from the solution.
\item \( (f \circ g)(1) \in [-19, -16] \)

 Distractor 1: Corresponds to reversing the composition.
\item \( \text{It is not possible to compose the two functions.} \)


\end{enumerate}

\textbf{General Comment:} $f$ composed with $g$ at $x$ means $f(g(x))$. The order matters!
}
\end{enumerate}

\end{document}