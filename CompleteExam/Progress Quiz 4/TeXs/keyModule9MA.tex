\documentclass{extbook}[14pt]
\usepackage{multicol, enumerate, enumitem, hyperref, color, soul, setspace, parskip, fancyhdr, amssymb, amsthm, amsmath, latexsym, units, mathtools}
\everymath{\displaystyle}
\usepackage[headsep=0.5cm,headheight=0cm, left=1 in,right= 1 in,top= 1 in,bottom= 1 in]{geometry}
\usepackage{dashrule}  % Package to use the command below to create lines between items
\newcommand{\litem}[1]{\item #1

\rule{\textwidth}{0.4pt}}
\pagestyle{fancy}
\lhead{}
\chead{Answer Key for Progress Quiz 4 Version A}
\rhead{}
\lfoot{5346-5907}
\cfoot{}
\rfoot{Summer C 2021}
\begin{document}
\textbf{This key should allow you to understand why you choose the option you did (beyond just getting a question right or wrong). \href{https://xronos.clas.ufl.edu/mac1105spring2020/courseDescriptionAndMisc/Exams/LearningFromResults}{More instructions on how to use this key can be found here}.}

\textbf{If you have a suggestion to make the keys better, \href{https://forms.gle/CZkbZmPbC9XALEE88}{please fill out the short survey here}.}

\textit{Note: This key is auto-generated and may contain issues and/or errors. The keys are reviewed after each exam to ensure grading is done accurately. If there are issues (like duplicate options), they are noted in the offline gradebook. The keys are a work-in-progress to give students as many resources to improve as possible.}

\rule{\textwidth}{0.4pt}

\begin{enumerate}\litem{
A town has an initial population of 70000. The town's population for the next 9 years is provided below. Which type of function would be most appropriate to model the town's population?


\begin{tabular}{c|c|c|c|c|c|c|c|c|c}
\textbf{Year} &1 &2 &3 &4 &5 &6 &7 &8 &9\tabularnewline \hline
\textbf{Pop} &70017 &70045 &70057 &70085 &70097 &70125 &70137 &70165 &70177\end{tabular}The solution is \( \text{Non-Linear Power} \), which is option A.\begin{enumerate}[label=\Alph*.]
\item \( \text{Non-Linear Power} \)

This suggests a growth faster than constant but slower than exponential.
\item \( \text{Exponential} \)

This suggests the fastest of growths that we know.
\item \( \text{Linear} \)

This suggests a constant growth. You would be able to add or subtract the same amount year-to-year if this is the correct answer.
\item \( \text{Logarithmic} \)

This suggests the slowest of growths that we know.
\item \( \text{None of the above} \)

Please contact the coordinator to discuss why you believe none of the options model the population.
\end{enumerate}

\textbf{General Comment:} We are trying to compare the growth rate of the population. Growth rates can be characterized from slowest to fastest as: logarithmic, indirect, linear, direct, exponential. The best way to approach this is to first compare it to linear (is it linear, faster than linear, or slower than linear)? If faster, is it as fast as exponential? If slower, is it as slow as logarithmic?
}
\litem{
For the information below, construct a linear model that describes the total time $T$ spent on the path in terms of the distance of a particular part of the path \textit{if we know that all parts of the path are equal length}.

\begin{center}
    \textit{ A bicyclist is training for a race on a hilly path. Their bike keeps track of their speed at any time, but not the distance traveled. Their speed traveling up a hill is 2 mph, 9 mph when traveling down a hill, and 5 mph when traveling along a flat portion. }
\end{center}
The solution is \( 0.811 D \), which is option A.\begin{enumerate}[label=\Alph*.]
\item \( 0.811 D \)

* This is the correct option.
\item \( 16.000 D \)

The coefficient here is calculated as if you were trying to model the distance on the total path.
\item \( 90.000 D \)

The coefficient here is calculated by multiplying the distances together rather than adding.
\item \( \text{The model can be found with the information provided, but isn't options 1-3.} \)

Since we know all parts of the path are equal length, we can treat all distance variables as the same variable, $D$.
\item \( \text{The model cannot be found with the information provided.} \)

If you chose this option, please contact the coordinator to discuss why you think we cannot model the situation.
\end{enumerate}

\textbf{General Comment:} Be sure you pay attention to the variable we are writing the model in terms of. To create the model with a single variable, we have to know that variable is the same throughout each path!
}
\litem{
Using the situation below, construct a linear model that describes the cost of the coffee beans $C(h)$ in terms of the weight of the high-quality coffee beans $h$.

\begin{center}
    \textit{ Veronica needs to prepare 230 of blended coffee beans selling for \$6.12 per pound. She has a high-quality bean that sells for \$6.92 a pound and a low-quality bean that sells for \$3.92 a pound. }
\end{center}
The solution is \( C(h) = 3.00 h + 901.60 \), which is option B.\begin{enumerate}[label=\Alph*.]
\item \( C(h) = 6.92 h \)

This models the cost of the high-quality bean only, not the blended beans.
\item \( C(h) = 3.00 h + 901.60 \)

* This is the correct option since the questions asked you to construct the cost model in terms of the weight of the high-quality bean.
\item \( C(h) = -3.00 h + 1591.60 \)

This would be correct if the question asked you to construct the cost model in terms of the weight of the low-quality bean.
\item \( C(h) = 5.42 h \)

This assumes that exactly half of the high- and low- quality beans are mixed to create the blended coffee beans.
\item \( \text{None of the above.} \)

If you chose this option, please talk to the coordinator to discuss why.
\end{enumerate}

\textbf{General Comment:} This is exactly like the chemistry mixture question from the homework! If you are having trouble with this problem, be sure to review the video for building linear models.
}
\litem{
What is the \textbf{best} way to describe the domain of the scenario below?

\begin{center}
    \textit{ Fred is a store manager at Publix. The store normally orders two pallets of water bottles a week and sells 1000 bottles per day. However, a hurricane is coming and Fred expects water bottle sales to increase tenfold for three days, then decrease by half of normal sales for four days. How many more pallets of water bottles should Fred order the week before the hurricane? }
\end{center}
The solution is \( \text{Subset of the Integers} \), which is option E.\begin{enumerate}[label=\Alph*.]
\item \( \text{Subset of the Natural numbers} \)

Recall that the Naturals are the counting numbers: 1, 2, 3, ...
\item \( \text{Proper subset of the Real numbers} \)

This means we have a domain of the Real numbers but need to throw out values based on the context.
\item \( \text{Subset of the Rational numbers} \)

Recall that the Rationals are fractions with Integers in the numerator and denominator.
\item \( \text{There is no restricted domain in this scenario} \)

This means we have a domain of the Real numbers and we don't need to remove any values even in the real-world context.
\item \( \text{Subset of the Integers} \)

Recall that the Integers are the positive and negative counting numbers: ..., -3, -2, -1, 0, 1, 2, 3, ... 
\end{enumerate}

\textbf{General Comment:} We often have to remove values in the domain when working with real-world models.
}
\litem{
A town has an initial population of 50000. The town's population for the next 9 years is provided below. Which type of function would be most appropriate to model the town's population?


\begin{tabular}{c|c|c|c|c|c|c|c|c|c}
\textbf{Year} &1 &2 &3 &4 &5 &6 &7 &8 &9\tabularnewline \hline
\textbf{Pop} &49978 &49966 &49938 &49926 &49898 &49886 &49858 &49846 &49818\end{tabular}The solution is \( \text{Non-Linear Power} \), which is option B.\begin{enumerate}[label=\Alph*.]
\item \( \text{Linear} \)

This suggests a constant growth. You would be able to add or subtract the same amount year-to-year if this is the correct answer.
\item \( \text{Non-Linear Power} \)

This suggests a growth faster than constant but slower than exponential.
\item \( \text{Exponential} \)

This suggests the fastest of growths that we know.
\item \( \text{Logarithmic} \)

This suggests the slowest of growths that we know.
\item \( \text{None of the above} \)

Please contact the coordinator to discuss why you believe none of the options model the population.
\end{enumerate}

\textbf{General Comment:} We are trying to compare the growth rate of the population. Growth rates can be characterized from slowest to fastest as: logarithmic, indirect, linear, direct, exponential. The best way to approach this is to first compare it to linear (is it linear, faster than linear, or slower than linear)? If faster, is it as fast as exponential? If slower, is it as slow as logarithmic?
}
\litem{
For the information below, construct a linear model that describes the total time $T$ spent on the path in terms of the distance of a particular part of the path \textit{if we know that all parts of the path are equal length}.

\begin{center}
    \textit{ A bicyclist is training for a race on a hilly path. Their bike keeps track of their speed at any time, but not the distance traveled. Their speed traveling up a hill is 3 mph, 7 mph when traveling down a hill, and 5 mph when traveling along a flat portion. }
\end{center}
The solution is \( 0.676 D \), which is option C.\begin{enumerate}[label=\Alph*.]
\item \( 15.000 D \)

The coefficient here is calculated as if you were trying to model the distance on the total path.
\item \( 105.000 D \)

The coefficient here is calculated by multiplying the distances together rather than adding.
\item \( 0.676 D \)

* This is the correct option.
\item \( \text{The model can be found with the information provided, but isn't options 1-3.} \)

Since we know all parts of the path are equal length, we can treat all distance variables as the same variable, $D$.
\item \( \text{The model cannot be found with the information provided.} \)

If you chose this option, please contact the coordinator to discuss why you think we cannot model the situation.
\end{enumerate}

\textbf{General Comment:} Be sure you pay attention to the variable we are writing the model in terms of. To create the model with a single variable, we have to know that variable is the same throughout each path!
}
\litem{
Using the situation below, construct a linear model that describes the cost of the coffee beans $C(h)$ in terms of the weight of the low-quality coffee beans $h$.

\begin{center}
    \textit{ Veronica needs to prepare 70 of blended coffee beans selling for \$5.39 per pound. She has a high-quality bean that sells for \$7.08 a pound and a low-quality bean that sells for \$4.76 a pound. }
\end{center}
The solution is \( C(h) = -2.32 h + 495.60 \), which is option A.\begin{enumerate}[label=\Alph*.]
\item \( C(h) = -2.32 h + 495.60 \)

* This is the correct option since the questions asked you to construct the cost model in terms of the weight of the low-quality bean.
\item \( C(h) = 5.92 h \)

This assumes that exactly half of the high- and low- quality beans are mixed to create the blended coffee beans.
\item \( C(h) = 4.76 h \)

This models the cost of the low-quality bean only, not the blended beans.
\item \( C(h) = 2.32 h + 333.20 \)

This would be correct if the question asked you to construct the cost model in terms of the weight of the high-quality bean.
\item \( \text{None of the above.} \)

If you chose this option, please talk to the coordinator to discuss why.
\end{enumerate}

\textbf{General Comment:} This is exactly like the chemistry mixture question from the homework! If you are having trouble with this problem, be sure to review the video for building linear models.
}
\litem{
What is the \textbf{best} way to describe the domain of the scenario below?

\begin{center}
    \textit{ Chemists commonly create a solution by mixing two products of differing concentrations together. A 10\% and 30\% solution can make an acid solution of some value between these, such as a 24\% acid solution. The chemist wants to make differing solution percentages of 7 liters each. }
\end{center}
The solution is \( \text{Proper subset of the Real numbers} \), which is option A.\begin{enumerate}[label=\Alph*.]
\item \( \text{Proper subset of the Real numbers} \)

This means we have a domain of the Real numbers but need to throw out values based on the context.
\item \( \text{Subset of the Natural numbers} \)

Recall that the Naturals are the counting numbers: 1, 2, 3, ...
\item \( \text{Subset of the Rational numbers} \)

Recall that the Rationals are fractions with Integers in the numerator and denominator.
\item \( \text{Subset of the Integers} \)

Recall that the Integers are the positive and negative counting numbers: ..., -3, -2, -1, 0, 1, 2, 3, ... 
\item \( \text{There is no restricted domain in this scenario} \)

This means we have a domain of the Real numbers and we don't need to remove any values even in the real-world context.
\end{enumerate}

\textbf{General Comment:} We often have to remove values in the domain when working with real-world models.
}
\litem{
For the information provided below, construct a linear model that describes her total budget, $B$, as a function of the number of months, $x$ she is at UF.

\begin{center}
    \textit{ Aubrey is a college student going into her first year at UF. She will receive Bright Futures, which covers her tuition plus a \$800 educational expense each year. Before college, Aubrey saved up \$10000. She knows she will need to pay \$1200 in rent a month, \$50 for food a week, and \$56 in other weekly expenses. }
\end{center}
The solution is \( \text{none of the above.} \), which is option E.\begin{enumerate}[label=\Alph*.]
\item \( B(x) = 10800 - 1624 x \)


\item \( B(x) = 800 x + 10000 \)

This treats the educational expense as something you get every month rather than a 1-time payment and is modeling Income, not Budget.
\item \( B(x) = 10800 - 1306 x \)

This treats weekly expenses as month expenses rather than multiplying each weekly expense.
\item \( B(x) = 10000 x + 800 \)

This treats the savings as something you get every month rather than a 1-time payment and is modeling Income, not Budget.
\item \( \text{None of the above.} \)

* This is the correct option as the model should be $B(x) = 1624 - 10800 x$.
\end{enumerate}

\textbf{General Comment:} This is a Costs, Profit, Revenue question! The most common issues here are: (1) not converting the weekly costs to monthly costs, (2) treating the one-time values like savings and educational expense as happening per month, and (3) not checking that your model is for cost, profit [income], or revenue [budget].
}
\litem{
For the information provided below, construct a linear model that describes her total costs, $C$, as a function of the number of months, $x$ she is at UF. 

\begin{center}
    \textit{ Aubrey is a college student going into her first year at UF. She will receive Bright Futures, which covers her tuition plus a \$800 educational expense each year. Before college, Aubrey saved up \$10000. She knows she will need to pay \$1200 in rent a month, \$70 for food a week, and \$40 in other weekly expenses. }
\end{center}
The solution is \( C(x) = 1640 x \), which is option D.\begin{enumerate}[label=\Alph*.]
\item \( C(x) = 1310 x \)

This treats weekly expenses as monthly expenses rather than multiplying each weekly expense by 4.
\item \( C(x) = 1640 \)

This describes the costs as if they are one-time only and not monthly.
\item \( C(x) = 1310 \)

This treats weekly expenses as month expenses rather than multiplying each weekly expense by 4 AND does not account for these expenses per month.
\item \( C(x) = 1640 x \)

* This is the correct option.
\item \( \text{None of the above.} \)

You may have chosen this as you thought you were modeling total income or total budget.
\end{enumerate}

\textbf{General Comment:} This is a Costs, Profit, Revenue question! The most common issues here are: (1) not converting the weekly costs to monthly costs, (2) treating the one-time values like savings and educational expense as happening per month, and (3) not checking that your model is for cost, profit [income], or revenue [budget].
}
\end{enumerate}

\end{document}