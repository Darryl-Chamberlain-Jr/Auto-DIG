\documentclass{extbook}[14pt]
\usepackage{multicol, enumerate, enumitem, hyperref, color, soul, setspace, parskip, fancyhdr, amssymb, amsthm, amsmath, bbm, latexsym, units, mathtools}
\everymath{\displaystyle}
\usepackage[headsep=0.5cm,headheight=0cm, left=1 in,right= 1 in,top= 1 in,bottom= 1 in]{geometry}
\usepackage{dashrule}  % Package to use the command below to create lines between items
\newcommand{\litem}[1]{\item #1

\rule{\textwidth}{0.4pt}}
\pagestyle{fancy}
\lhead{}
\chead{Answer Key for Makeup Progress Quiz 1 Version C}
\rhead{}
\lfoot{6018-3080}
\cfoot{}
\rfoot{Spring 2021}
\begin{document}
\textbf{This key should allow you to understand why you choose the option you did (beyond just getting a question right or wrong). \href{https://xronos.clas.ufl.edu/mac1105spring2020/courseDescriptionAndMisc/Exams/LearningFromResults}{More instructions on how to use this key can be found here}.}

\textbf{If you have a suggestion to make the keys better, \href{https://forms.gle/CZkbZmPbC9XALEE88}{please fill out the short survey here}.}

\textit{Note: This key is auto-generated and may contain issues and/or errors. The keys are reviewed after each exam to ensure grading is done accurately. If there are issues (like duplicate options), they are noted in the offline gradebook. The keys are a work-in-progress to give students as many resources to improve as possible.}

\rule{\textwidth}{0.4pt}

\begin{enumerate}\litem{
Simplify the expression below and choose the interval the simplification is contained within.
\[ 11 - 10 \div 6 * 13 - (8 * 12) \]The solution is \( -106.667 \), which is option A.\begin{enumerate}[label=\Alph*.]
\item \( [-106.67, -102.67] \)

* -106.667, which is the correct option.
\item \( [-225, -220] \)

 -224.000, which corresponds to not distributing a negative correctly.
\item \( [101.87, 112.87] \)

 106.872, which corresponds to not distributing addition and subtraction correctly.
\item \( [-92.13, -82.13] \)

 -85.128, which corresponds to an Order of Operations error: not reading left-to-right for multiplication/division.
\item \( \text{None of the above} \)

 You may have gotten this by making an unanticipated error. If you got a value that is not any of the others, please let the coordinator know so they can help you figure out what happened.
\end{enumerate}

\textbf{General Comment:} While you may remember (or were taught) PEMDAS is done in order, it is actually done as P/E/MD/AS. When we are at MD or AS, we read left to right.
}
\litem{
Simplify the expression below into the form $a+bi$. Then, choose the intervals that $a$ and $b$ belong to.
\[ (-5 + 4 i)(6 + 3 i) \]The solution is \( -42 + 9 i \), which is option E.\begin{enumerate}[label=\Alph*.]
\item \( a \in [-27, -17] \text{ and } b \in [37.2, 40.7] \)

 $-18 + 39 i$, which corresponds to adding a minus sign in the second term.
\item \( a \in [-33, -28] \text{ and } b \in [9.9, 14.8] \)

 $-30 + 12 i$, which corresponds to just multiplying the real terms to get the real part of the solution and the coefficients in the complex terms to get the complex part.
\item \( a \in [-42, -41] \text{ and } b \in [-10.5, -6.4] \)

 $-42 - 9 i$, which corresponds to adding a minus sign in both terms.
\item \( a \in [-27, -17] \text{ and } b \in [-39.8, -38.7] \)

 $-18 - 39 i$, which corresponds to adding a minus sign in the first term.
\item \( a \in [-42, -41] \text{ and } b \in [8.5, 9.7] \)

* $-42 + 9 i$, which is the correct option.
\end{enumerate}

\textbf{General Comment:} You can treat $i$ as a variable and distribute. Just remember that $i^2=-1$, so you can continue to reduce after you distribute.
}
\litem{
Simplify the expression below into the form $a+bi$. Then, choose the intervals that $a$ and $b$ belong to.
\[ \frac{36 - 11 i}{-6 + 3 i} \]The solution is \( -5.53  - 0.93 i \), which is option B.\begin{enumerate}[label=\Alph*.]
\item \( a \in [-5.7, -5.2] \text{ and } b \in [-43, -41.5] \)

 $-5.53  - 42.00 i$, which corresponds to forgetting to multiply the conjugate by the numerator.
\item \( a \in [-5.7, -5.2] \text{ and } b \in [-1.5, 0.5] \)

* $-5.53  - 0.93 i$, which is the correct option.
\item \( a \in [-6.15, -5.75] \text{ and } b \in [-4, -3] \)

 $-6.00  - 3.67 i$, which corresponds to just dividing the first term by the first term and the second by the second.
\item \( a \in [-4.7, -3.55] \text{ and } b \in [2.5, 4.5] \)

 $-4.07  + 3.87 i$, which corresponds to forgetting to multiply the conjugate by the numerator and not computing the conjugate correctly.
\item \( a \in [-249.7, -248.95] \text{ and } b \in [-1.5, 0.5] \)

 $-249.00  - 0.93 i$, which corresponds to forgetting to multiply the conjugate by the numerator and using a plus instead of a minus in the denominator.
\end{enumerate}

\textbf{General Comment:} Multiply the numerator and denominator by the *conjugate* of the denominator, then simplify. For example, if we have $2+3i$, the conjugate is $2-3i$.
}
\litem{
Choose the \textbf{smallest} set of Complex numbers that the number below belongs to.
\[ \frac{20}{-9}+\sqrt{-49}i \]The solution is \( \text{Rational} \), which is option D.\begin{enumerate}[label=\Alph*.]
\item \( \text{Pure Imaginary} \)

This is a Complex number $(a+bi)$ that \textbf{only} has an imaginary part like $2i$.
\item \( \text{Irrational} \)

These cannot be written as a fraction of Integers. Remember: $\pi$ is not an Integer!
\item \( \text{Nonreal Complex} \)

This is a Complex number $(a+bi)$ that is not Real (has $i$ as part of the number).
\item \( \text{Rational} \)

* This is the correct option!
\item \( \text{Not a Complex Number} \)

This is not a number. The only non-Complex number we know is dividing by 0 as this is not a number!
\end{enumerate}

\textbf{General Comment:} Be sure to simplify $i^2 = -1$. This may remove the imaginary portion for your number. If you are having trouble, you may want to look at the \textit{Subgroups of the Real Numbers} section.
}
\litem{
Simplify the expression below into the form $a+bi$. Then, choose the intervals that $a$ and $b$ belong to.
\[ (4 - 10 i)(-9 + 6 i) \]The solution is \( 24 + 114 i \), which is option C.\begin{enumerate}[label=\Alph*.]
\item \( a \in [-100, -94] \text{ and } b \in [-71, -63] \)

 $-96 - 66 i$, which corresponds to adding a minus sign in the first term.
\item \( a \in [19, 30] \text{ and } b \in [-114, -113] \)

 $24 - 114 i$, which corresponds to adding a minus sign in both terms.
\item \( a \in [19, 30] \text{ and } b \in [105, 120] \)

* $24 + 114 i$, which is the correct option.
\item \( a \in [-100, -94] \text{ and } b \in [65, 72] \)

 $-96 + 66 i$, which corresponds to adding a minus sign in the second term.
\item \( a \in [-37, -29] \text{ and } b \in [-63, -57] \)

 $-36 - 60 i$, which corresponds to just multiplying the real terms to get the real part of the solution and the coefficients in the complex terms to get the complex part.
\end{enumerate}

\textbf{General Comment:} You can treat $i$ as a variable and distribute. Just remember that $i^2=-1$, so you can continue to reduce after you distribute.
}
\litem{
Simplify the expression below into the form $a+bi$. Then, choose the intervals that $a$ and $b$ belong to.
\[ \frac{-72 - 33 i}{-7 + 6 i} \]The solution is \( 3.60  + 7.80 i \), which is option B.\begin{enumerate}[label=\Alph*.]
\item \( a \in [3, 4.5] \text{ and } b \in [662, 663.5] \)

 $3.60  + 663.00 i$, which corresponds to forgetting to multiply the conjugate by the numerator.
\item \( a \in [3, 4.5] \text{ and } b \in [6.5, 8.5] \)

* $3.60  + 7.80 i$, which is the correct option.
\item \( a \in [305, 306.5] \text{ and } b \in [6.5, 8.5] \)

 $306.00  + 7.80 i$, which corresponds to forgetting to multiply the conjugate by the numerator and using a plus instead of a minus in the denominator.
\item \( a \in [7.5, 9] \text{ and } b \in [-3, -2] \)

 $8.26  - 2.36 i$, which corresponds to forgetting to multiply the conjugate by the numerator and not computing the conjugate correctly.
\item \( a \in [10, 12] \text{ and } b \in [-7, -4.5] \)

 $10.29  - 5.50 i$, which corresponds to just dividing the first term by the first term and the second by the second.
\end{enumerate}

\textbf{General Comment:} Multiply the numerator and denominator by the *conjugate* of the denominator, then simplify. For example, if we have $2+3i$, the conjugate is $2-3i$.
}
\litem{
Simplify the expression below and choose the interval the simplification is contained within.
\[ 9 - 5 \div 15 * 12 - (18 * 4) \]The solution is \( -67.000 \), which is option D.\begin{enumerate}[label=\Alph*.]
\item \( [-53.2, -48.2] \)

 -52.000, which corresponds to not distributing a negative correctly.
\item \( [80.9, 82] \)

 80.972, which corresponds to not distributing addition and subtraction correctly.
\item \( [-63.8, -62] \)

 -63.028, which corresponds to an Order of Operations error: not reading left-to-right for multiplication/division.
\item \( [-70.4, -64.2] \)

* -67.000, which is the correct option.
\item \( \text{None of the above} \)

 You may have gotten this by making an unanticipated error. If you got a value that is not any of the others, please let the coordinator know so they can help you figure out what happened.
\end{enumerate}

\textbf{General Comment:} While you may remember (or were taught) PEMDAS is done in order, it is actually done as P/E/MD/AS. When we are at MD or AS, we read left to right.
}
\litem{
Choose the \textbf{smallest} set of Real numbers that the number below belongs to.
\[ \sqrt{\frac{93636}{289}} \]The solution is \( \text{Whole} \), which is option A.\begin{enumerate}[label=\Alph*.]
\item \( \text{Whole} \)

* This is the correct option!
\item \( \text{Integer} \)

These are the negative and positive counting numbers (..., -3, -2, -1, 0, 1, 2, 3, ...)
\item \( \text{Not a Real number} \)

These are Nonreal Complex numbers \textbf{OR} things that are not numbers (e.g., dividing by 0).
\item \( \text{Rational} \)

These are numbers that can be written as fraction of Integers (e.g., -2/3)
\item \( \text{Irrational} \)

These cannot be written as a fraction of Integers.
\end{enumerate}

\textbf{General Comment:} First, you \textbf{NEED} to simplify the expression. This question simplifies to $306$. 
 
 Be sure you look at the simplified fraction and not just the decimal expansion. Numbers such as 13, 17, and 19 provide \textbf{long but repeating/terminating decimal expansions!} 
 
 The only ways to *not* be a Real number are: dividing by 0 or taking the square root of a negative number. 
 
 Irrational numbers are more than just square root of 3: adding or subtracting values from square root of 3 is also irrational.
}
\litem{
Choose the \textbf{smallest} set of Real numbers that the number below belongs to.
\[ -\sqrt{\frac{74529}{441}} \]The solution is \( \text{Integer} \), which is option A.\begin{enumerate}[label=\Alph*.]
\item \( \text{Integer} \)

* This is the correct option!
\item \( \text{Whole} \)

These are the counting numbers with 0 (0, 1, 2, 3, ...)
\item \( \text{Irrational} \)

These cannot be written as a fraction of Integers.
\item \( \text{Not a Real number} \)

These are Nonreal Complex numbers \textbf{OR} things that are not numbers (e.g., dividing by 0).
\item \( \text{Rational} \)

These are numbers that can be written as fraction of Integers (e.g., -2/3)
\end{enumerate}

\textbf{General Comment:} First, you \textbf{NEED} to simplify the expression. This question simplifies to $-273$. 
 
 Be sure you look at the simplified fraction and not just the decimal expansion. Numbers such as 13, 17, and 19 provide \textbf{long but repeating/terminating decimal expansions!} 
 
 The only ways to *not* be a Real number are: dividing by 0 or taking the square root of a negative number. 
 
 Irrational numbers are more than just square root of 3: adding or subtracting values from square root of 3 is also irrational.
}
\litem{
Choose the \textbf{smallest} set of Complex numbers that the number below belongs to.
\[ \frac{0}{7 \pi}+\sqrt{5}i \]The solution is \( \text{Pure Imaginary} \), which is option C.\begin{enumerate}[label=\Alph*.]
\item \( \text{Not a Complex Number} \)

This is not a number. The only non-Complex number we know is dividing by 0 as this is not a number!
\item \( \text{Nonreal Complex} \)

This is a Complex number $(a+bi)$ that is not Real (has $i$ as part of the number).
\item \( \text{Pure Imaginary} \)

* This is the correct option!
\item \( \text{Irrational} \)

These cannot be written as a fraction of Integers. Remember: $\pi$ is not an Integer!
\item \( \text{Rational} \)

These are numbers that can be written as fraction of Integers (e.g., -2/3 + 5)
\end{enumerate}

\textbf{General Comment:} Be sure to simplify $i^2 = -1$. This may remove the imaginary portion for your number. If you are having trouble, you may want to look at the \textit{Subgroups of the Real Numbers} section.
}
\end{enumerate}

\end{document}