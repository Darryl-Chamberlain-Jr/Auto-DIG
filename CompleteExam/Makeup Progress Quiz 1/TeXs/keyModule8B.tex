\documentclass{extbook}[14pt]
\usepackage{multicol, enumerate, enumitem, hyperref, color, soul, setspace, parskip, fancyhdr, amssymb, amsthm, amsmath, bbm, latexsym, units, mathtools}
\everymath{\displaystyle}
\usepackage[headsep=0.5cm,headheight=0cm, left=1 in,right= 1 in,top= 1 in,bottom= 1 in]{geometry}
\usepackage{dashrule}  % Package to use the command below to create lines between items
\newcommand{\litem}[1]{\item #1

\rule{\textwidth}{0.4pt}}
\pagestyle{fancy}
\lhead{}
\chead{Answer Key for Makeup Progress Quiz 1 Version B}
\rhead{}
\lfoot{6018-3080}
\cfoot{}
\rfoot{Spring 2021}
\begin{document}
\textbf{This key should allow you to understand why you choose the option you did (beyond just getting a question right or wrong). \href{https://xronos.clas.ufl.edu/mac1105spring2020/courseDescriptionAndMisc/Exams/LearningFromResults}{More instructions on how to use this key can be found here}.}

\textbf{If you have a suggestion to make the keys better, \href{https://forms.gle/CZkbZmPbC9XALEE88}{please fill out the short survey here}.}

\textit{Note: This key is auto-generated and may contain issues and/or errors. The keys are reviewed after each exam to ensure grading is done accurately. If there are issues (like duplicate options), they are noted in the offline gradebook. The keys are a work-in-progress to give students as many resources to improve as possible.}

\rule{\textwidth}{0.4pt}

\begin{enumerate}\litem{
Which of the following intervals describes the Domain of the function below?
\[ f(x) = e^{x+4}+1 \]The solution is \( (-\infty, \infty) \), which is option E.\begin{enumerate}[label=\Alph*.]
\item \( [a, \infty), a \in [-2.6, -0.8] \)

$[-1, \infty)$, which corresponds to using the negative vertical shift AND flipping the Range interval AND including the endpoint.
\item \( (-\infty, a), a \in [-0.4, 2.4] \)

$(-\infty, 1)$, which corresponds to using the correct vertical shift *if we wanted the Range*.
\item \( (a, \infty), a \in [-2.6, -0.8] \)

$(-1, \infty)$, which corresponds to using the negative vertical shift AND flipping the Range interval.
\item \( (-\infty, a], a \in [-0.4, 2.4] \)

$(-\infty, 1]$, which corresponds to using the correct vertical shift *if we wanted the Range* AND including the endpoint.
\item \( (-\infty, \infty) \)

* This is the correct option.
\end{enumerate}

\textbf{General Comment:} \textbf{General Comments}: Domain of a basic exponential function is $(-\infty, \infty)$ while the Range is $(0, \infty)$. We can shift these intervals [and even flip when $a<0$!] to find the new Domain/Range.
}
\litem{
 Solve the equation for $x$ and choose the interval that contains $x$ (if it exists).
\[  16 = \ln{\sqrt[7]{\frac{26}{e^{5x}}}} \]The solution is \( x = -21.748 \), which is option C.\begin{enumerate}[label=\Alph*.]
\item \( x \in [-5.75, -4.75] \)

$x = -5.748$, which corresponds to treating any root as a square root.
\item \( x \in [-5.53, -0.53] \)

$x = -4.533$, which corresponds to thinking you need to take the natural log of on the left before reducing.
\item \( x \in [-23.75, -20.75] \)

* $x = -21.748$, which is the correct option.
\item \( \text{There is no Real solution to the equation.} \)

This corresponds to believing you cannot solve the equation.
\item \( \text{None of the above.} \)

This corresponds to making an unexpected error.
\end{enumerate}

\textbf{General Comment:} \textbf{General Comments}: After using the properties of logarithmic functions to break up the right-hand side, use $\ln(e) = 1$ to reduce the question to a linear function to solve. You can put $\ln(26)$ into a calculator if you are having trouble.
}
\litem{
Solve the equation for $x$ and choose the interval that contains the solution (if it exists).
\[ 3^{-5x-5} = 16^{-2x-4} \]The solution is \( x = -107.401 \), which is option A.\begin{enumerate}[label=\Alph*.]
\item \( x \in [-108.4, -103.4] \)

* $x = -107.401$, which is the correct option.
\item \( x \in [17.19, 21.19] \)

$x = 19.188$, which corresponds to distributing the $\ln(base)$ to the first term of the exponent only.
\item \( x \in [-2.33, 0.67] \)

$x = -0.333$, which corresponds to solving the numerators as equal while ignoring the bases are different.
\item \( x \in [-0.13, 7.87] \)

$x = 1.866$, which corresponds to distributing the $\ln(base)$ to the second term of the exponent only.
\item \( \text{There is no Real solution to the equation.} \)

This corresponds to believing there is no solution since the bases are not powers of each other.
\end{enumerate}

\textbf{General Comment:} \textbf{General Comments:} This question was written so that the bases could not be written the same. You will need to take the log of both sides.
}
\litem{
Solve the equation for $x$ and choose the interval that contains the solution (if it exists).
\[ 5^{-3x-2} = 49^{-5x+5} \]The solution is \( x = 1.550 \), which is option B.\begin{enumerate}[label=\Alph*.]
\item \( x \in [-1.2, 1.4] \)

$x = 0.478$, which corresponds to distributing the $\ln(base)$ to the first term of the exponent only.
\item \( x \in [0.8, 2.8] \)

* $x = 1.550$, which is the correct option.
\item \( x \in [10.4, 12.6] \)

$x = 11.339$, which corresponds to distributing the $\ln(base)$ to the second term of the exponent only.
\item \( x \in [2.6, 4.2] \)

$x = 3.500$, which corresponds to solving the numerators as equal while ignoring the bases are different.
\item \( \text{There is no Real solution to the equation.} \)

This corresponds to believing there is no solution since the bases are not powers of each other.
\end{enumerate}

\textbf{General Comment:} \textbf{General Comments:} This question was written so that the bases could not be written the same. You will need to take the log of both sides.
}
\litem{
Solve the equation for $x$ and choose the interval that contains the solution (if it exists).
\[ \log_{5}{(2x+8)}+5 = 2 \]The solution is \( x = -3.996 \), which is option D.\begin{enumerate}[label=\Alph*.]
\item \( x \in [-131.5, -124.5] \)

$x = -125.500$, which corresponds to reversing the base and exponent when converting.
\item \( x \in [-119.5, -114.5] \)

$x = -117.500$, which corresponds to reversing the base and exponent when converting and reversing the value with $x$.
\item \( x \in [5.5, 10.5] \)

$x = 8.500$, which corresponds to ignoring the vertical shift when converting to exponential form.
\item \( x \in [-5, 1] \)

* $x = -3.996$, which is the correct option.
\item \( \text{There is no Real solution to the equation.} \)

Corresponds to believing a negative coefficient within the log equation means there is no Real solution.
\end{enumerate}

\textbf{General Comment:} \textbf{General Comments:} First, get the equation in the form $\log_b{(cx+d)} = a$. Then, convert to $b^a = cx+d$ and solve.
}
\litem{
Which of the following intervals describes the Range of the function below?
\[ f(x) = e^{x+5}-1 \]The solution is \( (-1, \infty) \), which is option A.\begin{enumerate}[label=\Alph*.]
\item \( (a, \infty), a \in [-6, 0] \)

* $(-1, \infty)$, which is the correct option.
\item \( (-\infty, a], a \in [1, 7] \)

$(-\infty, 1]$, which corresponds to using the negative vertical shift AND flipping the Range interval AND including the endpoint.
\item \( (-\infty, a), a \in [1, 7] \)

$(-\infty, 1)$, which corresponds to using the negative vertical shift AND flipping the Range interval.
\item \( [a, \infty), a \in [-6, 0] \)

$[-1, \infty)$, which corresponds to including the endpoint.
\item \( (-\infty, \infty) \)

This corresponds to confusing range of an exponential function with the domain of an exponential function.
\end{enumerate}

\textbf{General Comment:} \textbf{General Comments}: Domain of a basic exponential function is $(-\infty, \infty)$ while the Range is $(0, \infty)$. We can shift these intervals [and even flip when $a<0$!] to find the new Domain/Range.
}
\litem{
Which of the following intervals describes the Domain of the function below?
\[ f(x) = \log_2{(x+4)}-9 \]The solution is \( (-4, \infty) \), which is option A.\begin{enumerate}[label=\Alph*.]
\item \( (a, \infty), a \in [-4.4, -1] \)

* $(-4, \infty)$, which is the correct option.
\item \( (-\infty, a), a \in [3.5, 4.7] \)

$(-\infty, 4)$, which corresponds to flipping the Domain. Remember: the general for is $a*\log(x-h)+k$, \textbf{where $a$ does not affect the domain}.
\item \( (-\infty, a], a \in [6.5, 10] \)

$(-\infty, 9]$, which corresponds to using the negative vertical shift AND including the endpoint AND flipping the domain.
\item \( [a, \infty), a \in [-11.3, -7.8] \)

$[-9, \infty)$, which corresponds to using the vertical shift when shifting the Domain AND including the endpoint.
\item \( (-\infty, \infty) \)

This corresponds to thinking of the range of the log function (or the domain of the exponential function).
\end{enumerate}

\textbf{General Comment:} \textbf{General Comments}: The domain of a basic logarithmic function is $(0, \infty)$ and the Range is $(-\infty, \infty)$. We can use shifts when finding the Domain, but the Range will always be all Real numbers.
}
\litem{
Solve the equation for $x$ and choose the interval that contains the solution (if it exists).
\[ \log_{4}{(4x+6)}+6 = 2 \]The solution is \( x = -1.499 \), which is option B.\begin{enumerate}[label=\Alph*.]
\item \( x \in [60, 63.7] \)

$x = 62.500$, which corresponds to reversing the base and exponent when converting.
\item \( x \in [-2.9, -0.4] \)

* $x = -1.499$, which is the correct option.
\item \( x \in [63.4, 67.3] \)

$x = 65.500$, which corresponds to reversing the base and exponent when converting and reversing the value with $x$.
\item \( x \in [2.4, 4.7] \)

$x = 2.500$, which corresponds to ignoring the vertical shift when converting to exponential form.
\item \( \text{There is no Real solution to the equation.} \)

Corresponds to believing a negative coefficient within the log equation means there is no Real solution.
\end{enumerate}

\textbf{General Comment:} \textbf{General Comments:} First, get the equation in the form $\log_b{(cx+d)} = a$. Then, convert to $b^a = cx+d$ and solve.
}
\litem{
 Solve the equation for $x$ and choose the interval that contains $x$ (if it exists).
\[  24 = \ln{\sqrt[4]{\frac{5}{e^{9x}}}} \]The solution is \( x = -10.488 \), which is option A.\begin{enumerate}[label=\Alph*.]
\item \( x \in [-12.49, -8.49] \)

* $x = -10.488$, which is the correct option.
\item \( x \in [-7.15, -3.15] \)

$x = -5.155$, which corresponds to treating any root as a square root.
\item \( x \in [-2.59, -0.59] \)

$x = -1.591$, which corresponds to thinking you need to take the natural log of on the left before reducing.
\item \( \text{There is no Real solution to the equation.} \)

This corresponds to believing you cannot solve the equation.
\item \( \text{None of the above.} \)

This corresponds to making an unexpected error.
\end{enumerate}

\textbf{General Comment:} \textbf{General Comments}: After using the properties of logarithmic functions to break up the right-hand side, use $\ln(e) = 1$ to reduce the question to a linear function to solve. You can put $\ln(5)$ into a calculator if you are having trouble.
}
\litem{
Which of the following intervals describes the Range of the function below?
\[ f(x) = \log_2{(x+7)}-9 \]The solution is \( (\infty, \infty) \), which is option E.\begin{enumerate}[label=\Alph*.]
\item \( (-\infty, a), a \in [-9.04, -8.97] \)

$(-\infty, -9)$, which corresponds to using the vertical shift while the Range is $(-\infty, \infty)$.
\item \( [a, \infty), a \in [6.59, 8.25] \)

$[7, \infty)$, which corresponds to using the negative of the horizontal shift AND including the endpoint.
\item \( (-\infty, a), a \in [8.76, 10.3] \)

$(-\infty, 9)$, which corresponds to using the using the negative of vertical shift on $(0, \infty)$.
\item \( [a, \infty), a \in [-7.18, -6.9] \)

$[-9, \infty)$, which corresponds to using the flipped Domain AND including the endpoint.
\item \( (-\infty, \infty) \)

*This is the correct option.
\end{enumerate}

\textbf{General Comment:} \textbf{General Comments}: The domain of a basic logarithmic function is $(0, \infty)$ and the Range is $(-\infty, \infty)$. We can use shifts when finding the Domain, but the Range will always be all Real numbers.
}
\end{enumerate}

\end{document}