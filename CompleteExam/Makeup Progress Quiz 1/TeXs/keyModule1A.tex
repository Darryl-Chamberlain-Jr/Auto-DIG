\documentclass{extbook}[14pt]
\usepackage{multicol, enumerate, enumitem, hyperref, color, soul, setspace, parskip, fancyhdr, amssymb, amsthm, amsmath, bbm, latexsym, units, mathtools}
\everymath{\displaystyle}
\usepackage[headsep=0.5cm,headheight=0cm, left=1 in,right= 1 in,top= 1 in,bottom= 1 in]{geometry}
\usepackage{dashrule}  % Package to use the command below to create lines between items
\newcommand{\litem}[1]{\item #1

\rule{\textwidth}{0.4pt}}
\pagestyle{fancy}
\lhead{}
\chead{Answer Key for Makeup Progress Quiz 1 Version A}
\rhead{}
\lfoot{6018-3080}
\cfoot{}
\rfoot{Spring 2021}
\begin{document}
\textbf{This key should allow you to understand why you choose the option you did (beyond just getting a question right or wrong). \href{https://xronos.clas.ufl.edu/mac1105spring2020/courseDescriptionAndMisc/Exams/LearningFromResults}{More instructions on how to use this key can be found here}.}

\textbf{If you have a suggestion to make the keys better, \href{https://forms.gle/CZkbZmPbC9XALEE88}{please fill out the short survey here}.}

\textit{Note: This key is auto-generated and may contain issues and/or errors. The keys are reviewed after each exam to ensure grading is done accurately. If there are issues (like duplicate options), they are noted in the offline gradebook. The keys are a work-in-progress to give students as many resources to improve as possible.}

\rule{\textwidth}{0.4pt}

\begin{enumerate}\litem{
Simplify the expression below and choose the interval the simplification is contained within.
\[ 16 - 1 \div 14 * 10 - (6 * 12) \]The solution is \( -56.714 \), which is option B.\begin{enumerate}[label=\Alph*.]
\item \( [111.34, 113.19] \)

 111.429, which corresponds to not distributing a negative correctly.
\item \( [-56.77, -56.66] \)

* -56.714, which is the correct option.
\item \( [86.76, 88.44] \)

 87.993, which corresponds to not distributing addition and subtraction correctly.
\item \( [-56.05, -55.02] \)

 -56.007, which corresponds to an Order of Operations error: not reading left-to-right for multiplication/division.
\item \( \text{None of the above} \)

 You may have gotten this by making an unanticipated error. If you got a value that is not any of the others, please let the coordinator know so they can help you figure out what happened.
\end{enumerate}

\textbf{General Comment:} While you may remember (or were taught) PEMDAS is done in order, it is actually done as P/E/MD/AS. When we are at MD or AS, we read left to right.
}
\litem{
Simplify the expression below into the form $a+bi$. Then, choose the intervals that $a$ and $b$ belong to.
\[ (-2 + 8 i)(-3 + 7 i) \]The solution is \( -50 - 38 i \), which is option A.\begin{enumerate}[label=\Alph*.]
\item \( a \in [-53, -42] \text{ and } b \in [-42, -37] \)

* $-50 - 38 i$, which is the correct option.
\item \( a \in [58, 67] \text{ and } b \in [10, 13] \)

 $62 + 10 i$, which corresponds to adding a minus sign in the first term.
\item \( a \in [58, 67] \text{ and } b \in [-10, -9] \)

 $62 - 10 i$, which corresponds to adding a minus sign in the second term.
\item \( a \in [5, 7] \text{ and } b \in [55, 59] \)

 $6 + 56 i$, which corresponds to just multiplying the real terms to get the real part of the solution and the coefficients in the complex terms to get the complex part.
\item \( a \in [-53, -42] \text{ and } b \in [36, 40] \)

 $-50 + 38 i$, which corresponds to adding a minus sign in both terms.
\end{enumerate}

\textbf{General Comment:} You can treat $i$ as a variable and distribute. Just remember that $i^2=-1$, so you can continue to reduce after you distribute.
}
\litem{
Simplify the expression below into the form $a+bi$. Then, choose the intervals that $a$ and $b$ belong to.
\[ \frac{18 + 44 i}{-3 - 5 i} \]The solution is \( -8.06  - 1.24 i \), which is option C.\begin{enumerate}[label=\Alph*.]
\item \( a \in [-6.5, -5] \text{ and } b \in [-10.5, -8] \)

 $-6.00  - 8.80 i$, which corresponds to just dividing the first term by the first term and the second by the second.
\item \( a \in [4, 6] \text{ and } b \in [-8.5, -5.5] \)

 $4.88  - 6.53 i$, which corresponds to forgetting to multiply the conjugate by the numerator and not computing the conjugate correctly.
\item \( a \in [-8.5, -7.5] \text{ and } b \in [-2, -0.5] \)

* $-8.06  - 1.24 i$, which is the correct option.
\item \( a \in [-8.5, -7.5] \text{ and } b \in [-42.5, -40.5] \)

 $-8.06  - 42.00 i$, which corresponds to forgetting to multiply the conjugate by the numerator.
\item \( a \in [-274.5, -273] \text{ and } b \in [-2, -0.5] \)

 $-274.00  - 1.24 i$, which corresponds to forgetting to multiply the conjugate by the numerator and using a plus instead of a minus in the denominator.
\end{enumerate}

\textbf{General Comment:} Multiply the numerator and denominator by the *conjugate* of the denominator, then simplify. For example, if we have $2+3i$, the conjugate is $2-3i$.
}
\litem{
Choose the \textbf{smallest} set of Complex numbers that the number below belongs to.
\[ \frac{-16}{-4}+\sqrt{-49}i \]The solution is \( \text{Rational} \), which is option C.\begin{enumerate}[label=\Alph*.]
\item \( \text{Irrational} \)

These cannot be written as a fraction of Integers. Remember: $\pi$ is not an Integer!
\item \( \text{Nonreal Complex} \)

This is a Complex number $(a+bi)$ that is not Real (has $i$ as part of the number).
\item \( \text{Rational} \)

* This is the correct option!
\item \( \text{Not a Complex Number} \)

This is not a number. The only non-Complex number we know is dividing by 0 as this is not a number!
\item \( \text{Pure Imaginary} \)

This is a Complex number $(a+bi)$ that \textbf{only} has an imaginary part like $2i$.
\end{enumerate}

\textbf{General Comment:} Be sure to simplify $i^2 = -1$. This may remove the imaginary portion for your number. If you are having trouble, you may want to look at the \textit{Subgroups of the Real Numbers} section.
}
\litem{
Simplify the expression below into the form $a+bi$. Then, choose the intervals that $a$ and $b$ belong to.
\[ (3 - 4 i)(9 + 8 i) \]The solution is \( 59 - 12 i \), which is option C.\begin{enumerate}[label=\Alph*.]
\item \( a \in [-8, -4] \text{ and } b \in [-62, -58] \)

 $-5 - 60 i$, which corresponds to adding a minus sign in the second term.
\item \( a \in [-8, -4] \text{ and } b \in [57, 65] \)

 $-5 + 60 i$, which corresponds to adding a minus sign in the first term.
\item \( a \in [58, 60] \text{ and } b \in [-13, -9] \)

* $59 - 12 i$, which is the correct option.
\item \( a \in [25, 29] \text{ and } b \in [-33, -31] \)

 $27 - 32 i$, which corresponds to just multiplying the real terms to get the real part of the solution and the coefficients in the complex terms to get the complex part.
\item \( a \in [58, 60] \text{ and } b \in [12, 15] \)

 $59 + 12 i$, which corresponds to adding a minus sign in both terms.
\end{enumerate}

\textbf{General Comment:} You can treat $i$ as a variable and distribute. Just remember that $i^2=-1$, so you can continue to reduce after you distribute.
}
\litem{
Simplify the expression below into the form $a+bi$. Then, choose the intervals that $a$ and $b$ belong to.
\[ \frac{27 - 44 i}{1 - 8 i} \]The solution is \( 5.83  + 2.65 i \), which is option E.\begin{enumerate}[label=\Alph*.]
\item \( a \in [-6, -4] \text{ and } b \in [-5, -3.5] \)

 $-5.00  - 4.00 i$, which corresponds to forgetting to multiply the conjugate by the numerator and not computing the conjugate correctly.
\item \( a \in [26, 28] \text{ and } b \in [3, 7] \)

 $27.00  + 5.50 i$, which corresponds to just dividing the first term by the first term and the second by the second.
\item \( a \in [5, 6] \text{ and } b \in [171, 172.5] \)

 $5.83  + 172.00 i$, which corresponds to forgetting to multiply the conjugate by the numerator.
\item \( a \in [378.5, 380] \text{ and } b \in [2, 3.5] \)

 $379.00  + 2.65 i$, which corresponds to forgetting to multiply the conjugate by the numerator and using a plus instead of a minus in the denominator.
\item \( a \in [5, 6] \text{ and } b \in [2, 3.5] \)

* $5.83  + 2.65 i$, which is the correct option.
\end{enumerate}

\textbf{General Comment:} Multiply the numerator and denominator by the *conjugate* of the denominator, then simplify. For example, if we have $2+3i$, the conjugate is $2-3i$.
}
\litem{
Simplify the expression below and choose the interval the simplification is contained within.
\[ 17 - 13 \div 6 * 5 - (8 * 9) \]The solution is \( -65.833 \), which is option D.\begin{enumerate}[label=\Alph*.]
\item \( [86.57, 89.57] \)

 88.567, which corresponds to not distributing addition and subtraction correctly.
\item \( [-20.5, -9.5] \)

 -16.500, which corresponds to not distributing a negative correctly.
\item \( [-56.43, -54.43] \)

 -55.433, which corresponds to an Order of Operations error: not reading left-to-right for multiplication/division.
\item \( [-72.83, -61.83] \)

* -65.833, which is the correct option.
\item \( \text{None of the above} \)

 You may have gotten this by making an unanticipated error. If you got a value that is not any of the others, please let the coordinator know so they can help you figure out what happened.
\end{enumerate}

\textbf{General Comment:} While you may remember (or were taught) PEMDAS is done in order, it is actually done as P/E/MD/AS. When we are at MD or AS, we read left to right.
}
\litem{
Choose the \textbf{smallest} set of Real numbers that the number below belongs to.
\[ -\sqrt{\frac{9}{0}} \]The solution is \( \text{Not a Real number} \), which is option E.\begin{enumerate}[label=\Alph*.]
\item \( \text{Irrational} \)

These cannot be written as a fraction of Integers.
\item \( \text{Integer} \)

These are the negative and positive counting numbers (..., -3, -2, -1, 0, 1, 2, 3, ...)
\item \( \text{Rational} \)

These are numbers that can be written as fraction of Integers (e.g., -2/3)
\item \( \text{Whole} \)

These are the counting numbers with 0 (0, 1, 2, 3, ...)
\item \( \text{Not a Real number} \)

* This is the correct option!
\end{enumerate}

\textbf{General Comment:} First, you \textbf{NEED} to simplify the expression. This question simplifies to $-\sqrt{\frac{9}{0}}$. 
 
 Be sure you look at the simplified fraction and not just the decimal expansion. Numbers such as 13, 17, and 19 provide \textbf{long but repeating/terminating decimal expansions!} 
 
 The only ways to *not* be a Real number are: dividing by 0 or taking the square root of a negative number. 
 
 Irrational numbers are more than just square root of 3: adding or subtracting values from square root of 3 is also irrational.
}
\litem{
Choose the \textbf{smallest} set of Real numbers that the number below belongs to.
\[ -\sqrt{\frac{1716}{13}} \]The solution is \( \text{Irrational} \), which is option E.\begin{enumerate}[label=\Alph*.]
\item \( \text{Whole} \)

These are the counting numbers with 0 (0, 1, 2, 3, ...)
\item \( \text{Integer} \)

These are the negative and positive counting numbers (..., -3, -2, -1, 0, 1, 2, 3, ...)
\item \( \text{Rational} \)

These are numbers that can be written as fraction of Integers (e.g., -2/3)
\item \( \text{Not a Real number} \)

These are Nonreal Complex numbers \textbf{OR} things that are not numbers (e.g., dividing by 0).
\item \( \text{Irrational} \)

* This is the correct option!
\end{enumerate}

\textbf{General Comment:} First, you \textbf{NEED} to simplify the expression. This question simplifies to $-\sqrt{132}$. 
 
 Be sure you look at the simplified fraction and not just the decimal expansion. Numbers such as 13, 17, and 19 provide \textbf{long but repeating/terminating decimal expansions!} 
 
 The only ways to *not* be a Real number are: dividing by 0 or taking the square root of a negative number. 
 
 Irrational numbers are more than just square root of 3: adding or subtracting values from square root of 3 is also irrational.
}
\litem{
Choose the \textbf{smallest} set of Complex numbers that the number below belongs to.
\[ \frac{\sqrt{91}}{13}+8i^2 \]The solution is \( \text{Irrational} \), which is option B.\begin{enumerate}[label=\Alph*.]
\item \( \text{Not a Complex Number} \)

This is not a number. The only non-Complex number we know is dividing by 0 as this is not a number!
\item \( \text{Irrational} \)

* This is the correct option!
\item \( \text{Rational} \)

These are numbers that can be written as fraction of Integers (e.g., -2/3 + 5)
\item \( \text{Nonreal Complex} \)

This is a Complex number $(a+bi)$ that is not Real (has $i$ as part of the number).
\item \( \text{Pure Imaginary} \)

This is a Complex number $(a+bi)$ that \textbf{only} has an imaginary part like $2i$.
\end{enumerate}

\textbf{General Comment:} Be sure to simplify $i^2 = -1$. This may remove the imaginary portion for your number. If you are having trouble, you may want to look at the \textit{Subgroups of the Real Numbers} section.
}
\end{enumerate}

\end{document}