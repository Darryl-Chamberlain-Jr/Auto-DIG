\documentclass{extbook}[14pt]
\usepackage{multicol, enumerate, enumitem, hyperref, color, soul, setspace, parskip, fancyhdr, amssymb, amsthm, amsmath, bbm, latexsym, units, mathtools}
\everymath{\displaystyle}
\usepackage[headsep=0.5cm,headheight=0cm, left=1 in,right= 1 in,top= 1 in,bottom= 1 in]{geometry}
\usepackage{dashrule}  % Package to use the command below to create lines between items
\newcommand{\litem}[1]{\item #1

\rule{\textwidth}{0.4pt}}
\pagestyle{fancy}
\lhead{}
\chead{Answer Key for Makeup Progress Quiz 1 Version A}
\rhead{}
\lfoot{6018-3080}
\cfoot{}
\rfoot{Spring 2021}
\begin{document}
\textbf{This key should allow you to understand why you choose the option you did (beyond just getting a question right or wrong). \href{https://xronos.clas.ufl.edu/mac1105spring2020/courseDescriptionAndMisc/Exams/LearningFromResults}{More instructions on how to use this key can be found here}.}

\textbf{If you have a suggestion to make the keys better, \href{https://forms.gle/CZkbZmPbC9XALEE88}{please fill out the short survey here}.}

\textit{Note: This key is auto-generated and may contain issues and/or errors. The keys are reviewed after each exam to ensure grading is done accurately. If there are issues (like duplicate options), they are noted in the offline gradebook. The keys are a work-in-progress to give students as many resources to improve as possible.}

\rule{\textwidth}{0.4pt}

\begin{enumerate}\litem{
Solve the linear inequality below. Then, choose the constant and interval combination that describes the solution set.
\[ \frac{10}{3} - \frac{10}{7} x \geq \frac{-3}{2} x + \frac{7}{5} \]The solution is \( [-27.067, \infty) \), which is option B.\begin{enumerate}[label=\Alph*.]
\item \( (-\infty, a], \text{ where } a \in [-30.07, -26.07] \)

 $(-\infty, -27.067]$, which corresponds to switching the direction of the interval. You likely did this if you did not flip the inequality when dividing by a negative!
\item \( [a, \infty), \text{ where } a \in [-28.07, -25.07] \)

* $[-27.067, \infty)$, which is the correct option.
\item \( (-\infty, a], \text{ where } a \in [26.07, 30.07] \)

 $(-\infty, 27.067]$, which corresponds to switching the direction of the interval AND negating the endpoint. You likely did this if you did not flip the inequality when dividing by a negative as well as not moving values over to a side properly.
\item \( [a, \infty), \text{ where } a \in [25.07, 29.07] \)

 $[27.067, \infty)$, which corresponds to negating the endpoint of the solution.
\item \( \text{None of the above}. \)

You may have chosen this if you thought the inequality did not match the ends of the intervals.
\end{enumerate}

\textbf{General Comment:} Remember that less/greater than or equal to includes the endpoint, while less/greater do not. Also, remember that you need to flip the inequality when you multiply or divide by a negative.
}
\litem{
Using an interval or intervals, describe all the $x$-values within or including a distance of the given values.
\[ \text{ Less than } 3 \text{ units from the number } -3. \]The solution is \( (-6, 0) \), which is option C.\begin{enumerate}[label=\Alph*.]
\item \( (-\infty, -6) \cup (0, \infty) \)

This describes the values more than 3 from -3
\item \( (-\infty, -6] \cup [0, \infty) \)

This describes the values no less than 3 from -3
\item \( (-6, 0) \)

This describes the values less than 3 from -3
\item \( [-6, 0] \)

This describes the values no more than 3 from -3
\item \( \text{None of the above} \)

You likely thought the values in the interval were not correct.
\end{enumerate}

\textbf{General Comment:} When thinking about this language, it helps to draw a number line and try points.
}
\litem{
Using an interval or intervals, describe all the $x$-values within or including a distance of the given values.
\[ \text{ No less than } 2 \text{ units from the number } -3. \]The solution is \( (-\infty, -5] \cup [-1, \infty) \), which is option B.\begin{enumerate}[label=\Alph*.]
\item \( (-\infty, -5) \cup (-1, \infty) \)

This describes the values more than 2 from -3
\item \( (-\infty, -5] \cup [-1, \infty) \)

This describes the values no less than 2 from -3
\item \( [-5, -1] \)

This describes the values no more than 2 from -3
\item \( (-5, -1) \)

This describes the values less than 2 from -3
\item \( \text{None of the above} \)

You likely thought the values in the interval were not correct.
\end{enumerate}

\textbf{General Comment:} When thinking about this language, it helps to draw a number line and try points.
}
\litem{
Solve the linear inequality below. Then, choose the constant and interval combination that describes the solution set.
\[ \frac{3}{4} + \frac{3}{5} x \geq \frac{6}{8} x - \frac{9}{6} \]The solution is \( (-\infty, 15.0] \), which is option C.\begin{enumerate}[label=\Alph*.]
\item \( (-\infty, a], \text{ where } a \in [-15, -12] \)

 $(-\infty, -15.0]$, which corresponds to negating the endpoint of the solution.
\item \( [a, \infty), \text{ where } a \in [12, 16] \)

 $[15.0, \infty)$, which corresponds to switching the direction of the interval. You likely did this if you did not flip the inequality when dividing by a negative!
\item \( (-\infty, a], \text{ where } a \in [15, 17] \)

* $(-\infty, 15.0]$, which is the correct option.
\item \( [a, \infty), \text{ where } a \in [-15, -14] \)

 $[-15.0, \infty)$, which corresponds to switching the direction of the interval AND negating the endpoint. You likely did this if you did not flip the inequality when dividing by a negative as well as not moving values over to a side properly.
\item \( \text{None of the above}. \)

You may have chosen this if you thought the inequality did not match the ends of the intervals.
\end{enumerate}

\textbf{General Comment:} Remember that less/greater than or equal to includes the endpoint, while less/greater do not. Also, remember that you need to flip the inequality when you multiply or divide by a negative.
}
\litem{
Solve the linear inequality below. Then, choose the constant and interval combination that describes the solution set.
\[ -4 - 6 x < \frac{-46 x + 5}{8} \leq -9 - 7 x \]The solution is \( (-18.50, -7.70] \), which is option B.\begin{enumerate}[label=\Alph*.]
\item \( (-\infty, a) \cup [b, \infty), \text{ where } a \in [-20.5, -14.5] \text{ and } b \in [-8.7, -3.7] \)

$(-\infty, -18.50) \cup [-7.70, \infty)$, which corresponds to displaying the and-inequality as an or-inequality.
\item \( (a, b], \text{ where } a \in [-21.5, -15.5] \text{ and } b \in [-10.7, -6.7] \)

* $(-18.50, -7.70]$, which is the correct option.
\item \( (-\infty, a] \cup (b, \infty), \text{ where } a \in [-20.5, -14.5] \text{ and } b \in [-9.7, -3.7] \)

$(-\infty, -18.50] \cup (-7.70, \infty)$, which corresponds to displaying the and-inequality as an or-inequality AND flipping the inequality.
\item \( [a, b), \text{ where } a \in [-19.5, -17.5] \text{ and } b \in [-7.7, -6.7] \)

$[-18.50, -7.70)$, which corresponds to flipping the inequality.
\item \( \text{None of the above.} \)


\end{enumerate}

\textbf{General Comment:} To solve, you will need to break up the compound inequality into two inequalities. Be sure to keep track of the inequality! It may be best to draw a number line and graph your solution.
}
\litem{
Solve the linear inequality below. Then, choose the constant and interval combination that describes the solution set.
\[ -4 + 5 x > 8 x \text{ or } 4 + 5 x < 6 x \]The solution is \( (-\infty, -1.333) \text{ or } (4.0, \infty) \), which is option D.\begin{enumerate}[label=\Alph*.]
\item \( (-\infty, a] \cup [b, \infty), \text{ where } a \in [-5, -2] \text{ and } b \in [1.3, 2.7] \)

Corresponds to including the endpoints AND negating.
\item \( (-\infty, a] \cup [b, \infty), \text{ where } a \in [-3.33, 1.67] \text{ and } b \in [3.5, 4.8] \)

Corresponds to including the endpoints (when they should be excluded).
\item \( (-\infty, a) \cup (b, \infty), \text{ where } a \in [-6.5, -2.8] \text{ and } b \in [-0.3, 2.8] \)

Corresponds to inverting the inequality and negating the solution.
\item \( (-\infty, a) \cup (b, \infty), \text{ where } a \in [-2.7, -0.6] \text{ and } b \in [3.3, 4.2] \)

 * Correct option.
\item \( (-\infty, \infty) \)

Corresponds to the variable canceling, which does not happen in this instance.
\end{enumerate}

\textbf{General Comment:} When multiplying or dividing by a negative, flip the sign.
}
\litem{
Solve the linear inequality below. Then, choose the constant and interval combination that describes the solution set.
\[ -4x -9 < 3x + 5 \]The solution is \( (-2.0, \infty) \), which is option A.\begin{enumerate}[label=\Alph*.]
\item \( (a, \infty), \text{ where } a \in [-6, 0] \)

* $(-2.0, \infty)$, which is the correct option.
\item \( (a, \infty), \text{ where } a \in [1, 5] \)

 $(2.0, \infty)$, which corresponds to negating the endpoint of the solution.
\item \( (-\infty, a), \text{ where } a \in [-4, 1] \)

 $(-\infty, -2.0)$, which corresponds to switching the direction of the interval. You likely did this if you did not flip the inequality when dividing by a negative!
\item \( (-\infty, a), \text{ where } a \in [2, 6] \)

 $(-\infty, 2.0)$, which corresponds to switching the direction of the interval AND negating the endpoint. You likely did this if you did not flip the inequality when dividing by a negative as well as not moving values over to a side properly.
\item \( \text{None of the above}. \)

You may have chosen this if you thought the inequality did not match the ends of the intervals.
\end{enumerate}

\textbf{General Comment:} Remember that less/greater than or equal to includes the endpoint, while less/greater do not. Also, remember that you need to flip the inequality when you multiply or divide by a negative.
}
\litem{
Solve the linear inequality below. Then, choose the constant and interval combination that describes the solution set.
\[ 7 + 3 x < \frac{77 x + 3}{9} \leq 3 + 8 x \]The solution is \( (1.20, 4.80] \), which is option C.\begin{enumerate}[label=\Alph*.]
\item \( [a, b), \text{ where } a \in [0.4, 3] \text{ and } b \in [4.8, 9.8] \)

$[1.20, 4.80)$, which corresponds to flipping the inequality.
\item \( (-\infty, a] \cup (b, \infty), \text{ where } a \in [-0.5, 2.7] \text{ and } b \in [0.8, 5.8] \)

$(-\infty, 1.20] \cup (4.80, \infty)$, which corresponds to displaying the and-inequality as an or-inequality AND flipping the inequality.
\item \( (a, b], \text{ where } a \in [0.2, 5.2] \text{ and } b \in [3.8, 8.8] \)

* $(1.20, 4.80]$, which is the correct option.
\item \( (-\infty, a) \cup [b, \infty), \text{ where } a \in [0.2, 2.2] \text{ and } b \in [4.8, 6.8] \)

$(-\infty, 1.20) \cup [4.80, \infty)$, which corresponds to displaying the and-inequality as an or-inequality.
\item \( \text{None of the above.} \)


\end{enumerate}

\textbf{General Comment:} To solve, you will need to break up the compound inequality into two inequalities. Be sure to keep track of the inequality! It may be best to draw a number line and graph your solution.
}
\litem{
Solve the linear inequality below. Then, choose the constant and interval combination that describes the solution set.
\[ -8 + 3 x > 6 x \text{ or } 9 - 3 x < 4 x \]The solution is \( (-\infty, -2.667) \text{ or } (1.286, \infty) \), which is option D.\begin{enumerate}[label=\Alph*.]
\item \( (-\infty, a] \cup [b, \infty), \text{ where } a \in [-3.3, -2] \text{ and } b \in [1.15, 1.68] \)

Corresponds to including the endpoints (when they should be excluded).
\item \( (-\infty, a) \cup (b, \infty), \text{ where } a \in [-2.62, -1.01] \text{ and } b \in [2.5, 3.7] \)

Corresponds to inverting the inequality and negating the solution.
\item \( (-\infty, a] \cup [b, \infty), \text{ where } a \in [-1.5, -1.1] \text{ and } b \in [2.64, 3.13] \)

Corresponds to including the endpoints AND negating.
\item \( (-\infty, a) \cup (b, \infty), \text{ where } a \in [-2.68, -1.33] \text{ and } b \in [0.8, 1.7] \)

 * Correct option.
\item \( (-\infty, \infty) \)

Corresponds to the variable canceling, which does not happen in this instance.
\end{enumerate}

\textbf{General Comment:} When multiplying or dividing by a negative, flip the sign.
}
\litem{
Solve the linear inequality below. Then, choose the constant and interval combination that describes the solution set.
\[ -6x -5 > 10x -10 \]The solution is \( (-\infty, 0.312) \), which is option B.\begin{enumerate}[label=\Alph*.]
\item \( (a, \infty), \text{ where } a \in [-1.14, -0.05] \)

 $(-0.312, \infty)$, which corresponds to switching the direction of the interval AND negating the endpoint. You likely did this if you did not flip the inequality when dividing by a negative as well as not moving values over to a side properly.
\item \( (-\infty, a), \text{ where } a \in [0.26, 0.9] \)

* $(-\infty, 0.312)$, which is the correct option.
\item \( (-\infty, a), \text{ where } a \in [-1.95, 0.11] \)

 $(-\infty, -0.312)$, which corresponds to negating the endpoint of the solution.
\item \( (a, \infty), \text{ where } a \in [0.25, 0.38] \)

 $(0.312, \infty)$, which corresponds to switching the direction of the interval. You likely did this if you did not flip the inequality when dividing by a negative!
\item \( \text{None of the above}. \)

You may have chosen this if you thought the inequality did not match the ends of the intervals.
\end{enumerate}

\textbf{General Comment:} Remember that less/greater than or equal to includes the endpoint, while less/greater do not. Also, remember that you need to flip the inequality when you multiply or divide by a negative.
}
\end{enumerate}

\end{document}