\documentclass{extbook}[14pt]
\usepackage{multicol, enumerate, enumitem, hyperref, color, soul, setspace, parskip, fancyhdr, amssymb, amsthm, amsmath, bbm, latexsym, units, mathtools}
\everymath{\displaystyle}
\usepackage[headsep=0.5cm,headheight=0cm, left=1 in,right= 1 in,top= 1 in,bottom= 1 in]{geometry}
\usepackage{dashrule}  % Package to use the command below to create lines between items
\newcommand{\litem}[1]{\item #1

\rule{\textwidth}{0.4pt}}
\pagestyle{fancy}
\lhead{}
\chead{Answer Key for Makeup Progress Quiz 3 Version A}
\rhead{}
\lfoot{4315-3397}
\cfoot{}
\rfoot{Fall 2020}
\begin{document}
\textbf{This key should allow you to understand why you choose the option you did (beyond just getting a question right or wrong). \href{https://xronos.clas.ufl.edu/mac1105spring2020/courseDescriptionAndMisc/Exams/LearningFromResults}{More instructions on how to use this key can be found here}.}

\textbf{If you have a suggestion to make the keys better, \href{https://forms.gle/CZkbZmPbC9XALEE88}{please fill out the short survey here}.}

\textit{Note: This key is auto-generated and may contain issues and/or errors. The keys are reviewed after each exam to ensure grading is done accurately. If there are issues (like duplicate options), they are noted in the offline gradebook. The keys are a work-in-progress to give students as many resources to improve as possible.}

\rule{\textwidth}{0.4pt}

\begin{enumerate}\litem{
Choose the \textbf{smallest} set of Real numbers that the number below belongs to.
\[ \sqrt{\frac{74529}{169}} \]

The solution is \( \text{Whole} \), which is option C.\begin{enumerate}[label=\Alph*.]
\item \( \text{Not a Real number} \)

These are Nonreal Complex numbers \textbf{OR} things that are not numbers (e.g., dividing by 0).
\item \( \text{Integer} \)

These are the negative and positive counting numbers (..., -3, -2, -1, 0, 1, 2, 3, ...)
\item \( \text{Whole} \)

* This is the correct option!
\item \( \text{Irrational} \)

These cannot be written as a fraction of Integers.
\item \( \text{Rational} \)

These are numbers that can be written as fraction of Integers (e.g., -2/3)
\end{enumerate}

\textbf{General Comment:} First, you \textbf{NEED} to simplify the expression. This question simplifies to $273$. 
 
 Be sure you look at the simplified fraction and not just the decimal expansion. Numbers such as 13, 17, and 19 provide \textbf{long but repeating/terminating decimal expansions!} 
 
 The only ways to *not* be a Real number are: dividing by 0 or taking the square root of a negative number. 
 
 Irrational numbers are more than just square root of 3: adding or subtracting values from square root of 3 is also irrational.
}
\litem{
Choose the \textbf{smallest} set of Complex numbers that the number below belongs to.
\[ \sqrt{\frac{1547}{13}}+\sqrt{55} i \]

The solution is \( \text{Nonreal Complex} \), which is option D.\begin{enumerate}[label=\Alph*.]
\item \( \text{Irrational} \)

These cannot be written as a fraction of Integers. Remember: $\pi$ is not an Integer!
\item \( \text{Pure Imaginary} \)

This is a Complex number $(a+bi)$ that \textbf{only} has an imaginary part like $2i$.
\item \( \text{Rational} \)

These are numbers that can be written as fraction of Integers (e.g., -2/3 + 5)
\item \( \text{Nonreal Complex} \)

* This is the correct option!
\item \( \text{Not a Complex Number} \)

This is not a number. The only non-Complex number we know is dividing by 0 as this is not a number!
\end{enumerate}

\textbf{General Comment:} Be sure to simplify $i^2 = -1$. This may remove the imaginary portion for your number. If you are having trouble, you may want to look at the \textit{Subgroups of the Real Numbers} section.
}
\litem{
Choose the \textbf{smallest} set of Real numbers that the number below belongs to.
\[ \sqrt{\frac{625}{169}} \]

The solution is \( \text{Rational} \), which is option A.\begin{enumerate}[label=\Alph*.]
\item \( \text{Rational} \)

* This is the correct option!
\item \( \text{Whole} \)

These are the counting numbers with 0 (0, 1, 2, 3, ...)
\item \( \text{Irrational} \)

These cannot be written as a fraction of Integers.
\item \( \text{Integer} \)

These are the negative and positive counting numbers (..., -3, -2, -1, 0, 1, 2, 3, ...)
\item \( \text{Not a Real number} \)

These are Nonreal Complex numbers \textbf{OR} things that are not numbers (e.g., dividing by 0).
\end{enumerate}

\textbf{General Comment:} First, you \textbf{NEED} to simplify the expression. This question simplifies to $\frac{25}{13}$. 
 
 Be sure you look at the simplified fraction and not just the decimal expansion. Numbers such as 13, 17, and 19 provide \textbf{long but repeating/terminating decimal expansions!} 
 
 The only ways to *not* be a Real number are: dividing by 0 or taking the square root of a negative number. 
 
 Irrational numbers are more than just square root of 3: adding or subtracting values from square root of 3 is also irrational.
}
\litem{
Simplify the expression below and choose the interval the simplification is contained within.
\[ 18 - 1^2 + 20 \div 12 * 15 \div 9 \]

The solution is \( 19.778 \), which is option D.\begin{enumerate}[label=\Alph*.]
\item \( [19, 19.44] \)

 19.012, which corresponds to two Order of Operations errors.
\item \( [20.95, 22.53] \)

 21.778, which corresponds to an Order of Operations error: multiplying by negative before squaring. For example: $(-3)^2 \neq -3^2$
\item \( [16.03, 17.42] \)

 17.012, which corresponds to an Order of Operations error: not reading left-to-right for multiplication/division.
\item \( [19.31, 19.82] \)

* 19.778, this is the correct option
\item \( \text{None of the above} \)

 You may have gotten this by making an unanticipated error. If you got a value that is not any of the others, please let the coordinator know so they can help you figure out what happened.
\end{enumerate}

\textbf{General Comment:} While you may remember (or were taught) PEMDAS is done in order, it is actually done as P/E/MD/AS. When we are at MD or AS, we read left to right.
}
\litem{
Choose the \textbf{smallest} set of Complex numbers that the number below belongs to.
\[ \frac{10}{2}+\sqrt{-9}i \]

The solution is \( \text{Rational} \), which is option C.\begin{enumerate}[label=\Alph*.]
\item \( \text{Not a Complex Number} \)

This is not a number. The only non-Complex number we know is dividing by 0 as this is not a number!
\item \( \text{Irrational} \)

These cannot be written as a fraction of Integers. Remember: $\pi$ is not an Integer!
\item \( \text{Rational} \)

* This is the correct option!
\item \( \text{Nonreal Complex} \)

This is a Complex number $(a+bi)$ that is not Real (has $i$ as part of the number).
\item \( \text{Pure Imaginary} \)

This is a Complex number $(a+bi)$ that \textbf{only} has an imaginary part like $2i$.
\end{enumerate}

\textbf{General Comment:} Be sure to simplify $i^2 = -1$. This may remove the imaginary portion for your number. If you are having trouble, you may want to look at the \textit{Subgroups of the Real Numbers} section.
}
\litem{
Simplify the expression below into the form $a+bi$. Then, choose the intervals that $a$ and $b$ belong to.
\[ \frac{-36 - 77 i}{-3 + i} \]

The solution is \( 3.10  + 26.70 i \), which is option C.\begin{enumerate}[label=\Alph*.]
\item \( a \in [2, 5] \text{ and } b \in [266.5, 267.5] \)

 $3.10  + 267.00 i$, which corresponds to forgetting to multiply the conjugate by the numerator.
\item \( a \in [17.5, 19.5] \text{ and } b \in [19, 20] \)

 $18.50  + 19.50 i$, which corresponds to forgetting to multiply the conjugate by the numerator and not computing the conjugate correctly.
\item \( a \in [2, 5] \text{ and } b \in [26, 27] \)

* $3.10  + 26.70 i$, which is the correct option.
\item \( a \in [11, 13.5] \text{ and } b \in [-77.5, -76] \)

 $12.00  - 77.00 i$, which corresponds to just dividing the first term by the first term and the second by the second.
\item \( a \in [29.5, 32.5] \text{ and } b \in [26, 27] \)

 $31.00  + 26.70 i$, which corresponds to forgetting to multiply the conjugate by the numerator and using a plus instead of a minus in the denominator.
\end{enumerate}

\textbf{General Comment:} Multiply the numerator and denominator by the *conjugate* of the denominator, then simplify. For example, if we have $2+3i$, the conjugate is $2-3i$.
}
\litem{
Simplify the expression below into the form $a+bi$. Then, choose the intervals that $a$ and $b$ belong to.
\[ (-10 + 4 i)(2 + 7 i) \]

The solution is \( -48 - 62 i \), which is option C.\begin{enumerate}[label=\Alph*.]
\item \( a \in [8, 12] \text{ and } b \in [76, 82] \)

 $8 + 78 i$, which corresponds to adding a minus sign in the second term.
\item \( a \in [-21, -17] \text{ and } b \in [24, 34] \)

 $-20 + 28 i$, which corresponds to just multiplying the real terms to get the real part of the solution and the coefficients in the complex terms to get the complex part.
\item \( a \in [-49, -45] \text{ and } b \in [-63, -61] \)

* $-48 - 62 i$, which is the correct option.
\item \( a \in [-49, -45] \text{ and } b \in [60, 69] \)

 $-48 + 62 i$, which corresponds to adding a minus sign in both terms.
\item \( a \in [8, 12] \text{ and } b \in [-78, -75] \)

 $8 - 78 i$, which corresponds to adding a minus sign in the first term.
\end{enumerate}

\textbf{General Comment:} You can treat $i$ as a variable and distribute. Just remember that $i^2=-1$, so you can continue to reduce after you distribute.
}
\litem{
Simplify the expression below into the form $a+bi$. Then, choose the intervals that $a$ and $b$ belong to.
\[ \frac{45 - 22 i}{-3 - 6 i} \]

The solution is \( -0.07  + 7.47 i \), which is option E.\begin{enumerate}[label=\Alph*.]
\item \( a \in [-15.5, -13] \text{ and } b \in [2, 4.5] \)

 $-15.00  + 3.67 i$, which corresponds to just dividing the first term by the first term and the second by the second.
\item \( a \in [-4, -2.5] \text{ and } b \in [7, 8] \)

 $-3.00  + 7.47 i$, which corresponds to forgetting to multiply the conjugate by the numerator and using a plus instead of a minus in the denominator.
\item \( a \in [-7.5, -5] \text{ and } b \in [-6, -4] \)

 $-5.93  - 4.53 i$, which corresponds to forgetting to multiply the conjugate by the numerator and not computing the conjugate correctly.
\item \( a \in [-0.5, 0.5] \text{ and } b \in [335.5, 337] \)

 $-0.07  + 336.00 i$, which corresponds to forgetting to multiply the conjugate by the numerator.
\item \( a \in [-0.5, 0.5] \text{ and } b \in [7, 8] \)

* $-0.07  + 7.47 i$, which is the correct option.
\end{enumerate}

\textbf{General Comment:} Multiply the numerator and denominator by the *conjugate* of the denominator, then simplify. For example, if we have $2+3i$, the conjugate is $2-3i$.
}
\litem{
Simplify the expression below and choose the interval the simplification is contained within.
\[ 16 - 15^2 + 19 \div 1 * 14 \div 18 \]

The solution is \( -194.222 \), which is option A.\begin{enumerate}[label=\Alph*.]
\item \( [-197.22, -193.22] \)

* -194.222, this is the correct option
\item \( [253.78, 259.78] \)

 255.778, which corresponds to an Order of Operations error: multiplying by negative before squaring. For example: $(-3)^2 \neq -3^2$
\item \( [238.08, 244.08] \)

 241.075, which corresponds to two Order of Operations errors.
\item \( [-211.92, -206.92] \)

 -208.925, which corresponds to an Order of Operations error: not reading left-to-right for multiplication/division.
\item \( \text{None of the above} \)

 You may have gotten this by making an unanticipated error. If you got a value that is not any of the others, please let the coordinator know so they can help you figure out what happened.
\end{enumerate}

\textbf{General Comment:} While you may remember (or were taught) PEMDAS is done in order, it is actually done as P/E/MD/AS. When we are at MD or AS, we read left to right.
}
\litem{
Simplify the expression below into the form $a+bi$. Then, choose the intervals that $a$ and $b$ belong to.
\[ (-3 + 5 i)(2 - 9 i) \]

The solution is \( 39 + 37 i \), which is option A.\begin{enumerate}[label=\Alph*.]
\item \( a \in [38, 49] \text{ and } b \in [35, 38] \)

* $39 + 37 i$, which is the correct option.
\item \( a \in [-51, -46] \text{ and } b \in [-20, -14] \)

 $-51 - 17 i$, which corresponds to adding a minus sign in the second term.
\item \( a \in [38, 49] \text{ and } b \in [-40, -34] \)

 $39 - 37 i$, which corresponds to adding a minus sign in both terms.
\item \( a \in [-51, -46] \text{ and } b \in [16, 20] \)

 $-51 + 17 i$, which corresponds to adding a minus sign in the first term.
\item \( a \in [-7, -1] \text{ and } b \in [-49, -44] \)

 $-6 - 45 i$, which corresponds to just multiplying the real terms to get the real part of the solution and the coefficients in the complex terms to get the complex part.
\end{enumerate}

\textbf{General Comment:} You can treat $i$ as a variable and distribute. Just remember that $i^2=-1$, so you can continue to reduce after you distribute.
}
\end{enumerate}

\end{document}