\documentclass{extbook}[14pt]
\usepackage{multicol, enumerate, enumitem, hyperref, color, soul, setspace, parskip, fancyhdr, amssymb, amsthm, amsmath, bbm, latexsym, units, mathtools}
\everymath{\displaystyle}
\usepackage[headsep=0.5cm,headheight=0cm, left=1 in,right= 1 in,top= 1 in,bottom= 1 in]{geometry}
\usepackage{dashrule}  % Package to use the command below to create lines between items
\newcommand{\litem}[1]{\item #1

\rule{\textwidth}{0.4pt}}
\pagestyle{fancy}
\lhead{}
\chead{Answer Key for Makeup Progress Quiz 3 Version A}
\rhead{}
\lfoot{4315-3397}
\cfoot{}
\rfoot{Fall 2020}
\begin{document}
\textbf{This key should allow you to understand why you choose the option you did (beyond just getting a question right or wrong). \href{https://xronos.clas.ufl.edu/mac1105spring2020/courseDescriptionAndMisc/Exams/LearningFromResults}{More instructions on how to use this key can be found here}.}

\textbf{If you have a suggestion to make the keys better, \href{https://forms.gle/CZkbZmPbC9XALEE88}{please fill out the short survey here}.}

\textit{Note: This key is auto-generated and may contain issues and/or errors. The keys are reviewed after each exam to ensure grading is done accurately. If there are issues (like duplicate options), they are noted in the offline gradebook. The keys are a work-in-progress to give students as many resources to improve as possible.}

\rule{\textwidth}{0.4pt}

\begin{enumerate}\litem{
Perform the division below. Then, find the intervals that correspond to the quotient in the form $ax^2+bx+c$ and remainder $r$.
\[ \frac{9x^{3} -27 x + 23}{x + 2} \]

The solution is \( 9x^{2} -18 x + 9 + \frac{5}{x + 2} \), which is option E.\begin{enumerate}[label=\Alph*.]
\item \( a \in [6, 15], b \in [14, 26], c \in [6, 13], \text{ and } r \in [39, 43]. \)

 You divided by the opposite of the factor.
\item \( a \in [6, 15], b \in [-32, -26], c \in [50, 62], \text{ and } r \in [-139, -137]. \)

 You multipled by the synthetic number and subtracted rather than adding during synthetic division.
\item \( a \in [-20, -13], b \in [-37, -32], c \in [-104, -98], \text{ and } r \in [-175, -173]. \)

 You divided by the opposite of the factor AND multipled the first factor rather than just bringing it down.
\item \( a \in [-20, -13], b \in [36, 39], c \in [-104, -98], \text{ and } r \in [220, 223]. \)

 You multipled by the synthetic number rather than bringing the first factor down.
\item \( a \in [6, 15], b \in [-18, -9], c \in [6, 13], \text{ and } r \in [0, 9]. \)

* This is the solution!
\end{enumerate}

\textbf{General Comment:} Be sure to synthetically divide by the zero of the denominator! Also, make sure to include 0 placeholders for missing terms.
}
\litem{
What are the \textit{possible Integer} roots of the polynomial below?
\[ f(x) = 5x^{3} +3 x^{2} +6 x + 3 \]

The solution is \( \pm 1,\pm 3 \), which is option A.\begin{enumerate}[label=\Alph*.]
\item \( \pm 1,\pm 3 \)

* This is the solution \textbf{since we asked for the possible Integer roots}!
\item \( \text{ All combinations of: }\frac{\pm 1,\pm 3}{\pm 1,\pm 5} \)

This would have been the solution \textbf{if asked for the possible Rational roots}!
\item \( \text{ All combinations of: }\frac{\pm 1,\pm 5}{\pm 1,\pm 3} \)

 Distractor 3: Corresponds to the plus or minus of the inverse quotient (an/a0) of the factors. 
\item \( \pm 1,\pm 5 \)

 Distractor 1: Corresponds to the plus or minus factors of a1 only.
\item \( \text{There is no formula or theorem that tells us all possible Integer roots.} \)

 Distractor 4: Corresponds to not recognizing Integers as a subset of Rationals.
\end{enumerate}

\textbf{General Comment:} We have a way to find the possible Rational roots. The possible Integer roots are the Integers in this list.
}
\litem{
Perform the division below. Then, find the intervals that correspond to the quotient in the form $ax^2+bx+c$ and remainder $r$.
\[ \frac{6x^{3} +21 x^{2} -30}{x + 3} \]

The solution is \( 6x^{2} +3 x -9 + \frac{-3}{x + 3} \), which is option E.\begin{enumerate}[label=\Alph*.]
\item \( a \in [-18, -14], b \in [-33, -30], c \in [-99, -97], \text{ and } r \in [-327, -324]. \)

 You divided by the opposite of the factor AND multipled the first factor rather than just bringing it down.
\item \( a \in [-1, 8], b \in [37, 44], c \in [116, 120], \text{ and } r \in [321, 324]. \)

 You divided by the opposite of the factor.
\item \( a \in [-18, -14], b \in [73, 78], c \in [-226, -223], \text{ and } r \in [641, 647]. \)

 You multipled by the synthetic number rather than bringing the first factor down.
\item \( a \in [-1, 8], b \in [-6, 1], c \in [10, 13], \text{ and } r \in [-79, -73]. \)

 You multipled by the synthetic number and subtracted rather than adding during synthetic division.
\item \( a \in [-1, 8], b \in [-1, 10], c \in [-14, -6], \text{ and } r \in [-7, -1]. \)

* This is the solution!
\end{enumerate}

\textbf{General Comment:} Be sure to synthetically divide by the zero of the denominator! Also, make sure to include 0 placeholders for missing terms.
}
\litem{
Perform the division below. Then, find the intervals that correspond to the quotient in the form $ax^2+bx+c$ and remainder $r$.
\[ \frac{20x^{3} -107 x^{2} +117 x -34}{x -4} \]

The solution is \( 20x^{2} -27 x + 9 + \frac{2}{x -4} \), which is option A.\begin{enumerate}[label=\Alph*.]
\item \( a \in [20, 23], \text{   } b \in [-30, -25], \text{   } c \in [7, 10], \text{   and   } r \in [-1, 7]. \)

* This is the solution!
\item \( a \in [20, 23], \text{   } b \in [-187, -185], \text{   } c \in [863, 866], \text{   and   } r \in [-3499, -3493]. \)

 You divided by the opposite of the factor.
\item \( a \in [75, 84], \text{   } b \in [-430, -426], \text{   } c \in [1821, 1828], \text{   and   } r \in [-7334, -7328]. \)

 You divided by the opposite of the factor AND multiplied the first factor rather than just bringing it down.
\item \( a \in [20, 23], \text{   } b \in [-52, -46], \text{   } c \in [-24, -22], \text{   and   } r \in [-108, -96]. \)

 You multiplied by the synthetic number and subtracted rather than adding during synthetic division.
\item \( a \in [75, 84], \text{   } b \in [206, 220], \text{   } c \in [966, 972], \text{   and   } r \in [3841, 3844]. \)

 You multiplied by the synthetic number rather than bringing the first factor down.
\end{enumerate}

\textbf{General Comment:} Be sure to synthetically divide by the zero of the denominator!
}
\litem{
What are the \textit{possible Rational} roots of the polynomial below?
\[ f(x) = 4x^{4} +2 x^{3} +7 x^{2} +6 x + 3 \]

The solution is \( \text{ All combinations of: }\frac{\pm 1,\pm 3}{\pm 1,\pm 2,\pm 4} \), which is option A.\begin{enumerate}[label=\Alph*.]
\item \( \text{ All combinations of: }\frac{\pm 1,\pm 3}{\pm 1,\pm 2,\pm 4} \)

* This is the solution \textbf{since we asked for the possible Rational roots}!
\item \( \pm 1,\pm 3 \)

This would have been the solution \textbf{if asked for the possible Integer roots}!
\item \( \pm 1,\pm 2,\pm 4 \)

 Distractor 1: Corresponds to the plus or minus factors of a1 only.
\item \( \text{ All combinations of: }\frac{\pm 1,\pm 2,\pm 4}{\pm 1,\pm 3} \)

 Distractor 3: Corresponds to the plus or minus of the inverse quotient (an/a0) of the factors. 
\item \( \text{ There is no formula or theorem that tells us all possible Rational roots.} \)

 Distractor 4: Corresponds to not recalling the theorem for rational roots of a polynomial.
\end{enumerate}

\textbf{General Comment:} We have a way to find the possible Rational roots. The possible Integer roots are the Integers in this list.
}
\litem{
Perform the division below. Then, find the intervals that correspond to the quotient in the form $ax^2+bx+c$ and remainder $r$.
\[ \frac{20x^{3} -54 x^{2} -96 x -36}{x -4} \]

The solution is \( 20x^{2} +26 x + 8 + \frac{-4}{x -4} \), which is option C.\begin{enumerate}[label=\Alph*.]
\item \( a \in [17, 23], \text{   } b \in [-136, -127], \text{   } c \in [436, 441], \text{   and   } r \in [-1798, -1793]. \)

 You divided by the opposite of the factor.
\item \( a \in [80, 83], \text{   } b \in [261, 270], \text{   } c \in [967, 969], \text{   and   } r \in [3833, 3837]. \)

 You multiplied by the synthetic number rather than bringing the first factor down.
\item \( a \in [17, 23], \text{   } b \in [26, 29], \text{   } c \in [5, 9], \text{   and   } r \in [-4, -2]. \)

* This is the solution!
\item \( a \in [17, 23], \text{   } b \in [4, 12], \text{   } c \in [-81, -75], \text{   and   } r \in [-271, -268]. \)

 You multiplied by the synthetic number and subtracted rather than adding during synthetic division.
\item \( a \in [80, 83], \text{   } b \in [-375, -370], \text{   } c \in [1397, 1403], \text{   and   } r \in [-5638, -5635]. \)

 You divided by the opposite of the factor AND multiplied the first factor rather than just bringing it down.
\end{enumerate}

\textbf{General Comment:} Be sure to synthetically divide by the zero of the denominator!
}
\litem{
Factor the polynomial below completely, knowing that $x+3$ is a factor. Then, choose the intervals the zeros of the polynomial belong to, where $z_1 \leq z_2 \leq z_3 \leq z_4$. \textit{To make the problem easier, all zeros are between -5 and 5.}
\[ f(x) = 20x^{4} +103 x^{3} -4 x^{2} -339 x + 180 \]

The solution is \( [-4, -3, 0.6, 1.25] \), which is option C.\begin{enumerate}[label=\Alph*.]
\item \( z_1 \in [-2.04, -1.58], \text{   }  z_2 \in [-0.86, -0.8], z_3 \in [2.92, 3.3], \text{   and   } z_4 \in [3.77, 4.03] \)

 Distractor 3: Corresponds to negatives of all zeros AND inversing rational roots.
\item \( z_1 \in [-3.47, -2.82], \text{   }  z_2 \in [-0.46, -0.21], z_3 \in [2.92, 3.3], \text{   and   } z_4 \in [3.77, 4.03] \)

 Distractor 4: Corresponds to moving factors from one rational to another.
\item \( z_1 \in [-4.67, -3.71], \text{   }  z_2 \in [-3.23, -2.91], z_3 \in [0.5, 0.64], \text{   and   } z_4 \in [0.85, 1.29] \)

* This is the solution!
\item \( z_1 \in [-4.67, -3.71], \text{   }  z_2 \in [-3.23, -2.91], z_3 \in [0.68, 1.09], \text{   and   } z_4 \in [1.31, 1.72] \)

 Distractor 2: Corresponds to inversing rational roots.
\item \( z_1 \in [-1.44, -0.82], \text{   }  z_2 \in [-0.79, -0.4], z_3 \in [2.92, 3.3], \text{   and   } z_4 \in [3.77, 4.03] \)

 Distractor 1: Corresponds to negatives of all zeros.
\end{enumerate}

\textbf{General Comment:} Remember to try the middle-most integers first as these normally are the zeros. Also, once you get it to a quadratic, you can use your other factoring techniques to finish factoring.
}
\litem{
Factor the polynomial below completely. Then, choose the intervals the zeros of the polynomial belong to, where $z_1 \leq z_2 \leq z_3$. \textit{To make the problem easier, all zeros are between -5 and 5.}
\[ f(x) = 8x^{3} -10 x^{2} -57 x + 45 \]

The solution is \( [-2.5, 0.75, 3] \), which is option D.\begin{enumerate}[label=\Alph*.]
\item \( z_1 \in [-3.07, -2.72], \text{   }  z_2 \in [-0.86, -0.71], \text{   and   } z_3 \in [2.34, 2.68] \)

 Distractor 1: Corresponds to negatives of all zeros.
\item \( z_1 \in [-3.07, -2.72], \text{   }  z_2 \in [-1.55, -1.2], \text{   and   } z_3 \in [-0.03, 0.76] \)

 Distractor 3: Corresponds to negatives of all zeros AND inversing rational roots.
\item \( z_1 \in [-0.52, -0.05], \text{   }  z_2 \in [1.01, 1.41], \text{   and   } z_3 \in [2.8, 3.19] \)

 Distractor 2: Corresponds to inversing rational roots.
\item \( z_1 \in [-2.84, -2.33], \text{   }  z_2 \in [0.4, 0.87], \text{   and   } z_3 \in [2.8, 3.19] \)

* This is the solution!
\item \( z_1 \in [-3.07, -2.72], \text{   }  z_2 \in [-0.63, -0.18], \text{   and   } z_3 \in [4.83, 5.39] \)

 Distractor 4: Corresponds to moving factors from one rational to another.
\end{enumerate}

\textbf{General Comment:} Remember to try the middle-most integers first as these normally are the zeros. Also, once you get it to a quadratic, you can use your other factoring techniques to finish factoring.
}
\litem{
Factor the polynomial below completely, knowing that $x-5$ is a factor. Then, choose the intervals the zeros of the polynomial belong to, where $z_1 \leq z_2 \leq z_3 \leq z_4$. \textit{To make the problem easier, all zeros are between -5 and 5.}
\[ f(x) = 12x^{4} -5 x^{3} -325 x^{2} +125 x + 625 \]

The solution is \( [-5, -1.25, 1.6666666666666667, 5] \), which is option A.\begin{enumerate}[label=\Alph*.]
\item \( z_1 \in [-5, -4], \text{   }  z_2 \in [-1.25, -1.18], z_3 \in [1.62, 1.74], \text{   and   } z_4 \in [5, 9] \)

* This is the solution!
\item \( z_1 \in [-5, -4], \text{   }  z_2 \in [-1.69, -1.55], z_3 \in [1.17, 1.47], \text{   and   } z_4 \in [5, 9] \)

 Distractor 1: Corresponds to negatives of all zeros.
\item \( z_1 \in [-5, -4], \text{   }  z_2 \in [-0.57, -0.41], z_3 \in [4.99, 5.11], \text{   and   } z_4 \in [5, 9] \)

 Distractor 4: Corresponds to moving factors from one rational to another.
\item \( z_1 \in [-5, -4], \text{   }  z_2 \in [-0.74, -0.58], z_3 \in [0.76, 0.81], \text{   and   } z_4 \in [5, 9] \)

 Distractor 3: Corresponds to negatives of all zeros AND inversing rational roots.
\item \( z_1 \in [-5, -4], \text{   }  z_2 \in [-0.94, -0.76], z_3 \in [0.49, 0.62], \text{   and   } z_4 \in [5, 9] \)

 Distractor 2: Corresponds to inversing rational roots.
\end{enumerate}

\textbf{General Comment:} Remember to try the middle-most integers first as these normally are the zeros. Also, once you get it to a quadratic, you can use your other factoring techniques to finish factoring.
}
\litem{
Factor the polynomial below completely. Then, choose the intervals the zeros of the polynomial belong to, where $z_1 \leq z_2 \leq z_3$. \textit{To make the problem easier, all zeros are between -5 and 5.}
\[ f(x) = 10x^{3} +39 x^{2} +18 x -27 \]

The solution is \( [-3, -1.5, 0.6] \), which is option D.\begin{enumerate}[label=\Alph*.]
\item \( z_1 \in [-1.67, -0.67], \text{   }  z_2 \in [0.52, 0.82], \text{   and   } z_3 \in [2, 3.2] \)

 Distractor 3: Corresponds to negatives of all zeros AND inversing rational roots.
\item \( z_1 \in [-3, -2], \text{   }  z_2 \in [0.04, 0.48], \text{   and   } z_3 \in [2, 3.2] \)

 Distractor 4: Corresponds to moving factors from one rational to another.
\item \( z_1 \in [-3, -2], \text{   }  z_2 \in [-0.8, -0.48], \text{   and   } z_3 \in [1.3, 1.7] \)

 Distractor 2: Corresponds to inversing rational roots.
\item \( z_1 \in [-3, -2], \text{   }  z_2 \in [-1.53, -1.4], \text{   and   } z_3 \in [0, 0.7] \)

* This is the solution!
\item \( z_1 \in [-0.6, 1.4], \text{   }  z_2 \in [1.33, 1.63], \text{   and   } z_3 \in [2, 3.2] \)

 Distractor 1: Corresponds to negatives of all zeros.
\end{enumerate}

\textbf{General Comment:} Remember to try the middle-most integers first as these normally are the zeros. Also, once you get it to a quadratic, you can use your other factoring techniques to finish factoring.
}
\end{enumerate}

\end{document}