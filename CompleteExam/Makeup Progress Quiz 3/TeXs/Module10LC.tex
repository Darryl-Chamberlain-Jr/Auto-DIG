\documentclass[14pt]{extbook}
\usepackage{multicol, enumerate, enumitem, hyperref, color, soul, setspace, parskip, fancyhdr} %General Packages
\usepackage{amssymb, amsthm, amsmath, latexsym, units, mathtools} %Math Packages
\everymath{\displaystyle} %All math in Display Style
% Packages with additional options
\usepackage[headsep=0.5cm,headheight=12pt, left=1 in,right= 1 in,top= 1 in,bottom= 1 in]{geometry}
\usepackage[usenames,dvipsnames]{xcolor}
\usepackage{dashrule}  % Package to use the command below to create lines between items
\newcommand{\litem}[1]{\item#1\hspace*{-1cm}\rule{\textwidth}{0.4pt}}
\pagestyle{fancy}
\lhead{Makeup Progress Quiz 3}
\chead{}
\rhead{Version C}
\lfoot{1648-1753}
\cfoot{}
\rfoot{Summer C 2021}
\begin{document}

\begin{enumerate}
\litem{
Factor the polynomial below completely. Then, choose the intervals the zeros of the polynomial belong to, where $z_1 \leq z_2 \leq z_3$. \textit{To make the problem easier, all zeros are between -5 and 5.}\[ f(x) = 15x^{3} +56 x^{2} +60 x + 16 \]\begin{enumerate}[label=\Alph*.]
\item \( z_1 \in [-2.37, -1.82], \text{   }  z_2 \in [-1.6, -0.7], \text{   and   } z_3 \in [-0.66, 0.01] \)
\item \( z_1 \in [-3.12, -2.21], \text{   }  z_2 \in [-2.2, -1.8], \text{   and   } z_3 \in [-1.16, -0.73] \)
\item \( z_1 \in [0.04, 0.25], \text{   }  z_2 \in [1.5, 2.3], \text{   and   } z_3 \in [3.59, 4.28] \)
\item \( z_1 \in [0.28, 0.66], \text{   }  z_2 \in [0, 1.8], \text{   and   } z_3 \in [1.8, 2.27] \)
\item \( z_1 \in [0.58, 0.8], \text{   }  z_2 \in [1.5, 2.3], \text{   and   } z_3 \in [2.39, 2.55] \)

\end{enumerate} }
\litem{
Perform the division below. Then, find the intervals that correspond to the quotient in the form $ax^2+bx+c$ and remainder $r$.\[ \frac{4x^{3} -28 x + 29}{x + 3} \]\begin{enumerate}[label=\Alph*.]
\item \( a \in [-13, -6], b \in [33.5, 37.2], c \in [-136, -135], \text{ and } r \in [434, 440]. \)
\item \( a \in [-3, 7], b \in [-16.3, -15.4], c \in [34, 40], \text{ and } r \in [-115, -110]. \)
\item \( a \in [-3, 7], b \in [11.9, 14.4], c \in [2, 15], \text{ and } r \in [50, 60]. \)
\item \( a \in [-13, -6], b \in [-38.6, -35.7], c \in [-136, -135], \text{ and } r \in [-383, -375]. \)
\item \( a \in [-3, 7], b \in [-12.1, -10.9], c \in [2, 15], \text{ and } r \in [3, 8]. \)

\end{enumerate} }
\litem{
Perform the division below. Then, find the intervals that correspond to the quotient in the form $ax^2+bx+c$ and remainder $r$.\[ \frac{20x^{3} -75 x^{2} +85 x -28}{x -2} \]\begin{enumerate}[label=\Alph*.]
\item \( a \in [19, 21], \text{   } b \in [-115, -113], \text{   } c \in [309, 317], \text{   and   } r \in [-661, -657]. \)
\item \( a \in [19, 21], \text{   } b \in [-55, -50], \text{   } c \in [27, 38], \text{   and   } r \in [0, 5]. \)
\item \( a \in [38, 44], \text{   } b \in [-161, -150], \text{   } c \in [395, 404], \text{   and   } r \in [-819, -814]. \)
\item \( a \in [38, 44], \text{   } b \in [5, 8], \text{   } c \in [93, 96], \text{   and   } r \in [161, 167]. \)
\item \( a \in [19, 21], \text{   } b \in [-43, -32], \text{   } c \in [14, 16], \text{   and   } r \in [0, 5]. \)

\end{enumerate} }
\litem{
Factor the polynomial below completely, knowing that $x -4$ is a factor. Then, choose the intervals the zeros of the polynomial belong to, where $z_1 \leq z_2 \leq z_3 \leq z_4$. \textit{To make the problem easier, all zeros are between -5 and 5.}\[ f(x) = 9x^{4} -54 x^{3} +47 x^{2} +150 x -200 \]\begin{enumerate}[label=\Alph*.]
\item \( z_1 \in [-1.2, 0.1], \text{   }  z_2 \in [-0.13, 1.43], z_3 \in [1.96, 2.19], \text{   and   } z_4 \in [3.9, 4.01] \)
\item \( z_1 \in [-4.2, -2.8], \text{   }  z_2 \in [-2.03, -1.86], z_3 \in [-0.6, -0.42], \text{   and   } z_4 \in [0.57, 0.63] \)
\item \( z_1 \in [-5.8, -4.5], \text{   }  z_2 \in [-4.7, -3.58], z_3 \in [-2.01, -1.7], \text{   and   } z_4 \in [0.47, 0.59] \)
\item \( z_1 \in [-2.4, -1.3], \text{   }  z_2 \in [0.99, 1.68], z_3 \in [1.96, 2.19], \text{   and   } z_4 \in [3.9, 4.01] \)
\item \( z_1 \in [-4.2, -2.8], \text{   }  z_2 \in [-2.03, -1.86], z_3 \in [-1.89, -1.42], \text{   and   } z_4 \in [1.56, 1.72] \)

\end{enumerate} }
\litem{
Factor the polynomial below completely, knowing that $x -2$ is a factor. Then, choose the intervals the zeros of the polynomial belong to, where $z_1 \leq z_2 \leq z_3 \leq z_4$. \textit{To make the problem easier, all zeros are between -5 and 5.}\[ f(x) = 6x^{4} -59 x^{3} +206 x^{2} -304 x + 160 \]\begin{enumerate}[label=\Alph*.]
\item \( z_1 \in [-4.28, -3.72], \text{   }  z_2 \in [-2.11, -1.43], z_3 \in [-0.78, -0.37], \text{   and   } z_4 \in [-0.51, 0.3] \)
\item \( z_1 \in [-4.28, -3.72], \text{   }  z_2 \in [-3.23, -2.11], z_3 \in [-2.42, -1.97], \text{   and   } z_4 \in [-2.07, -1.17] \)
\item \( z_1 \in [1.03, 1.82], \text{   }  z_2 \in [0.92, 2.24], z_3 \in [2.38, 2.55], \text{   and   } z_4 \in [3.29, 4.43] \)
\item \( z_1 \in [-0.57, 0.96], \text{   }  z_2 \in [-0.13, 1.04], z_3 \in [1.89, 2.08], \text{   and   } z_4 \in [3.29, 4.43] \)
\item \( z_1 \in [-4.28, -3.72], \text{   }  z_2 \in [-4.06, -3.6], z_3 \in [-2.42, -1.97], \text{   and   } z_4 \in [-0.89, -0.61] \)

\end{enumerate} }
\litem{
Perform the division below. Then, find the intervals that correspond to the quotient in the form $ax^2+bx+c$ and remainder $r$.\[ \frac{6x^{3} -2 x^{2} -44 x + 45}{x + 3} \]\begin{enumerate}[label=\Alph*.]
\item \( a \in [5, 9], \text{   } b \in [-23.4, -18.2], \text{   } c \in [15, 18], \text{   and   } r \in [-5, 1]. \)
\item \( a \in [-19, -17], \text{   } b \in [49.5, 54.2], \text{   } c \in [-209, -197], \text{   and   } r \in [644, 649]. \)
\item \( a \in [-19, -17], \text{   } b \in [-58.6, -53.9], \text{   } c \in [-217, -210], \text{   and   } r \in [-594, -589]. \)
\item \( a \in [5, 9], \text{   } b \in [15.1, 16.8], \text{   } c \in [-1, 11], \text{   and   } r \in [53, 60]. \)
\item \( a \in [5, 9], \text{   } b \in [-27, -25.5], \text{   } c \in [59, 63], \text{   and   } r \in [-195, -187]. \)

\end{enumerate} }
\litem{
Factor the polynomial below completely. Then, choose the intervals the zeros of the polynomial belong to, where $z_1 \leq z_2 \leq z_3$. \textit{To make the problem easier, all zeros are between -5 and 5.}\[ f(x) = 15x^{3} -89 x^{2} +62 x + 40 \]\begin{enumerate}[label=\Alph*.]
\item \( z_1 \in [-3.5, -1.5], \text{   }  z_2 \in [0.64, 0.8], \text{   and   } z_3 \in [4.8, 5.4] \)
\item \( z_1 \in [-5, -4], \text{   }  z_2 \in [-0.87, -0.74], \text{   and   } z_3 \in [2.1, 2.9] \)
\item \( z_1 \in [-5, -4], \text{   }  z_2 \in [-2.05, -1.27], \text{   and   } z_3 \in [0.3, 0.5] \)
\item \( z_1 \in [-5, -4], \text{   }  z_2 \in [-0.29, -0.11], \text{   and   } z_3 \in [1, 2.3] \)
\item \( z_1 \in [-1.4, 1.6], \text{   }  z_2 \in [0.97, 1.36], \text{   and   } z_3 \in [4.8, 5.4] \)

\end{enumerate} }
\litem{
What are the \textit{possible Rational} roots of the polynomial below?\[ f(x) = 5x^{4} +3 x^{3} +6 x^{2} +2 x + 6 \]\begin{enumerate}[label=\Alph*.]
\item \( \pm 1,\pm 5 \)
\item \( \text{ All combinations of: }\frac{\pm 1,\pm 5}{\pm 1,\pm 2,\pm 3,\pm 6} \)
\item \( \pm 1,\pm 2,\pm 3,\pm 6 \)
\item \( \text{ All combinations of: }\frac{\pm 1,\pm 2,\pm 3,\pm 6}{\pm 1,\pm 5} \)
\item \( \text{ There is no formula or theorem that tells us all possible Rational roots.} \)

\end{enumerate} }
\litem{
Perform the division below. Then, find the intervals that correspond to the quotient in the form $ax^2+bx+c$ and remainder $r$.\[ \frac{10x^{3} -30 x + 18}{x + 2} \]\begin{enumerate}[label=\Alph*.]
\item \( a \in [-20, -19], b \in [39, 42], c \in [-116, -107], \text{ and } r \in [237, 239]. \)
\item \( a \in [6, 14], b \in [-34, -29], c \in [58, 61], \text{ and } r \in [-164, -161]. \)
\item \( a \in [6, 14], b \in [-23, -13], c \in [8, 15], \text{ and } r \in [-7, 1]. \)
\item \( a \in [-20, -19], b \in [-42, -39], c \in [-116, -107], \text{ and } r \in [-202, -197]. \)
\item \( a \in [6, 14], b \in [14, 23], c \in [8, 15], \text{ and } r \in [36, 42]. \)

\end{enumerate} }
\litem{
What are the \textit{possible Integer} roots of the polynomial below?\[ f(x) = 5x^{3} +4 x^{2} +3 x + 4 \]\begin{enumerate}[label=\Alph*.]
\item \( \text{ All combinations of: }\frac{\pm 1,\pm 2,\pm 4}{\pm 1,\pm 5} \)
\item \( \text{ All combinations of: }\frac{\pm 1,\pm 5}{\pm 1,\pm 2,\pm 4} \)
\item \( \pm 1,\pm 5 \)
\item \( \pm 1,\pm 2,\pm 4 \)
\item \( \text{There is no formula or theorem that tells us all possible Integer roots.} \)

\end{enumerate} }
\end{enumerate}

\end{document}