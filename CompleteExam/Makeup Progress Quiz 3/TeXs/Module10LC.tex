\documentclass[14pt]{extbook}
\usepackage{multicol, enumerate, enumitem, hyperref, color, soul, setspace, parskip, fancyhdr} %General Packages
\usepackage{amssymb, amsthm, amsmath, bbm, latexsym, units, mathtools} %Math Packages
\everymath{\displaystyle} %All math in Display Style
% Packages with additional options
\usepackage[headsep=0.5cm,headheight=12pt, left=1 in,right= 1 in,top= 1 in,bottom= 1 in]{geometry}
\usepackage[usenames,dvipsnames]{xcolor}
\usepackage{dashrule}  % Package to use the command below to create lines between items
\newcommand{\litem}[1]{\item#1\hspace*{-1cm}\rule{\textwidth}{0.4pt}}
\pagestyle{fancy}
\lhead{Makeup Progress Quiz 3}
\chead{}
\rhead{Version C}
\lfoot{4315-3397}
\cfoot{}
\rfoot{Fall 2020}
\begin{document}

\begin{enumerate}
\litem{
Perform the division below. Then, find the intervals that correspond to the quotient in the form $ax^2+bx+c$ and remainder $r$.\[ \frac{12x^{3} -63 x^{2} + 77}{x -5} \]\begin{enumerate}[label=\Alph*.]
\item \( a \in [6, 15], b \in [-130, -120], c \in [604, 618], \text{ and } r \in [-2998, -2988]. \)
\item \( a \in [6, 15], b \in [-6, -1], c \in [-15, -14], \text{ and } r \in [2, 4]. \)
\item \( a \in [6, 15], b \in [-16, -14], c \in [-63, -54], \text{ and } r \in [-163, -160]. \)
\item \( a \in [57, 62], b \in [228, 238], c \in [1185, 1187], \text{ and } r \in [5999, 6003]. \)
\item \( a \in [57, 62], b \in [-365, -360], c \in [1814, 1822], \text{ and } r \in [-9003, -8995]. \)

\end{enumerate} }
\litem{
What are the \textit{possible Rational} roots of the polynomial below?\[ f(x) = 4x^{4} +4 x^{3} +2 x^{2} +3 x + 2 \]\begin{enumerate}[label=\Alph*.]
\item \( \text{ All combinations of: }\frac{\pm 1,\pm 2,\pm 4}{\pm 1,\pm 2} \)
\item \( \pm 1,\pm 2 \)
\item \( \pm 1,\pm 2,\pm 4 \)
\item \( \text{ All combinations of: }\frac{\pm 1,\pm 2}{\pm 1,\pm 2,\pm 4} \)
\item \( \text{ There is no formula or theorem that tells us all possible Rational roots.} \)

\end{enumerate} }
\litem{
Perform the division below. Then, find the intervals that correspond to the quotient in the form $ax^2+bx+c$ and remainder $r$.\[ \frac{25x^{3} -105 x^{2} + 83}{x -4} \]\begin{enumerate}[label=\Alph*.]
\item \( a \in [100, 103], b \in [295, 302], c \in [1174, 1190], \text{ and } r \in [4800, 4809]. \)
\item \( a \in [100, 103], b \in [-505, -499], c \in [2019, 2022], \text{ and } r \in [-7997, -7992]. \)
\item \( a \in [23, 33], b \in [-31, -28], c \in [-91, -89], \text{ and } r \in [-190, -184]. \)
\item \( a \in [23, 33], b \in [-8, -1], c \in [-20, -18], \text{ and } r \in [-5, 7]. \)
\item \( a \in [23, 33], b \in [-208, -202], c \in [817, 823], \text{ and } r \in [-3202, -3194]. \)

\end{enumerate} }
\litem{
Perform the division below. Then, find the intervals that correspond to the quotient in the form $ax^2+bx+c$ and remainder $r$.\[ \frac{25x^{3} -15 x^{2} -58 x -26}{x -2} \]\begin{enumerate}[label=\Alph*.]
\item \( a \in [50, 56], \text{   } b \in [-119, -112], \text{   } c \in [169, 173], \text{   and   } r \in [-373, -364]. \)
\item \( a \in [50, 56], \text{   } b \in [79, 90], \text{   } c \in [112, 113], \text{   and   } r \in [198, 205]. \)
\item \( a \in [24, 26], \text{   } b \in [34, 42], \text{   } c \in [11, 14], \text{   and   } r \in [-6, 2]. \)
\item \( a \in [24, 26], \text{   } b \in [10, 11], \text{   } c \in [-49, -45], \text{   and   } r \in [-76, -70]. \)
\item \( a \in [24, 26], \text{   } b \in [-70, -61], \text{   } c \in [70, 73], \text{   and   } r \in [-172, -166]. \)

\end{enumerate} }
\litem{
What are the \textit{possible Rational} roots of the polynomial below?\[ f(x) = 4x^{2} +5 x + 7 \]\begin{enumerate}[label=\Alph*.]
\item \( \pm 1,\pm 2,\pm 4 \)
\item \( \text{ All combinations of: }\frac{\pm 1,\pm 2,\pm 4}{\pm 1,\pm 7} \)
\item \( \text{ All combinations of: }\frac{\pm 1,\pm 7}{\pm 1,\pm 2,\pm 4} \)
\item \( \pm 1,\pm 7 \)
\item \( \text{ There is no formula or theorem that tells us all possible Rational roots.} \)

\end{enumerate} }
\litem{
Perform the division below. Then, find the intervals that correspond to the quotient in the form $ax^2+bx+c$ and remainder $r$.\[ \frac{10x^{3} +61 x^{2} +49 x -28}{x + 5} \]\begin{enumerate}[label=\Alph*.]
\item \( a \in [-52, -41], \text{   } b \in [-192, -185], \text{   } c \in [-896, -891], \text{   and   } r \in [-4511, -4507]. \)
\item \( a \in [5, 13], \text{   } b \in [104, 113], \text{   } c \in [601, 607], \text{   and   } r \in [2991, 2994]. \)
\item \( a \in [5, 13], \text{   } b \in [11, 14], \text{   } c \in [-8, -2], \text{   and   } r \in [2, 7]. \)
\item \( a \in [5, 13], \text{   } b \in [-5, 2], \text{   } c \in [40, 45], \text{   and   } r \in [-289, -283]. \)
\item \( a \in [-52, -41], \text{   } b \in [307, 316], \text{   } c \in [-1508, -1500], \text{   and   } r \in [7501, 7504]. \)

\end{enumerate} }
\litem{
Factor the polynomial below completely, knowing that $x-4$ is a factor. Then, choose the intervals the zeros of the polynomial belong to, where $z_1 \leq z_2 \leq z_3 \leq z_4$. \textit{To make the problem easier, all zeros are between -5 and 5.}\[ f(x) = 12x^{4} -115 x^{3} +381 x^{2} -512 x + 240 \]\begin{enumerate}[label=\Alph*.]
\item \( z_1 \in [-5.47, -4.86], \text{   }  z_2 \in [-5.17, -3.55], z_3 \in [-3.03, -2.94], \text{   and   } z_4 \in [-0.39, 0.33] \)
\item \( z_1 \in [1.22, 1.61], \text{   }  z_2 \in [1.31, 1.36], z_3 \in [2.75, 3.19], \text{   and   } z_4 \in [3.54, 4.41] \)
\item \( z_1 \in [-0.02, 0.89], \text{   }  z_2 \in [0.45, 0.97], z_3 \in [2.75, 3.19], \text{   and   } z_4 \in [3.54, 4.41] \)
\item \( z_1 \in [-4.33, -3.64], \text{   }  z_2 \in [-3.31, -2.03], z_3 \in [-2.18, -0.97], \text{   and   } z_4 \in [-1.67, -0.77] \)
\item \( z_1 \in [-4.33, -3.64], \text{   }  z_2 \in [-3.31, -2.03], z_3 \in [-1.05, -0.03], \text{   and   } z_4 \in [-1.08, -0.72] \)

\end{enumerate} }
\litem{
Factor the polynomial below completely. Then, choose the intervals the zeros of the polynomial belong to, where $z_1 \leq z_2 \leq z_3$. \textit{To make the problem easier, all zeros are between -5 and 5.}\[ f(x) = 6x^{3} +55 x^{2} +150 x + 125 \]\begin{enumerate}[label=\Alph*.]
\item \( z_1 \in [-5.83, -4.66], \text{   }  z_2 \in [-0.6, 0.4], \text{   and   } z_3 \in [-1.4, 0.6] \)
\item \( z_1 \in [1.35, 1.96], \text{   }  z_2 \in [1.5, 3.5], \text{   and   } z_3 \in [4, 7] \)
\item \( z_1 \in [-5.83, -4.66], \text{   }  z_2 \in [-2.5, -1.5], \text{   and   } z_3 \in [-1.67, -0.67] \)
\item \( z_1 \in [0.49, 1.39], \text{   }  z_2 \in [5, 6], \text{   and   } z_3 \in [4, 7] \)
\item \( z_1 \in [-0.16, 0.82], \text{   }  z_2 \in [-0.4, 1.6], \text{   and   } z_3 \in [4, 7] \)

\end{enumerate} }
\litem{
Factor the polynomial below completely, knowing that $x-2$ is a factor. Then, choose the intervals the zeros of the polynomial belong to, where $z_1 \leq z_2 \leq z_3 \leq z_4$. \textit{To make the problem easier, all zeros are between -5 and 5.}\[ f(x) = 20x^{4} -153 x^{3} +276 x^{2} -25 x -150 \]\begin{enumerate}[label=\Alph*.]
\item \( z_1 \in [-5.3, -2.5], \text{   }  z_2 \in [-2.38, -1.97], z_3 \in [-0.83, -0.49], \text{   and   } z_4 \in [1.29, 2.23] \)
\item \( z_1 \in [-0.9, 0.3], \text{   }  z_2 \in [1.18, 1.47], z_3 \in [1.91, 2.44], \text{   and   } z_4 \in [4.84, 5.29] \)
\item \( z_1 \in [-5.3, -2.5], \text{   }  z_2 \in [-5.23, -4.96], z_3 \in [-2.01, -1.59], \text{   and   } z_4 \in [-0.14, 0.28] \)
\item \( z_1 \in [-5.3, -2.5], \text{   }  z_2 \in [-2.38, -1.97], z_3 \in [-1.34, -1.18], \text{   and   } z_4 \in [0.41, 0.79] \)
\item \( z_1 \in [-1.8, -1.2], \text{   }  z_2 \in [0.73, 0.97], z_3 \in [1.91, 2.44], \text{   and   } z_4 \in [4.84, 5.29] \)

\end{enumerate} }
\litem{
Factor the polynomial below completely. Then, choose the intervals the zeros of the polynomial belong to, where $z_1 \leq z_2 \leq z_3$. \textit{To make the problem easier, all zeros are between -5 and 5.}\[ f(x) = 15x^{3} +53 x^{2} +8 x -48 \]\begin{enumerate}[label=\Alph*.]
\item \( z_1 \in [-0.95, -0.69], \text{   }  z_2 \in [1, 2.3], \text{   and   } z_3 \in [2.77, 3.19] \)
\item \( z_1 \in [-3.68, -2.98], \text{   }  z_2 \in [-1.3, -0.3], \text{   and   } z_3 \in [0.85, 2.37] \)
\item \( z_1 \in [-0.68, -0.21], \text{   }  z_2 \in [2.9, 3.6], \text{   and   } z_3 \in [3.97, 4.38] \)
\item \( z_1 \in [-3.68, -2.98], \text{   }  z_2 \in [-2, -1], \text{   and   } z_3 \in [0.15, 0.88] \)
\item \( z_1 \in [-1.52, -1.06], \text{   }  z_2 \in [-0.2, 1.1], \text{   and   } z_3 \in [2.77, 3.19] \)

\end{enumerate} }
\end{enumerate}

\end{document}