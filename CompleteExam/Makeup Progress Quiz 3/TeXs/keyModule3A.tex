\documentclass{extbook}[14pt]
\usepackage{multicol, enumerate, enumitem, hyperref, color, soul, setspace, parskip, fancyhdr, amssymb, amsthm, amsmath, bbm, latexsym, units, mathtools}
\everymath{\displaystyle}
\usepackage[headsep=0.5cm,headheight=0cm, left=1 in,right= 1 in,top= 1 in,bottom= 1 in]{geometry}
\usepackage{dashrule}  % Package to use the command below to create lines between items
\newcommand{\litem}[1]{\item #1

\rule{\textwidth}{0.4pt}}
\pagestyle{fancy}
\lhead{}
\chead{Answer Key for Makeup Progress Quiz 3 Version A}
\rhead{}
\lfoot{4315-3397}
\cfoot{}
\rfoot{Fall 2020}
\begin{document}
\textbf{This key should allow you to understand why you choose the option you did (beyond just getting a question right or wrong). \href{https://xronos.clas.ufl.edu/mac1105spring2020/courseDescriptionAndMisc/Exams/LearningFromResults}{More instructions on how to use this key can be found here}.}

\textbf{If you have a suggestion to make the keys better, \href{https://forms.gle/CZkbZmPbC9XALEE88}{please fill out the short survey here}.}

\textit{Note: This key is auto-generated and may contain issues and/or errors. The keys are reviewed after each exam to ensure grading is done accurately. If there are issues (like duplicate options), they are noted in the offline gradebook. The keys are a work-in-progress to give students as many resources to improve as possible.}

\rule{\textwidth}{0.4pt}

\begin{enumerate}\litem{
Solve the linear inequality below. Then, choose the constant and interval combination that describes the solution set.
\[ -9 + 5 x < \frac{46 x + 3}{7} \leq 3 + 6 x \]

The solution is \( (-6.00, 4.50] \), which is option C.\begin{enumerate}[label=\Alph*.]
\item \( (-\infty, a) \cup [b, \infty), \text{ where } a \in [-10, -3] \text{ and } b \in [-0.5, 6.5] \)

$(-\infty, -6.00) \cup [4.50, \infty)$, which corresponds to displaying the and-inequality as an or-inequality.
\item \( [a, b), \text{ where } a \in [-7, 0] \text{ and } b \in [3.5, 8.5] \)

$[-6.00, 4.50)$, which corresponds to flipping the inequality.
\item \( (a, b], \text{ where } a \in [-7, -3] \text{ and } b \in [1.5, 11.5] \)

* $(-6.00, 4.50]$, which is the correct option.
\item \( (-\infty, a] \cup (b, \infty), \text{ where } a \in [-10, -3] \text{ and } b \in [4.5, 5.5] \)

$(-\infty, -6.00] \cup (4.50, \infty)$, which corresponds to displaying the and-inequality as an or-inequality AND flipping the inequality.
\item \( \text{None of the above.} \)


\end{enumerate}

\textbf{General Comment:} To solve, you will need to break up the compound inequality into two inequalities. Be sure to keep track of the inequality! It may be best to draw a number line and graph your solution.
}
\litem{
Solve the linear inequality below. Then, choose the constant and interval combination that describes the solution set.
\[ -6 + 6 x > 9 x \text{ or } 6 + 5 x < 7 x \]

The solution is \( (-\infty, -2.0) \text{ or } (3.0, \infty) \), which is option B.\begin{enumerate}[label=\Alph*.]
\item \( (-\infty, a] \cup [b, \infty), \text{ where } a \in [-3.59, -2.22] \text{ and } b \in [0.3, 2.3] \)

Corresponds to including the endpoints AND negating.
\item \( (-\infty, a) \cup (b, \infty), \text{ where } a \in [-2.4, 0.5] \text{ and } b \in [2.9, 3.5] \)

 * Correct option.
\item \( (-\infty, a) \cup (b, \infty), \text{ where } a \in [-3.8, -2.7] \text{ and } b \in [1.66, 2.69] \)

Corresponds to inverting the inequality and negating the solution.
\item \( (-\infty, a] \cup [b, \infty), \text{ where } a \in [-2.78, -1.68] \text{ and } b \in [2.4, 3.3] \)

Corresponds to including the endpoints (when they should be excluded).
\item \( (-\infty, \infty) \)

Corresponds to the variable canceling, which does not happen in this instance.
\end{enumerate}

\textbf{General Comment:} When multiplying or dividing by a negative, flip the sign.
}
\litem{
Solve the linear inequality below. Then, choose the constant and interval combination that describes the solution set.
\[ -5 - 6 x < \frac{-34 x + 7}{6} \leq -5 - 9 x \]

The solution is \( (-18.50, -1.85] \), which is option D.\begin{enumerate}[label=\Alph*.]
\item \( [a, b), \text{ where } a \in [-18.5, -13.5] \text{ and } b \in [-1.85, 0.15] \)

$[-18.50, -1.85)$, which corresponds to flipping the inequality.
\item \( (-\infty, a] \cup (b, \infty), \text{ where } a \in [-19.5, -16.5] \text{ and } b \in [-5.85, -0.85] \)

$(-\infty, -18.50] \cup (-1.85, \infty)$, which corresponds to displaying the and-inequality as an or-inequality AND flipping the inequality.
\item \( (-\infty, a) \cup [b, \infty), \text{ where } a \in [-19.5, -13.5] \text{ and } b \in [-4.85, 1.15] \)

$(-\infty, -18.50) \cup [-1.85, \infty)$, which corresponds to displaying the and-inequality as an or-inequality.
\item \( (a, b], \text{ where } a \in [-24.5, -15.5] \text{ and } b \in [-1.85, -0.85] \)

* $(-18.50, -1.85]$, which is the correct option.
\item \( \text{None of the above.} \)


\end{enumerate}

\textbf{General Comment:} To solve, you will need to break up the compound inequality into two inequalities. Be sure to keep track of the inequality! It may be best to draw a number line and graph your solution.
}
\litem{
Solve the linear inequality below. Then, choose the constant and interval combination that describes the solution set.
\[ 6 + 3 x > 4 x \text{ or } 9 + 4 x < 5 x \]

The solution is \( (-\infty, 6.0) \text{ or } (9.0, \infty) \), which is option C.\begin{enumerate}[label=\Alph*.]
\item \( (-\infty, a] \cup [b, \infty), \text{ where } a \in [-11, -4] \text{ and } b \in [-10, -2] \)

Corresponds to including the endpoints AND negating.
\item \( (-\infty, a] \cup [b, \infty), \text{ where } a \in [4, 10] \text{ and } b \in [7, 11] \)

Corresponds to including the endpoints (when they should be excluded).
\item \( (-\infty, a) \cup (b, \infty), \text{ where } a \in [4, 14] \text{ and } b \in [9, 12] \)

 * Correct option.
\item \( (-\infty, a) \cup (b, \infty), \text{ where } a \in [-13, -8] \text{ and } b \in [-11, -4] \)

Corresponds to inverting the inequality and negating the solution.
\item \( (-\infty, \infty) \)

Corresponds to the variable canceling, which does not happen in this instance.
\end{enumerate}

\textbf{General Comment:} When multiplying or dividing by a negative, flip the sign.
}
\litem{
Solve the linear inequality below. Then, choose the constant and interval combination that describes the solution set.
\[ \frac{-7}{4} + \frac{9}{6} x > \frac{10}{5} x + \frac{6}{3} \]

The solution is \( (-\infty, -7.5) \), which is option A.\begin{enumerate}[label=\Alph*.]
\item \( (-\infty, a), \text{ where } a \in [-9.5, -5.5] \)

* $(-\infty, -7.5)$, which is the correct option.
\item \( (-\infty, a), \text{ where } a \in [5.5, 11.5] \)

 $(-\infty, 7.5)$, which corresponds to negating the endpoint of the solution.
\item \( (a, \infty), \text{ where } a \in [-9.5, -3.5] \)

 $(-7.5, \infty)$, which corresponds to switching the direction of the interval. You likely did this if you did not flip the inequality when dividing by a negative!
\item \( (a, \infty), \text{ where } a \in [5.5, 9.5] \)

 $(7.5, \infty)$, which corresponds to switching the direction of the interval AND negating the endpoint. You likely did this if you did not flip the inequality when dividing by a negative as well as not moving values over to a side properly.
\item \( \text{None of the above}. \)

You may have chosen this if you thought the inequality did not match the ends of the intervals.
\end{enumerate}

\textbf{General Comment:} Remember that less/greater than or equal to includes the endpoint, while less/greater do not. Also, remember that you need to flip the inequality when you multiply or divide by a negative.
}
\litem{
Solve the linear inequality below. Then, choose the constant and interval combination that describes the solution set.
\[ 6x -7 < 10x -8 \]

The solution is \( (0.25, \infty) \), which is option B.\begin{enumerate}[label=\Alph*.]
\item \( (-\infty, a), \text{ where } a \in [-0.86, 0.03] \)

 $(-\infty, -0.25)$, which corresponds to switching the direction of the interval AND negating the endpoint. You likely did this if you did not flip the inequality when dividing by a negative as well as not moving values over to a side properly.
\item \( (a, \infty), \text{ where } a \in [0.07, 0.81] \)

* $(0.25, \infty)$, which is the correct option.
\item \( (-\infty, a), \text{ where } a \in [-0.04, 0.49] \)

 $(-\infty, 0.25)$, which corresponds to switching the direction of the interval. You likely did this if you did not flip the inequality when dividing by a negative!
\item \( (a, \infty), \text{ where } a \in [-1, 0.11] \)

 $(-0.25, \infty)$, which corresponds to negating the endpoint of the solution.
\item \( \text{None of the above}. \)

You may have chosen this if you thought the inequality did not match the ends of the intervals.
\end{enumerate}

\textbf{General Comment:} Remember that less/greater than or equal to includes the endpoint, while less/greater do not. Also, remember that you need to flip the inequality when you multiply or divide by a negative.
}
\litem{
Using an interval or intervals, describe all the $x$-values within or including a distance of the given values.
\[ \text{ No less than } 2 \text{ units from the number } -4. \]

The solution is \( (-\infty, -6] \cup [-2, \infty) \), which is option B.\begin{enumerate}[label=\Alph*.]
\item \( (-\infty, -6) \cup (-2, \infty) \)

This describes the values more than 2 from -4
\item \( (-\infty, -6] \cup [-2, \infty) \)

This describes the values no less than 2 from -4
\item \( [-6, -2] \)

This describes the values no more than 2 from -4
\item \( (-6, -2) \)

This describes the values less than 2 from -4
\item \( \text{None of the above} \)

You likely thought the values in the interval were not correct.
\end{enumerate}

\textbf{General Comment:} When thinking about this language, it helps to draw a number line and try points.
}
\litem{
Using an interval or intervals, describe all the $x$-values within or including a distance of the given values.
\[ \text{ Less than } 10 \text{ units from the number } 2. \]

The solution is \( (-8, 12) \), which is option B.\begin{enumerate}[label=\Alph*.]
\item \( (-\infty, -8] \cup [12, \infty) \)

This describes the values no less than 10 from 2
\item \( (-8, 12) \)

This describes the values less than 10 from 2
\item \( [-8, 12] \)

This describes the values no more than 10 from 2
\item \( (-\infty, -8) \cup (12, \infty) \)

This describes the values more than 10 from 2
\item \( \text{None of the above} \)

You likely thought the values in the interval were not correct.
\end{enumerate}

\textbf{General Comment:} When thinking about this language, it helps to draw a number line and try points.
}
\litem{
Solve the linear inequality below. Then, choose the constant and interval combination that describes the solution set.
\[ \frac{-4}{9} - \frac{10}{8} x > \frac{-7}{7} x + \frac{8}{3} \]

The solution is \( (-\infty, -12.444) \), which is option B.\begin{enumerate}[label=\Alph*.]
\item \( (a, \infty), \text{ where } a \in [12.44, 14.44] \)

 $(12.444, \infty)$, which corresponds to switching the direction of the interval AND negating the endpoint. You likely did this if you did not flip the inequality when dividing by a negative as well as not moving values over to a side properly.
\item \( (-\infty, a), \text{ where } a \in [-15.44, -10.44] \)

* $(-\infty, -12.444)$, which is the correct option.
\item \( (-\infty, a), \text{ where } a \in [12.44, 16.44] \)

 $(-\infty, 12.444)$, which corresponds to negating the endpoint of the solution.
\item \( (a, \infty), \text{ where } a \in [-16.44, -11.44] \)

 $(-12.444, \infty)$, which corresponds to switching the direction of the interval. You likely did this if you did not flip the inequality when dividing by a negative!
\item \( \text{None of the above}. \)

You may have chosen this if you thought the inequality did not match the ends of the intervals.
\end{enumerate}

\textbf{General Comment:} Remember that less/greater than or equal to includes the endpoint, while less/greater do not. Also, remember that you need to flip the inequality when you multiply or divide by a negative.
}
\litem{
Solve the linear inequality below. Then, choose the constant and interval combination that describes the solution set.
\[ -6x -9 \leq 6x -3 \]

The solution is \( [-0.5, \infty) \), which is option A.\begin{enumerate}[label=\Alph*.]
\item \( [a, \infty), \text{ where } a \in [-0.6, -0.2] \)

* $[-0.5, \infty)$, which is the correct option.
\item \( (-\infty, a], \text{ where } a \in [0.44, 0.62] \)

 $(-\infty, 0.5]$, which corresponds to switching the direction of the interval AND negating the endpoint. You likely did this if you did not flip the inequality when dividing by a negative as well as not moving values over to a side properly.
\item \( [a, \infty), \text{ where } a \in [-0.2, 2.8] \)

 $[0.5, \infty)$, which corresponds to negating the endpoint of the solution.
\item \( (-\infty, a], \text{ where } a \in [-2.16, -0.23] \)

 $(-\infty, -0.5]$, which corresponds to switching the direction of the interval. You likely did this if you did not flip the inequality when dividing by a negative!
\item \( \text{None of the above}. \)

You may have chosen this if you thought the inequality did not match the ends of the intervals.
\end{enumerate}

\textbf{General Comment:} Remember that less/greater than or equal to includes the endpoint, while less/greater do not. Also, remember that you need to flip the inequality when you multiply or divide by a negative.
}
\end{enumerate}

\end{document}