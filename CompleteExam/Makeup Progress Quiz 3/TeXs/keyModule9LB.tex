\documentclass{extbook}[14pt]
\usepackage{multicol, enumerate, enumitem, hyperref, color, soul, setspace, parskip, fancyhdr, amssymb, amsthm, amsmath, bbm, latexsym, units, mathtools}
\everymath{\displaystyle}
\usepackage[headsep=0.5cm,headheight=0cm, left=1 in,right= 1 in,top= 1 in,bottom= 1 in]{geometry}
\usepackage{dashrule}  % Package to use the command below to create lines between items
\newcommand{\litem}[1]{\item #1

\rule{\textwidth}{0.4pt}}
\pagestyle{fancy}
\lhead{}
\chead{Answer Key for Makeup Progress Quiz 3 Version B}
\rhead{}
\lfoot{4315-3397}
\cfoot{}
\rfoot{Fall 2020}
\begin{document}
\textbf{This key should allow you to understand why you choose the option you did (beyond just getting a question right or wrong). \href{https://xronos.clas.ufl.edu/mac1105spring2020/courseDescriptionAndMisc/Exams/LearningFromResults}{More instructions on how to use this key can be found here}.}

\textbf{If you have a suggestion to make the keys better, \href{https://forms.gle/CZkbZmPbC9XALEE88}{please fill out the short survey here}.}

\textit{Note: This key is auto-generated and may contain issues and/or errors. The keys are reviewed after each exam to ensure grading is done accurately. If there are issues (like duplicate options), they are noted in the offline gradebook. The keys are a work-in-progress to give students as many resources to improve as possible.}

\rule{\textwidth}{0.4pt}

\begin{enumerate}\litem{
Find the inverse of the function below (if it exists). Then, evaluate the inverse at $x = 12$ and choose the interval that $f^{-1}(12)$ belongs to.
\[ f(x) = 4 x^2 + 3 \]

The solution is \( \text{ The function is not invertible for all Real numbers. } \), which is option E.\begin{enumerate}[label=\Alph*.]
\item \( f^{-1}(12) \in [3.29, 4.04] \)

 Distractor 4: This corresponds to both distractors 2 and 3.
\item \( f^{-1}(12) \in [1.45, 1.52] \)

 Distractor 1: This corresponds to trying to find the inverse even though the function is not 1-1. 
\item \( f^{-1}(12) \in [2.03, 3.45] \)

 Distractor 3: This corresponds to finding the (nonexistent) inverse and dividing by a negative.
\item \( f^{-1}(12) \in [1.76, 2.33] \)

 Distractor 2: This corresponds to finding the (nonexistent) inverse and not subtracting by the vertical shift.
\item \( \text{ The function is not invertible for all Real numbers. } \)

* This is the correct option.
\end{enumerate}

\textbf{General Comment:} Be sure you check that the function is 1-1 before trying to find the inverse!
}
\litem{
Multiply the following functions, then choose the domain of the resulting function from the list below.
\[ f(x) = \frac{2}{4x+29} \text{ and } g(x) = \frac{1}{3x+10} \]

The solution is \( \text{ The domain is all Real numbers except } x = -7.25 \text{ and } x = -3.3333333333333335 \), which is option D.\begin{enumerate}[label=\Alph*.]
\item \( \text{ The domain is all Real numbers greater than or equal to } x = a, \text{ where } a \in [-5, -2] \)


\item \( \text{ The domain is all Real numbers except } x = a, \text{ where } a \in [-3.25, -2.25] \)


\item \( \text{ The domain is all Real numbers less than or equal to } x = a, \text{ where } a \in [3, 5] \)


\item \( \text{ The domain is all Real numbers except } x = a \text{ and } x = b, \text{ where } a \in [-7.25, -3.25] \text{ and } b \in [-7.33, -1.33] \)


\item \( \text{ The domain is all Real numbers. } \)


\end{enumerate}

\textbf{General Comment:} The new domain is the intersection of the previous domains.
}
\litem{
Multiply the following functions, then choose the domain of the resulting function from the list below.
\[ f(x) = 5x^{3} +7 x^{2} + 1 \text{ and } g(x) = 8x + 7 \]

The solution is \( (-\infty, \infty) \), which is option E.\begin{enumerate}[label=\Alph*.]
\item \( \text{ The domain is all Real numbers less than or equal to } x = a, \text{ where } a \in [-4.67, -1.67] \)


\item \( \text{ The domain is all Real numbers greater than or equal to } x = a, \text{ where } a \in [1, 5] \)


\item \( \text{ The domain is all Real numbers except } x = a, \text{ where } a \in [1.4, 5.4] \)


\item \( \text{ The domain is all Real numbers except } x = a \text{ and } x = b, \text{ where } a \in [5.2, 8.2] \text{ and } b \in [-7.2, -3.2] \)


\item \( \text{ The domain is all Real numbers. } \)


\end{enumerate}

\textbf{General Comment:} The new domain is the intersection of the previous domains.
}
\litem{
Choose the interval below that $f$ composed with $g$ at $x=1$ is in.
\[ f(x) = -x^{3} +4 x^{2} -4 x \text{ and } g(x) = -3x^{3} +4 x^{2} -2 x \]

The solution is \( 9.0 \), which is option C.\begin{enumerate}[label=\Alph*.]
\item \( (f \circ g)(1) \in [12, 18] \)

 Distractor 3: Corresponds to being slightly off from the solution.
\item \( (f \circ g)(1) \in [-2, 4] \)

 Distractor 2: Corresponds to being slightly off from the solution.
\item \( (f \circ g)(1) \in [7, 11] \)

* This is the correct solution
\item \( (f \circ g)(1) \in [7, 11] \)

 Distractor 1: Corresponds to reversing the composition.
\item \( \text{It is not possible to compose the two functions.} \)


\end{enumerate}

\textbf{General Comment:} $f$ composed with $g$ at $x$ means $f(g(x))$. The order matters!
}
\litem{
Choose the interval below that $f$ composed with $g$ at $x=-2$ is in.
\[ f(x) = -2x^{3} -1 x^{2} +3 x -4 \text{ and } g(x) = x^{3} -1 x^{2} -4 x \]

The solution is \( 96.0 \), which is option A.\begin{enumerate}[label=\Alph*.]
\item \( (f \circ g)(-2) \in [95, 104] \)

* This is the correct solution
\item \( (f \circ g)(-2) \in [-8, -1] \)

 Distractor 1: Corresponds to reversing the composition.
\item \( (f \circ g)(-2) \in [85, 91] \)

 Distractor 2: Corresponds to being slightly off from the solution.
\item \( (f \circ g)(-2) \in [4, 12] \)

 Distractor 3: Corresponds to being slightly off from the solution.
\item \( \text{It is not possible to compose the two functions.} \)


\end{enumerate}

\textbf{General Comment:} $f$ composed with $g$ at $x$ means $f(g(x))$. The order matters!
}
\litem{
Find the inverse of the function below (if it exists). Then, evaluate the inverse at $x = 11$ and choose the interval that $f^{-1}(11)$ belongs to.
\[ f(x) = 5 x^2 + 2 \]

The solution is \( \text{ The function is not invertible for all Real numbers. } \), which is option E.\begin{enumerate}[label=\Alph*.]
\item \( f^{-1}(11) \in [4.54, 5.73] \)

 Distractor 4: This corresponds to both distractors 2 and 3.
\item \( f^{-1}(11) \in [2.68, 4.39] \)

 Distractor 3: This corresponds to finding the (nonexistent) inverse and dividing by a negative.
\item \( f^{-1}(11) \in [0.87, 1.6] \)

 Distractor 1: This corresponds to trying to find the inverse even though the function is not 1-1. 
\item \( f^{-1}(11) \in [1.39, 1.65] \)

 Distractor 2: This corresponds to finding the (nonexistent) inverse and not subtracting by the vertical shift.
\item \( \text{ The function is not invertible for all Real numbers. } \)

* This is the correct option.
\end{enumerate}

\textbf{General Comment:} Be sure you check that the function is 1-1 before trying to find the inverse!
}
\litem{
Determine whether the function below is 1-1.
\[ f(x) = 16 x^2 + 112 x + 196 \]

The solution is \( \text{no} \), which is option B.\begin{enumerate}[label=\Alph*.]
\item \( \text{No, because there is an $x$-value that goes to 2 different $y$-values.} \)

Corresponds to the Vertical Line test, which checks if an expression is a function.
\item \( \text{No, because there is a $y$-value that goes to 2 different $x$-values.} \)

* This is the solution.
\item \( \text{No, because the range of the function is not $(-\infty, \infty)$.} \)

Corresponds to believing 1-1 means the range is all Real numbers.
\item \( \text{Yes, the function is 1-1.} \)

Corresponds to believing the function passes the Horizontal Line test.
\item \( \text{No, because the domain of the function is not $(-\infty, \infty)$.} \)

Corresponds to believing 1-1 means the domain is all Real numbers.
\end{enumerate}

\textbf{General Comment:} There are only two valid options: The function is 1-1 OR No because there is a $y$-value that goes to 2 different $x$-values.
}
\litem{
Find the inverse of the function below. Then, evaluate the inverse at $x = 8$ and choose the interval that $f^{-1}(8)$ belongs to.
\[ f(x) = e^{x+4}+3 \]

The solution is \( f^{-1}(8) = -2.391 \), which is option B.\begin{enumerate}[label=\Alph*.]
\item \( f^{-1}(8) \in [5.38, 5.44] \)

 This solution corresponds to distractor 2.
\item \( f^{-1}(8) \in [-2.4, -2.36] \)

 This is the solution.
\item \( f^{-1}(8) \in [5.48, 5.53] \)

 This solution corresponds to distractor 4.
\item \( f^{-1}(8) \in [5.6, 5.63] \)

 This solution corresponds to distractor 1.
\item \( f^{-1}(8) \in [4.25, 4.39] \)

 This solution corresponds to distractor 3.
\end{enumerate}

\textbf{General Comment:} Natural log and exponential functions always have an inverse. Once you switch the $x$ and $y$, use the conversion $ e^y = x \leftrightarrow y=\ln(x)$.
}
\litem{
Determine whether the function below is 1-1.
\[ f(x) = (4 x + 14)^3 \]

The solution is \( \text{yes} \), which is option E.\begin{enumerate}[label=\Alph*.]
\item \( \text{No, because the domain of the function is not $(-\infty, \infty)$.} \)

Corresponds to believing 1-1 means the domain is all Real numbers.
\item \( \text{No, because there is an $x$-value that goes to 2 different $y$-values.} \)

Corresponds to the Vertical Line test, which checks if an expression is a function.
\item \( \text{No, because the range of the function is not $(-\infty, \infty)$.} \)

Corresponds to believing 1-1 means the range is all Real numbers.
\item \( \text{No, because there is a $y$-value that goes to 2 different $x$-values.} \)

Corresponds to the Horizontal Line test, which this function passes.
\item \( \text{Yes, the function is 1-1.} \)

* This is the solution.
\end{enumerate}

\textbf{General Comment:} There are only two valid options: The function is 1-1 OR No because there is a $y$-value that goes to 2 different $x$-values.
}
\litem{
Find the inverse of the function below. Then, evaluate the inverse at $x = 8$ and choose the interval that $f^{-1}(8)$ belongs to.
\[ f(x) = e^{x-5}+2 \]

The solution is \( f^{-1}(8) = 6.792 \), which is option A.\begin{enumerate}[label=\Alph*.]
\item \( f^{-1}(8) \in [6.77, 6.95] \)

 This is the solution.
\item \( f^{-1}(8) \in [4.14, 4.45] \)

 This solution corresponds to distractor 2.
\item \( f^{-1}(8) \in [2.84, 3.45] \)

 This solution corresponds to distractor 4.
\item \( f^{-1}(8) \in [4.34, 4.67] \)

 This solution corresponds to distractor 3.
\item \( f^{-1}(8) \in [-3.23, -3.06] \)

 This solution corresponds to distractor 1.
\end{enumerate}

\textbf{General Comment:} Natural log and exponential functions always have an inverse. Once you switch the $x$ and $y$, use the conversion $ e^y = x \leftrightarrow y=\ln(x)$.
}
\end{enumerate}

\end{document}