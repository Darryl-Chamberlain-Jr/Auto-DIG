\documentclass{extbook}[14pt]
\usepackage{multicol, enumerate, enumitem, hyperref, color, soul, setspace, parskip, fancyhdr, amssymb, amsthm, amsmath, latexsym, units, mathtools}
\everymath{\displaystyle}
\usepackage[headsep=0.5cm,headheight=0cm, left=1 in,right= 1 in,top= 1 in,bottom= 1 in]{geometry}
\usepackage{dashrule}  % Package to use the command below to create lines between items
\newcommand{\litem}[1]{\item #1

\rule{\textwidth}{0.4pt}}
\pagestyle{fancy}
\lhead{}
\chead{Answer Key for Makeup Progress Quiz 3 Version B}
\rhead{}
\lfoot{1648-1753}
\cfoot{}
\rfoot{Summer C 2021}
\begin{document}
\textbf{This key should allow you to understand why you choose the option you did (beyond just getting a question right or wrong). \href{https://xronos.clas.ufl.edu/mac1105spring2020/courseDescriptionAndMisc/Exams/LearningFromResults}{More instructions on how to use this key can be found here}.}

\textbf{If you have a suggestion to make the keys better, \href{https://forms.gle/CZkbZmPbC9XALEE88}{please fill out the short survey here}.}

\textit{Note: This key is auto-generated and may contain issues and/or errors. The keys are reviewed after each exam to ensure grading is done accurately. If there are issues (like duplicate options), they are noted in the offline gradebook. The keys are a work-in-progress to give students as many resources to improve as possible.}

\rule{\textwidth}{0.4pt}

\begin{enumerate}\litem{
Factor the polynomial below completely. Then, choose the intervals the zeros of the polynomial belong to, where $z_1 \leq z_2 \leq z_3$. \textit{To make the problem easier, all zeros are between -5 and 5.}
\[ f(x) = 9x^{3} -21 x^{2} -14 x + 40 \]The solution is \( [-1.33, 1.67, 2] \), which is option A.\begin{enumerate}[label=\Alph*.]
\item \( z_1 \in [-1.38, -0.8], \text{   }  z_2 \in [1.53, 1.81], \text{   and   } z_3 \in [1.79, 2.07] \)

* This is the solution!
\item \( z_1 \in [-5.06, -4.9], \text{   }  z_2 \in [-2.45, -1.94], \text{   and   } z_3 \in [0.12, 0.65] \)

 Distractor 4: Corresponds to moving factors from one rational to another.
\item \( z_1 \in [-2.51, -1.83], \text{   }  z_2 \in [-0.61, -0.47], \text{   and   } z_3 \in [0.65, 1] \)

 Distractor 3: Corresponds to negatives of all zeros AND inversing rational roots.
\item \( z_1 \in [-1.15, -0.16], \text{   }  z_2 \in [0.22, 0.94], \text{   and   } z_3 \in [1.79, 2.07] \)

 Distractor 2: Corresponds to inversing rational roots.
\item \( z_1 \in [-2.51, -1.83], \text{   }  z_2 \in [-1.7, -1.2], \text{   and   } z_3 \in [1.02, 1.56] \)

 Distractor 1: Corresponds to negatives of all zeros.
\end{enumerate}

\textbf{General Comment:} Remember to try the middle-most integers first as these normally are the zeros. Also, once you get it to a quadratic, you can use your other factoring techniques to finish factoring.
}
\litem{
Perform the division below. Then, find the intervals that correspond to the quotient in the form $ax^2+bx+c$ and remainder $r$.
\[ \frac{8x^{3} +28 x^{2} -39}{x + 3} \]The solution is \( 8x^{2} +4 x -12 + \frac{-3}{x + 3} \), which is option A.\begin{enumerate}[label=\Alph*.]
\item \( a \in [3, 15], b \in [2, 7], c \in [-16, -7], \text{ and } r \in [-7, 5]. \)

* This is the solution!
\item \( a \in [-28, -22], b \in [98, 105], c \in [-302, -297], \text{ and } r \in [859, 863]. \)

 You multipled by the synthetic number rather than bringing the first factor down.
\item \( a \in [3, 15], b \in [-7, 1], c \in [13, 19], \text{ and } r \in [-105, -99]. \)

 You multipled by the synthetic number and subtracted rather than adding during synthetic division.
\item \( a \in [-28, -22], b \in [-49, -43], c \in [-136, -131], \text{ and } r \in [-440, -428]. \)

 You divided by the opposite of the factor AND multipled the first factor rather than just bringing it down.
\item \( a \in [3, 15], b \in [51, 53], c \in [153, 159], \text{ and } r \in [428, 432]. \)

 You divided by the opposite of the factor.
\end{enumerate}

\textbf{General Comment:} Be sure to synthetically divide by the zero of the denominator! Also, make sure to include 0 placeholders for missing terms.
}
\litem{
Perform the division below. Then, find the intervals that correspond to the quotient in the form $ax^2+bx+c$ and remainder $r$.
\[ \frac{20x^{3} +72 x^{2} +28 x -20}{x + 3} \]The solution is \( 20x^{2} +12 x -8 + \frac{4}{x + 3} \), which is option E.\begin{enumerate}[label=\Alph*.]
\item \( a \in [-68, -56], \text{   } b \in [-109, -105], \text{   } c \in [-301, -291], \text{   and   } r \in [-908, -906]. \)

 You divided by the opposite of the factor AND multiplied the first factor rather than just bringing it down.
\item \( a \in [16, 24], \text{   } b \in [-10, -7], \text{   } c \in [58, 62], \text{   and   } r \in [-260, -257]. \)

 You multiplied by the synthetic number and subtracted rather than adding during synthetic division.
\item \( a \in [-68, -56], \text{   } b \in [251, 256], \text{   } c \in [-734, -727], \text{   and   } r \in [2164, 2169]. \)

 You multiplied by the synthetic number rather than bringing the first factor down.
\item \( a \in [16, 24], \text{   } b \in [132, 136], \text{   } c \in [423, 428], \text{   and   } r \in [1250, 1260]. \)

 You divided by the opposite of the factor.
\item \( a \in [16, 24], \text{   } b \in [7, 19], \text{   } c \in [-8, -3], \text{   and   } r \in [-1, 8]. \)

* This is the solution!
\end{enumerate}

\textbf{General Comment:} Be sure to synthetically divide by the zero of the denominator!
}
\litem{
Factor the polynomial below completely, knowing that $x + 5$ is a factor. Then, choose the intervals the zeros of the polynomial belong to, where $z_1 \leq z_2 \leq z_3 \leq z_4$. \textit{To make the problem easier, all zeros are between -5 and 5.}
\[ f(x) = 15x^{4} +11 x^{3} -257 x^{2} +297 x -90 \]The solution is \( [-5, 0.6, 0.667, 3] \), which is option E.\begin{enumerate}[label=\Alph*.]
\item \( z_1 \in [-3.8, -1.7], \text{   }  z_2 \in [-1.96, -1.25], z_3 \in [-1.68, -1.5], \text{   and   } z_4 \in [4.1, 6.1] \)

 Distractor 3: Corresponds to negatives of all zeros AND inversing rational roots.
\item \( z_1 \in [-3.8, -1.7], \text{   }  z_2 \in [-0.82, 0.09], z_3 \in [-0.64, -0.29], \text{   and   } z_4 \in [4.1, 6.1] \)

 Distractor 1: Corresponds to negatives of all zeros.
\item \( z_1 \in [-5.1, -3.6], \text{   }  z_2 \in [1.23, 1.89], z_3 \in [1.46, 1.93], \text{   and   } z_4 \in [1.7, 3.8] \)

 Distractor 2: Corresponds to inversing rational roots.
\item \( z_1 \in [-3.8, -1.7], \text{   }  z_2 \in [-2.05, -1.98], z_3 \in [-0.45, 0.33], \text{   and   } z_4 \in [4.1, 6.1] \)

 Distractor 4: Corresponds to moving factors from one rational to another.
\item \( z_1 \in [-5.1, -3.6], \text{   }  z_2 \in [-0.33, 1.48], z_3 \in [0.44, 0.93], \text{   and   } z_4 \in [1.7, 3.8] \)

* This is the solution!
\end{enumerate}

\textbf{General Comment:} Remember to try the middle-most integers first as these normally are the zeros. Also, once you get it to a quadratic, you can use your other factoring techniques to finish factoring.
}
\litem{
Factor the polynomial below completely, knowing that $x + 2$ is a factor. Then, choose the intervals the zeros of the polynomial belong to, where $z_1 \leq z_2 \leq z_3 \leq z_4$. \textit{To make the problem easier, all zeros are between -5 and 5.}
\[ f(x) = 20x^{4} +103 x^{3} +126 x^{2} -27 x -54 \]The solution is \( [-3, -2, -0.75, 0.6] \), which is option C.\begin{enumerate}[label=\Alph*.]
\item \( z_1 \in [-0.33, 0.04], \text{   }  z_2 \in [1.62, 2.17], z_3 \in [2.32, 3.41], \text{   and   } z_4 \in [2.81, 3.25] \)

 Distractor 4: Corresponds to moving factors from one rational to another.
\item \( z_1 \in [-3.06, -2.84], \text{   }  z_2 \in [-2.02, -1.92], z_3 \in [-1.34, -0.98], \text{   and   } z_4 \in [1.44, 1.82] \)

 Distractor 2: Corresponds to inversing rational roots.
\item \( z_1 \in [-3.06, -2.84], \text{   }  z_2 \in [-2.02, -1.92], z_3 \in [-1.15, -0.65], \text{   and   } z_4 \in [-0.07, 0.95] \)

* This is the solution!
\item \( z_1 \in [-1.75, -1.29], \text{   }  z_2 \in [1.23, 1.53], z_3 \in [1.95, 2.57], \text{   and   } z_4 \in [2.81, 3.25] \)

 Distractor 3: Corresponds to negatives of all zeros AND inversing rational roots.
\item \( z_1 \in [-0.71, -0.58], \text{   }  z_2 \in [0.64, 1.05], z_3 \in [1.95, 2.57], \text{   and   } z_4 \in [2.81, 3.25] \)

 Distractor 1: Corresponds to negatives of all zeros.
\end{enumerate}

\textbf{General Comment:} Remember to try the middle-most integers first as these normally are the zeros. Also, once you get it to a quadratic, you can use your other factoring techniques to finish factoring.
}
\litem{
Perform the division below. Then, find the intervals that correspond to the quotient in the form $ax^2+bx+c$ and remainder $r$.
\[ \frac{15x^{3} +101 x^{2} +138 x + 45}{x + 5} \]The solution is \( 15x^{2} +26 x + 8 + \frac{5}{x + 5} \), which is option D.\begin{enumerate}[label=\Alph*.]
\item \( a \in [-75, -72], \text{   } b \in [475, 477], \text{   } c \in [-2245, -2240], \text{   and   } r \in [11253, 11258]. \)

 You multiplied by the synthetic number rather than bringing the first factor down.
\item \( a \in [14, 16], \text{   } b \in [10, 15], \text{   } c \in [69, 78], \text{   and   } r \in [-389, -378]. \)

 You multiplied by the synthetic number and subtracted rather than adding during synthetic division.
\item \( a \in [14, 16], \text{   } b \in [174, 179], \text{   } c \in [1017, 1022], \text{   and   } r \in [5131, 5139]. \)

 You divided by the opposite of the factor.
\item \( a \in [14, 16], \text{   } b \in [25, 33], \text{   } c \in [4, 9], \text{   and   } r \in [1, 8]. \)

* This is the solution!
\item \( a \in [-75, -72], \text{   } b \in [-279, -271], \text{   } c \in [-1234, -1227], \text{   and   } r \in [-6115, -6111]. \)

 You divided by the opposite of the factor AND multiplied the first factor rather than just bringing it down.
\end{enumerate}

\textbf{General Comment:} Be sure to synthetically divide by the zero of the denominator!
}
\litem{
Factor the polynomial below completely. Then, choose the intervals the zeros of the polynomial belong to, where $z_1 \leq z_2 \leq z_3$. \textit{To make the problem easier, all zeros are between -5 and 5.}
\[ f(x) = 8x^{3} -26 x^{2} -5 x + 50 \]The solution is \( [-1.25, 2, 2.5] \), which is option C.\begin{enumerate}[label=\Alph*.]
\item \( z_1 \in [-0.81, -0.3], \text{   }  z_2 \in [0.3, 1.8], \text{   and   } z_3 \in [1.68, 2.12] \)

 Distractor 2: Corresponds to inversing rational roots.
\item \( z_1 \in [-5.59, -4.51], \text{   }  z_2 \in [-2.3, -1.5], \text{   and   } z_3 \in [0.41, 0.68] \)

 Distractor 4: Corresponds to moving factors from one rational to another.
\item \( z_1 \in [-1.3, -1.01], \text{   }  z_2 \in [1.6, 2.8], \text{   and   } z_3 \in [2.49, 2.51] \)

* This is the solution!
\item \( z_1 \in [-2.4, -1.96], \text{   }  z_2 \in [-0.8, -0.2], \text{   and   } z_3 \in [0.76, 1.14] \)

 Distractor 3: Corresponds to negatives of all zeros AND inversing rational roots.
\item \( z_1 \in [-2.51, -2.08], \text{   }  z_2 \in [-2.3, -1.5], \text{   and   } z_3 \in [1.22, 1.38] \)

 Distractor 1: Corresponds to negatives of all zeros.
\end{enumerate}

\textbf{General Comment:} Remember to try the middle-most integers first as these normally are the zeros. Also, once you get it to a quadratic, you can use your other factoring techniques to finish factoring.
}
\litem{
What are the \textit{possible Rational} roots of the polynomial below?
\[ f(x) = 3x^{3} +6 x^{2} +2 x + 5 \]The solution is \( \text{ All combinations of: }\frac{\pm 1,\pm 5}{\pm 1,\pm 3} \), which is option D.\begin{enumerate}[label=\Alph*.]
\item \( \pm 1,\pm 3 \)

 Distractor 1: Corresponds to the plus or minus factors of a1 only.
\item \( \text{ All combinations of: }\frac{\pm 1,\pm 3}{\pm 1,\pm 5} \)

 Distractor 3: Corresponds to the plus or minus of the inverse quotient (an/a0) of the factors. 
\item \( \pm 1,\pm 5 \)

This would have been the solution \textbf{if asked for the possible Integer roots}!
\item \( \text{ All combinations of: }\frac{\pm 1,\pm 5}{\pm 1,\pm 3} \)

* This is the solution \textbf{since we asked for the possible Rational roots}!
\item \( \text{ There is no formula or theorem that tells us all possible Rational roots.} \)

 Distractor 4: Corresponds to not recalling the theorem for rational roots of a polynomial.
\end{enumerate}

\textbf{General Comment:} We have a way to find the possible Rational roots. The possible Integer roots are the Integers in this list.
}
\litem{
Perform the division below. Then, find the intervals that correspond to the quotient in the form $ax^2+bx+c$ and remainder $r$.
\[ \frac{4x^{3} -14 x^{2} + 21}{x -3} \]The solution is \( 4x^{2} -2 x -6 + \frac{3}{x -3} \), which is option C.\begin{enumerate}[label=\Alph*.]
\item \( a \in [11, 17], b \in [17, 23], c \in [57, 72], \text{ and } r \in [217, 221]. \)

 You multipled by the synthetic number rather than bringing the first factor down.
\item \( a \in [3, 7], b \in [-6, -4], c \in [-16, -7], \text{ and } r \in [-8, -1]. \)

 You multipled by the synthetic number and subtracted rather than adding during synthetic division.
\item \( a \in [3, 7], b \in [-3, 1], c \in [-7, -3], \text{ and } r \in [-2, 4]. \)

* This is the solution!
\item \( a \in [11, 17], b \in [-55, -47], c \in [150, 152], \text{ and } r \in [-436, -428]. \)

 You divided by the opposite of the factor AND multipled the first factor rather than just bringing it down.
\item \( a \in [3, 7], b \in [-28, -21], c \in [74, 79], \text{ and } r \in [-215, -210]. \)

 You divided by the opposite of the factor.
\end{enumerate}

\textbf{General Comment:} Be sure to synthetically divide by the zero of the denominator! Also, make sure to include 0 placeholders for missing terms.
}
\litem{
What are the \textit{possible Integer} roots of the polynomial below?
\[ f(x) = 4x^{2} +7 x + 7 \]The solution is \( \pm 1,\pm 7 \), which is option C.\begin{enumerate}[label=\Alph*.]
\item \( \pm 1,\pm 2,\pm 4 \)

 Distractor 1: Corresponds to the plus or minus factors of a1 only.
\item \( \text{ All combinations of: }\frac{\pm 1,\pm 7}{\pm 1,\pm 2,\pm 4} \)

This would have been the solution \textbf{if asked for the possible Rational roots}!
\item \( \pm 1,\pm 7 \)

* This is the solution \textbf{since we asked for the possible Integer roots}!
\item \( \text{ All combinations of: }\frac{\pm 1,\pm 2,\pm 4}{\pm 1,\pm 7} \)

 Distractor 3: Corresponds to the plus or minus of the inverse quotient (an/a0) of the factors. 
\item \( \text{There is no formula or theorem that tells us all possible Integer roots.} \)

 Distractor 4: Corresponds to not recognizing Integers as a subset of Rationals.
\end{enumerate}

\textbf{General Comment:} We have a way to find the possible Rational roots. The possible Integer roots are the Integers in this list.
}
\end{enumerate}

\end{document}