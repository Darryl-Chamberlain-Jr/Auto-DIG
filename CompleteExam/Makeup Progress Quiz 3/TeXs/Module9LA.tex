\documentclass[14pt]{extbook}
\usepackage{multicol, enumerate, enumitem, hyperref, color, soul, setspace, parskip, fancyhdr} %General Packages
\usepackage{amssymb, amsthm, amsmath, latexsym, units, mathtools} %Math Packages
\everymath{\displaystyle} %All math in Display Style
% Packages with additional options
\usepackage[headsep=0.5cm,headheight=12pt, left=1 in,right= 1 in,top= 1 in,bottom= 1 in]{geometry}
\usepackage[usenames,dvipsnames]{xcolor}
\usepackage{dashrule}  % Package to use the command below to create lines between items
\newcommand{\litem}[1]{\item#1\hspace*{-1cm}\rule{\textwidth}{0.4pt}}
\pagestyle{fancy}
\lhead{Makeup Progress Quiz 3}
\chead{}
\rhead{Version A}
\lfoot{1648-1753}
\cfoot{}
\rfoot{Summer C 2021}
\begin{document}

\begin{enumerate}
\litem{
Determine whether the function below is 1-1.\[ f(x) = -24 x^2 - 12 x + 336 \]\begin{enumerate}[label=\Alph*.]
\item \( \text{No, because there is an $x$-value that goes to 2 different $y$-values.} \)
\item \( \text{No, because the domain of the function is not $(-\infty, \infty)$.} \)
\item \( \text{No, because there is a $y$-value that goes to 2 different $x$-values.} \)
\item \( \text{No, because the range of the function is not $(-\infty, \infty)$.} \)
\item \( \text{Yes, the function is 1-1.} \)

\end{enumerate} }
\litem{
Determine whether the function below is 1-1.\[ f(x) = 36 x^2 + 480 x + 1600 \]\begin{enumerate}[label=\Alph*.]
\item \( \text{No, because the domain of the function is not $(-\infty, \infty)$.} \)
\item \( \text{No, because there is an $x$-value that goes to 2 different $y$-values.} \)
\item \( \text{Yes, the function is 1-1.} \)
\item \( \text{No, because the range of the function is not $(-\infty, \infty)$.} \)
\item \( \text{No, because there is a $y$-value that goes to 2 different $x$-values.} \)

\end{enumerate} }
\litem{
Find the inverse of the function below (if it exists). Then, evaluate the inverse at $x = -10$ and choose the interval that $f^-1(-10)$ belongs to.\[ f(x) = \sqrt[3]{4 x + 5} \]\begin{enumerate}[label=\Alph*.]
\item \( f^{-1}(-10) \in [249.3, 253.6] \)
\item \( f^{-1}(-10) \in [-253.5, -249.2] \)
\item \( f^{-1}(-10) \in [246.5, 250.6] \)
\item \( f^{-1}(-10) \in [-250.2, -248.6] \)
\item \( \text{ The function is not invertible for all Real numbers. } \)

\end{enumerate} }
\litem{
Multiply the following functions, then choose the domain of the resulting function from the list below.\[ f(x) = 3x^{2} +x + 5 \text{ and } g(x) = 8x^{3} +5 x^{2} +5 x \]\begin{enumerate}[label=\Alph*.]
\item \( \text{ The domain is all Real numbers except } x = a, \text{ where } a \in [-10.25, 1.75] \)
\item \( \text{ The domain is all Real numbers less than or equal to } x = a, \text{ where } a \in [5.33, 12.33] \)
\item \( \text{ The domain is all Real numbers greater than or equal to } x = a, \text{ where } a \in [-13.67, -2.67] \)
\item \( \text{ The domain is all Real numbers except } x = a \text{ and } x = b, \text{ where } a \in [5.83, 7.83] \text{ and } b \in [4.67, 6.67] \)
\item \( \text{ The domain is all Real numbers. } \)

\end{enumerate} }
\litem{
Choose the interval below that $f$ composed with $g$ at $x=1$ is in.\[ f(x) = 2x^{3} -4 x^{2} +4 x \text{ and } g(x) = -2x^{3} +4 x^{2} +x + 1 \]\begin{enumerate}[label=\Alph*.]
\item \( (f \circ g)(1) \in [-8, 2] \)
\item \( (f \circ g)(1) \in [88, 95] \)
\item \( (f \circ g)(1) \in [1, 5] \)
\item \( (f \circ g)(1) \in [77, 87] \)
\item \( \text{It is not possible to compose the two functions.} \)

\end{enumerate} }
\litem{
Find the inverse of the function below. Then, evaluate the inverse at $x = 7$ and choose the interval that $f^-1(7)$ belongs to.\[ f(x) = e^{x-5}+3 \]\begin{enumerate}[label=\Alph*.]
\item \( f^{-1}(7) \in [2.62, 3.88] \)
\item \( f^{-1}(7) \in [-4.27, -3.07] \)
\item \( f^{-1}(7) \in [5.41, 5.89] \)
\item \( f^{-1}(7) \in [4.86, 5.34] \)
\item \( f^{-1}(7) \in [6.08, 7.06] \)

\end{enumerate} }
\litem{
Choose the interval below that $f$ composed with $g$ at $x=1$ is in.\[ f(x) = -2x^{3} + x^{2} -x \text{ and } g(x) = -2x^{3} -1 x^{2} -x + 4 \]\begin{enumerate}[label=\Alph*.]
\item \( (f \circ g)(1) \in [23.1, 25.2] \)
\item \( (f \circ g)(1) \in [8.9, 9.9] \)
\item \( (f \circ g)(1) \in [-1.3, 3.9] \)
\item \( (f \circ g)(1) \in [17.6, 18.8] \)
\item \( \text{It is not possible to compose the two functions.} \)

\end{enumerate} }
\litem{
Multiply the following functions, then choose the domain of the resulting function from the list below.\[ f(x) = 5x^{2} +8 x + 9 \text{ and } g(x) = 2x^{3} +4 x^{2} +x + 8 \]\begin{enumerate}[label=\Alph*.]
\item \( \text{ The domain is all Real numbers less than or equal to } x = a, \text{ where } a \in [-7.75, 2.25] \)
\item \( \text{ The domain is all Real numbers except } x = a, \text{ where } a \in [1.67, 10.67] \)
\item \( \text{ The domain is all Real numbers greater than or equal to } x = a, \text{ where } a \in [3.5, 8.5] \)
\item \( \text{ The domain is all Real numbers except } x = a \text{ and } x = b, \text{ where } a \in [3.2, 10.2] \text{ and } b \in [-8.67, -4.67] \)
\item \( \text{ The domain is all Real numbers. } \)

\end{enumerate} }
\litem{
Find the inverse of the function below (if it exists). Then, evaluate the inverse at $x = -10$ and choose the interval that $f^-1(-10)$ belongs to.\[ f(x) = 3 x^2 - 5 \]\begin{enumerate}[label=\Alph*.]
\item \( f^{-1}(-10) \in [1.29, 1.31] \)
\item \( f^{-1}(-10) \in [2.28, 2.31] \)
\item \( f^{-1}(-10) \in [3.27, 3.35] \)
\item \( f^{-1}(-10) \in [2.18, 2.29] \)
\item \( \text{ The function is not invertible for all Real numbers. } \)

\end{enumerate} }
\litem{
Find the inverse of the function below. Then, evaluate the inverse at $x = 9$ and choose the interval that $f^-1(9)$ belongs to.\[ f(x) = e^{x-5}+3 \]\begin{enumerate}[label=\Alph*.]
\item \( f^{-1}(9) \in [5.57, 5.67] \)
\item \( f^{-1}(9) \in [4.16, 4.4] \)
\item \( f^{-1}(9) \in [5.3, 5.53] \)
\item \( f^{-1}(9) \in [-3.25, -2.83] \)
\item \( f^{-1}(9) \in [6.79, 7.24] \)

\end{enumerate} }
\end{enumerate}

\end{document}