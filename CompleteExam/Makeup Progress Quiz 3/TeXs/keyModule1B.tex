\documentclass{extbook}[14pt]
\usepackage{multicol, enumerate, enumitem, hyperref, color, soul, setspace, parskip, fancyhdr, amssymb, amsthm, amsmath, latexsym, units, mathtools}
\everymath{\displaystyle}
\usepackage[headsep=0.5cm,headheight=0cm, left=1 in,right= 1 in,top= 1 in,bottom= 1 in]{geometry}
\usepackage{dashrule}  % Package to use the command below to create lines between items
\newcommand{\litem}[1]{\item #1

\rule{\textwidth}{0.4pt}}
\pagestyle{fancy}
\lhead{}
\chead{Answer Key for Makeup Progress Quiz 3 Version B}
\rhead{}
\lfoot{1648-1753}
\cfoot{}
\rfoot{Summer C 2021}
\begin{document}
\textbf{This key should allow you to understand why you choose the option you did (beyond just getting a question right or wrong). \href{https://xronos.clas.ufl.edu/mac1105spring2020/courseDescriptionAndMisc/Exams/LearningFromResults}{More instructions on how to use this key can be found here}.}

\textbf{If you have a suggestion to make the keys better, \href{https://forms.gle/CZkbZmPbC9XALEE88}{please fill out the short survey here}.}

\textit{Note: This key is auto-generated and may contain issues and/or errors. The keys are reviewed after each exam to ensure grading is done accurately. If there are issues (like duplicate options), they are noted in the offline gradebook. The keys are a work-in-progress to give students as many resources to improve as possible.}

\rule{\textwidth}{0.4pt}

\begin{enumerate}\litem{
Choose the \textbf{smallest} set of Complex numbers that the number below belongs to.
\[ \sqrt{\frac{1050}{0}}+\sqrt{210} i \]The solution is \( \text{Not a Complex Number} \), which is option A.\begin{enumerate}[label=\Alph*.]
\item \( \text{Not a Complex Number} \)

* This is the correct option!
\item \( \text{Rational} \)

These are numbers that can be written as fraction of Integers (e.g., -2/3 + 5)
\item \( \text{Irrational} \)

These cannot be written as a fraction of Integers. Remember: $\pi$ is not an Integer!
\item \( \text{Pure Imaginary} \)

This is a Complex number $(a+bi)$ that \textbf{only} has an imaginary part like $2i$.
\item \( \text{Nonreal Complex} \)

This is a Complex number $(a+bi)$ that is not Real (has $i$ as part of the number).
\end{enumerate}

\textbf{General Comment:} Be sure to simplify $i^2 = -1$. This may remove the imaginary portion for your number. If you are having trouble, you may want to look at the \textit{Subgroups of the Real Numbers} section.
}
\litem{
Choose the \textbf{smallest} set of Real numbers that the number below belongs to.
\[ -\sqrt{\frac{93636}{324}} \]The solution is \( \text{Integer} \), which is option E.\begin{enumerate}[label=\Alph*.]
\item \( \text{Whole} \)

These are the counting numbers with 0 (0, 1, 2, 3, ...)
\item \( \text{Irrational} \)

These cannot be written as a fraction of Integers.
\item \( \text{Rational} \)

These are numbers that can be written as fraction of Integers (e.g., -2/3)
\item \( \text{Not a Real number} \)

These are Nonreal Complex numbers \textbf{OR} things that are not numbers (e.g., dividing by 0).
\item \( \text{Integer} \)

* This is the correct option!
\end{enumerate}

\textbf{General Comment:} First, you \textbf{NEED} to simplify the expression. This question simplifies to $-306$. 
 
 Be sure you look at the simplified fraction and not just the decimal expansion. Numbers such as 13, 17, and 19 provide \textbf{long but repeating/terminating decimal expansions!} 
 
 The only ways to *not* be a Real number are: dividing by 0 or taking the square root of a negative number. 
 
 Irrational numbers are more than just square root of 3: adding or subtracting values from square root of 3 is also irrational.
}
\litem{
Choose the \textbf{smallest} set of Real numbers that the number below belongs to.
\[ \sqrt{\frac{193600}{400}} \]The solution is \( \text{Whole} \), which is option E.\begin{enumerate}[label=\Alph*.]
\item \( \text{Not a Real number} \)

These are Nonreal Complex numbers \textbf{OR} things that are not numbers (e.g., dividing by 0).
\item \( \text{Rational} \)

These are numbers that can be written as fraction of Integers (e.g., -2/3)
\item \( \text{Irrational} \)

These cannot be written as a fraction of Integers.
\item \( \text{Integer} \)

These are the negative and positive counting numbers (..., -3, -2, -1, 0, 1, 2, 3, ...)
\item \( \text{Whole} \)

* This is the correct option!
\end{enumerate}

\textbf{General Comment:} First, you \textbf{NEED} to simplify the expression. This question simplifies to $440$. 
 
 Be sure you look at the simplified fraction and not just the decimal expansion. Numbers such as 13, 17, and 19 provide \textbf{long but repeating/terminating decimal expansions!} 
 
 The only ways to *not* be a Real number are: dividing by 0 or taking the square root of a negative number. 
 
 Irrational numbers are more than just square root of 3: adding or subtracting values from square root of 3 is also irrational.
}
\litem{
Simplify the expression below into the form $a+bi$. Then, choose the intervals that $a$ and $b$ belong to.
\[ \frac{63 - 55 i}{3 + 2 i} \]The solution is \( 6.08  - 22.38 i \), which is option A.\begin{enumerate}[label=\Alph*.]
\item \( a \in [5.5, 7] \text{ and } b \in [-23.5, -21] \)

* $6.08  - 22.38 i$, which is the correct option.
\item \( a \in [19.5, 22.5] \text{ and } b \in [-28.5, -26.5] \)

 $21.00  - 27.50 i$, which corresponds to just dividing the first term by the first term and the second by the second.
\item \( a \in [5.5, 7] \text{ and } b \in [-293, -290.5] \)

 $6.08  - 291.00 i$, which corresponds to forgetting to multiply the conjugate by the numerator.
\item \( a \in [22, 24] \text{ and } b \in [-3.5, -2.5] \)

 $23.00  - 3.00 i$, which corresponds to forgetting to multiply the conjugate by the numerator and not computing the conjugate correctly.
\item \( a \in [78, 80.5] \text{ and } b \in [-23.5, -21] \)

 $79.00  - 22.38 i$, which corresponds to forgetting to multiply the conjugate by the numerator and using a plus instead of a minus in the denominator.
\end{enumerate}

\textbf{General Comment:} Multiply the numerator and denominator by the *conjugate* of the denominator, then simplify. For example, if we have $2+3i$, the conjugate is $2-3i$.
}
\litem{
Simplify the expression below and choose the interval the simplification is contained within.
\[ 12 - 5^2 + 17 \div 16 * 4 \div 20 \]The solution is \( -12.787 \), which is option A.\begin{enumerate}[label=\Alph*.]
\item \( [-12.81, -12.66] \)

* -12.787, this is the correct option
\item \( [37, 37.09] \)

 37.013, which corresponds to two Order of Operations errors.
\item \( [-13.16, -12.83] \)

 -12.987, which corresponds to an Order of Operations error: not reading left-to-right for multiplication/division.
\item \( [37.09, 37.36] \)

 37.212, which corresponds to an Order of Operations error: multiplying by negative before squaring. For example: $(-3)^2 \neq -3^2$
\item \( \text{None of the above} \)

 You may have gotten this by making an unanticipated error. If you got a value that is not any of the others, please let the coordinator know so they can help you figure out what happened.
\end{enumerate}

\textbf{General Comment:} While you may remember (or were taught) PEMDAS is done in order, it is actually done as P/E/MD/AS. When we are at MD or AS, we read left to right.
}
\litem{
Simplify the expression below into the form $a+bi$. Then, choose the intervals that $a$ and $b$ belong to.
\[ \frac{72 + 66 i}{-7 - i} \]The solution is \( -11.40  - 7.80 i \), which is option D.\begin{enumerate}[label=\Alph*.]
\item \( a \in [-9.5, -8] \text{ and } b \in [-12, -9] \)

 $-8.76  - 10.68 i$, which corresponds to forgetting to multiply the conjugate by the numerator and not computing the conjugate correctly.
\item \( a \in [-12, -10.5] \text{ and } b \in [-390.5, -389] \)

 $-11.40  - 390.00 i$, which corresponds to forgetting to multiply the conjugate by the numerator.
\item \( a \in [-571, -569.5] \text{ and } b \in [-8, -7] \)

 $-570.00  - 7.80 i$, which corresponds to forgetting to multiply the conjugate by the numerator and using a plus instead of a minus in the denominator.
\item \( a \in [-12, -10.5] \text{ and } b \in [-8, -7] \)

* $-11.40  - 7.80 i$, which is the correct option.
\item \( a \in [-11, -9] \text{ and } b \in [-68, -65.5] \)

 $-10.29  - 66.00 i$, which corresponds to just dividing the first term by the first term and the second by the second.
\end{enumerate}

\textbf{General Comment:} Multiply the numerator and denominator by the *conjugate* of the denominator, then simplify. For example, if we have $2+3i$, the conjugate is $2-3i$.
}
\litem{
Choose the \textbf{smallest} set of Complex numbers that the number below belongs to.
\[ \sqrt{\frac{2057}{11}}+\sqrt{110} i \]The solution is \( \text{Nonreal Complex} \), which is option D.\begin{enumerate}[label=\Alph*.]
\item \( \text{Pure Imaginary} \)

This is a Complex number $(a+bi)$ that \textbf{only} has an imaginary part like $2i$.
\item \( \text{Irrational} \)

These cannot be written as a fraction of Integers. Remember: $\pi$ is not an Integer!
\item \( \text{Rational} \)

These are numbers that can be written as fraction of Integers (e.g., -2/3 + 5)
\item \( \text{Nonreal Complex} \)

* This is the correct option!
\item \( \text{Not a Complex Number} \)

This is not a number. The only non-Complex number we know is dividing by 0 as this is not a number!
\end{enumerate}

\textbf{General Comment:} Be sure to simplify $i^2 = -1$. This may remove the imaginary portion for your number. If you are having trouble, you may want to look at the \textit{Subgroups of the Real Numbers} section.
}
\litem{
Simplify the expression below and choose the interval the simplification is contained within.
\[ 19 - 4^2 + 9 \div 3 * 17 \div 16 \]The solution is \( 6.188 \), which is option C.\begin{enumerate}[label=\Alph*.]
\item \( [33.42, 35.53] \)

 35.011, which corresponds to two Order of Operations errors.
\item \( [2.16, 5.17] \)

 3.011, which corresponds to an Order of Operations error: not reading left-to-right for multiplication/division.
\item \( [5.8, 7.07] \)

* 6.188, this is the correct option
\item \( [37.78, 38.61] \)

 38.188, which corresponds to an Order of Operations error: multiplying by negative before squaring. For example: $(-3)^2 \neq -3^2$
\item \( \text{None of the above} \)

 You may have gotten this by making an unanticipated error. If you got a value that is not any of the others, please let the coordinator know so they can help you figure out what happened.
\end{enumerate}

\textbf{General Comment:} While you may remember (or were taught) PEMDAS is done in order, it is actually done as P/E/MD/AS. When we are at MD or AS, we read left to right.
}
\litem{
Simplify the expression below into the form $a+bi$. Then, choose the intervals that $a$ and $b$ belong to.
\[ (-5 + 10 i)(-9 - 2 i) \]The solution is \( 65 - 80 i \), which is option E.\begin{enumerate}[label=\Alph*.]
\item \( a \in [23, 27] \text{ and } b \in [-103, -98] \)

 $25 - 100 i$, which corresponds to adding a minus sign in the second term.
\item \( a \in [45, 46] \text{ and } b \in [-20, -19] \)

 $45 - 20 i$, which corresponds to just multiplying the real terms to get the real part of the solution and the coefficients in the complex terms to get the complex part.
\item \( a \in [23, 27] \text{ and } b \in [93, 104] \)

 $25 + 100 i$, which corresponds to adding a minus sign in the first term.
\item \( a \in [64, 66] \text{ and } b \in [78, 86] \)

 $65 + 80 i$, which corresponds to adding a minus sign in both terms.
\item \( a \in [64, 66] \text{ and } b \in [-89, -79] \)

* $65 - 80 i$, which is the correct option.
\end{enumerate}

\textbf{General Comment:} You can treat $i$ as a variable and distribute. Just remember that $i^2=-1$, so you can continue to reduce after you distribute.
}
\litem{
Simplify the expression below into the form $a+bi$. Then, choose the intervals that $a$ and $b$ belong to.
\[ (-6 - 2 i)(4 + 9 i) \]The solution is \( -6 - 62 i \), which is option D.\begin{enumerate}[label=\Alph*.]
\item \( a \in [-42, -38] \text{ and } b \in [42, 47] \)

 $-42 + 46 i$, which corresponds to adding a minus sign in the second term.
\item \( a \in [-27, -22] \text{ and } b \in [-20, -17] \)

 $-24 - 18 i$, which corresponds to just multiplying the real terms to get the real part of the solution and the coefficients in the complex terms to get the complex part.
\item \( a \in [-11, -3] \text{ and } b \in [59, 64] \)

 $-6 + 62 i$, which corresponds to adding a minus sign in both terms.
\item \( a \in [-11, -3] \text{ and } b \in [-64, -59] \)

* $-6 - 62 i$, which is the correct option.
\item \( a \in [-42, -38] \text{ and } b \in [-47, -42] \)

 $-42 - 46 i$, which corresponds to adding a minus sign in the first term.
\end{enumerate}

\textbf{General Comment:} You can treat $i$ as a variable and distribute. Just remember that $i^2=-1$, so you can continue to reduce after you distribute.
}
\end{enumerate}

\end{document}