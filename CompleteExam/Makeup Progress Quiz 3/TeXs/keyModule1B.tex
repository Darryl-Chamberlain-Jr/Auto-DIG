\documentclass{extbook}[14pt]
\usepackage{multicol, enumerate, enumitem, hyperref, color, soul, setspace, parskip, fancyhdr, amssymb, amsthm, amsmath, bbm, latexsym, units, mathtools}
\everymath{\displaystyle}
\usepackage[headsep=0.5cm,headheight=0cm, left=1 in,right= 1 in,top= 1 in,bottom= 1 in]{geometry}
\usepackage{dashrule}  % Package to use the command below to create lines between items
\newcommand{\litem}[1]{\item #1

\rule{\textwidth}{0.4pt}}
\pagestyle{fancy}
\lhead{}
\chead{Answer Key for Makeup Progress Quiz 3 Version B}
\rhead{}
\lfoot{4315-3397}
\cfoot{}
\rfoot{Fall 2020}
\begin{document}
\textbf{This key should allow you to understand why you choose the option you did (beyond just getting a question right or wrong). \href{https://xronos.clas.ufl.edu/mac1105spring2020/courseDescriptionAndMisc/Exams/LearningFromResults}{More instructions on how to use this key can be found here}.}

\textbf{If you have a suggestion to make the keys better, \href{https://forms.gle/CZkbZmPbC9XALEE88}{please fill out the short survey here}.}

\textit{Note: This key is auto-generated and may contain issues and/or errors. The keys are reviewed after each exam to ensure grading is done accurately. If there are issues (like duplicate options), they are noted in the offline gradebook. The keys are a work-in-progress to give students as many resources to improve as possible.}

\rule{\textwidth}{0.4pt}

\begin{enumerate}\litem{
Choose the \textbf{smallest} set of Real numbers that the number below belongs to.
\[ -\sqrt{\frac{160000}{256}} \]

The solution is \( \text{Integer} \), which is option D.\begin{enumerate}[label=\Alph*.]
\item \( \text{Rational} \)

These are numbers that can be written as fraction of Integers (e.g., -2/3)
\item \( \text{Whole} \)

These are the counting numbers with 0 (0, 1, 2, 3, ...)
\item \( \text{Irrational} \)

These cannot be written as a fraction of Integers.
\item \( \text{Integer} \)

* This is the correct option!
\item \( \text{Not a Real number} \)

These are Nonreal Complex numbers \textbf{OR} things that are not numbers (e.g., dividing by 0).
\end{enumerate}

\textbf{General Comment:} First, you \textbf{NEED} to simplify the expression. This question simplifies to $-400$. 
 
 Be sure you look at the simplified fraction and not just the decimal expansion. Numbers such as 13, 17, and 19 provide \textbf{long but repeating/terminating decimal expansions!} 
 
 The only ways to *not* be a Real number are: dividing by 0 or taking the square root of a negative number. 
 
 Irrational numbers are more than just square root of 3: adding or subtracting values from square root of 3 is also irrational.
}
\litem{
Choose the \textbf{smallest} set of Complex numbers that the number below belongs to.
\[ \frac{\sqrt{70}}{9}+\sqrt{-3}i \]

The solution is \( \text{Irrational} \), which is option D.\begin{enumerate}[label=\Alph*.]
\item \( \text{Nonreal Complex} \)

This is a Complex number $(a+bi)$ that is not Real (has $i$ as part of the number).
\item \( \text{Not a Complex Number} \)

This is not a number. The only non-Complex number we know is dividing by 0 as this is not a number!
\item \( \text{Rational} \)

These are numbers that can be written as fraction of Integers (e.g., -2/3 + 5)
\item \( \text{Irrational} \)

* This is the correct option!
\item \( \text{Pure Imaginary} \)

This is a Complex number $(a+bi)$ that \textbf{only} has an imaginary part like $2i$.
\end{enumerate}

\textbf{General Comment:} Be sure to simplify $i^2 = -1$. This may remove the imaginary portion for your number. If you are having trouble, you may want to look at the \textit{Subgroups of the Real Numbers} section.
}
\litem{
Choose the \textbf{smallest} set of Real numbers that the number below belongs to.
\[ -\sqrt{\frac{765}{9}} \]

The solution is \( \text{Irrational} \), which is option C.\begin{enumerate}[label=\Alph*.]
\item \( \text{Rational} \)

These are numbers that can be written as fraction of Integers (e.g., -2/3)
\item \( \text{Whole} \)

These are the counting numbers with 0 (0, 1, 2, 3, ...)
\item \( \text{Irrational} \)

* This is the correct option!
\item \( \text{Not a Real number} \)

These are Nonreal Complex numbers \textbf{OR} things that are not numbers (e.g., dividing by 0).
\item \( \text{Integer} \)

These are the negative and positive counting numbers (..., -3, -2, -1, 0, 1, 2, 3, ...)
\end{enumerate}

\textbf{General Comment:} First, you \textbf{NEED} to simplify the expression. This question simplifies to $-\sqrt{85}$. 
 
 Be sure you look at the simplified fraction and not just the decimal expansion. Numbers such as 13, 17, and 19 provide \textbf{long but repeating/terminating decimal expansions!} 
 
 The only ways to *not* be a Real number are: dividing by 0 or taking the square root of a negative number. 
 
 Irrational numbers are more than just square root of 3: adding or subtracting values from square root of 3 is also irrational.
}
\litem{
Simplify the expression below and choose the interval the simplification is contained within.
\[ 16 - 10^2 + 18 \div 1 * 7 \div 12 \]

The solution is \( -73.500 \), which is option A.\begin{enumerate}[label=\Alph*.]
\item \( [-76.5, -70.5] \)

* -73.500, this is the correct option
\item \( [-84.79, -82.79] \)

 -83.786, which corresponds to an Order of Operations error: not reading left-to-right for multiplication/division.
\item \( [121.5, 130.5] \)

 126.500, which corresponds to an Order of Operations error: multiplying by negative before squaring. For example: $(-3)^2 \neq -3^2$
\item \( [114.21, 123.21] \)

 116.214, which corresponds to two Order of Operations errors.
\item \( \text{None of the above} \)

 You may have gotten this by making an unanticipated error. If you got a value that is not any of the others, please let the coordinator know so they can help you figure out what happened.
\end{enumerate}

\textbf{General Comment:} While you may remember (or were taught) PEMDAS is done in order, it is actually done as P/E/MD/AS. When we are at MD or AS, we read left to right.
}
\litem{
Choose the \textbf{smallest} set of Complex numbers that the number below belongs to.
\[ -\sqrt{\frac{1260}{15}}+2i^2 \]

The solution is \( \text{Irrational} \), which is option B.\begin{enumerate}[label=\Alph*.]
\item \( \text{Rational} \)

These are numbers that can be written as fraction of Integers (e.g., -2/3 + 5)
\item \( \text{Irrational} \)

* This is the correct option!
\item \( \text{Not a Complex Number} \)

This is not a number. The only non-Complex number we know is dividing by 0 as this is not a number!
\item \( \text{Nonreal Complex} \)

This is a Complex number $(a+bi)$ that is not Real (has $i$ as part of the number).
\item \( \text{Pure Imaginary} \)

This is a Complex number $(a+bi)$ that \textbf{only} has an imaginary part like $2i$.
\end{enumerate}

\textbf{General Comment:} Be sure to simplify $i^2 = -1$. This may remove the imaginary portion for your number. If you are having trouble, you may want to look at the \textit{Subgroups of the Real Numbers} section.
}
\litem{
Simplify the expression below into the form $a+bi$. Then, choose the intervals that $a$ and $b$ belong to.
\[ \frac{18 - 77 i}{8 - i} \]

The solution is \( 3.40  - 9.20 i \), which is option C.\begin{enumerate}[label=\Alph*.]
\item \( a \in [220.5, 221.5] \text{ and } b \in [-9.65, -8.95] \)

 $221.00  - 9.20 i$, which corresponds to forgetting to multiply the conjugate by the numerator and using a plus instead of a minus in the denominator.
\item \( a \in [0.5, 1.5] \text{ and } b \in [-10, -9.4] \)

 $1.03  - 9.75 i$, which corresponds to forgetting to multiply the conjugate by the numerator and not computing the conjugate correctly.
\item \( a \in [3, 4.5] \text{ and } b \in [-9.65, -8.95] \)

* $3.40  - 9.20 i$, which is the correct option.
\item \( a \in [1.5, 3] \text{ and } b \in [76.8, 77.05] \)

 $2.25  + 77.00 i$, which corresponds to just dividing the first term by the first term and the second by the second.
\item \( a \in [3, 4.5] \text{ and } b \in [-598.05, -597.7] \)

 $3.40  - 598.00 i$, which corresponds to forgetting to multiply the conjugate by the numerator.
\end{enumerate}

\textbf{General Comment:} Multiply the numerator and denominator by the *conjugate* of the denominator, then simplify. For example, if we have $2+3i$, the conjugate is $2-3i$.
}
\litem{
Simplify the expression below into the form $a+bi$. Then, choose the intervals that $a$ and $b$ belong to.
\[ (-2 + 9 i)(8 - 3 i) \]

The solution is \( 11 + 78 i \), which is option C.\begin{enumerate}[label=\Alph*.]
\item \( a \in [-44, -41] \text{ and } b \in [66, 69] \)

 $-43 + 66 i$, which corresponds to adding a minus sign in the second term.
\item \( a \in [7, 17] \text{ and } b \in [-86, -72] \)

 $11 - 78 i$, which corresponds to adding a minus sign in both terms.
\item \( a \in [7, 17] \text{ and } b \in [76, 82] \)

* $11 + 78 i$, which is the correct option.
\item \( a \in [-44, -41] \text{ and } b \in [-66, -59] \)

 $-43 - 66 i$, which corresponds to adding a minus sign in the first term.
\item \( a \in [-20, -10] \text{ and } b \in [-29, -26] \)

 $-16 - 27 i$, which corresponds to just multiplying the real terms to get the real part of the solution and the coefficients in the complex terms to get the complex part.
\end{enumerate}

\textbf{General Comment:} You can treat $i$ as a variable and distribute. Just remember that $i^2=-1$, so you can continue to reduce after you distribute.
}
\litem{
Simplify the expression below into the form $a+bi$. Then, choose the intervals that $a$ and $b$ belong to.
\[ \frac{-45 - 11 i}{4 - 8 i} \]

The solution is \( -1.15  - 5.05 i \), which is option D.\begin{enumerate}[label=\Alph*.]
\item \( a \in [-11.5, -10.5] \text{ and } b \in [0, 2] \)

 $-11.25  + 1.38 i$, which corresponds to just dividing the first term by the first term and the second by the second.
\item \( a \in [-1.5, -0.5] \text{ and } b \in [-405, -403] \)

 $-1.15  - 404.00 i$, which corresponds to forgetting to multiply the conjugate by the numerator.
\item \( a \in [-92.5, -91] \text{ and } b \in [-6, -4] \)

 $-92.00  - 5.05 i$, which corresponds to forgetting to multiply the conjugate by the numerator and using a plus instead of a minus in the denominator.
\item \( a \in [-1.5, -0.5] \text{ and } b \in [-6, -4] \)

* $-1.15  - 5.05 i$, which is the correct option.
\item \( a \in [-3.5, -2.5] \text{ and } b \in [3.5, 5.5] \)

 $-3.35  + 3.95 i$, which corresponds to forgetting to multiply the conjugate by the numerator and not computing the conjugate correctly.
\end{enumerate}

\textbf{General Comment:} Multiply the numerator and denominator by the *conjugate* of the denominator, then simplify. For example, if we have $2+3i$, the conjugate is $2-3i$.
}
\litem{
Simplify the expression below and choose the interval the simplification is contained within.
\[ 20 - 9^2 + 7 \div 1 * 10 \div 14 \]

The solution is \( -56.000 \), which is option C.\begin{enumerate}[label=\Alph*.]
\item \( [-64.95, -58.95] \)

 -60.950, which corresponds to an Order of Operations error: not reading left-to-right for multiplication/division.
\item \( [105, 109] \)

 106.000, which corresponds to an Order of Operations error: multiplying by negative before squaring. For example: $(-3)^2 \neq -3^2$
\item \( [-60, -50] \)

* -56.000, this is the correct option
\item \( [100.05, 102.05] \)

 101.050, which corresponds to two Order of Operations errors.
\item \( \text{None of the above} \)

 You may have gotten this by making an unanticipated error. If you got a value that is not any of the others, please let the coordinator know so they can help you figure out what happened.
\end{enumerate}

\textbf{General Comment:} While you may remember (or were taught) PEMDAS is done in order, it is actually done as P/E/MD/AS. When we are at MD or AS, we read left to right.
}
\litem{
Simplify the expression below into the form $a+bi$. Then, choose the intervals that $a$ and $b$ belong to.
\[ (8 - 6 i)(-2 + 3 i) \]

The solution is \( 2 + 36 i \), which is option B.\begin{enumerate}[label=\Alph*.]
\item \( a \in [-37, -29] \text{ and } b \in [10, 14] \)

 $-34 + 12 i$, which corresponds to adding a minus sign in the first term.
\item \( a \in [2, 6] \text{ and } b \in [32, 38] \)

* $2 + 36 i$, which is the correct option.
\item \( a \in [-20, -14] \text{ and } b \in [-23, -16] \)

 $-16 - 18 i$, which corresponds to just multiplying the real terms to get the real part of the solution and the coefficients in the complex terms to get the complex part.
\item \( a \in [2, 6] \text{ and } b \in [-36, -33] \)

 $2 - 36 i$, which corresponds to adding a minus sign in both terms.
\item \( a \in [-37, -29] \text{ and } b \in [-17, -4] \)

 $-34 - 12 i$, which corresponds to adding a minus sign in the second term.
\end{enumerate}

\textbf{General Comment:} You can treat $i$ as a variable and distribute. Just remember that $i^2=-1$, so you can continue to reduce after you distribute.
}
\end{enumerate}

\end{document}