\documentclass{extbook}[14pt]
\usepackage{multicol, enumerate, enumitem, hyperref, color, soul, setspace, parskip, fancyhdr, amssymb, amsthm, amsmath, latexsym, units, mathtools}
\everymath{\displaystyle}
\usepackage[headsep=0.5cm,headheight=0cm, left=1 in,right= 1 in,top= 1 in,bottom= 1 in]{geometry}
\usepackage{dashrule}  % Package to use the command below to create lines between items
\newcommand{\litem}[1]{\item #1

\rule{\textwidth}{0.4pt}}
\pagestyle{fancy}
\lhead{}
\chead{Answer Key for Makeup Progress Quiz 3 Version C}
\rhead{}
\lfoot{1648-1753}
\cfoot{}
\rfoot{Summer C 2021}
\begin{document}
\textbf{This key should allow you to understand why you choose the option you did (beyond just getting a question right or wrong). \href{https://xronos.clas.ufl.edu/mac1105spring2020/courseDescriptionAndMisc/Exams/LearningFromResults}{More instructions on how to use this key can be found here}.}

\textbf{If you have a suggestion to make the keys better, \href{https://forms.gle/CZkbZmPbC9XALEE88}{please fill out the short survey here}.}

\textit{Note: This key is auto-generated and may contain issues and/or errors. The keys are reviewed after each exam to ensure grading is done accurately. If there are issues (like duplicate options), they are noted in the offline gradebook. The keys are a work-in-progress to give students as many resources to improve as possible.}

\rule{\textwidth}{0.4pt}

\begin{enumerate}\litem{
Factor the polynomial below completely. Then, choose the intervals the zeros of the polynomial belong to, where $z_1 \leq z_2 \leq z_3$. \textit{To make the problem easier, all zeros are between -5 and 5.}
\[ f(x) = 15x^{3} +56 x^{2} +60 x + 16 \]The solution is \( [-2, -1.33, -0.4] \), which is option A.\begin{enumerate}[label=\Alph*.]
\item \( z_1 \in [-2.37, -1.82], \text{   }  z_2 \in [-1.6, -0.7], \text{   and   } z_3 \in [-0.66, 0.01] \)

* This is the solution!
\item \( z_1 \in [-3.12, -2.21], \text{   }  z_2 \in [-2.2, -1.8], \text{   and   } z_3 \in [-1.16, -0.73] \)

 Distractor 2: Corresponds to inversing rational roots.
\item \( z_1 \in [0.04, 0.25], \text{   }  z_2 \in [1.5, 2.3], \text{   and   } z_3 \in [3.59, 4.28] \)

 Distractor 4: Corresponds to moving factors from one rational to another.
\item \( z_1 \in [0.28, 0.66], \text{   }  z_2 \in [0, 1.8], \text{   and   } z_3 \in [1.8, 2.27] \)

 Distractor 1: Corresponds to negatives of all zeros.
\item \( z_1 \in [0.58, 0.8], \text{   }  z_2 \in [1.5, 2.3], \text{   and   } z_3 \in [2.39, 2.55] \)

 Distractor 3: Corresponds to negatives of all zeros AND inversing rational roots.
\end{enumerate}

\textbf{General Comment:} Remember to try the middle-most integers first as these normally are the zeros. Also, once you get it to a quadratic, you can use your other factoring techniques to finish factoring.
}
\litem{
Perform the division below. Then, find the intervals that correspond to the quotient in the form $ax^2+bx+c$ and remainder $r$.
\[ \frac{4x^{3} -28 x + 29}{x + 3} \]The solution is \( 4x^{2} -12 x + 8 + \frac{5}{x + 3} \), which is option E.\begin{enumerate}[label=\Alph*.]
\item \( a \in [-13, -6], b \in [33.5, 37.2], c \in [-136, -135], \text{ and } r \in [434, 440]. \)

 You multipled by the synthetic number rather than bringing the first factor down.
\item \( a \in [-3, 7], b \in [-16.3, -15.4], c \in [34, 40], \text{ and } r \in [-115, -110]. \)

 You multipled by the synthetic number and subtracted rather than adding during synthetic division.
\item \( a \in [-3, 7], b \in [11.9, 14.4], c \in [2, 15], \text{ and } r \in [50, 60]. \)

 You divided by the opposite of the factor.
\item \( a \in [-13, -6], b \in [-38.6, -35.7], c \in [-136, -135], \text{ and } r \in [-383, -375]. \)

 You divided by the opposite of the factor AND multipled the first factor rather than just bringing it down.
\item \( a \in [-3, 7], b \in [-12.1, -10.9], c \in [2, 15], \text{ and } r \in [3, 8]. \)

* This is the solution!
\end{enumerate}

\textbf{General Comment:} Be sure to synthetically divide by the zero of the denominator! Also, make sure to include 0 placeholders for missing terms.
}
\litem{
Perform the division below. Then, find the intervals that correspond to the quotient in the form $ax^2+bx+c$ and remainder $r$.
\[ \frac{20x^{3} -75 x^{2} +85 x -28}{x -2} \]The solution is \( 20x^{2} -35 x + 15 + \frac{2}{x -2} \), which is option E.\begin{enumerate}[label=\Alph*.]
\item \( a \in [19, 21], \text{   } b \in [-115, -113], \text{   } c \in [309, 317], \text{   and   } r \in [-661, -657]. \)

 You divided by the opposite of the factor.
\item \( a \in [19, 21], \text{   } b \in [-55, -50], \text{   } c \in [27, 38], \text{   and   } r \in [0, 5]. \)

 You multiplied by the synthetic number and subtracted rather than adding during synthetic division.
\item \( a \in [38, 44], \text{   } b \in [-161, -150], \text{   } c \in [395, 404], \text{   and   } r \in [-819, -814]. \)

 You divided by the opposite of the factor AND multiplied the first factor rather than just bringing it down.
\item \( a \in [38, 44], \text{   } b \in [5, 8], \text{   } c \in [93, 96], \text{   and   } r \in [161, 167]. \)

 You multiplied by the synthetic number rather than bringing the first factor down.
\item \( a \in [19, 21], \text{   } b \in [-43, -32], \text{   } c \in [14, 16], \text{   and   } r \in [0, 5]. \)

* This is the solution!
\end{enumerate}

\textbf{General Comment:} Be sure to synthetically divide by the zero of the denominator!
}
\litem{
Factor the polynomial below completely, knowing that $x -4$ is a factor. Then, choose the intervals the zeros of the polynomial belong to, where $z_1 \leq z_2 \leq z_3 \leq z_4$. \textit{To make the problem easier, all zeros are between -5 and 5.}
\[ f(x) = 9x^{4} -54 x^{3} +47 x^{2} +150 x -200 \]The solution is \( [-1.667, 1.667, 2, 4] \), which is option D.\begin{enumerate}[label=\Alph*.]
\item \( z_1 \in [-1.2, 0.1], \text{   }  z_2 \in [-0.13, 1.43], z_3 \in [1.96, 2.19], \text{   and   } z_4 \in [3.9, 4.01] \)

 Distractor 2: Corresponds to inversing rational roots.
\item \( z_1 \in [-4.2, -2.8], \text{   }  z_2 \in [-2.03, -1.86], z_3 \in [-0.6, -0.42], \text{   and   } z_4 \in [0.57, 0.63] \)

 Distractor 3: Corresponds to negatives of all zeros AND inversing rational roots.
\item \( z_1 \in [-5.8, -4.5], \text{   }  z_2 \in [-4.7, -3.58], z_3 \in [-2.01, -1.7], \text{   and   } z_4 \in [0.47, 0.59] \)

 Distractor 4: Corresponds to moving factors from one rational to another.
\item \( z_1 \in [-2.4, -1.3], \text{   }  z_2 \in [0.99, 1.68], z_3 \in [1.96, 2.19], \text{   and   } z_4 \in [3.9, 4.01] \)

* This is the solution!
\item \( z_1 \in [-4.2, -2.8], \text{   }  z_2 \in [-2.03, -1.86], z_3 \in [-1.89, -1.42], \text{   and   } z_4 \in [1.56, 1.72] \)

 Distractor 1: Corresponds to negatives of all zeros.
\end{enumerate}

\textbf{General Comment:} Remember to try the middle-most integers first as these normally are the zeros. Also, once you get it to a quadratic, you can use your other factoring techniques to finish factoring.
}
\litem{
Factor the polynomial below completely, knowing that $x -2$ is a factor. Then, choose the intervals the zeros of the polynomial belong to, where $z_1 \leq z_2 \leq z_3 \leq z_4$. \textit{To make the problem easier, all zeros are between -5 and 5.}
\[ f(x) = 6x^{4} -59 x^{3} +206 x^{2} -304 x + 160 \]The solution is \( [1.333, 2, 2.5, 4] \), which is option C.\begin{enumerate}[label=\Alph*.]
\item \( z_1 \in [-4.28, -3.72], \text{   }  z_2 \in [-2.11, -1.43], z_3 \in [-0.78, -0.37], \text{   and   } z_4 \in [-0.51, 0.3] \)

 Distractor 3: Corresponds to negatives of all zeros AND inversing rational roots.
\item \( z_1 \in [-4.28, -3.72], \text{   }  z_2 \in [-3.23, -2.11], z_3 \in [-2.42, -1.97], \text{   and   } z_4 \in [-2.07, -1.17] \)

 Distractor 1: Corresponds to negatives of all zeros.
\item \( z_1 \in [1.03, 1.82], \text{   }  z_2 \in [0.92, 2.24], z_3 \in [2.38, 2.55], \text{   and   } z_4 \in [3.29, 4.43] \)

* This is the solution!
\item \( z_1 \in [-0.57, 0.96], \text{   }  z_2 \in [-0.13, 1.04], z_3 \in [1.89, 2.08], \text{   and   } z_4 \in [3.29, 4.43] \)

 Distractor 2: Corresponds to inversing rational roots.
\item \( z_1 \in [-4.28, -3.72], \text{   }  z_2 \in [-4.06, -3.6], z_3 \in [-2.42, -1.97], \text{   and   } z_4 \in [-0.89, -0.61] \)

 Distractor 4: Corresponds to moving factors from one rational to another.
\end{enumerate}

\textbf{General Comment:} Remember to try the middle-most integers first as these normally are the zeros. Also, once you get it to a quadratic, you can use your other factoring techniques to finish factoring.
}
\litem{
Perform the division below. Then, find the intervals that correspond to the quotient in the form $ax^2+bx+c$ and remainder $r$.
\[ \frac{6x^{3} -2 x^{2} -44 x + 45}{x + 3} \]The solution is \( 6x^{2} -20 x + 16 + \frac{-3}{x + 3} \), which is option A.\begin{enumerate}[label=\Alph*.]
\item \( a \in [5, 9], \text{   } b \in [-23.4, -18.2], \text{   } c \in [15, 18], \text{   and   } r \in [-5, 1]. \)

* This is the solution!
\item \( a \in [-19, -17], \text{   } b \in [49.5, 54.2], \text{   } c \in [-209, -197], \text{   and   } r \in [644, 649]. \)

 You multiplied by the synthetic number rather than bringing the first factor down.
\item \( a \in [-19, -17], \text{   } b \in [-58.6, -53.9], \text{   } c \in [-217, -210], \text{   and   } r \in [-594, -589]. \)

 You divided by the opposite of the factor AND multiplied the first factor rather than just bringing it down.
\item \( a \in [5, 9], \text{   } b \in [15.1, 16.8], \text{   } c \in [-1, 11], \text{   and   } r \in [53, 60]. \)

 You divided by the opposite of the factor.
\item \( a \in [5, 9], \text{   } b \in [-27, -25.5], \text{   } c \in [59, 63], \text{   and   } r \in [-195, -187]. \)

 You multiplied by the synthetic number and subtracted rather than adding during synthetic division.
\end{enumerate}

\textbf{General Comment:} Be sure to synthetically divide by the zero of the denominator!
}
\litem{
Factor the polynomial below completely. Then, choose the intervals the zeros of the polynomial belong to, where $z_1 \leq z_2 \leq z_3$. \textit{To make the problem easier, all zeros are between -5 and 5.}
\[ f(x) = 15x^{3} -89 x^{2} +62 x + 40 \]The solution is \( [-0.4, 1.33, 5] \), which is option E.\begin{enumerate}[label=\Alph*.]
\item \( z_1 \in [-3.5, -1.5], \text{   }  z_2 \in [0.64, 0.8], \text{   and   } z_3 \in [4.8, 5.4] \)

 Distractor 2: Corresponds to inversing rational roots.
\item \( z_1 \in [-5, -4], \text{   }  z_2 \in [-0.87, -0.74], \text{   and   } z_3 \in [2.1, 2.9] \)

 Distractor 3: Corresponds to negatives of all zeros AND inversing rational roots.
\item \( z_1 \in [-5, -4], \text{   }  z_2 \in [-2.05, -1.27], \text{   and   } z_3 \in [0.3, 0.5] \)

 Distractor 1: Corresponds to negatives of all zeros.
\item \( z_1 \in [-5, -4], \text{   }  z_2 \in [-0.29, -0.11], \text{   and   } z_3 \in [1, 2.3] \)

 Distractor 4: Corresponds to moving factors from one rational to another.
\item \( z_1 \in [-1.4, 1.6], \text{   }  z_2 \in [0.97, 1.36], \text{   and   } z_3 \in [4.8, 5.4] \)

* This is the solution!
\end{enumerate}

\textbf{General Comment:} Remember to try the middle-most integers first as these normally are the zeros. Also, once you get it to a quadratic, you can use your other factoring techniques to finish factoring.
}
\litem{
What are the \textit{possible Rational} roots of the polynomial below?
\[ f(x) = 5x^{4} +3 x^{3} +6 x^{2} +2 x + 6 \]The solution is \( \text{ All combinations of: }\frac{\pm 1,\pm 2,\pm 3,\pm 6}{\pm 1,\pm 5} \), which is option D.\begin{enumerate}[label=\Alph*.]
\item \( \pm 1,\pm 5 \)

 Distractor 1: Corresponds to the plus or minus factors of a1 only.
\item \( \text{ All combinations of: }\frac{\pm 1,\pm 5}{\pm 1,\pm 2,\pm 3,\pm 6} \)

 Distractor 3: Corresponds to the plus or minus of the inverse quotient (an/a0) of the factors. 
\item \( \pm 1,\pm 2,\pm 3,\pm 6 \)

This would have been the solution \textbf{if asked for the possible Integer roots}!
\item \( \text{ All combinations of: }\frac{\pm 1,\pm 2,\pm 3,\pm 6}{\pm 1,\pm 5} \)

* This is the solution \textbf{since we asked for the possible Rational roots}!
\item \( \text{ There is no formula or theorem that tells us all possible Rational roots.} \)

 Distractor 4: Corresponds to not recalling the theorem for rational roots of a polynomial.
\end{enumerate}

\textbf{General Comment:} We have a way to find the possible Rational roots. The possible Integer roots are the Integers in this list.
}
\litem{
Perform the division below. Then, find the intervals that correspond to the quotient in the form $ax^2+bx+c$ and remainder $r$.
\[ \frac{10x^{3} -30 x + 18}{x + 2} \]The solution is \( 10x^{2} -20 x + 10 + \frac{-2}{x + 2} \), which is option C.\begin{enumerate}[label=\Alph*.]
\item \( a \in [-20, -19], b \in [39, 42], c \in [-116, -107], \text{ and } r \in [237, 239]. \)

 You multipled by the synthetic number rather than bringing the first factor down.
\item \( a \in [6, 14], b \in [-34, -29], c \in [58, 61], \text{ and } r \in [-164, -161]. \)

 You multipled by the synthetic number and subtracted rather than adding during synthetic division.
\item \( a \in [6, 14], b \in [-23, -13], c \in [8, 15], \text{ and } r \in [-7, 1]. \)

* This is the solution!
\item \( a \in [-20, -19], b \in [-42, -39], c \in [-116, -107], \text{ and } r \in [-202, -197]. \)

 You divided by the opposite of the factor AND multipled the first factor rather than just bringing it down.
\item \( a \in [6, 14], b \in [14, 23], c \in [8, 15], \text{ and } r \in [36, 42]. \)

 You divided by the opposite of the factor.
\end{enumerate}

\textbf{General Comment:} Be sure to synthetically divide by the zero of the denominator! Also, make sure to include 0 placeholders for missing terms.
}
\litem{
What are the \textit{possible Integer} roots of the polynomial below?
\[ f(x) = 5x^{3} +4 x^{2} +3 x + 4 \]The solution is \( \pm 1,\pm 2,\pm 4 \), which is option D.\begin{enumerate}[label=\Alph*.]
\item \( \text{ All combinations of: }\frac{\pm 1,\pm 2,\pm 4}{\pm 1,\pm 5} \)

This would have been the solution \textbf{if asked for the possible Rational roots}!
\item \( \text{ All combinations of: }\frac{\pm 1,\pm 5}{\pm 1,\pm 2,\pm 4} \)

 Distractor 3: Corresponds to the plus or minus of the inverse quotient (an/a0) of the factors. 
\item \( \pm 1,\pm 5 \)

 Distractor 1: Corresponds to the plus or minus factors of a1 only.
\item \( \pm 1,\pm 2,\pm 4 \)

* This is the solution \textbf{since we asked for the possible Integer roots}!
\item \( \text{There is no formula or theorem that tells us all possible Integer roots.} \)

 Distractor 4: Corresponds to not recognizing Integers as a subset of Rationals.
\end{enumerate}

\textbf{General Comment:} We have a way to find the possible Rational roots. The possible Integer roots are the Integers in this list.
}
\end{enumerate}

\end{document}