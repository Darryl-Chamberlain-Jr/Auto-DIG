\documentclass{extbook}[14pt]
\usepackage{multicol, enumerate, enumitem, hyperref, color, soul, setspace, parskip, fancyhdr, amssymb, amsthm, amsmath, bbm, latexsym, units, mathtools}
\everymath{\displaystyle}
\usepackage[headsep=0.5cm,headheight=0cm, left=1 in,right= 1 in,top= 1 in,bottom= 1 in]{geometry}
\usepackage{dashrule}  % Package to use the command below to create lines between items
\newcommand{\litem}[1]{\item #1

\rule{\textwidth}{0.4pt}}
\pagestyle{fancy}
\lhead{}
\chead{Answer Key for Makeup Progress Quiz 3 Version C}
\rhead{}
\lfoot{4315-3397}
\cfoot{}
\rfoot{Fall 2020}
\begin{document}
\textbf{This key should allow you to understand why you choose the option you did (beyond just getting a question right or wrong). \href{https://xronos.clas.ufl.edu/mac1105spring2020/courseDescriptionAndMisc/Exams/LearningFromResults}{More instructions on how to use this key can be found here}.}

\textbf{If you have a suggestion to make the keys better, \href{https://forms.gle/CZkbZmPbC9XALEE88}{please fill out the short survey here}.}

\textit{Note: This key is auto-generated and may contain issues and/or errors. The keys are reviewed after each exam to ensure grading is done accurately. If there are issues (like duplicate options), they are noted in the offline gradebook. The keys are a work-in-progress to give students as many resources to improve as possible.}

\rule{\textwidth}{0.4pt}

\begin{enumerate}\litem{
Perform the division below. Then, find the intervals that correspond to the quotient in the form $ax^2+bx+c$ and remainder $r$.
\[ \frac{12x^{3} -63 x^{2} + 77}{x -5} \]

The solution is \( 12x^{2} -3 x -15 + \frac{2}{x -5} \), which is option B.\begin{enumerate}[label=\Alph*.]
\item \( a \in [6, 15], b \in [-130, -120], c \in [604, 618], \text{ and } r \in [-2998, -2988]. \)

 You divided by the opposite of the factor.
\item \( a \in [6, 15], b \in [-6, -1], c \in [-15, -14], \text{ and } r \in [2, 4]. \)

* This is the solution!
\item \( a \in [6, 15], b \in [-16, -14], c \in [-63, -54], \text{ and } r \in [-163, -160]. \)

 You multipled by the synthetic number and subtracted rather than adding during synthetic division.
\item \( a \in [57, 62], b \in [228, 238], c \in [1185, 1187], \text{ and } r \in [5999, 6003]. \)

 You multipled by the synthetic number rather than bringing the first factor down.
\item \( a \in [57, 62], b \in [-365, -360], c \in [1814, 1822], \text{ and } r \in [-9003, -8995]. \)

 You divided by the opposite of the factor AND multipled the first factor rather than just bringing it down.
\end{enumerate}

\textbf{General Comment:} Be sure to synthetically divide by the zero of the denominator! Also, make sure to include 0 placeholders for missing terms.
}
\litem{
What are the \textit{possible Rational} roots of the polynomial below?
\[ f(x) = 4x^{4} +4 x^{3} +2 x^{2} +3 x + 2 \]

The solution is \( \text{ All combinations of: }\frac{\pm 1,\pm 2}{\pm 1,\pm 2,\pm 4} \), which is option D.\begin{enumerate}[label=\Alph*.]
\item \( \text{ All combinations of: }\frac{\pm 1,\pm 2,\pm 4}{\pm 1,\pm 2} \)

 Distractor 3: Corresponds to the plus or minus of the inverse quotient (an/a0) of the factors. 
\item \( \pm 1,\pm 2 \)

This would have been the solution \textbf{if asked for the possible Integer roots}!
\item \( \pm 1,\pm 2,\pm 4 \)

 Distractor 1: Corresponds to the plus or minus factors of a1 only.
\item \( \text{ All combinations of: }\frac{\pm 1,\pm 2}{\pm 1,\pm 2,\pm 4} \)

* This is the solution \textbf{since we asked for the possible Rational roots}!
\item \( \text{ There is no formula or theorem that tells us all possible Rational roots.} \)

 Distractor 4: Corresponds to not recalling the theorem for rational roots of a polynomial.
\end{enumerate}

\textbf{General Comment:} We have a way to find the possible Rational roots. The possible Integer roots are the Integers in this list.
}
\litem{
Perform the division below. Then, find the intervals that correspond to the quotient in the form $ax^2+bx+c$ and remainder $r$.
\[ \frac{25x^{3} -105 x^{2} + 83}{x -4} \]

The solution is \( 25x^{2} -5 x -20 + \frac{3}{x -4} \), which is option D.\begin{enumerate}[label=\Alph*.]
\item \( a \in [100, 103], b \in [295, 302], c \in [1174, 1190], \text{ and } r \in [4800, 4809]. \)

 You multipled by the synthetic number rather than bringing the first factor down.
\item \( a \in [100, 103], b \in [-505, -499], c \in [2019, 2022], \text{ and } r \in [-7997, -7992]. \)

 You divided by the opposite of the factor AND multipled the first factor rather than just bringing it down.
\item \( a \in [23, 33], b \in [-31, -28], c \in [-91, -89], \text{ and } r \in [-190, -184]. \)

 You multipled by the synthetic number and subtracted rather than adding during synthetic division.
\item \( a \in [23, 33], b \in [-8, -1], c \in [-20, -18], \text{ and } r \in [-5, 7]. \)

* This is the solution!
\item \( a \in [23, 33], b \in [-208, -202], c \in [817, 823], \text{ and } r \in [-3202, -3194]. \)

 You divided by the opposite of the factor.
\end{enumerate}

\textbf{General Comment:} Be sure to synthetically divide by the zero of the denominator! Also, make sure to include 0 placeholders for missing terms.
}
\litem{
Perform the division below. Then, find the intervals that correspond to the quotient in the form $ax^2+bx+c$ and remainder $r$.
\[ \frac{25x^{3} -15 x^{2} -58 x -26}{x -2} \]

The solution is \( 25x^{2} +35 x + 12 + \frac{-2}{x -2} \), which is option C.\begin{enumerate}[label=\Alph*.]
\item \( a \in [50, 56], \text{   } b \in [-119, -112], \text{   } c \in [169, 173], \text{   and   } r \in [-373, -364]. \)

 You divided by the opposite of the factor AND multiplied the first factor rather than just bringing it down.
\item \( a \in [50, 56], \text{   } b \in [79, 90], \text{   } c \in [112, 113], \text{   and   } r \in [198, 205]. \)

 You multiplied by the synthetic number rather than bringing the first factor down.
\item \( a \in [24, 26], \text{   } b \in [34, 42], \text{   } c \in [11, 14], \text{   and   } r \in [-6, 2]. \)

* This is the solution!
\item \( a \in [24, 26], \text{   } b \in [10, 11], \text{   } c \in [-49, -45], \text{   and   } r \in [-76, -70]. \)

 You multiplied by the synthetic number and subtracted rather than adding during synthetic division.
\item \( a \in [24, 26], \text{   } b \in [-70, -61], \text{   } c \in [70, 73], \text{   and   } r \in [-172, -166]. \)

 You divided by the opposite of the factor.
\end{enumerate}

\textbf{General Comment:} Be sure to synthetically divide by the zero of the denominator!
}
\litem{
What are the \textit{possible Rational} roots of the polynomial below?
\[ f(x) = 4x^{2} +5 x + 7 \]

The solution is \( \text{ All combinations of: }\frac{\pm 1,\pm 7}{\pm 1,\pm 2,\pm 4} \), which is option C.\begin{enumerate}[label=\Alph*.]
\item \( \pm 1,\pm 2,\pm 4 \)

 Distractor 1: Corresponds to the plus or minus factors of a1 only.
\item \( \text{ All combinations of: }\frac{\pm 1,\pm 2,\pm 4}{\pm 1,\pm 7} \)

 Distractor 3: Corresponds to the plus or minus of the inverse quotient (an/a0) of the factors. 
\item \( \text{ All combinations of: }\frac{\pm 1,\pm 7}{\pm 1,\pm 2,\pm 4} \)

* This is the solution \textbf{since we asked for the possible Rational roots}!
\item \( \pm 1,\pm 7 \)

This would have been the solution \textbf{if asked for the possible Integer roots}!
\item \( \text{ There is no formula or theorem that tells us all possible Rational roots.} \)

 Distractor 4: Corresponds to not recalling the theorem for rational roots of a polynomial.
\end{enumerate}

\textbf{General Comment:} We have a way to find the possible Rational roots. The possible Integer roots are the Integers in this list.
}
\litem{
Perform the division below. Then, find the intervals that correspond to the quotient in the form $ax^2+bx+c$ and remainder $r$.
\[ \frac{10x^{3} +61 x^{2} +49 x -28}{x + 5} \]

The solution is \( 10x^{2} +11 x -6 + \frac{2}{x + 5} \), which is option C.\begin{enumerate}[label=\Alph*.]
\item \( a \in [-52, -41], \text{   } b \in [-192, -185], \text{   } c \in [-896, -891], \text{   and   } r \in [-4511, -4507]. \)

 You divided by the opposite of the factor AND multiplied the first factor rather than just bringing it down.
\item \( a \in [5, 13], \text{   } b \in [104, 113], \text{   } c \in [601, 607], \text{   and   } r \in [2991, 2994]. \)

 You divided by the opposite of the factor.
\item \( a \in [5, 13], \text{   } b \in [11, 14], \text{   } c \in [-8, -2], \text{   and   } r \in [2, 7]. \)

* This is the solution!
\item \( a \in [5, 13], \text{   } b \in [-5, 2], \text{   } c \in [40, 45], \text{   and   } r \in [-289, -283]. \)

 You multiplied by the synthetic number and subtracted rather than adding during synthetic division.
\item \( a \in [-52, -41], \text{   } b \in [307, 316], \text{   } c \in [-1508, -1500], \text{   and   } r \in [7501, 7504]. \)

 You multiplied by the synthetic number rather than bringing the first factor down.
\end{enumerate}

\textbf{General Comment:} Be sure to synthetically divide by the zero of the denominator!
}
\litem{
Factor the polynomial below completely, knowing that $x-4$ is a factor. Then, choose the intervals the zeros of the polynomial belong to, where $z_1 \leq z_2 \leq z_3 \leq z_4$. \textit{To make the problem easier, all zeros are between -5 and 5.}
\[ f(x) = 12x^{4} -115 x^{3} +381 x^{2} -512 x + 240 \]

The solution is \( [1.25, 1.3333333333333333, 3, 4] \), which is option B.\begin{enumerate}[label=\Alph*.]
\item \( z_1 \in [-5.47, -4.86], \text{   }  z_2 \in [-5.17, -3.55], z_3 \in [-3.03, -2.94], \text{   and   } z_4 \in [-0.39, 0.33] \)

 Distractor 4: Corresponds to moving factors from one rational to another.
\item \( z_1 \in [1.22, 1.61], \text{   }  z_2 \in [1.31, 1.36], z_3 \in [2.75, 3.19], \text{   and   } z_4 \in [3.54, 4.41] \)

* This is the solution!
\item \( z_1 \in [-0.02, 0.89], \text{   }  z_2 \in [0.45, 0.97], z_3 \in [2.75, 3.19], \text{   and   } z_4 \in [3.54, 4.41] \)

 Distractor 2: Corresponds to inversing rational roots.
\item \( z_1 \in [-4.33, -3.64], \text{   }  z_2 \in [-3.31, -2.03], z_3 \in [-2.18, -0.97], \text{   and   } z_4 \in [-1.67, -0.77] \)

 Distractor 1: Corresponds to negatives of all zeros.
\item \( z_1 \in [-4.33, -3.64], \text{   }  z_2 \in [-3.31, -2.03], z_3 \in [-1.05, -0.03], \text{   and   } z_4 \in [-1.08, -0.72] \)

 Distractor 3: Corresponds to negatives of all zeros AND inversing rational roots.
\end{enumerate}

\textbf{General Comment:} Remember to try the middle-most integers first as these normally are the zeros. Also, once you get it to a quadratic, you can use your other factoring techniques to finish factoring.
}
\litem{
Factor the polynomial below completely. Then, choose the intervals the zeros of the polynomial belong to, where $z_1 \leq z_2 \leq z_3$. \textit{To make the problem easier, all zeros are between -5 and 5.}
\[ f(x) = 6x^{3} +55 x^{2} +150 x + 125 \]

The solution is \( [-5, -2.5, -1.6666666666666667] \), which is option C.\begin{enumerate}[label=\Alph*.]
\item \( z_1 \in [-5.83, -4.66], \text{   }  z_2 \in [-0.6, 0.4], \text{   and   } z_3 \in [-1.4, 0.6] \)

 Distractor 2: Corresponds to inversing rational roots.
\item \( z_1 \in [1.35, 1.96], \text{   }  z_2 \in [1.5, 3.5], \text{   and   } z_3 \in [4, 7] \)

 Distractor 1: Corresponds to negatives of all zeros.
\item \( z_1 \in [-5.83, -4.66], \text{   }  z_2 \in [-2.5, -1.5], \text{   and   } z_3 \in [-1.67, -0.67] \)

* This is the solution!
\item \( z_1 \in [0.49, 1.39], \text{   }  z_2 \in [5, 6], \text{   and   } z_3 \in [4, 7] \)

 Distractor 4: Corresponds to moving factors from one rational to another.
\item \( z_1 \in [-0.16, 0.82], \text{   }  z_2 \in [-0.4, 1.6], \text{   and   } z_3 \in [4, 7] \)

 Distractor 3: Corresponds to negatives of all zeros AND inversing rational roots.
\end{enumerate}

\textbf{General Comment:} Remember to try the middle-most integers first as these normally are the zeros. Also, once you get it to a quadratic, you can use your other factoring techniques to finish factoring.
}
\litem{
Factor the polynomial below completely, knowing that $x-2$ is a factor. Then, choose the intervals the zeros of the polynomial belong to, where $z_1 \leq z_2 \leq z_3 \leq z_4$. \textit{To make the problem easier, all zeros are between -5 and 5.}
\[ f(x) = 20x^{4} -153 x^{3} +276 x^{2} -25 x -150 \]

The solution is \( [-0.6, 1.25, 2, 5] \), which is option B.\begin{enumerate}[label=\Alph*.]
\item \( z_1 \in [-5.3, -2.5], \text{   }  z_2 \in [-2.38, -1.97], z_3 \in [-0.83, -0.49], \text{   and   } z_4 \in [1.29, 2.23] \)

 Distractor 3: Corresponds to negatives of all zeros AND inversing rational roots.
\item \( z_1 \in [-0.9, 0.3], \text{   }  z_2 \in [1.18, 1.47], z_3 \in [1.91, 2.44], \text{   and   } z_4 \in [4.84, 5.29] \)

* This is the solution!
\item \( z_1 \in [-5.3, -2.5], \text{   }  z_2 \in [-5.23, -4.96], z_3 \in [-2.01, -1.59], \text{   and   } z_4 \in [-0.14, 0.28] \)

 Distractor 4: Corresponds to moving factors from one rational to another.
\item \( z_1 \in [-5.3, -2.5], \text{   }  z_2 \in [-2.38, -1.97], z_3 \in [-1.34, -1.18], \text{   and   } z_4 \in [0.41, 0.79] \)

 Distractor 1: Corresponds to negatives of all zeros.
\item \( z_1 \in [-1.8, -1.2], \text{   }  z_2 \in [0.73, 0.97], z_3 \in [1.91, 2.44], \text{   and   } z_4 \in [4.84, 5.29] \)

 Distractor 2: Corresponds to inversing rational roots.
\end{enumerate}

\textbf{General Comment:} Remember to try the middle-most integers first as these normally are the zeros. Also, once you get it to a quadratic, you can use your other factoring techniques to finish factoring.
}
\litem{
Factor the polynomial below completely. Then, choose the intervals the zeros of the polynomial belong to, where $z_1 \leq z_2 \leq z_3$. \textit{To make the problem easier, all zeros are between -5 and 5.}
\[ f(x) = 15x^{3} +53 x^{2} +8 x -48 \]

The solution is \( [-3, -1.3333333333333333, 0.8] \), which is option D.\begin{enumerate}[label=\Alph*.]
\item \( z_1 \in [-0.95, -0.69], \text{   }  z_2 \in [1, 2.3], \text{   and   } z_3 \in [2.77, 3.19] \)

 Distractor 1: Corresponds to negatives of all zeros.
\item \( z_1 \in [-3.68, -2.98], \text{   }  z_2 \in [-1.3, -0.3], \text{   and   } z_3 \in [0.85, 2.37] \)

 Distractor 2: Corresponds to inversing rational roots.
\item \( z_1 \in [-0.68, -0.21], \text{   }  z_2 \in [2.9, 3.6], \text{   and   } z_3 \in [3.97, 4.38] \)

 Distractor 4: Corresponds to moving factors from one rational to another.
\item \( z_1 \in [-3.68, -2.98], \text{   }  z_2 \in [-2, -1], \text{   and   } z_3 \in [0.15, 0.88] \)

* This is the solution!
\item \( z_1 \in [-1.52, -1.06], \text{   }  z_2 \in [-0.2, 1.1], \text{   and   } z_3 \in [2.77, 3.19] \)

 Distractor 3: Corresponds to negatives of all zeros AND inversing rational roots.
\end{enumerate}

\textbf{General Comment:} Remember to try the middle-most integers first as these normally are the zeros. Also, once you get it to a quadratic, you can use your other factoring techniques to finish factoring.
}
\end{enumerate}

\end{document}