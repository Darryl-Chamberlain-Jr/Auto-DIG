\documentclass[14pt]{extbook}
\usepackage{multicol, enumerate, enumitem, hyperref, color, soul, setspace, parskip, fancyhdr} %General Packages
\usepackage{amssymb, amsthm, amsmath, latexsym, units, mathtools} %Math Packages
\everymath{\displaystyle} %All math in Display Style
% Packages with additional options
\usepackage[headsep=0.5cm,headheight=12pt, left=1 in,right= 1 in,top= 1 in,bottom= 1 in]{geometry}
\usepackage[usenames,dvipsnames]{xcolor}
\usepackage{dashrule}  % Package to use the command below to create lines between items
\newcommand{\litem}[1]{\item#1\hspace*{-1cm}\rule{\textwidth}{0.4pt}}
\pagestyle{fancy}
\lhead{Makeup Progress Quiz 3}
\chead{}
\rhead{Version C}
\lfoot{1648-1753}
\cfoot{}
\rfoot{Summer C 2021}
\begin{document}

\begin{enumerate}
\litem{
Determine whether the function below is 1-1.\[ f(x) = (5 x - 16)^3 \]\begin{enumerate}[label=\Alph*.]
\item \( \text{No, because the domain of the function is not $(-\infty, \infty)$.} \)
\item \( \text{No, because there is an $x$-value that goes to 2 different $y$-values.} \)
\item \( \text{No, because the range of the function is not $(-\infty, \infty)$.} \)
\item \( \text{No, because there is a $y$-value that goes to 2 different $x$-values.} \)
\item \( \text{Yes, the function is 1-1.} \)

\end{enumerate} }
\litem{
Determine whether the function below is 1-1.\[ f(x) = 9 x^2 - 30 x + 25 \]\begin{enumerate}[label=\Alph*.]
\item \( \text{Yes, the function is 1-1.} \)
\item \( \text{No, because the range of the function is not $(-\infty, \infty)$.} \)
\item \( \text{No, because there is an $x$-value that goes to 2 different $y$-values.} \)
\item \( \text{No, because the domain of the function is not $(-\infty, \infty)$.} \)
\item \( \text{No, because there is a $y$-value that goes to 2 different $x$-values.} \)

\end{enumerate} }
\litem{
Find the inverse of the function below (if it exists). Then, evaluate the inverse at $x = 12$ and choose the interval that $f^-1(12)$ belongs to.\[ f(x) = 3 x^2 + 2 \]\begin{enumerate}[label=\Alph*.]
\item \( f^{-1}(12) \in [4.74, 5.36] \)
\item \( f^{-1}(12) \in [2.03, 4.21] \)
\item \( f^{-1}(12) \in [6.45, 8.02] \)
\item \( f^{-1}(12) \in [1.74, 1.9] \)
\item \( \text{ The function is not invertible for all Real numbers. } \)

\end{enumerate} }
\litem{
Subtract the following functions, then choose the domain of the resulting function from the list below.\[ f(x) = 6x^{4} +6 x^{2} +7 x + 7 \text{ and } g(x) = \sqrt{-5x-15}  \]\begin{enumerate}[label=\Alph*.]
\item \( \text{ The domain is all Real numbers except } x = a, \text{ where } a \in [-14.4, 0.6] \)
\item \( \text{ The domain is all Real numbers less than or equal to } x = a, \text{ where } a \in [-4, 0] \)
\item \( \text{ The domain is all Real numbers greater than or equal to } x = a, \text{ where } a \in [-5.5, -0.5] \)
\item \( \text{ The domain is all Real numbers except } x = a \text{ and } x = b, \text{ where } a \in [-9.67, -4.67] \text{ and } b \in [-8.83, -4.83] \)
\item \( \text{ The domain is all Real numbers. } \)

\end{enumerate} }
\litem{
Choose the interval below that $f$ composed with $g$ at $x=1$ is in.\[ f(x) = -2x^{3} +2 x^{2} +x \text{ and } g(x) = 4x^{3} -2 x^{2} -2 x \]\begin{enumerate}[label=\Alph*.]
\item \( (f \circ g)(1) \in [-1.42, 0.53] \)
\item \( (f \circ g)(1) \in [-1.42, 0.53] \)
\item \( (f \circ g)(1) \in [4.53, 5.32] \)
\item \( (f \circ g)(1) \in [5.81, 6.62] \)
\item \( \text{It is not possible to compose the two functions.} \)

\end{enumerate} }
\litem{
Find the inverse of the function below. Then, evaluate the inverse at $x = 7$ and choose the interval that $f^-1(7)$ belongs to.\[ f(x) = e^{x-3}-3 \]\begin{enumerate}[label=\Alph*.]
\item \( f^{-1}(7) \in [5.1, 6.59] \)
\item \( f^{-1}(7) \in [-1.53, 0.15] \)
\item \( f^{-1}(7) \in [-1.53, 0.15] \)
\item \( f^{-1}(7) \in [-1.96, -1.38] \)
\item \( f^{-1}(7) \in [-1.96, -1.38] \)

\end{enumerate} }
\litem{
Choose the interval below that $f$ composed with $g$ at $x=1$ is in.\[ f(x) = x^{3} -1 x^{2} -3 x + 1 \text{ and } g(x) = 3x^{3} -3 x^{2} +2 x \]\begin{enumerate}[label=\Alph*.]
\item \( (f \circ g)(1) \in [5, 13] \)
\item \( (f \circ g)(1) \in [-49, -41] \)
\item \( (f \circ g)(1) \in [-4, 2] \)
\item \( (f \circ g)(1) \in [-43, -39] \)
\item \( \text{It is not possible to compose the two functions.} \)

\end{enumerate} }
\litem{
Multiply the following functions, then choose the domain of the resulting function from the list below.\[ f(x) = \sqrt{4x-26}  \text{ and } g(x) = 6x^{3} +9 x^{2} +8 x + 1 \]\begin{enumerate}[label=\Alph*.]
\item \( \text{ The domain is all Real numbers except } x = a, \text{ where } a \in [3.75, 11.75] \)
\item \( \text{ The domain is all Real numbers less than or equal to } x = a, \text{ where } a \in [2.67, 12.67] \)
\item \( \text{ The domain is all Real numbers greater than or equal to } x = a, \text{ where } a \in [4.5, 10.5] \)
\item \( \text{ The domain is all Real numbers except } x = a \text{ and } x = b, \text{ where } a \in [4.2, 8.2] \text{ and } b \in [-7.67, 0.33] \)
\item \( \text{ The domain is all Real numbers. } \)

\end{enumerate} }
\litem{
Find the inverse of the function below (if it exists). Then, evaluate the inverse at $x = 14$ and choose the interval that $f^-1(14)$ belongs to.\[ f(x) = 5 x^2 + 3 \]\begin{enumerate}[label=\Alph*.]
\item \( f^{-1}(14) \in [1.74, 1.9] \)
\item \( f^{-1}(14) \in [3.47, 3.7] \)
\item \( f^{-1}(14) \in [4.46, 4.89] \)
\item \( f^{-1}(14) \in [1.48, 1.69] \)
\item \( \text{ The function is not invertible for all Real numbers. } \)

\end{enumerate} }
\litem{
Find the inverse of the function below. Then, evaluate the inverse at $x = 8$ and choose the interval that $f^-1(8)$ belongs to.\[ f(x) = e^{x+4}-2 \]\begin{enumerate}[label=\Alph*.]
\item \( f^{-1}(8) \in [0.47, 0.65] \)
\item \( f^{-1}(8) \in [-1.94, -1.15] \)
\item \( f^{-1}(8) \in [-1.35, -0.49] \)
\item \( f^{-1}(8) \in [6.13, 6.88] \)
\item \( f^{-1}(8) \in [-0.27, 0.15] \)

\end{enumerate} }
\end{enumerate}

\end{document}