\documentclass{extbook}[14pt]
\usepackage{multicol, enumerate, enumitem, hyperref, color, soul, setspace, parskip, fancyhdr, amssymb, amsthm, amsmath, bbm, latexsym, units, mathtools}
\everymath{\displaystyle}
\usepackage[headsep=0.5cm,headheight=0cm, left=1 in,right= 1 in,top= 1 in,bottom= 1 in]{geometry}
\usepackage{dashrule}  % Package to use the command below to create lines between items
\newcommand{\litem}[1]{\item #1

\rule{\textwidth}{0.4pt}}
\pagestyle{fancy}
\lhead{}
\chead{Answer Key for Makeup Progress Quiz 3 Version A}
\rhead{}
\lfoot{4315-3397}
\cfoot{}
\rfoot{Fall 2020}
\begin{document}
\textbf{This key should allow you to understand why you choose the option you did (beyond just getting a question right or wrong). \href{https://xronos.clas.ufl.edu/mac1105spring2020/courseDescriptionAndMisc/Exams/LearningFromResults}{More instructions on how to use this key can be found here}.}

\textbf{If you have a suggestion to make the keys better, \href{https://forms.gle/CZkbZmPbC9XALEE88}{please fill out the short survey here}.}

\textit{Note: This key is auto-generated and may contain issues and/or errors. The keys are reviewed after each exam to ensure grading is done accurately. If there are issues (like duplicate options), they are noted in the offline gradebook. The keys are a work-in-progress to give students as many resources to improve as possible.}

\rule{\textwidth}{0.4pt}

\begin{enumerate}\litem{
Which of the following intervals describes the Range of the function below?
\[ f(x) = -\log_2{(x+6)}-1 \]

The solution is \( (\infty, \infty) \), which is option E.\begin{enumerate}[label=\Alph*.]
\item \( (-\infty, a), a \in [-2.2, 0.6] \)

$(-\infty, -1)$, which corresponds to using the vertical shift while the Range is $(-\infty, \infty)$.
\item \( [a, \infty), a \in [-6.5, -5.5] \)

$[-1, \infty)$, which corresponds to using the flipped Domain AND including the endpoint.
\item \( [a, \infty), a \in [3.7, 7.6] \)

$[6, \infty)$, which corresponds to using the negative of the horizontal shift AND including the endpoint.
\item \( (-\infty, a), a \in [0.5, 1.1] \)

$(-\infty, 1)$, which corresponds to using the using the negative of vertical shift on $(0, \infty)$.
\item \( (-\infty, \infty) \)

*This is the correct option.
\end{enumerate}

\textbf{General Comment:} \textbf{General Comments}: The domain of a basic logarithmic function is $(0, \infty)$ and the Range is $(-\infty, \infty)$. We can use shifts when finding the Domain, but the Range will always be all Real numbers.
}
\litem{
Which of the following intervals describes the Range of the function below?
\[ f(x) = -e^{x-1}+9 \]

The solution is \( (-\infty, 9) \), which is option C.\begin{enumerate}[label=\Alph*.]
\item \( (-\infty, a], a \in [5, 14] \)

$(-\infty, 9]$, which corresponds to including the endpoint.
\item \( (a, \infty), a \in [-11, -3] \)

$(-9, \infty)$, which corresponds to using the negative vertical shift AND flipping the Range interval.
\item \( (-\infty, a), a \in [5, 14] \)

* $(-\infty, 9)$, which is the correct option.
\item \( [a, \infty), a \in [-11, -3] \)

$[-9, \infty)$, which corresponds to using the negative vertical shift AND flipping the Range interval AND including the endpoint.
\item \( (-\infty, \infty) \)

This corresponds to confusing range of an exponential function with the domain of an exponential function.
\end{enumerate}

\textbf{General Comment:} \textbf{General Comments}: Domain of a basic exponential function is $(-\infty, \infty)$ while the Range is $(0, \infty)$. We can shift these intervals [and even flip when $a<0$!] to find the new Domain/Range.
}
\litem{
Solve the equation for $x$ and choose the interval that contains the solution (if it exists).
\[ 5^{-3x-5} = 27^{-2x+5} \]

The solution is \( x = 13.909 \), which is option B.\begin{enumerate}[label=\Alph*.]
\item \( x \in [-12, -7] \)

$x = -10.000$, which corresponds to solving the numerators as equal while ignoring the bases are different.
\item \( x \in [10.91, 14.91] \)

* $x = 13.909$, which is the correct option.
\item \( x \in [-25.53, -20.53] \)

$x = -24.526$, which corresponds to distributing the $\ln(base)$ to the second term of the exponent only.
\item \( x \in [4.67, 6.67] \)

$x = 5.671$, which corresponds to distributing the $\ln(base)$ to the first term of the exponent only.
\item \( \text{There is no Real solution to the equation.} \)

This corresponds to believing there is no solution since the bases are not powers of each other.
\end{enumerate}

\textbf{General Comment:} \textbf{General Comments:} This question was written so that the bases could not be written the same. You will need to take the log of both sides.
}
\litem{
Which of the following intervals describes the Range of the function below?
\[ f(x) = -e^{x-8}-5 \]

The solution is \( (-\infty, -5) \), which is option A.\begin{enumerate}[label=\Alph*.]
\item \( (-\infty, a), a \in [-8, -4] \)

* $(-\infty, -5)$, which is the correct option.
\item \( (a, \infty), a \in [4, 10] \)

$(5, \infty)$, which corresponds to using the negative vertical shift AND flipping the Range interval.
\item \( (-\infty, a], a \in [-8, -4] \)

$(-\infty, -5]$, which corresponds to including the endpoint.
\item \( [a, \infty), a \in [4, 10] \)

$[5, \infty)$, which corresponds to using the negative vertical shift AND flipping the Range interval AND including the endpoint.
\item \( (-\infty, \infty) \)

This corresponds to confusing range of an exponential function with the domain of an exponential function.
\end{enumerate}

\textbf{General Comment:} \textbf{General Comments}: Domain of a basic exponential function is $(-\infty, \infty)$ while the Range is $(0, \infty)$. We can shift these intervals [and even flip when $a<0$!] to find the new Domain/Range.
}
\litem{
 Solve the equation for $x$ and choose the interval that contains $x$ (if it exists).
\[  18 = \sqrt[3]{\frac{9}{e^{9x}}} \]

The solution is \( x = -0.719, \text{ which does not fit in any of the interval options.} \), which is option E.\begin{enumerate}[label=\Alph*.]
\item \( x \in [0.3, 0.89] \)

$x = 0.719$, which is the negative of the correct solution.
\item \( x \in [-0.67, -0.26] \)

$x = -0.398$, which corresponds to treating any root as a square root.
\item \( x \in [-6.75, -5.82] \)

$x = -6.244$, which corresponds to thinking you don't need to take the natural log of both sides before reducing, as if the right side already has a natural log.
\item \( \text{There is no Real solution to the equation.} \)

This corresponds to believing you cannot solve the equation.
\item \( \text{None of the above.} \)

* $x = -0.719$ is the correct solution and does not fit in any of the other intervals.
\end{enumerate}

\textbf{General Comment:} \textbf{General Comments}: After using the properties of logarithmic functions to break up the right-hand side, use $\ln(e) = 1$ to reduce the question to a linear function to solve. You can put $\ln(9)$ into a calculator if you are having trouble.
}
\litem{
Which of the following intervals describes the Domain of the function below?
\[ f(x) = \log_2{(x+5)}-2 \]

The solution is \( (-5, \infty) \), which is option C.\begin{enumerate}[label=\Alph*.]
\item \( (-\infty, a), a \in [3.8, 7.1] \)

$(-\infty, 5)$, which corresponds to flipping the Domain. Remember: the general for is $a*\log(x-h)+k$, \textbf{where $a$ does not affect the domain}.
\item \( (-\infty, a], a \in [1.5, 4.9] \)

$(-\infty, 2]$, which corresponds to using the negative vertical shift AND including the endpoint AND flipping the domain.
\item \( (a, \infty), a \in [-6.6, -4.1] \)

* $(-5, \infty)$, which is the correct option.
\item \( [a, \infty), a \in [-4.5, -1.2] \)

$[-2, \infty)$, which corresponds to using the vertical shift when shifting the Domain AND including the endpoint.
\item \( (-\infty, \infty) \)

This corresponds to thinking of the range of the log function (or the domain of the exponential function).
\end{enumerate}

\textbf{General Comment:} \textbf{General Comments}: The domain of a basic logarithmic function is $(0, \infty)$ and the Range is $(-\infty, \infty)$. We can use shifts when finding the Domain, but the Range will always be all Real numbers.
}
\litem{
 Solve the equation for $x$ and choose the interval that contains $x$ (if it exists).
\[  18 = \sqrt[5]{\frac{20}{e^{9x}}} \]

The solution is \( x = -1.273, \text{ which does not fit in any of the interval options.} \), which is option E.\begin{enumerate}[label=\Alph*.]
\item \( x \in [-10.88, -10.24] \)

$x = -10.333$, which corresponds to thinking you don't need to take the natural log of both sides before reducing, as if the right side already has a natural log.
\item \( x \in [-0.91, -0.06] \)

$x = -0.309$, which corresponds to treating any root as a square root.
\item \( x \in [0.74, 1.96] \)

$x = 1.273$, which is the negative of the correct solution.
\item \( \text{There is no Real solution to the equation.} \)

This corresponds to believing you cannot solve the equation.
\item \( \text{None of the above.} \)

* $x = -1.273$ is the correct solution and does not fit in any of the other intervals.
\end{enumerate}

\textbf{General Comment:} \textbf{General Comments}: After using the properties of logarithmic functions to break up the right-hand side, use $\ln(e) = 1$ to reduce the question to a linear function to solve. You can put $\ln(20)$ into a calculator if you are having trouble.
}
\litem{
Solve the equation for $x$ and choose the interval that contains the solution (if it exists).
\[ 5^{-3x+2} = \left(\frac{1}{9}\right)^{2x-3} \]

The solution is \( x = -7.774 \), which is option A.\begin{enumerate}[label=\Alph*.]
\item \( x \in [-8.2, -7.4] \)

* $x = -7.774$, which is the correct option.
\item \( x \in [10.7, 12.3] \)

$x = 11.524$, which corresponds to distributing the $\ln(base)$ to the first term of the exponent only.
\item \( x \in [-0.2, 2.9] \)

$x = 1.000$, which corresponds to solving the numerators as equal while ignoring the bases are different.
\item \( x \in [-1.2, 0.1] \)

$x = -0.675$, which corresponds to distributing the $\ln(base)$ to the second term of the exponent only.
\item \( \text{There is no Real solution to the equation.} \)

This corresponds to believing there is no solution since the bases are not powers of each other.
\end{enumerate}

\textbf{General Comment:} \textbf{General Comments:} This question was written so that the bases could not be written the same. You will need to take the log of both sides.
}
\litem{
Solve the equation for $x$ and choose the interval that contains the solution (if it exists).
\[ \log_{3}{(4x+8)}+4 = 2 \]

The solution is \( x = -1.972 \), which is option C.\begin{enumerate}[label=\Alph*.]
\item \( x \in [-4, -3.79] \)

$x = -4.000$, which corresponds to reversing the base and exponent when converting.
\item \( x \in [-0.49, 0.08] \)

$x = 0.000$, which corresponds to reversing the base and exponent when converting and reversing the value with $x$.
\item \( x \in [-1.98, -1.7] \)

* $x = -1.972$, which is the correct option.
\item \( x \in [0.23, 0.31] \)

$x = 0.250$, which corresponds to ignoring the vertical shift when converting to exponential form.
\item \( \text{There is no Real solution to the equation.} \)

Corresponds to believing a negative coefficient within the log equation means there is no Real solution.
\end{enumerate}

\textbf{General Comment:} \textbf{General Comments:} First, get the equation in the form $\log_b{(cx+d)} = a$. Then, convert to $b^a = cx+d$ and solve.
}
\litem{
Solve the equation for $x$ and choose the interval that contains the solution (if it exists).
\[ \log_{4}{(-2x+5)}+6 = 3 \]

The solution is \( x = 2.492 \), which is option C.\begin{enumerate}[label=\Alph*.]
\item \( x \in [-43, -42] \)

$x = -43.000$, which corresponds to reversing the base and exponent when converting and reversing the value with $x$.
\item \( x \in [-30.5, -23.5] \)

$x = -29.500$, which corresponds to ignoring the vertical shift when converting to exponential form.
\item \( x \in [0.49, 3.49] \)

* $x = 2.492$, which is the correct option.
\item \( x \in [-42, -36] \)

$x = -38.000$, which corresponds to reversing the base and exponent when converting.
\item \( \text{There is no Real solution to the equation.} \)

Corresponds to believing a negative coefficient within the log equation means there is no Real solution.
\end{enumerate}

\textbf{General Comment:} \textbf{General Comments:} First, get the equation in the form $\log_b{(cx+d)} = a$. Then, convert to $b^a = cx+d$ and solve.
}
\end{enumerate}

\end{document}