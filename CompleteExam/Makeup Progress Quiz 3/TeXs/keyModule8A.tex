\documentclass{extbook}[14pt]
\usepackage{multicol, enumerate, enumitem, hyperref, color, soul, setspace, parskip, fancyhdr, amssymb, amsthm, amsmath, latexsym, units, mathtools}
\everymath{\displaystyle}
\usepackage[headsep=0.5cm,headheight=0cm, left=1 in,right= 1 in,top= 1 in,bottom= 1 in]{geometry}
\usepackage{dashrule}  % Package to use the command below to create lines between items
\newcommand{\litem}[1]{\item #1

\rule{\textwidth}{0.4pt}}
\pagestyle{fancy}
\lhead{}
\chead{Answer Key for Makeup Progress Quiz 3 Version A}
\rhead{}
\lfoot{1648-1753}
\cfoot{}
\rfoot{Summer C 2021}
\begin{document}
\textbf{This key should allow you to understand why you choose the option you did (beyond just getting a question right or wrong). \href{https://xronos.clas.ufl.edu/mac1105spring2020/courseDescriptionAndMisc/Exams/LearningFromResults}{More instructions on how to use this key can be found here}.}

\textbf{If you have a suggestion to make the keys better, \href{https://forms.gle/CZkbZmPbC9XALEE88}{please fill out the short survey here}.}

\textit{Note: This key is auto-generated and may contain issues and/or errors. The keys are reviewed after each exam to ensure grading is done accurately. If there are issues (like duplicate options), they are noted in the offline gradebook. The keys are a work-in-progress to give students as many resources to improve as possible.}

\rule{\textwidth}{0.4pt}

\begin{enumerate}\litem{
Solve the equation for $x$ and choose the interval that contains the solution (if it exists).
\[ 4^{-5x+5} = 343^{-3x-2} \]The solution is \( x = -1.758 \), which is option B.\begin{enumerate}[label=\Alph*.]
\item \( x \in [-1.17, -0.46] \)

$x = -0.662$, which corresponds to distributing the $\ln(base)$ to the first term of the exponent only.
\item \( x \in [-2.69, -1.42] \)

* $x = -1.758$, which is the correct option.
\item \( x \in [9.02, 9.75] \)

$x = 9.303$, which corresponds to distributing the $\ln(base)$ to the second term of the exponent only.
\item \( x \in [3.35, 3.68] \)

$x = 3.500$, which corresponds to solving the numerators as equal while ignoring the bases are different.
\item \( \text{There is no Real solution to the equation.} \)

This corresponds to believing there is no solution since the bases are not powers of each other.
\end{enumerate}

\textbf{General Comment:} \textbf{General Comments:} This question was written so that the bases could not be written the same. You will need to take the log of both sides.
}
\litem{
Which of the following intervals describes the Domain of the function below?
\[ f(x) = e^{x+2}+7 \]The solution is \( (-\infty, \infty) \), which is option E.\begin{enumerate}[label=\Alph*.]
\item \( (a, \infty), a \in [-14, -4] \)

$(-7, \infty)$, which corresponds to using the negative vertical shift AND flipping the Range interval.
\item \( (-\infty, a), a \in [7, 13] \)

$(-\infty, 7)$, which corresponds to using the correct vertical shift *if we wanted the Range*.
\item \( (-\infty, a], a \in [7, 13] \)

$(-\infty, 7]$, which corresponds to using the correct vertical shift *if we wanted the Range* AND including the endpoint.
\item \( [a, \infty), a \in [-14, -4] \)

$[-7, \infty)$, which corresponds to using the negative vertical shift AND flipping the Range interval AND including the endpoint.
\item \( (-\infty, \infty) \)

* This is the correct option.
\end{enumerate}

\textbf{General Comment:} \textbf{General Comments}: Domain of a basic exponential function is $(-\infty, \infty)$ while the Range is $(0, \infty)$. We can shift these intervals [and even flip when $a<0$!] to find the new Domain/Range.
}
\litem{
Solve the equation for $x$ and choose the interval that contains the solution (if it exists).
\[ 2^{-3x+4} = \left(\frac{1}{25}\right)^{-2x-3} \]The solution is \( x = -0.808 \), which is option A.\begin{enumerate}[label=\Alph*.]
\item \( x \in [-0.9, 0.5] \)

* $x = -0.808$, which is the correct option.
\item \( x \in [6.1, 7.7] \)

$x = 7.000$, which corresponds to solving the numerators as equal while ignoring the bases are different.
\item \( x \in [-8.9, -5] \)

$x = -6.884$, which corresponds to distributing the $\ln(base)$ to the second term of the exponent only.
\item \( x \in [0.5, 1] \)

$x = 0.822$, which corresponds to distributing the $\ln(base)$ to the first term of the exponent only.
\item \( \text{There is no Real solution to the equation.} \)

This corresponds to believing there is no solution since the bases are not powers of each other.
\end{enumerate}

\textbf{General Comment:} \textbf{General Comments:} This question was written so that the bases could not be written the same. You will need to take the log of both sides.
}
\litem{
Solve the equation for $x$ and choose the interval that contains the solution (if it exists).
\[ \log_{5}{(2x+5)}+5 = 3 \]The solution is \( x = -2.480 \), which is option D.\begin{enumerate}[label=\Alph*.]
\item \( x \in [57, 62] \)

$x = 60.000$, which corresponds to ignoring the vertical shift when converting to exponential form.
\item \( x \in [-26.5, -14.5] \)

$x = -18.500$, which corresponds to reversing the base and exponent when converting.
\item \( x \in [-14.5, -11.5] \)

$x = -13.500$, which corresponds to reversing the base and exponent when converting and reversing the value with $x$.
\item \( x \in [-5.48, 2.52] \)

* $x = -2.480$, which is the correct option.
\item \( \text{There is no Real solution to the equation.} \)

Corresponds to believing a negative coefficient within the log equation means there is no Real solution.
\end{enumerate}

\textbf{General Comment:} \textbf{General Comments:} First, get the equation in the form $\log_b{(cx+d)} = a$. Then, convert to $b^a = cx+d$ and solve.
}
\litem{
 Solve the equation for $x$ and choose the interval that contains $x$ (if it exists).
\[  7 = \ln{\sqrt[7]{\frac{9}{e^{5x}}}} \]The solution is \( x = -9.361 \), which is option B.\begin{enumerate}[label=\Alph*.]
\item \( x \in [-2.4, -1.5] \)

$x = -2.361$, which corresponds to treating any root as a square root.
\item \( x \in [-9.7, -9.3] \)

* $x = -9.361$, which is the correct option.
\item \( x \in [-3.5, -2.9] \)

$x = -3.164$, which corresponds to thinking you need to take the natural log of on the left before reducing.
\item \( \text{There is no Real solution to the equation.} \)

This corresponds to believing you cannot solve the equation.
\item \( \text{None of the above.} \)

This corresponds to making an unexpected error.
\end{enumerate}

\textbf{General Comment:} \textbf{General Comments}: After using the properties of logarithmic functions to break up the right-hand side, use $\ln(e) = 1$ to reduce the question to a linear function to solve. You can put $\ln(9)$ into a calculator if you are having trouble.
}
\litem{
Which of the following intervals describes the Domain of the function below?
\[ f(x) = \log_2{(x-5)}+9 \]The solution is \( (5, \infty) \), which is option A.\begin{enumerate}[label=\Alph*.]
\item \( (a, \infty), a \in [4.7, 5.6] \)

* $(5, \infty)$, which is the correct option.
\item \( [a, \infty), a \in [8.7, 9.4] \)

$[9, \infty)$, which corresponds to using the vertical shift when shifting the Domain AND including the endpoint.
\item \( (-\infty, a), a \in [-5.2, -1.7] \)

$(-\infty, -5)$, which corresponds to flipping the Domain. Remember: the general for is $a*\log(x-h)+k$, \textbf{where $a$ does not affect the domain}.
\item \( (-\infty, a], a \in [-13.9, -7.3] \)

$(-\infty, -9]$, which corresponds to using the negative vertical shift AND including the endpoint AND flipping the domain.
\item \( (-\infty, \infty) \)

This corresponds to thinking of the range of the log function (or the domain of the exponential function).
\end{enumerate}

\textbf{General Comment:} \textbf{General Comments}: The domain of a basic logarithmic function is $(0, \infty)$ and the Range is $(-\infty, \infty)$. We can use shifts when finding the Domain, but the Range will always be all Real numbers.
}
\litem{
Which of the following intervals describes the Range of the function below?
\[ f(x) = e^{x-9}+6 \]The solution is \( (6, \infty) \), which is option D.\begin{enumerate}[label=\Alph*.]
\item \( (-\infty, a), a \in [-8, -1] \)

$(-\infty, -6)$, which corresponds to using the negative vertical shift AND flipping the Range interval.
\item \( [a, \infty), a \in [4, 7] \)

$[6, \infty)$, which corresponds to including the endpoint.
\item \( (-\infty, a], a \in [-8, -1] \)

$(-\infty, -6]$, which corresponds to using the negative vertical shift AND flipping the Range interval AND including the endpoint.
\item \( (a, \infty), a \in [4, 7] \)

* $(6, \infty)$, which is the correct option.
\item \( (-\infty, \infty) \)

This corresponds to confusing range of an exponential function with the domain of an exponential function.
\end{enumerate}

\textbf{General Comment:} \textbf{General Comments}: Domain of a basic exponential function is $(-\infty, \infty)$ while the Range is $(0, \infty)$. We can shift these intervals [and even flip when $a<0$!] to find the new Domain/Range.
}
\litem{
Solve the equation for $x$ and choose the interval that contains the solution (if it exists).
\[ \log_{3}{(-4x+6)}+4 = 2 \]The solution is \( x = 1.472 \), which is option D.\begin{enumerate}[label=\Alph*.]
\item \( x \in [0.27, 0.86] \)

$x = 0.500$, which corresponds to reversing the base and exponent when converting and reversing the value with $x$.
\item \( x \in [2.77, 4.53] \)

$x = 3.500$, which corresponds to reversing the base and exponent when converting.
\item \( x \in [-1.97, -0.05] \)

$x = -0.750$, which corresponds to ignoring the vertical shift when converting to exponential form.
\item \( x \in [1.42, 2.24] \)

* $x = 1.472$, which is the correct option.
\item \( \text{There is no Real solution to the equation.} \)

Corresponds to believing a negative coefficient within the log equation means there is no Real solution.
\end{enumerate}

\textbf{General Comment:} \textbf{General Comments:} First, get the equation in the form $\log_b{(cx+d)} = a$. Then, convert to $b^a = cx+d$ and solve.
}
\litem{
Which of the following intervals describes the Domain of the function below?
\[ f(x) = \log_2{(x-3)}-2 \]The solution is \( (3, \infty) \), which is option D.\begin{enumerate}[label=\Alph*.]
\item \( (-\infty, a], a \in [1.6, 2.21] \)

$(-\infty, 2]$, which corresponds to using the negative vertical shift AND including the endpoint AND flipping the domain.
\item \( [a, \infty), a \in [-2.49, -1.67] \)

$[-2, \infty)$, which corresponds to using the vertical shift when shifting the Domain AND including the endpoint.
\item \( (-\infty, a), a \in [-3.51, -2.15] \)

$(-\infty, -3)$, which corresponds to flipping the Domain. Remember: the general for is $a*\log(x-h)+k$, \textbf{where $a$ does not affect the domain}.
\item \( (a, \infty), a \in [2.45, 3.96] \)

* $(3, \infty)$, which is the correct option.
\item \( (-\infty, \infty) \)

This corresponds to thinking of the range of the log function (or the domain of the exponential function).
\end{enumerate}

\textbf{General Comment:} \textbf{General Comments}: The domain of a basic logarithmic function is $(0, \infty)$ and the Range is $(-\infty, \infty)$. We can use shifts when finding the Domain, but the Range will always be all Real numbers.
}
\litem{
 Solve the equation for $x$ and choose the interval that contains $x$ (if it exists).
\[  15 = \ln{\sqrt[5]{\frac{29}{e^{4x}}}} \]The solution is \( x = -17.908 \), which is option B.\begin{enumerate}[label=\Alph*.]
\item \( x \in [-5.23, -1.23] \)

$x = -4.227$, which corresponds to thinking you need to take the natural log of on the left before reducing.
\item \( x \in [-20.91, -16.91] \)

* $x = -17.908$, which is the correct option.
\item \( x \in [-6.66, -4.66] \)

$x = -6.658$, which corresponds to treating any root as a square root.
\item \( \text{There is no Real solution to the equation.} \)

This corresponds to believing you cannot solve the equation.
\item \( \text{None of the above.} \)

This corresponds to making an unexpected error.
\end{enumerate}

\textbf{General Comment:} \textbf{General Comments}: After using the properties of logarithmic functions to break up the right-hand side, use $\ln(e) = 1$ to reduce the question to a linear function to solve. You can put $\ln(29)$ into a calculator if you are having trouble.
}
\end{enumerate}

\end{document}