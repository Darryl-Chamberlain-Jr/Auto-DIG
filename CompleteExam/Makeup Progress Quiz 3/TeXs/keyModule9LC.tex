\documentclass{extbook}[14pt]
\usepackage{multicol, enumerate, enumitem, hyperref, color, soul, setspace, parskip, fancyhdr, amssymb, amsthm, amsmath, bbm, latexsym, units, mathtools}
\everymath{\displaystyle}
\usepackage[headsep=0.5cm,headheight=0cm, left=1 in,right= 1 in,top= 1 in,bottom= 1 in]{geometry}
\usepackage{dashrule}  % Package to use the command below to create lines between items
\newcommand{\litem}[1]{\item #1

\rule{\textwidth}{0.4pt}}
\pagestyle{fancy}
\lhead{}
\chead{Answer Key for Makeup Progress Quiz 3 Version C}
\rhead{}
\lfoot{4315-3397}
\cfoot{}
\rfoot{Fall 2020}
\begin{document}
\textbf{This key should allow you to understand why you choose the option you did (beyond just getting a question right or wrong). \href{https://xronos.clas.ufl.edu/mac1105spring2020/courseDescriptionAndMisc/Exams/LearningFromResults}{More instructions on how to use this key can be found here}.}

\textbf{If you have a suggestion to make the keys better, \href{https://forms.gle/CZkbZmPbC9XALEE88}{please fill out the short survey here}.}

\textit{Note: This key is auto-generated and may contain issues and/or errors. The keys are reviewed after each exam to ensure grading is done accurately. If there are issues (like duplicate options), they are noted in the offline gradebook. The keys are a work-in-progress to give students as many resources to improve as possible.}

\rule{\textwidth}{0.4pt}

\begin{enumerate}\litem{
Find the inverse of the function below (if it exists). Then, evaluate the inverse at $x = 13$ and choose the interval the $f^{-1}(13)$ belongs to.
\[ f(x) = \sqrt[3]{5 x - 4} \]

The solution is \( 440.2 \), which is option B.\begin{enumerate}[label=\Alph*.]
\item \( f^{-1}(13) \in [-439.6, -437.3] \)

 This solution corresponds to distractor 3.
\item \( f^{-1}(13) \in [440, 441] \)

* This is the correct solution.
\item \( f^{-1}(13) \in [-442.7, -439] \)

 This solution corresponds to distractor 2.
\item \( f^{-1}(13) \in [435.5, 439] \)

 Distractor 1: This corresponds to 
\item \( \text{ The function is not invertible for all Real numbers. } \)

 This solution corresponds to distractor 4.
\end{enumerate}

\textbf{General Comment:} Be sure you check that the function is 1-1 before trying to find the inverse!
}
\litem{
Subtract the following functions, then choose the domain of the resulting function from the list below.
\[ f(x) = \frac{2}{5x+34} \text{ and } g(x) = x^{4} +4 x^{3} + x^{2} +7 x \]

The solution is \( \text{ The domain is all Real numbers except } x = -6.8 \), which is option B.\begin{enumerate}[label=\Alph*.]
\item \( \text{ The domain is all Real numbers greater than or equal to } x = a, \text{ where } a \in [-6.5, 7.5] \)


\item \( \text{ The domain is all Real numbers except } x = a, \text{ where } a \in [-6.8, -2.8] \)


\item \( \text{ The domain is all Real numbers less than or equal to } x = a, \text{ where } a \in [2.2, 7.2] \)


\item \( \text{ The domain is all Real numbers except } x = a \text{ and } x = b, \text{ where } a \in [1.75, 4.75] \text{ and } b \in [-5.67, -1.67] \)


\item \( \text{ The domain is all Real numbers. } \)


\end{enumerate}

\textbf{General Comment:} The new domain is the intersection of the previous domains.
}
\litem{
Add the following functions, then choose the domain of the resulting function from the list below.
\[ f(x) = \sqrt{-6x-18}  \text{ and } g(x) = 4x^{2} +2 x + 4 \]

The solution is \( \text{ The domain is all Real numbers less than or equal to} x = -3.0. \), which is option B.\begin{enumerate}[label=\Alph*.]
\item \( \text{ The domain is all Real numbers except } x = a, \text{ where } a \in [-7.67, -3.67] \)


\item \( \text{ The domain is all Real numbers less than or equal to } x = a, \text{ where } a \in [-4, -1] \)


\item \( \text{ The domain is all Real numbers greater than or equal to } x = a, \text{ where } a \in [-7.4, -2.4] \)


\item \( \text{ The domain is all Real numbers except } x = a \text{ and } x = b, \text{ where } a \in [-6.2, -1.2] \text{ and } b \in [-8.17, -1.17] \)


\item \( \text{ The domain is all Real numbers. } \)


\end{enumerate}

\textbf{General Comment:} The new domain is the intersection of the previous domains.
}
\litem{
Choose the interval below that $f$ composed with $g$ at $x=1$ is in.
\[ f(x) = -3x^{3} +3 x^{2} +4 x \text{ and } g(x) = x^{3} -1 x^{2} -2 x \]

The solution is \( 28.0 \), which is option B.\begin{enumerate}[label=\Alph*.]
\item \( (f \circ g)(1) \in [14, 20] \)

 Distractor 2: Corresponds to being slightly off from the solution.
\item \( (f \circ g)(1) \in [28, 31] \)

* This is the correct solution
\item \( (f \circ g)(1) \in [47, 52] \)

 Distractor 3: Corresponds to being slightly off from the solution.
\item \( (f \circ g)(1) \in [37, 46] \)

 Distractor 1: Corresponds to reversing the composition.
\item \( \text{It is not possible to compose the two functions.} \)


\end{enumerate}

\textbf{General Comment:} $f$ composed with $g$ at $x$ means $f(g(x))$. The order matters!
}
\litem{
Choose the interval below that $f$ composed with $g$ at $x=-1$ is in.
\[ f(x) = x^{3} +2 x^{2} +4 x \text{ and } g(x) = -3x^{3} -1 x^{2} +2 x \]

The solution is \( 0.0 \), which is option B.\begin{enumerate}[label=\Alph*.]
\item \( (f \circ g)(-1) \in [56, 65] \)

 Distractor 3: Corresponds to being slightly off from the solution.
\item \( (f \circ g)(-1) \in [0, 1] \)

* This is the correct solution
\item \( (f \circ g)(-1) \in [6, 14] \)

 Distractor 2: Corresponds to being slightly off from the solution.
\item \( (f \circ g)(-1) \in [64, 69] \)

 Distractor 1: Corresponds to reversing the composition.
\item \( \text{It is not possible to compose the two functions.} \)


\end{enumerate}

\textbf{General Comment:} $f$ composed with $g$ at $x$ means $f(g(x))$. The order matters!
}
\litem{
Find the inverse of the function below (if it exists). Then, evaluate the inverse at $x = -11$ and choose the interval that $f^{-1}(-11)$ belongs to.
\[ f(x) = 5 x^2 - 3 \]

The solution is \( \text{ The function is not invertible for all Real numbers. } \), which is option E.\begin{enumerate}[label=\Alph*.]
\item \( f^{-1}(-11) \in [0.99, 1.41] \)

 Distractor 1: This corresponds to trying to find the inverse even though the function is not 1-1. 
\item \( f^{-1}(-11) \in [2.23, 2.55] \)

 Distractor 3: This corresponds to finding the (nonexistent) inverse and dividing by a negative.
\item \( f^{-1}(-11) \in [1.51, 1.87] \)

 Distractor 2: This corresponds to finding the (nonexistent) inverse and not subtracting by the vertical shift.
\item \( f^{-1}(-11) \in [5.15, 5.38] \)

 Distractor 4: This corresponds to both distractors 2 and 3.
\item \( \text{ The function is not invertible for all Real numbers. } \)

* This is the correct option.
\end{enumerate}

\textbf{General Comment:} Be sure you check that the function is 1-1 before trying to find the inverse!
}
\litem{
Determine whether the function below is 1-1.
\[ f(x) = 16 x^2 + 32 x - 425 \]

The solution is \( \text{no} \), which is option A.\begin{enumerate}[label=\Alph*.]
\item \( \text{No, because there is a $y$-value that goes to 2 different $x$-values.} \)

* This is the solution.
\item \( \text{No, because the domain of the function is not $(-\infty, \infty)$.} \)

Corresponds to believing 1-1 means the domain is all Real numbers.
\item \( \text{Yes, the function is 1-1.} \)

Corresponds to believing the function passes the Horizontal Line test.
\item \( \text{No, because there is an $x$-value that goes to 2 different $y$-values.} \)

Corresponds to the Vertical Line test, which checks if an expression is a function.
\item \( \text{No, because the range of the function is not $(-\infty, \infty)$.} \)

Corresponds to believing 1-1 means the range is all Real numbers.
\end{enumerate}

\textbf{General Comment:} There are only two valid options: The function is 1-1 OR No because there is a $y$-value that goes to 2 different $x$-values.
}
\litem{
Find the inverse of the function below. Then, evaluate the inverse at $x = 6$ and choose the interval that $f^{-1}(6)$ belongs to.
\[ f(x) = \ln{(x+4)}+2 \]

The solution is \( f^{-1}(6) = 50.598 \), which is option D.\begin{enumerate}[label=\Alph*.]
\item \( f^{-1}(6) \in [53.6, 65.6] \)

 This solution corresponds to distractor 3.
\item \( f^{-1}(6) \in [22024.47, 22031.47] \)

 This solution corresponds to distractor 4.
\item \( f^{-1}(6) \in [7.39, 12.39] \)

 This solution corresponds to distractor 2.
\item \( f^{-1}(6) \in [49.6, 54.6] \)

 This is the solution.
\item \( f^{-1}(6) \in [2974.96, 2983.96] \)

 This solution corresponds to distractor 1.
\end{enumerate}

\textbf{General Comment:} Natural log and exponential functions always have an inverse. Once you switch the $x$ and $y$, use the conversion $ e^y = x \leftrightarrow y=\ln(x)$.
}
\litem{
Determine whether the function below is 1-1.
\[ f(x) = 25 x^2 - 250 x + 625 \]

The solution is \( \text{no} \), which is option B.\begin{enumerate}[label=\Alph*.]
\item \( \text{Yes, the function is 1-1.} \)

Corresponds to believing the function passes the Horizontal Line test.
\item \( \text{No, because there is a $y$-value that goes to 2 different $x$-values.} \)

* This is the solution.
\item \( \text{No, because the range of the function is not $(-\infty, \infty)$.} \)

Corresponds to believing 1-1 means the range is all Real numbers.
\item \( \text{No, because there is an $x$-value that goes to 2 different $y$-values.} \)

Corresponds to the Vertical Line test, which checks if an expression is a function.
\item \( \text{No, because the domain of the function is not $(-\infty, \infty)$.} \)

Corresponds to believing 1-1 means the domain is all Real numbers.
\end{enumerate}

\textbf{General Comment:} There are only two valid options: The function is 1-1 OR No because there is a $y$-value that goes to 2 different $x$-values.
}
\litem{
Find the inverse of the function below. Then, evaluate the inverse at $x = 7$ and choose the interval that $f^{-1}(7)$ belongs to.
\[ f(x) = \ln{(x-2)}-2 \]

The solution is \( f^{-1}(7) = 8105.084 \), which is option C.\begin{enumerate}[label=\Alph*.]
\item \( f^{-1}(7) \in [148.41, 155.41] \)

 This solution corresponds to distractor 1.
\item \( f^{-1}(7) \in [8099.08, 8103.08] \)

 This solution corresponds to distractor 3.
\item \( f^{-1}(7) \in [8102.08, 8107.08] \)

 This is the solution.
\item \( f^{-1}(7) \in [146.41, 148.41] \)

 This solution corresponds to distractor 4.
\item \( f^{-1}(7) \in [8099.08, 8103.08] \)

 This solution corresponds to distractor 2.
\end{enumerate}

\textbf{General Comment:} Natural log and exponential functions always have an inverse. Once you switch the $x$ and $y$, use the conversion $ e^y = x \leftrightarrow y=\ln(x)$.
}
\end{enumerate}

\end{document}