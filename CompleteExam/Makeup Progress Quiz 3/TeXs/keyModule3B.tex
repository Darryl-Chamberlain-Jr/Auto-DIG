\documentclass{extbook}[14pt]
\usepackage{multicol, enumerate, enumitem, hyperref, color, soul, setspace, parskip, fancyhdr, amssymb, amsthm, amsmath, latexsym, units, mathtools}
\everymath{\displaystyle}
\usepackage[headsep=0.5cm,headheight=0cm, left=1 in,right= 1 in,top= 1 in,bottom= 1 in]{geometry}
\usepackage{dashrule}  % Package to use the command below to create lines between items
\newcommand{\litem}[1]{\item #1

\rule{\textwidth}{0.4pt}}
\pagestyle{fancy}
\lhead{}
\chead{Answer Key for Makeup Progress Quiz 3 Version B}
\rhead{}
\lfoot{1648-1753}
\cfoot{}
\rfoot{Summer C 2021}
\begin{document}
\textbf{This key should allow you to understand why you choose the option you did (beyond just getting a question right or wrong). \href{https://xronos.clas.ufl.edu/mac1105spring2020/courseDescriptionAndMisc/Exams/LearningFromResults}{More instructions on how to use this key can be found here}.}

\textbf{If you have a suggestion to make the keys better, \href{https://forms.gle/CZkbZmPbC9XALEE88}{please fill out the short survey here}.}

\textit{Note: This key is auto-generated and may contain issues and/or errors. The keys are reviewed after each exam to ensure grading is done accurately. If there are issues (like duplicate options), they are noted in the offline gradebook. The keys are a work-in-progress to give students as many resources to improve as possible.}

\rule{\textwidth}{0.4pt}

\begin{enumerate}\litem{
Using an interval or intervals, describe all the $x$-values within or including a distance of the given values.
\[ \text{ No more than } 6 \text{ units from the number } -6. \]The solution is \( [-12, 0] \), which is option C.\begin{enumerate}[label=\Alph*.]
\item \( (-\infty, -12] \cup [0, \infty) \)

This describes the values no less than 6 from -6
\item \( (-\infty, -12) \cup (0, \infty) \)

This describes the values more than 6 from -6
\item \( [-12, 0] \)

This describes the values no more than 6 from -6
\item \( (-12, 0) \)

This describes the values less than 6 from -6
\item \( \text{None of the above} \)

You likely thought the values in the interval were not correct.
\end{enumerate}

\textbf{General Comment:} When thinking about this language, it helps to draw a number line and try points.
}
\litem{
Solve the linear inequality below. Then, choose the constant and interval combination that describes the solution set.
\[ -5x + 7 \geq 5x -3 \]The solution is \( (-\infty, 1.0] \), which is option B.\begin{enumerate}[label=\Alph*.]
\item \( (-\infty, a], \text{ where } a \in [-2.6, -0.6] \)

 $(-\infty, -1.0]$, which corresponds to negating the endpoint of the solution.
\item \( (-\infty, a], \text{ where } a \in [0.6, 2.6] \)

* $(-\infty, 1.0]$, which is the correct option.
\item \( [a, \infty), \text{ where } a \in [0, 3.1] \)

 $[1.0, \infty)$, which corresponds to switching the direction of the interval. You likely did this if you did not flip the inequality when dividing by a negative!
\item \( [a, \infty), \text{ where } a \in [-1.5, -0.3] \)

 $[-1.0, \infty)$, which corresponds to switching the direction of the interval AND negating the endpoint. You likely did this if you did not flip the inequality when dividing by a negative as well as not moving values over to a side properly.
\item \( \text{None of the above}. \)

You may have chosen this if you thought the inequality did not match the ends of the intervals.
\end{enumerate}

\textbf{General Comment:} Remember that less/greater than or equal to includes the endpoint, while less/greater do not. Also, remember that you need to flip the inequality when you multiply or divide by a negative.
}
\litem{
Solve the linear inequality below. Then, choose the constant and interval combination that describes the solution set.
\[ -7 - 4 x \leq \frac{-27 x - 4}{8} < -3 - 4 x \]The solution is \( \text{None of the above.} \), which is option E.\begin{enumerate}[label=\Alph*.]
\item \( (-\infty, a) \cup [b, \infty), \text{ where } a \in [9, 13.5] \text{ and } b \in [3, 10.5] \)

$(-\infty, 10.40) \cup [4.00, \infty)$, which corresponds to displaying the and-inequality as an or-inequality AND flipping the inequality AND getting negatives of the actual endpoints.
\item \( [a, b), \text{ where } a \in [6.75, 12.75] \text{ and } b \in [0.75, 8.25] \)

$[10.40, 4.00)$, which is the correct interval but negatives of the actual endpoints.
\item \( (a, b], \text{ where } a \in [9, 12.75] \text{ and } b \in [3, 5.25] \)

$(10.40, 4.00]$, which corresponds to flipping the inequality and getting negatives of the actual endpoints.
\item \( (-\infty, a] \cup (b, \infty), \text{ where } a \in [8.25, 12] \text{ and } b \in [3, 9.75] \)

$(-\infty, 10.40] \cup (4.00, \infty)$, which corresponds to displaying the and-inequality as an or-inequality and getting negatives of the actual endpoints.
\item \( \text{None of the above.} \)

* This is correct as the answer should be $[-10.40, -4.00)$.
\end{enumerate}

\textbf{General Comment:} To solve, you will need to break up the compound inequality into two inequalities. Be sure to keep track of the inequality! It may be best to draw a number line and graph your solution.
}
\litem{
Solve the linear inequality below. Then, choose the constant and interval combination that describes the solution set.
\[ -7 + 5 x > 6 x \text{ or } -8 + 9 x < 11 x \]The solution is \( (-\infty, -7.0) \text{ or } (-4.0, \infty) \), which is option A.\begin{enumerate}[label=\Alph*.]
\item \( (-\infty, a) \cup (b, \infty), \text{ where } a \in [-7.5, -3.75] \text{ and } b \in [-4.5, -2.25] \)

 * Correct option.
\item \( (-\infty, a] \cup [b, \infty), \text{ where } a \in [-9, -5.25] \text{ and } b \in [-5.25, -1.5] \)

Corresponds to including the endpoints (when they should be excluded).
\item \( (-\infty, a) \cup (b, \infty), \text{ where } a \in [0, 4.5] \text{ and } b \in [5.25, 12] \)

Corresponds to inverting the inequality and negating the solution.
\item \( (-\infty, a] \cup [b, \infty), \text{ where } a \in [-2.25, 5.25] \text{ and } b \in [3.75, 9.75] \)

Corresponds to including the endpoints AND negating.
\item \( (-\infty, \infty) \)

Corresponds to the variable canceling, which does not happen in this instance.
\end{enumerate}

\textbf{General Comment:} When multiplying or dividing by a negative, flip the sign.
}
\litem{
Solve the linear inequality below. Then, choose the constant and interval combination that describes the solution set.
\[ \frac{10}{3} - \frac{4}{2} x > \frac{7}{6} x - \frac{8}{4} \]The solution is \( (-\infty, 1.684) \), which is option B.\begin{enumerate}[label=\Alph*.]
\item \( (-\infty, a), \text{ where } a \in [-4.5, 0.75] \)

 $(-\infty, -1.684)$, which corresponds to negating the endpoint of the solution.
\item \( (-\infty, a), \text{ where } a \in [-0.75, 2.25] \)

* $(-\infty, 1.684)$, which is the correct option.
\item \( (a, \infty), \text{ where } a \in [0.75, 2.25] \)

 $(1.684, \infty)$, which corresponds to switching the direction of the interval. You likely did this if you did not flip the inequality when dividing by a negative!
\item \( (a, \infty), \text{ where } a \in [-5.25, 0.75] \)

 $(-1.684, \infty)$, which corresponds to switching the direction of the interval AND negating the endpoint. You likely did this if you did not flip the inequality when dividing by a negative as well as not moving values over to a side properly.
\item \( \text{None of the above}. \)

You may have chosen this if you thought the inequality did not match the ends of the intervals.
\end{enumerate}

\textbf{General Comment:} Remember that less/greater than or equal to includes the endpoint, while less/greater do not. Also, remember that you need to flip the inequality when you multiply or divide by a negative.
}
\litem{
Solve the linear inequality below. Then, choose the constant and interval combination that describes the solution set.
\[ \frac{-8}{5} - \frac{10}{7} x > \frac{-5}{9} x + \frac{4}{8} \]The solution is \( (-\infty, -2.405) \), which is option B.\begin{enumerate}[label=\Alph*.]
\item \( (-\infty, a), \text{ where } a \in [0.75, 3.75] \)

 $(-\infty, 2.405)$, which corresponds to negating the endpoint of the solution.
\item \( (-\infty, a), \text{ where } a \in [-6, -1.5] \)

* $(-\infty, -2.405)$, which is the correct option.
\item \( (a, \infty), \text{ where } a \in [1.5, 6.75] \)

 $(2.405, \infty)$, which corresponds to switching the direction of the interval AND negating the endpoint. You likely did this if you did not flip the inequality when dividing by a negative as well as not moving values over to a side properly.
\item \( (a, \infty), \text{ where } a \in [-7.5, 0.75] \)

 $(-2.405, \infty)$, which corresponds to switching the direction of the interval. You likely did this if you did not flip the inequality when dividing by a negative!
\item \( \text{None of the above}. \)

You may have chosen this if you thought the inequality did not match the ends of the intervals.
\end{enumerate}

\textbf{General Comment:} Remember that less/greater than or equal to includes the endpoint, while less/greater do not. Also, remember that you need to flip the inequality when you multiply or divide by a negative.
}
\litem{
Solve the linear inequality below. Then, choose the constant and interval combination that describes the solution set.
\[ -7 + 4 x > 5 x \text{ or } 8 + 4 x < 5 x \]The solution is \( (-\infty, -7.0) \text{ or } (8.0, \infty) \), which is option D.\begin{enumerate}[label=\Alph*.]
\item \( (-\infty, a] \cup [b, \infty), \text{ where } a \in [-7.35, -5.33] \text{ and } b \in [7.2, 10.5] \)

Corresponds to including the endpoints (when they should be excluded).
\item \( (-\infty, a] \cup [b, \infty), \text{ where } a \in [-9.3, -7.95] \text{ and } b \in [5.02, 7.42] \)

Corresponds to including the endpoints AND negating.
\item \( (-\infty, a) \cup (b, \infty), \text{ where } a \in [-8.81, -7.65] \text{ and } b \in [6.46, 7.14] \)

Corresponds to inverting the inequality and negating the solution.
\item \( (-\infty, a) \cup (b, \infty), \text{ where } a \in [-7.34, -6.47] \text{ and } b \in [7.07, 8.38] \)

 * Correct option.
\item \( (-\infty, \infty) \)

Corresponds to the variable canceling, which does not happen in this instance.
\end{enumerate}

\textbf{General Comment:} When multiplying or dividing by a negative, flip the sign.
}
\litem{
Using an interval or intervals, describe all the $x$-values within or including a distance of the given values.
\[ \text{ No less than } 6 \text{ units from the number } 4. \]The solution is \( (-\infty, -2] \cup [10, \infty) \), which is option C.\begin{enumerate}[label=\Alph*.]
\item \( (-2, 10) \)

This describes the values less than 6 from 4
\item \( (-\infty, -2) \cup (10, \infty) \)

This describes the values more than 6 from 4
\item \( (-\infty, -2] \cup [10, \infty) \)

This describes the values no less than 6 from 4
\item \( [-2, 10] \)

This describes the values no more than 6 from 4
\item \( \text{None of the above} \)

You likely thought the values in the interval were not correct.
\end{enumerate}

\textbf{General Comment:} When thinking about this language, it helps to draw a number line and try points.
}
\litem{
Solve the linear inequality below. Then, choose the constant and interval combination that describes the solution set.
\[ 8 - 9 x < \frac{-22 x - 5}{4} \leq 8 - 6 x \]The solution is \( (2.64, 18.50] \), which is option C.\begin{enumerate}[label=\Alph*.]
\item \( (-\infty, a) \cup [b, \infty), \text{ where } a \in [0, 6.75] \text{ and } b \in [12, 21.75] \)

$(-\infty, 2.64) \cup [18.50, \infty)$, which corresponds to displaying the and-inequality as an or-inequality.
\item \( (-\infty, a] \cup (b, \infty), \text{ where } a \in [-1.5, 5.25] \text{ and } b \in [16.5, 21] \)

$(-\infty, 2.64] \cup (18.50, \infty)$, which corresponds to displaying the and-inequality as an or-inequality AND flipping the inequality.
\item \( (a, b], \text{ where } a \in [2.25, 5.25] \text{ and } b \in [18, 20.25] \)

* $(2.64, 18.50]$, which is the correct option.
\item \( [a, b), \text{ where } a \in [-0.75, 6.75] \text{ and } b \in [17.25, 23.25] \)

$[2.64, 18.50)$, which corresponds to flipping the inequality.
\item \( \text{None of the above.} \)


\end{enumerate}

\textbf{General Comment:} To solve, you will need to break up the compound inequality into two inequalities. Be sure to keep track of the inequality! It may be best to draw a number line and graph your solution.
}
\litem{
Solve the linear inequality below. Then, choose the constant and interval combination that describes the solution set.
\[ -4x -5 \leq 9x + 3 \]The solution is \( [-0.615, \infty) \), which is option A.\begin{enumerate}[label=\Alph*.]
\item \( [a, \infty), \text{ where } a \in [-1.05, -0.12] \)

* $[-0.615, \infty)$, which is the correct option.
\item \( (-\infty, a], \text{ where } a \in [-2.84, 0.51] \)

 $(-\infty, -0.615]$, which corresponds to switching the direction of the interval. You likely did this if you did not flip the inequality when dividing by a negative!
\item \( (-\infty, a], \text{ where } a \in [0.12, 0.84] \)

 $(-\infty, 0.615]$, which corresponds to switching the direction of the interval AND negating the endpoint. You likely did this if you did not flip the inequality when dividing by a negative as well as not moving values over to a side properly.
\item \( [a, \infty), \text{ where } a \in [-0.01, 1.34] \)

 $[0.615, \infty)$, which corresponds to negating the endpoint of the solution.
\item \( \text{None of the above}. \)

You may have chosen this if you thought the inequality did not match the ends of the intervals.
\end{enumerate}

\textbf{General Comment:} Remember that less/greater than or equal to includes the endpoint, while less/greater do not. Also, remember that you need to flip the inequality when you multiply or divide by a negative.
}
\end{enumerate}

\end{document}