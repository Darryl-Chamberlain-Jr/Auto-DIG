\documentclass[14pt]{extbook}
\usepackage{multicol, enumerate, enumitem, hyperref, color, soul, setspace, parskip, fancyhdr} %General Packages
\usepackage{amssymb, amsthm, amsmath, latexsym, units, mathtools} %Math Packages
\everymath{\displaystyle} %All math in Display Style
% Packages with additional options
\usepackage[headsep=0.5cm,headheight=12pt, left=1 in,right= 1 in,top= 1 in,bottom= 1 in]{geometry}
\usepackage[usenames,dvipsnames]{xcolor}
\usepackage{dashrule}  % Package to use the command below to create lines between items
\newcommand{\litem}[1]{\item#1\hspace*{-1cm}\rule{\textwidth}{0.4pt}}
\pagestyle{fancy}
\lhead{Makeup Progress Quiz 3}
\chead{}
\rhead{Version ALL}
\lfoot{1648-1753}
\cfoot{}
\rfoot{Summer C 2021}
\begin{document}

\begin{enumerate}
\litem{
Factor the polynomial below completely. Then, choose the intervals the zeros of the polynomial belong to, where $z_1 \leq z_2 \leq z_3$. \textit{To make the problem easier, all zeros are between -5 and 5.}\[ f(x) = 8x^{3} +14 x^{2} -63 x + 36 \]\begin{enumerate}[label=\Alph*.]
\item \( z_1 \in [-3.28, -2.73], \text{   }  z_2 \in [-0.41, -0.35], \text{   and   } z_3 \in [3.73, 4.24] \)
\item \( z_1 \in [-4.4, -3.98], \text{   }  z_2 \in [0.74, 0.77], \text{   and   } z_3 \in [1.49, 1.67] \)
\item \( z_1 \in [-1.44, -1.26], \text{   }  z_2 \in [-0.71, -0.63], \text{   and   } z_3 \in [3.73, 4.24] \)
\item \( z_1 \in [-1.57, -1.35], \text{   }  z_2 \in [-0.77, -0.71], \text{   and   } z_3 \in [3.73, 4.24] \)
\item \( z_1 \in [-4.4, -3.98], \text{   }  z_2 \in [0.62, 0.69], \text{   and   } z_3 \in [0.96, 1.36] \)

\end{enumerate} }
\litem{
Perform the division below. Then, find the intervals that correspond to the quotient in the form $ax^2+bx+c$ and remainder $r$.\[ \frac{8x^{3} -56 x -51}{x -3} \]\begin{enumerate}[label=\Alph*.]
\item \( a \in [8, 12], b \in [19, 29], c \in [13, 17], \text{ and } r \in [-4, 2]. \)
\item \( a \in [23, 29], b \in [-72, -69], c \in [160, 162], \text{ and } r \in [-533, -529]. \)
\item \( a \in [23, 29], b \in [69, 79], c \in [160, 162], \text{ and } r \in [426, 431]. \)
\item \( a \in [8, 12], b \in [11, 19], c \in [-24, -23], \text{ and } r \in [-99, -95]. \)
\item \( a \in [8, 12], b \in [-24, -19], c \in [13, 17], \text{ and } r \in [-99, -95]. \)

\end{enumerate} }
\litem{
Perform the division below. Then, find the intervals that correspond to the quotient in the form $ax^2+bx+c$ and remainder $r$.\[ \frac{15x^{3} +52 x^{2} -48 x -66}{x + 4} \]\begin{enumerate}[label=\Alph*.]
\item \( a \in [7, 17], \text{   } b \in [-25, -21], \text{   } c \in [63, 72], \text{   and   } r \in [-404, -400]. \)
\item \( a \in [-68, -58], \text{   } b \in [290, 297], \text{   } c \in [-1220, -1214], \text{   and   } r \in [4796, 4800]. \)
\item \( a \in [7, 17], \text{   } b \in [-13, -4], \text{   } c \in [-17, -14], \text{   and   } r \in [-4, 0]. \)
\item \( a \in [7, 17], \text{   } b \in [107, 113], \text{   } c \in [400, 403], \text{   and   } r \in [1534, 1539]. \)
\item \( a \in [-68, -58], \text{   } b \in [-192, -185], \text{   } c \in [-800, -796], \text{   and   } r \in [-3270, -3265]. \)

\end{enumerate} }
\litem{
Factor the polynomial below completely, knowing that $x + 4$ is a factor. Then, choose the intervals the zeros of the polynomial belong to, where $z_1 \leq z_2 \leq z_3 \leq z_4$. \textit{To make the problem easier, all zeros are between -5 and 5.}\[ f(x) = 12x^{4} +59 x^{3} -1 x^{2} -230 x -200 \]\begin{enumerate}[label=\Alph*.]
\item \( z_1 \in [-4.1, -3.6], \text{   }  z_2 \in [-0.85, -0.64], z_3 \in [-0.88, -0.5], \text{   and   } z_4 \in [0.6, 3] \)
\item \( z_1 \in [-4.1, -3.6], \text{   }  z_2 \in [-1.72, -1.62], z_3 \in [-1.32, -1.21], \text{   and   } z_4 \in [0.6, 3] \)
\item \( z_1 \in [-3, -0.6], \text{   }  z_2 \in [0.49, 0.85], z_3 \in [0.67, 1.33], \text{   and   } z_4 \in [2.3, 4.6] \)
\item \( z_1 \in [-3, -0.6], \text{   }  z_2 \in [1.13, 1.51], z_3 \in [1.35, 1.96], \text{   and   } z_4 \in [2.3, 4.6] \)
\item \( z_1 \in [-3, -0.6], \text{   }  z_2 \in [0.4, 0.55], z_3 \in [3.9, 4.15], \text{   and   } z_4 \in [4.3, 5.8] \)

\end{enumerate} }
\litem{
Factor the polynomial below completely, knowing that $x -2$ is a factor. Then, choose the intervals the zeros of the polynomial belong to, where $z_1 \leq z_2 \leq z_3 \leq z_4$. \textit{To make the problem easier, all zeros are between -5 and 5.}\[ f(x) = 4x^{4} -24 x^{3} +29 x^{2} +51 x -90 \]\begin{enumerate}[label=\Alph*.]
\item \( z_1 \in [-5.77, -4.75], \text{   }  z_2 \in [-3.17, -2.57], z_3 \in [-2.16, -1.25], \text{   and   } z_4 \in [0.7, 0.83] \)
\item \( z_1 \in [-1.04, 0.01], \text{   }  z_2 \in [-0.19, 1.44], z_3 \in [1.27, 2.41], \text{   and   } z_4 \in [2.99, 3.06] \)
\item \( z_1 \in [-3.2, -2.58], \text{   }  z_2 \in [-2.17, -1.53], z_3 \in [-0.87, 0.52], \text{   and   } z_4 \in [0.65, 0.68] \)
\item \( z_1 \in [-1.67, -0.78], \text{   }  z_2 \in [1.55, 2.52], z_3 \in [2.15, 2.77], \text{   and   } z_4 \in [2.99, 3.06] \)
\item \( z_1 \in [-3.2, -2.58], \text{   }  z_2 \in [-2.62, -2.43], z_3 \in [-2.16, -1.25], \text{   and   } z_4 \in [1.46, 1.55] \)

\end{enumerate} }
\litem{
Perform the division below. Then, find the intervals that correspond to the quotient in the form $ax^2+bx+c$ and remainder $r$.\[ \frac{8x^{3} +32 x^{2} -8 x -27}{x + 4} \]\begin{enumerate}[label=\Alph*.]
\item \( a \in [6, 21], \text{   } b \in [64, 68], \text{   } c \in [245, 251], \text{   and   } r \in [964, 969]. \)
\item \( a \in [6, 21], \text{   } b \in [-9, -4], \text{   } c \in [28, 36], \text{   and   } r \in [-190, -185]. \)
\item \( a \in [6, 21], \text{   } b \in [-3, 7], \text{   } c \in [-10, -1], \text{   and   } r \in [3, 8]. \)
\item \( a \in [-35, -28], \text{   } b \in [159, 163], \text{   } c \in [-648, -644], \text{   and   } r \in [2563, 2566]. \)
\item \( a \in [-35, -28], \text{   } b \in [-101, -90], \text{   } c \in [-394, -391], \text{   and   } r \in [-1596, -1594]. \)

\end{enumerate} }
\litem{
Factor the polynomial below completely. Then, choose the intervals the zeros of the polynomial belong to, where $z_1 \leq z_2 \leq z_3$. \textit{To make the problem easier, all zeros are between -5 and 5.}\[ f(x) = 9x^{3} -54 x^{2} +35 x + 50 \]\begin{enumerate}[label=\Alph*.]
\item \( z_1 \in [-1, 0.1], \text{   }  z_2 \in [1.58, 1.68], \text{   and   } z_3 \in [4.86, 5.29] \)
\item \( z_1 \in [-5.1, -4.7], \text{   }  z_2 \in [-1.67, -1.59], \text{   and   } z_3 \in [0.38, 1] \)
\item \( z_1 \in [-5.1, -4.7], \text{   }  z_2 \in [-0.61, -0.57], \text{   and   } z_3 \in [0.88, 1.74] \)
\item \( z_1 \in [-5.1, -4.7], \text{   }  z_2 \in [-0.58, -0.54], \text{   and   } z_3 \in [1.65, 2.08] \)
\item \( z_1 \in [-2, -1.4], \text{   }  z_2 \in [0.57, 0.61], \text{   and   } z_3 \in [4.86, 5.29] \)

\end{enumerate} }
\litem{
What are the \textit{possible Integer} roots of the polynomial below?\[ f(x) = 2x^{3} +3 x^{2} +7 x + 6 \]\begin{enumerate}[label=\Alph*.]
\item \( \text{ All combinations of: }\frac{\pm 1,\pm 2}{\pm 1,\pm 2,\pm 3,\pm 6} \)
\item \( \pm 1,\pm 2 \)
\item \( \pm 1,\pm 2,\pm 3,\pm 6 \)
\item \( \text{ All combinations of: }\frac{\pm 1,\pm 2,\pm 3,\pm 6}{\pm 1,\pm 2} \)
\item \( \text{There is no formula or theorem that tells us all possible Integer roots.} \)

\end{enumerate} }
\litem{
Perform the division below. Then, find the intervals that correspond to the quotient in the form $ax^2+bx+c$ and remainder $r$.\[ \frac{15x^{3} -65 x^{2} + 84}{x -4} \]\begin{enumerate}[label=\Alph*.]
\item \( a \in [13, 16], b \in [-20, -10], c \in [-61, -59], \text{ and } r \in [-100, -95]. \)
\item \( a \in [57, 64], b \in [173, 179], c \in [699, 703], \text{ and } r \in [2883, 2885]. \)
\item \( a \in [13, 16], b \in [-7, 0], c \in [-21, -17], \text{ and } r \in [3, 5]. \)
\item \( a \in [13, 16], b \in [-127, -116], c \in [497, 510], \text{ and } r \in [-1921, -1914]. \)
\item \( a \in [57, 64], b \in [-308, -301], c \in [1213, 1228], \text{ and } r \in [-4796, -4793]. \)

\end{enumerate} }
\litem{
What are the \textit{possible Rational} roots of the polynomial below?\[ f(x) = 5x^{3} +3 x^{2} +2 x + 6 \]\begin{enumerate}[label=\Alph*.]
\item \( \text{ All combinations of: }\frac{\pm 1,\pm 2,\pm 3,\pm 6}{\pm 1,\pm 5} \)
\item \( \pm 1,\pm 5 \)
\item \( \text{ All combinations of: }\frac{\pm 1,\pm 5}{\pm 1,\pm 2,\pm 3,\pm 6} \)
\item \( \pm 1,\pm 2,\pm 3,\pm 6 \)
\item \( \text{ There is no formula or theorem that tells us all possible Rational roots.} \)

\end{enumerate} }
\litem{
Factor the polynomial below completely. Then, choose the intervals the zeros of the polynomial belong to, where $z_1 \leq z_2 \leq z_3$. \textit{To make the problem easier, all zeros are between -5 and 5.}\[ f(x) = 9x^{3} -21 x^{2} -14 x + 40 \]\begin{enumerate}[label=\Alph*.]
\item \( z_1 \in [-1.38, -0.8], \text{   }  z_2 \in [1.53, 1.81], \text{   and   } z_3 \in [1.79, 2.07] \)
\item \( z_1 \in [-5.06, -4.9], \text{   }  z_2 \in [-2.45, -1.94], \text{   and   } z_3 \in [0.12, 0.65] \)
\item \( z_1 \in [-2.51, -1.83], \text{   }  z_2 \in [-0.61, -0.47], \text{   and   } z_3 \in [0.65, 1] \)
\item \( z_1 \in [-1.15, -0.16], \text{   }  z_2 \in [0.22, 0.94], \text{   and   } z_3 \in [1.79, 2.07] \)
\item \( z_1 \in [-2.51, -1.83], \text{   }  z_2 \in [-1.7, -1.2], \text{   and   } z_3 \in [1.02, 1.56] \)

\end{enumerate} }
\litem{
Perform the division below. Then, find the intervals that correspond to the quotient in the form $ax^2+bx+c$ and remainder $r$.\[ \frac{8x^{3} +28 x^{2} -39}{x + 3} \]\begin{enumerate}[label=\Alph*.]
\item \( a \in [3, 15], b \in [2, 7], c \in [-16, -7], \text{ and } r \in [-7, 5]. \)
\item \( a \in [-28, -22], b \in [98, 105], c \in [-302, -297], \text{ and } r \in [859, 863]. \)
\item \( a \in [3, 15], b \in [-7, 1], c \in [13, 19], \text{ and } r \in [-105, -99]. \)
\item \( a \in [-28, -22], b \in [-49, -43], c \in [-136, -131], \text{ and } r \in [-440, -428]. \)
\item \( a \in [3, 15], b \in [51, 53], c \in [153, 159], \text{ and } r \in [428, 432]. \)

\end{enumerate} }
\litem{
Perform the division below. Then, find the intervals that correspond to the quotient in the form $ax^2+bx+c$ and remainder $r$.\[ \frac{20x^{3} +72 x^{2} +28 x -20}{x + 3} \]\begin{enumerate}[label=\Alph*.]
\item \( a \in [-68, -56], \text{   } b \in [-109, -105], \text{   } c \in [-301, -291], \text{   and   } r \in [-908, -906]. \)
\item \( a \in [16, 24], \text{   } b \in [-10, -7], \text{   } c \in [58, 62], \text{   and   } r \in [-260, -257]. \)
\item \( a \in [-68, -56], \text{   } b \in [251, 256], \text{   } c \in [-734, -727], \text{   and   } r \in [2164, 2169]. \)
\item \( a \in [16, 24], \text{   } b \in [132, 136], \text{   } c \in [423, 428], \text{   and   } r \in [1250, 1260]. \)
\item \( a \in [16, 24], \text{   } b \in [7, 19], \text{   } c \in [-8, -3], \text{   and   } r \in [-1, 8]. \)

\end{enumerate} }
\litem{
Factor the polynomial below completely, knowing that $x + 5$ is a factor. Then, choose the intervals the zeros of the polynomial belong to, where $z_1 \leq z_2 \leq z_3 \leq z_4$. \textit{To make the problem easier, all zeros are between -5 and 5.}\[ f(x) = 15x^{4} +11 x^{3} -257 x^{2} +297 x -90 \]\begin{enumerate}[label=\Alph*.]
\item \( z_1 \in [-3.8, -1.7], \text{   }  z_2 \in [-1.96, -1.25], z_3 \in [-1.68, -1.5], \text{   and   } z_4 \in [4.1, 6.1] \)
\item \( z_1 \in [-3.8, -1.7], \text{   }  z_2 \in [-0.82, 0.09], z_3 \in [-0.64, -0.29], \text{   and   } z_4 \in [4.1, 6.1] \)
\item \( z_1 \in [-5.1, -3.6], \text{   }  z_2 \in [1.23, 1.89], z_3 \in [1.46, 1.93], \text{   and   } z_4 \in [1.7, 3.8] \)
\item \( z_1 \in [-3.8, -1.7], \text{   }  z_2 \in [-2.05, -1.98], z_3 \in [-0.45, 0.33], \text{   and   } z_4 \in [4.1, 6.1] \)
\item \( z_1 \in [-5.1, -3.6], \text{   }  z_2 \in [-0.33, 1.48], z_3 \in [0.44, 0.93], \text{   and   } z_4 \in [1.7, 3.8] \)

\end{enumerate} }
\litem{
Factor the polynomial below completely, knowing that $x + 2$ is a factor. Then, choose the intervals the zeros of the polynomial belong to, where $z_1 \leq z_2 \leq z_3 \leq z_4$. \textit{To make the problem easier, all zeros are between -5 and 5.}\[ f(x) = 20x^{4} +103 x^{3} +126 x^{2} -27 x -54 \]\begin{enumerate}[label=\Alph*.]
\item \( z_1 \in [-0.33, 0.04], \text{   }  z_2 \in [1.62, 2.17], z_3 \in [2.32, 3.41], \text{   and   } z_4 \in [2.81, 3.25] \)
\item \( z_1 \in [-3.06, -2.84], \text{   }  z_2 \in [-2.02, -1.92], z_3 \in [-1.34, -0.98], \text{   and   } z_4 \in [1.44, 1.82] \)
\item \( z_1 \in [-3.06, -2.84], \text{   }  z_2 \in [-2.02, -1.92], z_3 \in [-1.15, -0.65], \text{   and   } z_4 \in [-0.07, 0.95] \)
\item \( z_1 \in [-1.75, -1.29], \text{   }  z_2 \in [1.23, 1.53], z_3 \in [1.95, 2.57], \text{   and   } z_4 \in [2.81, 3.25] \)
\item \( z_1 \in [-0.71, -0.58], \text{   }  z_2 \in [0.64, 1.05], z_3 \in [1.95, 2.57], \text{   and   } z_4 \in [2.81, 3.25] \)

\end{enumerate} }
\litem{
Perform the division below. Then, find the intervals that correspond to the quotient in the form $ax^2+bx+c$ and remainder $r$.\[ \frac{15x^{3} +101 x^{2} +138 x + 45}{x + 5} \]\begin{enumerate}[label=\Alph*.]
\item \( a \in [-75, -72], \text{   } b \in [475, 477], \text{   } c \in [-2245, -2240], \text{   and   } r \in [11253, 11258]. \)
\item \( a \in [14, 16], \text{   } b \in [10, 15], \text{   } c \in [69, 78], \text{   and   } r \in [-389, -378]. \)
\item \( a \in [14, 16], \text{   } b \in [174, 179], \text{   } c \in [1017, 1022], \text{   and   } r \in [5131, 5139]. \)
\item \( a \in [14, 16], \text{   } b \in [25, 33], \text{   } c \in [4, 9], \text{   and   } r \in [1, 8]. \)
\item \( a \in [-75, -72], \text{   } b \in [-279, -271], \text{   } c \in [-1234, -1227], \text{   and   } r \in [-6115, -6111]. \)

\end{enumerate} }
\litem{
Factor the polynomial below completely. Then, choose the intervals the zeros of the polynomial belong to, where $z_1 \leq z_2 \leq z_3$. \textit{To make the problem easier, all zeros are between -5 and 5.}\[ f(x) = 8x^{3} -26 x^{2} -5 x + 50 \]\begin{enumerate}[label=\Alph*.]
\item \( z_1 \in [-0.81, -0.3], \text{   }  z_2 \in [0.3, 1.8], \text{   and   } z_3 \in [1.68, 2.12] \)
\item \( z_1 \in [-5.59, -4.51], \text{   }  z_2 \in [-2.3, -1.5], \text{   and   } z_3 \in [0.41, 0.68] \)
\item \( z_1 \in [-1.3, -1.01], \text{   }  z_2 \in [1.6, 2.8], \text{   and   } z_3 \in [2.49, 2.51] \)
\item \( z_1 \in [-2.4, -1.96], \text{   }  z_2 \in [-0.8, -0.2], \text{   and   } z_3 \in [0.76, 1.14] \)
\item \( z_1 \in [-2.51, -2.08], \text{   }  z_2 \in [-2.3, -1.5], \text{   and   } z_3 \in [1.22, 1.38] \)

\end{enumerate} }
\litem{
What are the \textit{possible Rational} roots of the polynomial below?\[ f(x) = 3x^{3} +6 x^{2} +2 x + 5 \]\begin{enumerate}[label=\Alph*.]
\item \( \pm 1,\pm 3 \)
\item \( \text{ All combinations of: }\frac{\pm 1,\pm 3}{\pm 1,\pm 5} \)
\item \( \pm 1,\pm 5 \)
\item \( \text{ All combinations of: }\frac{\pm 1,\pm 5}{\pm 1,\pm 3} \)
\item \( \text{ There is no formula or theorem that tells us all possible Rational roots.} \)

\end{enumerate} }
\litem{
Perform the division below. Then, find the intervals that correspond to the quotient in the form $ax^2+bx+c$ and remainder $r$.\[ \frac{4x^{3} -14 x^{2} + 21}{x -3} \]\begin{enumerate}[label=\Alph*.]
\item \( a \in [11, 17], b \in [17, 23], c \in [57, 72], \text{ and } r \in [217, 221]. \)
\item \( a \in [3, 7], b \in [-6, -4], c \in [-16, -7], \text{ and } r \in [-8, -1]. \)
\item \( a \in [3, 7], b \in [-3, 1], c \in [-7, -3], \text{ and } r \in [-2, 4]. \)
\item \( a \in [11, 17], b \in [-55, -47], c \in [150, 152], \text{ and } r \in [-436, -428]. \)
\item \( a \in [3, 7], b \in [-28, -21], c \in [74, 79], \text{ and } r \in [-215, -210]. \)

\end{enumerate} }
\litem{
What are the \textit{possible Integer} roots of the polynomial below?\[ f(x) = 4x^{2} +7 x + 7 \]\begin{enumerate}[label=\Alph*.]
\item \( \pm 1,\pm 2,\pm 4 \)
\item \( \text{ All combinations of: }\frac{\pm 1,\pm 7}{\pm 1,\pm 2,\pm 4} \)
\item \( \pm 1,\pm 7 \)
\item \( \text{ All combinations of: }\frac{\pm 1,\pm 2,\pm 4}{\pm 1,\pm 7} \)
\item \( \text{There is no formula or theorem that tells us all possible Integer roots.} \)

\end{enumerate} }
\litem{
Factor the polynomial below completely. Then, choose the intervals the zeros of the polynomial belong to, where $z_1 \leq z_2 \leq z_3$. \textit{To make the problem easier, all zeros are between -5 and 5.}\[ f(x) = 15x^{3} +56 x^{2} +60 x + 16 \]\begin{enumerate}[label=\Alph*.]
\item \( z_1 \in [-2.37, -1.82], \text{   }  z_2 \in [-1.6, -0.7], \text{   and   } z_3 \in [-0.66, 0.01] \)
\item \( z_1 \in [-3.12, -2.21], \text{   }  z_2 \in [-2.2, -1.8], \text{   and   } z_3 \in [-1.16, -0.73] \)
\item \( z_1 \in [0.04, 0.25], \text{   }  z_2 \in [1.5, 2.3], \text{   and   } z_3 \in [3.59, 4.28] \)
\item \( z_1 \in [0.28, 0.66], \text{   }  z_2 \in [0, 1.8], \text{   and   } z_3 \in [1.8, 2.27] \)
\item \( z_1 \in [0.58, 0.8], \text{   }  z_2 \in [1.5, 2.3], \text{   and   } z_3 \in [2.39, 2.55] \)

\end{enumerate} }
\litem{
Perform the division below. Then, find the intervals that correspond to the quotient in the form $ax^2+bx+c$ and remainder $r$.\[ \frac{4x^{3} -28 x + 29}{x + 3} \]\begin{enumerate}[label=\Alph*.]
\item \( a \in [-13, -6], b \in [33.5, 37.2], c \in [-136, -135], \text{ and } r \in [434, 440]. \)
\item \( a \in [-3, 7], b \in [-16.3, -15.4], c \in [34, 40], \text{ and } r \in [-115, -110]. \)
\item \( a \in [-3, 7], b \in [11.9, 14.4], c \in [2, 15], \text{ and } r \in [50, 60]. \)
\item \( a \in [-13, -6], b \in [-38.6, -35.7], c \in [-136, -135], \text{ and } r \in [-383, -375]. \)
\item \( a \in [-3, 7], b \in [-12.1, -10.9], c \in [2, 15], \text{ and } r \in [3, 8]. \)

\end{enumerate} }
\litem{
Perform the division below. Then, find the intervals that correspond to the quotient in the form $ax^2+bx+c$ and remainder $r$.\[ \frac{20x^{3} -75 x^{2} +85 x -28}{x -2} \]\begin{enumerate}[label=\Alph*.]
\item \( a \in [19, 21], \text{   } b \in [-115, -113], \text{   } c \in [309, 317], \text{   and   } r \in [-661, -657]. \)
\item \( a \in [19, 21], \text{   } b \in [-55, -50], \text{   } c \in [27, 38], \text{   and   } r \in [0, 5]. \)
\item \( a \in [38, 44], \text{   } b \in [-161, -150], \text{   } c \in [395, 404], \text{   and   } r \in [-819, -814]. \)
\item \( a \in [38, 44], \text{   } b \in [5, 8], \text{   } c \in [93, 96], \text{   and   } r \in [161, 167]. \)
\item \( a \in [19, 21], \text{   } b \in [-43, -32], \text{   } c \in [14, 16], \text{   and   } r \in [0, 5]. \)

\end{enumerate} }
\litem{
Factor the polynomial below completely, knowing that $x -4$ is a factor. Then, choose the intervals the zeros of the polynomial belong to, where $z_1 \leq z_2 \leq z_3 \leq z_4$. \textit{To make the problem easier, all zeros are between -5 and 5.}\[ f(x) = 9x^{4} -54 x^{3} +47 x^{2} +150 x -200 \]\begin{enumerate}[label=\Alph*.]
\item \( z_1 \in [-1.2, 0.1], \text{   }  z_2 \in [-0.13, 1.43], z_3 \in [1.96, 2.19], \text{   and   } z_4 \in [3.9, 4.01] \)
\item \( z_1 \in [-4.2, -2.8], \text{   }  z_2 \in [-2.03, -1.86], z_3 \in [-0.6, -0.42], \text{   and   } z_4 \in [0.57, 0.63] \)
\item \( z_1 \in [-5.8, -4.5], \text{   }  z_2 \in [-4.7, -3.58], z_3 \in [-2.01, -1.7], \text{   and   } z_4 \in [0.47, 0.59] \)
\item \( z_1 \in [-2.4, -1.3], \text{   }  z_2 \in [0.99, 1.68], z_3 \in [1.96, 2.19], \text{   and   } z_4 \in [3.9, 4.01] \)
\item \( z_1 \in [-4.2, -2.8], \text{   }  z_2 \in [-2.03, -1.86], z_3 \in [-1.89, -1.42], \text{   and   } z_4 \in [1.56, 1.72] \)

\end{enumerate} }
\litem{
Factor the polynomial below completely, knowing that $x -2$ is a factor. Then, choose the intervals the zeros of the polynomial belong to, where $z_1 \leq z_2 \leq z_3 \leq z_4$. \textit{To make the problem easier, all zeros are between -5 and 5.}\[ f(x) = 6x^{4} -59 x^{3} +206 x^{2} -304 x + 160 \]\begin{enumerate}[label=\Alph*.]
\item \( z_1 \in [-4.28, -3.72], \text{   }  z_2 \in [-2.11, -1.43], z_3 \in [-0.78, -0.37], \text{   and   } z_4 \in [-0.51, 0.3] \)
\item \( z_1 \in [-4.28, -3.72], \text{   }  z_2 \in [-3.23, -2.11], z_3 \in [-2.42, -1.97], \text{   and   } z_4 \in [-2.07, -1.17] \)
\item \( z_1 \in [1.03, 1.82], \text{   }  z_2 \in [0.92, 2.24], z_3 \in [2.38, 2.55], \text{   and   } z_4 \in [3.29, 4.43] \)
\item \( z_1 \in [-0.57, 0.96], \text{   }  z_2 \in [-0.13, 1.04], z_3 \in [1.89, 2.08], \text{   and   } z_4 \in [3.29, 4.43] \)
\item \( z_1 \in [-4.28, -3.72], \text{   }  z_2 \in [-4.06, -3.6], z_3 \in [-2.42, -1.97], \text{   and   } z_4 \in [-0.89, -0.61] \)

\end{enumerate} }
\litem{
Perform the division below. Then, find the intervals that correspond to the quotient in the form $ax^2+bx+c$ and remainder $r$.\[ \frac{6x^{3} -2 x^{2} -44 x + 45}{x + 3} \]\begin{enumerate}[label=\Alph*.]
\item \( a \in [5, 9], \text{   } b \in [-23.4, -18.2], \text{   } c \in [15, 18], \text{   and   } r \in [-5, 1]. \)
\item \( a \in [-19, -17], \text{   } b \in [49.5, 54.2], \text{   } c \in [-209, -197], \text{   and   } r \in [644, 649]. \)
\item \( a \in [-19, -17], \text{   } b \in [-58.6, -53.9], \text{   } c \in [-217, -210], \text{   and   } r \in [-594, -589]. \)
\item \( a \in [5, 9], \text{   } b \in [15.1, 16.8], \text{   } c \in [-1, 11], \text{   and   } r \in [53, 60]. \)
\item \( a \in [5, 9], \text{   } b \in [-27, -25.5], \text{   } c \in [59, 63], \text{   and   } r \in [-195, -187]. \)

\end{enumerate} }
\litem{
Factor the polynomial below completely. Then, choose the intervals the zeros of the polynomial belong to, where $z_1 \leq z_2 \leq z_3$. \textit{To make the problem easier, all zeros are between -5 and 5.}\[ f(x) = 15x^{3} -89 x^{2} +62 x + 40 \]\begin{enumerate}[label=\Alph*.]
\item \( z_1 \in [-3.5, -1.5], \text{   }  z_2 \in [0.64, 0.8], \text{   and   } z_3 \in [4.8, 5.4] \)
\item \( z_1 \in [-5, -4], \text{   }  z_2 \in [-0.87, -0.74], \text{   and   } z_3 \in [2.1, 2.9] \)
\item \( z_1 \in [-5, -4], \text{   }  z_2 \in [-2.05, -1.27], \text{   and   } z_3 \in [0.3, 0.5] \)
\item \( z_1 \in [-5, -4], \text{   }  z_2 \in [-0.29, -0.11], \text{   and   } z_3 \in [1, 2.3] \)
\item \( z_1 \in [-1.4, 1.6], \text{   }  z_2 \in [0.97, 1.36], \text{   and   } z_3 \in [4.8, 5.4] \)

\end{enumerate} }
\litem{
What are the \textit{possible Rational} roots of the polynomial below?\[ f(x) = 5x^{4} +3 x^{3} +6 x^{2} +2 x + 6 \]\begin{enumerate}[label=\Alph*.]
\item \( \pm 1,\pm 5 \)
\item \( \text{ All combinations of: }\frac{\pm 1,\pm 5}{\pm 1,\pm 2,\pm 3,\pm 6} \)
\item \( \pm 1,\pm 2,\pm 3,\pm 6 \)
\item \( \text{ All combinations of: }\frac{\pm 1,\pm 2,\pm 3,\pm 6}{\pm 1,\pm 5} \)
\item \( \text{ There is no formula or theorem that tells us all possible Rational roots.} \)

\end{enumerate} }
\litem{
Perform the division below. Then, find the intervals that correspond to the quotient in the form $ax^2+bx+c$ and remainder $r$.\[ \frac{10x^{3} -30 x + 18}{x + 2} \]\begin{enumerate}[label=\Alph*.]
\item \( a \in [-20, -19], b \in [39, 42], c \in [-116, -107], \text{ and } r \in [237, 239]. \)
\item \( a \in [6, 14], b \in [-34, -29], c \in [58, 61], \text{ and } r \in [-164, -161]. \)
\item \( a \in [6, 14], b \in [-23, -13], c \in [8, 15], \text{ and } r \in [-7, 1]. \)
\item \( a \in [-20, -19], b \in [-42, -39], c \in [-116, -107], \text{ and } r \in [-202, -197]. \)
\item \( a \in [6, 14], b \in [14, 23], c \in [8, 15], \text{ and } r \in [36, 42]. \)

\end{enumerate} }
\litem{
What are the \textit{possible Integer} roots of the polynomial below?\[ f(x) = 5x^{3} +4 x^{2} +3 x + 4 \]\begin{enumerate}[label=\Alph*.]
\item \( \text{ All combinations of: }\frac{\pm 1,\pm 2,\pm 4}{\pm 1,\pm 5} \)
\item \( \text{ All combinations of: }\frac{\pm 1,\pm 5}{\pm 1,\pm 2,\pm 4} \)
\item \( \pm 1,\pm 5 \)
\item \( \pm 1,\pm 2,\pm 4 \)
\item \( \text{There is no formula or theorem that tells us all possible Integer roots.} \)

\end{enumerate} }
\end{enumerate}

\end{document}