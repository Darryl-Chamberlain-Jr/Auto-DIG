\documentclass{extbook}[14pt]
\usepackage{multicol, enumerate, enumitem, hyperref, color, soul, setspace, parskip, fancyhdr, amssymb, amsthm, amsmath, latexsym, units, mathtools}
\everymath{\displaystyle}
\usepackage[headsep=0.5cm,headheight=0cm, left=1 in,right= 1 in,top= 1 in,bottom= 1 in]{geometry}
\usepackage{dashrule}  % Package to use the command below to create lines between items
\newcommand{\litem}[1]{\item #1

\rule{\textwidth}{0.4pt}}
\pagestyle{fancy}
\lhead{}
\chead{Answer Key for Progress Quiz 8 Version B}
\rhead{}
\lfoot{5493-4176}
\cfoot{}
\rfoot{Summer C 2021}
\begin{document}
\textbf{This key should allow you to understand why you choose the option you did (beyond just getting a question right or wrong). \href{https://xronos.clas.ufl.edu/mac1105spring2020/courseDescriptionAndMisc/Exams/LearningFromResults}{More instructions on how to use this key can be found here}.}

\textbf{If you have a suggestion to make the keys better, \href{https://forms.gle/CZkbZmPbC9XALEE88}{please fill out the short survey here}.}

\textit{Note: This key is auto-generated and may contain issues and/or errors. The keys are reviewed after each exam to ensure grading is done accurately. If there are issues (like duplicate options), they are noted in the offline gradebook. The keys are a work-in-progress to give students as many resources to improve as possible.}

\rule{\textwidth}{0.4pt}

\begin{enumerate}\litem{
Solve the linear inequality below. Then, choose the constant and interval combination that describes the solution set.
\[ 4x + 5 > 6x -7 \]The solution is \( (-\infty, 6.0) \), which is option B.\begin{enumerate}[label=\Alph*.]
\item \( (-\infty, a), \text{ where } a \in [-8, -1] \)

 $(-\infty, -6.0)$, which corresponds to negating the endpoint of the solution.
\item \( (-\infty, a), \text{ where } a \in [2, 7] \)

* $(-\infty, 6.0)$, which is the correct option.
\item \( (a, \infty), \text{ where } a \in [2, 7] \)

 $(6.0, \infty)$, which corresponds to switching the direction of the interval. You likely did this if you did not flip the inequality when dividing by a negative!
\item \( (a, \infty), \text{ where } a \in [-10, -4] \)

 $(-6.0, \infty)$, which corresponds to switching the direction of the interval AND negating the endpoint. You likely did this if you did not flip the inequality when dividing by a negative as well as not moving values over to a side properly.
\item \( \text{None of the above}. \)

You may have chosen this if you thought the inequality did not match the ends of the intervals.
\end{enumerate}

\textbf{General Comment:} Remember that less/greater than or equal to includes the endpoint, while less/greater do not. Also, remember that you need to flip the inequality when you multiply or divide by a negative.
}
\litem{
Solve the linear inequality below. Then, choose the constant and interval combination that describes the solution set.
\[ -7 - 3 x \leq \frac{-22 x - 8}{9} < 6 - 5 x \]The solution is \( \text{None of the above.} \), which is option E.\begin{enumerate}[label=\Alph*.]
\item \( [a, b), \text{ where } a \in [6, 15] \text{ and } b \in [-6, -1.5] \)

$[11.00, -2.70)$, which is the correct interval but negatives of the actual endpoints.
\item \( (a, b], \text{ where } a \in [4.5, 13.5] \text{ and } b \in [-7.5, -2.25] \)

$(11.00, -2.70]$, which corresponds to flipping the inequality and getting negatives of the actual endpoints.
\item \( (-\infty, a) \cup [b, \infty), \text{ where } a \in [8.25, 13.5] \text{ and } b \in [-3.75, -1.5] \)

$(-\infty, 11.00) \cup [-2.70, \infty)$, which corresponds to displaying the and-inequality as an or-inequality AND flipping the inequality AND getting negatives of the actual endpoints.
\item \( (-\infty, a] \cup (b, \infty), \text{ where } a \in [8.25, 13.5] \text{ and } b \in [-5.25, -2.25] \)

$(-\infty, 11.00] \cup (-2.70, \infty)$, which corresponds to displaying the and-inequality as an or-inequality and getting negatives of the actual endpoints.
\item \( \text{None of the above.} \)

* This is correct as the answer should be $[-11.00, 2.70)$.
\end{enumerate}

\textbf{General Comment:} To solve, you will need to break up the compound inequality into two inequalities. Be sure to keep track of the inequality! It may be best to draw a number line and graph your solution.
}
\litem{
Using an interval or intervals, describe all the $x$-values within or including a distance of the given values.
\[ \text{ No less than } 9 \text{ units from the number } 7. \]The solution is \( (-\infty, -2] \cup [16, \infty) \), which is option A.\begin{enumerate}[label=\Alph*.]
\item \( (-\infty, -2] \cup [16, \infty) \)

This describes the values no less than 9 from 7
\item \( [-2, 16] \)

This describes the values no more than 9 from 7
\item \( (-\infty, -2) \cup (16, \infty) \)

This describes the values more than 9 from 7
\item \( (-2, 16) \)

This describes the values less than 9 from 7
\item \( \text{None of the above} \)

You likely thought the values in the interval were not correct.
\end{enumerate}

\textbf{General Comment:} When thinking about this language, it helps to draw a number line and try points.
}
\litem{
Solve the linear inequality below. Then, choose the constant and interval combination that describes the solution set.
\[ -7 - 3 x \leq \frac{-21 x + 9}{8} < -9 - 7 x \]The solution is \( [-21.67, -2.31) \), which is option A.\begin{enumerate}[label=\Alph*.]
\item \( [a, b), \text{ where } a \in [-28.5, -16.5] \text{ and } b \in [-9, -0.75] \)

$[-21.67, -2.31)$, which is the correct option.
\item \( (-\infty, a) \cup [b, \infty), \text{ where } a \in [-23.25, -18.75] \text{ and } b \in [-7.5, -1.5] \)

$(-\infty, -21.67) \cup [-2.31, \infty)$, which corresponds to displaying the and-inequality as an or-inequality AND flipping the inequality.
\item \( (a, b], \text{ where } a \in [-23.25, -18] \text{ and } b \in [-8.25, 1.5] \)

$(-21.67, -2.31]$, which corresponds to flipping the inequality.
\item \( (-\infty, a] \cup (b, \infty), \text{ where } a \in [-24, -18] \text{ and } b \in [-3.75, 0] \)

$(-\infty, -21.67] \cup (-2.31, \infty)$, which corresponds to displaying the and-inequality as an or-inequality.
\item \( \text{None of the above.} \)


\end{enumerate}

\textbf{General Comment:} To solve, you will need to break up the compound inequality into two inequalities. Be sure to keep track of the inequality! It may be best to draw a number line and graph your solution.
}
\litem{
Using an interval or intervals, describe all the $x$-values within or including a distance of the given values.
\[ \text{ No more than } 3 \text{ units from the number } 2. \]The solution is \( \text{None of the above} \), which is option E.\begin{enumerate}[label=\Alph*.]
\item \( (-\infty, 1] \cup [5, \infty) \)

This describes the values no less than 2 from 3
\item \( [1, 5] \)

This describes the values no more than 2 from 3
\item \( (-\infty, 1) \cup (5, \infty) \)

This describes the values more than 2 from 3
\item \( (1, 5) \)

This describes the values less than 2 from 3
\item \( \text{None of the above} \)

Options A-D described the values [more/less than] 2 units from 3, which is the reverse of what the question asked.
\end{enumerate}

\textbf{General Comment:} When thinking about this language, it helps to draw a number line and try points.
}
\litem{
Solve the linear inequality below. Then, choose the constant and interval combination that describes the solution set.
\[ \frac{10}{7} - \frac{7}{9} x \geq \frac{4}{4} x - \frac{10}{3} \]The solution is \( (-\infty, 2.679] \), which is option C.\begin{enumerate}[label=\Alph*.]
\item \( [a, \infty), \text{ where } a \in [0, 4.5] \)

 $[2.679, \infty)$, which corresponds to switching the direction of the interval. You likely did this if you did not flip the inequality when dividing by a negative!
\item \( [a, \infty), \text{ where } a \in [-3.75, 0] \)

 $[-2.679, \infty)$, which corresponds to switching the direction of the interval AND negating the endpoint. You likely did this if you did not flip the inequality when dividing by a negative as well as not moving values over to a side properly.
\item \( (-\infty, a], \text{ where } a \in [0.75, 7.5] \)

* $(-\infty, 2.679]$, which is the correct option.
\item \( (-\infty, a], \text{ where } a \in [-5.25, 0] \)

 $(-\infty, -2.679]$, which corresponds to negating the endpoint of the solution.
\item \( \text{None of the above}. \)

You may have chosen this if you thought the inequality did not match the ends of the intervals.
\end{enumerate}

\textbf{General Comment:} Remember that less/greater than or equal to includes the endpoint, while less/greater do not. Also, remember that you need to flip the inequality when you multiply or divide by a negative.
}
\litem{
Solve the linear inequality below. Then, choose the constant and interval combination that describes the solution set.
\[ -9 + 5 x > 8 x \text{ or } 5 + 3 x < 4 x \]The solution is \( (-\infty, -3.0) \text{ or } (5.0, \infty) \), which is option C.\begin{enumerate}[label=\Alph*.]
\item \( (-\infty, a] \cup [b, \infty), \text{ where } a \in [-5.25, -3.15] \text{ and } b \in [1.05, 3.3] \)

Corresponds to including the endpoints AND negating.
\item \( (-\infty, a] \cup [b, \infty), \text{ where } a \in [-4.65, -2.4] \text{ and } b \in [4.58, 6.15] \)

Corresponds to including the endpoints (when they should be excluded).
\item \( (-\infty, a) \cup (b, \infty), \text{ where } a \in [-4.88, -2.62] \text{ and } b \in [3.52, 5.62] \)

 * Correct option.
\item \( (-\infty, a) \cup (b, \infty), \text{ where } a \in [-6.52, -3.82] \text{ and } b \in [2.62, 4.65] \)

Corresponds to inverting the inequality and negating the solution.
\item \( (-\infty, \infty) \)

Corresponds to the variable canceling, which does not happen in this instance.
\end{enumerate}

\textbf{General Comment:} When multiplying or dividing by a negative, flip the sign.
}
\litem{
Solve the linear inequality below. Then, choose the constant and interval combination that describes the solution set.
\[ \frac{-6}{7} + \frac{5}{8} x \leq \frac{10}{3} x + \frac{4}{2} \]The solution is \( [-1.055, \infty) \), which is option D.\begin{enumerate}[label=\Alph*.]
\item \( (-\infty, a], \text{ where } a \in [0, 1.5] \)

 $(-\infty, 1.055]$, which corresponds to switching the direction of the interval AND negating the endpoint. You likely did this if you did not flip the inequality when dividing by a negative as well as not moving values over to a side properly.
\item \( [a, \infty), \text{ where } a \in [0, 3] \)

 $[1.055, \infty)$, which corresponds to negating the endpoint of the solution.
\item \( (-\infty, a], \text{ where } a \in [-3, -0.75] \)

 $(-\infty, -1.055]$, which corresponds to switching the direction of the interval. You likely did this if you did not flip the inequality when dividing by a negative!
\item \( [a, \infty), \text{ where } a \in [-5.25, 0.75] \)

* $[-1.055, \infty)$, which is the correct option.
\item \( \text{None of the above}. \)

You may have chosen this if you thought the inequality did not match the ends of the intervals.
\end{enumerate}

\textbf{General Comment:} Remember that less/greater than or equal to includes the endpoint, while less/greater do not. Also, remember that you need to flip the inequality when you multiply or divide by a negative.
}
\litem{
Solve the linear inequality below. Then, choose the constant and interval combination that describes the solution set.
\[ -7x -3 \geq -4x -6 \]The solution is \( (-\infty, 1.0] \), which is option C.\begin{enumerate}[label=\Alph*.]
\item \( (-\infty, a], \text{ where } a \in [-2.2, 0.8] \)

 $(-\infty, -1.0]$, which corresponds to negating the endpoint of the solution.
\item \( [a, \infty), \text{ where } a \in [0.66, 1.54] \)

 $[1.0, \infty)$, which corresponds to switching the direction of the interval. You likely did this if you did not flip the inequality when dividing by a negative!
\item \( (-\infty, a], \text{ where } a \in [-0.2, 3] \)

* $(-\infty, 1.0]$, which is the correct option.
\item \( [a, \infty), \text{ where } a \in [-1.71, -0.18] \)

 $[-1.0, \infty)$, which corresponds to switching the direction of the interval AND negating the endpoint. You likely did this if you did not flip the inequality when dividing by a negative as well as not moving values over to a side properly.
\item \( \text{None of the above}. \)

You may have chosen this if you thought the inequality did not match the ends of the intervals.
\end{enumerate}

\textbf{General Comment:} Remember that less/greater than or equal to includes the endpoint, while less/greater do not. Also, remember that you need to flip the inequality when you multiply or divide by a negative.
}
\litem{
Solve the linear inequality below. Then, choose the constant and interval combination that describes the solution set.
\[ 7 + 5 x > 8 x \text{ or } 7 + 3 x < 4 x \]The solution is \( (-\infty, 2.333) \text{ or } (7.0, \infty) \), which is option D.\begin{enumerate}[label=\Alph*.]
\item \( (-\infty, a) \cup (b, \infty), \text{ where } a \in [-7.5, -6] \text{ and } b \in [-3, -1.5] \)

Corresponds to inverting the inequality and negating the solution.
\item \( (-\infty, a] \cup [b, \infty), \text{ where } a \in [0.75, 4.5] \text{ and } b \in [6, 12] \)

Corresponds to including the endpoints (when they should be excluded).
\item \( (-\infty, a] \cup [b, \infty), \text{ where } a \in [-8.25, -4.5] \text{ and } b \in [-5.25, 0.75] \)

Corresponds to including the endpoints AND negating.
\item \( (-\infty, a) \cup (b, \infty), \text{ where } a \in [0, 5.25] \text{ and } b \in [3, 13.5] \)

 * Correct option.
\item \( (-\infty, \infty) \)

Corresponds to the variable canceling, which does not happen in this instance.
\end{enumerate}

\textbf{General Comment:} When multiplying or dividing by a negative, flip the sign.
}
\end{enumerate}

\end{document}