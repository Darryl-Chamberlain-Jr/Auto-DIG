\documentclass{extbook}[14pt]
\usepackage{multicol, enumerate, enumitem, hyperref, color, soul, setspace, parskip, fancyhdr, amssymb, amsthm, amsmath, latexsym, units, mathtools}
\everymath{\displaystyle}
\usepackage[headsep=0.5cm,headheight=0cm, left=1 in,right= 1 in,top= 1 in,bottom= 1 in]{geometry}
\usepackage{dashrule}  % Package to use the command below to create lines between items
\newcommand{\litem}[1]{\item #1

\rule{\textwidth}{0.4pt}}
\pagestyle{fancy}
\lhead{}
\chead{Answer Key for Progress Quiz 8 Version A}
\rhead{}
\lfoot{5493-4176}
\cfoot{}
\rfoot{Summer C 2021}
\begin{document}
\textbf{This key should allow you to understand why you choose the option you did (beyond just getting a question right or wrong). \href{https://xronos.clas.ufl.edu/mac1105spring2020/courseDescriptionAndMisc/Exams/LearningFromResults}{More instructions on how to use this key can be found here}.}

\textbf{If you have a suggestion to make the keys better, \href{https://forms.gle/CZkbZmPbC9XALEE88}{please fill out the short survey here}.}

\textit{Note: This key is auto-generated and may contain issues and/or errors. The keys are reviewed after each exam to ensure grading is done accurately. If there are issues (like duplicate options), they are noted in the offline gradebook. The keys are a work-in-progress to give students as many resources to improve as possible.}

\rule{\textwidth}{0.4pt}

\begin{enumerate}\litem{
Solve the linear inequality below. Then, choose the constant and interval combination that describes the solution set.
\[ -3x -3 > 9x + 3 \]The solution is \( (-\infty, -0.5) \), which is option A.\begin{enumerate}[label=\Alph*.]
\item \( (-\infty, a), \text{ where } a \in [-2.3, 0.4] \)

* $(-\infty, -0.5)$, which is the correct option.
\item \( (a, \infty), \text{ where } a \in [-0.15, 1.68] \)

 $(0.5, \infty)$, which corresponds to switching the direction of the interval AND negating the endpoint. You likely did this if you did not flip the inequality when dividing by a negative as well as not moving values over to a side properly.
\item \( (-\infty, a), \text{ where } a \in [-0.4, 1.5] \)

 $(-\infty, 0.5)$, which corresponds to negating the endpoint of the solution.
\item \( (a, \infty), \text{ where } a \in [-1.53, -0.15] \)

 $(-0.5, \infty)$, which corresponds to switching the direction of the interval. You likely did this if you did not flip the inequality when dividing by a negative!
\item \( \text{None of the above}. \)

You may have chosen this if you thought the inequality did not match the ends of the intervals.
\end{enumerate}

\textbf{General Comment:} Remember that less/greater than or equal to includes the endpoint, while less/greater do not. Also, remember that you need to flip the inequality when you multiply or divide by a negative.
}
\litem{
Solve the linear inequality below. Then, choose the constant and interval combination that describes the solution set.
\[ -3 - 8 x < \frac{-44 x + 4}{7} \leq 7 - 7 x \]The solution is \( \text{None of the above.} \), which is option E.\begin{enumerate}[label=\Alph*.]
\item \( [a, b), \text{ where } a \in [0.3, 2.48] \text{ and } b \in [-13.5, -6.75] \)

$[2.08, -9.00)$, which corresponds to flipping the inequality and getting negatives of the actual endpoints.
\item \( (-\infty, a] \cup (b, \infty), \text{ where } a \in [0, 4.5] \text{ and } b \in [-12.75, -6.75] \)

$(-\infty, 2.08] \cup (-9.00, \infty)$, which corresponds to displaying the and-inequality as an or-inequality AND flipping the inequality AND getting negatives of the actual endpoints.
\item \( (-\infty, a) \cup [b, \infty), \text{ where } a \in [-0.75, 5.25] \text{ and } b \in [-11.25, -6.75] \)

$(-\infty, 2.08) \cup [-9.00, \infty)$, which corresponds to displaying the and-inequality as an or-inequality and getting negatives of the actual endpoints.
\item \( (a, b], \text{ where } a \in [-0.75, 3] \text{ and } b \in [-11.25, -8.25] \)

$(2.08, -9.00]$, which is the correct interval but negatives of the actual endpoints.
\item \( \text{None of the above.} \)

* This is correct as the answer should be $(-2.08, 9.00]$.
\end{enumerate}

\textbf{General Comment:} To solve, you will need to break up the compound inequality into two inequalities. Be sure to keep track of the inequality! It may be best to draw a number line and graph your solution.
}
\litem{
Using an interval or intervals, describe all the $x$-values within or including a distance of the given values.
\[ \text{ No less than } 6 \text{ units from the number } 10. \]The solution is \( (-\infty, 4] \cup [16, \infty) \), which is option D.\begin{enumerate}[label=\Alph*.]
\item \( [4, 16] \)

This describes the values no more than 6 from 10
\item \( (-\infty, 4) \cup (16, \infty) \)

This describes the values more than 6 from 10
\item \( (4, 16) \)

This describes the values less than 6 from 10
\item \( (-\infty, 4] \cup [16, \infty) \)

This describes the values no less than 6 from 10
\item \( \text{None of the above} \)

You likely thought the values in the interval were not correct.
\end{enumerate}

\textbf{General Comment:} When thinking about this language, it helps to draw a number line and try points.
}
\litem{
Solve the linear inequality below. Then, choose the constant and interval combination that describes the solution set.
\[ -8 - 5 x < \frac{-12 x + 7}{3} \leq 6 - 6 x \]The solution is \( (-10.33, 1.83] \), which is option D.\begin{enumerate}[label=\Alph*.]
\item \( [a, b), \text{ where } a \in [-11.25, -6] \text{ and } b \in [-0.75, 4.5] \)

$[-10.33, 1.83)$, which corresponds to flipping the inequality.
\item \( (-\infty, a) \cup [b, \infty), \text{ where } a \in [-13.5, -6] \text{ and } b \in [1.5, 2.25] \)

$(-\infty, -10.33) \cup [1.83, \infty)$, which corresponds to displaying the and-inequality as an or-inequality.
\item \( (-\infty, a] \cup (b, \infty), \text{ where } a \in [-12, -6.75] \text{ and } b \in [1.2, 2.7] \)

$(-\infty, -10.33] \cup (1.83, \infty)$, which corresponds to displaying the and-inequality as an or-inequality AND flipping the inequality.
\item \( (a, b], \text{ where } a \in [-12, -9.75] \text{ and } b \in [-1.5, 6] \)

* $(-10.33, 1.83]$, which is the correct option.
\item \( \text{None of the above.} \)


\end{enumerate}

\textbf{General Comment:} To solve, you will need to break up the compound inequality into two inequalities. Be sure to keep track of the inequality! It may be best to draw a number line and graph your solution.
}
\litem{
Using an interval or intervals, describe all the $x$-values within or including a distance of the given values.
\[ \text{ No more than } 2 \text{ units from the number } -8. \]The solution is \( [-10, -6] \), which is option D.\begin{enumerate}[label=\Alph*.]
\item \( (-10, -6) \)

This describes the values less than 2 from -8
\item \( (-\infty, -10) \cup (-6, \infty) \)

This describes the values more than 2 from -8
\item \( (-\infty, -10] \cup [-6, \infty) \)

This describes the values no less than 2 from -8
\item \( [-10, -6] \)

This describes the values no more than 2 from -8
\item \( \text{None of the above} \)

You likely thought the values in the interval were not correct.
\end{enumerate}

\textbf{General Comment:} When thinking about this language, it helps to draw a number line and try points.
}
\litem{
Solve the linear inequality below. Then, choose the constant and interval combination that describes the solution set.
\[ \frac{-6}{5} - \frac{4}{4} x < \frac{5}{9} x + \frac{3}{6} \]The solution is \( (-1.093, \infty) \), which is option D.\begin{enumerate}[label=\Alph*.]
\item \( (-\infty, a), \text{ where } a \in [-4.5, 0.75] \)

 $(-\infty, -1.093)$, which corresponds to switching the direction of the interval. You likely did this if you did not flip the inequality when dividing by a negative!
\item \( (a, \infty), \text{ where } a \in [-0.6, 1.57] \)

 $(1.093, \infty)$, which corresponds to negating the endpoint of the solution.
\item \( (-\infty, a), \text{ where } a \in [0, 2.25] \)

 $(-\infty, 1.093)$, which corresponds to switching the direction of the interval AND negating the endpoint. You likely did this if you did not flip the inequality when dividing by a negative as well as not moving values over to a side properly.
\item \( (a, \infty), \text{ where } a \in [-1.2, -0.07] \)

* $(-1.093, \infty)$, which is the correct option.
\item \( \text{None of the above}. \)

You may have chosen this if you thought the inequality did not match the ends of the intervals.
\end{enumerate}

\textbf{General Comment:} Remember that less/greater than or equal to includes the endpoint, while less/greater do not. Also, remember that you need to flip the inequality when you multiply or divide by a negative.
}
\litem{
Solve the linear inequality below. Then, choose the constant and interval combination that describes the solution set.
\[ -9 + 8 x > 11 x \text{ or } 5 + 7 x < 8 x \]The solution is \( (-\infty, -3.0) \text{ or } (5.0, \infty) \), which is option A.\begin{enumerate}[label=\Alph*.]
\item \( (-\infty, a) \cup (b, \infty), \text{ where } a \in [-3.52, -0.6] \text{ and } b \in [3.97, 7.8] \)

 * Correct option.
\item \( (-\infty, a) \cup (b, \infty), \text{ where } a \in [-6.83, -3.3] \text{ and } b \in [2.7, 4.05] \)

Corresponds to inverting the inequality and negating the solution.
\item \( (-\infty, a] \cup [b, \infty), \text{ where } a \in [-4.5, -2.25] \text{ and } b \in [4.12, 7.72] \)

Corresponds to including the endpoints (when they should be excluded).
\item \( (-\infty, a] \cup [b, \infty), \text{ where } a \in [-7.5, -3.75] \text{ and } b \in [2.1, 3.38] \)

Corresponds to including the endpoints AND negating.
\item \( (-\infty, \infty) \)

Corresponds to the variable canceling, which does not happen in this instance.
\end{enumerate}

\textbf{General Comment:} When multiplying or dividing by a negative, flip the sign.
}
\litem{
Solve the linear inequality below. Then, choose the constant and interval combination that describes the solution set.
\[ \frac{10}{4} - \frac{10}{8} x \leq \frac{8}{7} x - \frac{10}{3} \]The solution is \( [2.438, \infty) \), which is option A.\begin{enumerate}[label=\Alph*.]
\item \( [a, \infty), \text{ where } a \in [1.5, 3.75] \)

* $[2.438, \infty)$, which is the correct option.
\item \( [a, \infty), \text{ where } a \in [-3.75, -1.5] \)

 $[-2.438, \infty)$, which corresponds to negating the endpoint of the solution.
\item \( (-\infty, a], \text{ where } a \in [-3, 2.25] \)

 $(-\infty, -2.438]$, which corresponds to switching the direction of the interval AND negating the endpoint. You likely did this if you did not flip the inequality when dividing by a negative as well as not moving values over to a side properly.
\item \( (-\infty, a], \text{ where } a \in [0.75, 4.5] \)

 $(-\infty, 2.438]$, which corresponds to switching the direction of the interval. You likely did this if you did not flip the inequality when dividing by a negative!
\item \( \text{None of the above}. \)

You may have chosen this if you thought the inequality did not match the ends of the intervals.
\end{enumerate}

\textbf{General Comment:} Remember that less/greater than or equal to includes the endpoint, while less/greater do not. Also, remember that you need to flip the inequality when you multiply or divide by a negative.
}
\litem{
Solve the linear inequality below. Then, choose the constant and interval combination that describes the solution set.
\[ -7x + 6 \geq -4x + 4 \]The solution is \( (-\infty, 0.667] \), which is option B.\begin{enumerate}[label=\Alph*.]
\item \( (-\infty, a], \text{ where } a \in [-1.3, -0.3] \)

 $(-\infty, -0.667]$, which corresponds to negating the endpoint of the solution.
\item \( (-\infty, a], \text{ where } a \in [0, 4.3] \)

* $(-\infty, 0.667]$, which is the correct option.
\item \( [a, \infty), \text{ where } a \in [-3.3, -0.2] \)

 $[-0.667, \infty)$, which corresponds to switching the direction of the interval AND negating the endpoint. You likely did this if you did not flip the inequality when dividing by a negative as well as not moving values over to a side properly.
\item \( [a, \infty), \text{ where } a \in [-0.1, 2.6] \)

 $[0.667, \infty)$, which corresponds to switching the direction of the interval. You likely did this if you did not flip the inequality when dividing by a negative!
\item \( \text{None of the above}. \)

You may have chosen this if you thought the inequality did not match the ends of the intervals.
\end{enumerate}

\textbf{General Comment:} Remember that less/greater than or equal to includes the endpoint, while less/greater do not. Also, remember that you need to flip the inequality when you multiply or divide by a negative.
}
\litem{
Solve the linear inequality below. Then, choose the constant and interval combination that describes the solution set.
\[ -6 + 7 x > 9 x \text{ or } 6 + 4 x < 7 x \]The solution is \( (-\infty, -3.0) \text{ or } (2.0, \infty) \), which is option C.\begin{enumerate}[label=\Alph*.]
\item \( (-\infty, a] \cup [b, \infty), \text{ where } a \in [-2.15, -1.67] \text{ and } b \in [2.06, 4.54] \)

Corresponds to including the endpoints AND negating.
\item \( (-\infty, a] \cup [b, \infty), \text{ where } a \in [-3.44, -2.54] \text{ and } b \in [1.46, 2.53] \)

Corresponds to including the endpoints (when they should be excluded).
\item \( (-\infty, a) \cup (b, \infty), \text{ where } a \in [-5.33, -2.92] \text{ and } b \in [1.65, 2.77] \)

 * Correct option.
\item \( (-\infty, a) \cup (b, \infty), \text{ where } a \in [-2.4, -0.75] \text{ and } b \in [2.7, 4.88] \)

Corresponds to inverting the inequality and negating the solution.
\item \( (-\infty, \infty) \)

Corresponds to the variable canceling, which does not happen in this instance.
\end{enumerate}

\textbf{General Comment:} When multiplying or dividing by a negative, flip the sign.
}
\end{enumerate}

\end{document}