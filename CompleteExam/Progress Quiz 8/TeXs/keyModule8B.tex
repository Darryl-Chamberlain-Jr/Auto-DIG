\documentclass{extbook}[14pt]
\usepackage{multicol, enumerate, enumitem, hyperref, color, soul, setspace, parskip, fancyhdr, amssymb, amsthm, amsmath, latexsym, units, mathtools}
\everymath{\displaystyle}
\usepackage[headsep=0.5cm,headheight=0cm, left=1 in,right= 1 in,top= 1 in,bottom= 1 in]{geometry}
\usepackage{dashrule}  % Package to use the command below to create lines between items
\newcommand{\litem}[1]{\item #1

\rule{\textwidth}{0.4pt}}
\pagestyle{fancy}
\lhead{}
\chead{Answer Key for Progress Quiz 8 Version B}
\rhead{}
\lfoot{5493-4176}
\cfoot{}
\rfoot{Summer C 2021}
\begin{document}
\textbf{This key should allow you to understand why you choose the option you did (beyond just getting a question right or wrong). \href{https://xronos.clas.ufl.edu/mac1105spring2020/courseDescriptionAndMisc/Exams/LearningFromResults}{More instructions on how to use this key can be found here}.}

\textbf{If you have a suggestion to make the keys better, \href{https://forms.gle/CZkbZmPbC9XALEE88}{please fill out the short survey here}.}

\textit{Note: This key is auto-generated and may contain issues and/or errors. The keys are reviewed after each exam to ensure grading is done accurately. If there are issues (like duplicate options), they are noted in the offline gradebook. The keys are a work-in-progress to give students as many resources to improve as possible.}

\rule{\textwidth}{0.4pt}

\begin{enumerate}\litem{
 Solve the equation for $x$ and choose the interval that contains $x$ (if it exists).
\[  10 = \sqrt[5]{\frac{22}{e^{8x}}} \]The solution is \( x = -1.053 \), which is option C.\begin{enumerate}[label=\Alph*.]
\item \( x \in [-0.9, -0.18] \)

$x = -0.189$, which corresponds to treating any root as a square root.
\item \( x \in [-7.57, -5.58] \)

$x = -6.636$, which corresponds to thinking you don't need to take the natural log of both sides before reducing, as if the equation already had a natural log on the right side.
\item \( x \in [-1.85, -0.73] \)

* $x = -1.053$, which is the correct option.
\item \( \text{There is no Real solution to the equation.} \)

This corresponds to believing you cannot solve the equation.
\item \( \text{None of the above.} \)

This corresponds to making an unexpected error.
\end{enumerate}

\textbf{General Comment:} \textbf{General Comments}: After using the properties of logarithmic functions to break up the right-hand side, use $\ln(e) = 1$ to reduce the question to a linear function to solve. You can put $\ln(22)$ into a calculator if you are having trouble.
}
\litem{
Which of the following intervals describes the Range of the function below?
\[ f(x) = \log_2{(x-3)}+2 \]The solution is \( (\infty, \infty) \), which is option E.\begin{enumerate}[label=\Alph*.]
\item \( (-\infty, a), a \in [1.23, 2.99] \)

$(-\infty, 2)$, which corresponds to using the vertical shift while the Range is $(-\infty, \infty)$.
\item \( [a, \infty), a \in [2.99, 5.41] \)

$[2, \infty)$, which corresponds to using the flipped Domain AND including the endpoint.
\item \( [a, \infty), a \in [-4.36, -2.85] \)

$[-3, \infty)$, which corresponds to using the negative of the horizontal shift AND including the endpoint.
\item \( (-\infty, a), a \in [-2.45, -1.86] \)

$(-\infty, -2)$, which corresponds to using the using the negative of vertical shift on $(0, \infty)$.
\item \( (-\infty, \infty) \)

*This is the correct option.
\end{enumerate}

\textbf{General Comment:} \textbf{General Comments}: The domain of a basic logarithmic function is $(0, \infty)$ and the Range is $(-\infty, \infty)$. We can use shifts when finding the Domain, but the Range will always be all Real numbers.
}
\litem{
Solve the equation for $x$ and choose the interval that contains the solution (if it exists).
\[ 5^{-5x-4} = 27^{-3x+3} \]The solution is \( x = 8.871 \), which is option A.\begin{enumerate}[label=\Alph*.]
\item \( x \in [8.87, 9.87] \)

* $x = 8.871$, which is the correct option.
\item \( x \in [3.8, 5.8] \)

$x = 3.804$, which corresponds to distributing the $\ln(base)$ to the first term of the exponent only.
\item \( x \in [-3.5, -2.5] \)

$x = -3.500$, which corresponds to solving the numerators as equal while ignoring the bases are different.
\item \( x \in [-12.16, -7.16] \)

$x = -8.163$, which corresponds to distributing the $\ln(base)$ to the second term of the exponent only.
\item \( \text{There is no Real solution to the equation.} \)

This corresponds to believing there is no solution since the bases are not powers of each other.
\end{enumerate}

\textbf{General Comment:} \textbf{General Comments:} This question was written so that the bases could not be written the same. You will need to take the log of both sides.
}
\litem{
Which of the following intervals describes the Domain of the function below?
\[ f(x) = e^{x-3}-5 \]The solution is \( (-\infty, \infty) \), which is option E.\begin{enumerate}[label=\Alph*.]
\item \( (-\infty, a], a \in [-5, -3] \)

$(-\infty, -5]$, which corresponds to using the correct vertical shift *if we wanted the Range* AND including the endpoint.
\item \( (-\infty, a), a \in [-5, -3] \)

$(-\infty, -5)$, which corresponds to using the correct vertical shift *if we wanted the Range*.
\item \( [a, \infty), a \in [2, 12] \)

$[5, \infty)$, which corresponds to using the negative vertical shift AND flipping the Range interval AND including the endpoint.
\item \( (a, \infty), a \in [2, 12] \)

$(5, \infty)$, which corresponds to using the negative vertical shift AND flipping the Range interval.
\item \( (-\infty, \infty) \)

* This is the correct option.
\end{enumerate}

\textbf{General Comment:} \textbf{General Comments}: Domain of a basic exponential function is $(-\infty, \infty)$ while the Range is $(0, \infty)$. We can shift these intervals [and even flip when $a<0$!] to find the new Domain/Range.
}
\litem{
 Solve the equation for $x$ and choose the interval that contains $x$ (if it exists).
\[  12 = \sqrt[3]{\frac{24}{e^{6x}}} \]The solution is \( x = -0.713, \text{ which does not fit in any of the interval options.} \), which is option E.\begin{enumerate}[label=\Alph*.]
\item \( x \in [-6.68, -6.49] \)

$x = -6.530$, which corresponds to thinking you don't need to take the natural log of both sides before reducing, as if the right side already has a natural log.
\item \( x \in [-0.52, 0.11] \)

$x = -0.299$, which corresponds to treating any root as a square root.
\item \( x \in [0.53, 1.91] \)

$x = 0.713$, which is the negative of the correct solution.
\item \( \text{There is no Real solution to the equation.} \)

This corresponds to believing you cannot solve the equation.
\item \( \text{None of the above.} \)

* $x = -0.713$ is the correct solution and does not fit in any of the other intervals.
\end{enumerate}

\textbf{General Comment:} \textbf{General Comments}: After using the properties of logarithmic functions to break up the right-hand side, use $\ln(e) = 1$ to reduce the question to a linear function to solve. You can put $\ln(24)$ into a calculator if you are having trouble.
}
\litem{
Which of the following intervals describes the Range of the function below?
\[ f(x) = \log_2{(x-8)}-5 \]The solution is \( (\infty, \infty) \), which is option E.\begin{enumerate}[label=\Alph*.]
\item \( (-\infty, a), a \in [1.4, 5.1] \)

$(-\infty, 5)$, which corresponds to using the using the negative of vertical shift on $(0, \infty)$.
\item \( [a, \infty), a \in [7.8, 10.8] \)

$[-5, \infty)$, which corresponds to using the flipped Domain AND including the endpoint.
\item \( (-\infty, a), a \in [-7.6, -3.9] \)

$(-\infty, -5)$, which corresponds to using the vertical shift while the Range is $(-\infty, \infty)$.
\item \( [a, \infty), a \in [-8.6, -7.7] \)

$[-8, \infty)$, which corresponds to using the negative of the horizontal shift AND including the endpoint.
\item \( (-\infty, \infty) \)

*This is the correct option.
\end{enumerate}

\textbf{General Comment:} \textbf{General Comments}: The domain of a basic logarithmic function is $(0, \infty)$ and the Range is $(-\infty, \infty)$. We can use shifts when finding the Domain, but the Range will always be all Real numbers.
}
\litem{
Solve the equation for $x$ and choose the interval that contains the solution (if it exists).
\[ \log_{4}{(3x+6)}+4 = 3 \]The solution is \( x = -1.917 \), which is option A.\begin{enumerate}[label=\Alph*.]
\item \( x \in [-1.93, -1.85] \)

* $x = -1.917$, which is the correct option.
\item \( x \in [-1.68, -1.07] \)

$x = -1.667$, which corresponds to reversing the base and exponent when converting.
\item \( x \in [19.2, 20.36] \)

$x = 19.333$, which corresponds to ignoring the vertical shift when converting to exponential form.
\item \( x \in [2.25, 2.53] \)

$x = 2.333$, which corresponds to reversing the base and exponent when converting and reversing the value with $x$.
\item \( \text{There is no Real solution to the equation.} \)

Corresponds to believing a negative coefficient within the log equation means there is no Real solution.
\end{enumerate}

\textbf{General Comment:} \textbf{General Comments:} First, get the equation in the form $\log_b{(cx+d)} = a$. Then, convert to $b^a = cx+d$ and solve.
}
\litem{
Solve the equation for $x$ and choose the interval that contains the solution (if it exists).
\[ 3^{4x-2} = 125^{2x+2} \]The solution is \( x = -2.253 \), which is option A.\begin{enumerate}[label=\Alph*.]
\item \( x \in [-3.5, -1.3] \)

* $x = -2.253$, which is the correct option.
\item \( x \in [3.8, 7.3] \)

$x = 5.927$, which corresponds to distributing the $\ln(base)$ to the second term of the exponent only.
\item \( x \in [-1.6, 0.1] \)

$x = -0.760$, which corresponds to distributing the $\ln(base)$ to the first term of the exponent only.
\item \( x \in [1.5, 3] \)

$x = 2.000$, which corresponds to solving the numerators as equal while ignoring the bases are different.
\item \( \text{There is no Real solution to the equation.} \)

This corresponds to believing there is no solution since the bases are not powers of each other.
\end{enumerate}

\textbf{General Comment:} \textbf{General Comments:} This question was written so that the bases could not be written the same. You will need to take the log of both sides.
}
\litem{
Solve the equation for $x$ and choose the interval that contains the solution (if it exists).
\[ \log_{3}{(2x+5)}+6 = 3 \]The solution is \( x = -2.481 \), which is option B.\begin{enumerate}[label=\Alph*.]
\item \( x \in [-23, -13] \)

$x = -16.000$, which corresponds to reversing the base and exponent when converting.
\item \( x \in [-7.48, 0.52] \)

* $x = -2.481$, which is the correct option.
\item \( x \in [11, 20] \)

$x = 11.000$, which corresponds to ignoring the vertical shift when converting to exponential form.
\item \( x \in [-11, -4] \)

$x = -11.000$, which corresponds to reversing the base and exponent when converting and reversing the value with $x$.
\item \( \text{There is no Real solution to the equation.} \)

Corresponds to believing a negative coefficient within the log equation means there is no Real solution.
\end{enumerate}

\textbf{General Comment:} \textbf{General Comments:} First, get the equation in the form $\log_b{(cx+d)} = a$. Then, convert to $b^a = cx+d$ and solve.
}
\litem{
Which of the following intervals describes the Domain of the function below?
\[ f(x) = -e^{x-2}-6 \]The solution is \( (-\infty, \infty) \), which is option E.\begin{enumerate}[label=\Alph*.]
\item \( [a, \infty), a \in [5, 13] \)

$[6, \infty)$, which corresponds to using the negative vertical shift AND flipping the Range interval AND including the endpoint.
\item \( (a, \infty), a \in [5, 13] \)

$(6, \infty)$, which corresponds to using the negative vertical shift AND flipping the Range interval.
\item \( (-\infty, a), a \in [-7, -5] \)

$(-\infty, -6)$, which corresponds to using the correct vertical shift *if we wanted the Range*.
\item \( (-\infty, a], a \in [-7, -5] \)

$(-\infty, -6]$, which corresponds to using the correct vertical shift *if we wanted the Range* AND including the endpoint.
\item \( (-\infty, \infty) \)

* This is the correct option.
\end{enumerate}

\textbf{General Comment:} \textbf{General Comments}: Domain of a basic exponential function is $(-\infty, \infty)$ while the Range is $(0, \infty)$. We can shift these intervals [and even flip when $a<0$!] to find the new Domain/Range.
}
\end{enumerate}

\end{document}