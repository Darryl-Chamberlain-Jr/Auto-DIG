\documentclass{extbook}[14pt]
\usepackage{multicol, enumerate, enumitem, hyperref, color, soul, setspace, parskip, fancyhdr, amssymb, amsthm, amsmath, latexsym, units, mathtools}
\everymath{\displaystyle}
\usepackage[headsep=0.5cm,headheight=0cm, left=1 in,right= 1 in,top= 1 in,bottom= 1 in]{geometry}
\usepackage{dashrule}  % Package to use the command below to create lines between items
\newcommand{\litem}[1]{\item #1

\rule{\textwidth}{0.4pt}}
\pagestyle{fancy}
\lhead{}
\chead{Answer Key for Progress Quiz 8 Version C}
\rhead{}
\lfoot{5493-4176}
\cfoot{}
\rfoot{Summer C 2021}
\begin{document}
\textbf{This key should allow you to understand why you choose the option you did (beyond just getting a question right or wrong). \href{https://xronos.clas.ufl.edu/mac1105spring2020/courseDescriptionAndMisc/Exams/LearningFromResults}{More instructions on how to use this key can be found here}.}

\textbf{If you have a suggestion to make the keys better, \href{https://forms.gle/CZkbZmPbC9XALEE88}{please fill out the short survey here}.}

\textit{Note: This key is auto-generated and may contain issues and/or errors. The keys are reviewed after each exam to ensure grading is done accurately. If there are issues (like duplicate options), they are noted in the offline gradebook. The keys are a work-in-progress to give students as many resources to improve as possible.}

\rule{\textwidth}{0.4pt}

\begin{enumerate}\litem{
 Solve the equation for $x$ and choose the interval that contains $x$ (if it exists).
\[  13 = \sqrt[6]{\frac{26}{e^{6x}}} \]The solution is \( x = -2.022 \), which is option C.\begin{enumerate}[label=\Alph*.]
\item \( x \in [-15.2, -12.8] \)

$x = -13.543$, which corresponds to thinking you don't need to take the natural log of both sides before reducing, as if the equation already had a natural log on the right side.
\item \( x \in [-0.4, 0.5] \)

$x = -0.312$, which corresponds to treating any root as a square root.
\item \( x \in [-2.2, -1.3] \)

* $x = -2.022$, which is the correct option.
\item \( \text{There is no Real solution to the equation.} \)

This corresponds to believing you cannot solve the equation.
\item \( \text{None of the above.} \)

This corresponds to making an unexpected error.
\end{enumerate}

\textbf{General Comment:} \textbf{General Comments}: After using the properties of logarithmic functions to break up the right-hand side, use $\ln(e) = 1$ to reduce the question to a linear function to solve. You can put $\ln(26)$ into a calculator if you are having trouble.
}
\litem{
Which of the following intervals describes the Range of the function below?
\[ f(x) = \log_2{(x-5)}+8 \]The solution is \( (\infty, \infty) \), which is option E.\begin{enumerate}[label=\Alph*.]
\item \( (-\infty, a), a \in [-12, -7] \)

$(-\infty, -8)$, which corresponds to using the using the negative of vertical shift on $(0, \infty)$.
\item \( [a, \infty), a \in [0, 6] \)

$[8, \infty)$, which corresponds to using the flipped Domain AND including the endpoint.
\item \( (-\infty, a), a \in [7, 11] \)

$(-\infty, 8)$, which corresponds to using the vertical shift while the Range is $(-\infty, \infty)$.
\item \( [a, \infty), a \in [-7, -4] \)

$[-5, \infty)$, which corresponds to using the negative of the horizontal shift AND including the endpoint.
\item \( (-\infty, \infty) \)

*This is the correct option.
\end{enumerate}

\textbf{General Comment:} \textbf{General Comments}: The domain of a basic logarithmic function is $(0, \infty)$ and the Range is $(-\infty, \infty)$. We can use shifts when finding the Domain, but the Range will always be all Real numbers.
}
\litem{
Solve the equation for $x$ and choose the interval that contains the solution (if it exists).
\[ 3^{-5x-3} = \left(\frac{1}{125}\right)^{-2x+4} \]The solution is \( x = 1.057 \), which is option C.\begin{enumerate}[label=\Alph*.]
\item \( x \in [-3.4, -1.8] \)

$x = -2.333$, which corresponds to solving the numerators as equal while ignoring the bases are different.
\item \( x \in [4.7, 7] \)

$x = 5.339$, which corresponds to distributing the $\ln(base)$ to the second term of the exponent only.
\item \( x \in [0.1, 3.2] \)

* $x = 1.057$, which is the correct option.
\item \( x \in [-1.7, 0.1] \)

$x = -0.462$, which corresponds to distributing the $\ln(base)$ to the first term of the exponent only.
\item \( \text{There is no Real solution to the equation.} \)

This corresponds to believing there is no solution since the bases are not powers of each other.
\end{enumerate}

\textbf{General Comment:} \textbf{General Comments:} This question was written so that the bases could not be written the same. You will need to take the log of both sides.
}
\litem{
Which of the following intervals describes the Domain of the function below?
\[ f(x) = e^{x-4}-9 \]The solution is \( (-\infty, \infty) \), which is option E.\begin{enumerate}[label=\Alph*.]
\item \( (a, \infty), a \in [6, 15] \)

$(9, \infty)$, which corresponds to using the negative vertical shift AND flipping the Range interval.
\item \( (-\infty, a), a \in [-10, -7] \)

$(-\infty, -9)$, which corresponds to using the correct vertical shift *if we wanted the Range*.
\item \( [a, \infty), a \in [6, 15] \)

$[9, \infty)$, which corresponds to using the negative vertical shift AND flipping the Range interval AND including the endpoint.
\item \( (-\infty, a], a \in [-10, -7] \)

$(-\infty, -9]$, which corresponds to using the correct vertical shift *if we wanted the Range* AND including the endpoint.
\item \( (-\infty, \infty) \)

* This is the correct option.
\end{enumerate}

\textbf{General Comment:} \textbf{General Comments}: Domain of a basic exponential function is $(-\infty, \infty)$ while the Range is $(0, \infty)$. We can shift these intervals [and even flip when $a<0$!] to find the new Domain/Range.
}
\litem{
 Solve the equation for $x$ and choose the interval that contains $x$ (if it exists).
\[  21 = \ln{\sqrt[6]{\frac{22}{e^{3x}}}} \]The solution is \( x = -40.97 \), which is option B.\begin{enumerate}[label=\Alph*.]
\item \( x \in [-12.97, -11.97] \)

$x = -12.970$, which corresponds to treating any root as a square root.
\item \( x \in [-42.97, -36.97] \)

* $x = -40.970$, which is the correct option.
\item \( x \in [-9.12, -4.12] \)

$x = -7.119$, which corresponds to thinking you need to take the natural log of on the left before reducing.
\item \( \text{There is no Real solution to the equation.} \)

This corresponds to believing you cannot solve the equation.
\item \( \text{None of the above.} \)

This corresponds to making an unexpected error.
\end{enumerate}

\textbf{General Comment:} \textbf{General Comments}: After using the properties of logarithmic functions to break up the right-hand side, use $\ln(e) = 1$ to reduce the question to a linear function to solve. You can put $\ln(22)$ into a calculator if you are having trouble.
}
\litem{
Which of the following intervals describes the Domain of the function below?
\[ f(x) = \log_2{(x-5)}-8 \]The solution is \( (5, \infty) \), which is option C.\begin{enumerate}[label=\Alph*.]
\item \( (-\infty, a), a \in [-5.3, -3.2] \)

$(-\infty, -5)$, which corresponds to flipping the Domain. Remember: the general for is $a*\log(x-h)+k$, \textbf{where $a$ does not affect the domain}.
\item \( (-\infty, a], a \in [7.3, 12.2] \)

$(-\infty, 8]$, which corresponds to using the negative vertical shift AND including the endpoint AND flipping the domain.
\item \( (a, \infty), a \in [3.8, 5.2] \)

* $(5, \infty)$, which is the correct option.
\item \( [a, \infty), a \in [-9.5, -6.9] \)

$[-8, \infty)$, which corresponds to using the vertical shift when shifting the Domain AND including the endpoint.
\item \( (-\infty, \infty) \)

This corresponds to thinking of the range of the log function (or the domain of the exponential function).
\end{enumerate}

\textbf{General Comment:} \textbf{General Comments}: The domain of a basic logarithmic function is $(0, \infty)$ and the Range is $(-\infty, \infty)$. We can use shifts when finding the Domain, but the Range will always be all Real numbers.
}
\litem{
Solve the equation for $x$ and choose the interval that contains the solution (if it exists).
\[ \log_{4}{(-3x+7)}+4 = 2 \]The solution is \( x = 2.312 \), which is option C.\begin{enumerate}[label=\Alph*.]
\item \( x \in [-6, -2] \)

$x = -3.000$, which corresponds to reversing the base and exponent when converting.
\item \( x \in [-12.67, -5.67] \)

$x = -7.667$, which corresponds to reversing the base and exponent when converting and reversing the value with $x$.
\item \( x \in [-1.69, 6.31] \)

* $x = 2.312$, which is the correct option.
\item \( x \in [-6, -2] \)

$x = -3.000$, which corresponds to ignoring the vertical shift when converting to exponential form.
\item \( \text{There is no Real solution to the equation.} \)

Corresponds to believing a negative coefficient within the log equation means there is no Real solution.
\end{enumerate}

\textbf{General Comment:} \textbf{General Comments:} First, get the equation in the form $\log_b{(cx+d)} = a$. Then, convert to $b^a = cx+d$ and solve.
}
\litem{
Solve the equation for $x$ and choose the interval that contains the solution (if it exists).
\[ 2^{-3x+5} = \left(\frac{1}{9}\right)^{-4x-3} \]The solution is \( x = -0.288 \), which is option A.\begin{enumerate}[label=\Alph*.]
\item \( x \in [-1.29, 0.71] \)

* $x = -0.288$, which is the correct option.
\item \( x \in [2.13, 7.13] \)

$x = 3.126$, which corresponds to distributing the $\ln(base)$ to the second term of the exponent only.
\item \( x \in [-8, -6] \)

$x = -8.000$, which corresponds to solving the numerators as equal while ignoring the bases are different.
\item \( x \in [-0.26, 1.74] \)

$x = 0.736$, which corresponds to distributing the $\ln(base)$ to the first term of the exponent only.
\item \( \text{There is no Real solution to the equation.} \)

This corresponds to believing there is no solution since the bases are not powers of each other.
\end{enumerate}

\textbf{General Comment:} \textbf{General Comments:} This question was written so that the bases could not be written the same. You will need to take the log of both sides.
}
\litem{
Solve the equation for $x$ and choose the interval that contains the solution (if it exists).
\[ \log_{5}{(3x+7)}+6 = 3 \]The solution is \( x = -2.331 \), which is option A.\begin{enumerate}[label=\Alph*.]
\item \( x \in [-4.33, -1.33] \)

* $x = -2.331$, which is the correct option.
\item \( x \in [-89.33, -82.33] \)

$x = -83.333$, which corresponds to reversing the base and exponent when converting.
\item \( x \in [38.33, 41.33] \)

$x = 39.333$, which corresponds to ignoring the vertical shift when converting to exponential form.
\item \( x \in [-79.67, -76.67] \)

$x = -78.667$, which corresponds to reversing the base and exponent when converting and reversing the value with $x$.
\item \( \text{There is no Real solution to the equation.} \)

Corresponds to believing a negative coefficient within the log equation means there is no Real solution.
\end{enumerate}

\textbf{General Comment:} \textbf{General Comments:} First, get the equation in the form $\log_b{(cx+d)} = a$. Then, convert to $b^a = cx+d$ and solve.
}
\litem{
Which of the following intervals describes the Domain of the function below?
\[ f(x) = e^{x-1}-6 \]The solution is \( (-\infty, \infty) \), which is option E.\begin{enumerate}[label=\Alph*.]
\item \( (-\infty, a), a \in [-8, -5] \)

$(-\infty, -6)$, which corresponds to using the correct vertical shift *if we wanted the Range*.
\item \( (-\infty, a], a \in [-8, -5] \)

$(-\infty, -6]$, which corresponds to using the correct vertical shift *if we wanted the Range* AND including the endpoint.
\item \( (a, \infty), a \in [3, 10] \)

$(6, \infty)$, which corresponds to using the negative vertical shift AND flipping the Range interval.
\item \( [a, \infty), a \in [3, 10] \)

$[6, \infty)$, which corresponds to using the negative vertical shift AND flipping the Range interval AND including the endpoint.
\item \( (-\infty, \infty) \)

* This is the correct option.
\end{enumerate}

\textbf{General Comment:} \textbf{General Comments}: Domain of a basic exponential function is $(-\infty, \infty)$ while the Range is $(0, \infty)$. We can shift these intervals [and even flip when $a<0$!] to find the new Domain/Range.
}
\end{enumerate}

\end{document}