\documentclass[14pt]{extbook}
\usepackage{multicol, enumerate, enumitem, hyperref, color, soul, setspace, parskip, fancyhdr} %General Packages
\usepackage{amssymb, amsthm, amsmath, latexsym, units, mathtools} %Math Packages
\everymath{\displaystyle} %All math in Display Style
% Packages with additional options
\usepackage[headsep=0.5cm,headheight=12pt, left=1 in,right= 1 in,top= 1 in,bottom= 1 in]{geometry}
\usepackage[usenames,dvipsnames]{xcolor}
\usepackage{dashrule}  % Package to use the command below to create lines between items
\newcommand{\litem}[1]{\item#1\hspace*{-1cm}\rule{\textwidth}{0.4pt}}
\pagestyle{fancy}
\lhead{Progress Quiz 8}
\chead{}
\rhead{Version B}
\lfoot{5493-4176}
\cfoot{}
\rfoot{Summer C 2021}
\begin{document}

\begin{enumerate}
\litem{
What are the \textit{possible Rational} roots of the polynomial below?\[ f(x) = 5x^{2} +5 x + 4 \]\begin{enumerate}[label=\Alph*.]
\item \( \text{ All combinations of: }\frac{\pm 1,\pm 5}{\pm 1,\pm 2,\pm 4} \)
\item \( \pm 1,\pm 5 \)
\item \( \pm 1,\pm 2,\pm 4 \)
\item \( \text{ All combinations of: }\frac{\pm 1,\pm 2,\pm 4}{\pm 1,\pm 5} \)
\item \( \text{ There is no formula or theorem that tells us all possible Rational roots.} \)

\end{enumerate} }
\litem{
Factor the polynomial below completely. Then, choose the intervals the zeros of the polynomial belong to, where $z_1 \leq z_2 \leq z_3$. \textit{To make the problem easier, all zeros are between -5 and 5.}\[ f(x) = 9x^{3} +39 x^{2} -8 x -80 \]\begin{enumerate}[label=\Alph*.]
\item \( z_1 \in [-1.1, -0.4], \text{   }  z_2 \in [0.56, 0.75], \text{   and   } z_3 \in [3.49, 4.11] \)
\item \( z_1 \in [-2.6, -1.1], \text{   }  z_2 \in [1.59, 1.67], \text{   and   } z_3 \in [3.49, 4.11] \)
\item \( z_1 \in [-4.7, -3.7], \text{   }  z_2 \in [-1.72, -1.42], \text{   and   } z_3 \in [1.09, 1.96] \)
\item \( z_1 \in [-4.7, -3.7], \text{   }  z_2 \in [0.26, 0.56], \text{   and   } z_3 \in [3.49, 4.11] \)
\item \( z_1 \in [-4.7, -3.7], \text{   }  z_2 \in [-0.77, -0.4], \text{   and   } z_3 \in [0.37, 1.18] \)

\end{enumerate} }
\litem{
Factor the polynomial below completely, knowing that $x -5$ is a factor. Then, choose the intervals the zeros of the polynomial belong to, where $z_1 \leq z_2 \leq z_3 \leq z_4$. \textit{To make the problem easier, all zeros are between -5 and 5.}\[ f(x) = 15x^{4} -92 x^{3} +39 x^{2} +270 x -200 \]\begin{enumerate}[label=\Alph*.]
\item \( z_1 \in [-2.3, -1.61], \text{   }  z_2 \in [0.62, 0.99], z_3 \in [1.93, 2.23], \text{   and   } z_4 \in [4.89, 5.01] \)
\item \( z_1 \in [-5.63, -4.89], \text{   }  z_2 \in [-2.34, -1.66], z_3 \in [-0.87, -0.67], \text{   and   } z_4 \in [1.66, 1.72] \)
\item \( z_1 \in [-5.63, -4.89], \text{   }  z_2 \in [-2.34, -1.66], z_3 \in [-1.62, -1.11], \text{   and   } z_4 \in [0.45, 0.77] \)
\item \( z_1 \in [-1.59, 0.58], \text{   }  z_2 \in [1.2, 1.55], z_3 \in [1.93, 2.23], \text{   and   } z_4 \in [4.89, 5.01] \)
\item \( z_1 \in [-5.63, -4.89], \text{   }  z_2 \in [-4.07, -3.93], z_3 \in [-2.51, -1.78], \text{   and   } z_4 \in [0.05, 0.48] \)

\end{enumerate} }
\litem{
Factor the polynomial below completely, knowing that $x + 2$ is a factor. Then, choose the intervals the zeros of the polynomial belong to, where $z_1 \leq z_2 \leq z_3 \leq z_4$. \textit{To make the problem easier, all zeros are between -5 and 5.}\[ f(x) = 25x^{4} -30 x^{3} -92 x^{2} +120 x -32 \]\begin{enumerate}[label=\Alph*.]
\item \( z_1 \in [-2.17, -1.33], \text{   }  z_2 \in [0.53, 1.48], z_3 \in [1.78, 2.63], \text{   and   } z_4 \in [2.42, 2.65] \)
\item \( z_1 \in [-4.85, -3.25], \text{   }  z_2 \in [-2.97, -1.92], z_3 \in [-0.16, 0.52], \text{   and   } z_4 \in [1.76, 2.05] \)
\item \( z_1 \in [-3.07, -2.47], \text{   }  z_2 \in [-2.97, -1.92], z_3 \in [-1.63, -1.11], \text{   and   } z_4 \in [1.76, 2.05] \)
\item \( z_1 \in [-2.17, -1.33], \text{   }  z_2 \in [0.34, 0.92], z_3 \in [0.39, 0.81], \text{   and   } z_4 \in [1.76, 2.05] \)
\item \( z_1 \in [-2.17, -1.33], \text{   }  z_2 \in [-1.63, -0.5], z_3 \in [-0.48, -0.13], \text{   and   } z_4 \in [1.76, 2.05] \)

\end{enumerate} }
\litem{
Perform the division below. Then, find the intervals that correspond to the quotient in the form $ax^2+bx+c$ and remainder $r$.\[ \frac{16x^{3} -52 x^{2} +46 x -15}{x -2} \]\begin{enumerate}[label=\Alph*.]
\item \( a \in [32, 37], \text{   } b \in [8, 16], \text{   } c \in [69, 71], \text{   and   } r \in [122.2, 127.7]. \)
\item \( a \in [14, 24], \text{   } b \in [-23, -12], \text{   } c \in [6, 7], \text{   and   } r \in [-4.3, -0.7]. \)
\item \( a \in [14, 24], \text{   } b \in [-37, -31], \text{   } c \in [8, 14], \text{   and   } r \in [-6, -4.3]. \)
\item \( a \in [14, 24], \text{   } b \in [-88, -80], \text{   } c \in [210, 216], \text{   and   } r \in [-444, -440.8]. \)
\item \( a \in [32, 37], \text{   } b \in [-117, -112], \text{   } c \in [275, 283], \text{   and   } r \in [-571.4, -568.5]. \)

\end{enumerate} }
\litem{
Perform the division below. Then, find the intervals that correspond to the quotient in the form $ax^2+bx+c$ and remainder $r$.\[ \frac{10x^{3} +42 x^{2} -34}{x + 4} \]\begin{enumerate}[label=\Alph*.]
\item \( a \in [9, 11], b \in [-9, -7], c \in [39, 41], \text{ and } r \in [-237, -229]. \)
\item \( a \in [9, 11], b \in [2, 6], c \in [-9, -7], \text{ and } r \in [-7, 2]. \)
\item \( a \in [-42, -35], b \in [-123, -117], c \in [-484, -465], \text{ and } r \in [-1925, -1919]. \)
\item \( a \in [9, 11], b \in [81, 85], c \in [327, 330], \text{ and } r \in [1277, 1283]. \)
\item \( a \in [-42, -35], b \in [198, 207], c \in [-810, -806], \text{ and } r \in [3198, 3204]. \)

\end{enumerate} }
\litem{
Perform the division below. Then, find the intervals that correspond to the quotient in the form $ax^2+bx+c$ and remainder $r$.\[ \frac{12x^{3} -28 x^{2} + 18}{x -2} \]\begin{enumerate}[label=\Alph*.]
\item \( a \in [10, 14], b \in [-7, -3], c \in [-9, -5], \text{ and } r \in [-2, 9]. \)
\item \( a \in [10, 14], b \in [-52, -47], c \in [99, 105], \text{ and } r \in [-191, -188]. \)
\item \( a \in [10, 14], b \in [-16, -12], c \in [-16, -10], \text{ and } r \in [-2, 9]. \)
\item \( a \in [24, 28], b \in [18, 24], c \in [32, 42], \text{ and } r \in [94, 102]. \)
\item \( a \in [24, 28], b \in [-78, -75], c \in [145, 153], \text{ and } r \in [-289, -284]. \)

\end{enumerate} }
\litem{
What are the \textit{possible Rational} roots of the polynomial below?\[ f(x) = 3x^{3} +4 x^{2} +2 x + 4 \]\begin{enumerate}[label=\Alph*.]
\item \( \pm 1,\pm 3 \)
\item \( \pm 1,\pm 2,\pm 4 \)
\item \( \text{ All combinations of: }\frac{\pm 1,\pm 2,\pm 4}{\pm 1,\pm 3} \)
\item \( \text{ All combinations of: }\frac{\pm 1,\pm 3}{\pm 1,\pm 2,\pm 4} \)
\item \( \text{ There is no formula or theorem that tells us all possible Rational roots.} \)

\end{enumerate} }
\litem{
Perform the division below. Then, find the intervals that correspond to the quotient in the form $ax^2+bx+c$ and remainder $r$.\[ \frac{15x^{3} +25 x^{2} -20 x -18}{x + 2} \]\begin{enumerate}[label=\Alph*.]
\item \( a \in [10, 18], \text{   } b \in [-25, -12], \text{   } c \in [40, 42], \text{   and   } r \in [-140, -137]. \)
\item \( a \in [10, 18], \text{   } b \in [-7, 0], \text{   } c \in [-11, -8], \text{   and   } r \in [2, 4]. \)
\item \( a \in [10, 18], \text{   } b \in [52, 58], \text{   } c \in [85, 91], \text{   and   } r \in [160, 168]. \)
\item \( a \in [-35, -26], \text{   } b \in [-39, -30], \text{   } c \in [-93, -85], \text{   and   } r \in [-198, -197]. \)
\item \( a \in [-35, -26], \text{   } b \in [83, 90], \text{   } c \in [-190, -185], \text{   and   } r \in [360, 364]. \)

\end{enumerate} }
\litem{
Factor the polynomial below completely. Then, choose the intervals the zeros of the polynomial belong to, where $z_1 \leq z_2 \leq z_3$. \textit{To make the problem easier, all zeros are between -5 and 5.}\[ f(x) = 20x^{3} -123 x^{2} +121 x -30 \]\begin{enumerate}[label=\Alph*.]
\item \( z_1 \in [0.7, 2.1], \text{   }  z_2 \in [2.4, 2.6], \text{   and   } z_3 \in [4.71, 5.05] \)
\item \( z_1 \in [-5.8, -3.8], \text{   }  z_2 \in [-2.7, -1.6], \text{   and   } z_3 \in [-1.34, -1.16] \)
\item \( z_1 \in [-5.8, -3.8], \text{   }  z_2 \in [-1.5, -0.6], \text{   and   } z_3 \in [-0.48, -0.32] \)
\item \( z_1 \in [-0.4, 0.7], \text{   }  z_2 \in [0.5, 2], \text{   and   } z_3 \in [4.71, 5.05] \)
\item \( z_1 \in [-5.8, -3.8], \text{   }  z_2 \in [-3.3, -2.8], \text{   and   } z_3 \in [-0.11, -0.07] \)

\end{enumerate} }
\end{enumerate}

\end{document}