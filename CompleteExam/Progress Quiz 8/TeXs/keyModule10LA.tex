\documentclass{extbook}[14pt]
\usepackage{multicol, enumerate, enumitem, hyperref, color, soul, setspace, parskip, fancyhdr, amssymb, amsthm, amsmath, latexsym, units, mathtools}
\everymath{\displaystyle}
\usepackage[headsep=0.5cm,headheight=0cm, left=1 in,right= 1 in,top= 1 in,bottom= 1 in]{geometry}
\usepackage{dashrule}  % Package to use the command below to create lines between items
\newcommand{\litem}[1]{\item #1

\rule{\textwidth}{0.4pt}}
\pagestyle{fancy}
\lhead{}
\chead{Answer Key for Progress Quiz 8 Version A}
\rhead{}
\lfoot{5493-4176}
\cfoot{}
\rfoot{Summer C 2021}
\begin{document}
\textbf{This key should allow you to understand why you choose the option you did (beyond just getting a question right or wrong). \href{https://xronos.clas.ufl.edu/mac1105spring2020/courseDescriptionAndMisc/Exams/LearningFromResults}{More instructions on how to use this key can be found here}.}

\textbf{If you have a suggestion to make the keys better, \href{https://forms.gle/CZkbZmPbC9XALEE88}{please fill out the short survey here}.}

\textit{Note: This key is auto-generated and may contain issues and/or errors. The keys are reviewed after each exam to ensure grading is done accurately. If there are issues (like duplicate options), they are noted in the offline gradebook. The keys are a work-in-progress to give students as many resources to improve as possible.}

\rule{\textwidth}{0.4pt}

\begin{enumerate}\litem{
What are the \textit{possible Integer} roots of the polynomial below?
\[ f(x) = 6x^{3} +6 x^{2} +3 x + 3 \]The solution is \( \pm 1,\pm 3 \), which is option C.\begin{enumerate}[label=\Alph*.]
\item \( \text{ All combinations of: }\frac{\pm 1,\pm 2,\pm 3,\pm 6}{\pm 1,\pm 3} \)

 Distractor 3: Corresponds to the plus or minus of the inverse quotient (an/a0) of the factors. 
\item \( \text{ All combinations of: }\frac{\pm 1,\pm 3}{\pm 1,\pm 2,\pm 3,\pm 6} \)

This would have been the solution \textbf{if asked for the possible Rational roots}!
\item \( \pm 1,\pm 3 \)

* This is the solution \textbf{since we asked for the possible Integer roots}!
\item \( \pm 1,\pm 2,\pm 3,\pm 6 \)

 Distractor 1: Corresponds to the plus or minus factors of a1 only.
\item \( \text{There is no formula or theorem that tells us all possible Integer roots.} \)

 Distractor 4: Corresponds to not recognizing Integers as a subset of Rationals.
\end{enumerate}

\textbf{General Comment:} We have a way to find the possible Rational roots. The possible Integer roots are the Integers in this list.
}
\litem{
Factor the polynomial below completely. Then, choose the intervals the zeros of the polynomial belong to, where $z_1 \leq z_2 \leq z_3$. \textit{To make the problem easier, all zeros are between -5 and 5.}
\[ f(x) = 10x^{3} +69 x^{2} +126 x + 40 \]The solution is \( [-4, -2.5, -0.4] \), which is option C.\begin{enumerate}[label=\Alph*.]
\item \( z_1 \in [0.5, 0.51], \text{   }  z_2 \in [1.71, 2.09], \text{   and   } z_3 \in [3, 8] \)

 Distractor 4: Corresponds to moving factors from one rational to another.
\item \( z_1 \in [-4.04, -3.94], \text{   }  z_2 \in [-2.53, -2.26], \text{   and   } z_3 \in [-0.4, 1.6] \)

 Distractor 2: Corresponds to inversing rational roots.
\item \( z_1 \in [-4.04, -3.94], \text{   }  z_2 \in [-2.53, -2.26], \text{   and   } z_3 \in [-0.4, 1.6] \)

* This is the solution!
\item \( z_1 \in [0.26, 0.44], \text{   }  z_2 \in [2.47, 2.97], \text{   and   } z_3 \in [3, 8] \)

 Distractor 3: Corresponds to negatives of all zeros AND inversing rational roots.
\item \( z_1 \in [0.26, 0.44], \text{   }  z_2 \in [2.47, 2.97], \text{   and   } z_3 \in [3, 8] \)

 Distractor 1: Corresponds to negatives of all zeros.
\end{enumerate}

\textbf{General Comment:} Remember to try the middle-most integers first as these normally are the zeros. Also, once you get it to a quadratic, you can use your other factoring techniques to finish factoring.
}
\litem{
Factor the polynomial below completely, knowing that $x -5$ is a factor. Then, choose the intervals the zeros of the polynomial belong to, where $z_1 \leq z_2 \leq z_3 \leq z_4$. \textit{To make the problem easier, all zeros are between -5 and 5.}
\[ f(x) = 9x^{4} -72 x^{3} +143 x^{2} -20 x -100 \]The solution is \( [-0.667, 1.667, 2, 5] \), which is option C.\begin{enumerate}[label=\Alph*.]
\item \( z_1 \in [-1.83, -1.34], \text{   }  z_2 \in [0, 1.3], z_3 \in [1.98, 2.07], \text{   and   } z_4 \in [4.67, 5.09] \)

 Distractor 2: Corresponds to inversing rational roots.
\item \( z_1 \in [-5.3, -4.93], \text{   }  z_2 \in [-2.3, -1.3], z_3 \in [-0.62, -0.59], \text{   and   } z_4 \in [1.38, 1.7] \)

 Distractor 3: Corresponds to negatives of all zeros AND inversing rational roots.
\item \( z_1 \in [-0.84, -0.37], \text{   }  z_2 \in [0.9, 3.3], z_3 \in [1.98, 2.07], \text{   and   } z_4 \in [4.67, 5.09] \)

* This is the solution!
\item \( z_1 \in [-5.3, -4.93], \text{   }  z_2 \in [-2.3, -1.3], z_3 \in [-0.56, -0.53], \text{   and   } z_4 \in [1.66, 2.22] \)

 Distractor 4: Corresponds to moving factors from one rational to another.
\item \( z_1 \in [-5.3, -4.93], \text{   }  z_2 \in [-2.3, -1.3], z_3 \in [-1.73, -1.65], \text{   and   } z_4 \in [0.46, 1.24] \)

 Distractor 1: Corresponds to negatives of all zeros.
\end{enumerate}

\textbf{General Comment:} Remember to try the middle-most integers first as these normally are the zeros. Also, once you get it to a quadratic, you can use your other factoring techniques to finish factoring.
}
\litem{
Factor the polynomial below completely, knowing that $x -2$ is a factor. Then, choose the intervals the zeros of the polynomial belong to, where $z_1 \leq z_2 \leq z_3 \leq z_4$. \textit{To make the problem easier, all zeros are between -5 and 5.}
\[ f(x) = 20x^{4} -127 x^{3} +94 x^{2} +235 x -150 \]The solution is \( [-1.25, 0.6, 2, 5] \), which is option B.\begin{enumerate}[label=\Alph*.]
\item \( z_1 \in [-5, -4.55], \text{   }  z_2 \in [-2.9, -0.1], z_3 \in [-0.71, -0.18], \text{   and   } z_4 \in [1, 1.45] \)

 Distractor 1: Corresponds to negatives of all zeros.
\item \( z_1 \in [-1.28, -1.03], \text{   }  z_2 \in [0, 1.5], z_3 \in [1.97, 2.15], \text{   and   } z_4 \in [4.9, 5.35] \)

* This is the solution!
\item \( z_1 \in [-5, -4.55], \text{   }  z_2 \in [-3.9, -2.1], z_3 \in [-2.2, -1.98], \text{   and   } z_4 \in [-0.37, 0.46] \)

 Distractor 4: Corresponds to moving factors from one rational to another.
\item \( z_1 \in [-5, -4.55], \text{   }  z_2 \in [-2.9, -0.1], z_3 \in [-1.86, -1.52], \text{   and   } z_4 \in [0.53, 0.99] \)

 Distractor 3: Corresponds to negatives of all zeros AND inversing rational roots.
\item \( z_1 \in [-1.21, -0.52], \text{   }  z_2 \in [1.5, 1.8], z_3 \in [1.97, 2.15], \text{   and   } z_4 \in [4.9, 5.35] \)

 Distractor 2: Corresponds to inversing rational roots.
\end{enumerate}

\textbf{General Comment:} Remember to try the middle-most integers first as these normally are the zeros. Also, once you get it to a quadratic, you can use your other factoring techniques to finish factoring.
}
\litem{
Perform the division below. Then, find the intervals that correspond to the quotient in the form $ax^2+bx+c$ and remainder $r$.
\[ \frac{8x^{3} -10 x^{2} -32 x + 43}{x + 2} \]The solution is \( 8x^{2} -26 x + 20 + \frac{3}{x + 2} \), which is option D.\begin{enumerate}[label=\Alph*.]
\item \( a \in [7, 9], \text{   } b \in [-36, -33], \text{   } c \in [67, 76], \text{   and   } r \in [-172, -165]. \)

 You multiplied by the synthetic number and subtracted rather than adding during synthetic division.
\item \( a \in [-19, -14], \text{   } b \in [22, 23], \text{   } c \in [-78, -74], \text{   and   } r \in [195, 200]. \)

 You multiplied by the synthetic number rather than bringing the first factor down.
\item \( a \in [-19, -14], \text{   } b \in [-45, -41], \text{   } c \in [-120, -112], \text{   and   } r \in [-189, -187]. \)

 You divided by the opposite of the factor AND multiplied the first factor rather than just bringing it down.
\item \( a \in [7, 9], \text{   } b \in [-26, -23], \text{   } c \in [19, 25], \text{   and   } r \in [2, 11]. \)

* This is the solution!
\item \( a \in [7, 9], \text{   } b \in [4, 13], \text{   } c \in [-23, -14], \text{   and   } r \in [2, 11]. \)

 You divided by the opposite of the factor.
\end{enumerate}

\textbf{General Comment:} Be sure to synthetically divide by the zero of the denominator!
}
\litem{
Perform the division below. Then, find the intervals that correspond to the quotient in the form $ax^2+bx+c$ and remainder $r$.
\[ \frac{15x^{3} +63 x^{2} -43}{x + 4} \]The solution is \( 15x^{2} +3 x -12 + \frac{5}{x + 4} \), which is option A.\begin{enumerate}[label=\Alph*.]
\item \( a \in [14, 18], b \in [-3, 5], c \in [-14, -8], \text{ and } r \in [5, 9]. \)

* This is the solution!
\item \( a \in [14, 18], b \in [120, 126], c \in [492, 495], \text{ and } r \in [1925, 1931]. \)

 You divided by the opposite of the factor.
\item \( a \in [14, 18], b \in [-13, -8], c \in [60, 61], \text{ and } r \in [-348, -342]. \)

 You multipled by the synthetic number and subtracted rather than adding during synthetic division.
\item \( a \in [-64, -57], b \in [-184, -173], c \in [-709, -707], \text{ and } r \in [-2875, -2869]. \)

 You divided by the opposite of the factor AND multipled the first factor rather than just bringing it down.
\item \( a \in [-64, -57], b \in [303, 307], c \in [-1213, -1210], \text{ and } r \in [4803, 4809]. \)

 You multipled by the synthetic number rather than bringing the first factor down.
\end{enumerate}

\textbf{General Comment:} Be sure to synthetically divide by the zero of the denominator! Also, make sure to include 0 placeholders for missing terms.
}
\litem{
Perform the division below. Then, find the intervals that correspond to the quotient in the form $ax^2+bx+c$ and remainder $r$.
\[ \frac{4x^{3} +12 x^{2} -11}{x + 2} \]The solution is \( 4x^{2} +4 x -8 + \frac{5}{x + 2} \), which is option D.\begin{enumerate}[label=\Alph*.]
\item \( a \in [-8, -6], b \in [22, 32], c \in [-60, -55], \text{ and } r \in [101, 107]. \)

 You multipled by the synthetic number rather than bringing the first factor down.
\item \( a \in [-2, 7], b \in [0, 2], c \in [0, 2], \text{ and } r \in [-13, -10]. \)

 You multipled by the synthetic number and subtracted rather than adding during synthetic division.
\item \( a \in [-2, 7], b \in [17, 25], c \in [40, 41], \text{ and } r \in [64, 71]. \)

 You divided by the opposite of the factor.
\item \( a \in [-2, 7], b \in [2, 5], c \in [-11, -5], \text{ and } r \in [5, 9]. \)

* This is the solution!
\item \( a \in [-8, -6], b \in [-6, -3], c \in [-11, -5], \text{ and } r \in [-30, -25]. \)

 You divided by the opposite of the factor AND multipled the first factor rather than just bringing it down.
\end{enumerate}

\textbf{General Comment:} Be sure to synthetically divide by the zero of the denominator! Also, make sure to include 0 placeholders for missing terms.
}
\litem{
What are the \textit{possible Rational} roots of the polynomial below?
\[ f(x) = 7x^{2} +7 x + 5 \]The solution is \( \text{ All combinations of: }\frac{\pm 1,\pm 5}{\pm 1,\pm 7} \), which is option B.\begin{enumerate}[label=\Alph*.]
\item \( \pm 1,\pm 7 \)

 Distractor 1: Corresponds to the plus or minus factors of a1 only.
\item \( \text{ All combinations of: }\frac{\pm 1,\pm 5}{\pm 1,\pm 7} \)

* This is the solution \textbf{since we asked for the possible Rational roots}!
\item \( \text{ All combinations of: }\frac{\pm 1,\pm 7}{\pm 1,\pm 5} \)

 Distractor 3: Corresponds to the plus or minus of the inverse quotient (an/a0) of the factors. 
\item \( \pm 1,\pm 5 \)

This would have been the solution \textbf{if asked for the possible Integer roots}!
\item \( \text{ There is no formula or theorem that tells us all possible Rational roots.} \)

 Distractor 4: Corresponds to not recalling the theorem for rational roots of a polynomial.
\end{enumerate}

\textbf{General Comment:} We have a way to find the possible Rational roots. The possible Integer roots are the Integers in this list.
}
\litem{
Perform the division below. Then, find the intervals that correspond to the quotient in the form $ax^2+bx+c$ and remainder $r$.
\[ \frac{8x^{3} +6 x^{2} -32 x -19}{x -2} \]The solution is \( 8x^{2} +22 x + 12 + \frac{5}{x -2} \), which is option A.\begin{enumerate}[label=\Alph*.]
\item \( a \in [8, 10], \text{   } b \in [18, 29], \text{   } c \in [12, 16], \text{   and   } r \in [4, 8]. \)

* This is the solution!
\item \( a \in [16, 19], \text{   } b \in [-31, -20], \text{   } c \in [16, 25], \text{   and   } r \in [-59, -54]. \)

 You divided by the opposite of the factor AND multiplied the first factor rather than just bringing it down.
\item \( a \in [8, 10], \text{   } b \in [13, 17], \text{   } c \in [-22, -15], \text{   and   } r \in [-39, -36]. \)

 You multiplied by the synthetic number and subtracted rather than adding during synthetic division.
\item \( a \in [16, 19], \text{   } b \in [30, 43], \text{   } c \in [37, 48], \text{   and   } r \in [65, 72]. \)

 You multiplied by the synthetic number rather than bringing the first factor down.
\item \( a \in [8, 10], \text{   } b \in [-14, -8], \text{   } c \in [-15, -7], \text{   and   } r \in [4, 8]. \)

 You divided by the opposite of the factor.
\end{enumerate}

\textbf{General Comment:} Be sure to synthetically divide by the zero of the denominator!
}
\litem{
Factor the polynomial below completely. Then, choose the intervals the zeros of the polynomial belong to, where $z_1 \leq z_2 \leq z_3$. \textit{To make the problem easier, all zeros are between -5 and 5.}
\[ f(x) = 15x^{3} -23 x^{2} -58 x -24 \]The solution is \( [-0.8, -0.67, 3] \), which is option C.\begin{enumerate}[label=\Alph*.]
\item \( z_1 \in [-3.4, -2.1], \text{   }  z_2 \in [-0.39, 0.17], \text{   and   } z_3 \in [3.38, 4.26] \)

 Distractor 4: Corresponds to moving factors from one rational to another.
\item \( z_1 \in [-1.7, -1.3], \text{   }  z_2 \in [-1.49, -1.12], \text{   and   } z_3 \in [2.93, 3.49] \)

 Distractor 2: Corresponds to inversing rational roots.
\item \( z_1 \in [-1, 0.4], \text{   }  z_2 \in [-1.03, -0.38], \text{   and   } z_3 \in [2.93, 3.49] \)

* This is the solution!
\item \( z_1 \in [-3.4, -2.1], \text{   }  z_2 \in [0.52, 0.91], \text{   and   } z_3 \in [0.31, 1.05] \)

 Distractor 1: Corresponds to negatives of all zeros.
\item \( z_1 \in [-3.4, -2.1], \text{   }  z_2 \in [1.13, 1.64], \text{   and   } z_3 \in [1.47, 1.82] \)

 Distractor 3: Corresponds to negatives of all zeros AND inversing rational roots.
\end{enumerate}

\textbf{General Comment:} Remember to try the middle-most integers first as these normally are the zeros. Also, once you get it to a quadratic, you can use your other factoring techniques to finish factoring.
}
\end{enumerate}

\end{document}