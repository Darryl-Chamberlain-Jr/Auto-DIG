\documentclass{extbook}[14pt]
\usepackage{multicol, enumerate, enumitem, hyperref, color, soul, setspace, parskip, fancyhdr, amssymb, amsthm, amsmath, latexsym, units, mathtools}
\everymath{\displaystyle}
\usepackage[headsep=0.5cm,headheight=0cm, left=1 in,right= 1 in,top= 1 in,bottom= 1 in]{geometry}
\usepackage{dashrule}  % Package to use the command below to create lines between items
\newcommand{\litem}[1]{\item #1

\rule{\textwidth}{0.4pt}}
\pagestyle{fancy}
\lhead{}
\chead{Answer Key for Progress Quiz 8 Version A}
\rhead{}
\lfoot{5493-4176}
\cfoot{}
\rfoot{Summer C 2021}
\begin{document}
\textbf{This key should allow you to understand why you choose the option you did (beyond just getting a question right or wrong). \href{https://xronos.clas.ufl.edu/mac1105spring2020/courseDescriptionAndMisc/Exams/LearningFromResults}{More instructions on how to use this key can be found here}.}

\textbf{If you have a suggestion to make the keys better, \href{https://forms.gle/CZkbZmPbC9XALEE88}{please fill out the short survey here}.}

\textit{Note: This key is auto-generated and may contain issues and/or errors. The keys are reviewed after each exam to ensure grading is done accurately. If there are issues (like duplicate options), they are noted in the offline gradebook. The keys are a work-in-progress to give students as many resources to improve as possible.}

\rule{\textwidth}{0.4pt}

\begin{enumerate}\litem{
Find the inverse of the function below (if it exists). Then, evaluate the inverse at $x = 15$ and choose the interval that $f^-1(15)$ belongs to.
\[ f(x) = \sqrt[3]{3 x - 4} \]The solution is \( 1126.3333333333333 \), which is option B.\begin{enumerate}[label=\Alph*.]
\item \( f^{-1}(15) \in [-1128.5, -1124.2] \)

 This solution corresponds to distractor 2.
\item \( f^{-1}(15) \in [1124.3, 1128.9] \)

* This is the correct solution.
\item \( f^{-1}(15) \in [-1124.2, -1120.8] \)

 This solution corresponds to distractor 3.
\item \( f^{-1}(15) \in [1121.9, 1125.8] \)

 Distractor 1: This corresponds to 
\item \( \text{ The function is not invertible for all Real numbers. } \)

 This solution corresponds to distractor 4.
\end{enumerate}

\textbf{General Comment:} Be sure you check that the function is 1-1 before trying to find the inverse!
}
\litem{
Find the inverse of the function below. Then, evaluate the inverse at $x = 7$ and choose the interval that $f^-1(7)$ belongs to.
\[ f(x) = e^{x+5}-3 \]The solution is \( f^{-1}(7) = -2.697 \), which is option C.\begin{enumerate}[label=\Alph*.]
\item \( f^{-1}(7) \in [-2.55, -2.18] \)

 This solution corresponds to distractor 3.
\item \( f^{-1}(7) \in [-1.84, -0.93] \)

 This solution corresponds to distractor 2.
\item \( f^{-1}(7) \in [-3.14, -2.59] \)

 This is the solution.
\item \( f^{-1}(7) \in [6.87, 7.36] \)

 This solution corresponds to distractor 1.
\item \( f^{-1}(7) \in [-0.62, -0.27] \)

 This solution corresponds to distractor 4.
\end{enumerate}

\textbf{General Comment:} Natural log and exponential functions always have an inverse. Once you switch the $x$ and $y$, use the conversion $ e^y = x \leftrightarrow y=\ln(x)$.
}
\litem{
Find the inverse of the function below. Then, evaluate the inverse at $x = 7$ and choose the interval that $f^-1(7)$ belongs to.
\[ f(x) = e^{x-5}+3 \]The solution is \( f^{-1}(7) = 6.386 \), which is option C.\begin{enumerate}[label=\Alph*.]
\item \( f^{-1}(7) \in [5.43, 5.54] \)

 This solution corresponds to distractor 3.
\item \( f^{-1}(7) \in [5.12, 5.31] \)

 This solution corresponds to distractor 2.
\item \( f^{-1}(7) \in [6.27, 6.45] \)

 This is the solution.
\item \( f^{-1}(7) \in [3.54, 3.82] \)

 This solution corresponds to distractor 4.
\item \( f^{-1}(7) \in [-3.62, -3.52] \)

 This solution corresponds to distractor 1.
\end{enumerate}

\textbf{General Comment:} Natural log and exponential functions always have an inverse. Once you switch the $x$ and $y$, use the conversion $ e^y = x \leftrightarrow y=\ln(x)$.
}
\litem{
Multiply the following functions, then choose the domain of the resulting function from the list below.
\[ f(x) = 6x^{4} +4 x^{2} +7 x + 3 \text{ and } g(x) = \sqrt{-6x-27}  \]The solution is \( \text{ The domain is all Real numbers less than or equal to} x = -4.5. \), which is option C.\begin{enumerate}[label=\Alph*.]
\item \( \text{ The domain is all Real numbers except } x = a, \text{ where } a \in [6.25, 9.25] \)


\item \( \text{ The domain is all Real numbers greater than or equal to } x = a, \text{ where } a \in [3.5, 10.5] \)


\item \( \text{ The domain is all Real numbers less than or equal to } x = a, \text{ where } a \in [-12.5, -1.5] \)


\item \( \text{ The domain is all Real numbers except } x = a \text{ and } x = b, \text{ where } a \in [1.4, 5.4] \text{ and } b \in [1.25, 7.25] \)


\item \( \text{ The domain is all Real numbers. } \)


\end{enumerate}

\textbf{General Comment:} The new domain is the intersection of the previous domains.
}
\litem{
Determine whether the function below is 1-1.
\[ f(x) = 18 x^2 - 42 x - 196 \]The solution is \( \text{no} \), which is option A.\begin{enumerate}[label=\Alph*.]
\item \( \text{No, because there is a $y$-value that goes to 2 different $x$-values.} \)

* This is the solution.
\item \( \text{No, because the range of the function is not $(-\infty, \infty)$.} \)

Corresponds to believing 1-1 means the range is all Real numbers.
\item \( \text{Yes, the function is 1-1.} \)

Corresponds to believing the function passes the Horizontal Line test.
\item \( \text{No, because the domain of the function is not $(-\infty, \infty)$.} \)

Corresponds to believing 1-1 means the domain is all Real numbers.
\item \( \text{No, because there is an $x$-value that goes to 2 different $y$-values.} \)

Corresponds to the Vertical Line test, which checks if an expression is a function.
\end{enumerate}

\textbf{General Comment:} There are only two valid options: The function is 1-1 OR No because there is a $y$-value that goes to 2 different $x$-values.
}
\litem{
Choose the interval below that $f$ composed with $g$ at $x=-1$ is in.
\[ f(x) = 3x^{3} +2 x^{2} -4 x -4 \text{ and } g(x) = 3x^{3} + x^{2} +2 x + 3 \]The solution is \( -1.0 \), which is option B.\begin{enumerate}[label=\Alph*.]
\item \( (f \circ g)(-1) \in [1.9, 4.3] \)

 Distractor 2: Corresponds to being slightly off from the solution.
\item \( (f \circ g)(-1) \in [-2.4, 0.1] \)

* This is the correct solution
\item \( (f \circ g)(-1) \in [-2.4, 0.1] \)

 Distractor 1: Corresponds to reversing the composition.
\item \( (f \circ g)(-1) \in [5.6, 7.5] \)

 Distractor 3: Corresponds to being slightly off from the solution.
\item \( \text{It is not possible to compose the two functions.} \)


\end{enumerate}

\textbf{General Comment:} $f$ composed with $g$ at $x$ means $f(g(x))$. The order matters!
}
\litem{
Find the inverse of the function below (if it exists). Then, evaluate the inverse at $x = 12$ and choose the interval that $f^-1(12)$ belongs to.
\[ f(x) = \sqrt[3]{5 x + 2} \]The solution is \( 345.2 \), which is option D.\begin{enumerate}[label=\Alph*.]
\item \( f^{-1}(12) \in [-345.36, -344.52] \)

 This solution corresponds to distractor 2.
\item \( f^{-1}(12) \in [345.63, 346.11] \)

 Distractor 1: This corresponds to 
\item \( f^{-1}(12) \in [-346.21, -345.48] \)

 This solution corresponds to distractor 3.
\item \( f^{-1}(12) \in [344.56, 345.27] \)

* This is the correct solution.
\item \( \text{ The function is not invertible for all Real numbers. } \)

 This solution corresponds to distractor 4.
\end{enumerate}

\textbf{General Comment:} Be sure you check that the function is 1-1 before trying to find the inverse!
}
\litem{
Choose the interval below that $f$ composed with $g$ at $x=-1$ is in.
\[ f(x) = x^{3} -1 x^{2} -2 x \text{ and } g(x) = -3x^{3} +3 x^{2} -x -2 \]The solution is \( 90.0 \), which is option A.\begin{enumerate}[label=\Alph*.]
\item \( (f \circ g)(-1) \in [89, 92] \)

* This is the correct solution
\item \( (f \circ g)(-1) \in [-13, -8] \)

 Distractor 3: Corresponds to being slightly off from the solution.
\item \( (f \circ g)(-1) \in [81, 89] \)

 Distractor 2: Corresponds to being slightly off from the solution.
\item \( (f \circ g)(-1) \in [-2, 0] \)

 Distractor 1: Corresponds to reversing the composition.
\item \( \text{It is not possible to compose the two functions.} \)


\end{enumerate}

\textbf{General Comment:} $f$ composed with $g$ at $x$ means $f(g(x))$. The order matters!
}
\litem{
Subtract the following functions, then choose the domain of the resulting function from the list below.
\[ f(x) = \frac{3}{4x-23} \text{ and } g(x) = \frac{2}{4x-29} \]The solution is \( \text{ The domain is all Real numbers except } x = 5.75 \text{ and } x = 7.25 \), which is option D.\begin{enumerate}[label=\Alph*.]
\item \( \text{ The domain is all Real numbers except } x = a, \text{ where } a \in [6.67, 12.67] \)


\item \( \text{ The domain is all Real numbers greater than or equal to } x = a, \text{ where } a \in [-12, 1] \)


\item \( \text{ The domain is all Real numbers less than or equal to } x = a, \text{ where } a \in [0.4, 5.4] \)


\item \( \text{ The domain is all Real numbers except } x = a \text{ and } x = b, \text{ where } a \in [-2.25, 7.75] \text{ and } b \in [6.25, 10.25] \)


\item \( \text{ The domain is all Real numbers. } \)


\end{enumerate}

\textbf{General Comment:} The new domain is the intersection of the previous domains.
}
\litem{
Determine whether the function below is 1-1.
\[ f(x) = 18 x^2 + 15 x - 375 \]The solution is \( \text{no} \), which is option C.\begin{enumerate}[label=\Alph*.]
\item \( \text{No, because the domain of the function is not $(-\infty, \infty)$.} \)

Corresponds to believing 1-1 means the domain is all Real numbers.
\item \( \text{No, because there is an $x$-value that goes to 2 different $y$-values.} \)

Corresponds to the Vertical Line test, which checks if an expression is a function.
\item \( \text{No, because there is a $y$-value that goes to 2 different $x$-values.} \)

* This is the solution.
\item \( \text{Yes, the function is 1-1.} \)

Corresponds to believing the function passes the Horizontal Line test.
\item \( \text{No, because the range of the function is not $(-\infty, \infty)$.} \)

Corresponds to believing 1-1 means the range is all Real numbers.
\end{enumerate}

\textbf{General Comment:} There are only two valid options: The function is 1-1 OR No because there is a $y$-value that goes to 2 different $x$-values.
}
\end{enumerate}

\end{document}