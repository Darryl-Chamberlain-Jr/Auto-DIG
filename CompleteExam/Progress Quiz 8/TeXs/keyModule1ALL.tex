\documentclass{extbook}[14pt]
\usepackage{multicol, enumerate, enumitem, hyperref, color, soul, setspace, parskip, fancyhdr, amssymb, amsthm, amsmath, latexsym, units, mathtools}
\everymath{\displaystyle}
\usepackage[headsep=0.5cm,headheight=0cm, left=1 in,right= 1 in,top= 1 in,bottom= 1 in]{geometry}
\usepackage{dashrule}  % Package to use the command below to create lines between items
\newcommand{\litem}[1]{\item #1

\rule{\textwidth}{0.4pt}}
\pagestyle{fancy}
\lhead{}
\chead{Answer Key for Progress Quiz 8 Version ALL}
\rhead{}
\lfoot{5493-4176}
\cfoot{}
\rfoot{Summer C 2021}
\begin{document}
\textbf{This key should allow you to understand why you choose the option you did (beyond just getting a question right or wrong). \href{https://xronos.clas.ufl.edu/mac1105spring2020/courseDescriptionAndMisc/Exams/LearningFromResults}{More instructions on how to use this key can be found here}.}

\textbf{If you have a suggestion to make the keys better, \href{https://forms.gle/CZkbZmPbC9XALEE88}{please fill out the short survey here}.}

\textit{Note: This key is auto-generated and may contain issues and/or errors. The keys are reviewed after each exam to ensure grading is done accurately. If there are issues (like duplicate options), they are noted in the offline gradebook. The keys are a work-in-progress to give students as many resources to improve as possible.}

\rule{\textwidth}{0.4pt}

\begin{enumerate}\litem{
Simplify the expression below and choose the interval the simplification is contained within.
\[ 1 - 13 \div 17 * 20 - (14 * 8) \]The solution is \( -126.294 \), which is option B.\begin{enumerate}[label=\Alph*.]
\item \( [-228.35, -218.35] \)

 -226.353, which corresponds to not distributing a negative correctly.
\item \( [-131.29, -122.29] \)

* -126.294, which is the correct option.
\item \( [-111.04, -105.04] \)

 -111.038, which corresponds to an Order of Operations error: not reading left-to-right for multiplication/division.
\item \( [112.96, 115.96] \)

 112.962, which corresponds to not distributing addition and subtraction correctly.
\item \( \text{None of the above} \)

 You may have gotten this by making an unanticipated error. If you got a value that is not any of the others, please let the coordinator know so they can help you figure out what happened.
\end{enumerate}

\textbf{General Comment:} While you may remember (or were taught) PEMDAS is done in order, it is actually done as P/E/MD/AS. When we are at MD or AS, we read left to right.
}
\litem{
Choose the \textbf{smallest} set of Real numbers that the number below belongs to.
\[ \sqrt{\frac{19044}{36}} \]The solution is \( \text{Whole} \), which is option A.\begin{enumerate}[label=\Alph*.]
\item \( \text{Whole} \)

* This is the correct option!
\item \( \text{Not a Real number} \)

These are Nonreal Complex numbers \textbf{OR} things that are not numbers (e.g., dividing by 0).
\item \( \text{Irrational} \)

These cannot be written as a fraction of Integers.
\item \( \text{Integer} \)

These are the negative and positive counting numbers (..., -3, -2, -1, 0, 1, 2, 3, ...)
\item \( \text{Rational} \)

These are numbers that can be written as fraction of Integers (e.g., -2/3)
\end{enumerate}

\textbf{General Comment:} First, you \textbf{NEED} to simplify the expression. This question simplifies to $138$. 
 
 Be sure you look at the simplified fraction and not just the decimal expansion. Numbers such as 13, 17, and 19 provide \textbf{long but repeating/terminating decimal expansions!} 
 
 The only ways to *not* be a Real number are: dividing by 0 or taking the square root of a negative number. 
 
 Irrational numbers are more than just square root of 3: adding or subtracting values from square root of 3 is also irrational.
}
\litem{
Choose the \textbf{smallest} set of Real numbers that the number below belongs to.
\[ -\sqrt{\frac{28900}{289}} \]The solution is \( \text{Integer} \), which is option A.\begin{enumerate}[label=\Alph*.]
\item \( \text{Integer} \)

* This is the correct option!
\item \( \text{Rational} \)

These are numbers that can be written as fraction of Integers (e.g., -2/3)
\item \( \text{Irrational} \)

These cannot be written as a fraction of Integers.
\item \( \text{Not a Real number} \)

These are Nonreal Complex numbers \textbf{OR} things that are not numbers (e.g., dividing by 0).
\item \( \text{Whole} \)

These are the counting numbers with 0 (0, 1, 2, 3, ...)
\end{enumerate}

\textbf{General Comment:} First, you \textbf{NEED} to simplify the expression. This question simplifies to $-170$. 
 
 Be sure you look at the simplified fraction and not just the decimal expansion. Numbers such as 13, 17, and 19 provide \textbf{long but repeating/terminating decimal expansions!} 
 
 The only ways to *not* be a Real number are: dividing by 0 or taking the square root of a negative number. 
 
 Irrational numbers are more than just square root of 3: adding or subtracting values from square root of 3 is also irrational.
}
\litem{
Choose the \textbf{smallest} set of Complex numbers that the number below belongs to.
\[ \frac{-20}{-12}+64i^2 \]The solution is \( \text{Rational} \), which is option D.\begin{enumerate}[label=\Alph*.]
\item \( \text{Pure Imaginary} \)

This is a Complex number $(a+bi)$ that \textbf{only} has an imaginary part like $2i$.
\item \( \text{Nonreal Complex} \)

This is a Complex number $(a+bi)$ that is not Real (has $i$ as part of the number).
\item \( \text{Not a Complex Number} \)

This is not a number. The only non-Complex number we know is dividing by 0 as this is not a number!
\item \( \text{Rational} \)

* This is the correct option!
\item \( \text{Irrational} \)

These cannot be written as a fraction of Integers. Remember: $\pi$ is not an Integer!
\end{enumerate}

\textbf{General Comment:} Be sure to simplify $i^2 = -1$. This may remove the imaginary portion for your number. If you are having trouble, you may want to look at the \textit{Subgroups of the Real Numbers} section.
}
\litem{
Simplify the expression below into the form $a+bi$. Then, choose the intervals that $a$ and $b$ belong to.
\[ (10 + 2 i)(4 + 3 i) \]The solution is \( 34 + 38 i \), which is option B.\begin{enumerate}[label=\Alph*.]
\item \( a \in [42, 50] \text{ and } b \in [16, 26] \)

 $46 + 22 i$, which corresponds to adding a minus sign in the first term.
\item \( a \in [30, 37] \text{ and } b \in [35, 43] \)

* $34 + 38 i$, which is the correct option.
\item \( a \in [30, 37] \text{ and } b \in [-43, -36] \)

 $34 - 38 i$, which corresponds to adding a minus sign in both terms.
\item \( a \in [39, 44] \text{ and } b \in [5, 7] \)

 $40 + 6 i$, which corresponds to just multiplying the real terms to get the real part of the solution and the coefficients in the complex terms to get the complex part.
\item \( a \in [42, 50] \text{ and } b \in [-23, -19] \)

 $46 - 22 i$, which corresponds to adding a minus sign in the second term.
\end{enumerate}

\textbf{General Comment:} You can treat $i$ as a variable and distribute. Just remember that $i^2=-1$, so you can continue to reduce after you distribute.
}
\litem{
Simplify the expression below into the form $a+bi$. Then, choose the intervals that $a$ and $b$ belong to.
\[ \frac{63 - 33 i}{4 - 5 i} \]The solution is \( 10.17  + 4.46 i \), which is option C.\begin{enumerate}[label=\Alph*.]
\item \( a \in [416, 417.5] \text{ and } b \in [3.5, 5.5] \)

 $417.00  + 4.46 i$, which corresponds to forgetting to multiply the conjugate by the numerator and using a plus instead of a minus in the denominator.
\item \( a \in [9.5, 10.5] \text{ and } b \in [181.5, 184] \)

 $10.17  + 183.00 i$, which corresponds to forgetting to multiply the conjugate by the numerator.
\item \( a \in [9.5, 10.5] \text{ and } b \in [3.5, 5.5] \)

* $10.17  + 4.46 i$, which is the correct option.
\item \( a \in [1, 4] \text{ and } b \in [-11.5, -10.5] \)

 $2.12  - 10.90 i$, which corresponds to forgetting to multiply the conjugate by the numerator and not computing the conjugate correctly.
\item \( a \in [14.5, 16.5] \text{ and } b \in [5.5, 7.5] \)

 $15.75  + 6.60 i$, which corresponds to just dividing the first term by the first term and the second by the second.
\end{enumerate}

\textbf{General Comment:} Multiply the numerator and denominator by the *conjugate* of the denominator, then simplify. For example, if we have $2+3i$, the conjugate is $2-3i$.
}
\litem{
Simplify the expression below into the form $a+bi$. Then, choose the intervals that $a$ and $b$ belong to.
\[ (-9 - 4 i)(-6 + 2 i) \]The solution is \( 62 + 6 i \), which is option D.\begin{enumerate}[label=\Alph*.]
\item \( a \in [52, 58] \text{ and } b \in [-9.4, -7.8] \)

 $54 - 8 i$, which corresponds to just multiplying the real terms to get the real part of the solution and the coefficients in the complex terms to get the complex part.
\item \( a \in [46, 50] \text{ and } b \in [40.42, 42.03] \)

 $46 + 42 i$, which corresponds to adding a minus sign in the second term.
\item \( a \in [59, 64] \text{ and } b \in [-6.19, -5.59] \)

 $62 - 6 i$, which corresponds to adding a minus sign in both terms.
\item \( a \in [59, 64] \text{ and } b \in [4.87, 7.82] \)

* $62 + 6 i$, which is the correct option.
\item \( a \in [46, 50] \text{ and } b \in [-42.35, -41.86] \)

 $46 - 42 i$, which corresponds to adding a minus sign in the first term.
\end{enumerate}

\textbf{General Comment:} You can treat $i$ as a variable and distribute. Just remember that $i^2=-1$, so you can continue to reduce after you distribute.
}
\litem{
Simplify the expression below into the form $a+bi$. Then, choose the intervals that $a$ and $b$ belong to.
\[ \frac{-18 + 77 i}{6 - 3 i} \]The solution is \( -7.53  + 9.07 i \), which is option C.\begin{enumerate}[label=\Alph*.]
\item \( a \in [-339.5, -338] \text{ and } b \in [8.5, 10.5] \)

 $-339.00  + 9.07 i$, which corresponds to forgetting to multiply the conjugate by the numerator and using a plus instead of a minus in the denominator.
\item \( a \in [-4, -2] \text{ and } b \in [-26.5, -25] \)

 $-3.00  - 25.67 i$, which corresponds to just dividing the first term by the first term and the second by the second.
\item \( a \in [-9, -6] \text{ and } b \in [8.5, 10.5] \)

* $-7.53  + 9.07 i$, which is the correct option.
\item \( a \in [2, 3] \text{ and } b \in [11, 12] \)

 $2.73  + 11.47 i$, which corresponds to forgetting to multiply the conjugate by the numerator and not computing the conjugate correctly.
\item \( a \in [-9, -6] \text{ and } b \in [407.5, 409] \)

 $-7.53  + 408.00 i$, which corresponds to forgetting to multiply the conjugate by the numerator.
\end{enumerate}

\textbf{General Comment:} Multiply the numerator and denominator by the *conjugate* of the denominator, then simplify. For example, if we have $2+3i$, the conjugate is $2-3i$.
}
\litem{
Simplify the expression below and choose the interval the simplification is contained within.
\[ 20 - 2 \div 6 * 12 - (11 * 4) \]The solution is \( -28.000 \), which is option D.\begin{enumerate}[label=\Alph*.]
\item \( [-27.1, -23.8] \)

 -24.028, which corresponds to an Order of Operations error: not reading left-to-right for multiplication/division.
\item \( [19, 21.7] \)

 20.000, which corresponds to not distributing a negative correctly.
\item \( [61.4, 65.5] \)

 63.972, which corresponds to not distributing addition and subtraction correctly.
\item \( [-28.6, -25.1] \)

* -28.000, which is the correct option.
\item \( \text{None of the above} \)

 You may have gotten this by making an unanticipated error. If you got a value that is not any of the others, please let the coordinator know so they can help you figure out what happened.
\end{enumerate}

\textbf{General Comment:} While you may remember (or were taught) PEMDAS is done in order, it is actually done as P/E/MD/AS. When we are at MD or AS, we read left to right.
}
\litem{
Choose the \textbf{smallest} set of Complex numbers that the number below belongs to.
\[ \sqrt{\frac{450}{5}}+\sqrt{132} i \]The solution is \( \text{Nonreal Complex} \), which is option D.\begin{enumerate}[label=\Alph*.]
\item \( \text{Not a Complex Number} \)

This is not a number. The only non-Complex number we know is dividing by 0 as this is not a number!
\item \( \text{Rational} \)

These are numbers that can be written as fraction of Integers (e.g., -2/3 + 5)
\item \( \text{Pure Imaginary} \)

This is a Complex number $(a+bi)$ that \textbf{only} has an imaginary part like $2i$.
\item \( \text{Nonreal Complex} \)

* This is the correct option!
\item \( \text{Irrational} \)

These cannot be written as a fraction of Integers. Remember: $\pi$ is not an Integer!
\end{enumerate}

\textbf{General Comment:} Be sure to simplify $i^2 = -1$. This may remove the imaginary portion for your number. If you are having trouble, you may want to look at the \textit{Subgroups of the Real Numbers} section.
}
\litem{
Simplify the expression below and choose the interval the simplification is contained within.
\[ 8 - 6 \div 16 * 12 - (18 * 2) \]The solution is \( -32.500 \), which is option C.\begin{enumerate}[label=\Alph*.]
\item \( [-30.3, -28.4] \)

 -29.000, which corresponds to not distributing a negative correctly.
\item \( [43.6, 44.1] \)

 43.969, which corresponds to not distributing addition and subtraction correctly.
\item \( [-32.9, -30] \)

* -32.500, which is the correct option.
\item \( [-28.5, -25.6] \)

 -28.031, which corresponds to an Order of Operations error: not reading left-to-right for multiplication/division.
\item \( \text{None of the above} \)

 You may have gotten this by making an unanticipated error. If you got a value that is not any of the others, please let the coordinator know so they can help you figure out what happened.
\end{enumerate}

\textbf{General Comment:} While you may remember (or were taught) PEMDAS is done in order, it is actually done as P/E/MD/AS. When we are at MD or AS, we read left to right.
}
\litem{
Choose the \textbf{smallest} set of Real numbers that the number below belongs to.
\[ \sqrt{\frac{24}{0}} \]The solution is \( \text{Not a Real number} \), which is option B.\begin{enumerate}[label=\Alph*.]
\item \( \text{Whole} \)

These are the counting numbers with 0 (0, 1, 2, 3, ...)
\item \( \text{Not a Real number} \)

* This is the correct option!
\item \( \text{Integer} \)

These are the negative and positive counting numbers (..., -3, -2, -1, 0, 1, 2, 3, ...)
\item \( \text{Irrational} \)

These cannot be written as a fraction of Integers.
\item \( \text{Rational} \)

These are numbers that can be written as fraction of Integers (e.g., -2/3)
\end{enumerate}

\textbf{General Comment:} First, you \textbf{NEED} to simplify the expression. This question simplifies to $\sqrt{\frac{24}{0}}$. 
 
 Be sure you look at the simplified fraction and not just the decimal expansion. Numbers such as 13, 17, and 19 provide \textbf{long but repeating/terminating decimal expansions!} 
 
 The only ways to *not* be a Real number are: dividing by 0 or taking the square root of a negative number. 
 
 Irrational numbers are more than just square root of 3: adding or subtracting values from square root of 3 is also irrational.
}
\litem{
Choose the \textbf{smallest} set of Real numbers that the number below belongs to.
\[ \sqrt{\frac{-1716}{13}} \]The solution is \( \text{Not a Real number} \), which is option C.\begin{enumerate}[label=\Alph*.]
\item \( \text{Whole} \)

These are the counting numbers with 0 (0, 1, 2, 3, ...)
\item \( \text{Integer} \)

These are the negative and positive counting numbers (..., -3, -2, -1, 0, 1, 2, 3, ...)
\item \( \text{Not a Real number} \)

* This is the correct option!
\item \( \text{Rational} \)

These are numbers that can be written as fraction of Integers (e.g., -2/3)
\item \( \text{Irrational} \)

These cannot be written as a fraction of Integers.
\end{enumerate}

\textbf{General Comment:} First, you \textbf{NEED} to simplify the expression. This question simplifies to $\sqrt{132} i$. 
 
 Be sure you look at the simplified fraction and not just the decimal expansion. Numbers such as 13, 17, and 19 provide \textbf{long but repeating/terminating decimal expansions!} 
 
 The only ways to *not* be a Real number are: dividing by 0 or taking the square root of a negative number. 
 
 Irrational numbers are more than just square root of 3: adding or subtracting values from square root of 3 is also irrational.
}
\litem{
Choose the \textbf{smallest} set of Complex numbers that the number below belongs to.
\[ \sqrt{\frac{-1568}{14}}+\sqrt{0}i \]The solution is \( \text{Pure Imaginary} \), which is option B.\begin{enumerate}[label=\Alph*.]
\item \( \text{Irrational} \)

These cannot be written as a fraction of Integers. Remember: $\pi$ is not an Integer!
\item \( \text{Pure Imaginary} \)

* This is the correct option!
\item \( \text{Rational} \)

These are numbers that can be written as fraction of Integers (e.g., -2/3 + 5)
\item \( \text{Nonreal Complex} \)

This is a Complex number $(a+bi)$ that is not Real (has $i$ as part of the number).
\item \( \text{Not a Complex Number} \)

This is not a number. The only non-Complex number we know is dividing by 0 as this is not a number!
\end{enumerate}

\textbf{General Comment:} Be sure to simplify $i^2 = -1$. This may remove the imaginary portion for your number. If you are having trouble, you may want to look at the \textit{Subgroups of the Real Numbers} section.
}
\litem{
Simplify the expression below into the form $a+bi$. Then, choose the intervals that $a$ and $b$ belong to.
\[ (8 - 6 i)(-7 - 5 i) \]The solution is \( -86 + 2 i \), which is option E.\begin{enumerate}[label=\Alph*.]
\item \( a \in [-28, -18] \text{ and } b \in [81, 84] \)

 $-26 + 82 i$, which corresponds to adding a minus sign in the second term.
\item \( a \in [-87, -83] \text{ and } b \in [-4, 0] \)

 $-86 - 2 i$, which corresponds to adding a minus sign in both terms.
\item \( a \in [-56, -53] \text{ and } b \in [29, 33] \)

 $-56 + 30 i$, which corresponds to just multiplying the real terms to get the real part of the solution and the coefficients in the complex terms to get the complex part.
\item \( a \in [-28, -18] \text{ and } b \in [-85, -80] \)

 $-26 - 82 i$, which corresponds to adding a minus sign in the first term.
\item \( a \in [-87, -83] \text{ and } b \in [0, 5] \)

* $-86 + 2 i$, which is the correct option.
\end{enumerate}

\textbf{General Comment:} You can treat $i$ as a variable and distribute. Just remember that $i^2=-1$, so you can continue to reduce after you distribute.
}
\litem{
Simplify the expression below into the form $a+bi$. Then, choose the intervals that $a$ and $b$ belong to.
\[ \frac{-45 - 11 i}{3 - 4 i} \]The solution is \( -3.64  - 8.52 i \), which is option A.\begin{enumerate}[label=\Alph*.]
\item \( a \in [-5.5, -3.5] \text{ and } b \in [-9.5, -8] \)

* $-3.64  - 8.52 i$, which is the correct option.
\item \( a \in [-8, -6.5] \text{ and } b \in [5.5, 6.5] \)

 $-7.16  + 5.88 i$, which corresponds to forgetting to multiply the conjugate by the numerator and not computing the conjugate correctly.
\item \( a \in [-17, -14] \text{ and } b \in [1, 4] \)

 $-15.00  + 2.75 i$, which corresponds to just dividing the first term by the first term and the second by the second.
\item \( a \in [-5.5, -3.5] \text{ and } b \in [-213.5, -212.5] \)

 $-3.64  - 213.00 i$, which corresponds to forgetting to multiply the conjugate by the numerator.
\item \( a \in [-91.5, -90] \text{ and } b \in [-9.5, -8] \)

 $-91.00  - 8.52 i$, which corresponds to forgetting to multiply the conjugate by the numerator and using a plus instead of a minus in the denominator.
\end{enumerate}

\textbf{General Comment:} Multiply the numerator and denominator by the *conjugate* of the denominator, then simplify. For example, if we have $2+3i$, the conjugate is $2-3i$.
}
\litem{
Simplify the expression below into the form $a+bi$. Then, choose the intervals that $a$ and $b$ belong to.
\[ (-6 + 2 i)(-10 + 4 i) \]The solution is \( 52 - 44 i \), which is option E.\begin{enumerate}[label=\Alph*.]
\item \( a \in [58, 62] \text{ and } b \in [6, 15] \)

 $60 + 8 i$, which corresponds to just multiplying the real terms to get the real part of the solution and the coefficients in the complex terms to get the complex part.
\item \( a \in [66, 71] \text{ and } b \in [0, 5] \)

 $68 + 4 i$, which corresponds to adding a minus sign in the second term.
\item \( a \in [45, 55] \text{ and } b \in [37, 45] \)

 $52 + 44 i$, which corresponds to adding a minus sign in both terms.
\item \( a \in [66, 71] \text{ and } b \in [-7, -1] \)

 $68 - 4 i$, which corresponds to adding a minus sign in the first term.
\item \( a \in [45, 55] \text{ and } b \in [-45, -43] \)

* $52 - 44 i$, which is the correct option.
\end{enumerate}

\textbf{General Comment:} You can treat $i$ as a variable and distribute. Just remember that $i^2=-1$, so you can continue to reduce after you distribute.
}
\litem{
Simplify the expression below into the form $a+bi$. Then, choose the intervals that $a$ and $b$ belong to.
\[ \frac{-36 - 11 i}{6 - 2 i} \]The solution is \( -4.85  - 3.45 i \), which is option B.\begin{enumerate}[label=\Alph*.]
\item \( a \in [-5.99, -5.93] \text{ and } b \in [-1, 1] \)

 $-5.95  + 0.15 i$, which corresponds to forgetting to multiply the conjugate by the numerator and not computing the conjugate correctly.
\item \( a \in [-4.87, -4.85] \text{ and } b \in [-4.5, -3] \)

* $-4.85  - 3.45 i$, which is the correct option.
\item \( a \in [-6.01, -5.96] \text{ and } b \in [4, 6] \)

 $-6.00  + 5.50 i$, which corresponds to just dividing the first term by the first term and the second by the second.
\item \( a \in [-4.87, -4.85] \text{ and } b \in [-138.5, -137.5] \)

 $-4.85  - 138.00 i$, which corresponds to forgetting to multiply the conjugate by the numerator.
\item \( a \in [-194.02, -193.99] \text{ and } b \in [-4.5, -3] \)

 $-194.00  - 3.45 i$, which corresponds to forgetting to multiply the conjugate by the numerator and using a plus instead of a minus in the denominator.
\end{enumerate}

\textbf{General Comment:} Multiply the numerator and denominator by the *conjugate* of the denominator, then simplify. For example, if we have $2+3i$, the conjugate is $2-3i$.
}
\litem{
Simplify the expression below and choose the interval the simplification is contained within.
\[ 17 - 8 \div 15 * 9 - (13 * 5) \]The solution is \( -52.800 \), which is option A.\begin{enumerate}[label=\Alph*.]
\item \( [-54.8, -50.8] \)

* -52.800, which is the correct option.
\item \( [-50.06, -44.06] \)

 -48.059, which corresponds to an Order of Operations error: not reading left-to-right for multiplication/division.
\item \( [76.94, 82.94] \)

 81.941, which corresponds to not distributing addition and subtraction correctly.
\item \( [-8, -0] \)

 -4.000, which corresponds to not distributing a negative correctly.
\item \( \text{None of the above} \)

 You may have gotten this by making an unanticipated error. If you got a value that is not any of the others, please let the coordinator know so they can help you figure out what happened.
\end{enumerate}

\textbf{General Comment:} While you may remember (or were taught) PEMDAS is done in order, it is actually done as P/E/MD/AS. When we are at MD or AS, we read left to right.
}
\litem{
Choose the \textbf{smallest} set of Complex numbers that the number below belongs to.
\[ \sqrt{\frac{0}{49}}+\sqrt{5}i \]The solution is \( \text{Pure Imaginary} \), which is option C.\begin{enumerate}[label=\Alph*.]
\item \( \text{Rational} \)

These are numbers that can be written as fraction of Integers (e.g., -2/3 + 5)
\item \( \text{Nonreal Complex} \)

This is a Complex number $(a+bi)$ that is not Real (has $i$ as part of the number).
\item \( \text{Pure Imaginary} \)

* This is the correct option!
\item \( \text{Not a Complex Number} \)

This is not a number. The only non-Complex number we know is dividing by 0 as this is not a number!
\item \( \text{Irrational} \)

These cannot be written as a fraction of Integers. Remember: $\pi$ is not an Integer!
\end{enumerate}

\textbf{General Comment:} Be sure to simplify $i^2 = -1$. This may remove the imaginary portion for your number. If you are having trouble, you may want to look at the \textit{Subgroups of the Real Numbers} section.
}
\litem{
Simplify the expression below and choose the interval the simplification is contained within.
\[ 5 - 6 \div 16 * 19 - (4 * 18) \]The solution is \( -74.125 \), which is option C.\begin{enumerate}[label=\Alph*.]
\item \( [-114.25, -107.25] \)

 -110.250, which corresponds to not distributing a negative correctly.
\item \( [76.98, 80.98] \)

 76.980, which corresponds to not distributing addition and subtraction correctly.
\item \( [-78.12, -69.12] \)

* -74.125, which is the correct option.
\item \( [-69.02, -64.02] \)

 -67.020, which corresponds to an Order of Operations error: not reading left-to-right for multiplication/division.
\item \( \text{None of the above} \)

 You may have gotten this by making an unanticipated error. If you got a value that is not any of the others, please let the coordinator know so they can help you figure out what happened.
\end{enumerate}

\textbf{General Comment:} While you may remember (or were taught) PEMDAS is done in order, it is actually done as P/E/MD/AS. When we are at MD or AS, we read left to right.
}
\litem{
Choose the \textbf{smallest} set of Real numbers that the number below belongs to.
\[ -\sqrt{\frac{144}{49}} \]The solution is \( \text{Rational} \), which is option D.\begin{enumerate}[label=\Alph*.]
\item \( \text{Irrational} \)

These cannot be written as a fraction of Integers.
\item \( \text{Not a Real number} \)

These are Nonreal Complex numbers \textbf{OR} things that are not numbers (e.g., dividing by 0).
\item \( \text{Whole} \)

These are the counting numbers with 0 (0, 1, 2, 3, ...)
\item \( \text{Rational} \)

* This is the correct option!
\item \( \text{Integer} \)

These are the negative and positive counting numbers (..., -3, -2, -1, 0, 1, 2, 3, ...)
\end{enumerate}

\textbf{General Comment:} First, you \textbf{NEED} to simplify the expression. This question simplifies to $-\frac{12}{7}$. 
 
 Be sure you look at the simplified fraction and not just the decimal expansion. Numbers such as 13, 17, and 19 provide \textbf{long but repeating/terminating decimal expansions!} 
 
 The only ways to *not* be a Real number are: dividing by 0 or taking the square root of a negative number. 
 
 Irrational numbers are more than just square root of 3: adding or subtracting values from square root of 3 is also irrational.
}
\litem{
Choose the \textbf{smallest} set of Real numbers that the number below belongs to.
\[ -\sqrt{\frac{840}{14}} \]The solution is \( \text{Irrational} \), which is option B.\begin{enumerate}[label=\Alph*.]
\item \( \text{Not a Real number} \)

These are Nonreal Complex numbers \textbf{OR} things that are not numbers (e.g., dividing by 0).
\item \( \text{Irrational} \)

* This is the correct option!
\item \( \text{Rational} \)

These are numbers that can be written as fraction of Integers (e.g., -2/3)
\item \( \text{Integer} \)

These are the negative and positive counting numbers (..., -3, -2, -1, 0, 1, 2, 3, ...)
\item \( \text{Whole} \)

These are the counting numbers with 0 (0, 1, 2, 3, ...)
\end{enumerate}

\textbf{General Comment:} First, you \textbf{NEED} to simplify the expression. This question simplifies to $-\sqrt{60}$. 
 
 Be sure you look at the simplified fraction and not just the decimal expansion. Numbers such as 13, 17, and 19 provide \textbf{long but repeating/terminating decimal expansions!} 
 
 The only ways to *not* be a Real number are: dividing by 0 or taking the square root of a negative number. 
 
 Irrational numbers are more than just square root of 3: adding or subtracting values from square root of 3 is also irrational.
}
\litem{
Choose the \textbf{smallest} set of Complex numbers that the number below belongs to.
\[ \frac{\sqrt{143}}{6}+\sqrt{-4}i \]The solution is \( \text{Irrational} \), which is option B.\begin{enumerate}[label=\Alph*.]
\item \( \text{Rational} \)

These are numbers that can be written as fraction of Integers (e.g., -2/3 + 5)
\item \( \text{Irrational} \)

* This is the correct option!
\item \( \text{Pure Imaginary} \)

This is a Complex number $(a+bi)$ that \textbf{only} has an imaginary part like $2i$.
\item \( \text{Nonreal Complex} \)

This is a Complex number $(a+bi)$ that is not Real (has $i$ as part of the number).
\item \( \text{Not a Complex Number} \)

This is not a number. The only non-Complex number we know is dividing by 0 as this is not a number!
\end{enumerate}

\textbf{General Comment:} Be sure to simplify $i^2 = -1$. This may remove the imaginary portion for your number. If you are having trouble, you may want to look at the \textit{Subgroups of the Real Numbers} section.
}
\litem{
Simplify the expression below into the form $a+bi$. Then, choose the intervals that $a$ and $b$ belong to.
\[ (7 + 3 i)(-4 + 2 i) \]The solution is \( -34 + 2 i \), which is option A.\begin{enumerate}[label=\Alph*.]
\item \( a \in [-35, -33.3] \text{ and } b \in [0.3, 2.6] \)

* $-34 + 2 i$, which is the correct option.
\item \( a \in [-35, -33.3] \text{ and } b \in [-2.4, 1.6] \)

 $-34 - 2 i$, which corresponds to adding a minus sign in both terms.
\item \( a \in [-24.9, -18.6] \text{ and } b \in [-26.5, -22.1] \)

 $-22 - 26 i$, which corresponds to adding a minus sign in the second term.
\item \( a \in [-32, -27.2] \text{ and } b \in [5.6, 8.2] \)

 $-28 + 6 i$, which corresponds to just multiplying the real terms to get the real part of the solution and the coefficients in the complex terms to get the complex part.
\item \( a \in [-24.9, -18.6] \text{ and } b \in [24.6, 26.8] \)

 $-22 + 26 i$, which corresponds to adding a minus sign in the first term.
\end{enumerate}

\textbf{General Comment:} You can treat $i$ as a variable and distribute. Just remember that $i^2=-1$, so you can continue to reduce after you distribute.
}
\litem{
Simplify the expression below into the form $a+bi$. Then, choose the intervals that $a$ and $b$ belong to.
\[ \frac{27 - 44 i}{-5 + 8 i} \]The solution is \( -5.47  + 0.04 i \), which is option E.\begin{enumerate}[label=\Alph*.]
\item \( a \in [-487.16, -486.97] \text{ and } b \in [0, 0.7] \)

 $-487.00  + 0.04 i$, which corresponds to forgetting to multiply the conjugate by the numerator and using a plus instead of a minus in the denominator.
\item \( a \in [2.33, 2.62] \text{ and } b \in [4.55, 5.45] \)

 $2.44  + 4.90 i$, which corresponds to forgetting to multiply the conjugate by the numerator and not computing the conjugate correctly.
\item \( a \in [-5.45, -5.34] \text{ and } b \in [-5.55, -5.15] \)

 $-5.40  - 5.50 i$, which corresponds to just dividing the first term by the first term and the second by the second.
\item \( a \in [-5.51, -5.42] \text{ and } b \in [3.95, 4.45] \)

 $-5.47  + 4.00 i$, which corresponds to forgetting to multiply the conjugate by the numerator.
\item \( a \in [-5.51, -5.42] \text{ and } b \in [0, 0.7] \)

* $-5.47  + 0.04 i$, which is the correct option.
\end{enumerate}

\textbf{General Comment:} Multiply the numerator and denominator by the *conjugate* of the denominator, then simplify. For example, if we have $2+3i$, the conjugate is $2-3i$.
}
\litem{
Simplify the expression below into the form $a+bi$. Then, choose the intervals that $a$ and $b$ belong to.
\[ (-3 + 8 i)(2 - 6 i) \]The solution is \( 42 + 34 i \), which is option C.\begin{enumerate}[label=\Alph*.]
\item \( a \in [-7, -2] \text{ and } b \in [-48.7, -45.3] \)

 $-6 - 48 i$, which corresponds to just multiplying the real terms to get the real part of the solution and the coefficients in the complex terms to get the complex part.
\item \( a \in [-57, -50] \text{ and } b \in [-2.8, -1.7] \)

 $-54 - 2 i$, which corresponds to adding a minus sign in the second term.
\item \( a \in [33, 47] \text{ and } b \in [31.5, 36.6] \)

* $42 + 34 i$, which is the correct option.
\item \( a \in [-57, -50] \text{ and } b \in [-0.9, 2.4] \)

 $-54 + 2 i$, which corresponds to adding a minus sign in the first term.
\item \( a \in [33, 47] \text{ and } b \in [-35.5, -30.9] \)

 $42 - 34 i$, which corresponds to adding a minus sign in both terms.
\end{enumerate}

\textbf{General Comment:} You can treat $i$ as a variable and distribute. Just remember that $i^2=-1$, so you can continue to reduce after you distribute.
}
\litem{
Simplify the expression below into the form $a+bi$. Then, choose the intervals that $a$ and $b$ belong to.
\[ \frac{9 + 66 i}{-8 - 3 i} \]The solution is \( -3.70  - 6.86 i \), which is option D.\begin{enumerate}[label=\Alph*.]
\item \( a \in [-270.5, -269] \text{ and } b \in [-7, -5.9] \)

 $-270.00  - 6.86 i$, which corresponds to forgetting to multiply the conjugate by the numerator and using a plus instead of a minus in the denominator.
\item \( a \in [1.5, 3] \text{ and } b \in [-7.7, -7.25] \)

 $1.73  - 7.60 i$, which corresponds to forgetting to multiply the conjugate by the numerator and not computing the conjugate correctly.
\item \( a \in [-4.5, -3] \text{ and } b \in [-501.35, -500.4] \)

 $-3.70  - 501.00 i$, which corresponds to forgetting to multiply the conjugate by the numerator.
\item \( a \in [-4.5, -3] \text{ and } b \in [-7, -5.9] \)

* $-3.70  - 6.86 i$, which is the correct option.
\item \( a \in [-2.5, 0] \text{ and } b \in [-22.35, -21.45] \)

 $-1.12  - 22.00 i$, which corresponds to just dividing the first term by the first term and the second by the second.
\end{enumerate}

\textbf{General Comment:} Multiply the numerator and denominator by the *conjugate* of the denominator, then simplify. For example, if we have $2+3i$, the conjugate is $2-3i$.
}
\litem{
Simplify the expression below and choose the interval the simplification is contained within.
\[ 18 - 16^2 + 2 \div 8 * 14 \div 20 \]The solution is \( -237.825 \), which is option C.\begin{enumerate}[label=\Alph*.]
\item \( [-238.04, -237.96] \)

 -237.999, which corresponds to an Order of Operations error: not reading left-to-right for multiplication/division.
\item \( [273.92, 274.05] \)

 274.001, which corresponds to two Order of Operations errors.
\item \( [-237.85, -237.78] \)

* -237.825, this is the correct option
\item \( [274.15, 274.18] \)

 274.175, which corresponds to an Order of Operations error: multiplying by negative before squaring. For example: $(-3)^2 \neq -3^2$
\item \( \text{None of the above} \)

 You may have gotten this by making an unanticipated error. If you got a value that is not any of the others, please let the coordinator know so they can help you figure out what happened.
\end{enumerate}

\textbf{General Comment:} While you may remember (or were taught) PEMDAS is done in order, it is actually done as P/E/MD/AS. When we are at MD or AS, we read left to right.
}
\litem{
Choose the \textbf{smallest} set of Complex numbers that the number below belongs to.
\[ \sqrt{\frac{-490}{0}} i+\sqrt{176}i \]The solution is \( \text{Not a Complex Number} \), which is option C.\begin{enumerate}[label=\Alph*.]
\item \( \text{Pure Imaginary} \)

This is a Complex number $(a+bi)$ that \textbf{only} has an imaginary part like $2i$.
\item \( \text{Nonreal Complex} \)

This is a Complex number $(a+bi)$ that is not Real (has $i$ as part of the number).
\item \( \text{Not a Complex Number} \)

* This is the correct option!
\item \( \text{Rational} \)

These are numbers that can be written as fraction of Integers (e.g., -2/3 + 5)
\item \( \text{Irrational} \)

These cannot be written as a fraction of Integers. Remember: $\pi$ is not an Integer!
\end{enumerate}

\textbf{General Comment:} Be sure to simplify $i^2 = -1$. This may remove the imaginary portion for your number. If you are having trouble, you may want to look at the \textit{Subgroups of the Real Numbers} section.
}
\end{enumerate}

\end{document}