\documentclass{extbook}[14pt]
\usepackage{multicol, enumerate, enumitem, hyperref, color, soul, setspace, parskip, fancyhdr, amssymb, amsthm, amsmath, bbm, latexsym, units, mathtools}
\everymath{\displaystyle}
\usepackage[headsep=0.5cm,headheight=0cm, left=1 in,right= 1 in,top= 1 in,bottom= 1 in]{geometry}
\usepackage{dashrule}  % Package to use the command below to create lines between items
\newcommand{\litem}[1]{\item #1

\rule{\textwidth}{0.4pt}}
\pagestyle{fancy}
\lhead{}
\chead{Answer Key for Module8 Version C}
\rhead{}
\lfoot{1569-1502}
\cfoot{}
\rfoot{testing}
\begin{document}
\textbf{This key should allow you to understand why you choose the option you did (beyond just getting a question right or wrong). \href{https://xronos.clas.ufl.edu/mac1105spring2020/courseDescriptionAndMisc/Exams/LearningFromResults}{More instructions on how to use this key can be found here}.}

\textbf{If you have a suggestion to make the keys better, \href{https://forms.gle/CZkbZmPbC9XALEE88}{please fill out the short survey here}.}

\textit{Note: This key is auto-generated and may contain issues and/or errors. The keys are reviewed after each exam to ensure grading is done accurately. If there are issues (like duplicate options), they are noted in the offline gradebook. The keys are a work-in-progress to give students as many resources to improve as possible.}

\rule{\textwidth}{0.4pt}

\begin{enumerate}\litem{
Solve the equation for $x$ and choose the interval that contains the solution (if it exists).
\[ \log_{5}{(-3x+8)}+6 = 2 \]The solution is \( x = 2.666 \), which is option C.\begin{enumerate}[label=\Alph*.]
\item \( x \in [334.67, 341.67] \)

$x = 338.667$, which corresponds to reversing the base and exponent when converting and reversing the value with $x$.
\item \( x \in [-7.67, -2.67] \)

$x = -5.667$, which corresponds to ignoring the vertical shift when converting to exponential form.
\item \( x \in [-2.33, 7.67] \)

* $x = 2.666$, which is the correct option.
\item \( x \in [341, 349] \)

$x = 344.000$, which corresponds to reversing the base and exponent when converting.
\item \( \text{There is no Real solution to the equation.} \)

Corresponds to believing a negative coefficient within the log equation means there is no Real solution.
\end{enumerate}

\textbf{General Comment:} \textbf{General Comments:} First, get the equation in the form $\log_b{(cx+d)} = a$. Then, convert to $b^a = cx+d$ and solve.
}
\litem{
Which of the following intervals describes the Range of the function below?
\[ f(x) = -e^{x+6}+8 \]The solution is \( (-\infty, 8) \), which is option B.\begin{enumerate}[label=\Alph*.]
\item \( (a, \infty), a \in [-10, -2] \)

$(-8, \infty)$, which corresponds to using the negative vertical shift AND flipping the Range interval.
\item \( (-\infty, a), a \in [3, 13] \)

* $(-\infty, 8)$, which is the correct option.
\item \( [a, \infty), a \in [-10, -2] \)

$[-8, \infty)$, which corresponds to using the negative vertical shift AND flipping the Range interval AND including the endpoint.
\item \( (-\infty, a], a \in [3, 13] \)

$(-\infty, 8]$, which corresponds to including the endpoint.
\item \( (-\infty, \infty) \)

This corresponds to confusing range of an exponential function with the domain of an exponential function.
\end{enumerate}

\textbf{General Comment:} \textbf{General Comments}: Domain of a basic exponential function is $(-\infty, \infty)$ while the Range is $(0, \infty)$. We can shift these intervals [and even flip when $a<0$!] to find the new Domain/Range.
}
\litem{
 Solve the equation for $x$ and choose the interval that contains $x$ (if it exists).
\[  14 = \sqrt[5]{\frac{26}{e^{5x}}} \]The solution is \( x = -1.987 \), which is option C.\begin{enumerate}[label=\Alph*.]
\item \( x \in [-1.4, 0.6] \)

$x = -0.404$, which corresponds to treating any root as a square root.
\item \( x \in [-17.65, -11.65] \)

$x = -14.652$, which corresponds to thinking you don't need to take the natural log of both sides before reducing, as if the equation already had a natural log on the right side.
\item \( x \in [-5.99, -0.99] \)

* $x = -1.987$, which is the correct option.
\item \( \text{There is no Real solution to the equation.} \)

This corresponds to believing you cannot solve the equation.
\item \( \text{None of the above.} \)

This corresponds to making an unexpected error.
\end{enumerate}

\textbf{General Comment:} \textbf{General Comments}: After using the properties of logarithmic functions to break up the right-hand side, use $\ln(e) = 1$ to reduce the question to a linear function to solve. You can put $\ln(26)$ into a calculator if you are having trouble.
}
\litem{
Solve the equation for $x$ and choose the interval that contains the solution (if it exists).
\[ 3^{-2x+5} = \left(\frac{1}{49}\right)^{-3x-4} \]The solution is \( x = -0.726 \), which is option A.\begin{enumerate}[label=\Alph*.]
\item \( x \in [-0.73, 0.27] \)

* $x = -0.726$, which is the correct option.
\item \( x \in [-0.35, 5.65] \)

$x = 0.649$, which corresponds to distributing the $\ln(base)$ to the first term of the exponent only.
\item \( x \in [-12, -4] \)

$x = -9.000$, which corresponds to solving the numerators as equal while ignoring the bases are different.
\item \( x \in [8.07, 11.07] \)

$x = 10.074$, which corresponds to distributing the $\ln(base)$ to the second term of the exponent only.
\item \( \text{There is no Real solution to the equation.} \)

This corresponds to believing there is no solution since the bases are not powers of each other.
\end{enumerate}

\textbf{General Comment:} \textbf{General Comments:} This question was written so that the bases could not be written the same. You will need to take the log of both sides.
}
\litem{
Which of the following intervals describes the Domain of the function below?
\[ f(x) = \log_2{(x-6)}+9 \]The solution is \( (6, \infty) \), which is option B.\begin{enumerate}[label=\Alph*.]
\item \( (-\infty, a], a \in [-9.73, -8.65] \)

$(-\infty, -9]$, which corresponds to using the negative vertical shift AND including the endpoint AND flipping the domain.
\item \( (a, \infty), a \in [5.28, 6.86] \)

* $(6, \infty)$, which is the correct option.
\item \( (-\infty, a), a \in [-7.07, -4.61] \)

$(-\infty, -6)$, which corresponds to flipping the Domain. Remember: the general for is $a*\log(x-h)+k$, \textbf{where $a$ does not affect the domain}.
\item \( [a, \infty), a \in [7.74, 10.5] \)

$[9, \infty)$, which corresponds to using the vertical shift when shifting the Domain AND including the endpoint.
\item \( (-\infty, \infty) \)

This corresponds to thinking of the range of the log function (or the domain of the exponential function).
\end{enumerate}

\textbf{General Comment:} \textbf{General Comments}: The domain of a basic logarithmic function is $(0, \infty)$ and the Range is $(-\infty, \infty)$. We can use shifts when finding the Domain, but the Range will always be all Real numbers.
}
\litem{
 Solve the equation for $x$ and choose the interval that contains $x$ (if it exists).
\[  17 = \sqrt[7]{\frac{22}{e^{5x}}} \]The solution is \( x = -3.348 \), which is option A.\begin{enumerate}[label=\Alph*.]
\item \( x \in [-3.35, -2.35] \)

* $x = -3.348$, which is the correct option.
\item \( x \in [-28.42, -22.42] \)

$x = -24.418$, which corresponds to thinking you don't need to take the natural log of both sides before reducing, as if the equation already had a natural log on the right side.
\item \( x \in [-1.52, 3.48] \)

$x = -0.515$, which corresponds to treating any root as a square root.
\item \( \text{There is no Real solution to the equation.} \)

This corresponds to believing you cannot solve the equation.
\item \( \text{None of the above.} \)

This corresponds to making an unexpected error.
\end{enumerate}

\textbf{General Comment:} \textbf{General Comments}: After using the properties of logarithmic functions to break up the right-hand side, use $\ln(e) = 1$ to reduce the question to a linear function to solve. You can put $\ln(22)$ into a calculator if you are having trouble.
}
\litem{
Which of the following intervals describes the Range of the function below?
\[ f(x) = \log_2{(x+9)}+3 \]The solution is \( (\infty, \infty) \), which is option E.\begin{enumerate}[label=\Alph*.]
\item \( [a, \infty), a \in [4, 12] \)

$[9, \infty)$, which corresponds to using the negative of the horizontal shift AND including the endpoint.
\item \( (-\infty, a), a \in [-5, -1] \)

$(-\infty, -3)$, which corresponds to using the using the negative of vertical shift on $(0, \infty)$.
\item \( (-\infty, a), a \in [-1, 7] \)

$(-\infty, 3)$, which corresponds to using the vertical shift while the Range is $(-\infty, \infty)$.
\item \( [a, \infty), a \in [-15, -8] \)

$[3, \infty)$, which corresponds to using the flipped Domain AND including the endpoint.
\item \( (-\infty, \infty) \)

*This is the correct option.
\end{enumerate}

\textbf{General Comment:} \textbf{General Comments}: The domain of a basic logarithmic function is $(0, \infty)$ and the Range is $(-\infty, \infty)$. We can use shifts when finding the Domain, but the Range will always be all Real numbers.
}
\litem{
Solve the equation for $x$ and choose the interval that contains the solution (if it exists).
\[ 2^{-5x+3} = 49^{-3x-5} \]The solution is \( x = -2.624 \), which is option C.\begin{enumerate}[label=\Alph*.]
\item \( x \in [7.6, 11.3] \)

$x = 10.769$, which corresponds to distributing the $\ln(base)$ to the second term of the exponent only.
\item \( x \in [3.1, 4.2] \)

$x = 4.000$, which corresponds to solving the numerators as equal while ignoring the bases are different.
\item \( x \in [-3.2, -1.6] \)

* $x = -2.624$, which is the correct option.
\item \( x \in [-2.1, -0.3] \)

$x = -0.974$, which corresponds to distributing the $\ln(base)$ to the first term of the exponent only.
\item \( \text{There is no Real solution to the equation.} \)

This corresponds to believing there is no solution since the bases are not powers of each other.
\end{enumerate}

\textbf{General Comment:} \textbf{General Comments:} This question was written so that the bases could not be written the same. You will need to take the log of both sides.
}
\litem{
Which of the following intervals describes the Range of the function below?
\[ f(x) = -e^{x+6}+2 \]The solution is \( (-\infty, 2) \), which is option A.\begin{enumerate}[label=\Alph*.]
\item \( (-\infty, a), a \in [0, 7] \)

* $(-\infty, 2)$, which is the correct option.
\item \( (-\infty, a], a \in [0, 7] \)

$(-\infty, 2]$, which corresponds to including the endpoint.
\item \( (a, \infty), a \in [-3, -1] \)

$(-2, \infty)$, which corresponds to using the negative vertical shift AND flipping the Range interval.
\item \( [a, \infty), a \in [-3, -1] \)

$[-2, \infty)$, which corresponds to using the negative vertical shift AND flipping the Range interval AND including the endpoint.
\item \( (-\infty, \infty) \)

This corresponds to confusing range of an exponential function with the domain of an exponential function.
\end{enumerate}

\textbf{General Comment:} \textbf{General Comments}: Domain of a basic exponential function is $(-\infty, \infty)$ while the Range is $(0, \infty)$. We can shift these intervals [and even flip when $a<0$!] to find the new Domain/Range.
}
\litem{
Solve the equation for $x$ and choose the interval that contains the solution (if it exists).
\[ \log_{5}{(4x+6)}+5 = 2 \]The solution is \( x = -1.498 \), which is option A.\begin{enumerate}[label=\Alph*.]
\item \( x \in [-7, 1.1] \)

* $x = -1.498$, which is the correct option.
\item \( x \in [2.1, 5.9] \)

$x = 4.750$, which corresponds to ignoring the vertical shift when converting to exponential form.
\item \( x \in [-60.4, -58.6] \)

$x = -59.250$, which corresponds to reversing the base and exponent when converting and reversing the value with $x$.
\item \( x \in [-62.4, -61] \)

$x = -62.250$, which corresponds to reversing the base and exponent when converting.
\item \( \text{There is no Real solution to the equation.} \)

Corresponds to believing a negative coefficient within the log equation means there is no Real solution.
\end{enumerate}

\textbf{General Comment:} \textbf{General Comments:} First, get the equation in the form $\log_b{(cx+d)} = a$. Then, convert to $b^a = cx+d$ and solve.
}
\end{enumerate}

\end{document}