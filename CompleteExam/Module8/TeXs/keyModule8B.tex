\documentclass{extbook}[14pt]
\usepackage{multicol, enumerate, enumitem, hyperref, color, soul, setspace, parskip, fancyhdr, amssymb, amsthm, amsmath, bbm, latexsym, units, mathtools}
\everymath{\displaystyle}
\usepackage[headsep=0.5cm,headheight=0cm, left=1 in,right= 1 in,top= 1 in,bottom= 1 in]{geometry}
\usepackage{dashrule}  % Package to use the command below to create lines between items
\newcommand{\litem}[1]{\item #1

\rule{\textwidth}{0.4pt}}
\pagestyle{fancy}
\lhead{}
\chead{Answer Key for Module8 Version B}
\rhead{}
\lfoot{5107-4344}
\cfoot{}
\rfoot{Fall 2020}
\begin{document}
\textbf{This key should allow you to understand why you choose the option you did (beyond just getting a question right or wrong). \href{https://xronos.clas.ufl.edu/mac1105spring2020/courseDescriptionAndMisc/Exams/LearningFromResults}{More instructions on how to use this key can be found here}.}

\textbf{If you have a suggestion to make the keys better, \href{https://forms.gle/CZkbZmPbC9XALEE88}{please fill out the short survey here}.}

\textit{Note: This key is auto-generated and may contain issues and/or errors. The keys are reviewed after each exam to ensure grading is done accurately. If there are issues (like duplicate options), they are noted in the offline gradebook. The keys are a work-in-progress to give students as many resources to improve as possible.}

\rule{\textwidth}{0.4pt}

\begin{enumerate}\litem{
Which of the following intervals describes the Domain of the function below?
\[ f(x) = \log_2{(x+7)}-8 \]
The solution is \( (-7, \infty) \), which is option B.\begin{enumerate}[label=\Alph*.]
\item \( (-\infty, a), a \in [6.75, 7.62] \)

$(-\infty, 7)$, which corresponds to flipping the Domain. Remember: the general for is $a*\log(x-h)+k$, \textbf{where $a$ does not affect the domain}.
\item \( (a, \infty), a \in [-7.96, -6.29] \)

* $(-7, \infty)$, which is the correct option.
\item \( (-\infty, a], a \in [7.38, 8.33] \)

$(-\infty, 8]$, which corresponds to using the negative vertical shift AND including the endpoint AND flipping the domain.
\item \( [a, \infty), a \in [-8.69, -7.69] \)

$[-8, \infty)$, which corresponds to using the vertical shift when shifting the Domain AND including the endpoint.
\item \( (-\infty, \infty) \)

This corresponds to thinking of the range of the log function (or the domain of the exponential function).
\end{enumerate}

\textbf{General Comment:} \textbf{General Comments}: The domain of a basic logarithmic function is $(0, \infty)$ and the Range is $(-\infty, \infty)$. We can use shifts when finding the Domain, but the Range will always be all Real numbers.
}
\litem{
Solve the equation for $x$ and choose the interval that contains the solution (if it exists).
\[ \log_{4}{(4x+5)}+4 = 3 \]
The solution is \( x = -1.188 \), which is option B.\begin{enumerate}[label=\Alph*.]
\item \( x \in [1.21, 1.53] \)

$x = 1.500$, which corresponds to reversing the base and exponent when converting and reversing the value with $x$.
\item \( x \in [-1.24, -1.07] \)

* $x = -1.188$, which is the correct option.
\item \( x \in [14.47, 15.13] \)

$x = 14.750$, which corresponds to ignoring the vertical shift when converting to exponential form.
\item \( x \in [-1.05, -0.87] \)

$x = -1.000$, which corresponds to reversing the base and exponent when converting.
\item \( \text{There is no Real solution to the equation.} \)

Corresponds to believing a negative coefficient within the log equation means there is no Real solution.
\end{enumerate}

\textbf{General Comment:} \textbf{General Comments:} First, get the equation in the form $\log_b{(cx+d)} = a$. Then, convert to $b^a = cx+d$ and solve.
}
\litem{
Which of the following intervals describes the Range of the function below?
\[ f(x) = \log_2{(x-4)}+7 \]
The solution is \( (\infty, \infty) \), which is option E.\begin{enumerate}[label=\Alph*.]
\item \( (-\infty, a), a \in [5, 11] \)

$(-\infty, 7)$, which corresponds to using the vertical shift while the Range is $(-\infty, \infty)$.
\item \( [a, \infty), a \in [3, 6] \)

$[7, \infty)$, which corresponds to using the flipped Domain AND including the endpoint.
\item \( (-\infty, a), a \in [-7, -5] \)

$(-\infty, -7)$, which corresponds to using the using the negative of vertical shift on $(0, \infty)$.
\item \( [a, \infty), a \in [-6, 1] \)

$[-4, \infty)$, which corresponds to using the negative of the horizontal shift AND including the endpoint.
\item \( (-\infty, \infty) \)

*This is the correct option.
\end{enumerate}

\textbf{General Comment:} \textbf{General Comments}: The domain of a basic logarithmic function is $(0, \infty)$ and the Range is $(-\infty, \infty)$. We can use shifts when finding the Domain, but the Range will always be all Real numbers.
}
\litem{
 Solve the equation for $x$ and choose the interval that contains $x$ (if it exists).
\[  18 = \sqrt[3]{\frac{22}{e^{5x}}} \]
The solution is \( x = -1.116 \), which is option C.\begin{enumerate}[label=\Alph*.]
\item \( x \in [-0.6, 1.2] \)

$x = -0.538$, which corresponds to treating any root as a square root.
\item \( x \in [-12.1, -9.5] \)

$x = -11.418$, which corresponds to thinking you don't need to take the natural log of both sides before reducing, as if the equation already had a natural log on the right side.
\item \( x \in [-1.4, -1.1] \)

* $x = -1.116$, which is the correct option.
\item \( \text{There is no Real solution to the equation.} \)

This corresponds to believing you cannot solve the equation.
\item \( \text{None of the above.} \)

This corresponds to making an unexpected error.
\end{enumerate}

\textbf{General Comment:} \textbf{General Comments}: After using the properties of logarithmic functions to break up the right-hand side, use $\ln(e) = 1$ to reduce the question to a linear function to solve. You can put $\ln(22)$ into a calculator if you are having trouble.
}
\litem{
Which of the following intervals describes the Domain of the function below?
\[ f(x) = -e^{x-1}-5 \]
The solution is \( (-\infty, \infty) \), which is option E.\begin{enumerate}[label=\Alph*.]
\item \( (-\infty, a), a \in [-5, -4] \)

$(-\infty, -5)$, which corresponds to using the correct vertical shift *if we wanted the Range*.
\item \( (a, \infty), a \in [4, 7] \)

$(5, \infty)$, which corresponds to using the negative vertical shift AND flipping the Range interval.
\item \( (-\infty, a], a \in [-5, -4] \)

$(-\infty, -5]$, which corresponds to using the correct vertical shift *if we wanted the Range* AND including the endpoint.
\item \( [a, \infty), a \in [4, 7] \)

$[5, \infty)$, which corresponds to using the negative vertical shift AND flipping the Range interval AND including the endpoint.
\item \( (-\infty, \infty) \)

* This is the correct option.
\end{enumerate}

\textbf{General Comment:} \textbf{General Comments}: Domain of a basic exponential function is $(-\infty, \infty)$ while the Range is $(0, \infty)$. We can shift these intervals [and even flip when $a<0$!] to find the new Domain/Range.
}
\litem{
Solve the equation for $x$ and choose the interval that contains the solution (if it exists).
\[ 2^{5x+2} = \left(\frac{1}{9}\right)^{2x-5} \]
The solution is \( x = 1.221 \), which is option D.\begin{enumerate}[label=\Alph*.]
\item \( x \in [2.9, 4] \)

$x = 3.200$, which corresponds to distributing the $\ln(base)$ to the second term of the exponent only.
\item \( x \in [-1.7, 0.8] \)

$x = -0.891$, which corresponds to distributing the $\ln(base)$ to the first term of the exponent only.
\item \( x \in [-4.4, -1.8] \)

$x = -2.333$, which corresponds to solving the numerators as equal while ignoring the bases are different.
\item \( x \in [0.9, 3.1] \)

* $x = 1.221$, which is the correct option.
\item \( \text{There is no Real solution to the equation.} \)

This corresponds to believing there is no solution since the bases are not powers of each other.
\end{enumerate}

\textbf{General Comment:} \textbf{General Comments:} This question was written so that the bases could not be written the same. You will need to take the log of both sides.
}
\litem{
Which of the following intervals describes the Range of the function below?
\[ f(x) = -e^{x+2}+9 \]
The solution is \( (-\infty, 9) \), which is option B.\begin{enumerate}[label=\Alph*.]
\item \( (-\infty, a], a \in [5, 11] \)

$(-\infty, 9]$, which corresponds to including the endpoint.
\item \( (-\infty, a), a \in [5, 11] \)

* $(-\infty, 9)$, which is the correct option.
\item \( [a, \infty), a \in [-13, -7] \)

$[-9, \infty)$, which corresponds to using the negative vertical shift AND flipping the Range interval AND including the endpoint.
\item \( (a, \infty), a \in [-13, -7] \)

$(-9, \infty)$, which corresponds to using the negative vertical shift AND flipping the Range interval.
\item \( (-\infty, \infty) \)

This corresponds to confusing range of an exponential function with the domain of an exponential function.
\end{enumerate}

\textbf{General Comment:} \textbf{General Comments}: Domain of a basic exponential function is $(-\infty, \infty)$ while the Range is $(0, \infty)$. We can shift these intervals [and even flip when $a<0$!] to find the new Domain/Range.
}
\litem{
 Solve the equation for $x$ and choose the interval that contains $x$ (if it exists).
\[  6 = \ln{\sqrt[5]{\frac{24}{e^{6x}}}} \]
The solution is \( x = -4.47, \text{ which does not fit in any of the interval options.} \), which is option E.\begin{enumerate}[label=\Alph*.]
\item \( x \in [-1.8, -1.3] \)

$x = -1.470$, which corresponds to treating any root as a square root.
\item \( x \in [-3.5, -1.8] \)

$x = -2.023$, which corresponds to thinking you need to take the natural log of the left side before reducing.
\item \( x \in [2.3, 5.8] \)

$x = 4.470$, which is the negative of the correct solution.
\item \( \text{There is no Real solution to the equation.} \)

This corresponds to believing you cannot solve the equation.
\item \( \text{None of the above.} \)

*$x = -4.470$ is the correct solution and does not fit in any of the other intervals.
\end{enumerate}

\textbf{General Comment:} \textbf{General Comments}: After using the properties of logarithmic functions to break up the right-hand side, use $\ln(e) = 1$ to reduce the question to a linear function to solve. You can put $\ln(24)$ into a calculator if you are having trouble.
}
\litem{
Solve the equation for $x$ and choose the interval that contains the solution (if it exists).
\[ 2^{5x-2} = \left(\frac{1}{125}\right)^{4x+4} \]
The solution is \( x = -0.787 \), which is option C.\begin{enumerate}[label=\Alph*.]
\item \( x \in [4.4, 7] \)

$x = 6.000$, which corresponds to solving the numerators as equal while ignoring the bases are different.
\item \( x \in [-20.1, -16.4] \)

$x = -17.927$, which corresponds to distributing the $\ln(base)$ to the second term of the exponent only.
\item \( x \in [-1.7, -0.6] \)

* $x = -0.787$, which is the correct option.
\item \( x \in [-0.5, 3.1] \)

$x = 0.263$, which corresponds to distributing the $\ln(base)$ to the first term of the exponent only.
\item \( \text{There is no Real solution to the equation.} \)

This corresponds to believing there is no solution since the bases are not powers of each other.
\end{enumerate}

\textbf{General Comment:} \textbf{General Comments:} This question was written so that the bases could not be written the same. You will need to take the log of both sides.
}
\litem{
Solve the equation for $x$ and choose the interval that contains the solution (if it exists).
\[ \log_{4}{(-3x+5)}+4 = 3 \]
The solution is \( x = 1.583 \), which is option B.\begin{enumerate}[label=\Alph*.]
\item \( x \in [1.13, 1.53] \)

$x = 1.333$, which corresponds to reversing the base and exponent when converting.
\item \( x \in [1.43, 2.41] \)

* $x = 1.583$, which is the correct option.
\item \( x \in [-20.9, -19.02] \)

$x = -19.667$, which corresponds to ignoring the vertical shift when converting to exponential form.
\item \( x \in [-2.25, -1.33] \)

$x = -2.000$, which corresponds to reversing the base and exponent when converting and reversing the value with $x$.
\item \( \text{There is no Real solution to the equation.} \)

Corresponds to believing a negative coefficient within the log equation means there is no Real solution.
\end{enumerate}

\textbf{General Comment:} \textbf{General Comments:} First, get the equation in the form $\log_b{(cx+d)} = a$. Then, convert to $b^a = cx+d$ and solve.
}
\end{enumerate}

\end{document}