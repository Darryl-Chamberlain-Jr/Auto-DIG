\documentclass{extbook}[14pt]
\usepackage{multicol, enumerate, enumitem, hyperref, color, soul, setspace, parskip, fancyhdr, amssymb, amsthm, amsmath, bbm, latexsym, units, mathtools}
\everymath{\displaystyle}
\usepackage[headsep=0.5cm,headheight=0cm, left=1 in,right= 1 in,top= 1 in,bottom= 1 in]{geometry}
\usepackage{dashrule}  % Package to use the command below to create lines between items
\newcommand{\litem}[1]{\item #1

\rule{\textwidth}{0.4pt}}
\pagestyle{fancy}
\lhead{}
\chead{Answer Key for Progress Quiz 1 Version C}
\rhead{}
\lfoot{2654-6976}
\cfoot{}
\rfoot{Fall 2020}
\begin{document}
\textbf{This key should allow you to understand why you choose the option you did (beyond just getting a question right or wrong). \href{https://xronos.clas.ufl.edu/mac1105spring2020/courseDescriptionAndMisc/Exams/LearningFromResults}{More instructions on how to use this key can be found here}.}

\textbf{If you have a suggestion to make the keys better, \href{https://forms.gle/CZkbZmPbC9XALEE88}{please fill out the short survey here}.}

\textit{Note: This key is auto-generated and may contain issues and/or errors. The keys are reviewed after each exam to ensure grading is done accurately. If there are issues (like duplicate options), they are noted in the offline gradebook. The keys are a work-in-progress to give students as many resources to improve as possible.}

\rule{\textwidth}{0.4pt}

\begin{enumerate}\litem{
Simplify the expression below and choose the interval the simplification is contained within.
\[ 6 - 18^2 + 10 \div 13 * 7 \div 20 \]
The solution is \( -317.731 \), which is option D.\begin{enumerate}[label=\Alph*.]
\item \( [330.25, 330.46] \)

 330.269, which corresponds to an Order of Operations error: multiplying by negative before squaring. For example: $(-3)^2 \neq -3^2$
\item \( [329.72, 330.11] \)

 330.005, which corresponds to two Order of Operations errors.
\item \( [-318.08, -317.95] \)

 -317.995, which corresponds to an Order of Operations error: not reading left-to-right for multiplication/division.
\item \( [-317.81, -317.64] \)

* -317.731, this is the correct option
\item \( \text{None of the above} \)

 You may have gotten this by making an unanticipated error. If you got a value that is not any of the others, please let the coordinator know so they can help you figure out what happened.
\end{enumerate}

\textbf{General Comment:} While you may remember (or were taught) PEMDAS is done in order, it is actually done as P/E/MD/AS. When we are at MD or AS, we read left to right.
}
\litem{
Simplify the expression below into the form $a+bi$. Then, choose the intervals that $a$ and $b$ belong to.
\[ (-6 + 7 i)(-2 + 10 i) \]
The solution is \( -58 - 74 i \), which is option A.\begin{enumerate}[label=\Alph*.]
\item \( a \in [-61, -54] \text{ and } b \in [-75.5, -71.7] \)

* $-58 - 74 i$, which is the correct option.
\item \( a \in [-61, -54] \text{ and } b \in [71.7, 76.6] \)

 $-58 + 74 i$, which corresponds to adding a minus sign in both terms.
\item \( a \in [79, 83] \text{ and } b \in [45.2, 48.6] \)

 $82 + 46 i$, which corresponds to adding a minus sign in the second term.
\item \( a \in [79, 83] \text{ and } b \in [-47.3, -44.2] \)

 $82 - 46 i$, which corresponds to adding a minus sign in the first term.
\item \( a \in [12, 14] \text{ and } b \in [68.7, 73.3] \)

 $12 + 70 i$, which corresponds to just multiplying the real terms to get the real part of the solution and the coefficients in the complex terms to get the complex part.
\end{enumerate}

\textbf{General Comment:} You can treat $i$ as a variable and distribute. Just remember that $i^2=-1$, so you can continue to reduce after you distribute.
}
\litem{
Choose the \textbf{smallest} set of Complex numbers that the number below belongs to.
\[ \sqrt{\frac{640}{8}}+\sqrt{77} i \]
The solution is \( \text{Nonreal Complex} \), which is option D.\begin{enumerate}[label=\Alph*.]
\item \( \text{Irrational} \)

These cannot be written as a fraction of Integers. Remember: $\pi$ is not an Integer!
\item \( \text{Pure Imaginary} \)

This is a Complex number $(a+bi)$ that \textbf{only} has an imaginary part like $2i$.
\item \( \text{Rational} \)

These are numbers that can be written as fraction of Integers (e.g., -2/3 + 5)
\item \( \text{Nonreal Complex} \)

* This is the correct option!
\item \( \text{Not a Complex Number} \)

This is not a number. The only non-Complex number we know is dividing by 0 as this is not a number!
\end{enumerate}

\textbf{General Comment:} Be sure to simplify $i^2 = -1$. This may remove the imaginary portion for your number. If you are having trouble, you may want to look at the \textit{Subgroups of the Real Numbers} section.
}
\litem{
Choose the \textbf{smallest} set of Real numbers that the number below belongs to.
\[ -\sqrt{\frac{5929}{49}} \]
The solution is \( \text{Integer} \), which is option E.\begin{enumerate}[label=\Alph*.]
\item \( \text{Not a Real number} \)

These are Nonreal Complex numbers \textbf{OR} things that are not numbers (e.g., dividing by 0).
\item \( \text{Irrational} \)

These cannot be written as a fraction of Integers.
\item \( \text{Rational} \)

These are numbers that can be written as fraction of Integers (e.g., -2/3)
\item \( \text{Whole} \)

These are the counting numbers with 0 (0, 1, 2, 3, ...)
\item \( \text{Integer} \)

* This is the correct option!
\end{enumerate}

\textbf{General Comment:} First, you \textbf{NEED} to simplify the expression. This question simplifies to $-77$. 
 
 Be sure you look at the simplified fraction and not just the decimal expansion. Numbers such as 13, 17, and 19 provide \textbf{long but repeating/terminating decimal expansions!} 
 
 The only ways to *not* be a Real number are: dividing by 0 or taking the square root of a negative number. 
 
 Irrational numbers are more than just square root of 3: adding or subtracting values from square root of 3 is also irrational.
}
\litem{
Simplify the expression below into the form $a+bi$. Then, choose the intervals that $a$ and $b$ belong to.
\[ \frac{27 - 77 i}{2 + 8 i} \]
The solution is \( -8.26  - 5.44 i \), which is option E.\begin{enumerate}[label=\Alph*.]
\item \( a \in [-563, -561.5] \text{ and } b \in [-6, -4] \)

 $-562.00  - 5.44 i$, which corresponds to forgetting to multiply the conjugate by the numerator and using a plus instead of a minus in the denominator.
\item \( a \in [9, 10.5] \text{ and } b \in [-0.5, 1] \)

 $9.85  + 0.91 i$, which corresponds to forgetting to multiply the conjugate by the numerator and not computing the conjugate correctly.
\item \( a \in [-9.5, -7.5] \text{ and } b \in [-370.5, -369.5] \)

 $-8.26  - 370.00 i$, which corresponds to forgetting to multiply the conjugate by the numerator.
\item \( a \in [12, 14] \text{ and } b \in [-10, -9] \)

 $13.50  - 9.62 i$, which corresponds to just dividing the first term by the first term and the second by the second.
\item \( a \in [-9.5, -7.5] \text{ and } b \in [-6, -4] \)

* $-8.26  - 5.44 i$, which is the correct option.
\end{enumerate}

\textbf{General Comment:} Multiply the numerator and denominator by the *conjugate* of the denominator, then simplify. For example, if we have $2+3i$, the conjugate is $2-3i$.
}
\litem{
Simplify the expression below and choose the interval the simplification is contained within.
\[ 13 - 2^2 + 7 \div 17 * 16 \div 6 \]
The solution is \( 10.098 \), which is option D.\begin{enumerate}[label=\Alph*.]
\item \( [7.73, 9.14] \)

 9.004, which corresponds to an Order of Operations error: not reading left-to-right for multiplication/division.
\item \( [16.66, 17.49] \)

 17.004, which corresponds to two Order of Operations errors.
\item \( [17.06, 18.61] \)

 18.098, which corresponds to an Order of Operations error: multiplying by negative before squaring. For example: $(-3)^2 \neq -3^2$
\item \( [9.09, 10.8] \)

* 10.098, this is the correct option
\item \( \text{None of the above} \)

 You may have gotten this by making an unanticipated error. If you got a value that is not any of the others, please let the coordinator know so they can help you figure out what happened.
\end{enumerate}

\textbf{General Comment:} While you may remember (or were taught) PEMDAS is done in order, it is actually done as P/E/MD/AS. When we are at MD or AS, we read left to right.
}
\litem{
Choose the \textbf{smallest} set of Complex numbers that the number below belongs to.
\[ \sqrt{\frac{-1260}{0}} i+\sqrt{234}i \]
The solution is \( \text{Not a Complex Number} \), which is option B.\begin{enumerate}[label=\Alph*.]
\item \( \text{Pure Imaginary} \)

This is a Complex number $(a+bi)$ that \textbf{only} has an imaginary part like $2i$.
\item \( \text{Not a Complex Number} \)

* This is the correct option!
\item \( \text{Nonreal Complex} \)

This is a Complex number $(a+bi)$ that is not Real (has $i$ as part of the number).
\item \( \text{Irrational} \)

These cannot be written as a fraction of Integers. Remember: $\pi$ is not an Integer!
\item \( \text{Rational} \)

These are numbers that can be written as fraction of Integers (e.g., -2/3 + 5)
\end{enumerate}

\textbf{General Comment:} Be sure to simplify $i^2 = -1$. This may remove the imaginary portion for your number. If you are having trouble, you may want to look at the \textit{Subgroups of the Real Numbers} section.
}
\litem{
Choose the \textbf{smallest} set of Real numbers that the number below belongs to.
\[ -\sqrt{\frac{924}{11}} \]
The solution is \( \text{Irrational} \), which is option D.\begin{enumerate}[label=\Alph*.]
\item \( \text{Not a Real number} \)

These are Nonreal Complex numbers \textbf{OR} things that are not numbers (e.g., dividing by 0).
\item \( \text{Integer} \)

These are the negative and positive counting numbers (..., -3, -2, -1, 0, 1, 2, 3, ...)
\item \( \text{Rational} \)

These are numbers that can be written as fraction of Integers (e.g., -2/3)
\item \( \text{Irrational} \)

* This is the correct option!
\item \( \text{Whole} \)

These are the counting numbers with 0 (0, 1, 2, 3, ...)
\end{enumerate}

\textbf{General Comment:} First, you \textbf{NEED} to simplify the expression. This question simplifies to $-\sqrt{84}$. 
 
 Be sure you look at the simplified fraction and not just the decimal expansion. Numbers such as 13, 17, and 19 provide \textbf{long but repeating/terminating decimal expansions!} 
 
 The only ways to *not* be a Real number are: dividing by 0 or taking the square root of a negative number. 
 
 Irrational numbers are more than just square root of 3: adding or subtracting values from square root of 3 is also irrational.
}
\litem{
Simplify the expression below into the form $a+bi$. Then, choose the intervals that $a$ and $b$ belong to.
\[ (2 + 3 i)(7 - 10 i) \]
The solution is \( 44 + i \), which is option E.\begin{enumerate}[label=\Alph*.]
\item \( a \in [6, 18] \text{ and } b \in [-34, -25] \)

 $14 - 30 i$, which corresponds to just multiplying the real terms to get the real part of the solution and the coefficients in the complex terms to get the complex part.
\item \( a \in [40, 50] \text{ and } b \in [-5, 0] \)

 $44 - i$, which corresponds to adding a minus sign in both terms.
\item \( a \in [-17, -14] \text{ and } b \in [37, 42] \)

 $-16 + 41 i$, which corresponds to adding a minus sign in the second term.
\item \( a \in [-17, -14] \text{ and } b \in [-45, -38] \)

 $-16 - 41 i$, which corresponds to adding a minus sign in the first term.
\item \( a \in [40, 50] \text{ and } b \in [1, 3] \)

* $44 + i$, which is the correct option.
\end{enumerate}

\textbf{General Comment:} You can treat $i$ as a variable and distribute. Just remember that $i^2=-1$, so you can continue to reduce after you distribute.
}
\litem{
Simplify the expression below into the form $a+bi$. Then, choose the intervals that $a$ and $b$ belong to.
\[ \frac{-54 + 44 i}{-8 - 7 i} \]
The solution is \( 1.10  - 6.46 i \), which is option E.\begin{enumerate}[label=\Alph*.]
\item \( a \in [6.7, 7] \text{ and } b \in [-6.34, -6.04] \)

 $6.75  - 6.29 i$, which corresponds to just dividing the first term by the first term and the second by the second.
\item \( a \in [0.25, 1.35] \text{ and } b \in [-730.31, -729.96] \)

 $1.10  - 730.00 i$, which corresponds to forgetting to multiply the conjugate by the numerator.
\item \( a \in [123.15, 124.35] \text{ and } b \in [-6.62, -6.4] \)

 $124.00  - 6.46 i$, which corresponds to forgetting to multiply the conjugate by the numerator and using a plus instead of a minus in the denominator.
\item \( a \in [6.35, 6.7] \text{ and } b \in [0.12, 0.48] \)

 $6.55  + 0.23 i$, which corresponds to forgetting to multiply the conjugate by the numerator and not computing the conjugate correctly.
\item \( a \in [0.25, 1.35] \text{ and } b \in [-6.62, -6.4] \)

* $1.10  - 6.46 i$, which is the correct option.
\end{enumerate}

\textbf{General Comment:} Multiply the numerator and denominator by the *conjugate* of the denominator, then simplify. For example, if we have $2+3i$, the conjugate is $2-3i$.
}
\end{enumerate}

\end{document}