\documentclass{extbook}[14pt]
\usepackage{multicol, enumerate, enumitem, hyperref, color, soul, setspace, parskip, fancyhdr, amssymb, amsthm, amsmath, latexsym, units, mathtools}
\everymath{\displaystyle}
\usepackage[headsep=0.5cm,headheight=0cm, left=1 in,right= 1 in,top= 1 in,bottom= 1 in]{geometry}
\usepackage{dashrule}  % Package to use the command below to create lines between items
\newcommand{\litem}[1]{\item #1

\rule{\textwidth}{0.4pt}}
\pagestyle{fancy}
\lhead{}
\chead{Answer Key for Progress Quiz 1 Version C}
\rhead{}
\lfoot{3629-3146}
\cfoot{}
\rfoot{Summer C 2021}
\begin{document}
\textbf{This key should allow you to understand why you choose the option you did (beyond just getting a question right or wrong). \href{https://xronos.clas.ufl.edu/mac1105spring2020/courseDescriptionAndMisc/Exams/LearningFromResults}{More instructions on how to use this key can be found here}.}

\textbf{If you have a suggestion to make the keys better, \href{https://forms.gle/CZkbZmPbC9XALEE88}{please fill out the short survey here}.}

\textit{Note: This key is auto-generated and may contain issues and/or errors. The keys are reviewed after each exam to ensure grading is done accurately. If there are issues (like duplicate options), they are noted in the offline gradebook. The keys are a work-in-progress to give students as many resources to improve as possible.}

\rule{\textwidth}{0.4pt}

\begin{enumerate}\litem{
Simplify the expression below into the form $a+bi$. Then, choose the intervals that $a$ and $b$ belong to.
\[ \frac{-72 - 77 i}{-2 - 5 i} \]The solution is \( 18.24  - 7.10 i \), which is option D.\begin{enumerate}[label=\Alph*.]
\item \( a \in [17.5, 19] \text{ and } b \in [-207.5, -205] \)

 $18.24  - 206.00 i$, which corresponds to forgetting to multiply the conjugate by the numerator.
\item \( a \in [-10, -7.5] \text{ and } b \in [17, 19] \)

 $-8.31  + 17.72 i$, which corresponds to forgetting to multiply the conjugate by the numerator and not computing the conjugate correctly.
\item \( a \in [528, 530] \text{ and } b \in [-8.5, -5] \)

 $529.00  - 7.10 i$, which corresponds to forgetting to multiply the conjugate by the numerator and using a plus instead of a minus in the denominator.
\item \( a \in [17.5, 19] \text{ and } b \in [-8.5, -5] \)

* $18.24  - 7.10 i$, which is the correct option.
\item \( a \in [35.5, 36.5] \text{ and } b \in [15, 16] \)

 $36.00  + 15.40 i$, which corresponds to just dividing the first term by the first term and the second by the second.
\end{enumerate}

\textbf{General Comment:} Multiply the numerator and denominator by the *conjugate* of the denominator, then simplify. For example, if we have $2+3i$, the conjugate is $2-3i$.
}
\litem{
Simplify the expression below and choose the interval the simplification is contained within.
\[ 3 - 2^2 + 17 \div 4 * 14 \div 20 \]The solution is \( 1.975 \), which is option A.\begin{enumerate}[label=\Alph*.]
\item \( [1.2, 4.2] \)

* 1.975, this is the correct option
\item \( [-4.9, -0.1] \)

 -0.985, which corresponds to an Order of Operations error: not reading left-to-right for multiplication/division.
\item \( [4.6, 9.2] \)

 7.015, which corresponds to two Order of Operations errors.
\item \( [9.5, 10.2] \)

 9.975, which corresponds to an Order of Operations error: multiplying by negative before squaring. For example: $(-3)^2 \neq -3^2$
\item \( \text{None of the above} \)

 You may have gotten this by making an unanticipated error. If you got a value that is not any of the others, please let the coordinator know so they can help you figure out what happened.
\end{enumerate}

\textbf{General Comment:} While you may remember (or were taught) PEMDAS is done in order, it is actually done as P/E/MD/AS. When we are at MD or AS, we read left to right.
}
\litem{
Choose the \textbf{smallest} set of Real numbers that the number below belongs to.
\[ -\sqrt{\frac{2156}{14}} \]The solution is \( \text{Irrational} \), which is option C.\begin{enumerate}[label=\Alph*.]
\item \( \text{Integer} \)

These are the negative and positive counting numbers (..., -3, -2, -1, 0, 1, 2, 3, ...)
\item \( \text{Not a Real number} \)

These are Nonreal Complex numbers \textbf{OR} things that are not numbers (e.g., dividing by 0).
\item \( \text{Irrational} \)

* This is the correct option!
\item \( \text{Rational} \)

These are numbers that can be written as fraction of Integers (e.g., -2/3)
\item \( \text{Whole} \)

These are the counting numbers with 0 (0, 1, 2, 3, ...)
\end{enumerate}

\textbf{General Comment:} First, you \textbf{NEED} to simplify the expression. This question simplifies to $-\sqrt{154}$. 
 
 Be sure you look at the simplified fraction and not just the decimal expansion. Numbers such as 13, 17, and 19 provide \textbf{long but repeating/terminating decimal expansions!} 
 
 The only ways to *not* be a Real number are: dividing by 0 or taking the square root of a negative number. 
 
 Irrational numbers are more than just square root of 3: adding or subtracting values from square root of 3 is also irrational.
}
\litem{
Choose the \textbf{smallest} set of Complex numbers that the number below belongs to.
\[ \sqrt{\frac{1872}{8}}+4i^2 \]The solution is \( \text{Irrational} \), which is option D.\begin{enumerate}[label=\Alph*.]
\item \( \text{Not a Complex Number} \)

This is not a number. The only non-Complex number we know is dividing by 0 as this is not a number!
\item \( \text{Pure Imaginary} \)

This is a Complex number $(a+bi)$ that \textbf{only} has an imaginary part like $2i$.
\item \( \text{Rational} \)

These are numbers that can be written as fraction of Integers (e.g., -2/3 + 5)
\item \( \text{Irrational} \)

* This is the correct option!
\item \( \text{Nonreal Complex} \)

This is a Complex number $(a+bi)$ that is not Real (has $i$ as part of the number).
\end{enumerate}

\textbf{General Comment:} Be sure to simplify $i^2 = -1$. This may remove the imaginary portion for your number. If you are having trouble, you may want to look at the \textit{Subgroups of the Real Numbers} section.
}
\litem{
Simplify the expression below and choose the interval the simplification is contained within.
\[ 15 - 18^2 + 20 \div 2 * 12 \div 13 \]The solution is \( -299.769 \), which is option B.\begin{enumerate}[label=\Alph*.]
\item \( [346.23, 354.23] \)

 348.231, which corresponds to an Order of Operations error: multiplying by negative before squaring. For example: $(-3)^2 \neq -3^2$
\item \( [-300.77, -290.77] \)

* -299.769, this is the correct option
\item \( [338.06, 344.06] \)

 339.064, which corresponds to two Order of Operations errors.
\item \( [-311.94, -306.94] \)

 -308.936, which corresponds to an Order of Operations error: not reading left-to-right for multiplication/division.
\item \( \text{None of the above} \)

 You may have gotten this by making an unanticipated error. If you got a value that is not any of the others, please let the coordinator know so they can help you figure out what happened.
\end{enumerate}

\textbf{General Comment:} While you may remember (or were taught) PEMDAS is done in order, it is actually done as P/E/MD/AS. When we are at MD or AS, we read left to right.
}
\litem{
Simplify the expression below into the form $a+bi$. Then, choose the intervals that $a$ and $b$ belong to.
\[ \frac{-54 + 44 i}{-7 - 5 i} \]The solution is \( 2.14  - 7.81 i \), which is option E.\begin{enumerate}[label=\Alph*.]
\item \( a \in [2.1, 2.8] \text{ and } b \in [-579, -577.5] \)

 $2.14  - 578.00 i$, which corresponds to forgetting to multiply the conjugate by the numerator.
\item \( a \in [157.85, 159] \text{ and } b \in [-8.5, -7.5] \)

 $158.00  - 7.81 i$, which corresponds to forgetting to multiply the conjugate by the numerator and using a plus instead of a minus in the denominator.
\item \( a \in [7.75, 8.45] \text{ and } b \in [-1.5, 0] \)

 $8.08  - 0.51 i$, which corresponds to forgetting to multiply the conjugate by the numerator and not computing the conjugate correctly.
\item \( a \in [7.15, 8.05] \text{ and } b \in [-9.5, -8] \)

 $7.71  - 8.80 i$, which corresponds to just dividing the first term by the first term and the second by the second.
\item \( a \in [2.1, 2.8] \text{ and } b \in [-8.5, -7.5] \)

* $2.14  - 7.81 i$, which is the correct option.
\end{enumerate}

\textbf{General Comment:} Multiply the numerator and denominator by the *conjugate* of the denominator, then simplify. For example, if we have $2+3i$, the conjugate is $2-3i$.
}
\litem{
Choose the \textbf{smallest} set of Real numbers that the number below belongs to.
\[ \sqrt{\frac{441}{100}} \]The solution is \( \text{Rational} \), which is option B.\begin{enumerate}[label=\Alph*.]
\item \( \text{Irrational} \)

These cannot be written as a fraction of Integers.
\item \( \text{Rational} \)

* This is the correct option!
\item \( \text{Not a Real number} \)

These are Nonreal Complex numbers \textbf{OR} things that are not numbers (e.g., dividing by 0).
\item \( \text{Integer} \)

These are the negative and positive counting numbers (..., -3, -2, -1, 0, 1, 2, 3, ...)
\item \( \text{Whole} \)

These are the counting numbers with 0 (0, 1, 2, 3, ...)
\end{enumerate}

\textbf{General Comment:} First, you \textbf{NEED} to simplify the expression. This question simplifies to $\frac{21}{10}$. 
 
 Be sure you look at the simplified fraction and not just the decimal expansion. Numbers such as 13, 17, and 19 provide \textbf{long but repeating/terminating decimal expansions!} 
 
 The only ways to *not* be a Real number are: dividing by 0 or taking the square root of a negative number. 
 
 Irrational numbers are more than just square root of 3: adding or subtracting values from square root of 3 is also irrational.
}
\litem{
Simplify the expression below into the form $a+bi$. Then, choose the intervals that $a$ and $b$ belong to.
\[ (8 + 5 i)(4 + 9 i) \]The solution is \( -13 + 92 i \), which is option D.\begin{enumerate}[label=\Alph*.]
\item \( a \in [74, 82] \text{ and } b \in [49, 56] \)

 $77 + 52 i$, which corresponds to adding a minus sign in the first term.
\item \( a \in [31, 36] \text{ and } b \in [43, 51] \)

 $32 + 45 i$, which corresponds to just multiplying the real terms to get the real part of the solution and the coefficients in the complex terms to get the complex part.
\item \( a \in [-18, -9] \text{ and } b \in [-93, -88] \)

 $-13 - 92 i$, which corresponds to adding a minus sign in both terms.
\item \( a \in [-18, -9] \text{ and } b \in [88, 98] \)

* $-13 + 92 i$, which is the correct option.
\item \( a \in [74, 82] \text{ and } b \in [-60, -48] \)

 $77 - 52 i$, which corresponds to adding a minus sign in the second term.
\end{enumerate}

\textbf{General Comment:} You can treat $i$ as a variable and distribute. Just remember that $i^2=-1$, so you can continue to reduce after you distribute.
}
\litem{
Choose the \textbf{smallest} set of Complex numbers that the number below belongs to.
\[ \sqrt{\frac{1078}{0}}+\sqrt{90} i \]The solution is \( \text{Not a Complex Number} \), which is option D.\begin{enumerate}[label=\Alph*.]
\item \( \text{Pure Imaginary} \)

This is a Complex number $(a+bi)$ that \textbf{only} has an imaginary part like $2i$.
\item \( \text{Irrational} \)

These cannot be written as a fraction of Integers. Remember: $\pi$ is not an Integer!
\item \( \text{Nonreal Complex} \)

This is a Complex number $(a+bi)$ that is not Real (has $i$ as part of the number).
\item \( \text{Not a Complex Number} \)

* This is the correct option!
\item \( \text{Rational} \)

These are numbers that can be written as fraction of Integers (e.g., -2/3 + 5)
\end{enumerate}

\textbf{General Comment:} Be sure to simplify $i^2 = -1$. This may remove the imaginary portion for your number. If you are having trouble, you may want to look at the \textit{Subgroups of the Real Numbers} section.
}
\litem{
Simplify the expression below into the form $a+bi$. Then, choose the intervals that $a$ and $b$ belong to.
\[ (9 - 3 i)(-10 - 4 i) \]The solution is \( -102 - 6 i \), which is option D.\begin{enumerate}[label=\Alph*.]
\item \( a \in [-104, -96] \text{ and } b \in [1, 9] \)

 $-102 + 6 i$, which corresponds to adding a minus sign in both terms.
\item \( a \in [-93, -87] \text{ and } b \in [9, 15] \)

 $-90 + 12 i$, which corresponds to just multiplying the real terms to get the real part of the solution and the coefficients in the complex terms to get the complex part.
\item \( a \in [-81, -70] \text{ and } b \in [-69, -58] \)

 $-78 - 66 i$, which corresponds to adding a minus sign in the first term.
\item \( a \in [-104, -96] \text{ and } b \in [-11, 1] \)

* $-102 - 6 i$, which is the correct option.
\item \( a \in [-81, -70] \text{ and } b \in [66, 70] \)

 $-78 + 66 i$, which corresponds to adding a minus sign in the second term.
\end{enumerate}

\textbf{General Comment:} You can treat $i$ as a variable and distribute. Just remember that $i^2=-1$, so you can continue to reduce after you distribute.
}
\end{enumerate}

\end{document}