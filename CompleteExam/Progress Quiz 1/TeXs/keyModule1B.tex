\documentclass{extbook}[14pt]
\usepackage{multicol, enumerate, enumitem, hyperref, color, soul, setspace, parskip, fancyhdr, amssymb, amsthm, amsmath, latexsym, units, mathtools}
\everymath{\displaystyle}
\usepackage[headsep=0.5cm,headheight=0cm, left=1 in,right= 1 in,top= 1 in,bottom= 1 in]{geometry}
\usepackage{dashrule}  % Package to use the command below to create lines between items
\newcommand{\litem}[1]{\item #1

\rule{\textwidth}{0.4pt}}
\pagestyle{fancy}
\lhead{}
\chead{Answer Key for Progress Quiz 1 Version B}
\rhead{}
\lfoot{5899-4682}
\cfoot{}
\rfoot{Spring 2021}
\begin{document}
\textbf{This key should allow you to understand why you choose the option you did (beyond just getting a question right or wrong). \href{https://xronos.clas.ufl.edu/mac1105spring2020/courseDescriptionAndMisc/Exams/LearningFromResults}{More instructions on how to use this key can be found here}.}

\textbf{If you have a suggestion to make the keys better, \href{https://forms.gle/CZkbZmPbC9XALEE88}{please fill out the short survey here}.}

\textit{Note: This key is auto-generated and may contain issues and/or errors. The keys are reviewed after each exam to ensure grading is done accurately. If there are issues (like duplicate options), they are noted in the offline gradebook. The keys are a work-in-progress to give students as many resources to improve as possible.}

\rule{\textwidth}{0.4pt}

\begin{enumerate}\litem{
Choose the \textbf{smallest} set of Real numbers that the number below belongs to.
\[ -\sqrt{\frac{196}{289}} \]The solution is \( \text{Rational} \), which is option A.\begin{enumerate}[label=\Alph*.]
\item \( \text{Rational} \)

* This is the correct option!
\item \( \text{Irrational} \)

These cannot be written as a fraction of Integers.
\item \( \text{Whole} \)

These are the counting numbers with 0 (0, 1, 2, 3, ...)
\item \( \text{Not a Real number} \)

These are Nonreal Complex numbers \textbf{OR} things that are not numbers (e.g., dividing by 0).
\item \( \text{Integer} \)

These are the negative and positive counting numbers (..., -3, -2, -1, 0, 1, 2, 3, ...)
\end{enumerate}

\textbf{General Comment:} First, you \textbf{NEED} to simplify the expression. This question simplifies to $-\frac{14}{17}$. 
 
 Be sure you look at the simplified fraction and not just the decimal expansion. Numbers such as 13, 17, and 19 provide \textbf{long but repeating/terminating decimal expansions!} 
 
 The only ways to *not* be a Real number are: dividing by 0 or taking the square root of a negative number. 
 
 Irrational numbers are more than just square root of 3: adding or subtracting values from square root of 3 is also irrational.
}
\litem{
Choose the \textbf{smallest} set of Real numbers that the number below belongs to.
\[ \sqrt{\frac{35721}{441}} \]The solution is \( \text{Whole} \), which is option A.\begin{enumerate}[label=\Alph*.]
\item \( \text{Whole} \)

* This is the correct option!
\item \( \text{Rational} \)

These are numbers that can be written as fraction of Integers (e.g., -2/3)
\item \( \text{Not a Real number} \)

These are Nonreal Complex numbers \textbf{OR} things that are not numbers (e.g., dividing by 0).
\item \( \text{Irrational} \)

These cannot be written as a fraction of Integers.
\item \( \text{Integer} \)

These are the negative and positive counting numbers (..., -3, -2, -1, 0, 1, 2, 3, ...)
\end{enumerate}

\textbf{General Comment:} First, you \textbf{NEED} to simplify the expression. This question simplifies to $189$. 
 
 Be sure you look at the simplified fraction and not just the decimal expansion. Numbers such as 13, 17, and 19 provide \textbf{long but repeating/terminating decimal expansions!} 
 
 The only ways to *not* be a Real number are: dividing by 0 or taking the square root of a negative number. 
 
 Irrational numbers are more than just square root of 3: adding or subtracting values from square root of 3 is also irrational.
}
\litem{
Simplify the expression below into the form $a+bi$. Then, choose the intervals that $a$ and $b$ belong to.
\[ (-9 - 2 i)(7 - 10 i) \]The solution is \( -83 + 76 i \), which is option A.\begin{enumerate}[label=\Alph*.]
\item \( a \in [-83, -81] \text{ and } b \in [75, 82] \)

* $-83 + 76 i$, which is the correct option.
\item \( a \in [-51, -40] \text{ and } b \in [-107, -103] \)

 $-43 - 104 i$, which corresponds to adding a minus sign in the second term.
\item \( a \in [-66, -59] \text{ and } b \in [18, 25] \)

 $-63 + 20 i$, which corresponds to just multiplying the real terms to get the real part of the solution and the coefficients in the complex terms to get the complex part.
\item \( a \in [-83, -81] \text{ and } b \in [-80, -69] \)

 $-83 - 76 i$, which corresponds to adding a minus sign in both terms.
\item \( a \in [-51, -40] \text{ and } b \in [100, 105] \)

 $-43 + 104 i$, which corresponds to adding a minus sign in the first term.
\end{enumerate}

\textbf{General Comment:} You can treat $i$ as a variable and distribute. Just remember that $i^2=-1$, so you can continue to reduce after you distribute.
}
\litem{
Simplify the expression below into the form $a+bi$. Then, choose the intervals that $a$ and $b$ belong to.
\[ \frac{18 + 55 i}{4 + 6 i} \]The solution is \( 7.73  + 2.15 i \), which is option E.\begin{enumerate}[label=\Alph*.]
\item \( a \in [400.5, 402.5] \text{ and } b \in [1.5, 3.5] \)

 $402.00  + 2.15 i$, which corresponds to forgetting to multiply the conjugate by the numerator and using a plus instead of a minus in the denominator.
\item \( a \in [6.5, 8] \text{ and } b \in [111.5, 113] \)

 $7.73  + 112.00 i$, which corresponds to forgetting to multiply the conjugate by the numerator.
\item \( a \in [-6, -4] \text{ and } b \in [6, 7] \)

 $-4.96  + 6.31 i$, which corresponds to forgetting to multiply the conjugate by the numerator and not computing the conjugate correctly.
\item \( a \in [3, 6] \text{ and } b \in [8.5, 10] \)

 $4.50  + 9.17 i$, which corresponds to just dividing the first term by the first term and the second by the second.
\item \( a \in [6.5, 8] \text{ and } b \in [1.5, 3.5] \)

* $7.73  + 2.15 i$, which is the correct option.
\end{enumerate}

\textbf{General Comment:} Multiply the numerator and denominator by the *conjugate* of the denominator, then simplify. For example, if we have $2+3i$, the conjugate is $2-3i$.
}
\litem{
Choose the \textbf{smallest} set of Complex numbers that the number below belongs to.
\[ \frac{-5}{2}+\sqrt{-16}i \]The solution is \( \text{Rational} \), which is option C.\begin{enumerate}[label=\Alph*.]
\item \( \text{Not a Complex Number} \)

This is not a number. The only non-Complex number we know is dividing by 0 as this is not a number!
\item \( \text{Irrational} \)

These cannot be written as a fraction of Integers. Remember: $\pi$ is not an Integer!
\item \( \text{Rational} \)

* This is the correct option!
\item \( \text{Pure Imaginary} \)

This is a Complex number $(a+bi)$ that \textbf{only} has an imaginary part like $2i$.
\item \( \text{Nonreal Complex} \)

This is a Complex number $(a+bi)$ that is not Real (has $i$ as part of the number).
\end{enumerate}

\textbf{General Comment:} Be sure to simplify $i^2 = -1$. This may remove the imaginary portion for your number. If you are having trouble, you may want to look at the \textit{Subgroups of the Real Numbers} section.
}
\litem{
Simplify the expression below and choose the interval the simplification is contained within.
\[ 19 - 6^2 + 20 \div 9 * 8 \div 5 \]The solution is \( -13.444 \), which is option D.\begin{enumerate}[label=\Alph*.]
\item \( [56.6, 61.3] \)

 58.556, which corresponds to an Order of Operations error: multiplying by negative before squaring. For example: $(-3)^2 \neq -3^2$
\item \( [-19.2, -13.5] \)

 -16.944, which corresponds to an Order of Operations error: not reading left-to-right for multiplication/division.
\item \( [54.7, 57.5] \)

 55.056, which corresponds to two Order of Operations errors.
\item \( [-15, -12.5] \)

* -13.444, this is the correct option
\item \( \text{None of the above} \)

 You may have gotten this by making an unanticipated error. If you got a value that is not any of the others, please let the coordinator know so they can help you figure out what happened.
\end{enumerate}

\textbf{General Comment:} While you may remember (or were taught) PEMDAS is done in order, it is actually done as P/E/MD/AS. When we are at MD or AS, we read left to right.
}
\litem{
Simplify the expression below and choose the interval the simplification is contained within.
\[ 6 - 11 \div 2 * 7 - (1 * 12) \]The solution is \( -44.500 \), which is option A.\begin{enumerate}[label=\Alph*.]
\item \( [-51.5, -40.5] \)

* -44.500, which is the correct option.
\item \( [17.21, 24.21] \)

 17.214, which corresponds to not distributing addition and subtraction correctly.
\item \( [-402, -401] \)

 -402.000, which corresponds to not distributing a negative correctly.
\item \( [-10.79, -1.79] \)

 -6.786, which corresponds to an Order of Operations error: not reading left-to-right for multiplication/division.
\item \( \text{None of the above} \)

 You may have gotten this by making an unanticipated error. If you got a value that is not any of the others, please let the coordinator know so they can help you figure out what happened.
\end{enumerate}

\textbf{General Comment:} While you may remember (or were taught) PEMDAS is done in order, it is actually done as P/E/MD/AS. When we are at MD or AS, we read left to right.
}
\litem{
Choose the \textbf{smallest} set of Complex numbers that the number below belongs to.
\[ \frac{-20}{5}+\sqrt{-64}i \]The solution is \( \text{Rational} \), which is option C.\begin{enumerate}[label=\Alph*.]
\item \( \text{Nonreal Complex} \)

This is a Complex number $(a+bi)$ that is not Real (has $i$ as part of the number).
\item \( \text{Not a Complex Number} \)

This is not a number. The only non-Complex number we know is dividing by 0 as this is not a number!
\item \( \text{Rational} \)

* This is the correct option!
\item \( \text{Irrational} \)

These cannot be written as a fraction of Integers. Remember: $\pi$ is not an Integer!
\item \( \text{Pure Imaginary} \)

This is a Complex number $(a+bi)$ that \textbf{only} has an imaginary part like $2i$.
\end{enumerate}

\textbf{General Comment:} Be sure to simplify $i^2 = -1$. This may remove the imaginary portion for your number. If you are having trouble, you may want to look at the \textit{Subgroups of the Real Numbers} section.
}
\litem{
Simplify the expression below into the form $a+bi$. Then, choose the intervals that $a$ and $b$ belong to.
\[ (9 - 3 i)(8 + 4 i) \]The solution is \( 84 + 12 i \), which is option E.\begin{enumerate}[label=\Alph*.]
\item \( a \in [84, 85] \text{ and } b \in [-18, -7] \)

 $84 - 12 i$, which corresponds to adding a minus sign in both terms.
\item \( a \in [58, 65] \text{ and } b \in [-61, -56] \)

 $60 - 60 i$, which corresponds to adding a minus sign in the second term.
\item \( a \in [69, 73] \text{ and } b \in [-18, -7] \)

 $72 - 12 i$, which corresponds to just multiplying the real terms to get the real part of the solution and the coefficients in the complex terms to get the complex part.
\item \( a \in [58, 65] \text{ and } b \in [55, 64] \)

 $60 + 60 i$, which corresponds to adding a minus sign in the first term.
\item \( a \in [84, 85] \text{ and } b \in [10, 17] \)

* $84 + 12 i$, which is the correct option.
\end{enumerate}

\textbf{General Comment:} You can treat $i$ as a variable and distribute. Just remember that $i^2=-1$, so you can continue to reduce after you distribute.
}
\litem{
Simplify the expression below into the form $a+bi$. Then, choose the intervals that $a$ and $b$ belong to.
\[ \frac{63 - 66 i}{2 - 4 i} \]The solution is \( 19.50  + 6.00 i \), which is option A.\begin{enumerate}[label=\Alph*.]
\item \( a \in [19, 20] \text{ and } b \in [5, 7] \)

* $19.50  + 6.00 i$, which is the correct option.
\item \( a \in [19, 20] \text{ and } b \in [118.5, 120.5] \)

 $19.50  + 120.00 i$, which corresponds to forgetting to multiply the conjugate by the numerator.
\item \( a \in [30.5, 33] \text{ and } b \in [16, 17] \)

 $31.50  + 16.50 i$, which corresponds to just dividing the first term by the first term and the second by the second.
\item \( a \in [-8, -6.5] \text{ and } b \in [-19.5, -18.5] \)

 $-6.90  - 19.20 i$, which corresponds to forgetting to multiply the conjugate by the numerator and not computing the conjugate correctly.
\item \( a \in [389.5, 390.5] \text{ and } b \in [5, 7] \)

 $390.00  + 6.00 i$, which corresponds to forgetting to multiply the conjugate by the numerator and using a plus instead of a minus in the denominator.
\end{enumerate}

\textbf{General Comment:} Multiply the numerator and denominator by the *conjugate* of the denominator, then simplify. For example, if we have $2+3i$, the conjugate is $2-3i$.
}
\end{enumerate}

\end{document}