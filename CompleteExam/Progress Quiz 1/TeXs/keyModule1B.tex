\documentclass{extbook}[14pt]
\usepackage{multicol, enumerate, enumitem, hyperref, color, soul, setspace, parskip, fancyhdr, amssymb, amsthm, amsmath, bbm, latexsym, units, mathtools}
\everymath{\displaystyle}
\usepackage[headsep=0.5cm,headheight=0cm, left=1 in,right= 1 in,top= 1 in,bottom= 1 in]{geometry}
\usepackage{dashrule}  % Package to use the command below to create lines between items
\newcommand{\litem}[1]{\item #1

\rule{\textwidth}{0.4pt}}
\pagestyle{fancy}
\lhead{}
\chead{Answer Key for Progress Quiz 1 Version B}
\rhead{}
\lfoot{3939-9803}
\cfoot{}
\rfoot{Fall 2020}
\begin{document}
\textbf{This key should allow you to understand why you choose the option you did (beyond just getting a question right or wrong). \href{https://xronos.clas.ufl.edu/mac1105spring2020/courseDescriptionAndMisc/Exams/LearningFromResults}{More instructions on how to use this key can be found here}.}

\textbf{If you have a suggestion to make the keys better, \href{https://forms.gle/CZkbZmPbC9XALEE88}{please fill out the short survey here}.}

\textit{Note: This key is auto-generated and may contain issues and/or errors. The keys are reviewed after each exam to ensure grading is done accurately. If there are issues (like duplicate options), they are noted in the offline gradebook. The keys are a work-in-progress to give students as many resources to improve as possible.}

\rule{\textwidth}{0.4pt}

\begin{enumerate}\litem{
Simplify the expression below into the form $a+bi$. Then, choose the intervals that $a$ and $b$ belong to.
\[ \frac{9 - 77 i}{4 + 3 i} \]
The solution is \( -7.80  - 13.40 i \), which is option D.\begin{enumerate}[label=\Alph*.]
\item \( a \in [-195.5, -194.5] \text{ and } b \in [-14.5, -12.5] \)
 $-195.00  - 13.40 i$, which corresponds to forgetting to multiply the conjugate by the numerator and using a plus instead of a minus in the denominator.
\item \( a \in [-9.5, -7] \text{ and } b \in [-335.5, -334] \)
 $-7.80  - 335.00 i$, which corresponds to forgetting to multiply the conjugate by the numerator.
\item \( a \in [10, 11] \text{ and } b \in [-12.5, -10.5] \)
 $10.68  - 11.24 i$, which corresponds to forgetting to multiply the conjugate by the numerator and not computing the conjugate correctly.
\item \( a \in [-9.5, -7] \text{ and } b \in [-14.5, -12.5] \)
* $-7.80  - 13.40 i$, which is the correct option.
\item \( a \in [1.5, 3] \text{ and } b \in [-26.5, -24.5] \)
 $2.25  - 25.67 i$, which corresponds to just dividing the first term by the first term and the second by the second.
\end{enumerate}

\textbf{General Comment:} Multiply the numerator and denominator by the *conjugate* of the denominator, then simplify. For example, if we have $2+3i$, the conjugate is $2-3i$.
}
\litem{
Simplify the expression below into the form $a+bi$. Then, choose the intervals that $a$ and $b$ belong to.
\[ (-10 - 8 i)(-6 - 7 i) \]
The solution is \( 4 + 118 i \), which is option E.\begin{enumerate}[label=\Alph*.]
\item \( a \in [113, 118] \text{ and } b \in [-26, -18] \)
 $116 - 22 i$, which corresponds to adding a minus sign in the second term.
\item \( a \in [0, 7] \text{ and } b \in [-122, -115] \)
 $4 - 118 i$, which corresponds to adding a minus sign in both terms.
\item \( a \in [58, 69] \text{ and } b \in [54, 62] \)
 $60 + 56 i$, which corresponds to just multiplying the real terms to get the real part of the solution and the coefficients in the complex terms to get the complex part.
\item \( a \in [113, 118] \text{ and } b \in [18, 27] \)
 $116 + 22 i$, which corresponds to adding a minus sign in the first term.
\item \( a \in [0, 7] \text{ and } b \in [113, 124] \)
* $4 + 118 i$, which is the correct option.
\end{enumerate}

\textbf{General Comment:} You can treat $i$ as a variable and distribute. Just remember that $i^2=-1$, so you can continue to reduce after you distribute.
}

\litem{
Simplify the expression below and choose the interval the simplification is contained within.
\[ 19 - 3 \div 2 * 8 - (18 * 13) \]
The solution is \( -227.000 \), which is option C.\begin{enumerate}[label=\Alph*.]
\item \( [252.81, 255.81] \)
 252.812, which corresponds to not distributing addition and subtraction correctly.
\item \( [-146, -139] \)
 -143.000, which corresponds to not distributing a negative correctly.
\item \( [-231, -219] \)
* -227.000, which is the correct option.
\item \( [-221.19, -212.19] \)
 -215.188, which corresponds to an Order of Operations error: not reading left-to-right for multiplication/division.
\item \( \text{None of the above} \)
 You may have gotten this by making an unanticipated error. If you got a value that is not any of the others, please let the coordinator know so they can help you figure out what happened.
\end{enumerate}

\textbf{General Comment:} While you may remember (or were taught) PEMDAS is done in order, it is actually done as P/E/MD/AS. When we are at MD or AS, we read left to right.
}
\litem{
Choose the \textbf{smallest} set of Complex numbers that the number below belongs to.
\[ \sqrt{\frac{64}{169}} + 49i^2 \]
The solution is \( \text{Rational} \), which is option A.\begin{enumerate}[label=\Alph*.]
\item \( \text{Rational} \)
* This is the correct option!
\item \( \text{Nonreal Complex} \)
This is a Complex number $(a+bi)$ that is not Real (has $i$ as part of the number).
\item \( \text{Irrational} \)
These cannot be written as a fraction of Integers. Remember: $\pi$ is not an Integer!
\item \( \text{Pure Imaginary} \)
This is a Complex number $(a+bi)$ that \textbf{only} has an imaginary part like $2i$.
\item \( \text{Not a Complex Number} \)
This is not a number. The only non-Complex number we know is dividing by 0 as this is not a number!
\end{enumerate}

\textbf{General Comment:} Be sure to simplify $i^2 = -1$. This may remove the imaginary portion for your number. If you are having trouble, you may want to look at the \textit{Subgroups of the Real Numbers} section.
}
\litem{
Choose the \textbf{smallest} set of Real numbers that the number below belongs to.
\[ \sqrt{\frac{529}{25}} \]
The solution is \( \text{Rational} \), which is option B.\begin{enumerate}[label=\Alph*.]
\item \( \text{Irrational} \)
These cannot be written as a fraction of Integers.
\item \( \text{Rational} \)
* This is the correct option!
\item \( \text{Integer} \)
These are the negative and positive counting numbers (..., -3, -2, -1, 0, 1, 2, 3, ...)
\item \( \text{Whole} \)
These are the counting numbers with 0 (0, 1, 2, 3, ...)
\item \( \text{Not a Real number} \)
These are Nonreal Complex numbers \textbf{OR} things that are not numbers (e.g., dividing by 0).
\end{enumerate}

\textbf{General Comment:} First, you \textbf{NEED} to simplify the expression. This question simplifies to $\frac{23}{5}$. 
 
 Be sure you look at the simplified fraction and not just the decimal expansion. Numbers such as 13, 17, and 19 provide \textbf{long but repeating/terminating decimal expansions!} 
 
 The only ways to *not* be a Real number are: dividing by 0 or taking the square root of a negative number. 
 
 Irrational numbers are more than just square root of 3: adding or subtracting values from square root of 3 is also irrational.
}

\litem{
Simplify the expression below into the form $a+bi$. Then, choose the intervals that $a$ and $b$ belong to.
\[ \frac{72 - 11 i}{3 + 6 i} \]
The solution is \( 3.33  - 10.33 i \), which is option D.\begin{enumerate}[label=\Alph*.]
\item \( a \in [149.5, 151] \text{ and } b \in [-11.5, -9.5] \)
 $150.00  - 10.33 i$, which corresponds to forgetting to multiply the conjugate by the numerator and using a plus instead of a minus in the denominator.
\item \( a \in [23, 25.5] \text{ and } b \in [-2.5, -1.5] \)
 $24.00  - 1.83 i$, which corresponds to just dividing the first term by the first term and the second by the second.
\item \( a \in [1.5, 4] \text{ and } b \in [-466, -464.5] \)
 $3.33  - 465.00 i$, which corresponds to forgetting to multiply the conjugate by the numerator.
\item \( a \in [1.5, 4] \text{ and } b \in [-11.5, -9.5] \)
* $3.33  - 10.33 i$, which is the correct option.
\item \( a \in [5, 6.5] \text{ and } b \in [8.5, 10.5] \)
 $6.27  + 8.87 i$, which corresponds to forgetting to multiply the conjugate by the numerator and not computing the conjugate correctly.
\end{enumerate}

\textbf{General Comment:} Multiply the numerator and denominator by the *conjugate* of the denominator, then simplify. For example, if we have $2+3i$, the conjugate is $2-3i$.
}
\litem{
Simplify the expression below into the form $a+bi$. Then, choose the intervals that $a$ and $b$ belong to.
\[ (-6 - 8 i)(2 + 4 i) \]
The solution is \( 20 - 40 i \), which is option E.\begin{enumerate}[label=\Alph*.]
\item \( a \in [17, 23] \text{ and } b \in [35, 43] \)
 $20 + 40 i$, which corresponds to adding a minus sign in both terms.
\item \( a \in [-15, -10] \text{ and } b \in [-34, -30] \)
 $-12 - 32 i$, which corresponds to just multiplying the real terms to get the real part of the solution and the coefficients in the complex terms to get the complex part.
\item \( a \in [-45, -40] \text{ and } b \in [6, 9] \)
 $-44 + 8 i$, which corresponds to adding a minus sign in the second term.
\item \( a \in [-45, -40] \text{ and } b \in [-12, -5] \)
 $-44 - 8 i$, which corresponds to adding a minus sign in the first term.
\item \( a \in [17, 23] \text{ and } b \in [-41, -34] \)
* $20 - 40 i$, which is the correct option.
\end{enumerate}

\textbf{General Comment:} You can treat $i$ as a variable and distribute. Just remember that $i^2=-1$, so you can continue to reduce after you distribute.
}

\litem{
Simplify the expression below and choose the interval the simplification is contained within.
\[ 3 - 6^2 + 10 \div 19 * 9 \div 8 \]
The solution is \( -32.408 \), which is option A.\begin{enumerate}[label=\Alph*.]
\item \( [-32.78, -32.26] \)
* -32.408, this is the correct option
\item \( [-33.09, -32.91] \)
 -32.993, which corresponds to an Order of Operations error: not reading left-to-right for multiplication/division.
\item \( [39.55, 39.75] \)
 39.592, which corresponds to an Order of Operations error: multiplying by negative before squaring. For example: $(-3)^2 \neq -3^2$
\item \( [38.63, 39.25] \)
 39.007, which corresponds to two Order of Operations errors.
\item \( \text{None of the above} \)
 You may have gotten this by making an unanticipated error. If you got a value that is not any of the others, please let the coordinator know so they can help you figure out what happened.
\end{enumerate}

\textbf{General Comment:} While you may remember (or were taught) PEMDAS is done in order, it is actually done as P/E/MD/AS. When we are at MD or AS, we read left to right.
}
\litem{
Choose the \textbf{smallest} set of Complex numbers that the number below belongs to.
\[ \sqrt{\frac{576}{121}} + 9i^2 \]
The solution is \( \text{Rational} \), which is option A.\begin{enumerate}[label=\Alph*.]
\item \( \text{Rational} \)
* This is the correct option!
\item \( \text{Nonreal Complex} \)
This is a Complex number $(a+bi)$ that is not Real (has $i$ as part of the number).
\item \( \text{Pure Imaginary} \)
This is a Complex number $(a+bi)$ that \textbf{only} has an imaginary part like $2i$.
\item \( \text{Irrational} \)
These cannot be written as a fraction of Integers. Remember: $\pi$ is not an Integer!
\item \( \text{Not a Complex Number} \)
This is not a number. The only non-Complex number we know is dividing by 0 as this is not a number!
\end{enumerate}

\textbf{General Comment:} Be sure to simplify $i^2 = -1$. This may remove the imaginary portion for your number. If you are having trouble, you may want to look at the \textit{Subgroups of the Real Numbers} section.
}
\litem{
Choose the \textbf{smallest} set of Real numbers that the number below belongs to.
\[ -\sqrt{\frac{529}{400}} \]
The solution is \( \text{Rational} \), which is option B.\begin{enumerate}[label=\Alph*.]
\item \( \text{Integer} \)
These are the negative and positive counting numbers (..., -3, -2, -1, 0, 1, 2, 3, ...)
\item \( \text{Rational} \)
* This is the correct option!
\item \( \text{Not a Real number} \)
These are Nonreal Complex numbers \textbf{OR} things that are not numbers (e.g., dividing by 0).
\item \( \text{Irrational} \)
These cannot be written as a fraction of Integers.
\item \( \text{Whole} \)
These are the counting numbers with 0 (0, 1, 2, 3, ...)
\end{enumerate}

\textbf{General Comment:} First, you \textbf{NEED} to simplify the expression. This question simplifies to $-\frac{23}{20}$. 
 
 Be sure you look at the simplified fraction and not just the decimal expansion. Numbers such as 13, 17, and 19 provide \textbf{long but repeating/terminating decimal expansions!} 
 
 The only ways to *not* be a Real number are: dividing by 0 or taking the square root of a negative number. 
 
 Irrational numbers are more than just square root of 3: adding or subtracting values from square root of 3 is also irrational.
}
\end{enumerate}

\end{document}