\documentclass{extbook}[14pt]
\usepackage{multicol, enumerate, enumitem, hyperref, color, soul, setspace, parskip, fancyhdr, amssymb, amsthm, amsmath, latexsym, units, mathtools}
\everymath{\displaystyle}
\usepackage[headsep=0.5cm,headheight=0cm, left=1 in,right= 1 in,top= 1 in,bottom= 1 in]{geometry}
\usepackage{dashrule}  % Package to use the command below to create lines between items
\newcommand{\litem}[1]{\item #1

\rule{\textwidth}{0.4pt}}
\pagestyle{fancy}
\lhead{}
\chead{Answer Key for Progress Quiz 1 Version B}
\rhead{}
\lfoot{4082-7053}
\cfoot{}
\rfoot{test}
\begin{document}
\textbf{This key should allow you to understand why you choose the option you did (beyond just getting a question right or wrong). \href{https://xronos.clas.ufl.edu/mac1105spring2020/courseDescriptionAndMisc/Exams/LearningFromResults}{More instructions on how to use this key can be found here}.}

\textbf{If you have a suggestion to make the keys better, \href{https://forms.gle/CZkbZmPbC9XALEE88}{please fill out the short survey here}.}

\textit{Note: This key is auto-generated and may contain issues and/or errors. The keys are reviewed after each exam to ensure grading is done accurately. If there are issues (like duplicate options), they are noted in the offline gradebook. The keys are a work-in-progress to give students as many resources to improve as possible.}

\rule{\textwidth}{0.4pt}

\begin{enumerate}\litem{
Choose the \textbf{smallest} set of Real numbers that the number below belongs to.
\[ -\sqrt{\frac{8100}{25}} \]The solution is \( \text{Integer} \), which is option C.\begin{enumerate}[label=\Alph*.]
\item \( \text{Not a Real number} \)

These are Nonreal Complex numbers \textbf{OR} things that are not numbers (e.g., dividing by 0).
\item \( \text{Irrational} \)

These cannot be written as a fraction of Integers.
\item \( \text{Integer} \)

* This is the correct option!
\item \( \text{Whole} \)

These are the counting numbers with 0 (0, 1, 2, 3, ...)
\item \( \text{Rational} \)

These are numbers that can be written as fraction of Integers (e.g., -2/3)
\end{enumerate}

\textbf{General Comment:} First, you \textbf{NEED} to simplify the expression. This question simplifies to $-90$. 
 
 Be sure you look at the simplified fraction and not just the decimal expansion. Numbers such as 13, 17, and 19 provide \textbf{long but repeating/terminating decimal expansions!} 
 
 The only ways to *not* be a Real number are: dividing by 0 or taking the square root of a negative number. 
 
 Irrational numbers are more than just square root of 3: adding or subtracting values from square root of 3 is also irrational.
}
\litem{
Simplify the expression below and choose the interval the simplification is contained within.
\[ 19 - 6 \div 16 * 17 - (11 * 20) \]The solution is \( -207.375 \), which is option B.\begin{enumerate}[label=\Alph*.]
\item \( [233.98, 240.98] \)

 238.978, which corresponds to not distributing addition and subtraction correctly.
\item \( [-210.38, -206.38] \)

* -207.375, which is the correct option.
\item \( [-202.02, -198.02] \)

 -201.022, which corresponds to an Order of Operations error: not reading left-to-right for multiplication/division.
\item \( [29.5, 40.5] \)

 32.500, which corresponds to not distributing a negative correctly.
\item \( \text{None of the above} \)

 You may have gotten this by making an unanticipated error. If you got a value that is not any of the others, please let the coordinator know so they can help you figure out what happened.
\end{enumerate}

\textbf{General Comment:} While you may remember (or were taught) PEMDAS is done in order, it is actually done as P/E/MD/AS. When we are at MD or AS, we read left to right.
}
\litem{
Simplify the expression below and choose the interval the simplification is contained within.
\[ 12 - 13 \div 5 * 9 - (6 * 18) \]The solution is \( -119.400 \), which is option A.\begin{enumerate}[label=\Alph*.]
\item \( [-124.4, -115.4] \)

* -119.400, which is the correct option.
\item \( [113.71, 127.71] \)

 119.711, which corresponds to not distributing addition and subtraction correctly.
\item \( [-315.2, -310.2] \)

 -313.200, which corresponds to not distributing a negative correctly.
\item \( [-98.29, -91.29] \)

 -96.289, which corresponds to an Order of Operations error: not reading left-to-right for multiplication/division.
\item \( \text{None of the above} \)

 You may have gotten this by making an unanticipated error. If you got a value that is not any of the others, please let the coordinator know so they can help you figure out what happened.
\end{enumerate}

\textbf{General Comment:} While you may remember (or were taught) PEMDAS is done in order, it is actually done as P/E/MD/AS. When we are at MD or AS, we read left to right.
}
\litem{
Choose the \textbf{smallest} set of Complex numbers that the number below belongs to.
\[ \frac{-7}{5}+\sqrt{-36}i \]The solution is \( \text{Rational} \), which is option D.\begin{enumerate}[label=\Alph*.]
\item \( \text{Not a Complex Number} \)

This is not a number. The only non-Complex number we know is dividing by 0 as this is not a number!
\item \( \text{Pure Imaginary} \)

This is a Complex number $(a+bi)$ that \textbf{only} has an imaginary part like $2i$.
\item \( \text{Irrational} \)

These cannot be written as a fraction of Integers. Remember: $\pi$ is not an Integer!
\item \( \text{Rational} \)

* This is the correct option!
\item \( \text{Nonreal Complex} \)

This is a Complex number $(a+bi)$ that is not Real (has $i$ as part of the number).
\end{enumerate}

\textbf{General Comment:} Be sure to simplify $i^2 = -1$. This may remove the imaginary portion for your number. If you are having trouble, you may want to look at the \textit{Subgroups of the Real Numbers} section.
}
\litem{
Simplify the expression below into the form $a+bi$. Then, choose the intervals that $a$ and $b$ belong to.
\[ (8 - 5 i)(2 + 9 i) \]The solution is \( 61 + 62 i \), which is option A.\begin{enumerate}[label=\Alph*.]
\item \( a \in [59, 62] \text{ and } b \in [58, 63] \)

* $61 + 62 i$, which is the correct option.
\item \( a \in [13, 20] \text{ and } b \in [-49, -44] \)

 $16 - 45 i$, which corresponds to just multiplying the real terms to get the real part of the solution and the coefficients in the complex terms to get the complex part.
\item \( a \in [-29, -24] \text{ and } b \in [-86, -72] \)

 $-29 - 82 i$, which corresponds to adding a minus sign in the second term.
\item \( a \in [-29, -24] \text{ and } b \in [76, 86] \)

 $-29 + 82 i$, which corresponds to adding a minus sign in the first term.
\item \( a \in [59, 62] \text{ and } b \in [-64, -61] \)

 $61 - 62 i$, which corresponds to adding a minus sign in both terms.
\end{enumerate}

\textbf{General Comment:} You can treat $i$ as a variable and distribute. Just remember that $i^2=-1$, so you can continue to reduce after you distribute.
}
\litem{
Simplify the expression below into the form $a+bi$. Then, choose the intervals that $a$ and $b$ belong to.
\[ \frac{-27 + 77 i}{-8 + 6 i} \]The solution is \( 6.78  - 4.54 i \), which is option A.\begin{enumerate}[label=\Alph*.]
\item \( a \in [6, 7.5] \text{ and } b \in [-5.5, -3.5] \)

* $6.78  - 4.54 i$, which is the correct option.
\item \( a \in [6, 7.5] \text{ and } b \in [-455, -452.5] \)

 $6.78  - 454.00 i$, which corresponds to forgetting to multiply the conjugate by the numerator.
\item \( a \in [-3.5, -2] \text{ and } b \in [-8.5, -7.5] \)

 $-2.46  - 7.78 i$, which corresponds to forgetting to multiply the conjugate by the numerator and not computing the conjugate correctly.
\item \( a \in [677.5, 678.5] \text{ and } b \in [-5.5, -3.5] \)

 $678.00  - 4.54 i$, which corresponds to forgetting to multiply the conjugate by the numerator and using a plus instead of a minus in the denominator.
\item \( a \in [2.5, 4] \text{ and } b \in [11.5, 13.5] \)

 $3.38  + 12.83 i$, which corresponds to just dividing the first term by the first term and the second by the second.
\end{enumerate}

\textbf{General Comment:} Multiply the numerator and denominator by the *conjugate* of the denominator, then simplify. For example, if we have $2+3i$, the conjugate is $2-3i$.
}
\litem{
Choose the \textbf{smallest} set of Complex numbers that the number below belongs to.
\[ \frac{-11}{22}+\sqrt{77} i \]The solution is \( \text{Nonreal Complex} \), which is option D.\begin{enumerate}[label=\Alph*.]
\item \( \text{Not a Complex Number} \)

This is not a number. The only non-Complex number we know is dividing by 0 as this is not a number!
\item \( \text{Irrational} \)

These cannot be written as a fraction of Integers. Remember: $\pi$ is not an Integer!
\item \( \text{Rational} \)

These are numbers that can be written as fraction of Integers (e.g., -2/3 + 5)
\item \( \text{Nonreal Complex} \)

* This is the correct option!
\item \( \text{Pure Imaginary} \)

This is a Complex number $(a+bi)$ that \textbf{only} has an imaginary part like $2i$.
\end{enumerate}

\textbf{General Comment:} Be sure to simplify $i^2 = -1$. This may remove the imaginary portion for your number. If you are having trouble, you may want to look at the \textit{Subgroups of the Real Numbers} section.
}
\litem{
Simplify the expression below into the form $a+bi$. Then, choose the intervals that $a$ and $b$ belong to.
\[ (10 - 3 i)(9 - 2 i) \]The solution is \( 84 - 47 i \), which is option A.\begin{enumerate}[label=\Alph*.]
\item \( a \in [82.6, 84.1] \text{ and } b \in [-47.66, -46.81] \)

* $84 - 47 i$, which is the correct option.
\item \( a \in [93.2, 97.7] \text{ and } b \in [6.8, 7.95] \)

 $96 + 7 i$, which corresponds to adding a minus sign in the first term.
\item \( a \in [93.2, 97.7] \text{ and } b \in [-7.15, -4.69] \)

 $96 - 7 i$, which corresponds to adding a minus sign in the second term.
\item \( a \in [82.6, 84.1] \text{ and } b \in [46.23, 47.92] \)

 $84 + 47 i$, which corresponds to adding a minus sign in both terms.
\item \( a \in [85.4, 90.6] \text{ and } b \in [4.47, 6.03] \)

 $90 + 6 i$, which corresponds to just multiplying the real terms to get the real part of the solution and the coefficients in the complex terms to get the complex part.
\end{enumerate}

\textbf{General Comment:} You can treat $i$ as a variable and distribute. Just remember that $i^2=-1$, so you can continue to reduce after you distribute.
}
\litem{
Simplify the expression below into the form $a+bi$. Then, choose the intervals that $a$ and $b$ belong to.
\[ \frac{72 + 22 i}{-6 - i} \]The solution is \( -12.27  - 1.62 i \), which is option A.\begin{enumerate}[label=\Alph*.]
\item \( a \in [-12.45, -12.05] \text{ and } b \in [-2.5, -1] \)

* $-12.27  - 1.62 i$, which is the correct option.
\item \( a \in [-11.17, -10.85] \text{ and } b \in [-6, -4.5] \)

 $-11.08  - 5.51 i$, which corresponds to forgetting to multiply the conjugate by the numerator and not computing the conjugate correctly.
\item \( a \in [-12.12, -11.76] \text{ and } b \in [-23, -21.5] \)

 $-12.00  - 22.00 i$, which corresponds to just dividing the first term by the first term and the second by the second.
\item \( a \in [-12.45, -12.05] \text{ and } b \in [-60.5, -59.5] \)

 $-12.27  - 60.00 i$, which corresponds to forgetting to multiply the conjugate by the numerator.
\item \( a \in [-454.37, -453.81] \text{ and } b \in [-2.5, -1] \)

 $-454.00  - 1.62 i$, which corresponds to forgetting to multiply the conjugate by the numerator and using a plus instead of a minus in the denominator.
\end{enumerate}

\textbf{General Comment:} Multiply the numerator and denominator by the *conjugate* of the denominator, then simplify. For example, if we have $2+3i$, the conjugate is $2-3i$.
}
\litem{
Choose the \textbf{smallest} set of Real numbers that the number below belongs to.
\[ -\sqrt{\frac{1560}{12}} \]The solution is \( \text{Irrational} \), which is option B.\begin{enumerate}[label=\Alph*.]
\item \( \text{Not a Real number} \)

These are Nonreal Complex numbers \textbf{OR} things that are not numbers (e.g., dividing by 0).
\item \( \text{Irrational} \)

* This is the correct option!
\item \( \text{Whole} \)

These are the counting numbers with 0 (0, 1, 2, 3, ...)
\item \( \text{Integer} \)

These are the negative and positive counting numbers (..., -3, -2, -1, 0, 1, 2, 3, ...)
\item \( \text{Rational} \)

These are numbers that can be written as fraction of Integers (e.g., -2/3)
\end{enumerate}

\textbf{General Comment:} First, you \textbf{NEED} to simplify the expression. This question simplifies to $-\sqrt{130}$. 
 
 Be sure you look at the simplified fraction and not just the decimal expansion. Numbers such as 13, 17, and 19 provide \textbf{long but repeating/terminating decimal expansions!} 
 
 The only ways to *not* be a Real number are: dividing by 0 or taking the square root of a negative number. 
 
 Irrational numbers are more than just square root of 3: adding or subtracting values from square root of 3 is also irrational.
}
\end{enumerate}

\end{document}