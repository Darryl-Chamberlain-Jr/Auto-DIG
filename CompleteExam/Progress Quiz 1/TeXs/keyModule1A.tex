\documentclass{extbook}[14pt]
\usepackage{multicol, enumerate, enumitem, hyperref, color, soul, setspace, parskip, fancyhdr, amssymb, amsthm, amsmath, bbm, latexsym, units, mathtools}
\everymath{\displaystyle}
\usepackage[headsep=0.5cm,headheight=0cm, left=1 in,right= 1 in,top= 1 in,bottom= 1 in]{geometry}
\usepackage{dashrule}  % Package to use the command below to create lines between items
\newcommand{\litem}[1]{\item #1

\rule{\textwidth}{0.4pt}}
\pagestyle{fancy}
\lhead{}
\chead{Answer Key for Progress Quiz 1 Version A}
\rhead{}
\lfoot{3939-9803}
\cfoot{}
\rfoot{Fall 2020}
\begin{document}
\textbf{This key should allow you to understand why you choose the option you did (beyond just getting a question right or wrong). \href{https://xronos.clas.ufl.edu/mac1105spring2020/courseDescriptionAndMisc/Exams/LearningFromResults}{More instructions on how to use this key can be found here}.}

\textbf{If you have a suggestion to make the keys better, \href{https://forms.gle/CZkbZmPbC9XALEE88}{please fill out the short survey here}.}

\textit{Note: This key is auto-generated and may contain issues and/or errors. The keys are reviewed after each exam to ensure grading is done accurately. If there are issues (like duplicate options), they are noted in the offline gradebook. The keys are a work-in-progress to give students as many resources to improve as possible.}

\rule{\textwidth}{0.4pt}

\begin{enumerate}\litem{
Simplify the expression below into the form $a+bi$. Then, choose the intervals that $a$ and $b$ belong to.
\[ \frac{-54 - 11 i}{5 + 3 i} \]
The solution is \( -8.91  + 3.15 i \), which is option D.\begin{enumerate}[label=\Alph*.]
\item \( a \in [-12, -10.5] \text{ and } b \in [-4, -2] \)
 $-10.80  - 3.67 i$, which corresponds to just dividing the first term by the first term and the second by the second.
\item \( a \in [-303.5, -302.5] \text{ and } b \in [2, 4] \)
 $-303.00  + 3.15 i$, which corresponds to forgetting to multiply the conjugate by the numerator and using a plus instead of a minus in the denominator.
\item \( a \in [-7.5, -6.5] \text{ and } b \in [-7, -6] \)
 $-6.97  - 6.38 i$, which corresponds to forgetting to multiply the conjugate by the numerator and not computing the conjugate correctly.
\item \( a \in [-10, -8.5] \text{ and } b \in [2, 4] \)
* $-8.91  + 3.15 i$, which is the correct option.
\item \( a \in [-10, -8.5] \text{ and } b \in [105.5, 108.5] \)
 $-8.91  + 107.00 i$, which corresponds to forgetting to multiply the conjugate by the numerator.
\end{enumerate}

\textbf{General Comment:} Multiply the numerator and denominator by the *conjugate* of the denominator, then simplify. For example, if we have $2+3i$, the conjugate is $2-3i$.
}
\litem{
Simplify the expression below into the form $a+bi$. Then, choose the intervals that $a$ and $b$ belong to.
\[ (10 - 9 i)(8 + 6 i) \]
The solution is \( 134 - 12 i \), which is option B.\begin{enumerate}[label=\Alph*.]
\item \( a \in [75, 81] \text{ and } b \in [-59, -49] \)
 $80 - 54 i$, which corresponds to just multiplying the real terms to get the real part of the solution and the coefficients in the complex terms to get the complex part.
\item \( a \in [131, 140] \text{ and } b \in [-18, -2] \)
* $134 - 12 i$, which is the correct option.
\item \( a \in [20, 28] \text{ and } b \in [-134, -130] \)
 $26 - 132 i$, which corresponds to adding a minus sign in the second term.
\item \( a \in [20, 28] \text{ and } b \in [131, 135] \)
 $26 + 132 i$, which corresponds to adding a minus sign in the first term.
\item \( a \in [131, 140] \text{ and } b \in [7, 21] \)
 $134 + 12 i$, which corresponds to adding a minus sign in both terms.
\end{enumerate}

\textbf{General Comment:} You can treat $i$ as a variable and distribute. Just remember that $i^2=-1$, so you can continue to reduce after you distribute.
}
\litem{
Simplify the expression below and choose the interval the simplification is contained within.
\[ 1 - 8 \div 19 * 5 - (10 * 3) \]
The solution is \( -31.105 \), which is option C.\begin{enumerate}[label=\Alph*.]
\item \( [-30, -27.4] \)
 -29.084, which corresponds to an Order of Operations error: not reading left-to-right for multiplication/division.
\item \( [29, 33] \)
 30.916, which corresponds to not distributing addition and subtraction correctly.
\item \( [-32.3, -30.3] \)
* -31.105, which is the correct option.
\item \( [-34.5, -32.9] \)
 -33.316, which corresponds to not distributing a negative correctly.
\item \( \text{None of the above} \)
 You may have gotten this by making an unanticipated error. If you got a value that is not any of the others, please let the coordinator know so they can help you figure out what happened.
\end{enumerate}

\textbf{General Comment:} While you may remember (or were taught) PEMDAS is done in order, it is actually done as P/E/MD/AS. When we are at MD or AS, we read left to right.
}
\litem{
Choose the \textbf{smallest} set of Complex numbers that the number below belongs to.
\[ \sqrt{\frac{-560}{5}}+\sqrt{0}i \]
The solution is \( \text{Pure Imaginary} \), which is option C.\begin{enumerate}[label=\Alph*.]
\item \( \text{Not a Complex Number} \)
This is not a number. The only non-Complex number we know is dividing by 0 as this is not a number!
\item \( \text{Irrational} \)
These cannot be written as a fraction of Integers. Remember: $\pi$ is not an Integer!
\item \( \text{Pure Imaginary} \)
* This is the correct option!
\item \( \text{Nonreal Complex} \)
This is a Complex number $(a+bi)$ that is not Real (has $i$ as part of the number).
\item \( \text{Rational} \)
These are numbers that can be written as fraction of Integers (e.g., -2/3 + 5)
\end{enumerate}

\textbf{General Comment:} Be sure to simplify $i^2 = -1$. This may remove the imaginary portion for your number. If you are having trouble, you may want to look at the \textit{Subgroups of the Real Numbers} section.
}
\litem{
Choose the \textbf{smallest} set of Real numbers that the number below belongs to.
\[ \sqrt{\frac{1170}{10}} \]
The solution is \( \text{Irrational} \), which is option E.\begin{enumerate}[label=\Alph*.]
\item \( \text{Integer} \)
These are the negative and positive counting numbers (..., -3, -2, -1, 0, 1, 2, 3, ...)
\item \( \text{Not a Real number} \)
These are Nonreal Complex numbers \textbf{OR} things that are not numbers (e.g., dividing by 0).
\item \( \text{Rational} \)
These are numbers that can be written as fraction of Integers (e.g., -2/3)
\item \( \text{Whole} \)
These are the counting numbers with 0 (0, 1, 2, 3, ...)
\item \( \text{Irrational} \)
* This is the correct option!
\end{enumerate}

\textbf{General Comment:} First, you \textbf{NEED} to simplify the expression. This question simplifies to $\sqrt{117}$. 
 
 Be sure you look at the simplified fraction and not just the decimal expansion. Numbers such as 13, 17, and 19 provide \textbf{long but repeating/terminating decimal expansions!} 
 
 The only ways to *not* be a Real number are: dividing by 0 or taking the square root of a negative number. 
 
 Irrational numbers are more than just square root of 3: adding or subtracting values from square root of 3 is also irrational.
}
\litem{
Simplify the expression below into the form $a+bi$. Then, choose the intervals that $a$ and $b$ belong to.
\[ \frac{72 - 77 i}{3 - 4 i} \]
The solution is \( 20.96  + 2.28 i \), which is option D.\begin{enumerate}[label=\Alph*.]
\item \( a \in [-4.5, -2.5] \text{ and } b \in [-22, -19.5] \)
 $-3.68  - 20.76 i$, which corresponds to forgetting to multiply the conjugate by the numerator and not computing the conjugate correctly.
\item \( a \in [23.5, 24.5] \text{ and } b \in [18.5, 21] \)
 $24.00  + 19.25 i$, which corresponds to just dividing the first term by the first term and the second by the second.
\item \( a \in [523, 526.5] \text{ and } b \in [1.5, 3.5] \)
 $524.00  + 2.28 i$, which corresponds to forgetting to multiply the conjugate by the numerator and using a plus instead of a minus in the denominator.
\item \( a \in [19.5, 21.5] \text{ and } b \in [1.5, 3.5] \)
* $20.96  + 2.28 i$, which is the correct option.
\item \( a \in [19.5, 21.5] \text{ and } b \in [56, 57.5] \)
 $20.96  + 57.00 i$, which corresponds to forgetting to multiply the conjugate by the numerator.
\end{enumerate}

\textbf{General Comment:} Multiply the numerator and denominator by the *conjugate* of the denominator, then simplify. For example, if we have $2+3i$, the conjugate is $2-3i$.
}
\litem{
Simplify the expression below into the form $a+bi$. Then, choose the intervals that $a$ and $b$ belong to.
\[ (-3 - 7 i)(10 + 2 i) \]
The solution is \( -16 - 76 i \), which is option B.\begin{enumerate}[label=\Alph*.]
\item \( a \in [-44, -38] \text{ and } b \in [-66, -61] \)
 $-44 - 64 i$, which corresponds to adding a minus sign in the second term.
\item \( a \in [-20, -9] \text{ and } b \in [-82, -75] \)
* $-16 - 76 i$, which is the correct option.
\item \( a \in [-35, -26] \text{ and } b \in [-17, -11] \)
 $-30 - 14 i$, which corresponds to just multiplying the real terms to get the real part of the solution and the coefficients in the complex terms to get the complex part.
\item \( a \in [-44, -38] \text{ and } b \in [64, 68] \)
 $-44 + 64 i$, which corresponds to adding a minus sign in the first term.
\item \( a \in [-20, -9] \text{ and } b \in [70, 77] \)
 $-16 + 76 i$, which corresponds to adding a minus sign in both terms.
\end{enumerate}

\textbf{General Comment:} You can treat $i$ as a variable and distribute. Just remember that $i^2=-1$, so you can continue to reduce after you distribute.
}
\litem{
Simplify the expression below and choose the interval the simplification is contained within.
\[ 17 - 4^2 + 8 \div 7 * 20 \div 9 \]
The solution is \( 3.540 \), which is option B.\begin{enumerate}[label=\Alph*.]
\item \( [30.6, 33.4] \)
 33.006, which corresponds to two Order of Operations errors.
\item \( [2.1, 6.2] \)
* 3.540, this is the correct option
\item \( [35.2, 39.4] \)
 35.540, which corresponds to an Order of Operations error: multiplying by negative before squaring. For example: $(-3)^2 \neq -3^2$
\item \( [-1.4, 3] \)
 1.006, which corresponds to an Order of Operations error: not reading left-to-right for multiplication/division.
\item \( \text{None of the above} \)
 You may have gotten this by making an unanticipated error. If you got a value that is not any of the others, please let the coordinator know so they can help you figure out what happened.
\end{enumerate}

\textbf{General Comment:} While you may remember (or were taught) PEMDAS is done in order, it is actually done as P/E/MD/AS. When we are at MD or AS, we read left to right.
}
\litem{
Choose the \textbf{smallest} set of Complex numbers that the number below belongs to.
\[ -\sqrt{\frac{36}{49}} + 100i^2 \]
The solution is \( \text{Rational} \), which is option A.\begin{enumerate}[label=\Alph*.]
\item \( \text{Rational} \)
* This is the correct option!
\item \( \text{Irrational} \)
These cannot be written as a fraction of Integers. Remember: $\pi$ is not an Integer!
\item \( \text{Pure Imaginary} \)
This is a Complex number $(a+bi)$ that \textbf{only} has an imaginary part like $2i$.
\item \( \text{Not a Complex Number} \)
This is not a number. The only non-Complex number we know is dividing by 0 as this is not a number!
\item \( \text{Nonreal Complex} \)
This is a Complex number $(a+bi)$ that is not Real (has $i$ as part of the number).
\end{enumerate}

\textbf{General Comment:} Be sure to simplify $i^2 = -1$. This may remove the imaginary portion for your number. If you are having trouble, you may want to look at the \textit{Subgroups of the Real Numbers} section.
}
\litem{
Choose the \textbf{smallest} set of Real numbers that the number below belongs to.
\[ \sqrt{\frac{400}{5}} \]
The solution is \( \text{Irrational} \), which is option C.\begin{enumerate}[label=\Alph*.]
\item \( \text{Rational} \)
These are numbers that can be written as fraction of Integers (e.g., -2/3)
\item \( \text{Not a Real number} \)
These are Nonreal Complex numbers \textbf{OR} things that are not numbers (e.g., dividing by 0).
\item \( \text{Irrational} \)
* This is the correct option!
\item \( \text{Integer} \)
These are the negative and positive counting numbers (..., -3, -2, -1, 0, 1, 2, 3, ...)
\item \( \text{Whole} \)
These are the counting numbers with 0 (0, 1, 2, 3, ...)
\end{enumerate}

\textbf{General Comment:} First, you \textbf{NEED} to simplify the expression. This question simplifies to $\sqrt{80}$. 
 
 Be sure you look at the simplified fraction and not just the decimal expansion. Numbers such as 13, 17, and 19 provide \textbf{long but repeating/terminating decimal expansions!} 
 
 The only ways to *not* be a Real number are: dividing by 0 or taking the square root of a negative number. 
 
 Irrational numbers are more than just square root of 3: adding or subtracting values from square root of 3 is also irrational.
}
\end{enumerate}

\end{document}