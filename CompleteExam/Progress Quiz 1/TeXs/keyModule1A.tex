\documentclass{extbook}[14pt]
\usepackage{multicol, enumerate, enumitem, hyperref, color, soul, setspace, parskip, fancyhdr, amssymb, amsthm, amsmath, latexsym, units, mathtools}
\everymath{\displaystyle}
\usepackage[headsep=0.5cm,headheight=0cm, left=1 in,right= 1 in,top= 1 in,bottom= 1 in]{geometry}
\usepackage{dashrule}  % Package to use the command below to create lines between items
\newcommand{\litem}[1]{\item #1

\rule{\textwidth}{0.4pt}}
\pagestyle{fancy}
\lhead{}
\chead{Answer Key for Progress Quiz 1 Version A}
\rhead{}
\lfoot{5899-4682}
\cfoot{}
\rfoot{Spring 2021}
\begin{document}
\textbf{This key should allow you to understand why you choose the option you did (beyond just getting a question right or wrong). \href{https://xronos.clas.ufl.edu/mac1105spring2020/courseDescriptionAndMisc/Exams/LearningFromResults}{More instructions on how to use this key can be found here}.}

\textbf{If you have a suggestion to make the keys better, \href{https://forms.gle/CZkbZmPbC9XALEE88}{please fill out the short survey here}.}

\textit{Note: This key is auto-generated and may contain issues and/or errors. The keys are reviewed after each exam to ensure grading is done accurately. If there are issues (like duplicate options), they are noted in the offline gradebook. The keys are a work-in-progress to give students as many resources to improve as possible.}

\rule{\textwidth}{0.4pt}

\begin{enumerate}\litem{
Choose the \textbf{smallest} set of Real numbers that the number below belongs to.
\[ -\sqrt{\frac{1404}{12}} \]The solution is \( \text{Irrational} \), which is option A.\begin{enumerate}[label=\Alph*.]
\item \( \text{Irrational} \)

* This is the correct option!
\item \( \text{Not a Real number} \)

These are Nonreal Complex numbers \textbf{OR} things that are not numbers (e.g., dividing by 0).
\item \( \text{Rational} \)

These are numbers that can be written as fraction of Integers (e.g., -2/3)
\item \( \text{Integer} \)

These are the negative and positive counting numbers (..., -3, -2, -1, 0, 1, 2, 3, ...)
\item \( \text{Whole} \)

These are the counting numbers with 0 (0, 1, 2, 3, ...)
\end{enumerate}

\textbf{General Comment:} First, you \textbf{NEED} to simplify the expression. This question simplifies to $-\sqrt{117}$. 
 
 Be sure you look at the simplified fraction and not just the decimal expansion. Numbers such as 13, 17, and 19 provide \textbf{long but repeating/terminating decimal expansions!} 
 
 The only ways to *not* be a Real number are: dividing by 0 or taking the square root of a negative number. 
 
 Irrational numbers are more than just square root of 3: adding or subtracting values from square root of 3 is also irrational.
}
\litem{
Choose the \textbf{smallest} set of Real numbers that the number below belongs to.
\[ -\sqrt{\frac{81}{196}} \]The solution is \( \text{Rational} \), which is option E.\begin{enumerate}[label=\Alph*.]
\item \( \text{Not a Real number} \)

These are Nonreal Complex numbers \textbf{OR} things that are not numbers (e.g., dividing by 0).
\item \( \text{Irrational} \)

These cannot be written as a fraction of Integers.
\item \( \text{Integer} \)

These are the negative and positive counting numbers (..., -3, -2, -1, 0, 1, 2, 3, ...)
\item \( \text{Whole} \)

These are the counting numbers with 0 (0, 1, 2, 3, ...)
\item \( \text{Rational} \)

* This is the correct option!
\end{enumerate}

\textbf{General Comment:} First, you \textbf{NEED} to simplify the expression. This question simplifies to $-\frac{9}{14}$. 
 
 Be sure you look at the simplified fraction and not just the decimal expansion. Numbers such as 13, 17, and 19 provide \textbf{long but repeating/terminating decimal expansions!} 
 
 The only ways to *not* be a Real number are: dividing by 0 or taking the square root of a negative number. 
 
 Irrational numbers are more than just square root of 3: adding or subtracting values from square root of 3 is also irrational.
}
\litem{
Simplify the expression below into the form $a+bi$. Then, choose the intervals that $a$ and $b$ belong to.
\[ (-7 + 9 i)(-5 - 10 i) \]The solution is \( 125 + 25 i \), which is option B.\begin{enumerate}[label=\Alph*.]
\item \( a \in [-57, -54] \text{ and } b \in [-119, -114] \)

 $-55 - 115 i$, which corresponds to adding a minus sign in the second term.
\item \( a \in [122, 129] \text{ and } b \in [24, 27] \)

* $125 + 25 i$, which is the correct option.
\item \( a \in [-57, -54] \text{ and } b \in [111, 121] \)

 $-55 + 115 i$, which corresponds to adding a minus sign in the first term.
\item \( a \in [33, 36] \text{ and } b \in [-90, -86] \)

 $35 - 90 i$, which corresponds to just multiplying the real terms to get the real part of the solution and the coefficients in the complex terms to get the complex part.
\item \( a \in [122, 129] \text{ and } b \in [-25, -19] \)

 $125 - 25 i$, which corresponds to adding a minus sign in both terms.
\end{enumerate}

\textbf{General Comment:} You can treat $i$ as a variable and distribute. Just remember that $i^2=-1$, so you can continue to reduce after you distribute.
}
\litem{
Simplify the expression below into the form $a+bi$. Then, choose the intervals that $a$ and $b$ belong to.
\[ \frac{-9 + 66 i}{2 + 8 i} \]The solution is \( 7.50  + 3.00 i \), which is option A.\begin{enumerate}[label=\Alph*.]
\item \( a \in [7, 8] \text{ and } b \in [1.5, 5] \)

* $7.50  + 3.00 i$, which is the correct option.
\item \( a \in [-9, -7] \text{ and } b \in [0.5, 1.5] \)

 $-8.03  + 0.88 i$, which corresponds to forgetting to multiply the conjugate by the numerator and not computing the conjugate correctly.
\item \( a \in [509, 510.5] \text{ and } b \in [1.5, 5] \)

 $510.00  + 3.00 i$, which corresponds to forgetting to multiply the conjugate by the numerator and using a plus instead of a minus in the denominator.
\item \( a \in [-6, -3.5] \text{ and } b \in [6.5, 9.5] \)

 $-4.50  + 8.25 i$, which corresponds to just dividing the first term by the first term and the second by the second.
\item \( a \in [7, 8] \text{ and } b \in [203, 204.5] \)

 $7.50  + 204.00 i$, which corresponds to forgetting to multiply the conjugate by the numerator.
\end{enumerate}

\textbf{General Comment:} Multiply the numerator and denominator by the *conjugate* of the denominator, then simplify. For example, if we have $2+3i$, the conjugate is $2-3i$.
}
\litem{
Choose the \textbf{smallest} set of Complex numbers that the number below belongs to.
\[ \frac{-12}{-11}+\sqrt{-25}i \]The solution is \( \text{Rational} \), which is option B.\begin{enumerate}[label=\Alph*.]
\item \( \text{Nonreal Complex} \)

This is a Complex number $(a+bi)$ that is not Real (has $i$ as part of the number).
\item \( \text{Rational} \)

* This is the correct option!
\item \( \text{Irrational} \)

These cannot be written as a fraction of Integers. Remember: $\pi$ is not an Integer!
\item \( \text{Not a Complex Number} \)

This is not a number. The only non-Complex number we know is dividing by 0 as this is not a number!
\item \( \text{Pure Imaginary} \)

This is a Complex number $(a+bi)$ that \textbf{only} has an imaginary part like $2i$.
\end{enumerate}

\textbf{General Comment:} Be sure to simplify $i^2 = -1$. This may remove the imaginary portion for your number. If you are having trouble, you may want to look at the \textit{Subgroups of the Real Numbers} section.
}
\litem{
Simplify the expression below and choose the interval the simplification is contained within.
\[ 12 - 3 \div 10 * 15 - (9 * 2) \]The solution is \( -10.500 \), which is option C.\begin{enumerate}[label=\Alph*.]
\item \( [27.9, 33.7] \)

 29.980, which corresponds to not distributing addition and subtraction correctly.
\item \( [-5.4, -1] \)

 -3.000, which corresponds to not distributing a negative correctly.
\item \( [-11.7, -8] \)

* -10.500, which is the correct option.
\item \( [-8.6, -5.5] \)

 -6.020, which corresponds to an Order of Operations error: not reading left-to-right for multiplication/division.
\item \( \text{None of the above} \)

 You may have gotten this by making an unanticipated error. If you got a value that is not any of the others, please let the coordinator know so they can help you figure out what happened.
\end{enumerate}

\textbf{General Comment:} While you may remember (or were taught) PEMDAS is done in order, it is actually done as P/E/MD/AS. When we are at MD or AS, we read left to right.
}
\litem{
Simplify the expression below and choose the interval the simplification is contained within.
\[ 17 - 2 \div 1 * 9 - (16 * 13) \]The solution is \( -209.000 \), which is option B.\begin{enumerate}[label=\Alph*.]
\item \( [223.78, 225.78] \)

 224.778, which corresponds to not distributing addition and subtraction correctly.
\item \( [-210, -204] \)

* -209.000, which is the correct option.
\item \( [-194.22, -185.22] \)

 -191.222, which corresponds to an Order of Operations error: not reading left-to-right for multiplication/division.
\item \( [-224, -218] \)

 -221.000, which corresponds to not distributing a negative correctly.
\item \( \text{None of the above} \)

 You may have gotten this by making an unanticipated error. If you got a value that is not any of the others, please let the coordinator know so they can help you figure out what happened.
\end{enumerate}

\textbf{General Comment:} While you may remember (or were taught) PEMDAS is done in order, it is actually done as P/E/MD/AS. When we are at MD or AS, we read left to right.
}
\litem{
Choose the \textbf{smallest} set of Complex numbers that the number below belongs to.
\[ \frac{15}{-17}+4i^2 \]The solution is \( \text{Rational} \), which is option C.\begin{enumerate}[label=\Alph*.]
\item \( \text{Nonreal Complex} \)

This is a Complex number $(a+bi)$ that is not Real (has $i$ as part of the number).
\item \( \text{Pure Imaginary} \)

This is a Complex number $(a+bi)$ that \textbf{only} has an imaginary part like $2i$.
\item \( \text{Rational} \)

* This is the correct option!
\item \( \text{Irrational} \)

These cannot be written as a fraction of Integers. Remember: $\pi$ is not an Integer!
\item \( \text{Not a Complex Number} \)

This is not a number. The only non-Complex number we know is dividing by 0 as this is not a number!
\end{enumerate}

\textbf{General Comment:} Be sure to simplify $i^2 = -1$. This may remove the imaginary portion for your number. If you are having trouble, you may want to look at the \textit{Subgroups of the Real Numbers} section.
}
\litem{
Simplify the expression below into the form $a+bi$. Then, choose the intervals that $a$ and $b$ belong to.
\[ (10 - 4 i)(-6 + 7 i) \]The solution is \( -32 + 94 i \), which is option E.\begin{enumerate}[label=\Alph*.]
\item \( a \in [-63, -58] \text{ and } b \in [-34, -25] \)

 $-60 - 28 i$, which corresponds to just multiplying the real terms to get the real part of the solution and the coefficients in the complex terms to get the complex part.
\item \( a \in [-91, -85] \text{ and } b \in [-49, -40] \)

 $-88 - 46 i$, which corresponds to adding a minus sign in the second term.
\item \( a \in [-91, -85] \text{ and } b \in [46, 54] \)

 $-88 + 46 i$, which corresponds to adding a minus sign in the first term.
\item \( a \in [-38, -29] \text{ and } b \in [-96, -91] \)

 $-32 - 94 i$, which corresponds to adding a minus sign in both terms.
\item \( a \in [-38, -29] \text{ and } b \in [92, 95] \)

* $-32 + 94 i$, which is the correct option.
\end{enumerate}

\textbf{General Comment:} You can treat $i$ as a variable and distribute. Just remember that $i^2=-1$, so you can continue to reduce after you distribute.
}
\litem{
Simplify the expression below into the form $a+bi$. Then, choose the intervals that $a$ and $b$ belong to.
\[ \frac{27 - 55 i}{-7 + 6 i} \]The solution is \( -6.11  + 2.62 i \), which is option B.\begin{enumerate}[label=\Alph*.]
\item \( a \in [-7, -5] \text{ and } b \in [222, 224] \)

 $-6.11  + 223.00 i$, which corresponds to forgetting to multiply the conjugate by the numerator.
\item \( a \in [-7, -5] \text{ and } b \in [2.5, 3] \)

* $-6.11  + 2.62 i$, which is the correct option.
\item \( a \in [0.5, 2] \text{ and } b \in [5, 7] \)

 $1.66  + 6.44 i$, which corresponds to forgetting to multiply the conjugate by the numerator and not computing the conjugate correctly.
\item \( a \in [-520, -518] \text{ and } b \in [2.5, 3] \)

 $-519.00  + 2.62 i$, which corresponds to forgetting to multiply the conjugate by the numerator and using a plus instead of a minus in the denominator.
\item \( a \in [-4, -3] \text{ and } b \in [-10.5, -8.5] \)

 $-3.86  - 9.17 i$, which corresponds to just dividing the first term by the first term and the second by the second.
\end{enumerate}

\textbf{General Comment:} Multiply the numerator and denominator by the *conjugate* of the denominator, then simplify. For example, if we have $2+3i$, the conjugate is $2-3i$.
}
\end{enumerate}

\end{document}