\documentclass{extbook}[14pt]
\usepackage{multicol, enumerate, enumitem, hyperref, color, soul, setspace, parskip, fancyhdr, amssymb, amsthm, amsmath, latexsym, units, mathtools}
\everymath{\displaystyle}
\usepackage[headsep=0.5cm,headheight=0cm, left=1 in,right= 1 in,top= 1 in,bottom= 1 in]{geometry}
\usepackage{dashrule}  % Package to use the command below to create lines between items
\newcommand{\litem}[1]{\item #1

\rule{\textwidth}{0.4pt}}
\pagestyle{fancy}
\lhead{}
\chead{Answer Key for Progress Quiz 1 Version A}
\rhead{}
\lfoot{3629-3146}
\cfoot{}
\rfoot{Summer C 2021}
\begin{document}
\textbf{This key should allow you to understand why you choose the option you did (beyond just getting a question right or wrong). \href{https://xronos.clas.ufl.edu/mac1105spring2020/courseDescriptionAndMisc/Exams/LearningFromResults}{More instructions on how to use this key can be found here}.}

\textbf{If you have a suggestion to make the keys better, \href{https://forms.gle/CZkbZmPbC9XALEE88}{please fill out the short survey here}.}

\textit{Note: This key is auto-generated and may contain issues and/or errors. The keys are reviewed after each exam to ensure grading is done accurately. If there are issues (like duplicate options), they are noted in the offline gradebook. The keys are a work-in-progress to give students as many resources to improve as possible.}

\rule{\textwidth}{0.4pt}

\begin{enumerate}\litem{
Simplify the expression below into the form $a+bi$. Then, choose the intervals that $a$ and $b$ belong to.
\[ \frac{72 - 66 i}{-1 + 5 i} \]The solution is \( -15.46  - 11.31 i \), which is option D.\begin{enumerate}[label=\Alph*.]
\item \( a \in [-73, -71.5] \text{ and } b \in [-14.5, -12.5] \)

 $-72.00  - 13.20 i$, which corresponds to just dividing the first term by the first term and the second by the second.
\item \( a \in [-16.5, -15] \text{ and } b \in [-295, -293.5] \)

 $-15.46  - 294.00 i$, which corresponds to forgetting to multiply the conjugate by the numerator.
\item \( a \in [9, 11] \text{ and } b \in [16, 17.5] \)

 $9.92  + 16.38 i$, which corresponds to forgetting to multiply the conjugate by the numerator and not computing the conjugate correctly.
\item \( a \in [-16.5, -15] \text{ and } b \in [-11.5, -10.5] \)

* $-15.46  - 11.31 i$, which is the correct option.
\item \( a \in [-402.5, -401] \text{ and } b \in [-11.5, -10.5] \)

 $-402.00  - 11.31 i$, which corresponds to forgetting to multiply the conjugate by the numerator and using a plus instead of a minus in the denominator.
\end{enumerate}

\textbf{General Comment:} Multiply the numerator and denominator by the *conjugate* of the denominator, then simplify. For example, if we have $2+3i$, the conjugate is $2-3i$.
}
\litem{
Simplify the expression below and choose the interval the simplification is contained within.
\[ 17 - 14^2 + 5 \div 4 * 15 \div 13 \]The solution is \( -177.558 \), which is option A.\begin{enumerate}[label=\Alph*.]
\item \( [-177.68, -176.84] \)

* -177.558, this is the correct option
\item \( [-179.24, -177.84] \)

 -178.994, which corresponds to an Order of Operations error: not reading left-to-right for multiplication/division.
\item \( [213.8, 215.01] \)

 214.442, which corresponds to an Order of Operations error: multiplying by negative before squaring. For example: $(-3)^2 \neq -3^2$
\item \( [212.66, 214.13] \)

 213.006, which corresponds to two Order of Operations errors.
\item \( \text{None of the above} \)

 You may have gotten this by making an unanticipated error. If you got a value that is not any of the others, please let the coordinator know so they can help you figure out what happened.
\end{enumerate}

\textbf{General Comment:} While you may remember (or were taught) PEMDAS is done in order, it is actually done as P/E/MD/AS. When we are at MD or AS, we read left to right.
}
\litem{
Choose the \textbf{smallest} set of Real numbers that the number below belongs to.
\[ \sqrt{\frac{1980}{12}} \]The solution is \( \text{Irrational} \), which is option A.\begin{enumerate}[label=\Alph*.]
\item \( \text{Irrational} \)

* This is the correct option!
\item \( \text{Whole} \)

These are the counting numbers with 0 (0, 1, 2, 3, ...)
\item \( \text{Integer} \)

These are the negative and positive counting numbers (..., -3, -2, -1, 0, 1, 2, 3, ...)
\item \( \text{Not a Real number} \)

These are Nonreal Complex numbers \textbf{OR} things that are not numbers (e.g., dividing by 0).
\item \( \text{Rational} \)

These are numbers that can be written as fraction of Integers (e.g., -2/3)
\end{enumerate}

\textbf{General Comment:} First, you \textbf{NEED} to simplify the expression. This question simplifies to $\sqrt{165}$. 
 
 Be sure you look at the simplified fraction and not just the decimal expansion. Numbers such as 13, 17, and 19 provide \textbf{long but repeating/terminating decimal expansions!} 
 
 The only ways to *not* be a Real number are: dividing by 0 or taking the square root of a negative number. 
 
 Irrational numbers are more than just square root of 3: adding or subtracting values from square root of 3 is also irrational.
}
\litem{
Choose the \textbf{smallest} set of Complex numbers that the number below belongs to.
\[ \sqrt{\frac{-450}{10}} i+\sqrt{208}i \]The solution is \( \text{Nonreal Complex} \), which is option B.\begin{enumerate}[label=\Alph*.]
\item \( \text{Pure Imaginary} \)

This is a Complex number $(a+bi)$ that \textbf{only} has an imaginary part like $2i$.
\item \( \text{Nonreal Complex} \)

* This is the correct option!
\item \( \text{Irrational} \)

These cannot be written as a fraction of Integers. Remember: $\pi$ is not an Integer!
\item \( \text{Rational} \)

These are numbers that can be written as fraction of Integers (e.g., -2/3 + 5)
\item \( \text{Not a Complex Number} \)

This is not a number. The only non-Complex number we know is dividing by 0 as this is not a number!
\end{enumerate}

\textbf{General Comment:} Be sure to simplify $i^2 = -1$. This may remove the imaginary portion for your number. If you are having trouble, you may want to look at the \textit{Subgroups of the Real Numbers} section.
}
\litem{
Simplify the expression below and choose the interval the simplification is contained within.
\[ 9 - 19 \div 15 * 4 - (6 * 14) \]The solution is \( -80.067 \), which is option C.\begin{enumerate}[label=\Alph*.]
\item \( [-34.93, -23.93] \)

 -28.933, which corresponds to not distributing a negative correctly.
\item \( [87.68, 98.68] \)

 92.683, which corresponds to not distributing addition and subtraction correctly.
\item \( [-80.07, -76.07] \)

* -80.067, which is the correct option.
\item \( [-76.32, -71.32] \)

 -75.317, which corresponds to an Order of Operations error: not reading left-to-right for multiplication/division.
\item \( \text{None of the above} \)

 You may have gotten this by making an unanticipated error. If you got a value that is not any of the others, please let the coordinator know so they can help you figure out what happened.
\end{enumerate}

\textbf{General Comment:} While you may remember (or were taught) PEMDAS is done in order, it is actually done as P/E/MD/AS. When we are at MD or AS, we read left to right.
}
\litem{
Simplify the expression below into the form $a+bi$. Then, choose the intervals that $a$ and $b$ belong to.
\[ \frac{9 - 55 i}{3 + 6 i} \]The solution is \( -6.73  - 4.87 i \), which is option A.\begin{enumerate}[label=\Alph*.]
\item \( a \in [-8, -6.5] \text{ and } b \in [-6, -4] \)

* $-6.73  - 4.87 i$, which is the correct option.
\item \( a \in [-303.5, -302] \text{ and } b \in [-6, -4] \)

 $-303.00  - 4.87 i$, which corresponds to forgetting to multiply the conjugate by the numerator and using a plus instead of a minus in the denominator.
\item \( a \in [-8, -6.5] \text{ and } b \in [-220, -218.5] \)

 $-6.73  - 219.00 i$, which corresponds to forgetting to multiply the conjugate by the numerator.
\item \( a \in [7, 8.5] \text{ and } b \in [-3, -1.5] \)

 $7.93  - 2.47 i$, which corresponds to forgetting to multiply the conjugate by the numerator and not computing the conjugate correctly.
\item \( a \in [2, 4.5] \text{ and } b \in [-11.5, -8] \)

 $3.00  - 9.17 i$, which corresponds to just dividing the first term by the first term and the second by the second.
\end{enumerate}

\textbf{General Comment:} Multiply the numerator and denominator by the *conjugate* of the denominator, then simplify. For example, if we have $2+3i$, the conjugate is $2-3i$.
}
\litem{
Choose the \textbf{smallest} set of Real numbers that the number below belongs to.
\[ -\sqrt{\frac{180625}{625}} \]The solution is \( \text{Integer} \), which is option C.\begin{enumerate}[label=\Alph*.]
\item \( \text{Whole} \)

These are the counting numbers with 0 (0, 1, 2, 3, ...)
\item \( \text{Not a Real number} \)

These are Nonreal Complex numbers \textbf{OR} things that are not numbers (e.g., dividing by 0).
\item \( \text{Integer} \)

* This is the correct option!
\item \( \text{Irrational} \)

These cannot be written as a fraction of Integers.
\item \( \text{Rational} \)

These are numbers that can be written as fraction of Integers (e.g., -2/3)
\end{enumerate}

\textbf{General Comment:} First, you \textbf{NEED} to simplify the expression. This question simplifies to $-425$. 
 
 Be sure you look at the simplified fraction and not just the decimal expansion. Numbers such as 13, 17, and 19 provide \textbf{long but repeating/terminating decimal expansions!} 
 
 The only ways to *not* be a Real number are: dividing by 0 or taking the square root of a negative number. 
 
 Irrational numbers are more than just square root of 3: adding or subtracting values from square root of 3 is also irrational.
}
\litem{
Simplify the expression below into the form $a+bi$. Then, choose the intervals that $a$ and $b$ belong to.
\[ (7 - 4 i)(10 + 9 i) \]The solution is \( 106 + 23 i \), which is option D.\begin{enumerate}[label=\Alph*.]
\item \( a \in [30, 35] \text{ and } b \in [103, 105] \)

 $34 + 103 i$, which corresponds to adding a minus sign in the first term.
\item \( a \in [103, 112] \text{ and } b \in [-25, -19] \)

 $106 - 23 i$, which corresponds to adding a minus sign in both terms.
\item \( a \in [30, 35] \text{ and } b \in [-107, -101] \)

 $34 - 103 i$, which corresponds to adding a minus sign in the second term.
\item \( a \in [103, 112] \text{ and } b \in [19, 24] \)

* $106 + 23 i$, which is the correct option.
\item \( a \in [68, 76] \text{ and } b \in [-36, -32] \)

 $70 - 36 i$, which corresponds to just multiplying the real terms to get the real part of the solution and the coefficients in the complex terms to get the complex part.
\end{enumerate}

\textbf{General Comment:} You can treat $i$ as a variable and distribute. Just remember that $i^2=-1$, so you can continue to reduce after you distribute.
}
\litem{
Choose the \textbf{smallest} set of Complex numbers that the number below belongs to.
\[ \sqrt{\frac{484}{289}} + 25i^2 \]The solution is \( \text{Rational} \), which is option E.\begin{enumerate}[label=\Alph*.]
\item \( \text{Irrational} \)

These cannot be written as a fraction of Integers. Remember: $\pi$ is not an Integer!
\item \( \text{Nonreal Complex} \)

This is a Complex number $(a+bi)$ that is not Real (has $i$ as part of the number).
\item \( \text{Not a Complex Number} \)

This is not a number. The only non-Complex number we know is dividing by 0 as this is not a number!
\item \( \text{Pure Imaginary} \)

This is a Complex number $(a+bi)$ that \textbf{only} has an imaginary part like $2i$.
\item \( \text{Rational} \)

* This is the correct option!
\end{enumerate}

\textbf{General Comment:} Be sure to simplify $i^2 = -1$. This may remove the imaginary portion for your number. If you are having trouble, you may want to look at the \textit{Subgroups of the Real Numbers} section.
}
\litem{
Simplify the expression below into the form $a+bi$. Then, choose the intervals that $a$ and $b$ belong to.
\[ (-5 - 3 i)(-4 - 8 i) \]The solution is \( -4 + 52 i \), which is option C.\begin{enumerate}[label=\Alph*.]
\item \( a \in [38, 49] \text{ and } b \in [-33, -25] \)

 $44 - 28 i$, which corresponds to adding a minus sign in the second term.
\item \( a \in [38, 49] \text{ and } b \in [27, 29] \)

 $44 + 28 i$, which corresponds to adding a minus sign in the first term.
\item \( a \in [-4, 3] \text{ and } b \in [52, 54] \)

* $-4 + 52 i$, which is the correct option.
\item \( a \in [-4, 3] \text{ and } b \in [-53, -50] \)

 $-4 - 52 i$, which corresponds to adding a minus sign in both terms.
\item \( a \in [19, 28] \text{ and } b \in [23, 26] \)

 $20 + 24 i$, which corresponds to just multiplying the real terms to get the real part of the solution and the coefficients in the complex terms to get the complex part.
\end{enumerate}

\textbf{General Comment:} You can treat $i$ as a variable and distribute. Just remember that $i^2=-1$, so you can continue to reduce after you distribute.
}
\end{enumerate}

\end{document}