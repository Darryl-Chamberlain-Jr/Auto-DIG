\documentclass{extbook}[14pt]
\usepackage{multicol, enumerate, enumitem, hyperref, color, soul, setspace, parskip, fancyhdr, amssymb, amsthm, amsmath, bbm, latexsym, units, mathtools}
\everymath{\displaystyle}
\usepackage[headsep=0.5cm,headheight=0cm, left=1 in,right= 1 in,top= 1 in,bottom= 1 in]{geometry}
\usepackage{dashrule}  % Package to use the command below to create lines between items
\newcommand{\litem}[1]{\item #1

\rule{\textwidth}{0.4pt}}
\pagestyle{fancy}
\lhead{}
\chead{Answer Key for Progress Quiz 1 Version A}
\rhead{}
\lfoot{3735-1698}
\cfoot{}
\rfoot{Spring 2021}
\begin{document}
\textbf{This key should allow you to understand why you choose the option you did (beyond just getting a question right or wrong). \href{https://xronos.clas.ufl.edu/mac1105spring2020/courseDescriptionAndMisc/Exams/LearningFromResults}{More instructions on how to use this key can be found here}.}

\textbf{If you have a suggestion to make the keys better, \href{https://forms.gle/CZkbZmPbC9XALEE88}{please fill out the short survey here}.}

\textit{Note: This key is auto-generated and may contain issues and/or errors. The keys are reviewed after each exam to ensure grading is done accurately. If there are issues (like duplicate options), they are noted in the offline gradebook. The keys are a work-in-progress to give students as many resources to improve as possible.}

\rule{\textwidth}{0.4pt}

\begin{enumerate}\litem{
Simplify the expression below into the form $a+bi$. Then, choose the intervals that $a$ and $b$ belong to.
\[ \frac{54 + 44 i}{7 - 5 i} \]

The solution is \( 2.14  + 7.81 i \), which is option B.\begin{enumerate}[label=\Alph*.]
\item \( a \in [7.75, 8.5] \text{ and } b \in [0, 1] \)

 $8.08  + 0.51 i$, which corresponds to forgetting to multiply the conjugate by the numerator and not computing the conjugate correctly.
\item \( a \in [1.9, 2.35] \text{ and } b \in [6.5, 9] \)

* $2.14  + 7.81 i$, which is the correct option.
\item \( a \in [7.25, 7.9] \text{ and } b \in [-9, -8] \)

 $7.71  - 8.80 i$, which corresponds to just dividing the first term by the first term and the second by the second.
\item \( a \in [1.9, 2.35] \text{ and } b \in [577.5, 578.5] \)

 $2.14  + 578.00 i$, which corresponds to forgetting to multiply the conjugate by the numerator.
\item \( a \in [157.75, 158.4] \text{ and } b \in [6.5, 9] \)

 $158.00  + 7.81 i$, which corresponds to forgetting to multiply the conjugate by the numerator and using a plus instead of a minus in the denominator.
\end{enumerate}

\textbf{General Comment:} Multiply the numerator and denominator by the *conjugate* of the denominator, then simplify. For example, if we have $2+3i$, the conjugate is $2-3i$.
}
\litem{
Choose the \textbf{smallest} set of Real numbers that the number below belongs to.
\[ \sqrt{\frac{-1980}{10}} \]

The solution is \( \text{Not a Real number} \), which is option D.\begin{enumerate}[label=\Alph*.]
\item \( \text{Irrational} \)

These cannot be written as a fraction of Integers.
\item \( \text{Whole} \)

These are the counting numbers with 0 (0, 1, 2, 3, ...)
\item \( \text{Integer} \)

These are the negative and positive counting numbers (..., -3, -2, -1, 0, 1, 2, 3, ...)
\item \( \text{Not a Real number} \)

* This is the correct option!
\item \( \text{Rational} \)

These are numbers that can be written as fraction of Integers (e.g., -2/3)
\end{enumerate}

\textbf{General Comment:} First, you \textbf{NEED} to simplify the expression. This question simplifies to $\sqrt{198} i$. 
 
 Be sure you look at the simplified fraction and not just the decimal expansion. Numbers such as 13, 17, and 19 provide \textbf{long but repeating/terminating decimal expansions!} 
 
 The only ways to *not* be a Real number are: dividing by 0 or taking the square root of a negative number. 
 
 Irrational numbers are more than just square root of 3: adding or subtracting values from square root of 3 is also irrational.
}
\litem{
Choose the \textbf{smallest} set of Complex numbers that the number below belongs to.
\[ \sqrt{\frac{-935}{11}}+\sqrt{0}i \]

The solution is \( \text{Pure Imaginary} \), which is option D.\begin{enumerate}[label=\Alph*.]
\item \( \text{Nonreal Complex} \)

This is a Complex number $(a+bi)$ that is not Real (has $i$ as part of the number).
\item \( \text{Rational} \)

These are numbers that can be written as fraction of Integers (e.g., -2/3 + 5)
\item \( \text{Irrational} \)

These cannot be written as a fraction of Integers. Remember: $\pi$ is not an Integer!
\item \( \text{Pure Imaginary} \)

* This is the correct option!
\item \( \text{Not a Complex Number} \)

This is not a number. The only non-Complex number we know is dividing by 0 as this is not a number!
\end{enumerate}

\textbf{General Comment:} Be sure to simplify $i^2 = -1$. This may remove the imaginary portion for your number. If you are having trouble, you may want to look at the \textit{Subgroups of the Real Numbers} section.
}
\litem{
Simplify the expression below into the form $a+bi$. Then, choose the intervals that $a$ and $b$ belong to.
\[ (5 - 6 i)(-10 - 7 i) \]

The solution is \( -92 + 25 i \), which is option C.\begin{enumerate}[label=\Alph*.]
\item \( a \in [-92, -89] \text{ and } b \in [-27, -23] \)

 $-92 - 25 i$, which corresponds to adding a minus sign in both terms.
\item \( a \in [-8, -5] \text{ and } b \in [90, 99] \)

 $-8 + 95 i$, which corresponds to adding a minus sign in the second term.
\item \( a \in [-92, -89] \text{ and } b \in [24, 29] \)

* $-92 + 25 i$, which is the correct option.
\item \( a \in [-50, -47] \text{ and } b \in [41, 47] \)

 $-50 + 42 i$, which corresponds to just multiplying the real terms to get the real part of the solution and the coefficients in the complex terms to get the complex part.
\item \( a \in [-8, -5] \text{ and } b \in [-98, -92] \)

 $-8 - 95 i$, which corresponds to adding a minus sign in the first term.
\end{enumerate}

\textbf{General Comment:} You can treat $i$ as a variable and distribute. Just remember that $i^2=-1$, so you can continue to reduce after you distribute.
}
\litem{
Simplify the expression below into the form $a+bi$. Then, choose the intervals that $a$ and $b$ belong to.
\[ (-7 + 8 i)(4 + 6 i) \]

The solution is \( -76 - 10 i \), which is option B.\begin{enumerate}[label=\Alph*.]
\item \( a \in [18, 22] \text{ and } b \in [-74, -69] \)

 $20 - 74 i$, which corresponds to adding a minus sign in the first term.
\item \( a \in [-79, -66] \text{ and } b \in [-15, -9] \)

* $-76 - 10 i$, which is the correct option.
\item \( a \in [-79, -66] \text{ and } b \in [8, 11] \)

 $-76 + 10 i$, which corresponds to adding a minus sign in both terms.
\item \( a \in [-31, -25] \text{ and } b \in [43, 49] \)

 $-28 + 48 i$, which corresponds to just multiplying the real terms to get the real part of the solution and the coefficients in the complex terms to get the complex part.
\item \( a \in [18, 22] \text{ and } b \in [70, 80] \)

 $20 + 74 i$, which corresponds to adding a minus sign in the second term.
\end{enumerate}

\textbf{General Comment:} You can treat $i$ as a variable and distribute. Just remember that $i^2=-1$, so you can continue to reduce after you distribute.
}
\litem{
Simplify the expression below and choose the interval the simplification is contained within.
\[ 14 - 12^2 + 9 \div 5 * 10 \div 16 \]

The solution is \( -128.875 \), which is option D.\begin{enumerate}[label=\Alph*.]
\item \( [-130.14, -129.96] \)

 -129.989, which corresponds to an Order of Operations error: not reading left-to-right for multiplication/division.
\item \( [157.37, 158.98] \)

 158.011, which corresponds to two Order of Operations errors.
\item \( [158.86, 159.16] \)

 159.125, which corresponds to an Order of Operations error: multiplying by negative before squaring. For example: $(-3)^2 \neq -3^2$
\item \( [-129.04, -128.32] \)

* -128.875, this is the correct option
\item \( \text{None of the above} \)

 You may have gotten this by making an unanticipated error. If you got a value that is not any of the others, please let the coordinator know so they can help you figure out what happened.
\end{enumerate}

\textbf{General Comment:} While you may remember (or were taught) PEMDAS is done in order, it is actually done as P/E/MD/AS. When we are at MD or AS, we read left to right.
}
\litem{
Choose the \textbf{smallest} set of Complex numbers that the number below belongs to.
\[ \sqrt{\frac{1815}{11}}+2i^2 \]

The solution is \( \text{Irrational} \), which is option A.\begin{enumerate}[label=\Alph*.]
\item \( \text{Irrational} \)

* This is the correct option!
\item \( \text{Nonreal Complex} \)

This is a Complex number $(a+bi)$ that is not Real (has $i$ as part of the number).
\item \( \text{Pure Imaginary} \)

This is a Complex number $(a+bi)$ that \textbf{only} has an imaginary part like $2i$.
\item \( \text{Not a Complex Number} \)

This is not a number. The only non-Complex number we know is dividing by 0 as this is not a number!
\item \( \text{Rational} \)

These are numbers that can be written as fraction of Integers (e.g., -2/3 + 5)
\end{enumerate}

\textbf{General Comment:} Be sure to simplify $i^2 = -1$. This may remove the imaginary portion for your number. If you are having trouble, you may want to look at the \textit{Subgroups of the Real Numbers} section.
}
\litem{
Simplify the expression below and choose the interval the simplification is contained within.
\[ 10 - 6^2 + 4 \div 8 * 5 \div 18 \]

The solution is \( -25.861 \), which is option D.\begin{enumerate}[label=\Alph*.]
\item \( [46.1, 46.2] \)

 46.139, which corresponds to an Order of Operations error: multiplying by negative before squaring. For example: $(-3)^2 \neq -3^2$
\item \( [-26.02, -25.89] \)

 -25.994, which corresponds to an Order of Operations error: not reading left-to-right for multiplication/division.
\item \( [46, 46.06] \)

 46.006, which corresponds to two Order of Operations errors.
\item \( [-25.92, -25.83] \)

* -25.861, this is the correct option
\item \( \text{None of the above} \)

 You may have gotten this by making an unanticipated error. If you got a value that is not any of the others, please let the coordinator know so they can help you figure out what happened.
\end{enumerate}

\textbf{General Comment:} While you may remember (or were taught) PEMDAS is done in order, it is actually done as P/E/MD/AS. When we are at MD or AS, we read left to right.
}
\litem{
Choose the \textbf{smallest} set of Real numbers that the number below belongs to.
\[ -\sqrt{\frac{57600}{144}} \]

The solution is \( \text{Integer} \), which is option A.\begin{enumerate}[label=\Alph*.]
\item \( \text{Integer} \)

* This is the correct option!
\item \( \text{Irrational} \)

These cannot be written as a fraction of Integers.
\item \( \text{Whole} \)

These are the counting numbers with 0 (0, 1, 2, 3, ...)
\item \( \text{Not a Real number} \)

These are Nonreal Complex numbers \textbf{OR} things that are not numbers (e.g., dividing by 0).
\item \( \text{Rational} \)

These are numbers that can be written as fraction of Integers (e.g., -2/3)
\end{enumerate}

\textbf{General Comment:} First, you \textbf{NEED} to simplify the expression. This question simplifies to $-240$. 
 
 Be sure you look at the simplified fraction and not just the decimal expansion. Numbers such as 13, 17, and 19 provide \textbf{long but repeating/terminating decimal expansions!} 
 
 The only ways to *not* be a Real number are: dividing by 0 or taking the square root of a negative number. 
 
 Irrational numbers are more than just square root of 3: adding or subtracting values from square root of 3 is also irrational.
}
\litem{
Simplify the expression below into the form $a+bi$. Then, choose the intervals that $a$ and $b$ belong to.
\[ \frac{-36 - 77 i}{8 + 6 i} \]

The solution is \( -7.50  - 4.00 i \), which is option B.\begin{enumerate}[label=\Alph*.]
\item \( a \in [-9, -6.5] \text{ and } b \in [-401, -399.5] \)

 $-7.50  - 400.00 i$, which corresponds to forgetting to multiply the conjugate by the numerator.
\item \( a \in [-9, -6.5] \text{ and } b \in [-5, -3] \)

* $-7.50  - 4.00 i$, which is the correct option.
\item \( a \in [-750.5, -749.5] \text{ and } b \in [-5, -3] \)

 $-750.00  - 4.00 i$, which corresponds to forgetting to multiply the conjugate by the numerator and using a plus instead of a minus in the denominator.
\item \( a \in [-5, -3.5] \text{ and } b \in [-14, -12] \)

 $-4.50  - 12.83 i$, which corresponds to just dividing the first term by the first term and the second by the second.
\item \( a \in [0.5, 3] \text{ and } b \in [-9, -7.5] \)

 $1.74  - 8.32 i$, which corresponds to forgetting to multiply the conjugate by the numerator and not computing the conjugate correctly.
\end{enumerate}

\textbf{General Comment:} Multiply the numerator and denominator by the *conjugate* of the denominator, then simplify. For example, if we have $2+3i$, the conjugate is $2-3i$.
}
\end{enumerate}

\end{document}