\documentclass{extbook}[14pt]
\usepackage{multicol, enumerate, enumitem, hyperref, color, soul, setspace, parskip, fancyhdr, amssymb, amsthm, amsmath, bbm, latexsym, units, mathtools}
\everymath{\displaystyle}
\usepackage[headsep=0.5cm,headheight=0cm, left=1 in,right= 1 in,top= 1 in,bottom= 1 in]{geometry}
\usepackage{dashrule}  % Package to use the command below to create lines between items
\newcommand{\litem}[1]{\item #1

\rule{\textwidth}{0.4pt}}
\pagestyle{fancy}
\lhead{}
\chead{Answer Key for Makeup Progress Quiz -1 Version C}
\rhead{}
\lfoot{7547-2949}
\cfoot{}
\rfoot{Fall 2020}
\begin{document}
\textbf{This key should allow you to understand why you choose the option you did (beyond just getting a question right or wrong). \href{https://xronos.clas.ufl.edu/mac1105spring2020/courseDescriptionAndMisc/Exams/LearningFromResults}{More instructions on how to use this key can be found here}.}

\textbf{If you have a suggestion to make the keys better, \href{https://forms.gle/CZkbZmPbC9XALEE88}{please fill out the short survey here}.}

\textit{Note: This key is auto-generated and may contain issues and/or errors. The keys are reviewed after each exam to ensure grading is done accurately. If there are issues (like duplicate options), they are noted in the offline gradebook. The keys are a work-in-progress to give students as many resources to improve as possible.}

\rule{\textwidth}{0.4pt}

\begin{enumerate}\litem{
Solve the linear inequality below. Then, choose the constant and interval combination that describes the solution set.
\[ -3 + 8 x \leq \frac{60 x + 6}{7} < 6 + 8 x \]

The solution is \( [-6.75, 9.00) \), which is option C.\begin{enumerate}[label=\Alph*.]
\item \( (a, b], \text{ where } a \in [-7.75, 0.25] \text{ and } b \in [9, 10] \)

$(-6.75, 9.00]$, which corresponds to flipping the inequality.
\item \( (-\infty, a] \cup (b, \infty), \text{ where } a \in [-11.75, -1.75] \text{ and } b \in [7, 14] \)

$(-\infty, -6.75] \cup (9.00, \infty)$, which corresponds to displaying the and-inequality as an or-inequality.
\item \( [a, b), \text{ where } a \in [-6.75, -3.75] \text{ and } b \in [9, 15] \)

$[-6.75, 9.00)$, which is the correct option.
\item \( (-\infty, a) \cup [b, \infty), \text{ where } a \in [-10.75, -5.75] \text{ and } b \in [9, 16] \)

$(-\infty, -6.75) \cup [9.00, \infty)$, which corresponds to displaying the and-inequality as an or-inequality AND flipping the inequality.
\item \( \text{None of the above.} \)


\end{enumerate}

\textbf{General Comment:} To solve, you will need to break up the compound inequality into two inequalities. Be sure to keep track of the inequality! It may be best to draw a number line and graph your solution.
}
\litem{
Solve the linear inequality below. Then, choose the constant and interval combination that describes the solution set.
\[ 5x + 5 \leq 6x + 6 \]

The solution is \( [-1.0, \infty) \), which is option C.\begin{enumerate}[label=\Alph*.]
\item \( (-\infty, a], \text{ where } a \in [0, 4] \)

 $(-\infty, 1.0]$, which corresponds to switching the direction of the interval AND negating the endpoint. You likely did this if you did not flip the inequality when dividing by a negative as well as not moving values over to a side properly.
\item \( (-\infty, a], \text{ where } a \in [-2, 0] \)

 $(-\infty, -1.0]$, which corresponds to switching the direction of the interval. You likely did this if you did not flip the inequality when dividing by a negative!
\item \( [a, \infty), \text{ where } a \in [-2.7, -0.2] \)

* $[-1.0, \infty)$, which is the correct option.
\item \( [a, \infty), \text{ where } a \in [0.6, 1.9] \)

 $[1.0, \infty)$, which corresponds to negating the endpoint of the solution.
\item \( \text{None of the above}. \)

You may have chosen this if you thought the inequality did not match the ends of the intervals.
\end{enumerate}

\textbf{General Comment:} Remember that less/greater than or equal to includes the endpoint, while less/greater do not. Also, remember that you need to flip the inequality when you multiply or divide by a negative.
}
\litem{
Solve the linear inequality below. Then, choose the constant and interval combination that describes the solution set.
\[ \frac{-8}{6} - \frac{5}{3} x \leq \frac{3}{4} x + \frac{9}{8} \]

The solution is \( [-1.017, \infty) \), which is option C.\begin{enumerate}[label=\Alph*.]
\item \( (-\infty, a], \text{ where } a \in [0.02, 3.02] \)

 $(-\infty, 1.017]$, which corresponds to switching the direction of the interval AND negating the endpoint. You likely did this if you did not flip the inequality when dividing by a negative as well as not moving values over to a side properly.
\item \( [a, \infty), \text{ where } a \in [0.02, 4.02] \)

 $[1.017, \infty)$, which corresponds to negating the endpoint of the solution.
\item \( [a, \infty), \text{ where } a \in [-2.02, -0.02] \)

* $[-1.017, \infty)$, which is the correct option.
\item \( (-\infty, a], \text{ where } a \in [-2.02, -0.02] \)

 $(-\infty, -1.017]$, which corresponds to switching the direction of the interval. You likely did this if you did not flip the inequality when dividing by a negative!
\item \( \text{None of the above}. \)

You may have chosen this if you thought the inequality did not match the ends of the intervals.
\end{enumerate}

\textbf{General Comment:} Remember that less/greater than or equal to includes the endpoint, while less/greater do not. Also, remember that you need to flip the inequality when you multiply or divide by a negative.
}
\litem{
Solve the linear inequality below. Then, choose the constant and interval combination that describes the solution set.
\[ -3 + 4 x > 5 x \text{ or } -5 + 4 x < 7 x \]

The solution is \( (-\infty, -3.0) \text{ or } (-1.667, \infty) \), which is option D.\begin{enumerate}[label=\Alph*.]
\item \( (-\infty, a) \cup (b, \infty), \text{ where } a \in [0.67, 2.67] \text{ and } b \in [0, 6] \)

Corresponds to inverting the inequality and negating the solution.
\item \( (-\infty, a] \cup [b, \infty), \text{ where } a \in [-0.33, 8.67] \text{ and } b \in [2, 9] \)

Corresponds to including the endpoints AND negating.
\item \( (-\infty, a] \cup [b, \infty), \text{ where } a \in [-6, 0] \text{ and } b \in [-1.67, 2.33] \)

Corresponds to including the endpoints (when they should be excluded).
\item \( (-\infty, a) \cup (b, \infty), \text{ where } a \in [-6, 1] \text{ and } b \in [-4.67, 2.33] \)

 * Correct option.
\item \( (-\infty, \infty) \)

Corresponds to the variable canceling, which does not happen in this instance.
\end{enumerate}

\textbf{General Comment:} When multiplying or dividing by a negative, flip the sign.
}
\litem{
Using an interval or intervals, describe all the $x$-values within or including a distance of the given values.
\[ \text{ Less than } 4 \text{ units from the number } -4. \]

The solution is \( (-8, 0) \), which is option C.\begin{enumerate}[label=\Alph*.]
\item \( [-8, 0] \)

This describes the values no more than 4 from -4
\item \( (-\infty, -8) \cup (0, \infty) \)

This describes the values more than 4 from -4
\item \( (-8, 0) \)

This describes the values less than 4 from -4
\item \( (-\infty, -8] \cup [0, \infty) \)

This describes the values no less than 4 from -4
\item \( \text{None of the above} \)

You likely thought the values in the interval were not correct.
\end{enumerate}

\textbf{General Comment:} When thinking about this language, it helps to draw a number line and try points.
}
\litem{
Solve the linear inequality below. Then, choose the constant and interval combination that describes the solution set.
\[ 8 + 6 x > 9 x \text{ or } 9 + 7 x < 9 x \]

The solution is \( (-\infty, 2.667) \text{ or } (4.5, \infty) \), which is option D.\begin{enumerate}[label=\Alph*.]
\item \( (-\infty, a] \cup [b, \infty), \text{ where } a \in [-6.5, -3.5] \text{ and } b \in [-4.67, 2.33] \)

Corresponds to including the endpoints AND negating.
\item \( (-\infty, a] \cup [b, \infty), \text{ where } a \in [0.67, 6.67] \text{ and } b \in [3.5, 5.5] \)

Corresponds to including the endpoints (when they should be excluded).
\item \( (-\infty, a) \cup (b, \infty), \text{ where } a \in [-4.5, -2.5] \text{ and } b \in [-2.67, 1.33] \)

Corresponds to inverting the inequality and negating the solution.
\item \( (-\infty, a) \cup (b, \infty), \text{ where } a \in [-0.33, 4.67] \text{ and } b \in [3.5, 5.5] \)

 * Correct option.
\item \( (-\infty, \infty) \)

Corresponds to the variable canceling, which does not happen in this instance.
\end{enumerate}

\textbf{General Comment:} When multiplying or dividing by a negative, flip the sign.
}
\litem{
Solve the linear inequality below. Then, choose the constant and interval combination that describes the solution set.
\[ \frac{-9}{4} + \frac{4}{5} x \geq \frac{10}{9} x - \frac{3}{7} \]

The solution is \( (-\infty, -5.855] \), which is option C.\begin{enumerate}[label=\Alph*.]
\item \( [a, \infty), \text{ where } a \in [-8.86, -1.86] \)

 $[-5.855, \infty)$, which corresponds to switching the direction of the interval. You likely did this if you did not flip the inequality when dividing by a negative!
\item \( [a, \infty), \text{ where } a \in [3.86, 10.86] \)

 $[5.855, \infty)$, which corresponds to switching the direction of the interval AND negating the endpoint. You likely did this if you did not flip the inequality when dividing by a negative as well as not moving values over to a side properly.
\item \( (-\infty, a], \text{ where } a \in [-6.86, -4.86] \)

* $(-\infty, -5.855]$, which is the correct option.
\item \( (-\infty, a], \text{ where } a \in [2.86, 6.86] \)

 $(-\infty, 5.855]$, which corresponds to negating the endpoint of the solution.
\item \( \text{None of the above}. \)

You may have chosen this if you thought the inequality did not match the ends of the intervals.
\end{enumerate}

\textbf{General Comment:} Remember that less/greater than or equal to includes the endpoint, while less/greater do not. Also, remember that you need to flip the inequality when you multiply or divide by a negative.
}
\litem{
Solve the linear inequality below. Then, choose the constant and interval combination that describes the solution set.
\[ -6 + 8 x \leq \frac{32 x - 6}{3} < 5 + 9 x \]

The solution is \( [-1.50, 4.20) \), which is option B.\begin{enumerate}[label=\Alph*.]
\item \( (-\infty, a] \cup (b, \infty), \text{ where } a \in [-5.5, -0.5] \text{ and } b \in [0.2, 11.2] \)

$(-\infty, -1.50] \cup (4.20, \infty)$, which corresponds to displaying the and-inequality as an or-inequality.
\item \( [a, b), \text{ where } a \in [-3.6, -0.2] \text{ and } b \in [4.2, 6.2] \)

$[-1.50, 4.20)$, which is the correct option.
\item \( (a, b], \text{ where } a \in [-2, 1.2] \text{ and } b \in [2.2, 8.2] \)

$(-1.50, 4.20]$, which corresponds to flipping the inequality.
\item \( (-\infty, a) \cup [b, \infty), \text{ where } a \in [-1.5, -0.5] \text{ and } b \in [3.2, 9.2] \)

$(-\infty, -1.50) \cup [4.20, \infty)$, which corresponds to displaying the and-inequality as an or-inequality AND flipping the inequality.
\item \( \text{None of the above.} \)


\end{enumerate}

\textbf{General Comment:} To solve, you will need to break up the compound inequality into two inequalities. Be sure to keep track of the inequality! It may be best to draw a number line and graph your solution.
}
\litem{
Solve the linear inequality below. Then, choose the constant and interval combination that describes the solution set.
\[ -10x -7 \geq 4x + 4 \]

The solution is \( (-\infty, -0.786] \), which is option C.\begin{enumerate}[label=\Alph*.]
\item \( [a, \infty), \text{ where } a \in [-0.46, 2.58] \)

 $[0.786, \infty)$, which corresponds to switching the direction of the interval AND negating the endpoint. You likely did this if you did not flip the inequality when dividing by a negative as well as not moving values over to a side properly.
\item \( (-\infty, a], \text{ where } a \in [0.7, 1.7] \)

 $(-\infty, 0.786]$, which corresponds to negating the endpoint of the solution.
\item \( (-\infty, a], \text{ where } a \in [-1.9, -0.4] \)

* $(-\infty, -0.786]$, which is the correct option.
\item \( [a, \infty), \text{ where } a \in [-1.18, -0.2] \)

 $[-0.786, \infty)$, which corresponds to switching the direction of the interval. You likely did this if you did not flip the inequality when dividing by a negative!
\item \( \text{None of the above}. \)

You may have chosen this if you thought the inequality did not match the ends of the intervals.
\end{enumerate}

\textbf{General Comment:} Remember that less/greater than or equal to includes the endpoint, while less/greater do not. Also, remember that you need to flip the inequality when you multiply or divide by a negative.
}
\litem{
Using an interval or intervals, describe all the $x$-values within or including a distance of the given values.
\[ \text{ More than } 8 \text{ units from the number } 3. \]

The solution is \( (-\infty, -5) \cup (11, \infty) \), which is option A.\begin{enumerate}[label=\Alph*.]
\item \( (-\infty, -5) \cup (11, \infty) \)

This describes the values more than 8 from 3
\item \( (-\infty, -5] \cup [11, \infty) \)

This describes the values no less than 8 from 3
\item \( (-5, 11) \)

This describes the values less than 8 from 3
\item \( [-5, 11] \)

This describes the values no more than 8 from 3
\item \( \text{None of the above} \)

You likely thought the values in the interval were not correct.
\end{enumerate}

\textbf{General Comment:} When thinking about this language, it helps to draw a number line and try points.
}
\end{enumerate}

\end{document}