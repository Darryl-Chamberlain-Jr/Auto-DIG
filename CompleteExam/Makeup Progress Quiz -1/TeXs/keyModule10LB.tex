\documentclass{extbook}[14pt]
\usepackage{multicol, enumerate, enumitem, hyperref, color, soul, setspace, parskip, fancyhdr, amssymb, amsthm, amsmath, bbm, latexsym, units, mathtools}
\everymath{\displaystyle}
\usepackage[headsep=0.5cm,headheight=0cm, left=1 in,right= 1 in,top= 1 in,bottom= 1 in]{geometry}
\usepackage{dashrule}  % Package to use the command below to create lines between items
\newcommand{\litem}[1]{\item #1

\rule{\textwidth}{0.4pt}}
\pagestyle{fancy}
\lhead{}
\chead{Answer Key for Makeup Progress Quiz -1 Version B}
\rhead{}
\lfoot{7547-2949}
\cfoot{}
\rfoot{Fall 2020}
\begin{document}
\textbf{This key should allow you to understand why you choose the option you did (beyond just getting a question right or wrong). \href{https://xronos.clas.ufl.edu/mac1105spring2020/courseDescriptionAndMisc/Exams/LearningFromResults}{More instructions on how to use this key can be found here}.}

\textbf{If you have a suggestion to make the keys better, \href{https://forms.gle/CZkbZmPbC9XALEE88}{please fill out the short survey here}.}

\textit{Note: This key is auto-generated and may contain issues and/or errors. The keys are reviewed after each exam to ensure grading is done accurately. If there are issues (like duplicate options), they are noted in the offline gradebook. The keys are a work-in-progress to give students as many resources to improve as possible.}

\rule{\textwidth}{0.4pt}

\begin{enumerate}\litem{
What are the \textit{possible Rational} roots of the polynomial below?
\[ f(x) = 7x^{4} +6 x^{3} +4 x^{2} +4 x + 5 \]

The solution is \( \text{ All combinations of: }\frac{\pm 1,\pm 5}{\pm 1,\pm 7} \), which is option C.\begin{enumerate}[label=\Alph*.]
\item \( \pm 1,\pm 5 \)

This would have been the solution \textbf{if asked for the possible Integer roots}!
\item \( \text{ All combinations of: }\frac{\pm 1,\pm 7}{\pm 1,\pm 5} \)

 Distractor 3: Corresponds to the plus or minus of the inverse quotient (an/a0) of the factors. 
\item \( \text{ All combinations of: }\frac{\pm 1,\pm 5}{\pm 1,\pm 7} \)

* This is the solution \textbf{since we asked for the possible Rational roots}!
\item \( \pm 1,\pm 7 \)

 Distractor 1: Corresponds to the plus or minus factors of a1 only.
\item \( \text{ There is no formula or theorem that tells us all possible Rational roots.} \)

 Distractor 4: Corresponds to not recalling the theorem for rational roots of a polynomial.
\end{enumerate}

\textbf{General Comment:} We have a way to find the possible Rational roots. The possible Integer roots are the Integers in this list.
}
\litem{
Factor the polynomial below completely. Then, choose the intervals the zeros of the polynomial belong to, where $z_1 \leq z_2 \leq z_3$. \textit{To make the problem easier, all zeros are between -5 and 5.}
\[ f(x) = 15x^{3} -49 x^{2} +44 x -12 \]

The solution is \( [0.6, 0.6666666666666666, 2] \), which is option B.\begin{enumerate}[label=\Alph*.]
\item \( z_1 \in [-2.1, -1], \text{   }  z_2 \in [-1.14, -0.19], \text{   and   } z_3 \in [-1.08, -0.56] \)

 Distractor 1: Corresponds to negatives of all zeros.
\item \( z_1 \in [-0.1, 0.9], \text{   }  z_2 \in [0.07, 0.97], \text{   and   } z_3 \in [1.54, 2.39] \)

* This is the solution!
\item \( z_1 \in [-2.1, -1], \text{   }  z_2 \in [-2.02, -1.81], \text{   and   } z_3 \in [-0.39, -0.08] \)

 Distractor 4: Corresponds to moving factors from one rational to another.
\item \( z_1 \in [-2.1, -1], \text{   }  z_2 \in [-1.73, -1.51], \text{   and   } z_3 \in [-1.6, -1.14] \)

 Distractor 3: Corresponds to negatives of all zeros AND inversing rational roots.
\item \( z_1 \in [1, 2.5], \text{   }  z_2 \in [1.49, 1.7], \text{   and   } z_3 \in [1.54, 2.39] \)

 Distractor 2: Corresponds to inversing rational roots.
\end{enumerate}

\textbf{General Comment:} Remember to try the middle-most integers first as these normally are the zeros. Also, once you get it to a quadratic, you can use your other factoring techniques to finish factoring.
}
\litem{
What are the \textit{possible Rational} roots of the polynomial below?
\[ f(x) = 3x^{4} +7 x^{3} +2 x^{2} +5 x + 2 \]

The solution is \( \text{ All combinations of: }\frac{\pm 1,\pm 2}{\pm 1,\pm 3} \), which is option C.\begin{enumerate}[label=\Alph*.]
\item \( \pm 1,\pm 2 \)

This would have been the solution \textbf{if asked for the possible Integer roots}!
\item \( \text{ All combinations of: }\frac{\pm 1,\pm 3}{\pm 1,\pm 2} \)

 Distractor 3: Corresponds to the plus or minus of the inverse quotient (an/a0) of the factors. 
\item \( \text{ All combinations of: }\frac{\pm 1,\pm 2}{\pm 1,\pm 3} \)

* This is the solution \textbf{since we asked for the possible Rational roots}!
\item \( \pm 1,\pm 3 \)

 Distractor 1: Corresponds to the plus or minus factors of a1 only.
\item \( \text{ There is no formula or theorem that tells us all possible Rational roots.} \)

 Distractor 4: Corresponds to not recalling the theorem for rational roots of a polynomial.
\end{enumerate}

\textbf{General Comment:} We have a way to find the possible Rational roots. The possible Integer roots are the Integers in this list.
}
\litem{
Perform the division below. Then, find the intervals that correspond to the quotient in the form $ax^2+bx+c$ and remainder $r$.
\[ \frac{16x^{3} +32 x^{2} -4 x -12}{x + 2} \]

The solution is \( 16x^{2} -4 + \frac{-4}{x + 2} \), which is option E.\begin{enumerate}[label=\Alph*.]
\item \( a \in [-32, -31], \text{   } b \in [-32, -26], \text{   } c \in [-72, -63], \text{   and   } r \in [-151, -145]. \)

 You divided by the opposite of the factor AND multiplied the first factor rather than just bringing it down.
\item \( a \in [-32, -31], \text{   } b \in [95, 101], \text{   } c \in [-200, -193], \text{   and   } r \in [376, 387]. \)

 You multiplied by the synthetic number rather than bringing the first factor down.
\item \( a \in [16, 18], \text{   } b \in [-17, -15], \text{   } c \in [43, 45], \text{   and   } r \in [-147, -143]. \)

 You multiplied by the synthetic number and subtracted rather than adding during synthetic division.
\item \( a \in [16, 18], \text{   } b \in [60, 66], \text{   } c \in [124, 131], \text{   and   } r \in [235, 239]. \)

 You divided by the opposite of the factor.
\item \( a \in [16, 18], \text{   } b \in [-1, 3], \text{   } c \in [-15, 2], \text{   and   } r \in [-6, -1]. \)

* This is the solution!
\end{enumerate}

\textbf{General Comment:} Be sure to synthetically divide by the zero of the denominator!
}
\litem{
Perform the division below. Then, find the intervals that correspond to the quotient in the form $ax^2+bx+c$ and remainder $r$.
\[ \frac{4x^{3} -10 x^{2} -16 x + 38}{x -2} \]

The solution is \( 4x^{2} -2 x -20 + \frac{-2}{x -2} \), which is option A.\begin{enumerate}[label=\Alph*.]
\item \( a \in [2.9, 4.7], \text{   } b \in [-5.6, 0.9], \text{   } c \in [-21.9, -18.1], \text{   and   } r \in [-3, 0]. \)

* This is the solution!
\item \( a \in [2.9, 4.7], \text{   } b \in [-7, -4.7], \text{   } c \in [-22.3, -20.4], \text{   and   } r \in [12, 26]. \)

 You multiplied by the synthetic number and subtracted rather than adding during synthetic division.
\item \( a \in [4.3, 9.4], \text{   } b \in [5.5, 7.2], \text{   } c \in [-4.9, -1.2], \text{   and   } r \in [25, 34]. \)

 You multiplied by the synthetic number rather than bringing the first factor down.
\item \( a \in [4.3, 9.4], \text{   } b \in [-29, -25.9], \text{   } c \in [35.5, 41], \text{   and   } r \in [-35, -32]. \)

 You divided by the opposite of the factor AND multiplied the first factor rather than just bringing it down.
\item \( a \in [2.9, 4.7], \text{   } b \in [-18.6, -17.5], \text{   } c \in [18.1, 23.6], \text{   and   } r \in [-3, 0]. \)

 You divided by the opposite of the factor.
\end{enumerate}

\textbf{General Comment:} Be sure to synthetically divide by the zero of the denominator!
}
\litem{
Perform the division below. Then, find the intervals that correspond to the quotient in the form $ax^2+bx+c$ and remainder $r$.
\[ \frac{10x^{3} -70 x -63}{x -3} \]

The solution is \( 10x^{2} +30 x + 20 + \frac{-3}{x -3} \), which is option D.\begin{enumerate}[label=\Alph*.]
\item \( a \in [9, 11], b \in [12, 24], c \in [-32, -28], \text{ and } r \in [-125, -117]. \)

 You multipled by the synthetic number and subtracted rather than adding during synthetic division.
\item \( a \in [9, 11], b \in [-31, -28], c \in [18, 24], \text{ and } r \in [-125, -117]. \)

 You divided by the opposite of the factor.
\item \( a \in [30, 35], b \in [88, 93], c \in [198, 202], \text{ and } r \in [537, 540]. \)

 You multipled by the synthetic number rather than bringing the first factor down.
\item \( a \in [9, 11], b \in [28, 32], c \in [18, 24], \text{ and } r \in [-7, -2]. \)

* This is the solution!
\item \( a \in [30, 35], b \in [-93, -88], c \in [198, 202], \text{ and } r \in [-665, -660]. \)

 You divided by the opposite of the factor AND multipled the first factor rather than just bringing it down.
\end{enumerate}

\textbf{General Comment:} Be sure to synthetically divide by the zero of the denominator! Also, make sure to include 0 placeholders for missing terms.
}
\litem{
Factor the polynomial below completely, knowing that $x-2$ is a factor. Then, choose the intervals the zeros of the polynomial belong to, where $z_1 \leq z_2 \leq z_3 \leq z_4$. \textit{To make the problem easier, all zeros are between -5 and 5.}
\[ f(x) = 12x^{4} +25 x^{3} -114 x^{2} -48 x + 160 \]

The solution is \( [-4, -1.3333333333333333, 1.25, 2] \), which is option D.\begin{enumerate}[label=\Alph*.]
\item \( z_1 \in [-3.5, -1.7], \text{   }  z_2 \in [-0.43, -0.4], z_3 \in [3.99, 4.05], \text{   and   } z_4 \in [4, 9] \)

 Distractor 4: Corresponds to moving factors from one rational to another.
\item \( z_1 \in [-5.9, -2.3], \text{   }  z_2 \in [-0.78, -0.74], z_3 \in [0.79, 0.85], \text{   and   } z_4 \in [1, 3] \)

 Distractor 2: Corresponds to inversing rational roots.
\item \( z_1 \in [-3.5, -1.7], \text{   }  z_2 \in [-0.82, -0.79], z_3 \in [0.72, 0.76], \text{   and   } z_4 \in [4, 9] \)

 Distractor 3: Corresponds to negatives of all zeros AND inversing rational roots.
\item \( z_1 \in [-5.9, -2.3], \text{   }  z_2 \in [-1.38, -1.3], z_3 \in [1.23, 1.25], \text{   and   } z_4 \in [1, 3] \)

* This is the solution!
\item \( z_1 \in [-3.5, -1.7], \text{   }  z_2 \in [-1.32, -1.23], z_3 \in [1.33, 1.38], \text{   and   } z_4 \in [4, 9] \)

 Distractor 1: Corresponds to negatives of all zeros.
\end{enumerate}

\textbf{General Comment:} Remember to try the middle-most integers first as these normally are the zeros. Also, once you get it to a quadratic, you can use your other factoring techniques to finish factoring.
}
\litem{
Factor the polynomial below completely. Then, choose the intervals the zeros of the polynomial belong to, where $z_1 \leq z_2 \leq z_3$. \textit{To make the problem easier, all zeros are between -5 and 5.}
\[ f(x) = 6x^{3} +41 x^{2} +45 x -50 \]

The solution is \( [-5, -2.5, 0.6666666666666666] \), which is option C.\begin{enumerate}[label=\Alph*.]
\item \( z_1 \in [-5.23, -4.91], \text{   }  z_2 \in [-0.6, -0.1], \text{   and   } z_3 \in [1.2, 1.6] \)

 Distractor 2: Corresponds to inversing rational roots.
\item \( z_1 \in [-0.68, -0.37], \text{   }  z_2 \in [1.92, 2.62], \text{   and   } z_3 \in [3.9, 6.2] \)

 Distractor 1: Corresponds to negatives of all zeros.
\item \( z_1 \in [-5.23, -4.91], \text{   }  z_2 \in [-3.38, -2.35], \text{   and   } z_3 \in [0.4, 1.4] \)

* This is the solution!
\item \( z_1 \in [-1.59, -1.34], \text{   }  z_2 \in [0.25, 0.49], \text{   and   } z_3 \in [3.9, 6.2] \)

 Distractor 3: Corresponds to negatives of all zeros AND inversing rational roots.
\item \( z_1 \in [-0.49, -0.12], \text{   }  z_2 \in [4.57, 5.85], \text{   and   } z_3 \in [3.9, 6.2] \)

 Distractor 4: Corresponds to moving factors from one rational to another.
\end{enumerate}

\textbf{General Comment:} Remember to try the middle-most integers first as these normally are the zeros. Also, once you get it to a quadratic, you can use your other factoring techniques to finish factoring.
}
\litem{
Perform the division below. Then, find the intervals that correspond to the quotient in the form $ax^2+bx+c$ and remainder $r$.
\[ \frac{20x^{3} -84 x^{2} + 62}{x -4} \]

The solution is \( 20x^{2} -4 x -16 + \frac{-2}{x -4} \), which is option C.\begin{enumerate}[label=\Alph*.]
\item \( a \in [79, 83], b \in [-406, -397], c \in [1615, 1618], \text{ and } r \in [-6403, -6399]. \)

 You divided by the opposite of the factor AND multipled the first factor rather than just bringing it down.
\item \( a \in [79, 83], b \in [225, 245], c \in [942, 956], \text{ and } r \in [3838, 3846]. \)

 You multipled by the synthetic number rather than bringing the first factor down.
\item \( a \in [20, 27], b \in [-11, 3], c \in [-20, -12], \text{ and } r \in [-2, 1]. \)

* This is the solution!
\item \( a \in [20, 27], b \in [-28, -17], c \in [-75, -71], \text{ and } r \in [-155, -152]. \)

 You multipled by the synthetic number and subtracted rather than adding during synthetic division.
\item \( a \in [20, 27], b \in [-167, -161], c \in [651, 660], \text{ and } r \in [-2562, -2555]. \)

 You divided by the opposite of the factor.
\end{enumerate}

\textbf{General Comment:} Be sure to synthetically divide by the zero of the denominator! Also, make sure to include 0 placeholders for missing terms.
}
\litem{
Factor the polynomial below completely, knowing that $x-3$ is a factor. Then, choose the intervals the zeros of the polynomial belong to, where $z_1 \leq z_2 \leq z_3 \leq z_4$. \textit{To make the problem easier, all zeros are between -5 and 5.}
\[ f(x) = 10x^{4} -89 x^{3} +238 x^{2} -123 x -180 \]

The solution is \( [-0.6, 2.5, 3, 4] \), which is option A.\begin{enumerate}[label=\Alph*.]
\item \( z_1 \in [-0.6, 0.4], \text{   }  z_2 \in [1.1, 3.5], z_3 \in [2.77, 3.1], \text{   and   } z_4 \in [3.12, 4.1] \)

* This is the solution!
\item \( z_1 \in [-1.67, -0.67], \text{   }  z_2 \in [-0.8, 0.9], z_3 \in [2.77, 3.1], \text{   and   } z_4 \in [3.12, 4.1] \)

 Distractor 2: Corresponds to inversing rational roots.
\item \( z_1 \in [-6, -2], \text{   }  z_2 \in [-4.3, -1.2], z_3 \in [-0.41, -0.36], \text{   and   } z_4 \in [1.43, 2.5] \)

 Distractor 3: Corresponds to negatives of all zeros AND inversing rational roots.
\item \( z_1 \in [-6, -2], \text{   }  z_2 \in [-4.3, -1.2], z_3 \in [-3.11, -2.35], \text{   and   } z_4 \in [-0.23, 1.2] \)

 Distractor 1: Corresponds to negatives of all zeros.
\item \( z_1 \in [-6, -2], \text{   }  z_2 \in [-4.3, -1.2], z_3 \in [-0.85, -0.44], \text{   and   } z_4 \in [2.73, 3.26] \)

 Distractor 4: Corresponds to moving factors from one rational to another.
\end{enumerate}

\textbf{General Comment:} Remember to try the middle-most integers first as these normally are the zeros. Also, once you get it to a quadratic, you can use your other factoring techniques to finish factoring.
}
\end{enumerate}

\end{document}