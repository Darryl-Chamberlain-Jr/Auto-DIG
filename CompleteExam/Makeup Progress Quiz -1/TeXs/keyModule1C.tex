\documentclass{extbook}[14pt]
\usepackage{multicol, enumerate, enumitem, hyperref, color, soul, setspace, parskip, fancyhdr, amssymb, amsthm, amsmath, bbm, latexsym, units, mathtools}
\everymath{\displaystyle}
\usepackage[headsep=0.5cm,headheight=0cm, left=1 in,right= 1 in,top= 1 in,bottom= 1 in]{geometry}
\usepackage{dashrule}  % Package to use the command below to create lines between items
\newcommand{\litem}[1]{\item #1

\rule{\textwidth}{0.4pt}}
\pagestyle{fancy}
\lhead{}
\chead{Answer Key for Makeup Progress Quiz -1 Version C}
\rhead{}
\lfoot{7547-2949}
\cfoot{}
\rfoot{Fall 2020}
\begin{document}
\textbf{This key should allow you to understand why you choose the option you did (beyond just getting a question right or wrong). \href{https://xronos.clas.ufl.edu/mac1105spring2020/courseDescriptionAndMisc/Exams/LearningFromResults}{More instructions on how to use this key can be found here}.}

\textbf{If you have a suggestion to make the keys better, \href{https://forms.gle/CZkbZmPbC9XALEE88}{please fill out the short survey here}.}

\textit{Note: This key is auto-generated and may contain issues and/or errors. The keys are reviewed after each exam to ensure grading is done accurately. If there are issues (like duplicate options), they are noted in the offline gradebook. The keys are a work-in-progress to give students as many resources to improve as possible.}

\rule{\textwidth}{0.4pt}

\begin{enumerate}\litem{
Choose the \textbf{smallest} set of Real numbers that the number below belongs to.
\[ \sqrt{\frac{169}{400}} \]

The solution is \( \text{Rational} \), which is option A.\begin{enumerate}[label=\Alph*.]
\item \( \text{Rational} \)

* This is the correct option!
\item \( \text{Irrational} \)

These cannot be written as a fraction of Integers.
\item \( \text{Not a Real number} \)

These are Nonreal Complex numbers \textbf{OR} things that are not numbers (e.g., dividing by 0).
\item \( \text{Whole} \)

These are the counting numbers with 0 (0, 1, 2, 3, ...)
\item \( \text{Integer} \)

These are the negative and positive counting numbers (..., -3, -2, -1, 0, 1, 2, 3, ...)
\end{enumerate}

\textbf{General Comment:} First, you \textbf{NEED} to simplify the expression. This question simplifies to $\frac{13}{20}$. 
 
 Be sure you look at the simplified fraction and not just the decimal expansion. Numbers such as 13, 17, and 19 provide \textbf{long but repeating/terminating decimal expansions!} 
 
 The only ways to *not* be a Real number are: dividing by 0 or taking the square root of a negative number. 
 
 Irrational numbers are more than just square root of 3: adding or subtracting values from square root of 3 is also irrational.
}
\litem{
Simplify the expression below into the form $a+bi$. Then, choose the intervals that $a$ and $b$ belong to.
\[ (10 + 5 i)(6 + 4 i) \]

The solution is \( 40 + 70 i \), which is option E.\begin{enumerate}[label=\Alph*.]
\item \( a \in [79, 83] \text{ and } b \in [-14, -6] \)

 $80 - 10 i$, which corresponds to adding a minus sign in the second term.
\item \( a \in [35, 42] \text{ and } b \in [-72, -66] \)

 $40 - 70 i$, which corresponds to adding a minus sign in both terms.
\item \( a \in [53, 66] \text{ and } b \in [17, 25] \)

 $60 + 20 i$, which corresponds to just multiplying the real terms to get the real part of the solution and the coefficients in the complex terms to get the complex part.
\item \( a \in [79, 83] \text{ and } b \in [8, 14] \)

 $80 + 10 i$, which corresponds to adding a minus sign in the first term.
\item \( a \in [35, 42] \text{ and } b \in [67, 76] \)

* $40 + 70 i$, which is the correct option.
\end{enumerate}

\textbf{General Comment:} You can treat $i$ as a variable and distribute. Just remember that $i^2=-1$, so you can continue to reduce after you distribute.
}
\litem{
Simplify the expression below into the form $a+bi$. Then, choose the intervals that $a$ and $b$ belong to.
\[ \frac{-18 - 55 i}{-6 + 4 i} \]

The solution is \( -2.15  + 7.73 i \), which is option C.\begin{enumerate}[label=\Alph*.]
\item \( a \in [5, 8] \text{ and } b \in [4, 5.5] \)

 $6.31  + 4.96 i$, which corresponds to forgetting to multiply the conjugate by the numerator and not computing the conjugate correctly.
\item \( a \in [-2.5, -2] \text{ and } b \in [401.5, 403.5] \)

 $-2.15  + 402.00 i$, which corresponds to forgetting to multiply the conjugate by the numerator.
\item \( a \in [-2.5, -2] \text{ and } b \in [6, 8] \)

* $-2.15  + 7.73 i$, which is the correct option.
\item \( a \in [1.5, 4.5] \text{ and } b \in [-14, -12.5] \)

 $3.00  - 13.75 i$, which corresponds to just dividing the first term by the first term and the second by the second.
\item \( a \in [-112.5, -111.5] \text{ and } b \in [6, 8] \)

 $-112.00  + 7.73 i$, which corresponds to forgetting to multiply the conjugate by the numerator and using a plus instead of a minus in the denominator.
\end{enumerate}

\textbf{General Comment:} Multiply the numerator and denominator by the *conjugate* of the denominator, then simplify. For example, if we have $2+3i$, the conjugate is $2-3i$.
}
\litem{
Choose the \textbf{smallest} set of Complex numbers that the number below belongs to.
\[ -\sqrt{\frac{2925}{15}}+8i^2 \]

The solution is \( \text{Irrational} \), which is option B.\begin{enumerate}[label=\Alph*.]
\item \( \text{Pure Imaginary} \)

This is a Complex number $(a+bi)$ that \textbf{only} has an imaginary part like $2i$.
\item \( \text{Irrational} \)

* This is the correct option!
\item \( \text{Not a Complex Number} \)

This is not a number. The only non-Complex number we know is dividing by 0 as this is not a number!
\item \( \text{Nonreal Complex} \)

This is a Complex number $(a+bi)$ that is not Real (has $i$ as part of the number).
\item \( \text{Rational} \)

These are numbers that can be written as fraction of Integers (e.g., -2/3 + 5)
\end{enumerate}

\textbf{General Comment:} Be sure to simplify $i^2 = -1$. This may remove the imaginary portion for your number. If you are having trouble, you may want to look at the \textit{Subgroups of the Real Numbers} section.
}
\litem{
Simplify the expression below into the form $a+bi$. Then, choose the intervals that $a$ and $b$ belong to.
\[ \frac{18 + 77 i}{1 - 6 i} \]

The solution is \( -12.00  + 5.00 i \), which is option C.\begin{enumerate}[label=\Alph*.]
\item \( a \in [11.5, 13.5] \text{ and } b \in [-1.5, 0] \)

 $12.97  - 0.84 i$, which corresponds to forgetting to multiply the conjugate by the numerator and not computing the conjugate correctly.
\item \( a \in [-13, -11] \text{ and } b \in [184, 185.5] \)

 $-12.00  + 185.00 i$, which corresponds to forgetting to multiply the conjugate by the numerator.
\item \( a \in [-13, -11] \text{ and } b \in [4, 5.5] \)

* $-12.00  + 5.00 i$, which is the correct option.
\item \( a \in [-445, -442.5] \text{ and } b \in [4, 5.5] \)

 $-444.00  + 5.00 i$, which corresponds to forgetting to multiply the conjugate by the numerator and using a plus instead of a minus in the denominator.
\item \( a \in [17, 18.5] \text{ and } b \in [-13.5, -11.5] \)

 $18.00  - 12.83 i$, which corresponds to just dividing the first term by the first term and the second by the second.
\end{enumerate}

\textbf{General Comment:} Multiply the numerator and denominator by the *conjugate* of the denominator, then simplify. For example, if we have $2+3i$, the conjugate is $2-3i$.
}
\litem{
Simplify the expression below and choose the interval the simplification is contained within.
\[ 20 - 5 \div 14 * 13 - (15 * 16) \]

The solution is \( -224.643 \), which is option B.\begin{enumerate}[label=\Alph*.]
\item \( [255.97, 262.97] \)

 259.973, which corresponds to not distributing addition and subtraction correctly.
\item \( [-229.64, -221.64] \)

* -224.643, which is the correct option.
\item \( [1.71, 9.71] \)

 5.714, which corresponds to not distributing a negative correctly.
\item \( [-222.03, -214.03] \)

 -220.027, which corresponds to an Order of Operations error: not reading left-to-right for multiplication/division.
\item \( \text{None of the above} \)

 You may have gotten this by making an unanticipated error. If you got a value that is not any of the others, please let the coordinator know so they can help you figure out what happened.
\end{enumerate}

\textbf{General Comment:} While you may remember (or were taught) PEMDAS is done in order, it is actually done as P/E/MD/AS. When we are at MD or AS, we read left to right.
}
\litem{
Choose the \textbf{smallest} set of Real numbers that the number below belongs to.
\[ \sqrt{\frac{289}{100}} \]

The solution is \( \text{Rational} \), which is option D.\begin{enumerate}[label=\Alph*.]
\item \( \text{Irrational} \)

These cannot be written as a fraction of Integers.
\item \( \text{Whole} \)

These are the counting numbers with 0 (0, 1, 2, 3, ...)
\item \( \text{Integer} \)

These are the negative and positive counting numbers (..., -3, -2, -1, 0, 1, 2, 3, ...)
\item \( \text{Rational} \)

* This is the correct option!
\item \( \text{Not a Real number} \)

These are Nonreal Complex numbers \textbf{OR} things that are not numbers (e.g., dividing by 0).
\end{enumerate}

\textbf{General Comment:} First, you \textbf{NEED} to simplify the expression. This question simplifies to $\frac{17}{10}$. 
 
 Be sure you look at the simplified fraction and not just the decimal expansion. Numbers such as 13, 17, and 19 provide \textbf{long but repeating/terminating decimal expansions!} 
 
 The only ways to *not* be a Real number are: dividing by 0 or taking the square root of a negative number. 
 
 Irrational numbers are more than just square root of 3: adding or subtracting values from square root of 3 is also irrational.
}
\litem{
Choose the \textbf{smallest} set of Complex numbers that the number below belongs to.
\[ \frac{4}{-19}+81i^2 \]

The solution is \( \text{Rational} \), which is option B.\begin{enumerate}[label=\Alph*.]
\item \( \text{Pure Imaginary} \)

This is a Complex number $(a+bi)$ that \textbf{only} has an imaginary part like $2i$.
\item \( \text{Rational} \)

* This is the correct option!
\item \( \text{Nonreal Complex} \)

This is a Complex number $(a+bi)$ that is not Real (has $i$ as part of the number).
\item \( \text{Not a Complex Number} \)

This is not a number. The only non-Complex number we know is dividing by 0 as this is not a number!
\item \( \text{Irrational} \)

These cannot be written as a fraction of Integers. Remember: $\pi$ is not an Integer!
\end{enumerate}

\textbf{General Comment:} Be sure to simplify $i^2 = -1$. This may remove the imaginary portion for your number. If you are having trouble, you may want to look at the \textit{Subgroups of the Real Numbers} section.
}
\litem{
Simplify the expression below and choose the interval the simplification is contained within.
\[ 14 - 7^2 + 3 \div 16 * 12 \div 15 \]

The solution is \( -34.850 \), which is option B.\begin{enumerate}[label=\Alph*.]
\item \( [62.94, 63.1] \)

 63.001, which corresponds to two Order of Operations errors.
\item \( [-34.94, -34.81] \)

* -34.850, this is the correct option
\item \( [-35.04, -34.93] \)

 -34.999, which corresponds to an Order of Operations error: not reading left-to-right for multiplication/division.
\item \( [63.02, 63.36] \)

 63.150, which corresponds to an Order of Operations error: multiplying by negative before squaring. For example: $(-3)^2 \neq -3^2$
\item \( \text{None of the above} \)

 You may have gotten this by making an unanticipated error. If you got a value that is not any of the others, please let the coordinator know so they can help you figure out what happened.
\end{enumerate}

\textbf{General Comment:} While you may remember (or were taught) PEMDAS is done in order, it is actually done as P/E/MD/AS. When we are at MD or AS, we read left to right.
}
\litem{
Simplify the expression below into the form $a+bi$. Then, choose the intervals that $a$ and $b$ belong to.
\[ (-3 - 7 i)(4 - 8 i) \]

The solution is \( -68 - 4 i \), which is option E.\begin{enumerate}[label=\Alph*.]
\item \( a \in [44, 47] \text{ and } b \in [50, 55] \)

 $44 + 52 i$, which corresponds to adding a minus sign in the first term.
\item \( a \in [-17, -11] \text{ and } b \in [54, 57] \)

 $-12 + 56 i$, which corresponds to just multiplying the real terms to get the real part of the solution and the coefficients in the complex terms to get the complex part.
\item \( a \in [44, 47] \text{ and } b \in [-55, -51] \)

 $44 - 52 i$, which corresponds to adding a minus sign in the second term.
\item \( a \in [-71, -66] \text{ and } b \in [4, 5] \)

 $-68 + 4 i$, which corresponds to adding a minus sign in both terms.
\item \( a \in [-71, -66] \text{ and } b \in [-5, 2] \)

* $-68 - 4 i$, which is the correct option.
\end{enumerate}

\textbf{General Comment:} You can treat $i$ as a variable and distribute. Just remember that $i^2=-1$, so you can continue to reduce after you distribute.
}
\end{enumerate}

\end{document}