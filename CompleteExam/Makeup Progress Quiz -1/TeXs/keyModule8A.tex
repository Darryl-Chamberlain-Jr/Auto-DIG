\documentclass{extbook}[14pt]
\usepackage{multicol, enumerate, enumitem, hyperref, color, soul, setspace, parskip, fancyhdr, amssymb, amsthm, amsmath, bbm, latexsym, units, mathtools}
\everymath{\displaystyle}
\usepackage[headsep=0.5cm,headheight=0cm, left=1 in,right= 1 in,top= 1 in,bottom= 1 in]{geometry}
\usepackage{dashrule}  % Package to use the command below to create lines between items
\newcommand{\litem}[1]{\item #1

\rule{\textwidth}{0.4pt}}
\pagestyle{fancy}
\lhead{}
\chead{Answer Key for Makeup Progress Quiz -1 Version A}
\rhead{}
\lfoot{7547-2949}
\cfoot{}
\rfoot{Fall 2020}
\begin{document}
\textbf{This key should allow you to understand why you choose the option you did (beyond just getting a question right or wrong). \href{https://xronos.clas.ufl.edu/mac1105spring2020/courseDescriptionAndMisc/Exams/LearningFromResults}{More instructions on how to use this key can be found here}.}

\textbf{If you have a suggestion to make the keys better, \href{https://forms.gle/CZkbZmPbC9XALEE88}{please fill out the short survey here}.}

\textit{Note: This key is auto-generated and may contain issues and/or errors. The keys are reviewed after each exam to ensure grading is done accurately. If there are issues (like duplicate options), they are noted in the offline gradebook. The keys are a work-in-progress to give students as many resources to improve as possible.}

\rule{\textwidth}{0.4pt}

\begin{enumerate}\litem{
Solve the equation for $x$ and choose the interval that contains the solution (if it exists).
\[ \log_{4}{(-3x+7)}+5 = 3 \]

The solution is \( x = 2.312 \), which is option A.\begin{enumerate}[label=\Alph*.]
\item \( x \in [-2.69, 7.31] \)

* $x = 2.312$, which is the correct option.
\item \( x \in [-16.67, -5.67] \)

$x = -7.667$, which corresponds to reversing the base and exponent when converting and reversing the value with $x$.
\item \( x \in [-19, -14] \)

$x = -19.000$, which corresponds to ignoring the vertical shift when converting to exponential form.
\item \( x \in [-7, -1] \)

$x = -3.000$, which corresponds to reversing the base and exponent when converting.
\item \( \text{There is no Real solution to the equation.} \)

Corresponds to believing a negative coefficient within the log equation means there is no Real solution.
\end{enumerate}

\textbf{General Comment:} \textbf{General Comments:} First, get the equation in the form $\log_b{(cx+d)} = a$. Then, convert to $b^a = cx+d$ and solve.
}
\litem{
 Solve the equation for $x$ and choose the interval that contains $x$ (if it exists).
\[  21 = \ln{\sqrt[6]{\frac{21}{e^{4x}}}} \]

The solution is \( x = -30.739 \), which is option A.\begin{enumerate}[label=\Alph*.]
\item \( x \in [-31.74, -29.74] \)

* $x = -30.739$, which is the correct option.
\item \( x \in [-6.33, -4.33] \)

$x = -5.328$, which corresponds to thinking you need to take the natural log of on the left before reducing.
\item \( x \in [-9.74, -7.74] \)

$x = -9.739$, which corresponds to treating any root as a square root.
\item \( \text{There is no Real solution to the equation.} \)

This corresponds to believing you cannot solve the equation.
\item \( \text{None of the above.} \)

This corresponds to making an unexpected error.
\end{enumerate}

\textbf{General Comment:} \textbf{General Comments}: After using the properties of logarithmic functions to break up the right-hand side, use $\ln(e) = 1$ to reduce the question to a linear function to solve. You can put $\ln(21)$ into a calculator if you are having trouble.
}
\litem{
Which of the following intervals describes the Range of the function below?
\[ f(x) = e^{x-5}-8 \]

The solution is \( (-8, \infty) \), which is option B.\begin{enumerate}[label=\Alph*.]
\item \( [a, \infty), a \in [-15, -5] \)

$[-8, \infty)$, which corresponds to including the endpoint.
\item \( (a, \infty), a \in [-15, -5] \)

* $(-8, \infty)$, which is the correct option.
\item \( (-\infty, a), a \in [8, 11] \)

$(-\infty, 8)$, which corresponds to using the negative vertical shift AND flipping the Range interval.
\item \( (-\infty, a], a \in [8, 11] \)

$(-\infty, 8]$, which corresponds to using the negative vertical shift AND flipping the Range interval AND including the endpoint.
\item \( (-\infty, \infty) \)

This corresponds to confusing range of an exponential function with the domain of an exponential function.
\end{enumerate}

\textbf{General Comment:} \textbf{General Comments}: Domain of a basic exponential function is $(-\infty, \infty)$ while the Range is $(0, \infty)$. We can shift these intervals [and even flip when $a<0$!] to find the new Domain/Range.
}
\litem{
Which of the following intervals describes the Range of the function below?
\[ f(x) = -e^{x+9}+9 \]

The solution is \( (-\infty, 9) \), which is option B.\begin{enumerate}[label=\Alph*.]
\item \( [a, \infty), a \in [-12, -5] \)

$[-9, \infty)$, which corresponds to using the negative vertical shift AND flipping the Range interval AND including the endpoint.
\item \( (-\infty, a), a \in [8, 12] \)

* $(-\infty, 9)$, which is the correct option.
\item \( (-\infty, a], a \in [8, 12] \)

$(-\infty, 9]$, which corresponds to including the endpoint.
\item \( (a, \infty), a \in [-12, -5] \)

$(-9, \infty)$, which corresponds to using the negative vertical shift AND flipping the Range interval.
\item \( (-\infty, \infty) \)

This corresponds to confusing range of an exponential function with the domain of an exponential function.
\end{enumerate}

\textbf{General Comment:} \textbf{General Comments}: Domain of a basic exponential function is $(-\infty, \infty)$ while the Range is $(0, \infty)$. We can shift these intervals [and even flip when $a<0$!] to find the new Domain/Range.
}
\litem{
Which of the following intervals describes the Domain of the function below?
\[ f(x) = -\log_2{(x-5)}-3 \]

The solution is \( (5, \infty) \), which is option A.\begin{enumerate}[label=\Alph*.]
\item \( (a, \infty), a \in [4.2, 7.4] \)

* $(5, \infty)$, which is the correct option.
\item \( (-\infty, a], a \in [0.9, 3.3] \)

$(-\infty, 3]$, which corresponds to using the negative vertical shift AND including the endpoint AND flipping the domain.
\item \( [a, \infty), a \in [-4.7, -2.2] \)

$[-3, \infty)$, which corresponds to using the vertical shift when shifting the Domain AND including the endpoint.
\item \( (-\infty, a), a \in [-7.9, -4.7] \)

$(-\infty, -5)$, which corresponds to flipping the Domain. Remember: the general for is $a*\log(x-h)+k$, \textbf{where $a$ does not affect the domain}.
\item \( (-\infty, \infty) \)

This corresponds to thinking of the range of the log function (or the domain of the exponential function).
\end{enumerate}

\textbf{General Comment:} \textbf{General Comments}: The domain of a basic logarithmic function is $(0, \infty)$ and the Range is $(-\infty, \infty)$. We can use shifts when finding the Domain, but the Range will always be all Real numbers.
}
\litem{
 Solve the equation for $x$ and choose the interval that contains $x$ (if it exists).
\[  19 = \ln{\sqrt[5]{\frac{14}{e^{6x}}}} \]

The solution is \( x = -15.393, \text{ which does not fit in any of the interval options.} \), which is option E.\begin{enumerate}[label=\Alph*.]
\item \( x \in [14.39, 18.39] \)

$x = 15.393$, which is the negative of the correct solution.
\item \( x \in [-5.89, -1.89] \)

$x = -2.894$, which corresponds to thinking you need to take the natural log of the left side before reducing.
\item \( x \in [-7.89, -3.89] \)

$x = -5.893$, which corresponds to treating any root as a square root.
\item \( \text{There is no Real solution to the equation.} \)

This corresponds to believing you cannot solve the equation.
\item \( \text{None of the above.} \)

*$x = -15.393$ is the correct solution and does not fit in any of the other intervals.
\end{enumerate}

\textbf{General Comment:} \textbf{General Comments}: After using the properties of logarithmic functions to break up the right-hand side, use $\ln(e) = 1$ to reduce the question to a linear function to solve. You can put $\ln(14)$ into a calculator if you are having trouble.
}
\litem{
Which of the following intervals describes the Domain of the function below?
\[ f(x) = \log_2{(x-9)}+1 \]

The solution is \( (9, \infty) \), which is option C.\begin{enumerate}[label=\Alph*.]
\item \( (-\infty, a), a \in [-14, -5] \)

$(-\infty, -9)$, which corresponds to flipping the Domain. Remember: the general for is $a*\log(x-h)+k$, \textbf{where $a$ does not affect the domain}.
\item \( (-\infty, a], a \in [-5, 0] \)

$(-\infty, -1]$, which corresponds to using the negative vertical shift AND including the endpoint AND flipping the domain.
\item \( (a, \infty), a \in [5, 10] \)

* $(9, \infty)$, which is the correct option.
\item \( [a, \infty), a \in [0, 4] \)

$[1, \infty)$, which corresponds to using the vertical shift when shifting the Domain AND including the endpoint.
\item \( (-\infty, \infty) \)

This corresponds to thinking of the range of the log function (or the domain of the exponential function).
\end{enumerate}

\textbf{General Comment:} \textbf{General Comments}: The domain of a basic logarithmic function is $(0, \infty)$ and the Range is $(-\infty, \infty)$. We can use shifts when finding the Domain, but the Range will always be all Real numbers.
}
\litem{
Solve the equation for $x$ and choose the interval that contains the solution (if it exists).
\[ 4^{-5x-3} = 49^{-2x+4} \]

The solution is \( x = 23.148 \), which is option B.\begin{enumerate}[label=\Alph*.]
\item \( x \in [5.21, 13.21] \)

$x = 8.214$, which corresponds to distributing the $\ln(base)$ to the first term of the exponent only.
\item \( x \in [21.15, 25.15] \)

* $x = 23.148$, which is the correct option.
\item \( x \in [-6.58, -3.58] \)

$x = -6.575$, which corresponds to distributing the $\ln(base)$ to the second term of the exponent only.
\item \( x \in [-3.33, -0.33] \)

$x = -2.333$, which corresponds to solving the numerators as equal while ignoring the bases are different.
\item \( \text{There is no Real solution to the equation.} \)

This corresponds to believing there is no solution since the bases are not powers of each other.
\end{enumerate}

\textbf{General Comment:} \textbf{General Comments:} This question was written so that the bases could not be written the same. You will need to take the log of both sides.
}
\litem{
Solve the equation for $x$ and choose the interval that contains the solution (if it exists).
\[ 5^{4x+2} = \left(\frac{1}{27}\right)^{-2x-5} \]

The solution is \( x = -86.149 \), which is option D.\begin{enumerate}[label=\Alph*.]
\item \( x \in [0.21, 3.21] \)

$x = 2.210$, which corresponds to distributing the $\ln(base)$ to the second term of the exponent only.
\item \( x \in [41.48, 48.48] \)

$x = 45.478$, which corresponds to distributing the $\ln(base)$ to the first term of the exponent only.
\item \( x \in [-4.17, 1.83] \)

$x = -1.167$, which corresponds to solving the numerators as equal while ignoring the bases are different.
\item \( x \in [-87.15, -82.15] \)

* $x = -86.149$, which is the correct option.
\item \( \text{There is no Real solution to the equation.} \)

This corresponds to believing there is no solution since the bases are not powers of each other.
\end{enumerate}

\textbf{General Comment:} \textbf{General Comments:} This question was written so that the bases could not be written the same. You will need to take the log of both sides.
}
\litem{
Solve the equation for $x$ and choose the interval that contains the solution (if it exists).
\[ \log_{3}{(4x+7)}+6 = 2 \]

The solution is \( x = -1.747 \), which is option D.\begin{enumerate}[label=\Alph*.]
\item \( x \in [-0.5, 8.5] \)

$x = 0.500$, which corresponds to ignoring the vertical shift when converting to exponential form.
\item \( x \in [-14.25, -13.25] \)

$x = -14.250$, which corresponds to reversing the base and exponent when converting and reversing the value with $x$.
\item \( x \in [-20.75, -14.75] \)

$x = -17.750$, which corresponds to reversing the base and exponent when converting.
\item \( x \in [-1.75, 0.25] \)

* $x = -1.747$, which is the correct option.
\item \( \text{There is no Real solution to the equation.} \)

Corresponds to believing a negative coefficient within the log equation means there is no Real solution.
\end{enumerate}

\textbf{General Comment:} \textbf{General Comments:} First, get the equation in the form $\log_b{(cx+d)} = a$. Then, convert to $b^a = cx+d$ and solve.
}
\end{enumerate}

\end{document}