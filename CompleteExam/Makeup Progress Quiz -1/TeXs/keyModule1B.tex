\documentclass{extbook}[14pt]
\usepackage{multicol, enumerate, enumitem, hyperref, color, soul, setspace, parskip, fancyhdr, amssymb, amsthm, amsmath, bbm, latexsym, units, mathtools}
\everymath{\displaystyle}
\usepackage[headsep=0.5cm,headheight=0cm, left=1 in,right= 1 in,top= 1 in,bottom= 1 in]{geometry}
\usepackage{dashrule}  % Package to use the command below to create lines between items
\newcommand{\litem}[1]{\item #1

\rule{\textwidth}{0.4pt}}
\pagestyle{fancy}
\lhead{}
\chead{Answer Key for Makeup Progress Quiz -1 Version B}
\rhead{}
\lfoot{7547-2949}
\cfoot{}
\rfoot{Fall 2020}
\begin{document}
\textbf{This key should allow you to understand why you choose the option you did (beyond just getting a question right or wrong). \href{https://xronos.clas.ufl.edu/mac1105spring2020/courseDescriptionAndMisc/Exams/LearningFromResults}{More instructions on how to use this key can be found here}.}

\textbf{If you have a suggestion to make the keys better, \href{https://forms.gle/CZkbZmPbC9XALEE88}{please fill out the short survey here}.}

\textit{Note: This key is auto-generated and may contain issues and/or errors. The keys are reviewed after each exam to ensure grading is done accurately. If there are issues (like duplicate options), they are noted in the offline gradebook. The keys are a work-in-progress to give students as many resources to improve as possible.}

\rule{\textwidth}{0.4pt}

\begin{enumerate}\litem{
Choose the \textbf{smallest} set of Real numbers that the number below belongs to.
\[ \sqrt{\frac{3600}{100}} \]

The solution is \( \text{Whole} \), which is option E.\begin{enumerate}[label=\Alph*.]
\item \( \text{Rational} \)

These are numbers that can be written as fraction of Integers (e.g., -2/3)
\item \( \text{Not a Real number} \)

These are Nonreal Complex numbers \textbf{OR} things that are not numbers (e.g., dividing by 0).
\item \( \text{Irrational} \)

These cannot be written as a fraction of Integers.
\item \( \text{Integer} \)

These are the negative and positive counting numbers (..., -3, -2, -1, 0, 1, 2, 3, ...)
\item \( \text{Whole} \)

* This is the correct option!
\end{enumerate}

\textbf{General Comment:} First, you \textbf{NEED} to simplify the expression. This question simplifies to $60$. 
 
 Be sure you look at the simplified fraction and not just the decimal expansion. Numbers such as 13, 17, and 19 provide \textbf{long but repeating/terminating decimal expansions!} 
 
 The only ways to *not* be a Real number are: dividing by 0 or taking the square root of a negative number. 
 
 Irrational numbers are more than just square root of 3: adding or subtracting values from square root of 3 is also irrational.
}
\litem{
Simplify the expression below into the form $a+bi$. Then, choose the intervals that $a$ and $b$ belong to.
\[ (-6 - 8 i)(-4 + 7 i) \]

The solution is \( 80 - 10 i \), which is option C.\begin{enumerate}[label=\Alph*.]
\item \( a \in [-33, -24] \text{ and } b \in [74, 75] \)

 $-32 + 74 i$, which corresponds to adding a minus sign in the second term.
\item \( a \in [18, 26] \text{ and } b \in [-57, -55] \)

 $24 - 56 i$, which corresponds to just multiplying the real terms to get the real part of the solution and the coefficients in the complex terms to get the complex part.
\item \( a \in [80, 82] \text{ and } b \in [-10, -7] \)

* $80 - 10 i$, which is the correct option.
\item \( a \in [-33, -24] \text{ and } b \in [-79, -73] \)

 $-32 - 74 i$, which corresponds to adding a minus sign in the first term.
\item \( a \in [80, 82] \text{ and } b \in [8, 11] \)

 $80 + 10 i$, which corresponds to adding a minus sign in both terms.
\end{enumerate}

\textbf{General Comment:} You can treat $i$ as a variable and distribute. Just remember that $i^2=-1$, so you can continue to reduce after you distribute.
}
\litem{
Simplify the expression below into the form $a+bi$. Then, choose the intervals that $a$ and $b$ belong to.
\[ \frac{63 - 66 i}{-5 - 3 i} \]

The solution is \( -3.44  + 15.26 i \), which is option D.\begin{enumerate}[label=\Alph*.]
\item \( a \in [-14, -12] \text{ and } b \in [21, 22.5] \)

 $-12.60  + 22.00 i$, which corresponds to just dividing the first term by the first term and the second by the second.
\item \( a \in [-5.5, -2.5] \text{ and } b \in [517.5, 520] \)

 $-3.44  + 519.00 i$, which corresponds to forgetting to multiply the conjugate by the numerator.
\item \( a \in [-16.5, -14] \text{ and } b \in [3, 5] \)

 $-15.09  + 4.15 i$, which corresponds to forgetting to multiply the conjugate by the numerator and not computing the conjugate correctly.
\item \( a \in [-5.5, -2.5] \text{ and } b \in [14, 16.5] \)

* $-3.44  + 15.26 i$, which is the correct option.
\item \( a \in [-117.5, -116] \text{ and } b \in [14, 16.5] \)

 $-117.00  + 15.26 i$, which corresponds to forgetting to multiply the conjugate by the numerator and using a plus instead of a minus in the denominator.
\end{enumerate}

\textbf{General Comment:} Multiply the numerator and denominator by the *conjugate* of the denominator, then simplify. For example, if we have $2+3i$, the conjugate is $2-3i$.
}
\litem{
Choose the \textbf{smallest} set of Complex numbers that the number below belongs to.
\[ \sqrt{\frac{3120}{15}}+\sqrt{154} i \]

The solution is \( \text{Nonreal Complex} \), which is option E.\begin{enumerate}[label=\Alph*.]
\item \( \text{Not a Complex Number} \)

This is not a number. The only non-Complex number we know is dividing by 0 as this is not a number!
\item \( \text{Irrational} \)

These cannot be written as a fraction of Integers. Remember: $\pi$ is not an Integer!
\item \( \text{Rational} \)

These are numbers that can be written as fraction of Integers (e.g., -2/3 + 5)
\item \( \text{Pure Imaginary} \)

This is a Complex number $(a+bi)$ that \textbf{only} has an imaginary part like $2i$.
\item \( \text{Nonreal Complex} \)

* This is the correct option!
\end{enumerate}

\textbf{General Comment:} Be sure to simplify $i^2 = -1$. This may remove the imaginary portion for your number. If you are having trouble, you may want to look at the \textit{Subgroups of the Real Numbers} section.
}
\litem{
Simplify the expression below into the form $a+bi$. Then, choose the intervals that $a$ and $b$ belong to.
\[ \frac{45 + 22 i}{-1 + 3 i} \]

The solution is \( 2.10  - 15.70 i \), which is option A.\begin{enumerate}[label=\Alph*.]
\item \( a \in [1.5, 3.5] \text{ and } b \in [-16, -14.5] \)

* $2.10  - 15.70 i$, which is the correct option.
\item \( a \in [-12, -10] \text{ and } b \in [10.5, 12] \)

 $-11.10  + 11.30 i$, which corresponds to forgetting to multiply the conjugate by the numerator and not computing the conjugate correctly.
\item \( a \in [-45.5, -44] \text{ and } b \in [6.5, 8] \)

 $-45.00  + 7.33 i$, which corresponds to just dividing the first term by the first term and the second by the second.
\item \( a \in [1.5, 3.5] \text{ and } b \in [-157.5, -156.5] \)

 $2.10  - 157.00 i$, which corresponds to forgetting to multiply the conjugate by the numerator.
\item \( a \in [20, 21.5] \text{ and } b \in [-16, -14.5] \)

 $21.00  - 15.70 i$, which corresponds to forgetting to multiply the conjugate by the numerator and using a plus instead of a minus in the denominator.
\end{enumerate}

\textbf{General Comment:} Multiply the numerator and denominator by the *conjugate* of the denominator, then simplify. For example, if we have $2+3i$, the conjugate is $2-3i$.
}
\litem{
Simplify the expression below and choose the interval the simplification is contained within.
\[ 6 - 7 \div 5 * 14 - (1 * 19) \]

The solution is \( -32.600 \), which is option D.\begin{enumerate}[label=\Alph*.]
\item \( [-13.1, -9.1] \)

 -13.100, which corresponds to an Order of Operations error: not reading left-to-right for multiplication/division.
\item \( [24.9, 25.9] \)

 24.900, which corresponds to not distributing addition and subtraction correctly.
\item \( [-277.4, -274.4] \)

 -277.400, which corresponds to not distributing a negative correctly.
\item \( [-33.6, -27.6] \)

* -32.600, which is the correct option.
\item \( \text{None of the above} \)

 You may have gotten this by making an unanticipated error. If you got a value that is not any of the others, please let the coordinator know so they can help you figure out what happened.
\end{enumerate}

\textbf{General Comment:} While you may remember (or were taught) PEMDAS is done in order, it is actually done as P/E/MD/AS. When we are at MD or AS, we read left to right.
}
\litem{
Choose the \textbf{smallest} set of Real numbers that the number below belongs to.
\[ -\sqrt{\frac{41616}{289}} \]

The solution is \( \text{Integer} \), which is option E.\begin{enumerate}[label=\Alph*.]
\item \( \text{Whole} \)

These are the counting numbers with 0 (0, 1, 2, 3, ...)
\item \( \text{Irrational} \)

These cannot be written as a fraction of Integers.
\item \( \text{Not a Real number} \)

These are Nonreal Complex numbers \textbf{OR} things that are not numbers (e.g., dividing by 0).
\item \( \text{Rational} \)

These are numbers that can be written as fraction of Integers (e.g., -2/3)
\item \( \text{Integer} \)

* This is the correct option!
\end{enumerate}

\textbf{General Comment:} First, you \textbf{NEED} to simplify the expression. This question simplifies to $-204$. 
 
 Be sure you look at the simplified fraction and not just the decimal expansion. Numbers such as 13, 17, and 19 provide \textbf{long but repeating/terminating decimal expansions!} 
 
 The only ways to *not* be a Real number are: dividing by 0 or taking the square root of a negative number. 
 
 Irrational numbers are more than just square root of 3: adding or subtracting values from square root of 3 is also irrational.
}
\litem{
Choose the \textbf{smallest} set of Complex numbers that the number below belongs to.
\[ \frac{\sqrt{182}}{19}+\sqrt{-4}i \]

The solution is \( \text{Irrational} \), which is option E.\begin{enumerate}[label=\Alph*.]
\item \( \text{Not a Complex Number} \)

This is not a number. The only non-Complex number we know is dividing by 0 as this is not a number!
\item \( \text{Rational} \)

These are numbers that can be written as fraction of Integers (e.g., -2/3 + 5)
\item \( \text{Nonreal Complex} \)

This is a Complex number $(a+bi)$ that is not Real (has $i$ as part of the number).
\item \( \text{Pure Imaginary} \)

This is a Complex number $(a+bi)$ that \textbf{only} has an imaginary part like $2i$.
\item \( \text{Irrational} \)

* This is the correct option!
\end{enumerate}

\textbf{General Comment:} Be sure to simplify $i^2 = -1$. This may remove the imaginary portion for your number. If you are having trouble, you may want to look at the \textit{Subgroups of the Real Numbers} section.
}
\litem{
Simplify the expression below and choose the interval the simplification is contained within.
\[ 9 - 10^2 + 6 \div 5 * 7 \div 2 \]

The solution is \( -86.800 \), which is option A.\begin{enumerate}[label=\Alph*.]
\item \( [-87.8, -84.8] \)

* -86.800, this is the correct option
\item \( [-93.91, -88.91] \)

 -90.914, which corresponds to an Order of Operations error: not reading left-to-right for multiplication/division.
\item \( [108.09, 112.09] \)

 109.086, which corresponds to two Order of Operations errors.
\item \( [113.2, 122.2] \)

 113.200, which corresponds to an Order of Operations error: multiplying by negative before squaring. For example: $(-3)^2 \neq -3^2$
\item \( \text{None of the above} \)

 You may have gotten this by making an unanticipated error. If you got a value that is not any of the others, please let the coordinator know so they can help you figure out what happened.
\end{enumerate}

\textbf{General Comment:} While you may remember (or were taught) PEMDAS is done in order, it is actually done as P/E/MD/AS. When we are at MD or AS, we read left to right.
}
\litem{
Simplify the expression below into the form $a+bi$. Then, choose the intervals that $a$ and $b$ belong to.
\[ (-4 + 3 i)(-5 - 9 i) \]

The solution is \( 47 + 21 i \), which is option A.\begin{enumerate}[label=\Alph*.]
\item \( a \in [47, 48] \text{ and } b \in [18, 28] \)

* $47 + 21 i$, which is the correct option.
\item \( a \in [-17, -3] \text{ and } b \in [-58, -49] \)

 $-7 - 51 i$, which corresponds to adding a minus sign in the second term.
\item \( a \in [47, 48] \text{ and } b \in [-23, -15] \)

 $47 - 21 i$, which corresponds to adding a minus sign in both terms.
\item \( a \in [17, 27] \text{ and } b \in [-28, -26] \)

 $20 - 27 i$, which corresponds to just multiplying the real terms to get the real part of the solution and the coefficients in the complex terms to get the complex part.
\item \( a \in [-17, -3] \text{ and } b \in [47, 54] \)

 $-7 + 51 i$, which corresponds to adding a minus sign in the first term.
\end{enumerate}

\textbf{General Comment:} You can treat $i$ as a variable and distribute. Just remember that $i^2=-1$, so you can continue to reduce after you distribute.
}
\end{enumerate}

\end{document}