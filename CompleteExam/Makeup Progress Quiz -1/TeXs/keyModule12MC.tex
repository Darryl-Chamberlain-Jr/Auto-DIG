\documentclass{extbook}[14pt]
\usepackage{multicol, enumerate, enumitem, hyperref, color, soul, setspace, parskip, fancyhdr, amssymb, amsthm, amsmath, bbm, latexsym, units, mathtools}
\everymath{\displaystyle}
\usepackage[headsep=0.5cm,headheight=0cm, left=1 in,right= 1 in,top= 1 in,bottom= 1 in]{geometry}
\usepackage{dashrule}  % Package to use the command below to create lines between items
\newcommand{\litem}[1]{\item #1

\rule{\textwidth}{0.4pt}}
\pagestyle{fancy}
\lhead{}
\chead{Answer Key for Makeup Progress Quiz -1 Version C}
\rhead{}
\lfoot{7547-2949}
\cfoot{}
\rfoot{Fall 2020}
\begin{document}
\textbf{This key should allow you to understand why you choose the option you did (beyond just getting a question right or wrong). \href{https://xronos.clas.ufl.edu/mac1105spring2020/courseDescriptionAndMisc/Exams/LearningFromResults}{More instructions on how to use this key can be found here}.}

\textbf{If you have a suggestion to make the keys better, \href{https://forms.gle/CZkbZmPbC9XALEE88}{please fill out the short survey here}.}

\textit{Note: This key is auto-generated and may contain issues and/or errors. The keys are reviewed after each exam to ensure grading is done accurately. If there are issues (like duplicate options), they are noted in the offline gradebook. The keys are a work-in-progress to give students as many resources to improve as possible.}

\rule{\textwidth}{0.4pt}

\begin{enumerate}\litem{
Solve the modeling problem below, if possible.

\begin{center}
    \textit{ A new virus is spreading throughout the world. There were initially 5 many cases reported, but the number of confirmed cases has doubled every 3 days. How long will it be until there are at least 1000000 confirmed cases? }
\end{center}


The solution is \( \text{About } 53 \text{ days} \), which is option C.\begin{enumerate}[label=\Alph*.]
\item \( \text{About } 16 \text{ days} \)

You modeled the situation with $e$ as the base and did not apply the properties of log correctly.
\item \( \text{About } 18 \text{ days} \)

You modeled the situation correctly but did not apply the properties of log correctly.
\item \( \text{About } 53 \text{ days} \)

* This is the correct option.
\item \( \text{About } 37 \text{ days} \)

You modeled the situation with $e$ as the base, but solved correctly otherwise.
\item \( \text{There is not enough information to solve the problem.} \)

If you chose this option, please contact the coordinator to discuss why you think this is the case.
\end{enumerate}

\textbf{General Comment:} Set up the model the same as in Module 11M. Then, plug in 1000000 and solve for $d$ in your model.
}
\litem{
Determine the appropriate model for the graph of points below.

\begin{center}
    \includegraphics[width=0.5\textwidth]{../Figures/identifyModelGraph12CopyC.png}
\end{center}




The solution is \( \text{Linear model} \), which is option C.\begin{enumerate}[label=\Alph*.]
\item \( \text{Exponential model} \)

For this to be the correct option, we want an extremely slow change early, then a rapid change later.
\item \( \text{Non-linear Power model} \)

For this to be the correct option, we need to see a polynomial or rational shape.
\item \( \text{Linear model} \)

For this to be the correct option, we need to see a mostly straight line of points.
\item \( \text{Logarithmic model} \)

For this to be the correct option, we want a rapid change early, then an extremely slow change later.
\item \( \text{None of the above} \)

For this to be the correct option, we want to see no pattern in the points.
\end{enumerate}

\textbf{General Comment:} This question is testing if you can associate the models with their graphical representation. If you are having trouble, go back to the corresponding Core module to learn about the specific function you are having trouble recognizing.
}
\litem{
For the scenario below, use the model for the volume of a cylinder as $V = \pi r^2 h$ to find the coefficient for the model of the new volume $V_{	ext{new}} = k r^2 h$.

\begin{center}
    \textit{ Pepsi wants to increase the volume of soda in their cans. They've decided to decrease the radius by 17 percent and increase the height by 12 percent. They want to model the new volume based on the radius and height of the original cans. }
\end{center}


The solution is \( k = 2.42395 \), which is option C.\begin{enumerate}[label=\Alph*.]
\item \( k = 0.01090 \)

This corresponds to the model: $V = \pi (0.17 r)^2 (0.12 h)$.
\item \( k = 0.77157 \)

This corresponds to the model: $V = (0.83 r)^2 (1.12 h)$.
\item \( k = 2.42395 \)

* This is the correct option and corresponds to the model: $V = \pi (0.83 r)^2 (1.12 h)$.
\item \( k = 0.00347 \)

This corresponds to the model: $V = (0.17 r)^2 (0.12 h)$.
\item \( \text{None of the above.} \)

If you chose this, please talk with the coordinator to discuss why you believe none of the options are correct.
\end{enumerate}

\textbf{General Comment:} When calculating the new dimensions, you need to add/subtract from 100\%. For example, a 10\% increase in height would result in 110\% of the original height: $1.1h_{old} = h_{new}$.
}
\litem{
Determine the appropriate model for the graph of points below.

\begin{center}
    \includegraphics[width=0.5\textwidth]{../Figures/identifyModelGraph12C.png}
\end{center}




The solution is \( \text{Non-linear Power model} \), which is option D.\begin{enumerate}[label=\Alph*.]
\item \( \text{Logarithmic model} \)

For this to be the correct option, we want a rapid change early, then an extremely slow change later.
\item \( \text{Linear model} \)

For this to be the correct option, we need to see a mostly straight line of points.
\item \( \text{Exponential model} \)

For this to be the correct option, we want an extremely slow change early, then a rapid change later.
\item \( \text{Non-linear Power model} \)

For this to be the correct option, we need to see a polynomial or rational shape.
\item \( \text{None of the above} \)

For this to be the correct option, we want to see no pattern in the points.
\end{enumerate}

\textbf{General Comment:} This question is testing if you can associate the models with their graphical representation. If you are having trouble, go back to the corresponding Core module to learn about the specific function you are having trouble recognizing.
}
\litem{
Solve the modeling problem below, if possible.

\begin{center}
    \textit{ In CHM2045L, Brittany created a 23 liter 25 percent solution of chemical $\chi$ using two different solution percentages of chemical $\chi$. When she went to write her lab report, she realized she forgot to write the amount of each solution she used! If she remembers she used 18 percent and 46 percent solutions, what was the amount she used of the 46 percent solution? }
\end{center}


The solution is \( 5.75 \), which is option B.\begin{enumerate}[label=\Alph*.]
\item \( 12.70 \)

This was a random value. If this was not a guess, contact the coordinator to talk about how you got this value.
\item \( 5.75 \)

*This is the correct option.
\item \( 17.25 \)

This is the concentration of 18 percent solution.
\item \( 11.50 \)

This would be correct if Brittany used equal parts of each solution.
\item \( \text{There is not enough information to solve the problem.} \)

You may have chose this if you thought you needed to know how much of the second solution was used in the problem. Remember that the total minus the first solution would give you the second amount used.
\end{enumerate}

\textbf{General Comment:} Build the model exactly as you did in Module 9M. Then, solve for the volume you are looking for.
}
\litem{
For the scenario below, use the model for the volume of a cylinder as $V = \pi r^2 h$.

\begin{center}
    \textit{ Pringles wants to add 28 \text{percent} more chips to their cylinder cans and minimize the design change of their cans. They've decided that the best way to minimize the design change is to increase the radius and height by the same percentage. What should this increase be? }
\end{center}


The solution is \( \text{About } 9 \text{ percent} \), which is option B.\begin{enumerate}[label=\Alph*.]
\item \( \text{About } 14 \text{ percent} \)

This corresponds to treating both radius and height as equal contributors and not solving correctly.
\item \( \text{About } 9 \text{ percent} \)

* This is the correct option.
\item \( \text{About } 3 \text{ percent} \)

This corresponds to not solving for the increase properly.
\item \( \text{About } 13 \text{ percent} \)

This corresponds to solving correctly but treating both radius and height as equal contributors to the volume.
\item \( \text{None of the above} \)

If you chose this, please contact the coordinator to discus how you solved the problem.
\end{enumerate}

\textbf{General Comment:} Remember that when plugging the increases of values in, you need to treat it as that percentage above 100. For example, a 5 percent increase means 105 percent.
}
\litem{
The temperature of an object, $T$, in a different surrounding temperature $T_s$ will behave according to the formula $T(t) = Ae^{kt} + T_s$, where $t$ is minutes, $A$ is a constant, and k is a constant. Use this formula and the situation below to construct a model that describes the uranium's temperature, $T$, based on the amount of time t (in minutes) that have passed. Choose the correct constant $k$ from the options below.

\begin{center}
    \textit{ Uranium is taken out of the reactor with a temperature of $130^{\circ}$ C and is placed into a $12^{\circ}$ C bath to cool. After 23 minutes, the uranium has cooled to $88^{\circ}$ C. }
\end{center}


The solution is \( k = -0.01913 \), which is option D.\begin{enumerate}[label=\Alph*.]
\item \( k = -0.03209 \)

This uses $A$ as the initial temperature and solves for $k$ correctly.
\item \( k = -0.02334 \)

This uses $A$ as the initial temperature and solves for $k$ incorrectly.
\item \( k = -0.03263 \)

This uses $A$ correctly but solves for $k$ incorrectly.
\item \( k = -0.01913 \)

* This is the correct option.
\item \( \text{None of the above} \)

If you chose this, please contact the coordinator to discuss why you believe none of the other answers are correct.
\end{enumerate}

\textbf{General Comment:} The initial temperature is when $t = 0$. Unlike power models, that means $A$ is not the initial temperature!
}
\litem{
Solve the modeling problem below, if possible.

\begin{center}
    \textit{ A new virus is spreading throughout the world. There were initially 4 many cases reported, but the number of confirmed cases has tripled every 5 days. How long will it be until there are at least 1000 confirmed cases? }
\end{center}


The solution is \( \text{About } 26 \text{ days} \), which is option D.\begin{enumerate}[label=\Alph*.]
\item \( \text{About } 28 \text{ days} \)

You modeled the situation with $e$ as the base, but solved correctly otherwise.
\item \( \text{About } 14 \text{ days} \)

You modeled the situation correctly but did not apply the properties of log correctly.
\item \( \text{About } 15 \text{ days} \)

You modeled the situation with $e$ as the base and did not apply the properties of log correctly.
\item \( \text{About } 26 \text{ days} \)

* This is the correct option.
\item \( \text{There is not enough information to solve the problem.} \)

If you chose this option, please contact the coordinator to discuss why you think this is the case.
\end{enumerate}

\textbf{General Comment:} Set up the model the same as in Module 11M. Then, plug in 1000 and solve for $d$ in your model.
}
\litem{
Solve the modeling problem below, if possible.

\begin{center}
    \textit{ In CHM2045L, Brittany created a 27 liter 17 percent solution of chemical $\chi$ using two different solution percentages of chemical $\chi$. When she went to write her lab report, she realized she forgot to write the amount of each solution she used! If she remembers she used 12 percent and 28 percent solutions, what was the amount she used of the 12 percent solution? }
\end{center}


The solution is \( 18.56 \), which is option A.\begin{enumerate}[label=\Alph*.]
\item \( 18.56 \)

*This is the correct option.
\item \( 15.37 \)

This was a random value. If this was not a guess, contact the coordinator to talk about how you got this value.
\item \( 13.50 \)

This would be correct if Brittany used equal parts of each solution.
\item \( 8.44 \)

This is the concentration of 28 percent solution.
\item \( \text{There is not enough information to solve the problem.} \)

You may have chose this if you thought you needed to know how much of the second solution was used in the problem. Remember that the total minus the first solution would give you the second amount used.
\end{enumerate}

\textbf{General Comment:} Build the model exactly as you did in Module 9M. Then, solve for the volume you are looking for.
}
\litem{
For the scenario below, use the model for the volume of a cylinder as $V = \pi r^2 h$.

\begin{center}
    \textit{ Pringles wants to add 27 \text{percent} more chips to their cylinder cans and minimize the design change of their cans. They've decided that the best way to minimize the design change is to increase the radius and height by the same percentage. What should this increase be? }
\end{center}


The solution is \( \text{About } 8 \text{ percent} \), which is option D.\begin{enumerate}[label=\Alph*.]
\item \( \text{About } 14 \text{ percent} \)

This corresponds to treating both radius and height as equal contributors and not solving correctly.
\item \( \text{About } 3 \text{ percent} \)

This corresponds to not solving for the increase properly.
\item \( \text{About } 13 \text{ percent} \)

This corresponds to solving correctly but treating both radius and height as equal contributors to the volume.
\item \( \text{About } 8 \text{ percent} \)

* This is the correct option.
\item \( \text{None of the above} \)

If you chose this, please contact the coordinator to discus how you solved the problem.
\end{enumerate}

\textbf{General Comment:} Remember that when plugging the increases of values in, you need to treat it as that percentage above 100. For example, a 5 percent increase means 105 percent.
}
\end{enumerate}

\end{document}