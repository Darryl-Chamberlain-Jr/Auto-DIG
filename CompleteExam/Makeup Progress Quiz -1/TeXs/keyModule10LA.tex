\documentclass{extbook}[14pt]
\usepackage{multicol, enumerate, enumitem, hyperref, color, soul, setspace, parskip, fancyhdr, amssymb, amsthm, amsmath, bbm, latexsym, units, mathtools}
\everymath{\displaystyle}
\usepackage[headsep=0.5cm,headheight=0cm, left=1 in,right= 1 in,top= 1 in,bottom= 1 in]{geometry}
\usepackage{dashrule}  % Package to use the command below to create lines between items
\newcommand{\litem}[1]{\item #1

\rule{\textwidth}{0.4pt}}
\pagestyle{fancy}
\lhead{}
\chead{Answer Key for Makeup Progress Quiz -1 Version A}
\rhead{}
\lfoot{7547-2949}
\cfoot{}
\rfoot{Fall 2020}
\begin{document}
\textbf{This key should allow you to understand why you choose the option you did (beyond just getting a question right or wrong). \href{https://xronos.clas.ufl.edu/mac1105spring2020/courseDescriptionAndMisc/Exams/LearningFromResults}{More instructions on how to use this key can be found here}.}

\textbf{If you have a suggestion to make the keys better, \href{https://forms.gle/CZkbZmPbC9XALEE88}{please fill out the short survey here}.}

\textit{Note: This key is auto-generated and may contain issues and/or errors. The keys are reviewed after each exam to ensure grading is done accurately. If there are issues (like duplicate options), they are noted in the offline gradebook. The keys are a work-in-progress to give students as many resources to improve as possible.}

\rule{\textwidth}{0.4pt}

\begin{enumerate}\litem{
What are the \textit{possible Integer} roots of the polynomial below?
\[ f(x) = 2x^{3} +7 x^{2} +5 x + 3 \]

The solution is \( \pm 1,\pm 3 \), which is option A.\begin{enumerate}[label=\Alph*.]
\item \( \pm 1,\pm 3 \)

* This is the solution \textbf{since we asked for the possible Integer roots}!
\item \( \text{ All combinations of: }\frac{\pm 1,\pm 3}{\pm 1,\pm 2} \)

This would have been the solution \textbf{if asked for the possible Rational roots}!
\item \( \pm 1,\pm 2 \)

 Distractor 1: Corresponds to the plus or minus factors of a1 only.
\item \( \text{ All combinations of: }\frac{\pm 1,\pm 2}{\pm 1,\pm 3} \)

 Distractor 3: Corresponds to the plus or minus of the inverse quotient (an/a0) of the factors. 
\item \( \text{There is no formula or theorem that tells us all possible Integer roots.} \)

 Distractor 4: Corresponds to not recognizing Integers as a subset of Rationals.
\end{enumerate}

\textbf{General Comment:} We have a way to find the possible Rational roots. The possible Integer roots are the Integers in this list.
}
\litem{
Factor the polynomial below completely. Then, choose the intervals the zeros of the polynomial belong to, where $z_1 \leq z_2 \leq z_3$. \textit{To make the problem easier, all zeros are between -5 and 5.}
\[ f(x) = 10x^{3} -11 x^{2} -106 x -40 \]

The solution is \( [-2.5, -0.4, 4] \), which is option A.\begin{enumerate}[label=\Alph*.]
\item \( z_1 \in [-2.5, -0.5], \text{   }  z_2 \in [-0.49, -0.31], \text{   and   } z_3 \in [2.6, 4.9] \)

* This is the solution!
\item \( z_1 \in [-6, -3], \text{   }  z_2 \in [0.21, 0.47], \text{   and   } z_3 \in [0.9, 2.8] \)

 Distractor 1: Corresponds to negatives of all zeros.
\item \( z_1 \in [-6, -3], \text{   }  z_2 \in [0.21, 0.47], \text{   and   } z_3 \in [0.9, 2.8] \)

 Distractor 3: Corresponds to negatives of all zeros AND inversing rational roots.
\item \( z_1 \in [-2.5, -0.5], \text{   }  z_2 \in [-0.49, -0.31], \text{   and   } z_3 \in [2.6, 4.9] \)

 Distractor 2: Corresponds to inversing rational roots.
\item \( z_1 \in [-6, -3], \text{   }  z_2 \in [0, 0.2], \text{   and   } z_3 \in [4.4, 5.1] \)

 Distractor 4: Corresponds to moving factors from one rational to another.
\end{enumerate}

\textbf{General Comment:} Remember to try the middle-most integers first as these normally are the zeros. Also, once you get it to a quadratic, you can use your other factoring techniques to finish factoring.
}
\litem{
What are the \textit{possible Rational} roots of the polynomial below?
\[ f(x) = 4x^{3} +7 x^{2} +5 x + 7 \]

The solution is \( \text{ All combinations of: }\frac{\pm 1,\pm 7}{\pm 1,\pm 2,\pm 4} \), which is option C.\begin{enumerate}[label=\Alph*.]
\item \( \pm 1,\pm 2,\pm 4 \)

 Distractor 1: Corresponds to the plus or minus factors of a1 only.
\item \( \text{ All combinations of: }\frac{\pm 1,\pm 2,\pm 4}{\pm 1,\pm 7} \)

 Distractor 3: Corresponds to the plus or minus of the inverse quotient (an/a0) of the factors. 
\item \( \text{ All combinations of: }\frac{\pm 1,\pm 7}{\pm 1,\pm 2,\pm 4} \)

* This is the solution \textbf{since we asked for the possible Rational roots}!
\item \( \pm 1,\pm 7 \)

This would have been the solution \textbf{if asked for the possible Integer roots}!
\item \( \text{ There is no formula or theorem that tells us all possible Rational roots.} \)

 Distractor 4: Corresponds to not recalling the theorem for rational roots of a polynomial.
\end{enumerate}

\textbf{General Comment:} We have a way to find the possible Rational roots. The possible Integer roots are the Integers in this list.
}
\litem{
Perform the division below. Then, find the intervals that correspond to the quotient in the form $ax^2+bx+c$ and remainder $r$.
\[ \frac{4x^{3} +10 x^{2} -18 x -41}{x + 3} \]

The solution is \( 4x^{2} -2 x -12 + \frac{-5}{x + 3} \), which is option B.\begin{enumerate}[label=\Alph*.]
\item \( a \in [-16, -10], \text{   } b \in [43, 47], \text{   } c \in [-158, -153], \text{   and   } r \in [424, 434]. \)

 You multiplied by the synthetic number rather than bringing the first factor down.
\item \( a \in [2, 5], \text{   } b \in [-4, 2], \text{   } c \in [-15, -7], \text{   and   } r \in [-8, -1]. \)

* This is the solution!
\item \( a \in [2, 5], \text{   } b \in [-13, -3], \text{   } c \in [5, 12], \text{   and   } r \in [-68, -64]. \)

 You multiplied by the synthetic number and subtracted rather than adding during synthetic division.
\item \( a \in [-16, -10], \text{   } b \in [-27, -18], \text{   } c \in [-97, -90], \text{   and   } r \in [-331, -326]. \)

 You divided by the opposite of the factor AND multiplied the first factor rather than just bringing it down.
\item \( a \in [2, 5], \text{   } b \in [21, 25], \text{   } c \in [48, 51], \text{   and   } r \in [103, 109]. \)

 You divided by the opposite of the factor.
\end{enumerate}

\textbf{General Comment:} Be sure to synthetically divide by the zero of the denominator!
}
\litem{
Perform the division below. Then, find the intervals that correspond to the quotient in the form $ax^2+bx+c$ and remainder $r$.
\[ \frac{25x^{3} -25 x^{2} -125 x -72}{x -3} \]

The solution is \( 25x^{2} +50 x + 25 + \frac{3}{x -3} \), which is option E.\begin{enumerate}[label=\Alph*.]
\item \( a \in [21, 30], \text{   } b \in [-103, -97], \text{   } c \in [170, 177], \text{   and   } r \in [-597, -595]. \)

 You divided by the opposite of the factor.
\item \( a \in [72, 79], \text{   } b \in [-252, -246], \text{   } c \in [622, 630], \text{   and   } r \in [-1949, -1943]. \)

 You divided by the opposite of the factor AND multiplied the first factor rather than just bringing it down.
\item \( a \in [72, 79], \text{   } b \in [199, 204], \text{   } c \in [470, 477], \text{   and   } r \in [1347, 1356]. \)

 You multiplied by the synthetic number rather than bringing the first factor down.
\item \( a \in [21, 30], \text{   } b \in [19, 30], \text{   } c \in [-76, -74], \text{   and   } r \in [-224, -221]. \)

 You multiplied by the synthetic number and subtracted rather than adding during synthetic division.
\item \( a \in [21, 30], \text{   } b \in [47, 55], \text{   } c \in [24, 27], \text{   and   } r \in [1, 7]. \)

* This is the solution!
\end{enumerate}

\textbf{General Comment:} Be sure to synthetically divide by the zero of the denominator!
}
\litem{
Perform the division below. Then, find the intervals that correspond to the quotient in the form $ax^2+bx+c$ and remainder $r$.
\[ \frac{15x^{3} +65 x^{2} -82}{x + 4} \]

The solution is \( 15x^{2} +5 x -20 + \frac{-2}{x + 4} \), which is option C.\begin{enumerate}[label=\Alph*.]
\item \( a \in [-61, -54], b \in [305, 306], c \in [-1229, -1212], \text{ and } r \in [4794, 4805]. \)

 You multipled by the synthetic number rather than bringing the first factor down.
\item \( a \in [10, 19], b \in [-10, -5], c \in [46, 54], \text{ and } r \in [-334, -325]. \)

 You multipled by the synthetic number and subtracted rather than adding during synthetic division.
\item \( a \in [10, 19], b \in [3, 7], c \in [-22, -17], \text{ and } r \in [-3, -1]. \)

* This is the solution!
\item \( a \in [-61, -54], b \in [-181, -171], c \in [-705, -698], \text{ and } r \in [-2885, -2879]. \)

 You divided by the opposite of the factor AND multipled the first factor rather than just bringing it down.
\item \( a \in [10, 19], b \in [121, 130], c \in [500, 507], \text{ and } r \in [1913, 1921]. \)

 You divided by the opposite of the factor.
\end{enumerate}

\textbf{General Comment:} Be sure to synthetically divide by the zero of the denominator! Also, make sure to include 0 placeholders for missing terms.
}
\litem{
Factor the polynomial below completely, knowing that $x+3$ is a factor. Then, choose the intervals the zeros of the polynomial belong to, where $z_1 \leq z_2 \leq z_3 \leq z_4$. \textit{To make the problem easier, all zeros are between -5 and 5.}
\[ f(x) = 10x^{4} +77 x^{3} +157 x^{2} -144 \]

The solution is \( [-4, -3, -1.5, 0.8] \), which is option A.\begin{enumerate}[label=\Alph*.]
\item \( z_1 \in [-4.47, -3.95], \text{   }  z_2 \in [-3.43, -1.5], z_3 \in [-2.16, -1.33], \text{   and   } z_4 \in [0.55, 0.98] \)

* This is the solution!
\item \( z_1 \in [-0.43, -0.27], \text{   }  z_2 \in [2.29, 3.4], z_3 \in [2.32, 3.36], \text{   and   } z_4 \in [3.5, 4.38] \)

 Distractor 4: Corresponds to moving factors from one rational to another.
\item \( z_1 \in [-1.37, -1.06], \text{   }  z_2 \in [0.46, 0.91], z_3 \in [2.32, 3.36], \text{   and   } z_4 \in [3.5, 4.38] \)

 Distractor 3: Corresponds to negatives of all zeros AND inversing rational roots.
\item \( z_1 \in [-0.85, -0.49], \text{   }  z_2 \in [1.26, 1.77], z_3 \in [2.32, 3.36], \text{   and   } z_4 \in [3.5, 4.38] \)

 Distractor 1: Corresponds to negatives of all zeros.
\item \( z_1 \in [-4.47, -3.95], \text{   }  z_2 \in [-3.43, -1.5], z_3 \in [-1.04, -0.04], \text{   and   } z_4 \in [1.01, 2.3] \)

 Distractor 2: Corresponds to inversing rational roots.
\end{enumerate}

\textbf{General Comment:} Remember to try the middle-most integers first as these normally are the zeros. Also, once you get it to a quadratic, you can use your other factoring techniques to finish factoring.
}
\litem{
Factor the polynomial below completely. Then, choose the intervals the zeros of the polynomial belong to, where $z_1 \leq z_2 \leq z_3$. \textit{To make the problem easier, all zeros are between -5 and 5.}
\[ f(x) = 25x^{3} -95 x^{2} -26 x + 24 \]

The solution is \( [-0.6, 0.4, 4] \), which is option E.\begin{enumerate}[label=\Alph*.]
\item \( z_1 \in [-5.8, -2.8], \text{   }  z_2 \in [-0.61, -0.32], \text{   and   } z_3 \in [0.3, 1.4] \)

 Distractor 1: Corresponds to negatives of all zeros.
\item \( z_1 \in [-5.8, -2.8], \text{   }  z_2 \in [-2.15, -1.87], \text{   and   } z_3 \in [0, 0.5] \)

 Distractor 4: Corresponds to moving factors from one rational to another.
\item \( z_1 \in [-1.9, -0.8], \text{   }  z_2 \in [2.17, 2.85], \text{   and   } z_3 \in [2.9, 5] \)

 Distractor 2: Corresponds to inversing rational roots.
\item \( z_1 \in [-5.8, -2.8], \text{   }  z_2 \in [-2.88, -2.26], \text{   and   } z_3 \in [1.5, 2.3] \)

 Distractor 3: Corresponds to negatives of all zeros AND inversing rational roots.
\item \( z_1 \in [-1, 1], \text{   }  z_2 \in [-0.28, 0.53], \text{   and   } z_3 \in [2.9, 5] \)

* This is the solution!
\end{enumerate}

\textbf{General Comment:} Remember to try the middle-most integers first as these normally are the zeros. Also, once you get it to a quadratic, you can use your other factoring techniques to finish factoring.
}
\litem{
Perform the division below. Then, find the intervals that correspond to the quotient in the form $ax^2+bx+c$ and remainder $r$.
\[ \frac{16x^{3} -84 x^{2} + 102}{x -5} \]

The solution is \( 16x^{2} -4 x -20 + \frac{2}{x -5} \), which is option E.\begin{enumerate}[label=\Alph*.]
\item \( a \in [79, 82], b \in [-485, -478], c \in [2415, 2426], \text{ and } r \in [-11998, -11992]. \)

 You divided by the opposite of the factor AND multipled the first factor rather than just bringing it down.
\item \( a \in [79, 82], b \in [312, 321], c \in [1576, 1582], \text{ and } r \in [8002, 8007]. \)

 You multipled by the synthetic number rather than bringing the first factor down.
\item \( a \in [14, 20], b \in [-22, -19], c \in [-84, -79], \text{ and } r \in [-219, -214]. \)

 You multipled by the synthetic number and subtracted rather than adding during synthetic division.
\item \( a \in [14, 20], b \in [-166, -162], c \in [818, 822], \text{ and } r \in [-4008, -3996]. \)

 You divided by the opposite of the factor.
\item \( a \in [14, 20], b \in [-7, -2], c \in [-23, -17], \text{ and } r \in [-6, 5]. \)

* This is the solution!
\end{enumerate}

\textbf{General Comment:} Be sure to synthetically divide by the zero of the denominator! Also, make sure to include 0 placeholders for missing terms.
}
\litem{
Factor the polynomial below completely, knowing that $x+3$ is a factor. Then, choose the intervals the zeros of the polynomial belong to, where $z_1 \leq z_2 \leq z_3 \leq z_4$. \textit{To make the problem easier, all zeros are between -5 and 5.}
\[ f(x) = 25x^{4} +5 x^{3} -231 x^{2} -45 x + 54 \]

The solution is \( [-3, -0.6, 0.4, 3] \), which is option E.\begin{enumerate}[label=\Alph*.]
\item \( z_1 \in [-4, 1], \text{   }  z_2 \in [-0.51, -0.34], z_3 \in [0.48, 0.82], \text{   and   } z_4 \in [-2, 7] \)

 Distractor 1: Corresponds to negatives of all zeros.
\item \( z_1 \in [-4, 1], \text{   }  z_2 \in [-2.54, -2.36], z_3 \in [1.66, 1.68], \text{   and   } z_4 \in [-2, 7] \)

 Distractor 3: Corresponds to negatives of all zeros AND inversing rational roots.
\item \( z_1 \in [-4, 1], \text{   }  z_2 \in [-0.38, 0.25], z_3 \in [2.84, 3.14], \text{   and   } z_4 \in [-2, 7] \)

 Distractor 4: Corresponds to moving factors from one rational to another.
\item \( z_1 \in [-4, 1], \text{   }  z_2 \in [-1.84, -1.58], z_3 \in [2.45, 2.54], \text{   and   } z_4 \in [-2, 7] \)

 Distractor 2: Corresponds to inversing rational roots.
\item \( z_1 \in [-4, 1], \text{   }  z_2 \in [-0.72, -0.47], z_3 \in [0.36, 0.52], \text{   and   } z_4 \in [-2, 7] \)

* This is the solution!
\end{enumerate}

\textbf{General Comment:} Remember to try the middle-most integers first as these normally are the zeros. Also, once you get it to a quadratic, you can use your other factoring techniques to finish factoring.
}
\end{enumerate}

\end{document}