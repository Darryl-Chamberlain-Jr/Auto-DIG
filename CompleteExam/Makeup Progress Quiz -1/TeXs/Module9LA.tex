\documentclass[14pt]{extbook}
\usepackage{multicol, enumerate, enumitem, hyperref, color, soul, setspace, parskip, fancyhdr} %General Packages
\usepackage{amssymb, amsthm, amsmath, bbm, latexsym, units, mathtools} %Math Packages
\everymath{\displaystyle} %All math in Display Style
% Packages with additional options
\usepackage[headsep=0.5cm,headheight=12pt, left=1 in,right= 1 in,top= 1 in,bottom= 1 in]{geometry}
\usepackage[usenames,dvipsnames]{xcolor}
\usepackage{dashrule}  % Package to use the command below to create lines between items
\newcommand{\litem}[1]{\item#1\hspace*{-1cm}\rule{\textwidth}{0.4pt}}
\pagestyle{fancy}
\lhead{Makeup Progress Quiz -1}
\chead{}
\rhead{Version A}
\lfoot{7547-2949}
\cfoot{}
\rfoot{Fall 2020}
\begin{document}

\begin{enumerate}
\litem{
Find the inverse of the function below. Then, evaluate the inverse at $x = 8$ and choose the interval that $f^{-1}(8)$ belongs to.\[ f(x) = e^{x-4}+2 \]\begin{enumerate}[label=\Alph*.]
\item \( f^{-1}(8) \in [4.33, 4.59] \)
\item \( f^{-1}(8) \in [4.17, 4.34] \)
\item \( f^{-1}(8) \in [5.36, 6] \)
\item \( f^{-1}(8) \in [-2.43, -2.18] \)
\item \( f^{-1}(8) \in [3.28, 3.69] \)

\end{enumerate} }
\litem{
Subtract the following functions, then choose the domain of the resulting function from the list below.\[ f(x) = 6x^{2} +6 x + 5 \text{ and } g(x) = 5x + 1 \]\begin{enumerate}[label=\Alph*.]
\item \( \text{ The domain is all Real numbers less than or equal to } x = a, \text{ where } a \in [0.2, 9.2] \)
\item \( \text{ The domain is all Real numbers except } x = a, \text{ where } a \in [1.75, 7.75] \)
\item \( \text{ The domain is all Real numbers greater than or equal to } x = a, \text{ where } a \in [3.33, 9.33] \)
\item \( \text{ The domain is all Real numbers except } x = a \text{ and } x = b, \text{ where } a \in [2.2, 9.2] \text{ and } b \in [4.25, 6.25] \)
\item \( \text{ The domain is all Real numbers. } \)

\end{enumerate} }
\litem{
Choose the interval below that $f$ composed with $g$ at $x=-1$ is in.\[ f(x) = x^{3} + x^{2} +3 x \text{ and } g(x) = 4x^{3} +3 x^{2} -3 x + 2 \]\begin{enumerate}[label=\Alph*.]
\item \( (f \circ g)(-1) \in [-71, -67] \)
\item \( (f \circ g)(-1) \in [100, 107] \)
\item \( (f \circ g)(-1) \in [-82, -78] \)
\item \( (f \circ g)(-1) \in [91, 95] \)
\item \( \text{It is not possible to compose the two functions.} \)

\end{enumerate} }
\litem{
Find the inverse of the function below (if it exists). Then, evaluate the inverse at $x = 10$ and choose the interval that $f^{-1}(10)$ belongs to.\[ f(x) = 3 x^2 - 4 \]\begin{enumerate}[label=\Alph*.]
\item \( f^{-1}(10) \in [1.92, 3.29] \)
\item \( f^{-1}(10) \in [4.08, 5.39] \)
\item \( f^{-1}(10) \in [0.22, 2.01] \)
\item \( f^{-1}(10) \in [5.04, 6.9] \)
\item \( \text{ The function is not invertible for all Real numbers. } \)

\end{enumerate} }
\litem{
Add the following functions, then choose the domain of the resulting function from the list below.\[ f(x) = \frac{4}{6x+19} \text{ and } g(x) = \frac{4}{5x-26} \]\begin{enumerate}[label=\Alph*.]
\item \( \text{ The domain is all Real numbers less than or equal to } x = a, \text{ where } a \in [1.17, 5.17] \)
\item \( \text{ The domain is all Real numbers greater than or equal to } x = a, \text{ where } a \in [5.25, 10.25] \)
\item \( \text{ The domain is all Real numbers except } x = a, \text{ where } a \in [-7.6, -0.6] \)
\item \( \text{ The domain is all Real numbers except } x = a \text{ and } x = b, \text{ where } a \in [-5.17, 2.83] \text{ and } b \in [1.2, 12.2] \)
\item \( \text{ The domain is all Real numbers. } \)

\end{enumerate} }
\litem{
Determine whether the function below is 1-1.\[ f(x) = \sqrt{-3 x - 8} \]\begin{enumerate}[label=\Alph*.]
\item \( \text{No, because there is a $y$-value that goes to 2 different $x$-values.} \)
\item \( \text{No, because the domain of the function is not $(-\infty, \infty)$.} \)
\item \( \text{Yes, the function is 1-1.} \)
\item \( \text{No, because the range of the function is not $(-\infty, \infty)$.} \)
\item \( \text{No, because there is an $x$-value that goes to 2 different $y$-values.} \)

\end{enumerate} }
\litem{
Choose the interval below that $f$ composed with $g$ at $x=-1$ is in.\[ f(x) = -2x^{3} -4 x^{2} +3 x \text{ and } g(x) = -x^{3} -3 x^{2} +x + 1 \]\begin{enumerate}[label=\Alph*.]
\item \( (f \circ g)(-1) \in [34, 44] \)
\item \( (f \circ g)(-1) \in [3, 5] \)
\item \( (f \circ g)(-1) \in [-6, -3] \)
\item \( (f \circ g)(-1) \in [42, 50] \)
\item \( \text{It is not possible to compose the two functions.} \)

\end{enumerate} }
\litem{
Find the inverse of the function below (if it exists). Then, evaluate the inverse at $x = -15$ and choose the interval that $f^{-1}(-15)$ belongs to.\[ f(x) = 4 x^2 - 5 \]\begin{enumerate}[label=\Alph*.]
\item \( f^{-1}(-15) \in [0.4, 2.05] \)
\item \( f^{-1}(-15) \in [3.75, 4.87] \)
\item \( f^{-1}(-15) \in [2.15, 2.66] \)
\item \( f^{-1}(-15) \in [6.34, 7.06] \)
\item \( \text{ The function is not invertible for all Real numbers. } \)

\end{enumerate} }
\litem{
Determine whether the function below is 1-1.\[ f(x) = \sqrt{3 x + 13} \]\begin{enumerate}[label=\Alph*.]
\item \( \text{No, because there is an $x$-value that goes to 2 different $y$-values.} \)
\item \( \text{No, because there is a $y$-value that goes to 2 different $x$-values.} \)
\item \( \text{No, because the range of the function is not $(-\infty, \infty)$.} \)
\item \( \text{No, because the domain of the function is not $(-\infty, \infty)$.} \)
\item \( \text{Yes, the function is 1-1.} \)

\end{enumerate} }
\litem{
Find the inverse of the function below. Then, evaluate the inverse at $x = 10$ and choose the interval that $f^{-1}(10)$ belongs to.\[ f(x) = e^{x+4}-5 \]\begin{enumerate}[label=\Alph*.]
\item \( f^{-1}(10) \in [-2.43, -2.36] \)
\item \( f^{-1}(10) \in [-3.33, -3.13] \)
\item \( f^{-1}(10) \in [-3.54, -3.21] \)
\item \( f^{-1}(10) \in [6.68, 6.81] \)
\item \( f^{-1}(10) \in [-1.34, -1.16] \)

\end{enumerate} }
\end{enumerate}

\end{document}