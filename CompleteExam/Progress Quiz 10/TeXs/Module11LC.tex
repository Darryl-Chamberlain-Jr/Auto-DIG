\documentclass[14pt]{extbook}
\usepackage{multicol, enumerate, enumitem, hyperref, color, soul, setspace, parskip, fancyhdr} %General Packages
\usepackage{amssymb, amsthm, amsmath, bbm, latexsym, units, mathtools} %Math Packages
\everymath{\displaystyle} %All math in Display Style
% Packages with additional options
\usepackage[headsep=0.5cm,headheight=12pt, left=1 in,right= 1 in,top= 1 in,bottom= 1 in]{geometry}
\usepackage[usenames,dvipsnames]{xcolor}
\usepackage{dashrule}  % Package to use the command below to create lines between items
\newcommand{\litem}[1]{\item#1\hspace*{-1cm}\rule{\textwidth}{0.4pt}}
\pagestyle{fancy}
\lhead{Progress Quiz 10}
\chead{}
\rhead{Version C}
\lfoot{6232-9639}
\cfoot{}
\rfoot{Fall 2020}
\begin{document}

\begin{enumerate}
\litem{
Evaluate the one-sided limit of the function $f(x)$ below, if possible.\[ \lim_{x \rightarrow -9^+} \frac{-2}{(x-9)^9}+1 \]\begin{enumerate}[label=\Alph*.]
\item \( \infty \)
\item \( -\infty \)
\item \( f(-9) \)
\item \( \text{The limit does not exist} \)
\item \( \text{None of the above} \)

\end{enumerate} }
\litem{
To estimate the one-sided limit of the function below as $x$ approaches 4 from the right, which of the following sets of numbers should you use?\[ \frac{\frac{4}{x} - 1}{x - 4} \]\begin{enumerate}[label=\Alph*.]
\item \( \{ 3.9000, 3.9900, 3.9990, 3.9999 \} \)
\item \( \{ 4.0000, 3.9000, 3.9900, 3.9990 \} \)
\item \( \{ 4.0000, 4.1000, 4.0100, 4.0010 \} \)
\item \( \{ 3.9000, 3.9900, 4.0100, 4.1000 \} \)
\item \( \{ 4.1000, 4.0100, 4.0010, 4.0001 \} \)

\end{enumerate} }
\litem{
Evaluate the one-sided limit of the function $f(x)$ below, if possible.\[ \lim_{x \rightarrow 8^-} \frac{-2}{(x-8)^7}+4 \]\begin{enumerate}[label=\Alph*.]
\item \( \infty \)
\item \( -\infty \)
\item \( f(8) \)
\item \( \text{The limit does not exist} \)
\item \( \text{None of the above} \)

\end{enumerate} }
\litem{
For the graph below, find the value(s) $a$ that makes the statement true: $ \displaystyle \lim_{x \rightarrow a} f(x)$ does not exist.
\begin{center}
    \includegraphics[width=0.5\textwidth]{../Figures/evaluateLimitGraphicallyCopyC.png}
\end{center}
\begin{enumerate}[label=\Alph*.]
\item \( 1 \)
\item \( 3 \)
\item \( -2 \)
\item \( \text{Multiple } a \text{ make the statement true}. \)
\item \( \text{No } a \text{ make the statement true}. \)

\end{enumerate} }
\litem{
Evaluate the limit below, if possible.\[ \lim_{x \rightarrow 9} \frac{\sqrt{6x - 18} - 6}{5x - 45} \]\begin{enumerate}[label=\Alph*.]
\item \( 0.490 \)
\item \( \infty \)
\item \( 0.083 \)
\item \( 0.100 \)
\item \( \text{None of the above} \)

\end{enumerate} }
\litem{
For the graph below, evaluate the limit: $ \displaystyle \lim_{x \rightarrow 3} f(x)$.
\begin{center}
    \includegraphics[width=0.5\textwidth]{../Figures/evaluateLimitGraphicallyC.png}
\end{center}
\begin{enumerate}[label=\Alph*.]
\item \( 1 \)
\item \( -\infty \)
\item \( -2 \)
\item \( \text{The limit does not exist} \)
\item \( \text{None of the above} \)

\end{enumerate} }
\litem{
Based on the information below, which of the following statements is always true?
\begin{center}
    \textit{ As $x$ approaches $4$, $f(x)$ approaches $11.333$. }
\end{center}
\begin{enumerate}[label=\Alph*.]
\item \( f(x) = 11.333 \text{ when } x \text{ is close to } 4 \)
\item \( f(x) = 4 \text{ when } x \text{ is close to } 11.333 \)
\item \( f(x) \text{ is close to or exactly } 4 \text{ when } x \text{ is close to } 11.333 \)
\item \( f(x) \text{ is close to or exactly } 11.333 \text{ when } x \text{ is close to } 4 \)
\item \( \text{None of the above are always true.} \)

\end{enumerate} }
\litem{
Evaluate the limit below, if possible.\[ \lim_{x \rightarrow 8} \frac{\sqrt{8x - 48} - 4}{4x - 32} \]\begin{enumerate}[label=\Alph*.]
\item \( \infty \)
\item \( 0.125 \)
\item \( 0.707 \)
\item \( 0.250 \)
\item \( \text{None of the above} \)

\end{enumerate} }
\litem{
Based on the information below, which of the following statements is always true?
\begin{center}
    \textit{ $f(x)$ approaches $\infty$ as $x$ approaches $3$. }
\end{center}
\begin{enumerate}[label=\Alph*.]
\item \( x \text{ is undefined when } f(x) \text{ is close to or exactly } \infty. \)
\item \( f(x) \text{ is close to or exactly } 3 \text{ when } x \text{ is large enough}. \)
\item \( f(x) \text{ is close to or exactly } \infty \text{ when } x \text{ is large enough}. \)
\item \( f(x) \text{ is undefined when } x \text{ is close to or exactly } 3. \)
\item \( \text{None of the above are always true.} \)

\end{enumerate} }
\litem{
To estimate the one-sided limit of the function below as $x$ approaches 9 from the right, which of the following sets of numbers should you use?\[ \frac{\frac{9}{x} - 1}{x - 9} \]\begin{enumerate}[label=\Alph*.]
\item \( \{ 8.9000, 8.9900, 8.9990, 8.9999 \} \)
\item \( \{ 9.0000, 9.1000, 9.0100, 9.0010 \} \)
\item \( \{ 9.0000, 8.9000, 8.9900, 8.9990 \} \)
\item \( \{ 9.1000, 9.0100, 9.0010, 9.0001 \} \)
\item \( \{ 8.9000, 8.9900, 9.0100, 9.1000 \} \)

\end{enumerate} }
\end{enumerate}

\end{document}