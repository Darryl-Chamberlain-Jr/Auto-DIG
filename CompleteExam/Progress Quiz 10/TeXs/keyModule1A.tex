\documentclass{extbook}[14pt]
\usepackage{multicol, enumerate, enumitem, hyperref, color, soul, setspace, parskip, fancyhdr, amssymb, amsthm, amsmath, bbm, latexsym, units, mathtools}
\everymath{\displaystyle}
\usepackage[headsep=0.5cm,headheight=0cm, left=1 in,right= 1 in,top= 1 in,bottom= 1 in]{geometry}
\usepackage{dashrule}  % Package to use the command below to create lines between items
\newcommand{\litem}[1]{\item #1

\rule{\textwidth}{0.4pt}}
\pagestyle{fancy}
\lhead{}
\chead{Answer Key for Progress Quiz 10 Version A}
\rhead{}
\lfoot{6232-9639}
\cfoot{}
\rfoot{Fall 2020}
\begin{document}
\textbf{This key should allow you to understand why you choose the option you did (beyond just getting a question right or wrong). \href{https://xronos.clas.ufl.edu/mac1105spring2020/courseDescriptionAndMisc/Exams/LearningFromResults}{More instructions on how to use this key can be found here}.}

\textbf{If you have a suggestion to make the keys better, \href{https://forms.gle/CZkbZmPbC9XALEE88}{please fill out the short survey here}.}

\textit{Note: This key is auto-generated and may contain issues and/or errors. The keys are reviewed after each exam to ensure grading is done accurately. If there are issues (like duplicate options), they are noted in the offline gradebook. The keys are a work-in-progress to give students as many resources to improve as possible.}

\rule{\textwidth}{0.4pt}

\begin{enumerate}\litem{
Choose the \textbf{smallest} set of Complex numbers that the number below belongs to.
\[ \sqrt{\frac{36}{361}}+\sqrt{70} i \]

The solution is \( \text{Nonreal Complex} \), which is option C.\begin{enumerate}[label=\Alph*.]
\item \( \text{Not a Complex Number} \)

This is not a number. The only non-Complex number we know is dividing by 0 as this is not a number!
\item \( \text{Pure Imaginary} \)

This is a Complex number $(a+bi)$ that \textbf{only} has an imaginary part like $2i$.
\item \( \text{Nonreal Complex} \)

* This is the correct option!
\item \( \text{Irrational} \)

These cannot be written as a fraction of Integers. Remember: $\pi$ is not an Integer!
\item \( \text{Rational} \)

These are numbers that can be written as fraction of Integers (e.g., -2/3 + 5)
\end{enumerate}

\textbf{General Comment:} Be sure to simplify $i^2 = -1$. This may remove the imaginary portion for your number. If you are having trouble, you may want to look at the \textit{Subgroups of the Real Numbers} section.
}
\litem{
Simplify the expression below and choose the interval the simplification is contained within.
\[ 13 - 17 \div 8 * 12 - (10 * 6) \]

The solution is \( -72.500 \), which is option C.\begin{enumerate}[label=\Alph*.]
\item \( [71.82, 75.82] \)

 72.823, which corresponds to not distributing addition and subtraction correctly.
\item \( [-136, -129] \)

 -135.000, which corresponds to not distributing a negative correctly.
\item \( [-75.5, -65.5] \)

* -72.500, which is the correct option.
\item \( [-50.18, -46.18] \)

 -47.177, which corresponds to an Order of Operations error: not reading left-to-right for multiplication/division.
\item \( \text{None of the above} \)

 You may have gotten this by making an unanticipated error. If you got a value that is not any of the others, please let the coordinator know so they can help you figure out what happened.
\end{enumerate}

\textbf{General Comment:} While you may remember (or were taught) PEMDAS is done in order, it is actually done as P/E/MD/AS. When we are at MD or AS, we read left to right.
}
\litem{
Choose the \textbf{smallest} set of Real numbers that the number below belongs to.
\[ \sqrt{\frac{361}{529}} \]

The solution is \( \text{Rational} \), which is option C.\begin{enumerate}[label=\Alph*.]
\item \( \text{Irrational} \)

These cannot be written as a fraction of Integers.
\item \( \text{Not a Real number} \)

These are Nonreal Complex numbers \textbf{OR} things that are not numbers (e.g., dividing by 0).
\item \( \text{Rational} \)

* This is the correct option!
\item \( \text{Integer} \)

These are the negative and positive counting numbers (..., -3, -2, -1, 0, 1, 2, 3, ...)
\item \( \text{Whole} \)

These are the counting numbers with 0 (0, 1, 2, 3, ...)
\end{enumerate}

\textbf{General Comment:} First, you \textbf{NEED} to simplify the expression. This question simplifies to $\frac{19}{23}$. 
 
 Be sure you look at the simplified fraction and not just the decimal expansion. Numbers such as 13, 17, and 19 provide \textbf{long but repeating/terminating decimal expansions!} 
 
 The only ways to *not* be a Real number are: dividing by 0 or taking the square root of a negative number. 
 
 Irrational numbers are more than just square root of 3: adding or subtracting values from square root of 3 is also irrational.
}
\litem{
Simplify the expression below into the form $a+bi$. Then, choose the intervals that $a$ and $b$ belong to.
\[ (8 - 7 i)(-10 - 4 i) \]

The solution is \( -108 + 38 i \), which is option E.\begin{enumerate}[label=\Alph*.]
\item \( a \in [-57, -45] \text{ and } b \in [-104, -100] \)

 $-52 - 102 i$, which corresponds to adding a minus sign in the first term.
\item \( a \in [-109, -107] \text{ and } b \in [-40, -36] \)

 $-108 - 38 i$, which corresponds to adding a minus sign in both terms.
\item \( a \in [-57, -45] \text{ and } b \in [97, 107] \)

 $-52 + 102 i$, which corresponds to adding a minus sign in the second term.
\item \( a \in [-80, -75] \text{ and } b \in [26, 34] \)

 $-80 + 28 i$, which corresponds to just multiplying the real terms to get the real part of the solution and the coefficients in the complex terms to get the complex part.
\item \( a \in [-109, -107] \text{ and } b \in [33, 46] \)

* $-108 + 38 i$, which is the correct option.
\end{enumerate}

\textbf{General Comment:} You can treat $i$ as a variable and distribute. Just remember that $i^2=-1$, so you can continue to reduce after you distribute.
}
\litem{
Choose the \textbf{smallest} set of Complex numbers that the number below belongs to.
\[ \frac{\sqrt{110}}{18}+8i^2 \]

The solution is \( \text{Irrational} \), which is option A.\begin{enumerate}[label=\Alph*.]
\item \( \text{Irrational} \)

* This is the correct option!
\item \( \text{Rational} \)

These are numbers that can be written as fraction of Integers (e.g., -2/3 + 5)
\item \( \text{Pure Imaginary} \)

This is a Complex number $(a+bi)$ that \textbf{only} has an imaginary part like $2i$.
\item \( \text{Not a Complex Number} \)

This is not a number. The only non-Complex number we know is dividing by 0 as this is not a number!
\item \( \text{Nonreal Complex} \)

This is a Complex number $(a+bi)$ that is not Real (has $i$ as part of the number).
\end{enumerate}

\textbf{General Comment:} Be sure to simplify $i^2 = -1$. This may remove the imaginary portion for your number. If you are having trouble, you may want to look at the \textit{Subgroups of the Real Numbers} section.
}
\litem{
Simplify the expression below into the form $a+bi$. Then, choose the intervals that $a$ and $b$ belong to.
\[ (6 + 3 i)(-9 + 4 i) \]

The solution is \( -66 - 3 i \), which is option D.\begin{enumerate}[label=\Alph*.]
\item \( a \in [-44, -34] \text{ and } b \in [-51, -44] \)

 $-42 - 51 i$, which corresponds to adding a minus sign in the second term.
\item \( a \in [-57, -50] \text{ and } b \in [9, 14] \)

 $-54 + 12 i$, which corresponds to just multiplying the real terms to get the real part of the solution and the coefficients in the complex terms to get the complex part.
\item \( a \in [-44, -34] \text{ and } b \in [45, 56] \)

 $-42 + 51 i$, which corresponds to adding a minus sign in the first term.
\item \( a \in [-68, -65] \text{ and } b \in [-4, 0] \)

* $-66 - 3 i$, which is the correct option.
\item \( a \in [-68, -65] \text{ and } b \in [0, 8] \)

 $-66 + 3 i$, which corresponds to adding a minus sign in both terms.
\end{enumerate}

\textbf{General Comment:} You can treat $i$ as a variable and distribute. Just remember that $i^2=-1$, so you can continue to reduce after you distribute.
}
\litem{
Choose the \textbf{smallest} set of Real numbers that the number below belongs to.
\[ -\sqrt{\frac{1386}{14}} \]

The solution is \( \text{Irrational} \), which is option C.\begin{enumerate}[label=\Alph*.]
\item \( \text{Not a Real number} \)

These are Nonreal Complex numbers \textbf{OR} things that are not numbers (e.g., dividing by 0).
\item \( \text{Rational} \)

These are numbers that can be written as fraction of Integers (e.g., -2/3)
\item \( \text{Irrational} \)

* This is the correct option!
\item \( \text{Whole} \)

These are the counting numbers with 0 (0, 1, 2, 3, ...)
\item \( \text{Integer} \)

These are the negative and positive counting numbers (..., -3, -2, -1, 0, 1, 2, 3, ...)
\end{enumerate}

\textbf{General Comment:} First, you \textbf{NEED} to simplify the expression. This question simplifies to $-\sqrt{99}$. 
 
 Be sure you look at the simplified fraction and not just the decimal expansion. Numbers such as 13, 17, and 19 provide \textbf{long but repeating/terminating decimal expansions!} 
 
 The only ways to *not* be a Real number are: dividing by 0 or taking the square root of a negative number. 
 
 Irrational numbers are more than just square root of 3: adding or subtracting values from square root of 3 is also irrational.
}
\litem{
Simplify the expression below into the form $a+bi$. Then, choose the intervals that $a$ and $b$ belong to.
\[ \frac{-18 - 88 i}{-4 - 5 i} \]

The solution is \( 12.49  + 6.39 i \), which is option B.\begin{enumerate}[label=\Alph*.]
\item \( a \in [4, 6] \text{ and } b \in [17, 18.5] \)

 $4.50  + 17.60 i$, which corresponds to just dividing the first term by the first term and the second by the second.
\item \( a \in [12, 13.5] \text{ and } b \in [5, 7.5] \)

* $12.49  + 6.39 i$, which is the correct option.
\item \( a \in [510.5, 512.5] \text{ and } b \in [5, 7.5] \)

 $512.00  + 6.39 i$, which corresponds to forgetting to multiply the conjugate by the numerator and using a plus instead of a minus in the denominator.
\item \( a \in [-9.5, -8.5] \text{ and } b \in [10, 13] \)

 $-8.98  + 10.78 i$, which corresponds to forgetting to multiply the conjugate by the numerator and not computing the conjugate correctly.
\item \( a \in [12, 13.5] \text{ and } b \in [261.5, 263] \)

 $12.49  + 262.00 i$, which corresponds to forgetting to multiply the conjugate by the numerator.
\end{enumerate}

\textbf{General Comment:} Multiply the numerator and denominator by the *conjugate* of the denominator, then simplify. For example, if we have $2+3i$, the conjugate is $2-3i$.
}
\litem{
Simplify the expression below and choose the interval the simplification is contained within.
\[ 6 - 13^2 + 3 \div 15 * 18 \div 5 \]

The solution is \( -162.280 \), which is option A.\begin{enumerate}[label=\Alph*.]
\item \( [-162.9, -161.7] \)

* -162.280, this is the correct option
\item \( [173.91, 175.11] \)

 175.002, which corresponds to two Order of Operations errors.
\item \( [-163.46, -162.96] \)

 -162.998, which corresponds to an Order of Operations error: not reading left-to-right for multiplication/division.
\item \( [175.1, 176.25] \)

 175.720, which corresponds to an Order of Operations error: multiplying by negative before squaring. For example: $(-3)^2 \neq -3^2$
\item \( \text{None of the above} \)

 You may have gotten this by making an unanticipated error. If you got a value that is not any of the others, please let the coordinator know so they can help you figure out what happened.
\end{enumerate}

\textbf{General Comment:} While you may remember (or were taught) PEMDAS is done in order, it is actually done as P/E/MD/AS. When we are at MD or AS, we read left to right.
}
\litem{
Simplify the expression below into the form $a+bi$. Then, choose the intervals that $a$ and $b$ belong to.
\[ \frac{-9 + 88 i}{-7 - 6 i} \]

The solution is \( -5.47  - 7.88 i \), which is option C.\begin{enumerate}[label=\Alph*.]
\item \( a \in [1, 2] \text{ and } b \in [-16, -14] \)

 $1.29  - 14.67 i$, which corresponds to just dividing the first term by the first term and the second by the second.
\item \( a \in [-465.5, -464.5] \text{ and } b \in [-8.5, -7.5] \)

 $-465.00  - 7.88 i$, which corresponds to forgetting to multiply the conjugate by the numerator and using a plus instead of a minus in the denominator.
\item \( a \in [-7, -4.5] \text{ and } b \in [-8.5, -7.5] \)

* $-5.47  - 7.88 i$, which is the correct option.
\item \( a \in [-7, -4.5] \text{ and } b \in [-670.5, -669.5] \)

 $-5.47  - 670.00 i$, which corresponds to forgetting to multiply the conjugate by the numerator.
\item \( a \in [6, 8] \text{ and } b \in [-7, -6] \)

 $6.95  - 6.61 i$, which corresponds to forgetting to multiply the conjugate by the numerator and not computing the conjugate correctly.
\end{enumerate}

\textbf{General Comment:} Multiply the numerator and denominator by the *conjugate* of the denominator, then simplify. For example, if we have $2+3i$, the conjugate is $2-3i$.
}
\end{enumerate}

\end{document}