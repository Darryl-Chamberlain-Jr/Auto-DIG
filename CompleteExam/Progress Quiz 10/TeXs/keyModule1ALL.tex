\documentclass{extbook}[14pt]
\usepackage{multicol, enumerate, enumitem, hyperref, color, soul, setspace, parskip, fancyhdr, amssymb, amsthm, amsmath, latexsym, units, mathtools}
\everymath{\displaystyle}
\usepackage[headsep=0.5cm,headheight=0cm, left=1 in,right= 1 in,top= 1 in,bottom= 1 in]{geometry}
\usepackage{dashrule}  % Package to use the command below to create lines between items
\newcommand{\litem}[1]{\item #1

\rule{\textwidth}{0.4pt}}
\pagestyle{fancy}
\lhead{}
\chead{Answer Key for Progress Quiz 10 Version ALL}
\rhead{}
\lfoot{5170-5105}
\cfoot{}
\rfoot{Summer C 2021}
\begin{document}
\textbf{This key should allow you to understand why you choose the option you did (beyond just getting a question right or wrong). \href{https://xronos.clas.ufl.edu/mac1105spring2020/courseDescriptionAndMisc/Exams/LearningFromResults}{More instructions on how to use this key can be found here}.}

\textbf{If you have a suggestion to make the keys better, \href{https://forms.gle/CZkbZmPbC9XALEE88}{please fill out the short survey here}.}

\textit{Note: This key is auto-generated and may contain issues and/or errors. The keys are reviewed after each exam to ensure grading is done accurately. If there are issues (like duplicate options), they are noted in the offline gradebook. The keys are a work-in-progress to give students as many resources to improve as possible.}

\rule{\textwidth}{0.4pt}

\begin{enumerate}\litem{
Choose the \textbf{smallest} set of Complex numbers that the number below belongs to.
\[ \sqrt{\frac{169}{0}}+\sqrt{60} i \]The solution is \( \text{Not a Complex Number} \), which is option B.\begin{enumerate}[label=\Alph*.]
\item \( \text{Irrational} \)

These cannot be written as a fraction of Integers. Remember: $\pi$ is not an Integer!
\item \( \text{Not a Complex Number} \)

* This is the correct option!
\item \( \text{Pure Imaginary} \)

This is a Complex number $(a+bi)$ that \textbf{only} has an imaginary part like $2i$.
\item \( \text{Rational} \)

These are numbers that can be written as fraction of Integers (e.g., -2/3 + 5)
\item \( \text{Nonreal Complex} \)

This is a Complex number $(a+bi)$ that is not Real (has $i$ as part of the number).
\end{enumerate}

\textbf{General Comment:} Be sure to simplify $i^2 = -1$. This may remove the imaginary portion for your number. If you are having trouble, you may want to look at the \textit{Subgroups of the Real Numbers} section.
}
\litem{
Simplify the expression below and choose the interval the simplification is contained within.
\[ 8 - 1 \div 20 * 5 - (4 * 13) \]The solution is \( -44.250 \), which is option D.\begin{enumerate}[label=\Alph*.]
\item \( [59.76, 60.65] \)

 59.990, which corresponds to not distributing addition and subtraction correctly.
\item \( [48.71, 49.16] \)

 48.750, which corresponds to not distributing a negative correctly.
\item \( [-44.2, -43.15] \)

 -44.010, which corresponds to an Order of Operations error: not reading left-to-right for multiplication/division.
\item \( [-44.61, -44.11] \)

* -44.250, which is the correct option.
\item \( \text{None of the above} \)

 You may have gotten this by making an unanticipated error. If you got a value that is not any of the others, please let the coordinator know so they can help you figure out what happened.
\end{enumerate}

\textbf{General Comment:} While you may remember (or were taught) PEMDAS is done in order, it is actually done as P/E/MD/AS. When we are at MD or AS, we read left to right.
}
\litem{
Simplify the expression below into the form $a+bi$. Then, choose the intervals that $a$ and $b$ belong to.
\[ (-6 - 2 i)(8 - 9 i) \]The solution is \( -66 + 38 i \), which is option A.\begin{enumerate}[label=\Alph*.]
\item \( a \in [-70, -63] \text{ and } b \in [38, 47] \)

* $-66 + 38 i$, which is the correct option.
\item \( a \in [-51, -47] \text{ and } b \in [17, 20] \)

 $-48 + 18 i$, which corresponds to just multiplying the real terms to get the real part of the solution and the coefficients in the complex terms to get the complex part.
\item \( a \in [-33, -26] \text{ and } b \in [-77, -66] \)

 $-30 - 70 i$, which corresponds to adding a minus sign in the second term.
\item \( a \in [-70, -63] \text{ and } b \in [-43, -36] \)

 $-66 - 38 i$, which corresponds to adding a minus sign in both terms.
\item \( a \in [-33, -26] \text{ and } b \in [68, 77] \)

 $-30 + 70 i$, which corresponds to adding a minus sign in the first term.
\end{enumerate}

\textbf{General Comment:} You can treat $i$ as a variable and distribute. Just remember that $i^2=-1$, so you can continue to reduce after you distribute.
}
\litem{
Simplify the expression below and choose the interval the simplification is contained within.
\[ 15 - 11^2 + 3 \div 10 * 7 \div 12 \]The solution is \( -105.825 \), which is option B.\begin{enumerate}[label=\Alph*.]
\item \( [135.9, 136.06] \)

 136.004, which corresponds to two Order of Operations errors.
\item \( [-105.89, -105.76] \)

* -105.825, this is the correct option
\item \( [-106.13, -105.98] \)

 -105.996, which corresponds to an Order of Operations error: not reading left-to-right for multiplication/division.
\item \( [136.12, 136.33] \)

 136.175, which corresponds to an Order of Operations error: multiplying by negative before squaring. For example: $(-3)^2 \neq -3^2$
\item \( \text{None of the above} \)

 You may have gotten this by making an unanticipated error. If you got a value that is not any of the others, please let the coordinator know so they can help you figure out what happened.
\end{enumerate}

\textbf{General Comment:} While you may remember (or were taught) PEMDAS is done in order, it is actually done as P/E/MD/AS. When we are at MD or AS, we read left to right.
}
\litem{
Choose the \textbf{smallest} set of Complex numbers that the number below belongs to.
\[ \sqrt{\frac{0}{484}}+\sqrt{10}i \]The solution is \( \text{Pure Imaginary} \), which is option E.\begin{enumerate}[label=\Alph*.]
\item \( \text{Rational} \)

These are numbers that can be written as fraction of Integers (e.g., -2/3 + 5)
\item \( \text{Not a Complex Number} \)

This is not a number. The only non-Complex number we know is dividing by 0 as this is not a number!
\item \( \text{Nonreal Complex} \)

This is a Complex number $(a+bi)$ that is not Real (has $i$ as part of the number).
\item \( \text{Irrational} \)

These cannot be written as a fraction of Integers. Remember: $\pi$ is not an Integer!
\item \( \text{Pure Imaginary} \)

* This is the correct option!
\end{enumerate}

\textbf{General Comment:} Be sure to simplify $i^2 = -1$. This may remove the imaginary portion for your number. If you are having trouble, you may want to look at the \textit{Subgroups of the Real Numbers} section.
}
\litem{
Simplify the expression below into the form $a+bi$. Then, choose the intervals that $a$ and $b$ belong to.
\[ \frac{63 + 33 i}{8 - 5 i} \]The solution is \( 3.81  + 6.51 i \), which is option B.\begin{enumerate}[label=\Alph*.]
\item \( a \in [7.45, 7.55] \text{ and } b \in [-1, 0] \)

 $7.52  - 0.57 i$, which corresponds to forgetting to multiply the conjugate by the numerator and not computing the conjugate correctly.
\item \( a \in [3.7, 4.4] \text{ and } b \in [5.5, 7.5] \)

* $3.81  + 6.51 i$, which is the correct option.
\item \( a \in [3.7, 4.4] \text{ and } b \in [578.5, 580.5] \)

 $3.81  + 579.00 i$, which corresponds to forgetting to multiply the conjugate by the numerator.
\item \( a \in [7.85, 7.95] \text{ and } b \in [-7, -6] \)

 $7.88  - 6.60 i$, which corresponds to just dividing the first term by the first term and the second by the second.
\item \( a \in [338.9, 339.05] \text{ and } b \in [5.5, 7.5] \)

 $339.00  + 6.51 i$, which corresponds to forgetting to multiply the conjugate by the numerator and using a plus instead of a minus in the denominator.
\end{enumerate}

\textbf{General Comment:} Multiply the numerator and denominator by the *conjugate* of the denominator, then simplify. For example, if we have $2+3i$, the conjugate is $2-3i$.
}
\litem{
Simplify the expression below into the form $a+bi$. Then, choose the intervals that $a$ and $b$ belong to.
\[ (2 - 5 i)(-9 + 3 i) \]The solution is \( -3 + 51 i \), which is option D.\begin{enumerate}[label=\Alph*.]
\item \( a \in [-5, 4] \text{ and } b \in [-56, -44] \)

 $-3 - 51 i$, which corresponds to adding a minus sign in both terms.
\item \( a \in [-22, -15] \text{ and } b \in [-17, -13] \)

 $-18 - 15 i$, which corresponds to just multiplying the real terms to get the real part of the solution and the coefficients in the complex terms to get the complex part.
\item \( a \in [-35, -31] \text{ and } b \in [38, 47] \)

 $-33 + 39 i$, which corresponds to adding a minus sign in the second term.
\item \( a \in [-5, 4] \text{ and } b \in [49, 52] \)

* $-3 + 51 i$, which is the correct option.
\item \( a \in [-35, -31] \text{ and } b \in [-46, -31] \)

 $-33 - 39 i$, which corresponds to adding a minus sign in the first term.
\end{enumerate}

\textbf{General Comment:} You can treat $i$ as a variable and distribute. Just remember that $i^2=-1$, so you can continue to reduce after you distribute.
}
\litem{
Choose the \textbf{smallest} set of Real numbers that the number below belongs to.
\[ -\sqrt{\frac{15}{0}} \]The solution is \( \text{Not a Real number} \), which is option A.\begin{enumerate}[label=\Alph*.]
\item \( \text{Not a Real number} \)

* This is the correct option!
\item \( \text{Irrational} \)

These cannot be written as a fraction of Integers.
\item \( \text{Rational} \)

These are numbers that can be written as fraction of Integers (e.g., -2/3)
\item \( \text{Integer} \)

These are the negative and positive counting numbers (..., -3, -2, -1, 0, 1, 2, 3, ...)
\item \( \text{Whole} \)

These are the counting numbers with 0 (0, 1, 2, 3, ...)
\end{enumerate}

\textbf{General Comment:} First, you \textbf{NEED} to simplify the expression. This question simplifies to $-\sqrt{\frac{15}{0}}$. 
 
 Be sure you look at the simplified fraction and not just the decimal expansion. Numbers such as 13, 17, and 19 provide \textbf{long but repeating/terminating decimal expansions!} 
 
 The only ways to *not* be a Real number are: dividing by 0 or taking the square root of a negative number. 
 
 Irrational numbers are more than just square root of 3: adding or subtracting values from square root of 3 is also irrational.
}
\litem{
Simplify the expression below into the form $a+bi$. Then, choose the intervals that $a$ and $b$ belong to.
\[ \frac{-54 + 88 i}{7 + 5 i} \]The solution is \( 0.84  + 11.97 i \), which is option E.\begin{enumerate}[label=\Alph*.]
\item \( a \in [0, 1] \text{ and } b \in [884.5, 886.5] \)

 $0.84  + 886.00 i$, which corresponds to forgetting to multiply the conjugate by the numerator.
\item \( a \in [-9, -7] \text{ and } b \in [16.5, 18] \)

 $-7.71  + 17.60 i$, which corresponds to just dividing the first term by the first term and the second by the second.
\item \( a \in [-12, -10.5] \text{ and } b \in [4, 6] \)

 $-11.05  + 4.68 i$, which corresponds to forgetting to multiply the conjugate by the numerator and not computing the conjugate correctly.
\item \( a \in [60.5, 63] \text{ and } b \in [11, 13] \)

 $62.00  + 11.97 i$, which corresponds to forgetting to multiply the conjugate by the numerator and using a plus instead of a minus in the denominator.
\item \( a \in [0, 1] \text{ and } b \in [11, 13] \)

* $0.84  + 11.97 i$, which is the correct option.
\end{enumerate}

\textbf{General Comment:} Multiply the numerator and denominator by the *conjugate* of the denominator, then simplify. For example, if we have $2+3i$, the conjugate is $2-3i$.
}
\litem{
Choose the \textbf{smallest} set of Real numbers that the number below belongs to.
\[ \sqrt{\frac{1760}{10}} \]The solution is \( \text{Irrational} \), which is option C.\begin{enumerate}[label=\Alph*.]
\item \( \text{Integer} \)

These are the negative and positive counting numbers (..., -3, -2, -1, 0, 1, 2, 3, ...)
\item \( \text{Not a Real number} \)

These are Nonreal Complex numbers \textbf{OR} things that are not numbers (e.g., dividing by 0).
\item \( \text{Irrational} \)

* This is the correct option!
\item \( \text{Whole} \)

These are the counting numbers with 0 (0, 1, 2, 3, ...)
\item \( \text{Rational} \)

These are numbers that can be written as fraction of Integers (e.g., -2/3)
\end{enumerate}

\textbf{General Comment:} First, you \textbf{NEED} to simplify the expression. This question simplifies to $\sqrt{176}$. 
 
 Be sure you look at the simplified fraction and not just the decimal expansion. Numbers such as 13, 17, and 19 provide \textbf{long but repeating/terminating decimal expansions!} 
 
 The only ways to *not* be a Real number are: dividing by 0 or taking the square root of a negative number. 
 
 Irrational numbers are more than just square root of 3: adding or subtracting values from square root of 3 is also irrational.
}
\litem{
Choose the \textbf{smallest} set of Complex numbers that the number below belongs to.
\[ \sqrt{\frac{-2380}{0}}+\sqrt{63} \]The solution is \( \text{Not a Complex Number} \), which is option B.\begin{enumerate}[label=\Alph*.]
\item \( \text{Nonreal Complex} \)

This is a Complex number $(a+bi)$ that is not Real (has $i$ as part of the number).
\item \( \text{Not a Complex Number} \)

* This is the correct option!
\item \( \text{Rational} \)

These are numbers that can be written as fraction of Integers (e.g., -2/3 + 5)
\item \( \text{Pure Imaginary} \)

This is a Complex number $(a+bi)$ that \textbf{only} has an imaginary part like $2i$.
\item \( \text{Irrational} \)

These cannot be written as a fraction of Integers. Remember: $\pi$ is not an Integer!
\end{enumerate}

\textbf{General Comment:} Be sure to simplify $i^2 = -1$. This may remove the imaginary portion for your number. If you are having trouble, you may want to look at the \textit{Subgroups of the Real Numbers} section.
}
\litem{
Simplify the expression below and choose the interval the simplification is contained within.
\[ 1 - 2 \div 9 * 20 - (5 * 15) \]The solution is \( -78.444 \), which is option D.\begin{enumerate}[label=\Alph*.]
\item \( [-75.01, -72.01] \)

 -74.011, which corresponds to an Order of Operations error: not reading left-to-right for multiplication/division.
\item \( [-126.67, -123.67] \)

 -126.667, which corresponds to not distributing a negative correctly.
\item \( [71.99, 82.99] \)

 75.989, which corresponds to not distributing addition and subtraction correctly.
\item \( [-83.44, -77.44] \)

* -78.444, which is the correct option.
\item \( \text{None of the above} \)

 You may have gotten this by making an unanticipated error. If you got a value that is not any of the others, please let the coordinator know so they can help you figure out what happened.
\end{enumerate}

\textbf{General Comment:} While you may remember (or were taught) PEMDAS is done in order, it is actually done as P/E/MD/AS. When we are at MD or AS, we read left to right.
}
\litem{
Simplify the expression below into the form $a+bi$. Then, choose the intervals that $a$ and $b$ belong to.
\[ (5 - 2 i)(8 + 10 i) \]The solution is \( 60 + 34 i \), which is option B.\begin{enumerate}[label=\Alph*.]
\item \( a \in [59, 63] \text{ and } b \in [-40, -32] \)

 $60 - 34 i$, which corresponds to adding a minus sign in both terms.
\item \( a \in [59, 63] \text{ and } b \in [31, 37] \)

* $60 + 34 i$, which is the correct option.
\item \( a \in [15, 22] \text{ and } b \in [65, 71] \)

 $20 + 66 i$, which corresponds to adding a minus sign in the first term.
\item \( a \in [39, 44] \text{ and } b \in [-21, -19] \)

 $40 - 20 i$, which corresponds to just multiplying the real terms to get the real part of the solution and the coefficients in the complex terms to get the complex part.
\item \( a \in [15, 22] \text{ and } b \in [-70, -57] \)

 $20 - 66 i$, which corresponds to adding a minus sign in the second term.
\end{enumerate}

\textbf{General Comment:} You can treat $i$ as a variable and distribute. Just remember that $i^2=-1$, so you can continue to reduce after you distribute.
}
\litem{
Simplify the expression below and choose the interval the simplification is contained within.
\[ 6 - 18^2 + 17 \div 11 * 5 \div 10 \]The solution is \( -317.227 \), which is option D.\begin{enumerate}[label=\Alph*.]
\item \( [-318.23, -317.26] \)

 -317.969, which corresponds to an Order of Operations error: not reading left-to-right for multiplication/division.
\item \( [329.64, 330.72] \)

 330.031, which corresponds to two Order of Operations errors.
\item \( [330.23, 331.02] \)

 330.773, which corresponds to an Order of Operations error: multiplying by negative before squaring. For example: $(-3)^2 \neq -3^2$
\item \( [-317.26, -317.09] \)

* -317.227, this is the correct option
\item \( \text{None of the above} \)

 You may have gotten this by making an unanticipated error. If you got a value that is not any of the others, please let the coordinator know so they can help you figure out what happened.
\end{enumerate}

\textbf{General Comment:} While you may remember (or were taught) PEMDAS is done in order, it is actually done as P/E/MD/AS. When we are at MD or AS, we read left to right.
}
\litem{
Choose the \textbf{smallest} set of Complex numbers that the number below belongs to.
\[ \sqrt{\frac{1560}{8}}+7i^2 \]The solution is \( \text{Irrational} \), which is option E.\begin{enumerate}[label=\Alph*.]
\item \( \text{Rational} \)

These are numbers that can be written as fraction of Integers (e.g., -2/3 + 5)
\item \( \text{Nonreal Complex} \)

This is a Complex number $(a+bi)$ that is not Real (has $i$ as part of the number).
\item \( \text{Pure Imaginary} \)

This is a Complex number $(a+bi)$ that \textbf{only} has an imaginary part like $2i$.
\item \( \text{Not a Complex Number} \)

This is not a number. The only non-Complex number we know is dividing by 0 as this is not a number!
\item \( \text{Irrational} \)

* This is the correct option!
\end{enumerate}

\textbf{General Comment:} Be sure to simplify $i^2 = -1$. This may remove the imaginary portion for your number. If you are having trouble, you may want to look at the \textit{Subgroups of the Real Numbers} section.
}
\litem{
Simplify the expression below into the form $a+bi$. Then, choose the intervals that $a$ and $b$ belong to.
\[ \frac{63 + 44 i}{6 + 5 i} \]The solution is \( 9.80  - 0.84 i \), which is option C.\begin{enumerate}[label=\Alph*.]
\item \( a \in [1.9, 3.1] \text{ and } b \in [9, 11] \)

 $2.59  + 9.49 i$, which corresponds to forgetting to multiply the conjugate by the numerator and not computing the conjugate correctly.
\item \( a \in [9.6, 9.85] \text{ and } b \in [-51.5, -50] \)

 $9.80  - 51.00 i$, which corresponds to forgetting to multiply the conjugate by the numerator.
\item \( a \in [9.6, 9.85] \text{ and } b \in [-2.5, 0] \)

* $9.80  - 0.84 i$, which is the correct option.
\item \( a \in [597.5, 598.55] \text{ and } b \in [-2.5, 0] \)

 $598.00  - 0.84 i$, which corresponds to forgetting to multiply the conjugate by the numerator and using a plus instead of a minus in the denominator.
\item \( a \in [10.25, 10.9] \text{ and } b \in [8, 9] \)

 $10.50  + 8.80 i$, which corresponds to just dividing the first term by the first term and the second by the second.
\end{enumerate}

\textbf{General Comment:} Multiply the numerator and denominator by the *conjugate* of the denominator, then simplify. For example, if we have $2+3i$, the conjugate is $2-3i$.
}
\litem{
Simplify the expression below into the form $a+bi$. Then, choose the intervals that $a$ and $b$ belong to.
\[ (-3 - 10 i)(8 + 5 i) \]The solution is \( 26 - 95 i \), which is option B.\begin{enumerate}[label=\Alph*.]
\item \( a \in [23, 31] \text{ and } b \in [87, 103] \)

 $26 + 95 i$, which corresponds to adding a minus sign in both terms.
\item \( a \in [23, 31] \text{ and } b \in [-99, -90] \)

* $26 - 95 i$, which is the correct option.
\item \( a \in [-74, -70] \text{ and } b \in [-69, -60] \)

 $-74 - 65 i$, which corresponds to adding a minus sign in the second term.
\item \( a \in [-26, -16] \text{ and } b \in [-50, -48] \)

 $-24 - 50 i$, which corresponds to just multiplying the real terms to get the real part of the solution and the coefficients in the complex terms to get the complex part.
\item \( a \in [-74, -70] \text{ and } b \in [64, 66] \)

 $-74 + 65 i$, which corresponds to adding a minus sign in the first term.
\end{enumerate}

\textbf{General Comment:} You can treat $i$ as a variable and distribute. Just remember that $i^2=-1$, so you can continue to reduce after you distribute.
}
\litem{
Choose the \textbf{smallest} set of Real numbers that the number below belongs to.
\[ -\sqrt{\frac{8100}{25}} \]The solution is \( \text{Integer} \), which is option E.\begin{enumerate}[label=\Alph*.]
\item \( \text{Rational} \)

These are numbers that can be written as fraction of Integers (e.g., -2/3)
\item \( \text{Irrational} \)

These cannot be written as a fraction of Integers.
\item \( \text{Not a Real number} \)

These are Nonreal Complex numbers \textbf{OR} things that are not numbers (e.g., dividing by 0).
\item \( \text{Whole} \)

These are the counting numbers with 0 (0, 1, 2, 3, ...)
\item \( \text{Integer} \)

* This is the correct option!
\end{enumerate}

\textbf{General Comment:} First, you \textbf{NEED} to simplify the expression. This question simplifies to $-90$. 
 
 Be sure you look at the simplified fraction and not just the decimal expansion. Numbers such as 13, 17, and 19 provide \textbf{long but repeating/terminating decimal expansions!} 
 
 The only ways to *not* be a Real number are: dividing by 0 or taking the square root of a negative number. 
 
 Irrational numbers are more than just square root of 3: adding or subtracting values from square root of 3 is also irrational.
}
\litem{
Simplify the expression below into the form $a+bi$. Then, choose the intervals that $a$ and $b$ belong to.
\[ \frac{27 + 44 i}{-2 - i} \]The solution is \( -19.60  - 12.20 i \), which is option B.\begin{enumerate}[label=\Alph*.]
\item \( a \in [-15, -13] \text{ and } b \in [-44.5, -43.5] \)

 $-13.50  - 44.00 i$, which corresponds to just dividing the first term by the first term and the second by the second.
\item \( a \in [-20.5, -17.5] \text{ and } b \in [-13, -11.5] \)

* $-19.60  - 12.20 i$, which is the correct option.
\item \( a \in [-99, -97.5] \text{ and } b \in [-13, -11.5] \)

 $-98.00  - 12.20 i$, which corresponds to forgetting to multiply the conjugate by the numerator and using a plus instead of a minus in the denominator.
\item \( a \in [-20.5, -17.5] \text{ and } b \in [-62.5, -60] \)

 $-19.60  - 61.00 i$, which corresponds to forgetting to multiply the conjugate by the numerator.
\item \( a \in [-2.5, -1.5] \text{ and } b \in [-23.5, -22] \)

 $-2.00  - 23.00 i$, which corresponds to forgetting to multiply the conjugate by the numerator and not computing the conjugate correctly.
\end{enumerate}

\textbf{General Comment:} Multiply the numerator and denominator by the *conjugate* of the denominator, then simplify. For example, if we have $2+3i$, the conjugate is $2-3i$.
}
\litem{
Choose the \textbf{smallest} set of Real numbers that the number below belongs to.
\[ \sqrt{\frac{40000}{100}} \]The solution is \( \text{Whole} \), which is option E.\begin{enumerate}[label=\Alph*.]
\item \( \text{Rational} \)

These are numbers that can be written as fraction of Integers (e.g., -2/3)
\item \( \text{Irrational} \)

These cannot be written as a fraction of Integers.
\item \( \text{Integer} \)

These are the negative and positive counting numbers (..., -3, -2, -1, 0, 1, 2, 3, ...)
\item \( \text{Not a Real number} \)

These are Nonreal Complex numbers \textbf{OR} things that are not numbers (e.g., dividing by 0).
\item \( \text{Whole} \)

* This is the correct option!
\end{enumerate}

\textbf{General Comment:} First, you \textbf{NEED} to simplify the expression. This question simplifies to $200$. 
 
 Be sure you look at the simplified fraction and not just the decimal expansion. Numbers such as 13, 17, and 19 provide \textbf{long but repeating/terminating decimal expansions!} 
 
 The only ways to *not* be a Real number are: dividing by 0 or taking the square root of a negative number. 
 
 Irrational numbers are more than just square root of 3: adding or subtracting values from square root of 3 is also irrational.
}
\litem{
Choose the \textbf{smallest} set of Complex numbers that the number below belongs to.
\[ \sqrt{\frac{0}{144}}+\sqrt{4}i \]The solution is \( \text{Pure Imaginary} \), which is option D.\begin{enumerate}[label=\Alph*.]
\item \( \text{Rational} \)

These are numbers that can be written as fraction of Integers (e.g., -2/3 + 5)
\item \( \text{Not a Complex Number} \)

This is not a number. The only non-Complex number we know is dividing by 0 as this is not a number!
\item \( \text{Nonreal Complex} \)

This is a Complex number $(a+bi)$ that is not Real (has $i$ as part of the number).
\item \( \text{Pure Imaginary} \)

* This is the correct option!
\item \( \text{Irrational} \)

These cannot be written as a fraction of Integers. Remember: $\pi$ is not an Integer!
\end{enumerate}

\textbf{General Comment:} Be sure to simplify $i^2 = -1$. This may remove the imaginary portion for your number. If you are having trouble, you may want to look at the \textit{Subgroups of the Real Numbers} section.
}
\litem{
Simplify the expression below and choose the interval the simplification is contained within.
\[ 19 - 11^2 + 20 \div 7 * 13 \div 17 \]The solution is \( -99.815 \), which is option D.\begin{enumerate}[label=\Alph*.]
\item \( [141, 142.7] \)

 142.185, which corresponds to an Order of Operations error: multiplying by negative before squaring. For example: $(-3)^2 \neq -3^2$
\item \( [-104.1, -101.6] \)

 -101.987, which corresponds to an Order of Operations error: not reading left-to-right for multiplication/division.
\item \( [136.5, 141.4] \)

 140.013, which corresponds to two Order of Operations errors.
\item \( [-100.3, -98.1] \)

* -99.815, this is the correct option
\item \( \text{None of the above} \)

 You may have gotten this by making an unanticipated error. If you got a value that is not any of the others, please let the coordinator know so they can help you figure out what happened.
\end{enumerate}

\textbf{General Comment:} While you may remember (or were taught) PEMDAS is done in order, it is actually done as P/E/MD/AS. When we are at MD or AS, we read left to right.
}
\litem{
Simplify the expression below into the form $a+bi$. Then, choose the intervals that $a$ and $b$ belong to.
\[ (8 + 6 i)(2 + 3 i) \]The solution is \( -2 + 36 i \), which is option E.\begin{enumerate}[label=\Alph*.]
\item \( a \in [-12, -1] \text{ and } b \in [-45, -33] \)

 $-2 - 36 i$, which corresponds to adding a minus sign in both terms.
\item \( a \in [33, 39] \text{ and } b \in [9, 15] \)

 $34 + 12 i$, which corresponds to adding a minus sign in the first term.
\item \( a \in [14, 21] \text{ and } b \in [17, 23] \)

 $16 + 18 i$, which corresponds to just multiplying the real terms to get the real part of the solution and the coefficients in the complex terms to get the complex part.
\item \( a \in [33, 39] \text{ and } b \in [-13, -3] \)

 $34 - 12 i$, which corresponds to adding a minus sign in the second term.
\item \( a \in [-12, -1] \text{ and } b \in [35, 40] \)

* $-2 + 36 i$, which is the correct option.
\end{enumerate}

\textbf{General Comment:} You can treat $i$ as a variable and distribute. Just remember that $i^2=-1$, so you can continue to reduce after you distribute.
}
\litem{
Simplify the expression below and choose the interval the simplification is contained within.
\[ 19 - 9^2 + 8 \div 5 * 4 \div 20 \]The solution is \( -61.680 \), which is option A.\begin{enumerate}[label=\Alph*.]
\item \( [-61.93, -61.53] \)

* -61.680, this is the correct option
\item \( [100, 100.05] \)

 100.020, which corresponds to two Order of Operations errors.
\item \( [-62.15, -61.94] \)

 -61.980, which corresponds to an Order of Operations error: not reading left-to-right for multiplication/division.
\item \( [100.18, 100.45] \)

 100.320, which corresponds to an Order of Operations error: multiplying by negative before squaring. For example: $(-3)^2 \neq -3^2$
\item \( \text{None of the above} \)

 You may have gotten this by making an unanticipated error. If you got a value that is not any of the others, please let the coordinator know so they can help you figure out what happened.
\end{enumerate}

\textbf{General Comment:} While you may remember (or were taught) PEMDAS is done in order, it is actually done as P/E/MD/AS. When we are at MD or AS, we read left to right.
}
\litem{
Choose the \textbf{smallest} set of Complex numbers that the number below belongs to.
\[ \frac{\sqrt{65}}{18}+\sqrt{-2}i \]The solution is \( \text{Irrational} \), which is option E.\begin{enumerate}[label=\Alph*.]
\item \( \text{Nonreal Complex} \)

This is a Complex number $(a+bi)$ that is not Real (has $i$ as part of the number).
\item \( \text{Not a Complex Number} \)

This is not a number. The only non-Complex number we know is dividing by 0 as this is not a number!
\item \( \text{Pure Imaginary} \)

This is a Complex number $(a+bi)$ that \textbf{only} has an imaginary part like $2i$.
\item \( \text{Rational} \)

These are numbers that can be written as fraction of Integers (e.g., -2/3 + 5)
\item \( \text{Irrational} \)

* This is the correct option!
\end{enumerate}

\textbf{General Comment:} Be sure to simplify $i^2 = -1$. This may remove the imaginary portion for your number. If you are having trouble, you may want to look at the \textit{Subgroups of the Real Numbers} section.
}
\litem{
Simplify the expression below into the form $a+bi$. Then, choose the intervals that $a$ and $b$ belong to.
\[ \frac{-72 - 55 i}{-7 + 3 i} \]The solution is \( 5.84  + 10.36 i \), which is option E.\begin{enumerate}[label=\Alph*.]
\item \( a \in [11, 12] \text{ and } b \in [1.5, 3] \)

 $11.53  + 2.91 i$, which corresponds to forgetting to multiply the conjugate by the numerator and not computing the conjugate correctly.
\item \( a \in [4.5, 7.5] \text{ and } b \in [600.5, 602.5] \)

 $5.84  + 601.00 i$, which corresponds to forgetting to multiply the conjugate by the numerator.
\item \( a \in [338.5, 339.5] \text{ and } b \in [9.5, 12] \)

 $339.00  + 10.36 i$, which corresponds to forgetting to multiply the conjugate by the numerator and using a plus instead of a minus in the denominator.
\item \( a \in [10, 11.5] \text{ and } b \in [-19, -17.5] \)

 $10.29  - 18.33 i$, which corresponds to just dividing the first term by the first term and the second by the second.
\item \( a \in [4.5, 7.5] \text{ and } b \in [9.5, 12] \)

* $5.84  + 10.36 i$, which is the correct option.
\end{enumerate}

\textbf{General Comment:} Multiply the numerator and denominator by the *conjugate* of the denominator, then simplify. For example, if we have $2+3i$, the conjugate is $2-3i$.
}
\litem{
Simplify the expression below into the form $a+bi$. Then, choose the intervals that $a$ and $b$ belong to.
\[ (3 - 10 i)(-5 + 8 i) \]The solution is \( 65 + 74 i \), which is option A.\begin{enumerate}[label=\Alph*.]
\item \( a \in [63, 66] \text{ and } b \in [72, 75] \)

* $65 + 74 i$, which is the correct option.
\item \( a \in [-17, -12] \text{ and } b \in [-84, -79] \)

 $-15 - 80 i$, which corresponds to just multiplying the real terms to get the real part of the solution and the coefficients in the complex terms to get the complex part.
\item \( a \in [63, 66] \text{ and } b \in [-74, -69] \)

 $65 - 74 i$, which corresponds to adding a minus sign in both terms.
\item \( a \in [-102, -92] \text{ and } b \in [24, 29] \)

 $-95 + 26 i$, which corresponds to adding a minus sign in the second term.
\item \( a \in [-102, -92] \text{ and } b \in [-31, -22] \)

 $-95 - 26 i$, which corresponds to adding a minus sign in the first term.
\end{enumerate}

\textbf{General Comment:} You can treat $i$ as a variable and distribute. Just remember that $i^2=-1$, so you can continue to reduce after you distribute.
}
\litem{
Choose the \textbf{smallest} set of Real numbers that the number below belongs to.
\[ \sqrt{\frac{121}{324}} \]The solution is \( \text{Rational} \), which is option C.\begin{enumerate}[label=\Alph*.]
\item \( \text{Integer} \)

These are the negative and positive counting numbers (..., -3, -2, -1, 0, 1, 2, 3, ...)
\item \( \text{Whole} \)

These are the counting numbers with 0 (0, 1, 2, 3, ...)
\item \( \text{Rational} \)

* This is the correct option!
\item \( \text{Not a Real number} \)

These are Nonreal Complex numbers \textbf{OR} things that are not numbers (e.g., dividing by 0).
\item \( \text{Irrational} \)

These cannot be written as a fraction of Integers.
\end{enumerate}

\textbf{General Comment:} First, you \textbf{NEED} to simplify the expression. This question simplifies to $\frac{11}{18}$. 
 
 Be sure you look at the simplified fraction and not just the decimal expansion. Numbers such as 13, 17, and 19 provide \textbf{long but repeating/terminating decimal expansions!} 
 
 The only ways to *not* be a Real number are: dividing by 0 or taking the square root of a negative number. 
 
 Irrational numbers are more than just square root of 3: adding or subtracting values from square root of 3 is also irrational.
}
\litem{
Simplify the expression below into the form $a+bi$. Then, choose the intervals that $a$ and $b$ belong to.
\[ \frac{36 - 88 i}{2 + i} \]The solution is \( -3.20  - 42.40 i \), which is option D.\begin{enumerate}[label=\Alph*.]
\item \( a \in [17.5, 18.5] \text{ and } b \in [-89, -87] \)

 $18.00  - 88.00 i$, which corresponds to just dividing the first term by the first term and the second by the second.
\item \( a \in [-4.5, -2] \text{ and } b \in [-213, -211] \)

 $-3.20  - 212.00 i$, which corresponds to forgetting to multiply the conjugate by the numerator.
\item \( a \in [31, 33.5] \text{ and } b \in [-29.5, -27.5] \)

 $32.00  - 28.00 i$, which corresponds to forgetting to multiply the conjugate by the numerator and not computing the conjugate correctly.
\item \( a \in [-4.5, -2] \text{ and } b \in [-43, -41.5] \)

* $-3.20  - 42.40 i$, which is the correct option.
\item \( a \in [-16.5, -15.5] \text{ and } b \in [-43, -41.5] \)

 $-16.00  - 42.40 i$, which corresponds to forgetting to multiply the conjugate by the numerator and using a plus instead of a minus in the denominator.
\end{enumerate}

\textbf{General Comment:} Multiply the numerator and denominator by the *conjugate* of the denominator, then simplify. For example, if we have $2+3i$, the conjugate is $2-3i$.
}
\litem{
Choose the \textbf{smallest} set of Real numbers that the number below belongs to.
\[ \sqrt{\frac{3600}{36}} \]The solution is \( \text{Whole} \), which is option B.\begin{enumerate}[label=\Alph*.]
\item \( \text{Irrational} \)

These cannot be written as a fraction of Integers.
\item \( \text{Whole} \)

* This is the correct option!
\item \( \text{Integer} \)

These are the negative and positive counting numbers (..., -3, -2, -1, 0, 1, 2, 3, ...)
\item \( \text{Not a Real number} \)

These are Nonreal Complex numbers \textbf{OR} things that are not numbers (e.g., dividing by 0).
\item \( \text{Rational} \)

These are numbers that can be written as fraction of Integers (e.g., -2/3)
\end{enumerate}

\textbf{General Comment:} First, you \textbf{NEED} to simplify the expression. This question simplifies to $60$. 
 
 Be sure you look at the simplified fraction and not just the decimal expansion. Numbers such as 13, 17, and 19 provide \textbf{long but repeating/terminating decimal expansions!} 
 
 The only ways to *not* be a Real number are: dividing by 0 or taking the square root of a negative number. 
 
 Irrational numbers are more than just square root of 3: adding or subtracting values from square root of 3 is also irrational.
}
\end{enumerate}

\end{document}