\documentclass[14pt]{extbook}
\usepackage{multicol, enumerate, enumitem, hyperref, color, soul, setspace, parskip, fancyhdr} %General Packages
\usepackage{amssymb, amsthm, amsmath, latexsym, units, mathtools} %Math Packages
\everymath{\displaystyle} %All math in Display Style
% Packages with additional options
\usepackage[headsep=0.5cm,headheight=12pt, left=1 in,right= 1 in,top= 1 in,bottom= 1 in]{geometry}
\usepackage[usenames,dvipsnames]{xcolor}
\usepackage{dashrule}  % Package to use the command below to create lines between items
\newcommand{\litem}[1]{\item#1\hspace*{-1cm}\rule{\textwidth}{0.4pt}}
\pagestyle{fancy}
\lhead{Progress Quiz 10}
\chead{}
\rhead{Version ALL}
\lfoot{5170-5105}
\cfoot{}
\rfoot{Summer C 2021}
\begin{document}

\begin{enumerate}
\litem{
Choose the interval below that $f$ composed with $g$ at $x=-1$ is in.\[ f(x) = x^{3} -3 x^{2} -2 x + 1 \text{ and } g(x) = 4x^{3} +4 x^{2} -x \]\begin{enumerate}[label=\Alph*.]
\item \( (f \circ g)(-1) \in [9.3, 10.38] \)
\item \( (f \circ g)(-1) \in [-3.32, -2.98] \)
\item \( (f \circ g)(-1) \in [1.64, 2.57] \)
\item \( (f \circ g)(-1) \in [-0.35, 1.66] \)
\item \( \text{It is not possible to compose the two functions.} \)

\end{enumerate} }
\litem{
Determine whether the function below is 1-1.\[ f(x) = 36 x^2 + 480 x + 1600 \]\begin{enumerate}[label=\Alph*.]
\item \( \text{No, because there is a $y$-value that goes to 2 different $x$-values.} \)
\item \( \text{Yes, the function is 1-1.} \)
\item \( \text{No, because the domain of the function is not $(-\infty, \infty)$.} \)
\item \( \text{No, because the range of the function is not $(-\infty, \infty)$.} \)
\item \( \text{No, because there is an $x$-value that goes to 2 different $y$-values.} \)

\end{enumerate} }
\litem{
Multiply the following functions, then choose the domain of the resulting function from the list below.\[ f(x) = \sqrt{-4x+11}  \text{ and } g(x) = 6x + 4 \]\begin{enumerate}[label=\Alph*.]
\item \( \text{ The domain is all Real numbers less than or equal to } x = a, \text{ where } a \in [-0.25, 5.75] \)
\item \( \text{ The domain is all Real numbers except } x = a, \text{ where } a \in [-6.8, -0.8] \)
\item \( \text{ The domain is all Real numbers greater than or equal to } x = a, \text{ where } a \in [-12.4, -2.4] \)
\item \( \text{ The domain is all Real numbers except } x = a \text{ and } x = b, \text{ where } a \in [-9.4, -1.4] \text{ and } b \in [2.33, 14.33] \)
\item \( \text{ The domain is all Real numbers. } \)

\end{enumerate} }
\litem{
Find the inverse of the function below. Then, evaluate the inverse at $x = 10$ and choose the interval that $f^-1(10)$ belongs to.\[ f(x) = e^{x-5}+2 \]\begin{enumerate}[label=\Alph*.]
\item \( f^{-1}(10) \in [3.11, 3.77] \)
\item \( f^{-1}(10) \in [6.93, 7.45] \)
\item \( f^{-1}(10) \in [-2.94, -2.59] \)
\item \( f^{-1}(10) \in [4.02, 4.57] \)
\item \( f^{-1}(10) \in [4.57, 4.97] \)

\end{enumerate} }
\litem{
Find the inverse of the function below (if it exists). Then, evaluate the inverse at $x = -13$ and choose the interval that $f^-1(-13)$ belongs to.\[ f(x) = \sqrt[3]{3 x + 5} \]\begin{enumerate}[label=\Alph*.]
\item \( f^{-1}(-13) \in [722.67, 732.67] \)
\item \( f^{-1}(-13) \in [734, 740] \)
\item \( f^{-1}(-13) \in [-735, -733] \)
\item \( f^{-1}(-13) \in [-732.67, -723.67] \)
\item \( \text{ The function is not invertible for all Real numbers. } \)

\end{enumerate} }
\litem{
Choose the interval below that $f$ composed with $g$ at $x=1$ is in.\[ f(x) = 2x^{3} -1 x^{2} +4 x -4 \text{ and } g(x) = -2x^{3} + x^{2} +2 x + 1 \]\begin{enumerate}[label=\Alph*.]
\item \( (f \circ g)(1) \in [15, 25] \)
\item \( (f \circ g)(1) \in [-8, -5] \)
\item \( (f \circ g)(1) \in [-1, 3] \)
\item \( (f \circ g)(1) \in [23, 36] \)
\item \( \text{It is not possible to compose the two functions.} \)

\end{enumerate} }
\litem{
Add the following functions, then choose the domain of the resulting function from the list below.\[ f(x) = 8x^{4} +8 x^{3} +4 x^{2} +x \text{ and } g(x) = \frac{5}{5x+22} \]\begin{enumerate}[label=\Alph*.]
\item \( \text{ The domain is all Real numbers except } x = a, \text{ where } a \in [-6.4, 0.6] \)
\item \( \text{ The domain is all Real numbers greater than or equal to } x = a, \text{ where } a \in [-8.33, -0.33] \)
\item \( \text{ The domain is all Real numbers less than or equal to } x = a, \text{ where } a \in [-1.17, 6.83] \)
\item \( \text{ The domain is all Real numbers except } x = a \text{ and } x = b, \text{ where } a \in [3.67, 16.67] \text{ and } b \in [-9.17, -5.17] \)
\item \( \text{ The domain is all Real numbers. } \)

\end{enumerate} }
\litem{
Find the inverse of the function below. Then, evaluate the inverse at $x = 8$ and choose the interval that $f^-1(8)$ belongs to.\[ f(x) = e^{x+2}+5 \]\begin{enumerate}[label=\Alph*.]
\item \( f^{-1}(8) \in [6.73, 6.89] \)
\item \( f^{-1}(8) \in [7.35, 7.89] \)
\item \( f^{-1}(8) \in [-1.17, -0.58] \)
\item \( f^{-1}(8) \in [2.92, 3.25] \)
\item \( f^{-1}(8) \in [6.92, 7.48] \)

\end{enumerate} }
\litem{
Determine whether the function below is 1-1.\[ f(x) = 20 x^2 - 68 x - 736 \]\begin{enumerate}[label=\Alph*.]
\item \( \text{Yes, the function is 1-1.} \)
\item \( \text{No, because the domain of the function is not $(-\infty, \infty)$.} \)
\item \( \text{No, because there is a $y$-value that goes to 2 different $x$-values.} \)
\item \( \text{No, because there is an $x$-value that goes to 2 different $y$-values.} \)
\item \( \text{No, because the range of the function is not $(-\infty, \infty)$.} \)

\end{enumerate} }
\litem{
Find the inverse of the function below (if it exists). Then, evaluate the inverse at $x = -10$ and choose the interval that $f^-1(-10)$ belongs to.\[ f(x) = 4 x^2 - 5 \]\begin{enumerate}[label=\Alph*.]
\item \( f^{-1}(-10) \in [1.26, 2.12] \)
\item \( f^{-1}(-10) \in [2.98, 3.66] \)
\item \( f^{-1}(-10) \in [1.05, 1.21] \)
\item \( f^{-1}(-10) \in [3.84, 4.42] \)
\item \( \text{ The function is not invertible for all Real numbers. } \)

\end{enumerate} }
\litem{
Choose the interval below that $f$ composed with $g$ at $x=1$ is in.\[ f(x) = 2x^{3} + x^{2} -3 x \text{ and } g(x) = -x^{3} +2 x^{2} -3 x \]\begin{enumerate}[label=\Alph*.]
\item \( (f \circ g)(1) \in [-8, -5] \)
\item \( (f \circ g)(1) \in [-8, -5] \)
\item \( (f \circ g)(1) \in [-3, 2] \)
\item \( (f \circ g)(1) \in [-16, -10] \)
\item \( \text{It is not possible to compose the two functions.} \)

\end{enumerate} }
\litem{
Determine whether the function below is 1-1.\[ f(x) = 16 x^2 - 80 x + 100 \]\begin{enumerate}[label=\Alph*.]
\item \( \text{No, because the domain of the function is not $(-\infty, \infty)$.} \)
\item \( \text{No, because there is an $x$-value that goes to 2 different $y$-values.} \)
\item \( \text{No, because the range of the function is not $(-\infty, \infty)$.} \)
\item \( \text{No, because there is a $y$-value that goes to 2 different $x$-values.} \)
\item \( \text{Yes, the function is 1-1.} \)

\end{enumerate} }
\litem{
Multiply the following functions, then choose the domain of the resulting function from the list below.\[ f(x) = \sqrt{-3x+12}  \text{ and } g(x) = 5x + 5 \]\begin{enumerate}[label=\Alph*.]
\item \( \text{ The domain is all Real numbers greater than or equal to } x = a, \text{ where } a \in [-6.25, -2.25] \)
\item \( \text{ The domain is all Real numbers except } x = a, \text{ where } a \in [-6.4, 0.6] \)
\item \( \text{ The domain is all Real numbers less than or equal to } x = a, \text{ where } a \in [4, 6] \)
\item \( \text{ The domain is all Real numbers except } x = a \text{ and } x = b, \text{ where } a \in [-7.8, -0.8] \text{ and } b \in [3.75, 7.75] \)
\item \( \text{ The domain is all Real numbers. } \)

\end{enumerate} }
\litem{
Find the inverse of the function below. Then, evaluate the inverse at $x = 8$ and choose the interval that $f^-1(8)$ belongs to.\[ f(x) = e^{x+4}-2 \]\begin{enumerate}[label=\Alph*.]
\item \( f^{-1}(8) \in [-0.66, -0.48] \)
\item \( f^{-1}(8) \in [6.02, 6.73] \)
\item \( f^{-1}(8) \in [-0.35, -0.08] \)
\item \( f^{-1}(8) \in [-1.75, -1.28] \)
\item \( f^{-1}(8) \in [0.38, 0.63] \)

\end{enumerate} }
\litem{
Find the inverse of the function below (if it exists). Then, evaluate the inverse at $x = -12$ and choose the interval that $f^-1(-12)$ belongs to.\[ f(x) = 4 x^2 - 3 \]\begin{enumerate}[label=\Alph*.]
\item \( f^{-1}(-12) \in [3.32, 4.02] \)
\item \( f^{-1}(-12) \in [1.6, 2.66] \)
\item \( f^{-1}(-12) \in [5.18, 5.92] \)
\item \( f^{-1}(-12) \in [1.28, 1.69] \)
\item \( \text{ The function is not invertible for all Real numbers. } \)

\end{enumerate} }
\litem{
Choose the interval below that $f$ composed with $g$ at $x=1$ is in.\[ f(x) = -2x^{3} +2 x^{2} +3 x -1 \text{ and } g(x) = -x^{3} +2 x^{2} -2 x + 4 \]\begin{enumerate}[label=\Alph*.]
\item \( (f \circ g)(1) \in [-7, 5] \)
\item \( (f \circ g)(1) \in [-28, -27] \)
\item \( (f \circ g)(1) \in [-14, -6] \)
\item \( (f \circ g)(1) \in [-26, -22] \)
\item \( \text{It is not possible to compose the two functions.} \)

\end{enumerate} }
\litem{
Multiply the following functions, then choose the domain of the resulting function from the list below.\[ f(x) = 5x^{2} +3 x + 3 \text{ and } g(x) = 2x^{4} + x^{3} +7 x^{2} +8 x + 9 \]\begin{enumerate}[label=\Alph*.]
\item \( \text{ The domain is all Real numbers less than or equal to } x = a, \text{ where } a \in [3.5, 8.5] \)
\item \( \text{ The domain is all Real numbers except } x = a, \text{ where } a \in [4.8, 6.8] \)
\item \( \text{ The domain is all Real numbers greater than or equal to } x = a, \text{ where } a \in [-12.67, -5.67] \)
\item \( \text{ The domain is all Real numbers except } x = a \text{ and } x = b, \text{ where } a \in [4.2, 8.2] \text{ and } b \in [-10.6, -2.6] \)
\item \( \text{ The domain is all Real numbers. } \)

\end{enumerate} }
\litem{
Find the inverse of the function below. Then, evaluate the inverse at $x = 5$ and choose the interval that $f^-1(5)$ belongs to.\[ f(x) = e^{x+2}-3 \]\begin{enumerate}[label=\Alph*.]
\item \( f^{-1}(5) \in [-2.77, -1.99] \)
\item \( f^{-1}(5) \in [-1.09, -0.58] \)
\item \( f^{-1}(5) \in [-0.74, 0.63] \)
\item \( f^{-1}(5) \in [-1.97, -1.86] \)
\item \( f^{-1}(5) \in [3.78, 4.92] \)

\end{enumerate} }
\litem{
Determine whether the function below is 1-1.\[ f(x) = 12 x^2 - 114 x + 252 \]\begin{enumerate}[label=\Alph*.]
\item \( \text{No, because there is an $x$-value that goes to 2 different $y$-values.} \)
\item \( \text{No, because there is a $y$-value that goes to 2 different $x$-values.} \)
\item \( \text{Yes, the function is 1-1.} \)
\item \( \text{No, because the domain of the function is not $(-\infty, \infty)$.} \)
\item \( \text{No, because the range of the function is not $(-\infty, \infty)$.} \)

\end{enumerate} }
\litem{
Find the inverse of the function below (if it exists). Then, evaluate the inverse at $x = -13$ and choose the interval that $f^-1(-13)$ belongs to.\[ f(x) = \sqrt[3]{5 x + 4} \]\begin{enumerate}[label=\Alph*.]
\item \( f^{-1}(-13) \in [438.59, 439.36] \)
\item \( f^{-1}(-13) \in [-439.72, -437.92] \)
\item \( f^{-1}(-13) \in [-440.63, -439.67] \)
\item \( f^{-1}(-13) \in [439.31, 440.26] \)
\item \( \text{ The function is not invertible for all Real numbers. } \)

\end{enumerate} }
\litem{
Choose the interval below that $f$ composed with $g$ at $x=-1$ is in.\[ f(x) = 3x^{3} +3 x^{2} -2 x \text{ and } g(x) = -2x^{3} -3 x^{2} -2 x -3 \]\begin{enumerate}[label=\Alph*.]
\item \( (f \circ g)(-1) \in [-35.4, -34.9] \)
\item \( (f \circ g)(-1) \in [-11.2, -7.5] \)
\item \( (f \circ g)(-1) \in [-4.3, 0.9] \)
\item \( (f \circ g)(-1) \in [-33.6, -28.9] \)
\item \( \text{It is not possible to compose the two functions.} \)

\end{enumerate} }
\litem{
Determine whether the function below is 1-1.\[ f(x) = (5 x - 18)^3 \]\begin{enumerate}[label=\Alph*.]
\item \( \text{Yes, the function is 1-1.} \)
\item \( \text{No, because there is a $y$-value that goes to 2 different $x$-values.} \)
\item \( \text{No, because the domain of the function is not $(-\infty, \infty)$.} \)
\item \( \text{No, because the range of the function is not $(-\infty, \infty)$.} \)
\item \( \text{No, because there is an $x$-value that goes to 2 different $y$-values.} \)

\end{enumerate} }
\litem{
Multiply the following functions, then choose the domain of the resulting function from the list below.\[ f(x) = 5x^{4} +9 x^{3} +3 x^{2} +4 x + 8 \text{ and } g(x) = 2x + 2 \]\begin{enumerate}[label=\Alph*.]
\item \( \text{ The domain is all Real numbers except } x = a, \text{ where } a \in [-9.2, -1.2] \)
\item \( \text{ The domain is all Real numbers less than or equal to } x = a, \text{ where } a \in [1.5, 7.5] \)
\item \( \text{ The domain is all Real numbers greater than or equal to } x = a, \text{ where } a \in [-11.33, 0.67] \)
\item \( \text{ The domain is all Real numbers except } x = a \text{ and } x = b, \text{ where } a \in [3.25, 6.25] \text{ and } b \in [-10.2, -3.2] \)
\item \( \text{ The domain is all Real numbers. } \)

\end{enumerate} }
\litem{
Find the inverse of the function below. Then, evaluate the inverse at $x = 7$ and choose the interval that $f^-1(7)$ belongs to.\[ f(x) = \ln{(x-3)}+5 \]\begin{enumerate}[label=\Alph*.]
\item \( f^{-1}(7) \in [8.39, 11.39] \)
\item \( f^{-1}(7) \in [162757.79, 162759.79] \)
\item \( f^{-1}(7) \in [22027.47, 22034.47] \)
\item \( f^{-1}(7) \in [56.6, 63.6] \)
\item \( f^{-1}(7) \in [-0.61, 5.39] \)

\end{enumerate} }
\litem{
Find the inverse of the function below (if it exists). Then, evaluate the inverse at $x = -11$ and choose the interval that $f^-1(-11)$ belongs to.\[ f(x) = 3 x^2 - 4 \]\begin{enumerate}[label=\Alph*.]
\item \( f^{-1}(-11) \in [1.94, 2.92] \)
\item \( f^{-1}(-11) \in [7.43, 7.82] \)
\item \( f^{-1}(-11) \in [1.19, 1.93] \)
\item \( f^{-1}(-11) \in [4.35, 5.19] \)
\item \( \text{ The function is not invertible for all Real numbers. } \)

\end{enumerate} }
\litem{
Choose the interval below that $f$ composed with $g$ at $x=1$ is in.\[ f(x) = -x^{3} -2 x^{2} +2 x + 3 \text{ and } g(x) = -4x^{3} -1 x^{2} +2 x + 2 \]\begin{enumerate}[label=\Alph*.]
\item \( (f \circ g)(1) \in [0, 6] \)
\item \( (f \circ g)(1) \in [-30, -25] \)
\item \( (f \circ g)(1) \in [-5, -2] \)
\item \( (f \circ g)(1) \in [-25, -22] \)
\item \( \text{It is not possible to compose the two functions.} \)

\end{enumerate} }
\litem{
Multiply the following functions, then choose the domain of the resulting function from the list below.\[ f(x) = 4x^{4} +2 x^{3} +2 x + 9 \text{ and } g(x) = 4x + 3 \]\begin{enumerate}[label=\Alph*.]
\item \( \text{ The domain is all Real numbers greater than or equal to } x = a, \text{ where } a \in [-7, 1] \)
\item \( \text{ The domain is all Real numbers except } x = a, \text{ where } a \in [-4.4, 0.6] \)
\item \( \text{ The domain is all Real numbers less than or equal to } x = a, \text{ where } a \in [1.67, 3.67] \)
\item \( \text{ The domain is all Real numbers except } x = a \text{ and } x = b, \text{ where } a \in [7.2, 15.2] \text{ and } b \in [-7.6, -1.6] \)
\item \( \text{ The domain is all Real numbers. } \)

\end{enumerate} }
\litem{
Find the inverse of the function below. Then, evaluate the inverse at $x = 7$ and choose the interval that $f^-1(7)$ belongs to.\[ f(x) = e^{x-2}-3 \]\begin{enumerate}[label=\Alph*.]
\item \( f^{-1}(7) \in [-0.52, 1.86] \)
\item \( f^{-1}(7) \in [-1.47, -0.87] \)
\item \( f^{-1}(7) \in [3.95, 4.49] \)
\item \( f^{-1}(7) \in [-1, -0.76] \)
\item \( f^{-1}(7) \in [-2.16, -1.55] \)

\end{enumerate} }
\litem{
Determine whether the function below is 1-1.\[ f(x) = 16 x^2 + 128 x + 256 \]\begin{enumerate}[label=\Alph*.]
\item \( \text{No, because there is a $y$-value that goes to 2 different $x$-values.} \)
\item \( \text{No, because the range of the function is not $(-\infty, \infty)$.} \)
\item \( \text{No, because the domain of the function is not $(-\infty, \infty)$.} \)
\item \( \text{No, because there is an $x$-value that goes to 2 different $y$-values.} \)
\item \( \text{Yes, the function is 1-1.} \)

\end{enumerate} }
\litem{
Find the inverse of the function below (if it exists). Then, evaluate the inverse at $x = -15$ and choose the interval that $f^-1(-15)$ belongs to.\[ f(x) = \sqrt[3]{2 x - 3} \]\begin{enumerate}[label=\Alph*.]
\item \( f^{-1}(-15) \in [1684, 1688.7] \)
\item \( f^{-1}(-15) \in [1686.8, 1691.3] \)
\item \( f^{-1}(-15) \in [-1691.2, -1687.3] \)
\item \( f^{-1}(-15) \in [-1687.2, -1683.8] \)
\item \( \text{ The function is not invertible for all Real numbers. } \)

\end{enumerate} }
\end{enumerate}

\end{document}