\documentclass[14pt]{extbook}
\usepackage{multicol, enumerate, enumitem, hyperref, color, soul, setspace, parskip, fancyhdr} %General Packages
\usepackage{amssymb, amsthm, amsmath, bbm, latexsym, units, mathtools} %Math Packages
\everymath{\displaystyle} %All math in Display Style
% Packages with additional options
\usepackage[headsep=0.5cm,headheight=12pt, left=1 in,right= 1 in,top= 1 in,bottom= 1 in]{geometry}
\usepackage[usenames,dvipsnames]{xcolor}
\usepackage{dashrule}  % Package to use the command below to create lines between items
\newcommand{\litem}[1]{\item#1\hspace*{-1cm}\rule{\textwidth}{0.4pt}}
\pagestyle{fancy}
\lhead{Progress Quiz 10}
\chead{}
\rhead{Version C}
\lfoot{6232-9639}
\cfoot{}
\rfoot{Fall 2020}
\begin{document}

\begin{enumerate}
\litem{
Perform the division below. Then, find the intervals that correspond to the quotient in the form $ax^2+bx+c$ and remainder $r$.\[ \frac{20x^{3} +87 x^{2} -80 x -72}{x + 5} \]\begin{enumerate}[label=\Alph*.]
\item \( a \in [19, 26], \text{   } b \in [185, 190], \text{   } c \in [847, 860], \text{   and   } r \in [4200, 4206]. \)
\item \( a \in [19, 26], \text{   } b \in [-42, -29], \text{   } c \in [112, 121], \text{   and   } r \in [-785, -778]. \)
\item \( a \in [-101, -95], \text{   } b \in [-414, -407], \text{   } c \in [-2147, -2143], \text{   and   } r \in [-10799, -10794]. \)
\item \( a \in [-101, -95], \text{   } b \in [585, 592], \text{   } c \in [-3015, -3012], \text{   and   } r \in [14999, 15004]. \)
\item \( a \in [19, 26], \text{   } b \in [-16, -11], \text{   } c \in [-16, -4], \text{   and   } r \in [-2, 6]. \)

\end{enumerate} }
\litem{
What are the \textit{possible Rational} roots of the polynomial below?\[ f(x) = 4x^{4} +6 x^{3} +7 x^{2} +7 x + 2 \]\begin{enumerate}[label=\Alph*.]
\item \( \pm 1,\pm 2 \)
\item \( \text{ All combinations of: }\frac{\pm 1,\pm 2}{\pm 1,\pm 2,\pm 4} \)
\item \( \pm 1,\pm 2,\pm 4 \)
\item \( \text{ All combinations of: }\frac{\pm 1,\pm 2,\pm 4}{\pm 1,\pm 2} \)
\item \( \text{ There is no formula or theorem that tells us all possible Rational roots.} \)

\end{enumerate} }
\litem{
Factor the polynomial below completely, knowing that $x+3$ is a factor. Then, choose the intervals the zeros of the polynomial belong to, where $z_1 \leq z_2 \leq z_3 \leq z_4$. \textit{To make the problem easier, all zeros are between -5 and 5.}\[ f(x) = 20x^{4} +127 x^{3} +46 x^{2} -415 x + 150 \]\begin{enumerate}[label=\Alph*.]
\item \( z_1 \in [-3.5, -1.5], \text{   }  z_2 \in [-1.11, -0.49], z_3 \in [2.9, 3.27], \text{   and   } z_4 \in [4.3, 6.2] \)
\item \( z_1 \in [-5, -3], \text{   }  z_2 \in [-3.21, -2.66], z_3 \in [0.66, 0.81], \text{   and   } z_4 \in [2.3, 2.6] \)
\item \( z_1 \in [-1.25, 2.75], \text{   }  z_2 \in [-0.65, -0.3], z_3 \in [2.9, 3.27], \text{   and   } z_4 \in [4.3, 6.2] \)
\item \( z_1 \in [-5, -3], \text{   }  z_2 \in [-3.21, -2.66], z_3 \in [-0.12, 0.4], \text{   and   } z_4 \in [-0.3, 1.8] \)
\item \( z_1 \in [-5, -3], \text{   }  z_2 \in [-0.2, 0.37], z_3 \in [2.9, 3.27], \text{   and   } z_4 \in [4.3, 6.2] \)

\end{enumerate} }
\litem{
Perform the division below. Then, find the intervals that correspond to the quotient in the form $ax^2+bx+c$ and remainder $r$.\[ \frac{20x^{3} -76 x^{2} -32 x + 59}{x -4} \]\begin{enumerate}[label=\Alph*.]
\item \( a \in [18, 25], \text{   } b \in [-158, -152], \text{   } c \in [589, 597], \text{   and   } r \in [-2311, -2305]. \)
\item \( a \in [76, 85], \text{   } b \in [-399, -395], \text{   } c \in [1550, 1562], \text{   and   } r \in [-6154, -6144]. \)
\item \( a \in [76, 85], \text{   } b \in [242, 248], \text{   } c \in [943, 945], \text{   and   } r \in [3830, 3837]. \)
\item \( a \in [18, 25], \text{   } b \in [-21, -11], \text{   } c \in [-82, -77], \text{   and   } r \in [-181, -175]. \)
\item \( a \in [18, 25], \text{   } b \in [2, 6], \text{   } c \in [-19, -12], \text{   and   } r \in [-7, -1]. \)

\end{enumerate} }
\litem{
Perform the division below. Then, find the intervals that correspond to the quotient in the form $ax^2+bx+c$ and remainder $r$.\[ \frac{15x^{3} -35 x^{2} + 24}{x -2} \]\begin{enumerate}[label=\Alph*.]
\item \( a \in [28, 31], b \in [23, 32], c \in [47, 54], \text{ and } r \in [116, 130]. \)
\item \( a \in [15, 19], b \in [-22, -14], c \in [-23, -16], \text{ and } r \in [0, 7]. \)
\item \( a \in [15, 19], b \in [-7, -1], c \in [-17, -8], \text{ and } r \in [0, 7]. \)
\item \( a \in [15, 19], b \in [-65, -59], c \in [129, 134], \text{ and } r \in [-241, -231]. \)
\item \( a \in [28, 31], b \in [-104, -92], c \in [189, 194], \text{ and } r \in [-362, -355]. \)

\end{enumerate} }
\litem{
What are the \textit{possible Rational} roots of the polynomial below?\[ f(x) = 6x^{3} +2 x^{2} +7 x + 7 \]\begin{enumerate}[label=\Alph*.]
\item \( \text{ All combinations of: }\frac{\pm 1,\pm 7}{\pm 1,\pm 2,\pm 3,\pm 6} \)
\item \( \text{ All combinations of: }\frac{\pm 1,\pm 2,\pm 3,\pm 6}{\pm 1,\pm 7} \)
\item \( \pm 1,\pm 7 \)
\item \( \pm 1,\pm 2,\pm 3,\pm 6 \)
\item \( \text{ There is no formula or theorem that tells us all possible Rational roots.} \)

\end{enumerate} }
\litem{
Perform the division below. Then, find the intervals that correspond to the quotient in the form $ax^2+bx+c$ and remainder $r$.\[ \frac{6x^{3} -42 x + 38}{x + 3} \]\begin{enumerate}[label=\Alph*.]
\item \( a \in [1, 11], b \in [-18, -14], c \in [11, 20], \text{ and } r \in [2, 8]. \)
\item \( a \in [-21, -13], b \in [46, 58], c \in [-206, -203], \text{ and } r \in [644, 651]. \)
\item \( a \in [1, 11], b \in [15, 25], c \in [11, 20], \text{ and } r \in [70, 78]. \)
\item \( a \in [-21, -13], b \in [-59, -48], c \in [-206, -203], \text{ and } r \in [-576, -567]. \)
\item \( a \in [1, 11], b \in [-25, -23], c \in [54, 59], \text{ and } r \in [-181, -170]. \)

\end{enumerate} }
\litem{
Factor the polynomial below completely. Then, choose the intervals the zeros of the polynomial belong to, where $z_1 \leq z_2 \leq z_3$. \textit{To make the problem easier, all zeros are between -5 and 5.}\[ f(x) = 6x^{3} +29 x^{2} -20 x -75 \]\begin{enumerate}[label=\Alph*.]
\item \( z_1 \in [-0.76, -0.43], \text{   }  z_2 \in [0.2, 1.1], \text{   and   } z_3 \in [4, 8] \)
\item \( z_1 \in [-1.72, -1.37], \text{   }  z_2 \in [1, 1.8], \text{   and   } z_3 \in [4, 8] \)
\item \( z_1 \in [-5.03, -4.73], \text{   }  z_2 \in [-1.4, 0.4], \text{   and   } z_3 \in [0.6, 1.6] \)
\item \( z_1 \in [-5.03, -4.73], \text{   }  z_2 \in [-3.4, -0.7], \text{   and   } z_3 \in [0.67, 2.67] \)
\item \( z_1 \in [-1.13, -0.74], \text{   }  z_2 \in [2.5, 3.6], \text{   and   } z_3 \in [4, 8] \)

\end{enumerate} }
\litem{
Factor the polynomial below completely. Then, choose the intervals the zeros of the polynomial belong to, where $z_1 \leq z_2 \leq z_3$. \textit{To make the problem easier, all zeros are between -5 and 5.}\[ f(x) = 4x^{3} -49 x -60 \]\begin{enumerate}[label=\Alph*.]
\item \( z_1 \in [-2.5, -1.5], \text{   }  z_2 \in [-1.97, -0.97], \text{   and   } z_3 \in [3.2, 4.3] \)
\item \( z_1 \in [-6, -3], \text{   }  z_2 \in [0.27, 0.52], \text{   and   } z_3 \in [-0.1, 0.8] \)
\item \( z_1 \in [-0.67, 0.33], \text{   }  z_2 \in [-0.59, -0.2], \text{   and   } z_3 \in [3.2, 4.3] \)
\item \( z_1 \in [-6, -3], \text{   }  z_2 \in [0.52, 0.99], \text{   and   } z_3 \in [4.2, 6.5] \)
\item \( z_1 \in [-6, -3], \text{   }  z_2 \in [1.43, 2.03], \text{   and   } z_3 \in [2, 3.6] \)

\end{enumerate} }
\litem{
Factor the polynomial below completely, knowing that $x-3$ is a factor. Then, choose the intervals the zeros of the polynomial belong to, where $z_1 \leq z_2 \leq z_3 \leq z_4$. \textit{To make the problem easier, all zeros are between -5 and 5.}\[ f(x) = 6x^{4} -7 x^{3} -118 x^{2} +305 x -150 \]\begin{enumerate}[label=\Alph*.]
\item \( z_1 \in [-4.1, -2.1], \text{   }  z_2 \in [-1.61, -1.37], z_3 \in [-0.54, -0.06], \text{   and   } z_4 \in [4.9, 6.7] \)
\item \( z_1 \in [-5.9, -3.7], \text{   }  z_2 \in [0.34, 0.44], z_3 \in [1.16, 1.7], \text{   and   } z_4 \in [2.1, 3.9] \)
\item \( z_1 \in [-4.1, -2.1], \text{   }  z_2 \in [-2.02, -1.74], z_3 \in [-0.91, -0.7], \text{   and   } z_4 \in [4.9, 6.7] \)
\item \( z_1 \in [-4.1, -2.1], \text{   }  z_2 \in [-2.63, -2.41], z_3 \in [-0.68, -0.44], \text{   and   } z_4 \in [4.9, 6.7] \)
\item \( z_1 \in [-5.9, -3.7], \text{   }  z_2 \in [0.46, 0.72], z_3 \in [2.32, 2.9], \text{   and   } z_4 \in [2.1, 3.9] \)

\end{enumerate} }
\end{enumerate}

\end{document}