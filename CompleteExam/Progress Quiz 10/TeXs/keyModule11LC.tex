\documentclass{extbook}[14pt]
\usepackage{multicol, enumerate, enumitem, hyperref, color, soul, setspace, parskip, fancyhdr, amssymb, amsthm, amsmath, latexsym, units, mathtools}
\everymath{\displaystyle}
\usepackage[headsep=0.5cm,headheight=0cm, left=1 in,right= 1 in,top= 1 in,bottom= 1 in]{geometry}
\usepackage{dashrule}  % Package to use the command below to create lines between items
\newcommand{\litem}[1]{\item #1

\rule{\textwidth}{0.4pt}}
\pagestyle{fancy}
\lhead{}
\chead{Answer Key for Progress Quiz 10 Version C}
\rhead{}
\lfoot{5170-5105}
\cfoot{}
\rfoot{Summer C 2021}
\begin{document}
\textbf{This key should allow you to understand why you choose the option you did (beyond just getting a question right or wrong). \href{https://xronos.clas.ufl.edu/mac1105spring2020/courseDescriptionAndMisc/Exams/LearningFromResults}{More instructions on how to use this key can be found here}.}

\textbf{If you have a suggestion to make the keys better, \href{https://forms.gle/CZkbZmPbC9XALEE88}{please fill out the short survey here}.}

\textit{Note: This key is auto-generated and may contain issues and/or errors. The keys are reviewed after each exam to ensure grading is done accurately. If there are issues (like duplicate options), they are noted in the offline gradebook. The keys are a work-in-progress to give students as many resources to improve as possible.}

\rule{\textwidth}{0.4pt}

\begin{enumerate}\litem{
Evaluate the one-sided limit of the function $f(x)$ below, if possible.
\[ \lim_{x \rightarrow -6^-} \frac{-4}{(x+6)^5}+4 \]The solution is \( \infty \), which is option A.\begin{enumerate}[label=\Alph*.]
\item \( \infty \)


\item \( f(-6) \)


\item \( -\infty \)


\item \( \text{The limit does not exist} \)


\item \( \text{None of the above} \)


\end{enumerate}

\textbf{General Comment:} \textbf{General comments:} You should be able to graph the rational function displayed. If not, go back to Module 7 to learn about the general shape of rational functions.
}
\litem{
Evaluate the limit below, if possible.
\[ \lim_{x \rightarrow 9} \frac{\sqrt{6x - 18} - 6}{4x - 36} \]The solution is \( 0.125 \), which is option B.\begin{enumerate}[label=\Alph*.]
\item \( 0.083 \)

You likely memorized how to solve the similar homework problem and used the same formula here.
\item \( 0.125 \)

* This is the correct option.
\item \( 0.612 \)

You likely tried to use a shortcut to find the limit of a function that only works for when the numerator/denominator are polynomials.
\item \( \infty \)

You likely believed that since the denominator is equal to 0, the limit is infinity.
\item \( \text{None of the above} \)

If you got a limit that does not match any of the above, please contact the coordinator.
\end{enumerate}

\textbf{General Comment:} \textbf{General comments:} It is difficult to imagine the graph of this function, so you need to test values close to $x = 9$.
}
\litem{
Based on the information below, which of the following statements is always true?

\begin{center}
    \textit{ As $x$ approaches $8$, $f(x)$ approaches $16.975$. }
\end{center}
The solution is \( \text{None of the above are always true.} \), which is option E.\begin{enumerate}[label=\Alph*.]
\item \( f(16) = 8 \)


\item \( f(8) = 16 \)


\item \( f(16) \text{ is close to or exactly } 8 \)


\item \( f(8) \text{ is close to or exactly } 16 \)


\item \( \text{None of the above are always true.} \)


\end{enumerate}

\textbf{General Comment:} The limit tells you what happens as the $x$-values approach $8$. It says \textbf{absolutely nothing} about what is happening exactly at $f(8)$!
}
\litem{
For the graph below, find the value(s) $a$ that makes the statement true: $ \displaystyle \lim_{x \rightarrow a} f(x)$ does not exist.

\begin{center}
    \includegraphics[width=0.5\textwidth]{../Figures/evaluateLimitGraphicallyC.png}
\end{center}


The solution is \( 1 \), which is option C.\begin{enumerate}[label=\Alph*.]
\item \( -2 \)


\item \( 3 \)


\item \( 1 \)


\item \( \text{Multiple } a \text{ make the statement true}. \)


\item \( \text{No } a \text{ make the statement true}. \)


\end{enumerate}

\textbf{General Comment:} \textbf{General Comments:} Remember that the limit does not exist if the left-hand and right-hand limits do not match.
}
\litem{
Evaluate the one-sided limit of the function $f(x)$ below, if possible.
\[ \lim_{x \rightarrow -7^+} \frac{-2}{(x-7)^9}+8 \]The solution is \( f(-7) \), which is option B.\begin{enumerate}[label=\Alph*.]
\item \( \infty \)


\item \( f(-7) \)


\item \( -\infty \)


\item \( \text{The limit does not exist} \)


\item \( \text{None of the above} \)


\end{enumerate}

\textbf{General Comment:} \textbf{General comments:} You should be able to graph the rational function displayed. If not, go back to Module 7 to learn about the general shape of rational functions.
}
\litem{
To estimate the one-sided limit of the function below as $x$ approaches 10 from the left, which of the following sets of numbers should you use?
\[ \frac{\frac{10}{x} - 1}{x - 10} \]The solution is \( \{ 9.9000, 9.9900, 9.9990, 9.9999 \} \), which is option B.\begin{enumerate}[label=\Alph*.]
\item \( \{ 10.0000, 9.9000, 9.9900, 9.9990 \} \)

If we get $\frac{0}{0}$ or $\frac{\infty}{\infty}$, the value 10 doesn't help us estimate the limit.
\item \( \{ 9.9000, 9.9900, 9.9990, 9.9999 \} \)

This is correct!
\item \( \{ 10.0000, 10.1000, 10.0100, 10.0010 \} \)

If we get $\frac{0}{0}$ or $\frac{\infty}{\infty}$, the value 10 doesn't help us estimate the limit.
\item \( \{ 9.9000, 9.9900, 10.0100, 10.1000 \} \)

These values would estimate the limit at the point and not a one-sided limit.
\item \( \{ 10.1000, 10.0100, 10.0010, 10.0001 \} \)

These values would estimate the limit of 10 on the right.
\end{enumerate}

\textbf{General Comment:} \textbf{General Comments:} To evaluate a one-sided limit, we want to put numbers close to the limit. We can't use the limit value itself if it results in $\frac{0}{0}$ or $\frac{\infty}{\infty}$
}
\litem{
Evaluate the limit below, if possible.
\[ \lim_{x \rightarrow 5} \frac{\sqrt{9x - 29} - 4}{6x - 30} \]The solution is \( 0.188 \), which is option C.\begin{enumerate}[label=\Alph*.]
\item \( \infty \)

You likely believed that since the denominator is equal to 0, the limit is infinity.
\item \( 0.021 \)

You likely learned L'Hospital's Rule in a previous course, but misapplied it here.
\item \( 0.188 \)

* This is the correct option.
\item \( 0.125 \)

You likely memorized how to solve the similar homework problem and used the same formula here.
\item \( \text{None of the above} \)

If you got a limit that does not match any of the above, please contact the coordinator.
\end{enumerate}

\textbf{General Comment:} \textbf{General comments:} It is difficult to imagine the graph of this function, so you need to test values close to $x = 5$.
}
\litem{
Based on the information below, which of the following statements is always true?

\begin{center}
    \textit{ $f(x)$ approaches $0.883$ as $x$ approaches $4$. }
\end{center}
The solution is \( f(x) \text{ is close to or exactly } 0.883 \text{ when } x \text{ is close to } 4 \), which is option A.\begin{enumerate}[label=\Alph*.]
\item \( f(x) \text{ is close to or exactly } 0.883 \text{ when } x \text{ is close to } 4 \)


\item \( f(x) \text{ is close to or exactly } 4 \text{ when } x \text{ is close to } 0.883 \)


\item \( f(x) = 0.883 \text{ when } x \text{ is close to } 4 \)


\item \( f(x) = 4 \text{ when } x \text{ is close to } 0.883 \)


\item \( \text{None of the above are always true.} \)


\end{enumerate}

\textbf{General Comment:} The limit tells you what happens as the $x$-values approach $4$. It says \textbf{absolutely nothing} about what is happening exactly at $f(4)$!
}
\litem{
To estimate the one-sided limit of the function below as $x$ approaches 2 from the left, which of the following sets of numbers should you use?
\[ \frac{\frac{2}{x} - 1}{x - 2} \]The solution is \( \{ 1.9000, 1.9900, 1.9990, 1.9999 \} \), which is option A.\begin{enumerate}[label=\Alph*.]
\item \( \{ 1.9000, 1.9900, 1.9990, 1.9999 \} \)

This is correct!
\item \( \{ 2.1000, 2.0100, 2.0010, 2.0001 \} \)

These values would estimate the limit of 2 on the right.
\item \( \{ 2.0000, 1.9000, 1.9900, 1.9990 \} \)

If we get $\frac{0}{0}$ or $\frac{\infty}{\infty}$, the value 2 doesn't help us estimate the limit.
\item \( \{ 2.0000, 2.1000, 2.0100, 2.0010 \} \)

If we get $\frac{0}{0}$ or $\frac{\infty}{\infty}$, the value 2 doesn't help us estimate the limit.
\item \( \{ 1.9000, 1.9900, 2.0100, 2.1000 \} \)

These values would estimate the limit at the point and not a one-sided limit.
\end{enumerate}

\textbf{General Comment:} \textbf{General Comments:} To evaluate a one-sided limit, we want to put numbers close to the limit. We can't use the limit value itself if it results in $\frac{0}{0}$ or $\frac{\infty}{\infty}$
}
\litem{
For the graph below, find the value(s) $a$ that makes the statement true: $ \displaystyle \lim_{x \rightarrow a} f(x)$ does not exist.

\begin{center}
    \includegraphics[width=0.5\textwidth]{../Figures/evaluateLimitGraphicallyCopyC.png}
\end{center}


The solution is \( 1 \), which is option A.\begin{enumerate}[label=\Alph*.]
\item \( 1 \)


\item \( -2 \)


\item \( 3 \)


\item \( \text{Multiple } a \text{ make the statement true}. \)


\item \( \text{No } a \text{ make the statement true}. \)


\end{enumerate}

\textbf{General Comment:} \textbf{General Comments:} Remember that the limit does not exist if the left-hand and right-hand limits do not match.
}
\end{enumerate}

\end{document}