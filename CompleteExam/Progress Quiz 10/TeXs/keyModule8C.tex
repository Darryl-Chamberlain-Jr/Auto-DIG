\documentclass{extbook}[14pt]
\usepackage{multicol, enumerate, enumitem, hyperref, color, soul, setspace, parskip, fancyhdr, amssymb, amsthm, amsmath, latexsym, units, mathtools}
\everymath{\displaystyle}
\usepackage[headsep=0.5cm,headheight=0cm, left=1 in,right= 1 in,top= 1 in,bottom= 1 in]{geometry}
\usepackage{dashrule}  % Package to use the command below to create lines between items
\newcommand{\litem}[1]{\item #1

\rule{\textwidth}{0.4pt}}
\pagestyle{fancy}
\lhead{}
\chead{Answer Key for Progress Quiz 10 Version C}
\rhead{}
\lfoot{5170-5105}
\cfoot{}
\rfoot{Summer C 2021}
\begin{document}
\textbf{This key should allow you to understand why you choose the option you did (beyond just getting a question right or wrong). \href{https://xronos.clas.ufl.edu/mac1105spring2020/courseDescriptionAndMisc/Exams/LearningFromResults}{More instructions on how to use this key can be found here}.}

\textbf{If you have a suggestion to make the keys better, \href{https://forms.gle/CZkbZmPbC9XALEE88}{please fill out the short survey here}.}

\textit{Note: This key is auto-generated and may contain issues and/or errors. The keys are reviewed after each exam to ensure grading is done accurately. If there are issues (like duplicate options), they are noted in the offline gradebook. The keys are a work-in-progress to give students as many resources to improve as possible.}

\rule{\textwidth}{0.4pt}

\begin{enumerate}\litem{
 Solve the equation for $x$ and choose the interval that contains $x$ (if it exists).
\[  22 = \ln{\sqrt[5]{\frac{9}{e^{9x}}}} \]The solution is \( x = -11.978 \), which is option B.\begin{enumerate}[label=\Alph*.]
\item \( x \in [-2, -1] \)

$x = -1.961$, which corresponds to thinking you need to take the natural log of on the left before reducing.
\item \( x \in [-13.7, -10.5] \)

* $x = -11.978$, which is the correct option.
\item \( x \in [-5.5, -3.7] \)

$x = -4.645$, which corresponds to treating any root as a square root.
\item \( \text{There is no Real solution to the equation.} \)

This corresponds to believing you cannot solve the equation.
\item \( \text{None of the above.} \)

This corresponds to making an unexpected error.
\end{enumerate}

\textbf{General Comment:} \textbf{General Comments}: After using the properties of logarithmic functions to break up the right-hand side, use $\ln(e) = 1$ to reduce the question to a linear function to solve. You can put $\ln(9)$ into a calculator if you are having trouble.
}
\litem{
Which of the following intervals describes the Range of the function below?
\[ f(x) = -\log_2{(x-3)}+7 \]The solution is \( (\infty, \infty) \), which is option E.\begin{enumerate}[label=\Alph*.]
\item \( (-\infty, a), a \in [-8, -4] \)

$(-\infty, -7)$, which corresponds to using the using the negative of vertical shift on $(0, \infty)$.
\item \( (-\infty, a), a \in [5, 9] \)

$(-\infty, 7)$, which corresponds to using the vertical shift while the Range is $(-\infty, \infty)$.
\item \( [a, \infty), a \in [0, 4] \)

$[7, \infty)$, which corresponds to using the flipped Domain AND including the endpoint.
\item \( [a, \infty), a \in [-4, -2] \)

$[-3, \infty)$, which corresponds to using the negative of the horizontal shift AND including the endpoint.
\item \( (-\infty, \infty) \)

*This is the correct option.
\end{enumerate}

\textbf{General Comment:} \textbf{General Comments}: The domain of a basic logarithmic function is $(0, \infty)$ and the Range is $(-\infty, \infty)$. We can use shifts when finding the Domain, but the Range will always be all Real numbers.
}
\litem{
Which of the following intervals describes the Domain of the function below?
\[ f(x) = e^{x-8}+2 \]The solution is \( (-\infty, \infty) \), which is option E.\begin{enumerate}[label=\Alph*.]
\item \( [a, \infty), a \in [-4.1, 0.9] \)

$[-2, \infty)$, which corresponds to using the negative vertical shift AND flipping the Range interval AND including the endpoint.
\item \( (-\infty, a], a \in [0.9, 4.1] \)

$(-\infty, 2]$, which corresponds to using the correct vertical shift *if we wanted the Range* AND including the endpoint.
\item \( (a, \infty), a \in [-4.1, 0.9] \)

$(-2, \infty)$, which corresponds to using the negative vertical shift AND flipping the Range interval.
\item \( (-\infty, a), a \in [0.9, 4.1] \)

$(-\infty, 2)$, which corresponds to using the correct vertical shift *if we wanted the Range*.
\item \( (-\infty, \infty) \)

* This is the correct option.
\end{enumerate}

\textbf{General Comment:} \textbf{General Comments}: Domain of a basic exponential function is $(-\infty, \infty)$ while the Range is $(0, \infty)$. We can shift these intervals [and even flip when $a<0$!] to find the new Domain/Range.
}
\litem{
 Solve the equation for $x$ and choose the interval that contains $x$ (if it exists).
\[  11 = \ln{\sqrt[6]{\frac{7}{e^{6x}}}} \]The solution is \( x = -10.676 \), which is option C.\begin{enumerate}[label=\Alph*.]
\item \( x \in [-4.5, -2.9] \)

$x = -3.342$, which corresponds to treating any root as a square root.
\item \( x \in [-3.1, -1.5] \)

$x = -2.722$, which corresponds to thinking you need to take the natural log of on the left before reducing.
\item \( x \in [-12.6, -10.4] \)

* $x = -10.676$, which is the correct option.
\item \( \text{There is no Real solution to the equation.} \)

This corresponds to believing you cannot solve the equation.
\item \( \text{None of the above.} \)

This corresponds to making an unexpected error.
\end{enumerate}

\textbf{General Comment:} \textbf{General Comments}: After using the properties of logarithmic functions to break up the right-hand side, use $\ln(e) = 1$ to reduce the question to a linear function to solve. You can put $\ln(7)$ into a calculator if you are having trouble.
}
\litem{
Solve the equation for $x$ and choose the interval that contains the solution (if it exists).
\[ \log_{4}{(4x+5)}+5 = 2 \]The solution is \( x = -1.246 \), which is option D.\begin{enumerate}[label=\Alph*.]
\item \( x \in [16.2, 20.3] \)

$x = 19.000$, which corresponds to reversing the base and exponent when converting.
\item \( x \in [21.1, 23.6] \)

$x = 21.500$, which corresponds to reversing the base and exponent when converting and reversing the value with $x$.
\item \( x \in [2.4, 3.2] \)

$x = 2.750$, which corresponds to ignoring the vertical shift when converting to exponential form.
\item \( x \in [-2.6, -0.1] \)

* $x = -1.246$, which is the correct option.
\item \( \text{There is no Real solution to the equation.} \)

Corresponds to believing a negative coefficient within the log equation means there is no Real solution.
\end{enumerate}

\textbf{General Comment:} \textbf{General Comments:} First, get the equation in the form $\log_b{(cx+d)} = a$. Then, convert to $b^a = cx+d$ and solve.
}
\litem{
Solve the equation for $x$ and choose the interval that contains the solution (if it exists).
\[ 3^{3x-3} = 16^{2x+3} \]The solution is \( x = -5.163 \), which is option C.\begin{enumerate}[label=\Alph*.]
\item \( x \in [11.61, 12.61] \)

$x = 11.614$, which corresponds to distributing the $\ln(base)$ to the second term of the exponent only.
\item \( x \in [-3.67, -0.67] \)

$x = -2.667$, which corresponds to distributing the $\ln(base)$ to the first term of the exponent only.
\item \( x \in [-8.16, -3.16] \)

* $x = -5.163$, which is the correct option.
\item \( x \in [5, 10] \)

$x = 6.000$, which corresponds to solving the numerators as equal while ignoring the bases are different.
\item \( \text{There is no Real solution to the equation.} \)

This corresponds to believing there is no solution since the bases are not powers of each other.
\end{enumerate}

\textbf{General Comment:} \textbf{General Comments:} This question was written so that the bases could not be written the same. You will need to take the log of both sides.
}
\litem{
Which of the following intervals describes the Domain of the function below?
\[ f(x) = \log_2{(x+7)}+1 \]The solution is \( (-7, \infty) \), which is option A.\begin{enumerate}[label=\Alph*.]
\item \( (a, \infty), a \in [-8.9, -5] \)

* $(-7, \infty)$, which is the correct option.
\item \( (-\infty, a), a \in [5.9, 9.7] \)

$(-\infty, 7)$, which corresponds to flipping the Domain. Remember: the general for is $a*\log(x-h)+k$, \textbf{where $a$ does not affect the domain}.
\item \( (-\infty, a], a \in [-2.2, -0.9] \)

$(-\infty, -1]$, which corresponds to using the negative vertical shift AND including the endpoint AND flipping the domain.
\item \( [a, \infty), a \in [0.3, 2.2] \)

$[1, \infty)$, which corresponds to using the vertical shift when shifting the Domain AND including the endpoint.
\item \( (-\infty, \infty) \)

This corresponds to thinking of the range of the log function (or the domain of the exponential function).
\end{enumerate}

\textbf{General Comment:} \textbf{General Comments}: The domain of a basic logarithmic function is $(0, \infty)$ and the Range is $(-\infty, \infty)$. We can use shifts when finding the Domain, but the Range will always be all Real numbers.
}
\litem{
Which of the following intervals describes the Domain of the function below?
\[ f(x) = -e^{x-6}+2 \]The solution is \( (-\infty, \infty) \), which is option E.\begin{enumerate}[label=\Alph*.]
\item \( [a, \infty), a \in [-6, -1] \)

$[-2, \infty)$, which corresponds to using the negative vertical shift AND flipping the Range interval AND including the endpoint.
\item \( (-\infty, a), a \in [0, 3] \)

$(-\infty, 2)$, which corresponds to using the correct vertical shift *if we wanted the Range*.
\item \( (-\infty, a], a \in [0, 3] \)

$(-\infty, 2]$, which corresponds to using the correct vertical shift *if we wanted the Range* AND including the endpoint.
\item \( (a, \infty), a \in [-6, -1] \)

$(-2, \infty)$, which corresponds to using the negative vertical shift AND flipping the Range interval.
\item \( (-\infty, \infty) \)

* This is the correct option.
\end{enumerate}

\textbf{General Comment:} \textbf{General Comments}: Domain of a basic exponential function is $(-\infty, \infty)$ while the Range is $(0, \infty)$. We can shift these intervals [and even flip when $a<0$!] to find the new Domain/Range.
}
\litem{
Solve the equation for $x$ and choose the interval that contains the solution (if it exists).
\[ 3^{-3x+2} = \left(\frac{1}{343}\right)^{2x-5} \]The solution is \( x = 3.221 \), which is option B.\begin{enumerate}[label=\Alph*.]
\item \( x \in [-1.8, -0.4] \)

$x = -0.835$, which corresponds to distributing the $\ln(base)$ to the first term of the exponent only.
\item \( x \in [3, 5.2] \)

* $x = 3.221$, which is the correct option.
\item \( x \in [0.9, 2.1] \)

$x = 1.400$, which corresponds to solving the numerators as equal while ignoring the bases are different.
\item \( x \in [-6.4, -4.7] \)

$x = -5.398$, which corresponds to distributing the $\ln(base)$ to the second term of the exponent only.
\item \( \text{There is no Real solution to the equation.} \)

This corresponds to believing there is no solution since the bases are not powers of each other.
\end{enumerate}

\textbf{General Comment:} \textbf{General Comments:} This question was written so that the bases could not be written the same. You will need to take the log of both sides.
}
\litem{
Solve the equation for $x$ and choose the interval that contains the solution (if it exists).
\[ \log_{4}{(2x+7)}+4 = 2 \]The solution is \( x = -3.469 \), which is option A.\begin{enumerate}[label=\Alph*.]
\item \( x \in [-6.47, -2.47] \)

* $x = -3.469$, which is the correct option.
\item \( x \in [2.5, 5.5] \)

$x = 4.500$, which corresponds to reversing the base and exponent when converting.
\item \( x \in [2.5, 5.5] \)

$x = 4.500$, which corresponds to ignoring the vertical shift when converting to exponential form.
\item \( x \in [11.5, 12.5] \)

$x = 11.500$, which corresponds to reversing the base and exponent when converting and reversing the value with $x$.
\item \( \text{There is no Real solution to the equation.} \)

Corresponds to believing a negative coefficient within the log equation means there is no Real solution.
\end{enumerate}

\textbf{General Comment:} \textbf{General Comments:} First, get the equation in the form $\log_b{(cx+d)} = a$. Then, convert to $b^a = cx+d$ and solve.
}
\end{enumerate}

\end{document}