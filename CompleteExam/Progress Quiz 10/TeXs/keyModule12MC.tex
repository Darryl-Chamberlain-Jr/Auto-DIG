\documentclass{extbook}[14pt]
\usepackage{multicol, enumerate, enumitem, hyperref, color, soul, setspace, parskip, fancyhdr, amssymb, amsthm, amsmath, latexsym, units, mathtools}
\everymath{\displaystyle}
\usepackage[headsep=0.5cm,headheight=0cm, left=1 in,right= 1 in,top= 1 in,bottom= 1 in]{geometry}
\usepackage{dashrule}  % Package to use the command below to create lines between items
\newcommand{\litem}[1]{\item #1

\rule{\textwidth}{0.4pt}}
\pagestyle{fancy}
\lhead{}
\chead{Answer Key for Progress Quiz 10 Version C}
\rhead{}
\lfoot{5170-5105}
\cfoot{}
\rfoot{Summer C 2021}
\begin{document}
\textbf{This key should allow you to understand why you choose the option you did (beyond just getting a question right or wrong). \href{https://xronos.clas.ufl.edu/mac1105spring2020/courseDescriptionAndMisc/Exams/LearningFromResults}{More instructions on how to use this key can be found here}.}

\textbf{If you have a suggestion to make the keys better, \href{https://forms.gle/CZkbZmPbC9XALEE88}{please fill out the short survey here}.}

\textit{Note: This key is auto-generated and may contain issues and/or errors. The keys are reviewed after each exam to ensure grading is done accurately. If there are issues (like duplicate options), they are noted in the offline gradebook. The keys are a work-in-progress to give students as many resources to improve as possible.}

\rule{\textwidth}{0.4pt}

\begin{enumerate}\litem{
Determine the appropriate model for the graph of points below.

\begin{center}
    \includegraphics[width=0.5\textwidth]{../Figures/identifyModelGraph12C.png}
\end{center}


The solution is \( \text{Linear model} \), which is option A.\begin{enumerate}[label=\Alph*.]
\item \( \text{Linear model} \)

For this to be the correct option, we need to see a mostly straight line of points.
\item \( \text{Logarithmic model} \)

For this to be the correct option, we want a rapid change early, then an extremely slow change later.
\item \( \text{Exponential model} \)

For this to be the correct option, we want an extremely slow change early, then a rapid change later.
\item \( \text{Non-linear Power model} \)

For this to be the correct option, we need to see a polynomial or rational shape.
\item \( \text{None of the above} \)

For this to be the correct option, we want to see no pattern in the points.
\end{enumerate}

\textbf{General Comment:} This question is testing if you can associate the models with their graphical representation. If you are having trouble, go back to the corresponding Core module to learn about the specific function you are having trouble recognizing.
}
\litem{
For the scenario below, use the model for the volume of a cylinder as $V = \pi r^2 h$.

\begin{center}
    \textit{ Pringles wants to add 41 \text{percent} more chips to their cylinder cans and minimize the design change of their cans. They've decided that the best way to minimize the design change is to increase the radius and height by the same percentage. What should this increase be? }
\end{center}
The solution is \( \text{About } 12 \text{ percent} \), which is option C.\begin{enumerate}[label=\Alph*.]
\item \( \text{About } 20 \text{ percent} \)

This corresponds to treating both radius and height as equal contributors and not solving correctly.
\item \( \text{About } 19 \text{ percent} \)

This corresponds to solving correctly but treating both radius and height as equal contributors to the volume.
\item \( \text{About } 12 \text{ percent} \)

* This is the correct option.
\item \( \text{About } 3 \text{ percent} \)

This corresponds to not solving for the increase properly.
\item \( \text{None of the above} \)

If you chose this, please contact the coordinator to discus how you solved the problem.
\end{enumerate}

\textbf{General Comment:} Remember that when plugging the increases of values in, you need to treat it as that percentage above 100. For example, a 5 percent increase means 105 percent.
}
\litem{
Solve the modeling problem below, if possible.

\begin{center}
    \textit{ A new virus is spreading throughout the world. There were initially 4 many cases reported, but the number of confirmed cases has quadrupled every 4 days. How long will it be until there are at least 100000 confirmed cases? }
\end{center}
The solution is \( \text{About } 30 \text{ days} \), which is option D.\begin{enumerate}[label=\Alph*.]
\item \( \text{About } 20 \text{ days} \)

You modeled the situation with $e$ as the base and did not apply the properties of log correctly.
\item \( \text{About } 17 \text{ days} \)

You modeled the situation correctly but did not apply the properties of log correctly.
\item \( \text{About } 41 \text{ days} \)

You modeled the situation with $e$ as the base, but solved correctly otherwise.
\item \( \text{About } 30 \text{ days} \)

* This is the correct option.
\item \( \text{There is not enough information to solve the problem.} \)

If you chose this option, please contact the coordinator to discuss why you think this is the case.
\end{enumerate}

\textbf{General Comment:} Set up the model the same as in Module 11M. Then, plug in 100000 and solve for $d$ in your model.
}
\litem{
Solve the modeling problem below, if possible.

\begin{center}
    \textit{ In CHM2045L, Brittany created a 29 liter 21 percent solution of chemical $\chi$ using two different solution percentages of chemical $\chi$. When she went to write her lab report, she realized she forgot to write the amount of each solution she used! If she remembers she used 12 percent and 34 percent solutions, what was the amount she used of the 12 percent solution? }
\end{center}
The solution is \( 17.14 liters \), which is option A.\begin{enumerate}[label=\Alph*.]
\item \( 17.14 liters \)

*This is the correct option.
\item \( 11.86 liters \)

This is the concentration of 34 percent solution.
\item \( 14.50 liters \)

This would be correct if Brittany used equal parts of each solution.
\item \( 13.53 liters \)

This was a random value. If this was not a guess, contact the coordinator to talk about how you got this value.
\item \( \text{There is not enough information to solve the problem.} \)

You may have chose this if you thought you needed to know how much of the second solution was used in the problem. Remember that the total minus the first solution would give you the second amount used.
\end{enumerate}

\textbf{General Comment:} Build the model exactly as you did in Module 9M. Then, solve for the volume you are looking for.
}
\litem{
Determine the appropriate model for the graph of points below.

\begin{center}
    \includegraphics[width=0.5\textwidth]{../Figures/identifyModelGraph12CopyC.png}
\end{center}


The solution is \( \text{Exponential model} \), which is option B.\begin{enumerate}[label=\Alph*.]
\item \( \text{Linear model} \)

For this to be the correct option, we need to see a mostly straight line of points.
\item \( \text{Exponential model} \)

For this to be the correct option, we want an extremely slow change early, then a rapid change later.
\item \( \text{Non-linear Power model} \)

For this to be the correct option, we need to see a polynomial or rational shape.
\item \( \text{Logarithmic model} \)

For this to be the correct option, we want a rapid change early, then an extremely slow change later.
\item \( \text{None of the above} \)

For this to be the correct option, we want to see no pattern in the points.
\end{enumerate}

\textbf{General Comment:} This question is testing if you can associate the models with their graphical representation. If you are having trouble, go back to the corresponding Core module to learn about the specific function you are having trouble recognizing.
}
\litem{
For the scenario below, model the rate of vibration (cm/s) of the string in terms of the length of the string. Then determine the variation constant $k$ of the model (if possible). The constant should be in terms of cm and s.

\begin{center}
    \textit{ The rate of vibration of a string under constant tension varies based on the type of string and the length of the string. The rate of vibration of string $\omega$ increases as the square length of the string decreases. For example, when string $\omega$ is 2 mm long, the rate of vibration is 27 cm/s. }
\end{center}
The solution is \( k = 1.08 \), which is option C.\begin{enumerate}[label=\Alph*.]
\item \( k = 675.00 \)

This option uses the model $R = kl^{2}$ as if this is a direct variation.
\item \( k = 108.00 \)

This option uses the correct model, $R = \frac{k}{l^{2}}$, but does not convert from mm to cm so that the units match.
\item \( k = 1.08 \)

* This is the correct option, which corresponds to the model $R = \frac{k}{l^{2}}$ AND converts from mm to cm.
\item \( k = 6.75 \)

This option uses the model $R = kl^{2}$ as if this is a direct variation AND does not convert from mm to cm so that the units match.
\item \( \text{None of the above.} \)

Talk with the coordinator if you chose this option.
\end{enumerate}

\textbf{General Comment:} The most common mistake on this question is to not convert mm to cm! When modeling, you need to make sure all of the units for your variables are compatible.
}
\litem{
Solve the modeling problem below, if possible.

\begin{center}
    \textit{ A new virus is spreading throughout the world. There were initially 3 many cases reported, but the number of confirmed cases has tripled every 5 days. How long will it be until there are at least 100000 confirmed cases? }
\end{center}
The solution is \( \text{About } 48 \text{ days} \), which is option B.\begin{enumerate}[label=\Alph*.]
\item \( \text{About } 28 \text{ days} \)

You modeled the situation with $e$ as the base and did not apply the properties of log correctly.
\item \( \text{About } 48 \text{ days} \)

* This is the correct option.
\item \( \text{About } 53 \text{ days} \)

You modeled the situation with $e$ as the base, but solved correctly otherwise.
\item \( \text{About } 27 \text{ days} \)

You modeled the situation correctly but did not apply the properties of log correctly.
\item \( \text{There is not enough information to solve the problem.} \)

If you chose this option, please contact the coordinator to discuss why you think this is the case.
\end{enumerate}

\textbf{General Comment:} Set up the model the same as in Module 11M. Then, plug in 100000 and solve for $d$ in your model.
}
\litem{
For the scenario below, find the variation constant $k$ of the model (if possible).

\begin{center}
    \textit{ In an alternative galaxy, the square of the time, $T$ (Earth years), required for a planet to orbit Sun $\chi$ increases as the cube of the distance, $d$ (AUs), that the planet is from Sun $\chi$ increases. For example, when Ea's average distance from Sun $\chi$ is 2, it takes 70 Earth days to complete an orbit. }
\end{center}
The solution is \( k = 612.500 \), which is option D.\begin{enumerate}[label=\Alph*.]
\item \( k = 39200.000 \)

This corresponds to the model $T^{2} = \frac{k}{d^{3}}$.
\item \( k = 6.641 \)

This corresponds to the model $T^{1/2} = k d^{1/3}$.
\item \( k = 4.028 \)

This copies the constant used in the homework.
\item \( k = 612.500 \)

* This is the correct option corresponding to the model $T^{2} = k d^{3}$.
\item \( \text{Unable to compute the constant based on the information given.} \)

This corresponds to believing you cannot determine the type of model from the information given.
\end{enumerate}

\textbf{General Comment:} Since $T$ increases proportionally as $d$ increases, we know this is a direct variation model.
}
\litem{
For the scenario below, use the model for the volume of a cylinder as $V = \pi r^2 h$.

\begin{center}
    \textit{ Pringles wants to add 36 \text{percent} more chips to their cylinder cans and minimize the design change of their cans. They've decided that the best way to minimize the design change is to increase the radius and height by the same percentage. What should this increase be? }
\end{center}
The solution is \( \text{About } 11 \text{ percent} \), which is option A.\begin{enumerate}[label=\Alph*.]
\item \( \text{About } 11 \text{ percent} \)

* This is the correct option.
\item \( \text{About } 18 \text{ percent} \)

This corresponds to treating both radius and height as equal contributors and not solving correctly.
\item \( \text{About } 3 \text{ percent} \)

This corresponds to not solving for the increase properly.
\item \( \text{About } 17 \text{ percent} \)

This corresponds to solving correctly but treating both radius and height as equal contributors to the volume.
\item \( \text{None of the above} \)

If you chose this, please contact the coordinator to discus how you solved the problem.
\end{enumerate}

\textbf{General Comment:} Remember that when plugging the increases of values in, you need to treat it as that percentage above 100. For example, a 5 percent increase means 105 percent.
}
\litem{
Solve the modeling problem below, if possible.

\begin{center}
    \textit{ In CHM2045L, Brittany created a 19 liter 15 percent solution of chemical $\chi$ using two different solution percentages of chemical $\chi$. When she went to write her lab report, she realized she forgot to write the amount of each solution she used! If she remembers she used 7 percent and 19 percent solutions, what was the amount she used of the 7 percent solution? }
\end{center}
The solution is \( 6.33 liters \), which is option D.\begin{enumerate}[label=\Alph*.]
\item \( 8.99 liters \)

This was a random value. If this was not a guess, contact the coordinator to talk about how you got this value.
\item \( 12.67 liters \)

This is the concentration of 19 percent solution.
\item \( 9.50 liters \)

This would be correct if Brittany used equal parts of each solution.
\item \( 6.33 liters \)

*This is the correct option.
\item \( \text{There is not enough information to solve the problem.} \)

You may have chose this if you thought you needed to know how much of the second solution was used in the problem. Remember that the total minus the first solution would give you the second amount used.
\end{enumerate}

\textbf{General Comment:} Build the model exactly as you did in Module 9M. Then, solve for the volume you are looking for.
}
\end{enumerate}

\end{document}