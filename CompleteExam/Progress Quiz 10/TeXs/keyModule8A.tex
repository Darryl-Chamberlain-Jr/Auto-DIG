\documentclass{extbook}[14pt]
\usepackage{multicol, enumerate, enumitem, hyperref, color, soul, setspace, parskip, fancyhdr, amssymb, amsthm, amsmath, bbm, latexsym, units, mathtools}
\everymath{\displaystyle}
\usepackage[headsep=0.5cm,headheight=0cm, left=1 in,right= 1 in,top= 1 in,bottom= 1 in]{geometry}
\usepackage{dashrule}  % Package to use the command below to create lines between items
\newcommand{\litem}[1]{\item #1

\rule{\textwidth}{0.4pt}}
\pagestyle{fancy}
\lhead{}
\chead{Answer Key for Progress Quiz 10 Version A}
\rhead{}
\lfoot{6232-9639}
\cfoot{}
\rfoot{Fall 2020}
\begin{document}
\textbf{This key should allow you to understand why you choose the option you did (beyond just getting a question right or wrong). \href{https://xronos.clas.ufl.edu/mac1105spring2020/courseDescriptionAndMisc/Exams/LearningFromResults}{More instructions on how to use this key can be found here}.}

\textbf{If you have a suggestion to make the keys better, \href{https://forms.gle/CZkbZmPbC9XALEE88}{please fill out the short survey here}.}

\textit{Note: This key is auto-generated and may contain issues and/or errors. The keys are reviewed after each exam to ensure grading is done accurately. If there are issues (like duplicate options), they are noted in the offline gradebook. The keys are a work-in-progress to give students as many resources to improve as possible.}

\rule{\textwidth}{0.4pt}

\begin{enumerate}\litem{
Which of the following intervals describes the Range of the function below?
\[ f(x) = -\log_2{(x+5)}+9 \]

The solution is \( (\infty, \infty) \), which is option E.\begin{enumerate}[label=\Alph*.]
\item \( [a, \infty), a \in [-5, -4] \)

$[9, \infty)$, which corresponds to using the flipped Domain AND including the endpoint.
\item \( (-\infty, a), a \in [-10, -6] \)

$(-\infty, -9)$, which corresponds to using the using the negative of vertical shift on $(0, \infty)$.
\item \( [a, \infty), a \in [5, 6] \)

$[5, \infty)$, which corresponds to using the negative of the horizontal shift AND including the endpoint.
\item \( (-\infty, a), a \in [8, 13] \)

$(-\infty, 9)$, which corresponds to using the vertical shift while the Range is $(-\infty, \infty)$.
\item \( (-\infty, \infty) \)

*This is the correct option.
\end{enumerate}

\textbf{General Comment:} \textbf{General Comments}: The domain of a basic logarithmic function is $(0, \infty)$ and the Range is $(-\infty, \infty)$. We can use shifts when finding the Domain, but the Range will always be all Real numbers.
}
\litem{
 Solve the equation for $x$ and choose the interval that contains $x$ (if it exists).
\[  8 = \sqrt[6]{\frac{27}{e^{5x}}} \]

The solution is \( x = -1.836 \), which is option A.\begin{enumerate}[label=\Alph*.]
\item \( x \in [-2.7, -1.3] \)

* $x = -1.836$, which is the correct option.
\item \( x \in [-1.2, 0.4] \)

$x = -0.173$, which corresponds to treating any root as a square root.
\item \( x \in [-12.2, -9.4] \)

$x = -10.259$, which corresponds to thinking you don't need to take the natural log of both sides before reducing, as if the equation already had a natural log on the right side.
\item \( \text{There is no Real solution to the equation.} \)

This corresponds to believing you cannot solve the equation.
\item \( \text{None of the above.} \)

This corresponds to making an unexpected error.
\end{enumerate}

\textbf{General Comment:} \textbf{General Comments}: After using the properties of logarithmic functions to break up the right-hand side, use $\ln(e) = 1$ to reduce the question to a linear function to solve. You can put $\ln(27)$ into a calculator if you are having trouble.
}
\litem{
Which of the following intervals describes the Range of the function below?
\[ f(x) = e^{x-7}-8 \]

The solution is \( (-8, \infty) \), which is option D.\begin{enumerate}[label=\Alph*.]
\item \( (-\infty, a], a \in [5, 11] \)

$(-\infty, 8]$, which corresponds to using the negative vertical shift AND flipping the Range interval AND including the endpoint.
\item \( (-\infty, a), a \in [5, 11] \)

$(-\infty, 8)$, which corresponds to using the negative vertical shift AND flipping the Range interval.
\item \( [a, \infty), a \in [-9, -6] \)

$[-8, \infty)$, which corresponds to including the endpoint.
\item \( (a, \infty), a \in [-9, -6] \)

* $(-8, \infty)$, which is the correct option.
\item \( (-\infty, \infty) \)

This corresponds to confusing range of an exponential function with the domain of an exponential function.
\end{enumerate}

\textbf{General Comment:} \textbf{General Comments}: Domain of a basic exponential function is $(-\infty, \infty)$ while the Range is $(0, \infty)$. We can shift these intervals [and even flip when $a<0$!] to find the new Domain/Range.
}
\litem{
Solve the equation for $x$ and choose the interval that contains the solution (if it exists).
\[ 3^{-3x+5} = \left(\frac{1}{343}\right)^{-4x-4} \]

The solution is \( x = -0.670 \), which is option D.\begin{enumerate}[label=\Alph*.]
\item \( x \in [17.71, 18.2] \)

$x = 17.858$, which corresponds to distributing the $\ln(base)$ to the second term of the exponent only.
\item \( x \in [0.13, 0.64] \)

$x = 0.338$, which corresponds to distributing the $\ln(base)$ to the first term of the exponent only.
\item \( x \in [-9.49, -7.92] \)

$x = -9.000$, which corresponds to solving the numerators as equal while ignoring the bases are different.
\item \( x \in [-0.88, 0.05] \)

* $x = -0.670$, which is the correct option.
\item \( \text{There is no Real solution to the equation.} \)

This corresponds to believing there is no solution since the bases are not powers of each other.
\end{enumerate}

\textbf{General Comment:} \textbf{General Comments:} This question was written so that the bases could not be written the same. You will need to take the log of both sides.
}
\litem{
Which of the following intervals describes the Domain of the function below?
\[ f(x) = \log_2{(x-5)}+6 \]

The solution is \( (5, \infty) \), which is option B.\begin{enumerate}[label=\Alph*.]
\item \( [a, \infty), a \in [5.33, 6.35] \)

$[6, \infty)$, which corresponds to using the vertical shift when shifting the Domain AND including the endpoint.
\item \( (a, \infty), a \in [4.83, 5.97] \)

* $(5, \infty)$, which is the correct option.
\item \( (-\infty, a), a \in [-5.24, -4.83] \)

$(-\infty, -5)$, which corresponds to flipping the Domain. Remember: the general for is $a*\log(x-h)+k$, \textbf{where $a$ does not affect the domain}.
\item \( (-\infty, a], a \in [-7.22, -5.58] \)

$(-\infty, -6]$, which corresponds to using the negative vertical shift AND including the endpoint AND flipping the domain.
\item \( (-\infty, \infty) \)

This corresponds to thinking of the range of the log function (or the domain of the exponential function).
\end{enumerate}

\textbf{General Comment:} \textbf{General Comments}: The domain of a basic logarithmic function is $(0, \infty)$ and the Range is $(-\infty, \infty)$. We can use shifts when finding the Domain, but the Range will always be all Real numbers.
}
\litem{
 Solve the equation for $x$ and choose the interval that contains $x$ (if it exists).
\[  24 = \ln{\sqrt[3]{\frac{28}{e^{3x}}}} \]

The solution is \( x = -22.889 \), which is option A.\begin{enumerate}[label=\Alph*.]
\item \( x \in [-23.89, -18.89] \)

* $x = -22.889$, which is the correct option.
\item \( x \in [-18.89, -13.89] \)

$x = -14.889$, which corresponds to treating any root as a square root.
\item \( x \in [-6.29, -2.29] \)

$x = -4.289$, which corresponds to thinking you need to take the natural log of on the left before reducing.
\item \( \text{There is no Real solution to the equation.} \)

This corresponds to believing you cannot solve the equation.
\item \( \text{None of the above.} \)

This corresponds to making an unexpected error.
\end{enumerate}

\textbf{General Comment:} \textbf{General Comments}: After using the properties of logarithmic functions to break up the right-hand side, use $\ln(e) = 1$ to reduce the question to a linear function to solve. You can put $\ln(28)$ into a calculator if you are having trouble.
}
\litem{
Solve the equation for $x$ and choose the interval that contains the solution (if it exists).
\[ \log_{2}{(4x+6)}+6 = 2 \]

The solution is \( x = -1.484 \), which is option B.\begin{enumerate}[label=\Alph*.]
\item \( x \in [1.16, 2.8] \)

$x = 2.500$, which corresponds to reversing the base and exponent when converting.
\item \( x \in [-2.74, -1.15] \)

* $x = -1.484$, which is the correct option.
\item \( x \in [4.15, 6.6] \)

$x = 5.500$, which corresponds to reversing the base and exponent when converting and reversing the value with $x$.
\item \( x \in [-0.58, -0.22] \)

$x = -0.500$, which corresponds to ignoring the vertical shift when converting to exponential form.
\item \( \text{There is no Real solution to the equation.} \)

Corresponds to believing a negative coefficient within the log equation means there is no Real solution.
\end{enumerate}

\textbf{General Comment:} \textbf{General Comments:} First, get the equation in the form $\log_b{(cx+d)} = a$. Then, convert to $b^a = cx+d$ and solve.
}
\litem{
Solve the equation for $x$ and choose the interval that contains the solution (if it exists).
\[ 4^{-5x-2} = 49^{-4x+5} \]

The solution is \( x = 2.574 \), which is option A.\begin{enumerate}[label=\Alph*.]
\item \( x \in [2.1, 5.1] \)

* $x = 2.574$, which is the correct option.
\item \( x \in [-7.9, -6.2] \)

$x = -7.000$, which corresponds to solving the numerators as equal while ignoring the bases are different.
\item \( x \in [-1.4, 1.5] \)

$x = 0.811$, which corresponds to distributing the $\ln(base)$ to the first term of the exponent only.
\item \( x \in [-24.2, -21] \)

$x = -22.232$, which corresponds to distributing the $\ln(base)$ to the second term of the exponent only.
\item \( \text{There is no Real solution to the equation.} \)

This corresponds to believing there is no solution since the bases are not powers of each other.
\end{enumerate}

\textbf{General Comment:} \textbf{General Comments:} This question was written so that the bases could not be written the same. You will need to take the log of both sides.
}
\litem{
Solve the equation for $x$ and choose the interval that contains the solution (if it exists).
\[ \log_{4}{(2x+7)}+4 = 2 \]

The solution is \( x = -3.469 \), which is option A.\begin{enumerate}[label=\Alph*.]
\item \( x \in [-11.47, -2.47] \)

* $x = -3.469$, which is the correct option.
\item \( x \in [3.5, 7.5] \)

$x = 4.500$, which corresponds to reversing the base and exponent when converting.
\item \( x \in [7.5, 12.5] \)

$x = 11.500$, which corresponds to reversing the base and exponent when converting and reversing the value with $x$.
\item \( x \in [3.5, 7.5] \)

$x = 4.500$, which corresponds to ignoring the vertical shift when converting to exponential form.
\item \( \text{There is no Real solution to the equation.} \)

Corresponds to believing a negative coefficient within the log equation means there is no Real solution.
\end{enumerate}

\textbf{General Comment:} \textbf{General Comments:} First, get the equation in the form $\log_b{(cx+d)} = a$. Then, convert to $b^a = cx+d$ and solve.
}
\litem{
Which of the following intervals describes the Domain of the function below?
\[ f(x) = e^{x+9}-5 \]

The solution is \( (-\infty, \infty) \), which is option E.\begin{enumerate}[label=\Alph*.]
\item \( [a, \infty), a \in [4, 8] \)

$[5, \infty)$, which corresponds to using the negative vertical shift AND flipping the Range interval AND including the endpoint.
\item \( (-\infty, a], a \in [-8, -3] \)

$(-\infty, -5]$, which corresponds to using the correct vertical shift *if we wanted the Range* AND including the endpoint.
\item \( (-\infty, a), a \in [-8, -3] \)

$(-\infty, -5)$, which corresponds to using the correct vertical shift *if we wanted the Range*.
\item \( (a, \infty), a \in [4, 8] \)

$(5, \infty)$, which corresponds to using the negative vertical shift AND flipping the Range interval.
\item \( (-\infty, \infty) \)

* This is the correct option.
\end{enumerate}

\textbf{General Comment:} \textbf{General Comments}: Domain of a basic exponential function is $(-\infty, \infty)$ while the Range is $(0, \infty)$. We can shift these intervals [and even flip when $a<0$!] to find the new Domain/Range.
}
\end{enumerate}

\end{document}