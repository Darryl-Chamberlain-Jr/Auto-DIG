\documentclass{extbook}[14pt]
\usepackage{multicol, enumerate, enumitem, hyperref, color, soul, setspace, parskip, fancyhdr, amssymb, amsthm, amsmath, bbm, latexsym, units, mathtools}
\everymath{\displaystyle}
\usepackage[headsep=0.5cm,headheight=0cm, left=1 in,right= 1 in,top= 1 in,bottom= 1 in]{geometry}
\usepackage{dashrule}  % Package to use the command below to create lines between items
\newcommand{\litem}[1]{\item #1

\rule{\textwidth}{0.4pt}}
\pagestyle{fancy}
\lhead{}
\chead{Answer Key for Progress Quiz 10 Version C}
\rhead{}
\lfoot{6232-9639}
\cfoot{}
\rfoot{Fall 2020}
\begin{document}
\textbf{This key should allow you to understand why you choose the option you did (beyond just getting a question right or wrong). \href{https://xronos.clas.ufl.edu/mac1105spring2020/courseDescriptionAndMisc/Exams/LearningFromResults}{More instructions on how to use this key can be found here}.}

\textbf{If you have a suggestion to make the keys better, \href{https://forms.gle/CZkbZmPbC9XALEE88}{please fill out the short survey here}.}

\textit{Note: This key is auto-generated and may contain issues and/or errors. The keys are reviewed after each exam to ensure grading is done accurately. If there are issues (like duplicate options), they are noted in the offline gradebook. The keys are a work-in-progress to give students as many resources to improve as possible.}

\rule{\textwidth}{0.4pt}

\begin{enumerate}\litem{
Using an interval or intervals, describe all the $x$-values within or including a distance of the given values.
\[ \text{ No more than } 2 \text{ units from the number } 3. \]

The solution is \( [1, 5] \), which is option B.\begin{enumerate}[label=\Alph*.]
\item \( (-\infty, 1] \cup [5, \infty) \)

This describes the values no less than 2 from 3
\item \( [1, 5] \)

This describes the values no more than 2 from 3
\item \( (-\infty, 1) \cup (5, \infty) \)

This describes the values more than 2 from 3
\item \( (1, 5) \)

This describes the values less than 2 from 3
\item \( \text{None of the above} \)

You likely thought the values in the interval were not correct.
\end{enumerate}

\textbf{General Comment:} When thinking about this language, it helps to draw a number line and try points.
}
\litem{
Solve the linear inequality below. Then, choose the constant and interval combination that describes the solution set.
\[ -10x + 6 \leq 7x + 8 \]

The solution is \( [-0.118, \infty) \), which is option B.\begin{enumerate}[label=\Alph*.]
\item \( [a, \infty), \text{ where } a \in [-0.03, 0.84] \)

 $[0.118, \infty)$, which corresponds to negating the endpoint of the solution.
\item \( [a, \infty), \text{ where } a \in [-0.23, -0.04] \)

* $[-0.118, \infty)$, which is the correct option.
\item \( (-\infty, a], \text{ where } a \in [0, 0.21] \)

 $(-\infty, 0.118]$, which corresponds to switching the direction of the interval AND negating the endpoint. You likely did this if you did not flip the inequality when dividing by a negative as well as not moving values over to a side properly.
\item \( (-\infty, a], \text{ where } a \in [-0.13, 0.01] \)

 $(-\infty, -0.118]$, which corresponds to switching the direction of the interval. You likely did this if you did not flip the inequality when dividing by a negative!
\item \( \text{None of the above}. \)

You may have chosen this if you thought the inequality did not match the ends of the intervals.
\end{enumerate}

\textbf{General Comment:} Remember that less/greater than or equal to includes the endpoint, while less/greater do not. Also, remember that you need to flip the inequality when you multiply or divide by a negative.
}
\litem{
Solve the linear inequality below. Then, choose the constant and interval combination that describes the solution set.
\[ -7 + 9 x > 10 x \text{ or } 8 + 7 x < 9 x \]

The solution is \( (-\infty, -7.0) \text{ or } (4.0, \infty) \), which is option D.\begin{enumerate}[label=\Alph*.]
\item \( (-\infty, a] \cup [b, \infty), \text{ where } a \in [-9, -6] \text{ and } b \in [3.3, 4.3] \)

Corresponds to including the endpoints (when they should be excluded).
\item \( (-\infty, a] \cup [b, \infty), \text{ where } a \in [-5, -2] \text{ and } b \in [5.1, 10] \)

Corresponds to including the endpoints AND negating.
\item \( (-\infty, a) \cup (b, \infty), \text{ where } a \in [-5, -1] \text{ and } b \in [6, 8] \)

Corresponds to inverting the inequality and negating the solution.
\item \( (-\infty, a) \cup (b, \infty), \text{ where } a \in [-11, -6] \text{ and } b \in [-3, 6] \)

 * Correct option.
\item \( (-\infty, \infty) \)

Corresponds to the variable canceling, which does not happen in this instance.
\end{enumerate}

\textbf{General Comment:} When multiplying or dividing by a negative, flip the sign.
}
\litem{
Solve the linear inequality below. Then, choose the constant and interval combination that describes the solution set.
\[ -9 - 8 x \leq \frac{-54 x + 6}{8} < 9 - 7 x \]

The solution is \( [-7.80, 33.00) \), which is option D.\begin{enumerate}[label=\Alph*.]
\item \( (a, b], \text{ where } a \in [-12.8, -4.8] \text{ and } b \in [33, 36] \)

$(-7.80, 33.00]$, which corresponds to flipping the inequality.
\item \( (-\infty, a) \cup [b, \infty), \text{ where } a \in [-8.8, -3.8] \text{ and } b \in [31, 37] \)

$(-\infty, -7.80) \cup [33.00, \infty)$, which corresponds to displaying the and-inequality as an or-inequality AND flipping the inequality.
\item \( (-\infty, a] \cup (b, \infty), \text{ where } a \in [-11.8, -1.8] \text{ and } b \in [33, 35] \)

$(-\infty, -7.80] \cup (33.00, \infty)$, which corresponds to displaying the and-inequality as an or-inequality.
\item \( [a, b), \text{ where } a \in [-7.8, -3.8] \text{ and } b \in [33, 39] \)

$[-7.80, 33.00)$, which is the correct option.
\item \( \text{None of the above.} \)


\end{enumerate}

\textbf{General Comment:} To solve, you will need to break up the compound inequality into two inequalities. Be sure to keep track of the inequality! It may be best to draw a number line and graph your solution.
}
\litem{
Solve the linear inequality below. Then, choose the constant and interval combination that describes the solution set.
\[ -4 + 8 x > 9 x \text{ or } 3 + 6 x < 7 x \]

The solution is \( (-\infty, -4.0) \text{ or } (3.0, \infty) \), which is option C.\begin{enumerate}[label=\Alph*.]
\item \( (-\infty, a] \cup [b, \infty), \text{ where } a \in [-3.34, -2.57] \text{ and } b \in [3.8, 4.5] \)

Corresponds to including the endpoints AND negating.
\item \( (-\infty, a) \cup (b, \infty), \text{ where } a \in [-3.5, -1.2] \text{ and } b \in [3.31, 4.55] \)

Corresponds to inverting the inequality and negating the solution.
\item \( (-\infty, a) \cup (b, \infty), \text{ where } a \in [-5.5, -3.6] \text{ and } b \in [2.76, 3.25] \)

 * Correct option.
\item \( (-\infty, a] \cup [b, \infty), \text{ where } a \in [-4.31, -3.55] \text{ and } b \in [1.2, 3.5] \)

Corresponds to including the endpoints (when they should be excluded).
\item \( (-\infty, \infty) \)

Corresponds to the variable canceling, which does not happen in this instance.
\end{enumerate}

\textbf{General Comment:} When multiplying or dividing by a negative, flip the sign.
}
\litem{
Using an interval or intervals, describe all the $x$-values within or including a distance of the given values.
\[ \text{ More than } 3 \text{ units from the number } -9. \]

The solution is \( (-\infty, -12) \cup (-6, \infty) \), which is option C.\begin{enumerate}[label=\Alph*.]
\item \( (-12, -6) \)

This describes the values less than 3 from -9
\item \( (-\infty, -12] \cup [-6, \infty) \)

This describes the values no less than 3 from -9
\item \( (-\infty, -12) \cup (-6, \infty) \)

This describes the values more than 3 from -9
\item \( [-12, -6] \)

This describes the values no more than 3 from -9
\item \( \text{None of the above} \)

You likely thought the values in the interval were not correct.
\end{enumerate}

\textbf{General Comment:} When thinking about this language, it helps to draw a number line and try points.
}
\litem{
Solve the linear inequality below. Then, choose the constant and interval combination that describes the solution set.
\[ -6 + 9 x \leq \frac{75 x - 6}{8} < -9 + 4 x \]

The solution is \( [-14.00, -1.53) \), which is option B.\begin{enumerate}[label=\Alph*.]
\item \( (-\infty, a] \cup (b, \infty), \text{ where } a \in [-15, -13] \text{ and } b \in [-2.53, -0.53] \)

$(-\infty, -14.00] \cup (-1.53, \infty)$, which corresponds to displaying the and-inequality as an or-inequality.
\item \( [a, b), \text{ where } a \in [-17, -13] \text{ and } b \in [-1.9, -1.1] \)

$[-14.00, -1.53)$, which is the correct option.
\item \( (-\infty, a) \cup [b, \infty), \text{ where } a \in [-17, -13] \text{ and } b \in [-2.53, 1.47] \)

$(-\infty, -14.00) \cup [-1.53, \infty)$, which corresponds to displaying the and-inequality as an or-inequality AND flipping the inequality.
\item \( (a, b], \text{ where } a \in [-14, -10] \text{ and } b \in [-2.53, 1.47] \)

$(-14.00, -1.53]$, which corresponds to flipping the inequality.
\item \( \text{None of the above.} \)


\end{enumerate}

\textbf{General Comment:} To solve, you will need to break up the compound inequality into two inequalities. Be sure to keep track of the inequality! It may be best to draw a number line and graph your solution.
}
\litem{
Solve the linear inequality below. Then, choose the constant and interval combination that describes the solution set.
\[ \frac{3}{6} - \frac{4}{4} x > \frac{3}{9} x + \frac{10}{5} \]

The solution is \( (-\infty, -1.125) \), which is option C.\begin{enumerate}[label=\Alph*.]
\item \( (a, \infty), \text{ where } a \in [-0.88, 3.12] \)

 $(1.125, \infty)$, which corresponds to switching the direction of the interval AND negating the endpoint. You likely did this if you did not flip the inequality when dividing by a negative as well as not moving values over to a side properly.
\item \( (-\infty, a), \text{ where } a \in [-0.88, 5.12] \)

 $(-\infty, 1.125)$, which corresponds to negating the endpoint of the solution.
\item \( (-\infty, a), \text{ where } a \in [-2.12, -0.12] \)

* $(-\infty, -1.125)$, which is the correct option.
\item \( (a, \infty), \text{ where } a \in [-3.12, 0.88] \)

 $(-1.125, \infty)$, which corresponds to switching the direction of the interval. You likely did this if you did not flip the inequality when dividing by a negative!
\item \( \text{None of the above}. \)

You may have chosen this if you thought the inequality did not match the ends of the intervals.
\end{enumerate}

\textbf{General Comment:} Remember that less/greater than or equal to includes the endpoint, while less/greater do not. Also, remember that you need to flip the inequality when you multiply or divide by a negative.
}
\litem{
Solve the linear inequality below. Then, choose the constant and interval combination that describes the solution set.
\[ \frac{5}{4} - \frac{9}{8} x \leq \frac{-6}{6} x - \frac{5}{2} \]

The solution is \( [30.0, \infty) \), which is option B.\begin{enumerate}[label=\Alph*.]
\item \( (-\infty, a], \text{ where } a \in [29, 31] \)

 $(-\infty, 30.0]$, which corresponds to switching the direction of the interval. You likely did this if you did not flip the inequality when dividing by a negative!
\item \( [a, \infty), \text{ where } a \in [30, 31] \)

* $[30.0, \infty)$, which is the correct option.
\item \( [a, \infty), \text{ where } a \in [-32, -28] \)

 $[-30.0, \infty)$, which corresponds to negating the endpoint of the solution.
\item \( (-\infty, a], \text{ where } a \in [-32, -29] \)

 $(-\infty, -30.0]$, which corresponds to switching the direction of the interval AND negating the endpoint. You likely did this if you did not flip the inequality when dividing by a negative as well as not moving values over to a side properly.
\item \( \text{None of the above}. \)

You may have chosen this if you thought the inequality did not match the ends of the intervals.
\end{enumerate}

\textbf{General Comment:} Remember that less/greater than or equal to includes the endpoint, while less/greater do not. Also, remember that you need to flip the inequality when you multiply or divide by a negative.
}
\litem{
Solve the linear inequality below. Then, choose the constant and interval combination that describes the solution set.
\[ -9x + 10 > -6x -5 \]

The solution is \( (-\infty, 5.0) \), which is option C.\begin{enumerate}[label=\Alph*.]
\item \( (-\infty, a), \text{ where } a \in [-8, -2] \)

 $(-\infty, -5.0)$, which corresponds to negating the endpoint of the solution.
\item \( (a, \infty), \text{ where } a \in [3, 6] \)

 $(5.0, \infty)$, which corresponds to switching the direction of the interval. You likely did this if you did not flip the inequality when dividing by a negative!
\item \( (-\infty, a), \text{ where } a \in [5, 9] \)

* $(-\infty, 5.0)$, which is the correct option.
\item \( (a, \infty), \text{ where } a \in [-12, -4] \)

 $(-5.0, \infty)$, which corresponds to switching the direction of the interval AND negating the endpoint. You likely did this if you did not flip the inequality when dividing by a negative as well as not moving values over to a side properly.
\item \( \text{None of the above}. \)

You may have chosen this if you thought the inequality did not match the ends of the intervals.
\end{enumerate}

\textbf{General Comment:} Remember that less/greater than or equal to includes the endpoint, while less/greater do not. Also, remember that you need to flip the inequality when you multiply or divide by a negative.
}
\end{enumerate}

\end{document}