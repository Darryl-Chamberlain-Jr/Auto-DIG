\documentclass{extbook}[14pt]
\usepackage{multicol, enumerate, enumitem, hyperref, color, soul, setspace, parskip, fancyhdr, amssymb, amsthm, amsmath, latexsym, units, mathtools}
\everymath{\displaystyle}
\usepackage[headsep=0.5cm,headheight=0cm, left=1 in,right= 1 in,top= 1 in,bottom= 1 in]{geometry}
\usepackage{dashrule}  % Package to use the command below to create lines between items
\newcommand{\litem}[1]{\item #1

\rule{\textwidth}{0.4pt}}
\pagestyle{fancy}
\lhead{}
\chead{Answer Key for Progress Quiz 10 Version C}
\rhead{}
\lfoot{5170-5105}
\cfoot{}
\rfoot{Summer C 2021}
\begin{document}
\textbf{This key should allow you to understand why you choose the option you did (beyond just getting a question right or wrong). \href{https://xronos.clas.ufl.edu/mac1105spring2020/courseDescriptionAndMisc/Exams/LearningFromResults}{More instructions on how to use this key can be found here}.}

\textbf{If you have a suggestion to make the keys better, \href{https://forms.gle/CZkbZmPbC9XALEE88}{please fill out the short survey here}.}

\textit{Note: This key is auto-generated and may contain issues and/or errors. The keys are reviewed after each exam to ensure grading is done accurately. If there are issues (like duplicate options), they are noted in the offline gradebook. The keys are a work-in-progress to give students as many resources to improve as possible.}

\rule{\textwidth}{0.4pt}

\begin{enumerate}\litem{
Solve the linear inequality below. Then, choose the constant and interval combination that describes the solution set.
\[ 5 + 6 x < \frac{36 x + 4}{4} \leq 4 + 8 x \]The solution is \( \text{None of the above.} \), which is option E.\begin{enumerate}[label=\Alph*.]
\item \( (-\infty, a) \cup [b, \infty), \text{ where } a \in [-4.5, 0.75] \text{ and } b \in [-5.25, 2.25] \)

$(-\infty, -1.33) \cup [-3.00, \infty)$, which corresponds to displaying the and-inequality as an or-inequality and getting negatives of the actual endpoints.
\item \( (-\infty, a] \cup (b, \infty), \text{ where } a \in [-2.55, 0.38] \text{ and } b \in [-6.75, -1.5] \)

$(-\infty, -1.33] \cup (-3.00, \infty)$, which corresponds to displaying the and-inequality as an or-inequality AND flipping the inequality AND getting negatives of the actual endpoints.
\item \( [a, b), \text{ where } a \in [-1.72, -0.22] \text{ and } b \in [-5.25, 2.25] \)

$[-1.33, -3.00)$, which corresponds to flipping the inequality and getting negatives of the actual endpoints.
\item \( (a, b], \text{ where } a \in [-4.5, -0.75] \text{ and } b \in [-7.5, 0] \)

$(-1.33, -3.00]$, which is the correct interval but negatives of the actual endpoints.
\item \( \text{None of the above.} \)

* This is correct as the answer should be $(1.33, 3.00]$.
\end{enumerate}

\textbf{General Comment:} To solve, you will need to break up the compound inequality into two inequalities. Be sure to keep track of the inequality! It may be best to draw a number line and graph your solution.
}
\litem{
Solve the linear inequality below. Then, choose the constant and interval combination that describes the solution set.
\[ \frac{-5}{4} - \frac{7}{8} x \leq \frac{-4}{5} x + \frac{8}{3} \]The solution is \( [-52.222, \infty) \), which is option C.\begin{enumerate}[label=\Alph*.]
\item \( (-\infty, a], \text{ where } a \in [51.75, 54.75] \)

 $(-\infty, 52.222]$, which corresponds to switching the direction of the interval AND negating the endpoint. You likely did this if you did not flip the inequality when dividing by a negative as well as not moving values over to a side properly.
\item \( [a, \infty), \text{ where } a \in [51, 53.25] \)

 $[52.222, \infty)$, which corresponds to negating the endpoint of the solution.
\item \( [a, \infty), \text{ where } a \in [-54.75, -51.75] \)

* $[-52.222, \infty)$, which is the correct option.
\item \( (-\infty, a], \text{ where } a \in [-54, -46.5] \)

 $(-\infty, -52.222]$, which corresponds to switching the direction of the interval. You likely did this if you did not flip the inequality when dividing by a negative!
\item \( \text{None of the above}. \)

You may have chosen this if you thought the inequality did not match the ends of the intervals.
\end{enumerate}

\textbf{General Comment:} Remember that less/greater than or equal to includes the endpoint, while less/greater do not. Also, remember that you need to flip the inequality when you multiply or divide by a negative.
}
\litem{
Solve the linear inequality below. Then, choose the constant and interval combination that describes the solution set.
\[ -10x + 8 > 6x -5 \]The solution is \( (-\infty, 0.812) \), which is option C.\begin{enumerate}[label=\Alph*.]
\item \( (a, \infty), \text{ where } a \in [-0.3, 1.8] \)

 $(0.812, \infty)$, which corresponds to switching the direction of the interval. You likely did this if you did not flip the inequality when dividing by a negative!
\item \( (a, \infty), \text{ where } a \in [-1.8, -0.1] \)

 $(-0.812, \infty)$, which corresponds to switching the direction of the interval AND negating the endpoint. You likely did this if you did not flip the inequality when dividing by a negative as well as not moving values over to a side properly.
\item \( (-\infty, a), \text{ where } a \in [-0.4, 2.4] \)

* $(-\infty, 0.812)$, which is the correct option.
\item \( (-\infty, a), \text{ where } a \in [-1.5, 0.2] \)

 $(-\infty, -0.812)$, which corresponds to negating the endpoint of the solution.
\item \( \text{None of the above}. \)

You may have chosen this if you thought the inequality did not match the ends of the intervals.
\end{enumerate}

\textbf{General Comment:} Remember that less/greater than or equal to includes the endpoint, while less/greater do not. Also, remember that you need to flip the inequality when you multiply or divide by a negative.
}
\litem{
Solve the linear inequality below. Then, choose the constant and interval combination that describes the solution set.
\[ -10x + 4 \leq 7x + 7 \]The solution is \( [-0.176, \infty) \), which is option B.\begin{enumerate}[label=\Alph*.]
\item \( (-\infty, a], \text{ where } a \in [-0.27, -0.05] \)

 $(-\infty, -0.176]$, which corresponds to switching the direction of the interval. You likely did this if you did not flip the inequality when dividing by a negative!
\item \( [a, \infty), \text{ where } a \in [-0.38, -0] \)

* $[-0.176, \infty)$, which is the correct option.
\item \( [a, \infty), \text{ where } a \in [0.03, 0.36] \)

 $[0.176, \infty)$, which corresponds to negating the endpoint of the solution.
\item \( (-\infty, a], \text{ where } a \in [-0.06, 0.27] \)

 $(-\infty, 0.176]$, which corresponds to switching the direction of the interval AND negating the endpoint. You likely did this if you did not flip the inequality when dividing by a negative as well as not moving values over to a side properly.
\item \( \text{None of the above}. \)

You may have chosen this if you thought the inequality did not match the ends of the intervals.
\end{enumerate}

\textbf{General Comment:} Remember that less/greater than or equal to includes the endpoint, while less/greater do not. Also, remember that you need to flip the inequality when you multiply or divide by a negative.
}
\litem{
Using an interval or intervals, describe all the $x$-values within or including a distance of the given values.
\[ \text{ No more than } 5 \text{ units from the number } -10. \]The solution is \( [-15, -5] \), which is option C.\begin{enumerate}[label=\Alph*.]
\item \( (-15, -5) \)

This describes the values less than 5 from -10
\item \( (-\infty, -15) \cup (-5, \infty) \)

This describes the values more than 5 from -10
\item \( [-15, -5] \)

This describes the values no more than 5 from -10
\item \( (-\infty, -15] \cup [-5, \infty) \)

This describes the values no less than 5 from -10
\item \( \text{None of the above} \)

You likely thought the values in the interval were not correct.
\end{enumerate}

\textbf{General Comment:} When thinking about this language, it helps to draw a number line and try points.
}
\litem{
Solve the linear inequality below. Then, choose the constant and interval combination that describes the solution set.
\[ -4 - 3 x < \frac{-22 x + 6}{9} \leq -7 - 5 x \]The solution is \( \text{None of the above.} \), which is option E.\begin{enumerate}[label=\Alph*.]
\item \( (-\infty, a] \cup (b, \infty), \text{ where } a \in [6, 13.5] \text{ and } b \in [-1.5, 6] \)

$(-\infty, 8.40] \cup (3.00, \infty)$, which corresponds to displaying the and-inequality as an or-inequality AND flipping the inequality AND getting negatives of the actual endpoints.
\item \( [a, b), \text{ where } a \in [3, 9.75] \text{ and } b \in [2.25, 5.25] \)

$[8.40, 3.00)$, which corresponds to flipping the inequality and getting negatives of the actual endpoints.
\item \( (a, b], \text{ where } a \in [3.75, 9.75] \text{ and } b \in [-0.75, 5.25] \)

$(8.40, 3.00]$, which is the correct interval but negatives of the actual endpoints.
\item \( (-\infty, a) \cup [b, \infty), \text{ where } a \in [6.75, 13.5] \text{ and } b \in [2.25, 6] \)

$(-\infty, 8.40) \cup [3.00, \infty)$, which corresponds to displaying the and-inequality as an or-inequality and getting negatives of the actual endpoints.
\item \( \text{None of the above.} \)

* This is correct as the answer should be $(-8.40, -3.00]$.
\end{enumerate}

\textbf{General Comment:} To solve, you will need to break up the compound inequality into two inequalities. Be sure to keep track of the inequality! It may be best to draw a number line and graph your solution.
}
\litem{
Solve the linear inequality below. Then, choose the constant and interval combination that describes the solution set.
\[ -3 + 4 x > 7 x \text{ or } 5 + 9 x < 10 x \]The solution is \( (-\infty, -1.0) \text{ or } (5.0, \infty) \), which is option D.\begin{enumerate}[label=\Alph*.]
\item \( (-\infty, a] \cup [b, \infty), \text{ where } a \in [-2.25, 1.5] \text{ and } b \in [4.5, 6.75] \)

Corresponds to including the endpoints (when they should be excluded).
\item \( (-\infty, a) \cup (b, \infty), \text{ where } a \in [-7.95, -3.23] \text{ and } b \in [-0.75, 4.5] \)

Corresponds to inverting the inequality and negating the solution.
\item \( (-\infty, a] \cup [b, \infty), \text{ where } a \in [-8.25, -3] \text{ and } b \in [-1.5, 3] \)

Corresponds to including the endpoints AND negating.
\item \( (-\infty, a) \cup (b, \infty), \text{ where } a \in [-1.43, 1.43] \text{ and } b \in [3, 10.5] \)

 * Correct option.
\item \( (-\infty, \infty) \)

Corresponds to the variable canceling, which does not happen in this instance.
\end{enumerate}

\textbf{General Comment:} When multiplying or dividing by a negative, flip the sign.
}
\litem{
Solve the linear inequality below. Then, choose the constant and interval combination that describes the solution set.
\[ 5 + 4 x > 7 x \text{ or } 7 + 4 x < 5 x \]The solution is \( (-\infty, 1.667) \text{ or } (7.0, \infty) \), which is option D.\begin{enumerate}[label=\Alph*.]
\item \( (-\infty, a] \cup [b, \infty), \text{ where } a \in [-2.25, 6] \text{ and } b \in [4.5, 9.75] \)

Corresponds to including the endpoints (when they should be excluded).
\item \( (-\infty, a] \cup [b, \infty), \text{ where } a \in [-10.5, -6] \text{ and } b \in [-4.5, 0] \)

Corresponds to including the endpoints AND negating.
\item \( (-\infty, a) \cup (b, \infty), \text{ where } a \in [-9.75, -3] \text{ and } b \in [-5.25, -1.5] \)

Corresponds to inverting the inequality and negating the solution.
\item \( (-\infty, a) \cup (b, \infty), \text{ where } a \in [0.75, 4.5] \text{ and } b \in [3, 10.5] \)

 * Correct option.
\item \( (-\infty, \infty) \)

Corresponds to the variable canceling, which does not happen in this instance.
\end{enumerate}

\textbf{General Comment:} When multiplying or dividing by a negative, flip the sign.
}
\litem{
Solve the linear inequality below. Then, choose the constant and interval combination that describes the solution set.
\[ \frac{8}{2} - \frac{10}{3} x > \frac{5}{6} x - \frac{9}{5} \]The solution is \( (-\infty, 1.392) \), which is option A.\begin{enumerate}[label=\Alph*.]
\item \( (-\infty, a), \text{ where } a \in [-0.75, 3.75] \)

* $(-\infty, 1.392)$, which is the correct option.
\item \( (-\infty, a), \text{ where } a \in [-6.75, 0] \)

 $(-\infty, -1.392)$, which corresponds to negating the endpoint of the solution.
\item \( (a, \infty), \text{ where } a \in [0, 3.75] \)

 $(1.392, \infty)$, which corresponds to switching the direction of the interval. You likely did this if you did not flip the inequality when dividing by a negative!
\item \( (a, \infty), \text{ where } a \in [-2.25, 0] \)

 $(-1.392, \infty)$, which corresponds to switching the direction of the interval AND negating the endpoint. You likely did this if you did not flip the inequality when dividing by a negative as well as not moving values over to a side properly.
\item \( \text{None of the above}. \)

You may have chosen this if you thought the inequality did not match the ends of the intervals.
\end{enumerate}

\textbf{General Comment:} Remember that less/greater than or equal to includes the endpoint, while less/greater do not. Also, remember that you need to flip the inequality when you multiply or divide by a negative.
}
\litem{
Using an interval or intervals, describe all the $x$-values within or including a distance of the given values.
\[ \text{ Less than } 9 \text{ units from the number } 6. \]The solution is \( (-3, 15) \), which is option D.\begin{enumerate}[label=\Alph*.]
\item \( (-\infty, -3) \cup (15, \infty) \)

This describes the values more than 9 from 6
\item \( [-3, 15] \)

This describes the values no more than 9 from 6
\item \( (-\infty, -3] \cup [15, \infty) \)

This describes the values no less than 9 from 6
\item \( (-3, 15) \)

This describes the values less than 9 from 6
\item \( \text{None of the above} \)

You likely thought the values in the interval were not correct.
\end{enumerate}

\textbf{General Comment:} When thinking about this language, it helps to draw a number line and try points.
}
\end{enumerate}

\end{document}