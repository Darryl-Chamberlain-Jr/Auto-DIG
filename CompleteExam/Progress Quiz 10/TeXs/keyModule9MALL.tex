\documentclass{extbook}[14pt]
\usepackage{multicol, enumerate, enumitem, hyperref, color, soul, setspace, parskip, fancyhdr, amssymb, amsthm, amsmath, latexsym, units, mathtools}
\everymath{\displaystyle}
\usepackage[headsep=0.5cm,headheight=0cm, left=1 in,right= 1 in,top= 1 in,bottom= 1 in]{geometry}
\usepackage{dashrule}  % Package to use the command below to create lines between items
\newcommand{\litem}[1]{\item #1

\rule{\textwidth}{0.4pt}}
\pagestyle{fancy}
\lhead{}
\chead{Answer Key for Progress Quiz 10 Version ALL}
\rhead{}
\lfoot{5170-5105}
\cfoot{}
\rfoot{Summer C 2021}
\begin{document}
\textbf{This key should allow you to understand why you choose the option you did (beyond just getting a question right or wrong). \href{https://xronos.clas.ufl.edu/mac1105spring2020/courseDescriptionAndMisc/Exams/LearningFromResults}{More instructions on how to use this key can be found here}.}

\textbf{If you have a suggestion to make the keys better, \href{https://forms.gle/CZkbZmPbC9XALEE88}{please fill out the short survey here}.}

\textit{Note: This key is auto-generated and may contain issues and/or errors. The keys are reviewed after each exam to ensure grading is done accurately. If there are issues (like duplicate options), they are noted in the offline gradebook. The keys are a work-in-progress to give students as many resources to improve as possible.}

\rule{\textwidth}{0.4pt}

\begin{enumerate}\litem{
What is the \textbf{best} way to describe the domain of the scenario below?

\begin{center}
    \textit{ Veronica needs to prepare 170 lbs of blended coffee beans to sell for \$4.71 per pound. She has a high-quality bean that sells for \$6.00 a pound and a low-quality been that sells for \$3.25 a pound. }
\end{center}
The solution is \( \text{Subset of the Rational numbers} \), which is option D.\begin{enumerate}[label=\Alph*.]
\item \( \text{Subset of the Integers} \)

Recall that the Integers are the positive and negative counting numbers: ..., -3, -2, -1, 0, 1, 2, 3, ... 
\item \( \text{Proper subset of the Real numbers} \)

This means we have a domain of the Real numbers but need to throw out values based on the context.
\item \( \text{There is no restricted domain in this scenario} \)

This means we have a domain of the Real numbers and we don't need to remove any values even in the real-world context.
\item \( \text{Subset of the Rational numbers} \)

Recall that the Rationals are fractions with Integers in the numerator and denominator.
\item \( \text{Subset of the Natural numbers} \)

Recall that the Naturals are the counting numbers: 1, 2, 3, ...
\end{enumerate}

\textbf{General Comment:} We often have to remove values in the domain when working with real-world models.
}
\litem{
A town has an initial population of 100000. The town's population for the next 9 years is provided below. Which type of function would be most appropriate to model the town's population?


\begin{tabular}{c|c|c|c|c|c|c|c|c|c}
\textbf{Year} &1 &2 &3 &4 &5 &6 &7 &8 &9\tabularnewline \hline
\textbf{Pop} &100060 &100120 &100240 &100480 &100960 &101920 &103840 &107680 &115360\end{tabular}The solution is \( \text{Exponential} \), which is option D.\begin{enumerate}[label=\Alph*.]
\item \( \text{Logarithmic} \)

This suggests the slowest of growths that we know.
\item \( \text{Linear} \)

This suggests a constant growth. You would be able to add or subtract the same amount year-to-year if this is the correct answer.
\item \( \text{Non-Linear Power} \)

This suggests a growth faster than constant but slower than exponential.
\item \( \text{Exponential} \)

This suggests the fastest of growths that we know.
\item \( \text{None of the above} \)

Please contact the coordinator to discuss why you believe none of the options model the population.
\end{enumerate}

\textbf{General Comment:} We are trying to compare the growth rate of the population. Growth rates can be characterized from slowest to fastest as: logarithmic, indirect, linear, direct, exponential. The best way to approach this is to first compare it to linear (is it linear, faster than linear, or slower than linear)? If faster, is it as fast as exponential? If slower, is it as slow as logarithmic?
}
\litem{
For the information provided below, construct a linear model that describes her total income, $I$, as a function of the number of months, $x$ she is at UF.

\begin{center}
    \textit{ Aubrey is a college student going into her first year at UF. She will receive Bright Futures, which covers her tuition plus a \$800 educational expense each year. Before college, Aubrey saved up \$10000. She knows she will need to pay \$700 in rent a month, \$80 for food a week, and \$56 in other weekly expenses. }
\end{center}
The solution is \( \text{none of the above.} \), which is option E.\begin{enumerate}[label=\Alph*.]
\item \( I(x) = 836 x \)

This treats weekly expenses as monthly expenses rather than multiplying each weekly expense by 4.
\item \( I(x) = 1244 x \)

This describes the monthly costs, not the monthly income.
\item \( I(x) = 836 \)

This treats weekly expenses as month expenses rather than multiplying each weekly expense by 4 AND does not account for these expenses per month.
\item \( I(x) = 1244 \)

This describes the costs as if they are one-time only and not monthly.
\item \( \text{None of the above.} \)

* This is the correct option as the model should be $I(x) = 10800$.
\end{enumerate}

\textbf{General Comment:} This is a Costs, Profit, Revenue question! The most common issues here are: (1) not converting the weekly costs to monthly costs, (2) treating the one-time values like savings and educational expense as happening per month, and (3) not checking that your model is for cost, profit [income], or revenue [budget].
}
\litem{
Using the situation below, construct a linear model that describes the cost of the coffee beans $C(h)$ in terms of the weight of the low-quality coffee beans $h$.

\begin{center}
    \textit{ Veronica needs to prepare 130 of blended coffee beans selling for \$3.49 per pound. She has a high-quality bean that sells for \$4.66 a pound and a low-quality bean that sells for \$2.94 a pound. }
\end{center}
The solution is \( C(h) = -1.72 h + 605.80 \), which is option C.\begin{enumerate}[label=\Alph*.]
\item \( C(h) = 3.80 h \)

This assumes that exactly half of the high- and low- quality beans are mixed to create the blended coffee beans.
\item \( C(h) = 2.94 h \)

This models the cost of the low-quality bean only, not the blended beans.
\item \( C(h) = -1.72 h + 605.80 \)

* This is the correct option since the questions asked you to construct the cost model in terms of the weight of the low-quality bean.
\item \( C(h) = 1.72 h + 382.20 \)

This would be correct if the question asked you to construct the cost model in terms of the weight of the high-quality bean.
\item \( \text{None of the above.} \)

If you chose this option, please talk to the coordinator to discuss why.
\end{enumerate}

\textbf{General Comment:} This is exactly like the chemistry mixture question from the homework! If you are having trouble with this problem, be sure to review the video for building linear models.
}
\litem{
What is the \textbf{best} way to describe the domain of the scenario below?

\begin{center}
    \textit{ Fred is a store manager at Publix. The store normally orders two pallets of water bottles a week and sells 1000 bottles per day. However, a hurricane is coming and Fred expects water bottle sales to increase tenfold for three days, then decrease by half of normal sales for four days. How many more pallets of water bottles should Fred order the week before the hurricane? }
\end{center}
The solution is \( \text{Subset of the Integers} \), which is option C.\begin{enumerate}[label=\Alph*.]
\item \( \text{Proper subset of the Real numbers} \)

This means we have a domain of the Real numbers but need to throw out values based on the context.
\item \( \text{There is no restricted domain in this scenario} \)

This means we have a domain of the Real numbers and we don't need to remove any values even in the real-world context.
\item \( \text{Subset of the Integers} \)

Recall that the Integers are the positive and negative counting numbers: ..., -3, -2, -1, 0, 1, 2, 3, ... 
\item \( \text{Subset of the Rational numbers} \)

Recall that the Rationals are fractions with Integers in the numerator and denominator.
\item \( \text{Subset of the Natural numbers} \)

Recall that the Naturals are the counting numbers: 1, 2, 3, ...
\end{enumerate}

\textbf{General Comment:} We often have to remove values in the domain when working with real-world models.
}
\litem{
A town has an initial population of 50000. The town's population for the next 9 years is provided below. Which type of function would be most appropriate to model the town's population?


\begin{tabular}{c|c|c|c|c|c|c|c|c|c}
\textbf{Year} &1 &2 &3 &4 &5 &6 &7 &8 &9\tabularnewline \hline
\textbf{Pop} &50060 &50120 &50240 &50480 &50960 &51920 &53840 &57680 &65360\end{tabular}The solution is \( \text{Exponential} \), which is option C.\begin{enumerate}[label=\Alph*.]
\item \( \text{Linear} \)

This suggests a constant growth. You would be able to add or subtract the same amount year-to-year if this is the correct answer.
\item \( \text{Non-Linear Power} \)

This suggests a growth faster than constant but slower than exponential.
\item \( \text{Exponential} \)

This suggests the fastest of growths that we know.
\item \( \text{Logarithmic} \)

This suggests the slowest of growths that we know.
\item \( \text{None of the above} \)

Please contact the coordinator to discuss why you believe none of the options model the population.
\end{enumerate}

\textbf{General Comment:} We are trying to compare the growth rate of the population. Growth rates can be characterized from slowest to fastest as: logarithmic, indirect, linear, direct, exponential. The best way to approach this is to first compare it to linear (is it linear, faster than linear, or slower than linear)? If faster, is it as fast as exponential? If slower, is it as slow as logarithmic?
}
\litem{
For the information provided below, construct a linear model that describes the total distance of the path, $D$, in terms of the time spent on a particular path \textit{if we know that the time spent on each path was equal}.

\begin{center}
    \textit{ A bicyclist is training for a race on a hilly path. Their bike keeps track of their speed at any time, but not the distance traveled. Their speed traveling up a hill is 5 mph, 10 mph when traveling down a hill, and 7 mph when traveling along a flat portion. }
\end{center}
The solution is \( 22 t \), which is option A.\begin{enumerate}[label=\Alph*.]
\item \( 22 t \)

* This is the correct option since time spent on each path is equal.
\item \( 0.443 t \)

The coefficient here is calculated as if you were trying to model the time on the total path.
\item \( 350 t \)

The coefficient here is calculated by multiplying the speeds together rather than adding them.
\item \( \text{The model can be found with the information provided, but isn't options 1-3.} \)

Since the time spent on each path was equal, we can treat all time variables as the same variable, $t$.
\item \( \text{The model cannot be found with the information provided.} \)

If you chose this option, please contact the coordinator to discuss why you think we cannot model the situation.
\end{enumerate}

\textbf{General Comment:} Be sure you pay attention to the variable we are writing the model in terms of. To create the model with a single variable, we have to know that variable is the same throughout each path!
}
\litem{
For the information below, construct a linear model that describes the total time $T$ spent on the path in terms of the distance of a particular part of the path \textit{if we know that the time spent on each path was equal}.

\begin{center}
    \textit{ A bicyclist is training for a race on a hilly path. Their bike keeps track of their speed at any time, but not the distance traveled. Their speed traveling up a hill is 4 mph, 9 mph when traveling down a hill, and 5 mph when traveling along a flat portion. }
\end{center}
The solution is \( \text{The model can be found with the information provided, but isn't options 1-3.} \), which is option D.\begin{enumerate}[label=\Alph*.]
\item \( 18.000 D \)

The coefficient here is calculated as if you were trying to model the distance on the total path.
\item \( 180.000 D \)

The coefficient here is calculated by multiplying the distances together rather than adding.
\item \( 0.561 D \)

This would be correct if we knew all parts of the path are equal length.
\item \( \text{The model can be found with the information provided, but isn't options 1-3.} \)

* This is the correct option. Since the time spent on each path was equal, the distance of each path must be different. The model would be $0.250D_u + 0.111D_d + 0.200D_f$, where $D_u$ is distance traveling up the hill, $D_d$ is distance traveling down, and $D_f$ is distance traveling on a flat part.
\item \( \text{The model cannot be found with the information provided.} \)

If you chose this option, please contact the coordinator to discuss why you think we cannot model the situation.
\end{enumerate}

\textbf{General Comment:} Be sure you pay attention to the variable we are writing the model in terms of. To create the model with a single variable, we have to know that variable is the same throughout each path!
}
\litem{
Using the situation below, construct a linear model that describes the cost of the coffee beans $C(h)$ in terms of the weight of the high-quality coffee beans $h$.

\begin{center}
    \textit{ Veronica needs to prepare 180 of blended coffee beans selling for \$4.11 per pound. She has a high-quality bean that sells for \$4.63 a pound and a low-quality bean that sells for \$3.36 a pound. }
\end{center}
The solution is \( C(h) = 1.27 h + 604.80 \), which is option D.\begin{enumerate}[label=\Alph*.]
\item \( C(h) = 4.63 h \)

This models the cost of the high-quality bean only, not the blended beans.
\item \( C(h) = -1.27 h + 833.40 \)

This would be correct if the question asked you to construct the cost model in terms of the weight of the low-quality bean.
\item \( C(h) = 4.00 h \)

This assumes that exactly half of the high- and low- quality beans are mixed to create the blended coffee beans.
\item \( C(h) = 1.27 h + 604.80 \)

* This is the correct option since the questions asked you to construct the cost model in terms of the weight of the high-quality bean.
\item \( \text{None of the above.} \)

If you chose this option, please talk to the coordinator to discuss why.
\end{enumerate}

\textbf{General Comment:} This is exactly like the chemistry mixture question from the homework! If you are having trouble with this problem, be sure to review the video for building linear models.
}
\litem{
For the information provided below, construct a linear model that describes her total costs, $C$, as a function of the number of months, $x$ she is at UF. 

\begin{center}
    \textit{ Aubrey is a college student going into her first year at UF. She will receive Bright Futures, which covers her tuition plus a \$800 educational expense each year. Before college, Aubrey saved up \$11000. She knows she will need to pay \$900 in rent a month, \$80 for food a week, and \$56 in other weekly expenses. }
\end{center}
The solution is \( C(x) = 1444 x \), which is option A.\begin{enumerate}[label=\Alph*.]
\item \( C(x) = 1444 x \)

* This is the correct option.
\item \( C(x) = 1036 \)

This treats weekly expenses as month expenses rather than multiplying each weekly expense by 4 AND does not account for these expenses per month.
\item \( C(x) = 1036 x \)

This treats weekly expenses as monthly expenses rather than multiplying each weekly expense by 4.
\item \( C(x) = 1444 \)

This describes the costs as if they are one-time only and not monthly.
\item \( \text{None of the above.} \)

You may have chosen this as you thought you were modeling total income or total budget.
\end{enumerate}

\textbf{General Comment:} This is a Costs, Profit, Revenue question! The most common issues here are: (1) not converting the weekly costs to monthly costs, (2) treating the one-time values like savings and educational expense as happening per month, and (3) not checking that your model is for cost, profit [income], or revenue [budget].
}
\litem{
What is the \textbf{best} way to describe the domain of the scenario below?

\begin{center}
    \textit{ Hannah plans to pay off a no-interest loan from her parents. Her loan balance is \$1,000. She plans to pay \$35 at the end of every week until her balance is \$0. How many weeks will it be until she has paid off her loan? }
\end{center}
The solution is \( \text{Subset of the Natural numbers} \), which is option A.\begin{enumerate}[label=\Alph*.]
\item \( \text{Subset of the Natural numbers} \)

Recall that the Naturals are the counting numbers: 1, 2, 3, ...
\item \( \text{Proper subset of the Real numbers} \)

This means we have a domain of the Real numbers but need to throw out values based on the context.
\item \( \text{There is no restricted domain in this scenario} \)

This means we have a domain of the Real numbers and we don't need to remove any values even in the real-world context.
\item \( \text{Subset of the Rational numbers} \)

Recall that the Rationals are fractions with Integers in the numerator and denominator.
\item \( \text{Subset of the Integers} \)

Recall that the Integers are the positive and negative counting numbers: ..., -3, -2, -1, 0, 1, 2, 3, ... 
\end{enumerate}

\textbf{General Comment:} We often have to remove values in the domain when working with real-world models.
}
\litem{
A town has an initial population of 70000. The town's population for the next 9 years is provided below. Which type of function would be most appropriate to model the town's population?


\begin{tabular}{c|c|c|c|c|c|c|c|c|c}
\textbf{Year} &1 &2 &3 &4 &5 &6 &7 &8 &9\tabularnewline \hline
\textbf{Pop} &70027 &70057 &70095 &70125 &70147 &70177 &70215 &70245 &70267\end{tabular}The solution is \( \text{Non-Linear Power} \), which is option A.\begin{enumerate}[label=\Alph*.]
\item \( \text{Non-Linear Power} \)

This suggests a growth faster than constant but slower than exponential.
\item \( \text{Exponential} \)

This suggests the fastest of growths that we know.
\item \( \text{Linear} \)

This suggests a constant growth. You would be able to add or subtract the same amount year-to-year if this is the correct answer.
\item \( \text{Logarithmic} \)

This suggests the slowest of growths that we know.
\item \( \text{None of the above} \)

Please contact the coordinator to discuss why you believe none of the options model the population.
\end{enumerate}

\textbf{General Comment:} We are trying to compare the growth rate of the population. Growth rates can be characterized from slowest to fastest as: logarithmic, indirect, linear, direct, exponential. The best way to approach this is to first compare it to linear (is it linear, faster than linear, or slower than linear)? If faster, is it as fast as exponential? If slower, is it as slow as logarithmic?
}
\litem{
For the information provided below, construct a linear model that describes her total costs, $C$, as a function of the number of months, $x$ she is at UF. 

\begin{center}
    \textit{ Aubrey is a college student going into her first year at UF. She will receive Bright Futures, which covers her tuition plus a \$400 educational expense each year. Before college, Aubrey saved up \$10000. She knows she will need to pay \$700 in rent a month, \$40 for food a week, and \$32 in other weekly expenses. }
\end{center}
The solution is \( \text{None of the above.} \), which is option E.\begin{enumerate}[label=\Alph*.]
\item \( C(x) = 10400 \)

This describes the student's income, not costs.
\item \( C(x) = 772 x \)

This treats weekly expenses as monthly expenses rather than multiplying each weekly expense by 4.
\item \( C(x) = 772 \)

This treats weekly expenses as month expenses rather than multiplying each weekly expense by 4 AND does not account for these expenses per month.
\item \( C(x) = 10400 x \)

This describes the student's income as if they received the savings and educational expense each month.
\item \( \text{None of the above.} \)

* This is the correct option as the model should be $C(x) = 988 x$.
\end{enumerate}

\textbf{General Comment:} This is a Costs, Profit, Revenue question! The most common issues here are: (1) not converting the weekly costs to monthly costs, (2) treating the one-time values like savings and educational expense as happening per month, and (3) not checking that your model is for cost, profit [income], or revenue [budget].
}
\litem{
Using the situation below, construct a linear model that describes the cost of the coffee beans $C(h)$ in terms of the weight of the low-quality coffee beans $h$.

\begin{center}
    \textit{ Veronica needs to prepare 220 of blended coffee beans selling for \$4.85 per pound. She has a high-quality bean that sells for \$5.58 a pound and a low-quality bean that sells for \$4.16 a pound. }
\end{center}
The solution is \( C(h) = -1.42 h + 1227.60 \), which is option A.\begin{enumerate}[label=\Alph*.]
\item \( C(h) = -1.42 h + 1227.60 \)

* This is the correct option since the questions asked you to construct the cost model in terms of the weight of the low-quality bean.
\item \( C(h) = 4.16 h \)

This models the cost of the low-quality bean only, not the blended beans.
\item \( C(h) = 4.87 h \)

This assumes that exactly half of the high- and low- quality beans are mixed to create the blended coffee beans.
\item \( C(h) = 1.42 h + 915.20 \)

This would be correct if the question asked you to construct the cost model in terms of the weight of the high-quality bean.
\item \( \text{None of the above.} \)

If you chose this option, please talk to the coordinator to discuss why.
\end{enumerate}

\textbf{General Comment:} This is exactly like the chemistry mixture question from the homework! If you are having trouble with this problem, be sure to review the video for building linear models.
}
\litem{
What is the \textbf{best} way to describe the domain of the scenario below?

\begin{center}
    \textit{ Fred is a store manager at Publix. The store normally orders two pallets of water bottles a week and sells 1000 bottles per day. However, a hurricane is coming and Fred expects water bottle sales to increase tenfold for three days, then decrease by half of normal sales for four days. How many more pallets of water bottles should Fred order the week before the hurricane? }
\end{center}
The solution is \( \text{Subset of the Integers} \), which is option E.\begin{enumerate}[label=\Alph*.]
\item \( \text{Subset of the Natural numbers} \)

Recall that the Naturals are the counting numbers: 1, 2, 3, ...
\item \( \text{Subset of the Rational numbers} \)

Recall that the Rationals are fractions with Integers in the numerator and denominator.
\item \( \text{Proper subset of the Real numbers} \)

This means we have a domain of the Real numbers but need to throw out values based on the context.
\item \( \text{There is no restricted domain in this scenario} \)

This means we have a domain of the Real numbers and we don't need to remove any values even in the real-world context.
\item \( \text{Subset of the Integers} \)

Recall that the Integers are the positive and negative counting numbers: ..., -3, -2, -1, 0, 1, 2, 3, ... 
\end{enumerate}

\textbf{General Comment:} We often have to remove values in the domain when working with real-world models.
}
\litem{
A town has an initial population of 40000. The town's population for the next 9 years is provided below. Which type of function would be most appropriate to model the town's population?


\begin{tabular}{c|c|c|c|c|c|c|c|c|c}
\textbf{Year} &1 &2 &3 &4 &5 &6 &7 &8 &9\tabularnewline \hline
\textbf{Pop} &40055 &40097 &40147 &40205 &40255 &40297 &40347 &40405 &40455\end{tabular}The solution is \( \text{Non-Linear Power} \), which is option B.\begin{enumerate}[label=\Alph*.]
\item \( \text{Linear} \)

This suggests a constant growth. You would be able to add or subtract the same amount year-to-year if this is the correct answer.
\item \( \text{Non-Linear Power} \)

This suggests a growth faster than constant but slower than exponential.
\item \( \text{Exponential} \)

This suggests the fastest of growths that we know.
\item \( \text{Logarithmic} \)

This suggests the slowest of growths that we know.
\item \( \text{None of the above} \)

Please contact the coordinator to discuss why you believe none of the options model the population.
\end{enumerate}

\textbf{General Comment:} We are trying to compare the growth rate of the population. Growth rates can be characterized from slowest to fastest as: logarithmic, indirect, linear, direct, exponential. The best way to approach this is to first compare it to linear (is it linear, faster than linear, or slower than linear)? If faster, is it as fast as exponential? If slower, is it as slow as logarithmic?
}
\litem{
For the information below, construct a linear model that describes the total time $T$ spent on the path in terms of the distance of a particular part of the path \textit{if we know that all parts of the path are equal length}.

\begin{center}
    \textit{ A bicyclist is training for a race on a hilly path. Their bike keeps track of their speed at any time, but not the distance traveled. Their speed traveling up a hill is 4 mph, 7 mph when traveling down a hill, and 5 mph when traveling along a flat portion. }
\end{center}
The solution is \( 0.593 D \), which is option C.\begin{enumerate}[label=\Alph*.]
\item \( 16.000 D \)

The coefficient here is calculated as if you were trying to model the distance on the total path.
\item \( 140.000 D \)

The coefficient here is calculated by multiplying the distances together rather than adding.
\item \( 0.593 D \)

* This is the correct option.
\item \( \text{The model can be found with the information provided, but isn't options 1-3.} \)

Since we know all parts of the path are equal length, we can treat all distance variables as the same variable, $D$.
\item \( \text{The model cannot be found with the information provided.} \)

If you chose this option, please contact the coordinator to discuss why you think we cannot model the situation.
\end{enumerate}

\textbf{General Comment:} Be sure you pay attention to the variable we are writing the model in terms of. To create the model with a single variable, we have to know that variable is the same throughout each path!
}
\litem{
For the information below, construct a linear model that describes the total time $T$ spent on the path in terms of the distance of a particular part of the path \textit{if we know that the time spent on each path was equal}.

\begin{center}
    \textit{ A bicyclist is training for a race on a hilly path. Their bike keeps track of their speed at any time, but not the distance traveled. Their speed traveling up a hill is 4 mph, 9 mph when traveling down a hill, and 5 mph when traveling along a flat portion. }
\end{center}
The solution is \( \text{The model can be found with the information provided, but isn't options 1-3.} \), which is option D.\begin{enumerate}[label=\Alph*.]
\item \( 18.000 D \)

The coefficient here is calculated as if you were trying to model the distance on the total path.
\item \( 180.000 D \)

The coefficient here is calculated by multiplying the distances together rather than adding.
\item \( 0.561 D \)

This would be correct if we knew all parts of the path are equal length.
\item \( \text{The model can be found with the information provided, but isn't options 1-3.} \)

* This is the correct option. Since the time spent on each path was equal, the distance of each path must be different. The model would be $0.250D_u + 0.111D_d + 0.200D_f$, where $D_u$ is distance traveling up the hill, $D_d$ is distance traveling down, and $D_f$ is distance traveling on a flat part.
\item \( \text{The model cannot be found with the information provided.} \)

If you chose this option, please contact the coordinator to discuss why you think we cannot model the situation.
\end{enumerate}

\textbf{General Comment:} Be sure you pay attention to the variable we are writing the model in terms of. To create the model with a single variable, we have to know that variable is the same throughout each path!
}
\litem{
Using the situation below, construct a linear model that describes the cost of the coffee beans $C(h)$ in terms of the weight of the high-quality coffee beans $h$.

\begin{center}
    \textit{ Veronica needs to prepare 70 of blended coffee beans selling for \$4.19 per pound. She has a high-quality bean that sells for \$5.46 a pound and a low-quality bean that sells for \$3.09 a pound. }
\end{center}
The solution is \( C(h) = 2.37 h + 216.30 \), which is option C.\begin{enumerate}[label=\Alph*.]
\item \( C(h) = 4.28 h \)

This assumes that exactly half of the high- and low- quality beans are mixed to create the blended coffee beans.
\item \( C(h) = -2.37 h + 382.20 \)

This would be correct if the question asked you to construct the cost model in terms of the weight of the low-quality bean.
\item \( C(h) = 2.37 h + 216.30 \)

* This is the correct option since the questions asked you to construct the cost model in terms of the weight of the high-quality bean.
\item \( C(h) = 5.46 h \)

This models the cost of the high-quality bean only, not the blended beans.
\item \( \text{None of the above.} \)

If you chose this option, please talk to the coordinator to discuss why.
\end{enumerate}

\textbf{General Comment:} This is exactly like the chemistry mixture question from the homework! If you are having trouble with this problem, be sure to review the video for building linear models.
}
\litem{
For the information provided below, construct a linear model that describes her total budget, $B$, as a function of the number of months, $x$ she is at UF.

\begin{center}
    \textit{ Aubrey is a college student going into her first year at UF. She will receive Bright Futures, which covers her tuition plus a \$800 educational expense each year. Before college, Aubrey saved up \$8000. She knows she will need to pay \$900 in rent a month, \$70 for food a week, and \$40 in other weekly expenses. }
\end{center}
The solution is \( \text{none of the above.} \), which is option E.\begin{enumerate}[label=\Alph*.]
\item \( B(x) = 8800 - 1340 x \)


\item \( B(x) = 800 x + 8000 \)

This treats the educational expense as something you get every month rather than a 1-time payment and is modeling Income, not Budget.
\item \( B(x) = 8000 x + 800 \)

This treats the savings as something you get every month rather than a 1-time payment and is modeling Income, not Budget.
\item \( B(x) = 8800 - 1010 x \)

This treats weekly expenses as month expenses rather than multiplying each weekly expense.
\item \( \text{None of the above.} \)

* This is the correct option as the model should be $B(x) = 1340 - 8800 x$.
\end{enumerate}

\textbf{General Comment:} This is a Costs, Profit, Revenue question! The most common issues here are: (1) not converting the weekly costs to monthly costs, (2) treating the one-time values like savings and educational expense as happening per month, and (3) not checking that your model is for cost, profit [income], or revenue [budget].
}
\litem{
What is the \textbf{best} way to describe the domain of the scenario below?

\begin{center}
    \textit{ Bridges on highways often have expansion joints, which are small gaps in the roadway between one bridge section and the next. The gaps are put there so the bridge will have room to expand when the weather gets hot. Assume the gap width varies constantly with the temperature. Suppose a bridge has a gap of 1.3 cm when the temperature is 22 degrees C and that the gap narrows to 0.9 cm when the temperature warms to 30 degrees C. }
\end{center}
The solution is \( \text{There is no restricted domain in this scenario} \), which is option D.\begin{enumerate}[label=\Alph*.]
\item \( \text{Subset of the Integers} \)

Recall that the Integers are the positive and negative counting numbers: ..., -3, -2, -1, 0, 1, 2, 3, ... 
\item \( \text{Proper subset of the Real numbers} \)

This means we have a domain of the Real numbers but need to throw out values based on the context.
\item \( \text{Subset of the Rational numbers} \)

Recall that the Rationals are fractions with Integers in the numerator and denominator.
\item \( \text{There is no restricted domain in this scenario} \)

This means we have a domain of the Real numbers and we don't need to remove any values even in the real-world context.
\item \( \text{Subset of the Natural numbers} \)

Recall that the Naturals are the counting numbers: 1, 2, 3, ...
\end{enumerate}

\textbf{General Comment:} We often have to remove values in the domain when working with real-world models.
}
\litem{
A town has an initial population of 80000. The town's population for the next 9 years is provided below. Which type of function would be most appropriate to model the town's population?


\begin{tabular}{c|c|c|c|c|c|c|c|c|c}
\textbf{Year} &1 &2 &3 &4 &5 &6 &7 &8 &9\tabularnewline \hline
\textbf{Pop} &80027 &80057 &80095 &80125 &80147 &80177 &80215 &80245 &80267\end{tabular}The solution is \( \text{Non-Linear Power} \), which is option A.\begin{enumerate}[label=\Alph*.]
\item \( \text{Non-Linear Power} \)

This suggests a growth faster than constant but slower than exponential.
\item \( \text{Logarithmic} \)

This suggests the slowest of growths that we know.
\item \( \text{Exponential} \)

This suggests the fastest of growths that we know.
\item \( \text{Linear} \)

This suggests a constant growth. You would be able to add or subtract the same amount year-to-year if this is the correct answer.
\item \( \text{None of the above} \)

Please contact the coordinator to discuss why you believe none of the options model the population.
\end{enumerate}

\textbf{General Comment:} We are trying to compare the growth rate of the population. Growth rates can be characterized from slowest to fastest as: logarithmic, indirect, linear, direct, exponential. The best way to approach this is to first compare it to linear (is it linear, faster than linear, or slower than linear)? If faster, is it as fast as exponential? If slower, is it as slow as logarithmic?
}
\litem{
For the information provided below, construct a linear model that describes her total income, $I$, as a function of the number of months, $x$ she is at UF.

\begin{center}
    \textit{ Aubrey is a college student going into her first year at UF. She will receive Bright Futures, which covers her tuition plus a \$600 educational expense each year. Before college, Aubrey saved up \$9000. She knows she will need to pay \$1100 in rent a month, \$50 for food a week, and \$32 in other weekly expenses. }
\end{center}
The solution is \( I(x) = 9600 \), which is option D.\begin{enumerate}[label=\Alph*.]
\item \( I(x) = 9600 x \)

This treats the educational expense and savings as something you get every month rather than a 1-time payment.
\item \( I(x) = 9000 x + 600 \)

This treats the savings as something you get every month rather than a 1-time payment.
\item \( I(x) = 600 x + 9000 \)

This treats the educational expense as something you get every month rather than a 1-time payment.
\item \( I(x) = 9600 \)

* This is the correct option.
\item \( \text{None of the above.} \)

You may have chosen this as you thought you were modeling total costs or total budget.
\end{enumerate}

\textbf{General Comment:} This is a Costs, Profit, Revenue question! The most common issues here are: (1) not converting the weekly costs to monthly costs, (2) treating the one-time values like savings and educational expense as happening per month, and (3) not checking that your model is for cost, profit [income], or revenue [budget].
}
\litem{
Using the situation below, construct a linear model that describes the cost of the coffee beans $C(h)$ in terms of the weight of the low-quality coffee beans $h$.

\begin{center}
    \textit{ Veronica needs to prepare 150 of blended coffee beans selling for \$4.76 per pound. She has a high-quality bean that sells for \$5.29 a pound and a low-quality bean that sells for \$2.96 a pound. }
\end{center}
The solution is \( C(h) = -2.33 h + 793.50 \), which is option C.\begin{enumerate}[label=\Alph*.]
\item \( C(h) = 2.96 h \)

This models the cost of the low-quality bean only, not the blended beans.
\item \( C(h) = 4.12 h \)

This assumes that exactly half of the high- and low- quality beans are mixed to create the blended coffee beans.
\item \( C(h) = -2.33 h + 793.50 \)

* This is the correct option since the questions asked you to construct the cost model in terms of the weight of the low-quality bean.
\item \( C(h) = 2.33 h + 444.00 \)

This would be correct if the question asked you to construct the cost model in terms of the weight of the high-quality bean.
\item \( \text{None of the above.} \)

If you chose this option, please talk to the coordinator to discuss why.
\end{enumerate}

\textbf{General Comment:} This is exactly like the chemistry mixture question from the homework! If you are having trouble with this problem, be sure to review the video for building linear models.
}
\litem{
What is the \textbf{best} way to describe the domain of the scenario below?

\begin{center}
    \textit{ Veronica needs to prepare 170 lbs of blended coffee beans to sell for \$4.71 per pound. She has a high-quality bean that sells for \$6.00 a pound and a low-quality been that sells for \$3.25 a pound. }
\end{center}
The solution is \( \text{Subset of the Rational numbers} \), which is option A.\begin{enumerate}[label=\Alph*.]
\item \( \text{Subset of the Rational numbers} \)

Recall that the Rationals are fractions with Integers in the numerator and denominator.
\item \( \text{Proper subset of the Real numbers} \)

This means we have a domain of the Real numbers but need to throw out values based on the context.
\item \( \text{Subset of the Natural numbers} \)

Recall that the Naturals are the counting numbers: 1, 2, 3, ...
\item \( \text{Subset of the Integers} \)

Recall that the Integers are the positive and negative counting numbers: ..., -3, -2, -1, 0, 1, 2, 3, ... 
\item \( \text{There is no restricted domain in this scenario} \)

This means we have a domain of the Real numbers and we don't need to remove any values even in the real-world context.
\end{enumerate}

\textbf{General Comment:} We often have to remove values in the domain when working with real-world models.
}
\litem{
A town has an initial population of 50000. The town's population for the next 9 years is provided below. Which type of function would be most appropriate to model the town's population?


\begin{tabular}{c|c|c|c|c|c|c|c|c|c}
\textbf{Year} &1 &2 &3 &4 &5 &6 &7 &8 &9\tabularnewline \hline
\textbf{Pop} &49950 &49900 &49850 &49800 &49750 &49700 &49650 &49600 &49550\end{tabular}The solution is \( \text{Linear} \), which is option C.\begin{enumerate}[label=\Alph*.]
\item \( \text{Exponential} \)

This suggests the fastest of growths that we know.
\item \( \text{Logarithmic} \)

This suggests the slowest of growths that we know.
\item \( \text{Linear} \)

This suggests a constant growth. You would be able to add or subtract the same amount year-to-year if this is the correct answer.
\item \( \text{Non-Linear Power} \)

This suggests a growth faster than constant but slower than exponential.
\item \( \text{None of the above} \)

Please contact the coordinator to discuss why you believe none of the options model the population.
\end{enumerate}

\textbf{General Comment:} We are trying to compare the growth rate of the population. Growth rates can be characterized from slowest to fastest as: logarithmic, indirect, linear, direct, exponential. The best way to approach this is to first compare it to linear (is it linear, faster than linear, or slower than linear)? If faster, is it as fast as exponential? If slower, is it as slow as logarithmic?
}
\litem{
For the information below, construct a linear model that describes the total time $T$ spent on the path in terms of the distance of a particular part of the path \textit{if we know that the time spent on each path was equal}.

\begin{center}
    \textit{ A bicyclist is training for a race on a hilly path. Their bike keeps track of their speed at any time, but not the distance traveled. Their speed traveling up a hill is 5 mph, 11 mph when traveling down a hill, and 7 mph when traveling along a flat portion. }
\end{center}
The solution is \( \text{The model can be found with the information provided, but isn't options 1-3.} \), which is option D.\begin{enumerate}[label=\Alph*.]
\item \( 0.434 D \)

This would be correct if we knew all parts of the path are equal length.
\item \( 23.000 D \)

The coefficient here is calculated as if you were trying to model the distance on the total path.
\item \( 385.000 D \)

The coefficient here is calculated by multiplying the distances together rather than adding.
\item \( \text{The model can be found with the information provided, but isn't options 1-3.} \)

* This is the correct option. Since the time spent on each path was equal, the distance of each path must be different. The model would be $0.200D_u + 0.091D_d + 0.143D_f$, where $D_u$ is distance traveling up the hill, $D_d$ is distance traveling down, and $D_f$ is distance traveling on a flat part.
\item \( \text{The model cannot be found with the information provided.} \)

If you chose this option, please contact the coordinator to discuss why you think we cannot model the situation.
\end{enumerate}

\textbf{General Comment:} Be sure you pay attention to the variable we are writing the model in terms of. To create the model with a single variable, we have to know that variable is the same throughout each path!
}
\litem{
For the information below, construct a linear model that describes the total time $T$ spent on the path in terms of the distance of a particular part of the path \textit{if we know that the time spent on each path was equal}.

\begin{center}
    \textit{ A bicyclist is training for a race on a hilly path. Their bike keeps track of their speed at any time, but not the distance traveled. Their speed traveling up a hill is 4 mph, 9 mph when traveling down a hill, and 7 mph when traveling along a flat portion. }
\end{center}
The solution is \( \text{The model can be found with the information provided, but isn't options 1-3.} \), which is option D.\begin{enumerate}[label=\Alph*.]
\item \( 252.000 D \)

The coefficient here is calculated by multiplying the distances together rather than adding.
\item \( 0.504 D \)

This would be correct if we knew all parts of the path are equal length.
\item \( 20.000 D \)

The coefficient here is calculated as if you were trying to model the distance on the total path.
\item \( \text{The model can be found with the information provided, but isn't options 1-3.} \)

* This is the correct option. Since the time spent on each path was equal, the distance of each path must be different. The model would be $0.250D_u + 0.111D_d + 0.143D_f$, where $D_u$ is distance traveling up the hill, $D_d$ is distance traveling down, and $D_f$ is distance traveling on a flat part.
\item \( \text{The model cannot be found with the information provided.} \)

If you chose this option, please contact the coordinator to discuss why you think we cannot model the situation.
\end{enumerate}

\textbf{General Comment:} Be sure you pay attention to the variable we are writing the model in terms of. To create the model with a single variable, we have to know that variable is the same throughout each path!
}
\litem{
Using the situation below, construct a linear model that describes the cost of the coffee beans $C(h)$ in terms of the weight of the high-quality coffee beans $h$.

\begin{center}
    \textit{ Veronica needs to prepare 250 of blended coffee beans selling for \$5.05 per pound. She has a high-quality bean that sells for \$5.60 a pound and a low-quality bean that sells for \$3.58 a pound. }
\end{center}
The solution is \( C(h) = 2.02 h + 895.00 \), which is option A.\begin{enumerate}[label=\Alph*.]
\item \( C(h) = 2.02 h + 895.00 \)

* This is the correct option since the questions asked you to construct the cost model in terms of the weight of the high-quality bean.
\item \( C(h) = 4.59 h \)

This assumes that exactly half of the high- and low- quality beans are mixed to create the blended coffee beans.
\item \( C(h) = -2.02 h + 1400.00 \)

This would be correct if the question asked you to construct the cost model in terms of the weight of the low-quality bean.
\item \( C(h) = 5.60 h \)

This models the cost of the high-quality bean only, not the blended beans.
\item \( \text{None of the above.} \)

If you chose this option, please talk to the coordinator to discuss why.
\end{enumerate}

\textbf{General Comment:} This is exactly like the chemistry mixture question from the homework! If you are having trouble with this problem, be sure to review the video for building linear models.
}
\litem{
For the information provided below, construct a linear model that describes her total costs, $C$, as a function of the number of months, $x$ she is at UF. 

\begin{center}
    \textit{ Aubrey is a college student going into her first year at UF. She will receive Bright Futures, which covers her tuition plus a \$400 educational expense each year. Before college, Aubrey saved up \$8000. She knows she will need to pay \$700 in rent a month, \$40 for food a week, and \$56 in other weekly expenses. }
\end{center}
The solution is \( \text{None of the above.} \), which is option E.\begin{enumerate}[label=\Alph*.]
\item \( C(x) = 8400 x \)

This describes the student's income as if they received the savings and educational expense each month.
\item \( C(x) = 8400 \)

This describes the student's income, not costs.
\item \( C(x) = 796 \)

This treats weekly expenses as month expenses rather than multiplying each weekly expense by 4 AND does not account for these expenses per month.
\item \( C(x) = 796 x \)

This treats weekly expenses as monthly expenses rather than multiplying each weekly expense by 4.
\item \( \text{None of the above.} \)

* This is the correct option as the model should be $C(x) = 1084 x$.
\end{enumerate}

\textbf{General Comment:} This is a Costs, Profit, Revenue question! The most common issues here are: (1) not converting the weekly costs to monthly costs, (2) treating the one-time values like savings and educational expense as happening per month, and (3) not checking that your model is for cost, profit [income], or revenue [budget].
}
\end{enumerate}

\end{document}