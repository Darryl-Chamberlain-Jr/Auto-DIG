\documentclass{extbook}[14pt]
\usepackage{multicol, enumerate, enumitem, hyperref, color, soul, setspace, parskip, fancyhdr, amssymb, amsthm, amsmath, latexsym, units, mathtools}
\everymath{\displaystyle}
\usepackage[headsep=0.5cm,headheight=0cm, left=1 in,right= 1 in,top= 1 in,bottom= 1 in]{geometry}
\usepackage{dashrule}  % Package to use the command below to create lines between items
\newcommand{\litem}[1]{\item #1

\rule{\textwidth}{0.4pt}}
\pagestyle{fancy}
\lhead{}
\chead{Answer Key for Progress Quiz 10 Version A}
\rhead{}
\lfoot{5170-5105}
\cfoot{}
\rfoot{Summer C 2021}
\begin{document}
\textbf{This key should allow you to understand why you choose the option you did (beyond just getting a question right or wrong). \href{https://xronos.clas.ufl.edu/mac1105spring2020/courseDescriptionAndMisc/Exams/LearningFromResults}{More instructions on how to use this key can be found here}.}

\textbf{If you have a suggestion to make the keys better, \href{https://forms.gle/CZkbZmPbC9XALEE88}{please fill out the short survey here}.}

\textit{Note: This key is auto-generated and may contain issues and/or errors. The keys are reviewed after each exam to ensure grading is done accurately. If there are issues (like duplicate options), they are noted in the offline gradebook. The keys are a work-in-progress to give students as many resources to improve as possible.}

\rule{\textwidth}{0.4pt}

\begin{enumerate}\litem{
Factor the polynomial below completely. Then, choose the intervals the zeros of the polynomial belong to, where $z_1 \leq z_2 \leq z_3$. \textit{To make the problem easier, all zeros are between -5 and 5.}
\[ f(x) = 8x^{3} -34 x^{2} -39 x + 45 \]The solution is \( [-1.5, 0.75, 5] \), which is option C.\begin{enumerate}[label=\Alph*.]
\item \( z_1 \in [-5.9, -3.5], \text{   }  z_2 \in [-1.82, -1.31], \text{   and   } z_3 \in [0.6, 0.73] \)

 Distractor 3: Corresponds to negatives of all zeros AND inversing rational roots.
\item \( z_1 \in [-1, -0.1], \text{   }  z_2 \in [0.87, 1.6], \text{   and   } z_3 \in [4.82, 5.23] \)

 Distractor 2: Corresponds to inversing rational roots.
\item \( z_1 \in [-2.7, -0.7], \text{   }  z_2 \in [-0.01, 1.07], \text{   and   } z_3 \in [4.82, 5.23] \)

* This is the solution!
\item \( z_1 \in [-5.9, -3.5], \text{   }  z_2 \in [-1.01, -0.49], \text{   and   } z_3 \in [1.5, 1.58] \)

 Distractor 1: Corresponds to negatives of all zeros.
\item \( z_1 \in [-5.9, -3.5], \text{   }  z_2 \in [-3.25, -2.68], \text{   and   } z_3 \in [0.18, 0.44] \)

 Distractor 4: Corresponds to moving factors from one rational to another.
\end{enumerate}

\textbf{General Comment:} Remember to try the middle-most integers first as these normally are the zeros. Also, once you get it to a quadratic, you can use your other factoring techniques to finish factoring.
}
\litem{
What are the \textit{possible Integer} roots of the polynomial below?
\[ f(x) = 4x^{3} +4 x^{2} +5 x + 5 \]The solution is \( \pm 1,\pm 5 \), which is option D.\begin{enumerate}[label=\Alph*.]
\item \( \pm 1,\pm 2,\pm 4 \)

 Distractor 1: Corresponds to the plus or minus factors of a1 only.
\item \( \text{ All combinations of: }\frac{\pm 1,\pm 5}{\pm 1,\pm 2,\pm 4} \)

This would have been the solution \textbf{if asked for the possible Rational roots}!
\item \( \text{ All combinations of: }\frac{\pm 1,\pm 2,\pm 4}{\pm 1,\pm 5} \)

 Distractor 3: Corresponds to the plus or minus of the inverse quotient (an/a0) of the factors. 
\item \( \pm 1,\pm 5 \)

* This is the solution \textbf{since we asked for the possible Integer roots}!
\item \( \text{There is no formula or theorem that tells us all possible Integer roots.} \)

 Distractor 4: Corresponds to not recognizing Integers as a subset of Rationals.
\end{enumerate}

\textbf{General Comment:} We have a way to find the possible Rational roots. The possible Integer roots are the Integers in this list.
}
\litem{
Factor the polynomial below completely. Then, choose the intervals the zeros of the polynomial belong to, where $z_1 \leq z_2 \leq z_3$. \textit{To make the problem easier, all zeros are between -5 and 5.}
\[ f(x) = 9x^{3} +27 x^{2} -82 x + 40 \]The solution is \( [-5, 0.67, 1.33] \), which is option E.\begin{enumerate}[label=\Alph*.]
\item \( z_1 \in [-5.3, -4.79], \text{   }  z_2 \in [0.74, 0.79], \text{   and   } z_3 \in [1.47, 1.57] \)

 Distractor 2: Corresponds to inversing rational roots.
\item \( z_1 \in [-1.7, -1.42], \text{   }  z_2 \in [-0.8, -0.68], \text{   and   } z_3 \in [4.86, 5.23] \)

 Distractor 3: Corresponds to negatives of all zeros AND inversing rational roots.
\item \( z_1 \in [-1.36, -0.85], \text{   }  z_2 \in [-0.67, -0.56], \text{   and   } z_3 \in [4.86, 5.23] \)

 Distractor 1: Corresponds to negatives of all zeros.
\item \( z_1 \in [-4.43, -3.76], \text{   }  z_2 \in [-0.38, -0.13], \text{   and   } z_3 \in [4.86, 5.23] \)

 Distractor 4: Corresponds to moving factors from one rational to another.
\item \( z_1 \in [-5.3, -4.79], \text{   }  z_2 \in [0.65, 0.68], \text{   and   } z_3 \in [1.07, 1.37] \)

* This is the solution!
\end{enumerate}

\textbf{General Comment:} Remember to try the middle-most integers first as these normally are the zeros. Also, once you get it to a quadratic, you can use your other factoring techniques to finish factoring.
}
\litem{
Perform the division below. Then, find the intervals that correspond to the quotient in the form $ax^2+bx+c$ and remainder $r$.
\[ \frac{25x^{3} -105 x^{2} + 83}{x -4} \]The solution is \( 25x^{2} -5 x -20 + \frac{3}{x -4} \), which is option B.\begin{enumerate}[label=\Alph*.]
\item \( a \in [95, 103], b \in [-507, -499], c \in [2020, 2024], \text{ and } r \in [-8000, -7994]. \)

 You divided by the opposite of the factor AND multipled the first factor rather than just bringing it down.
\item \( a \in [22, 30], b \in [-9, -4], c \in [-20, -19], \text{ and } r \in [2, 8]. \)

* This is the solution!
\item \( a \in [22, 30], b \in [-30, -28], c \in [-96, -87], \text{ and } r \in [-189, -184]. \)

 You multipled by the synthetic number and subtracted rather than adding during synthetic division.
\item \( a \in [22, 30], b \in [-207, -201], c \in [817, 822], \text{ and } r \in [-3198, -3194]. \)

 You divided by the opposite of the factor.
\item \( a \in [95, 103], b \in [292, 297], c \in [1179, 1182], \text{ and } r \in [4802, 4808]. \)

 You multipled by the synthetic number rather than bringing the first factor down.
\end{enumerate}

\textbf{General Comment:} Be sure to synthetically divide by the zero of the denominator! Also, make sure to include 0 placeholders for missing terms.
}
\litem{
Perform the division below. Then, find the intervals that correspond to the quotient in the form $ax^2+bx+c$ and remainder $r$.
\[ \frac{10x^{3} +41 x^{2} +51 x + 22}{x + 2} \]The solution is \( 10x^{2} +21 x + 9 + \frac{4}{x + 2} \), which is option B.\begin{enumerate}[label=\Alph*.]
\item \( a \in [-22, -19], \text{   } b \in [79, 86], \text{   } c \in [-112, -106], \text{   and   } r \in [241, 249]. \)

 You multiplied by the synthetic number rather than bringing the first factor down.
\item \( a \in [9, 13], \text{   } b \in [17, 23], \text{   } c \in [6, 10], \text{   and   } r \in [-1, 9]. \)

* This is the solution!
\item \( a \in [9, 13], \text{   } b \in [9, 15], \text{   } c \in [16, 27], \text{   and   } r \in [-36, -29]. \)

 You multiplied by the synthetic number and subtracted rather than adding during synthetic division.
\item \( a \in [9, 13], \text{   } b \in [57, 66], \text{   } c \in [173, 175], \text{   and   } r \in [362, 369]. \)

 You divided by the opposite of the factor.
\item \( a \in [-22, -19], \text{   } b \in [0, 7], \text{   } c \in [52, 56], \text{   and   } r \in [126, 134]. \)

 You divided by the opposite of the factor AND multiplied the first factor rather than just bringing it down.
\end{enumerate}

\textbf{General Comment:} Be sure to synthetically divide by the zero of the denominator!
}
\litem{
Perform the division below. Then, find the intervals that correspond to the quotient in the form $ax^2+bx+c$ and remainder $r$.
\[ \frac{8x^{3} -24 x + 14}{x + 2} \]The solution is \( 8x^{2} -16 x + 8 + \frac{-2}{x + 2} \), which is option C.\begin{enumerate}[label=\Alph*.]
\item \( a \in [6, 14], b \in [-28, -20], c \in [45, 53], \text{ and } r \in [-132, -124]. \)

 You multipled by the synthetic number and subtracted rather than adding during synthetic division.
\item \( a \in [-24, -15], b \in [28, 35], c \in [-88, -87], \text{ and } r \in [182, 199]. \)

 You multipled by the synthetic number rather than bringing the first factor down.
\item \( a \in [6, 14], b \in [-19, -14], c \in [1, 14], \text{ and } r \in [-4, 0]. \)

* This is the solution!
\item \( a \in [6, 14], b \in [16, 23], c \in [1, 14], \text{ and } r \in [30, 36]. \)

 You divided by the opposite of the factor.
\item \( a \in [-24, -15], b \in [-36, -28], c \in [-88, -87], \text{ and } r \in [-166, -160]. \)

 You divided by the opposite of the factor AND multipled the first factor rather than just bringing it down.
\end{enumerate}

\textbf{General Comment:} Be sure to synthetically divide by the zero of the denominator! Also, make sure to include 0 placeholders for missing terms.
}
\litem{
Factor the polynomial below completely, knowing that $x -5$ is a factor. Then, choose the intervals the zeros of the polynomial belong to, where $z_1 \leq z_2 \leq z_3 \leq z_4$. \textit{To make the problem easier, all zeros are between -5 and 5.}
\[ f(x) = 12x^{4} -47 x^{3} -102 x^{2} +155 x + 150 \]The solution is \( [-2, -0.75, 1.667, 5] \), which is option B.\begin{enumerate}[label=\Alph*.]
\item \( z_1 \in [-2, -1], \text{   }  z_2 \in [-1.51, -1.18], z_3 \in [0.49, 0.63], \text{   and   } z_4 \in [4.6, 5.8] \)

 Distractor 2: Corresponds to inversing rational roots.
\item \( z_1 \in [-2, -1], \text{   }  z_2 \in [-0.78, -0.61], z_3 \in [1.5, 1.69], \text{   and   } z_4 \in [4.6, 5.8] \)

* This is the solution!
\item \( z_1 \in [-7, -3], \text{   }  z_2 \in [-0.65, -0.53], z_3 \in [1.28, 1.5], \text{   and   } z_4 \in [1, 2.4] \)

 Distractor 3: Corresponds to negatives of all zeros AND inversing rational roots.
\item \( z_1 \in [-7, -3], \text{   }  z_2 \in [-0.47, -0.33], z_3 \in [1.9, 2.12], \text{   and   } z_4 \in [2.9, 3.6] \)

 Distractor 4: Corresponds to moving factors from one rational to another.
\item \( z_1 \in [-7, -3], \text{   }  z_2 \in [-1.68, -1.5], z_3 \in [0.69, 0.99], \text{   and   } z_4 \in [1, 2.4] \)

 Distractor 1: Corresponds to negatives of all zeros.
\end{enumerate}

\textbf{General Comment:} Remember to try the middle-most integers first as these normally are the zeros. Also, once you get it to a quadratic, you can use your other factoring techniques to finish factoring.
}
\litem{
Factor the polynomial below completely, knowing that $x + 4$ is a factor. Then, choose the intervals the zeros of the polynomial belong to, where $z_1 \leq z_2 \leq z_3 \leq z_4$. \textit{To make the problem easier, all zeros are between -5 and 5.}
\[ f(x) = 10x^{4} + x^{3} -133 x^{2} +122 x + 120 \]The solution is \( [-4, -0.6, 2, 2.5] \), which is option D.\begin{enumerate}[label=\Alph*.]
\item \( z_1 \in [-3.44, -2.2], \text{   }  z_2 \in [-2.14, -1.9], z_3 \in [0.55, 0.83], \text{   and   } z_4 \in [3.48, 4.06] \)

 Distractor 1: Corresponds to negatives of all zeros.
\item \( z_1 \in [-2.11, -1.78], \text{   }  z_2 \in [-0.44, -0.24], z_3 \in [1.31, 1.72], \text{   and   } z_4 \in [3.48, 4.06] \)

 Distractor 3: Corresponds to negatives of all zeros AND inversing rational roots.
\item \( z_1 \in [-2.11, -1.78], \text{   }  z_2 \in [-0.56, -0.46], z_3 \in [2.94, 3.09], \text{   and   } z_4 \in [3.48, 4.06] \)

 Distractor 4: Corresponds to moving factors from one rational to another.
\item \( z_1 \in [-4.38, -3.92], \text{   }  z_2 \in [-0.61, -0.56], z_3 \in [1.78, 2.09], \text{   and   } z_4 \in [2.27, 3.18] \)

* This is the solution!
\item \( z_1 \in [-4.38, -3.92], \text{   }  z_2 \in [-1.79, -1.59], z_3 \in [0.23, 0.47], \text{   and   } z_4 \in [1.69, 2.08] \)

 Distractor 2: Corresponds to inversing rational roots.
\end{enumerate}

\textbf{General Comment:} Remember to try the middle-most integers first as these normally are the zeros. Also, once you get it to a quadratic, you can use your other factoring techniques to finish factoring.
}
\litem{
What are the \textit{possible Integer} roots of the polynomial below?
\[ f(x) = 5x^{3} +4 x^{2} +4 x + 2 \]The solution is \( \pm 1,\pm 2 \), which is option B.\begin{enumerate}[label=\Alph*.]
\item \( \text{ All combinations of: }\frac{\pm 1,\pm 2}{\pm 1,\pm 5} \)

This would have been the solution \textbf{if asked for the possible Rational roots}!
\item \( \pm 1,\pm 2 \)

* This is the solution \textbf{since we asked for the possible Integer roots}!
\item \( \text{ All combinations of: }\frac{\pm 1,\pm 5}{\pm 1,\pm 2} \)

 Distractor 3: Corresponds to the plus or minus of the inverse quotient (an/a0) of the factors. 
\item \( \pm 1,\pm 5 \)

 Distractor 1: Corresponds to the plus or minus factors of a1 only.
\item \( \text{There is no formula or theorem that tells us all possible Integer roots.} \)

 Distractor 4: Corresponds to not recognizing Integers as a subset of Rationals.
\end{enumerate}

\textbf{General Comment:} We have a way to find the possible Rational roots. The possible Integer roots are the Integers in this list.
}
\litem{
Perform the division below. Then, find the intervals that correspond to the quotient in the form $ax^2+bx+c$ and remainder $r$.
\[ \frac{6x^{3} +5 x^{2} -49 x -55}{x -3} \]The solution is \( 6x^{2} +23 x + 20 + \frac{5}{x -3} \), which is option E.\begin{enumerate}[label=\Alph*.]
\item \( a \in [18, 20], \text{   } b \in [57, 63], \text{   } c \in [126, 132], \text{   and   } r \in [328, 334]. \)

 You multiplied by the synthetic number rather than bringing the first factor down.
\item \( a \in [18, 20], \text{   } b \in [-51, -44], \text{   } c \in [95, 102], \text{   and   } r \in [-349, -348]. \)

 You divided by the opposite of the factor AND multiplied the first factor rather than just bringing it down.
\item \( a \in [6, 14], \text{   } b \in [-13, -12], \text{   } c \in [-13, -8], \text{   and   } r \in [-29, -20]. \)

 You divided by the opposite of the factor.
\item \( a \in [6, 14], \text{   } b \in [13, 20], \text{   } c \in [-17, -14], \text{   and   } r \in [-86, -83]. \)

 You multiplied by the synthetic number and subtracted rather than adding during synthetic division.
\item \( a \in [6, 14], \text{   } b \in [21, 24], \text{   } c \in [20, 23], \text{   and   } r \in [2, 10]. \)

* This is the solution!
\end{enumerate}

\textbf{General Comment:} Be sure to synthetically divide by the zero of the denominator!
}
\end{enumerate}

\end{document}