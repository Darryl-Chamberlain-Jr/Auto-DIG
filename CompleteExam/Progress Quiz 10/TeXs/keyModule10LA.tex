\documentclass{extbook}[14pt]
\usepackage{multicol, enumerate, enumitem, hyperref, color, soul, setspace, parskip, fancyhdr, amssymb, amsthm, amsmath, bbm, latexsym, units, mathtools}
\everymath{\displaystyle}
\usepackage[headsep=0.5cm,headheight=0cm, left=1 in,right= 1 in,top= 1 in,bottom= 1 in]{geometry}
\usepackage{dashrule}  % Package to use the command below to create lines between items
\newcommand{\litem}[1]{\item #1

\rule{\textwidth}{0.4pt}}
\pagestyle{fancy}
\lhead{}
\chead{Answer Key for Progress Quiz 10 Version A}
\rhead{}
\lfoot{6232-9639}
\cfoot{}
\rfoot{Fall 2020}
\begin{document}
\textbf{This key should allow you to understand why you choose the option you did (beyond just getting a question right or wrong). \href{https://xronos.clas.ufl.edu/mac1105spring2020/courseDescriptionAndMisc/Exams/LearningFromResults}{More instructions on how to use this key can be found here}.}

\textbf{If you have a suggestion to make the keys better, \href{https://forms.gle/CZkbZmPbC9XALEE88}{please fill out the short survey here}.}

\textit{Note: This key is auto-generated and may contain issues and/or errors. The keys are reviewed after each exam to ensure grading is done accurately. If there are issues (like duplicate options), they are noted in the offline gradebook. The keys are a work-in-progress to give students as many resources to improve as possible.}

\rule{\textwidth}{0.4pt}

\begin{enumerate}\litem{
Perform the division below. Then, find the intervals that correspond to the quotient in the form $ax^2+bx+c$ and remainder $r$.
\[ \frac{9x^{3} -6 x^{2} -51 x -39}{x -3} \]

The solution is \( 9x^{2} +21 x + 12 + \frac{-3}{x -3} \), which is option A.\begin{enumerate}[label=\Alph*.]
\item \( a \in [7, 16], \text{   } b \in [20, 28], \text{   } c \in [7, 14], \text{   and   } r \in [-5, 1]. \)

* This is the solution!
\item \( a \in [7, 16], \text{   } b \in [12, 13], \text{   } c \in [-29, -21], \text{   and   } r \in [-93, -91]. \)

 You multiplied by the synthetic number and subtracted rather than adding during synthetic division.
\item \( a \in [22, 30], \text{   } b \in [74, 79], \text{   } c \in [174, 179], \text{   and   } r \in [482, 484]. \)

 You multiplied by the synthetic number rather than bringing the first factor down.
\item \( a \in [7, 16], \text{   } b \in [-41, -28], \text{   } c \in [47, 55], \text{   and   } r \in [-184, -176]. \)

 You divided by the opposite of the factor.
\item \( a \in [22, 30], \text{   } b \in [-91, -84], \text{   } c \in [209, 214], \text{   and   } r \in [-673, -663]. \)

 You divided by the opposite of the factor AND multiplied the first factor rather than just bringing it down.
\end{enumerate}

\textbf{General Comment:} Be sure to synthetically divide by the zero of the denominator!
}
\litem{
What are the \textit{possible Integer} roots of the polynomial below?
\[ f(x) = 2x^{4} +3 x^{3} +6 x^{2} +5 x + 6 \]

The solution is \( \pm 1,\pm 2,\pm 3,\pm 6 \), which is option B.\begin{enumerate}[label=\Alph*.]
\item \( \text{ All combinations of: }\frac{\pm 1,\pm 2}{\pm 1,\pm 2,\pm 3,\pm 6} \)

 Distractor 3: Corresponds to the plus or minus of the inverse quotient (an/a0) of the factors. 
\item \( \pm 1,\pm 2,\pm 3,\pm 6 \)

* This is the solution \textbf{since we asked for the possible Integer roots}!
\item \( \pm 1,\pm 2 \)

 Distractor 1: Corresponds to the plus or minus factors of a1 only.
\item \( \text{ All combinations of: }\frac{\pm 1,\pm 2,\pm 3,\pm 6}{\pm 1,\pm 2} \)

This would have been the solution \textbf{if asked for the possible Rational roots}!
\item \( \text{There is no formula or theorem that tells us all possible Integer roots.} \)

 Distractor 4: Corresponds to not recognizing Integers as a subset of Rationals.
\end{enumerate}

\textbf{General Comment:} We have a way to find the possible Rational roots. The possible Integer roots are the Integers in this list.
}
\litem{
Factor the polynomial below completely, knowing that $x-2$ is a factor. Then, choose the intervals the zeros of the polynomial belong to, where $z_1 \leq z_2 \leq z_3 \leq z_4$. \textit{To make the problem easier, all zeros are between -5 and 5.}
\[ f(x) = 12x^{4} +35 x^{3} -23 x^{2} -140 x -100 \]

The solution is \( [-2, -1.6666666666666667, -1.25, 2] \), which is option E.\begin{enumerate}[label=\Alph*.]
\item \( z_1 \in [-6, -1], \text{   }  z_2 \in [0.2, 0.57], z_3 \in [1.92, 2.12], \text{   and   } z_4 \in [4, 9] \)

 Distractor 4: Corresponds to moving factors from one rational to another.
\item \( z_1 \in [-6, -1], \text{   }  z_2 \in [1.22, 1.29], z_3 \in [1.54, 1.77], \text{   and   } z_4 \in [0, 4] \)

 Distractor 1: Corresponds to negatives of all zeros.
\item \( z_1 \in [-6, -1], \text{   }  z_2 \in [0.59, 0.96], z_3 \in [0.61, 0.88], \text{   and   } z_4 \in [0, 4] \)

 Distractor 3: Corresponds to negatives of all zeros AND inversing rational roots.
\item \( z_1 \in [-6, -1], \text{   }  z_2 \in [-1.42, -0.64], z_3 \in [-0.94, -0.45], \text{   and   } z_4 \in [0, 4] \)

 Distractor 2: Corresponds to inversing rational roots.
\item \( z_1 \in [-6, -1], \text{   }  z_2 \in [-1.72, -1.33], z_3 \in [-1.28, -1.01], \text{   and   } z_4 \in [0, 4] \)

* This is the solution!
\end{enumerate}

\textbf{General Comment:} Remember to try the middle-most integers first as these normally are the zeros. Also, once you get it to a quadratic, you can use your other factoring techniques to finish factoring.
}
\litem{
Perform the division below. Then, find the intervals that correspond to the quotient in the form $ax^2+bx+c$ and remainder $r$.
\[ \frac{15x^{3} +66 x^{2} +15 x -31}{x + 4} \]

The solution is \( 15x^{2} +6 x -9 + \frac{5}{x + 4} \), which is option C.\begin{enumerate}[label=\Alph*.]
\item \( a \in [-61, -59], \text{   } b \in [302, 308], \text{   } c \in [-1214, -1208], \text{   and   } r \in [4805, 4810]. \)

 You multiplied by the synthetic number rather than bringing the first factor down.
\item \( a \in [14, 17], \text{   } b \in [120, 132], \text{   } c \in [519, 521], \text{   and   } r \in [2043, 2046]. \)

 You divided by the opposite of the factor.
\item \( a \in [14, 17], \text{   } b \in [-1, 8], \text{   } c \in [-10, -4], \text{   and   } r \in [0, 10]. \)

* This is the solution!
\item \( a \in [-61, -59], \text{   } b \in [-174, -173], \text{   } c \in [-681, -677], \text{   and   } r \in [-2762, -2750]. \)

 You divided by the opposite of the factor AND multiplied the first factor rather than just bringing it down.
\item \( a \in [14, 17], \text{   } b \in [-15, -8], \text{   } c \in [60, 62], \text{   and   } r \in [-333, -330]. \)

 You multiplied by the synthetic number and subtracted rather than adding during synthetic division.
\end{enumerate}

\textbf{General Comment:} Be sure to synthetically divide by the zero of the denominator!
}
\litem{
Perform the division below. Then, find the intervals that correspond to the quotient in the form $ax^2+bx+c$ and remainder $r$.
\[ \frac{4x^{3} +12 x^{2} -11}{x + 2} \]

The solution is \( 4x^{2} +4 x -8 + \frac{5}{x + 2} \), which is option A.\begin{enumerate}[label=\Alph*.]
\item \( a \in [2, 7], b \in [3.6, 4.5], c \in [-10, -6], \text{ and } r \in [3, 11]. \)

* This is the solution!
\item \( a \in [2, 7], b \in [-2, 1.1], c \in [-3, 3], \text{ and } r \in [-11, -9]. \)

 You multipled by the synthetic number and subtracted rather than adding during synthetic division.
\item \( a \in [-13, -7], b \in [26.7, 30.3], c \in [-56, -51], \text{ and } r \in [98, 104]. \)

 You multipled by the synthetic number rather than bringing the first factor down.
\item \( a \in [-13, -7], b \in [-7.6, -3.8], c \in [-10, -6], \text{ and } r \in [-27, -25]. \)

 You divided by the opposite of the factor AND multipled the first factor rather than just bringing it down.
\item \( a \in [2, 7], b \in [15.7, 20.2], c \in [39, 41], \text{ and } r \in [68, 73]. \)

 You divided by the opposite of the factor.
\end{enumerate}

\textbf{General Comment:} Be sure to synthetically divide by the zero of the denominator! Also, make sure to include 0 placeholders for missing terms.
}
\litem{
What are the \textit{possible Rational} roots of the polynomial below?
\[ f(x) = 2x^{2} +3 x + 7 \]

The solution is \( \text{ All combinations of: }\frac{\pm 1,\pm 7}{\pm 1,\pm 2} \), which is option C.\begin{enumerate}[label=\Alph*.]
\item \( \pm 1,\pm 7 \)

This would have been the solution \textbf{if asked for the possible Integer roots}!
\item \( \pm 1,\pm 2 \)

 Distractor 1: Corresponds to the plus or minus factors of a1 only.
\item \( \text{ All combinations of: }\frac{\pm 1,\pm 7}{\pm 1,\pm 2} \)

* This is the solution \textbf{since we asked for the possible Rational roots}!
\item \( \text{ All combinations of: }\frac{\pm 1,\pm 2}{\pm 1,\pm 7} \)

 Distractor 3: Corresponds to the plus or minus of the inverse quotient (an/a0) of the factors. 
\item \( \text{ There is no formula or theorem that tells us all possible Rational roots.} \)

 Distractor 4: Corresponds to not recalling the theorem for rational roots of a polynomial.
\end{enumerate}

\textbf{General Comment:} We have a way to find the possible Rational roots. The possible Integer roots are the Integers in this list.
}
\litem{
Perform the division below. Then, find the intervals that correspond to the quotient in the form $ax^2+bx+c$ and remainder $r$.
\[ \frac{12x^{3} +65 x^{2} -122}{x + 5} \]

The solution is \( 12x^{2} +5 x -25 + \frac{3}{x + 5} \), which is option A.\begin{enumerate}[label=\Alph*.]
\item \( a \in [10, 14], b \in [5, 9], c \in [-28, -24], \text{ and } r \in [2, 5]. \)

* This is the solution!
\item \( a \in [-63, -57], b \in [363, 367], c \in [-1829, -1822], \text{ and } r \in [9003, 9008]. \)

 You multipled by the synthetic number rather than bringing the first factor down.
\item \( a \in [10, 14], b \in [115, 126], c \in [624, 629], \text{ and } r \in [2998, 3004]. \)

 You divided by the opposite of the factor.
\item \( a \in [-63, -57], b \in [-238, -228], c \in [-1181, -1172], \text{ and } r \in [-5999, -5996]. \)

 You divided by the opposite of the factor AND multipled the first factor rather than just bringing it down.
\item \( a \in [10, 14], b \in [-9, -6], c \in [40, 45], \text{ and } r \in [-378, -373]. \)

 You multipled by the synthetic number and subtracted rather than adding during synthetic division.
\end{enumerate}

\textbf{General Comment:} Be sure to synthetically divide by the zero of the denominator! Also, make sure to include 0 placeholders for missing terms.
}
\litem{
Factor the polynomial below completely. Then, choose the intervals the zeros of the polynomial belong to, where $z_1 \leq z_2 \leq z_3$. \textit{To make the problem easier, all zeros are between -5 and 5.}
\[ f(x) = 15x^{3} +31 x^{2} -50 x -24 \]

The solution is \( [-3, -0.4, 1.3333333333333333] \), which is option E.\begin{enumerate}[label=\Alph*.]
\item \( z_1 \in [-1.8, -1.17], \text{   }  z_2 \in [0.16, 0.41], \text{   and   } z_3 \in [2.57, 3.15] \)

 Distractor 1: Corresponds to negatives of all zeros.
\item \( z_1 \in [-1.24, -0.6], \text{   }  z_2 \in [2.46, 2.75], \text{   and   } z_3 \in [2.57, 3.15] \)

 Distractor 3: Corresponds to negatives of all zeros AND inversing rational roots.
\item \( z_1 \in [-3.68, -2.89], \text{   }  z_2 \in [-3.01, -2.32], \text{   and   } z_3 \in [0.45, 1.14] \)

 Distractor 2: Corresponds to inversing rational roots.
\item \( z_1 \in [-4.14, -3.63], \text{   }  z_2 \in [-0.24, 0.29], \text{   and   } z_3 \in [2.57, 3.15] \)

 Distractor 4: Corresponds to moving factors from one rational to another.
\item \( z_1 \in [-3.68, -2.89], \text{   }  z_2 \in [-0.43, -0.33], \text{   and   } z_3 \in [1.27, 1.74] \)

* This is the solution!
\end{enumerate}

\textbf{General Comment:} Remember to try the middle-most integers first as these normally are the zeros. Also, once you get it to a quadratic, you can use your other factoring techniques to finish factoring.
}
\litem{
Factor the polynomial below completely. Then, choose the intervals the zeros of the polynomial belong to, where $z_1 \leq z_2 \leq z_3$. \textit{To make the problem easier, all zeros are between -5 and 5.}
\[ f(x) = 15x^{3} -64 x^{2} +12 x + 16 \]

The solution is \( [-0.4, 0.6666666666666666, 4] \), which is option E.\begin{enumerate}[label=\Alph*.]
\item \( z_1 \in [-5, -3], \text{   }  z_2 \in [-0.78, -0.27], \text{   and   } z_3 \in [0.22, 0.71] \)

 Distractor 1: Corresponds to negatives of all zeros.
\item \( z_1 \in [-5, -3], \text{   }  z_2 \in [-1.61, -0.92], \text{   and   } z_3 \in [2.35, 2.71] \)

 Distractor 3: Corresponds to negatives of all zeros AND inversing rational roots.
\item \( z_1 \in [-5, -3], \text{   }  z_2 \in [-2.24, -1.98], \text{   and   } z_3 \in [0.04, 0.16] \)

 Distractor 4: Corresponds to moving factors from one rational to another.
\item \( z_1 \in [-2.5, -1.5], \text{   }  z_2 \in [1.14, 1.56], \text{   and   } z_3 \in [3.73, 4.19] \)

 Distractor 2: Corresponds to inversing rational roots.
\item \( z_1 \in [-1.4, 1.6], \text{   }  z_2 \in [0.54, 1.28], \text{   and   } z_3 \in [3.73, 4.19] \)

* This is the solution!
\end{enumerate}

\textbf{General Comment:} Remember to try the middle-most integers first as these normally are the zeros. Also, once you get it to a quadratic, you can use your other factoring techniques to finish factoring.
}
\litem{
Factor the polynomial below completely, knowing that $x+2$ is a factor. Then, choose the intervals the zeros of the polynomial belong to, where $z_1 \leq z_2 \leq z_3 \leq z_4$. \textit{To make the problem easier, all zeros are between -5 and 5.}
\[ f(x) = 20x^{4} +129 x^{3} +194 x^{2} -48 x -160 \]

The solution is \( [-4, -2, -1.25, 0.8] \), which is option C.\begin{enumerate}[label=\Alph*.]
\item \( z_1 \in [-4.33, -3.88], \text{   }  z_2 \in [0.19, 0.53], z_3 \in [1.37, 2.44], \text{   and   } z_4 \in [3.2, 5.3] \)

 Distractor 4: Corresponds to moving factors from one rational to another.
\item \( z_1 \in [-4.33, -3.88], \text{   }  z_2 \in [-2.12, -1.88], z_3 \in [-1.01, -0.29], \text{   and   } z_4 \in [1, 1.5] \)

 Distractor 2: Corresponds to inversing rational roots.
\item \( z_1 \in [-4.33, -3.88], \text{   }  z_2 \in [-2.12, -1.88], z_3 \in [-1.98, -1.21], \text{   and   } z_4 \in [0.2, 0.9] \)

* This is the solution!
\item \( z_1 \in [-0.99, -0.08], \text{   }  z_2 \in [1.24, 1.6], z_3 \in [1.37, 2.44], \text{   and   } z_4 \in [3.2, 5.3] \)

 Distractor 1: Corresponds to negatives of all zeros.
\item \( z_1 \in [-1.61, -1.07], \text{   }  z_2 \in [0.76, 1.2], z_3 \in [1.37, 2.44], \text{   and   } z_4 \in [3.2, 5.3] \)

 Distractor 3: Corresponds to negatives of all zeros AND inversing rational roots.
\end{enumerate}

\textbf{General Comment:} Remember to try the middle-most integers first as these normally are the zeros. Also, once you get it to a quadratic, you can use your other factoring techniques to finish factoring.
}
\end{enumerate}

\end{document}