\documentclass{extbook}[14pt]
\usepackage{multicol, enumerate, enumitem, hyperref, color, soul, setspace, parskip, fancyhdr, amssymb, amsthm, amsmath, latexsym, units, mathtools}
\everymath{\displaystyle}
\usepackage[headsep=0.5cm,headheight=0cm, left=1 in,right= 1 in,top= 1 in,bottom= 1 in]{geometry}
\usepackage{dashrule}  % Package to use the command below to create lines between items
\newcommand{\litem}[1]{\item #1

\rule{\textwidth}{0.4pt}}
\pagestyle{fancy}
\lhead{}
\chead{Answer Key for Progress Quiz 10 Version B}
\rhead{}
\lfoot{1995-1928}
\cfoot{}
\rfoot{test}
\begin{document}
\textbf{This key should allow you to understand why you choose the option you did (beyond just getting a question right or wrong). \href{https://xronos.clas.ufl.edu/mac1105spring2020/courseDescriptionAndMisc/Exams/LearningFromResults}{More instructions on how to use this key can be found here}.}

\textbf{If you have a suggestion to make the keys better, \href{https://forms.gle/CZkbZmPbC9XALEE88}{please fill out the short survey here}.}

\textit{Note: This key is auto-generated and may contain issues and/or errors. The keys are reviewed after each exam to ensure grading is done accurately. If there are issues (like duplicate options), they are noted in the offline gradebook. The keys are a work-in-progress to give students as many resources to improve as possible.}

\rule{\textwidth}{0.4pt}

\begin{enumerate}\litem{
Simplify the expression below into the form $a+bi$. Then, choose the intervals that $a$ and $b$ belong to.
\[ (-7 - 4 i)(3 + 6 i) \]The solution is \( 3 - 54 i \), which is option A.\begin{enumerate}[label=\Alph*.]
\item \( a \in [3, 5] \text{ and } b \in [-58, -47] \)

* $3 - 54 i$, which is the correct option.
\item \( a \in [-26, -18] \text{ and } b \in [-24, -18] \)

 $-21 - 24 i$, which corresponds to just multiplying the real terms to get the real part of the solution and the coefficients in the complex terms to get the complex part.
\item \( a \in [3, 5] \text{ and } b \in [50, 61] \)

 $3 + 54 i$, which corresponds to adding a minus sign in both terms.
\item \( a \in [-45, -42] \text{ and } b \in [-31, -29] \)

 $-45 - 30 i$, which corresponds to adding a minus sign in the first term.
\item \( a \in [-45, -42] \text{ and } b \in [27, 34] \)

 $-45 + 30 i$, which corresponds to adding a minus sign in the second term.
\end{enumerate}

\textbf{General Comment:} You can treat $i$ as a variable and distribute. Just remember that $i^2=-1$, so you can continue to reduce after you distribute.
}
\litem{
Choose the \textbf{smallest} set of Complex numbers that the number below belongs to.
\[ \sqrt{\frac{-390}{6}}+\sqrt{234} \]The solution is \( \text{Nonreal Complex} \), which is option B.\begin{enumerate}[label=\Alph*.]
\item \( \text{Rational} \)

These are numbers that can be written as fraction of Integers (e.g., -2/3 + 5)
\item \( \text{Nonreal Complex} \)

* This is the correct option!
\item \( \text{Pure Imaginary} \)

This is a Complex number $(a+bi)$ that \textbf{only} has an imaginary part like $2i$.
\item \( \text{Irrational} \)

These cannot be written as a fraction of Integers. Remember: $\pi$ is not an Integer!
\item \( \text{Not a Complex Number} \)

This is not a number. The only non-Complex number we know is dividing by 0 as this is not a number!
\end{enumerate}

\textbf{General Comment:} Be sure to simplify $i^2 = -1$. This may remove the imaginary portion for your number. If you are having trouble, you may want to look at the \textit{Subgroups of the Real Numbers} section.
}
\litem{
Simplify the expression below and choose the interval the simplification is contained within.
\[ 6 - 7^2 + 1 \div 10 * 19 \div 4 \]The solution is \( -42.525 \), which is option C.\begin{enumerate}[label=\Alph*.]
\item \( [-43.27, -42.62] \)

 -42.999, which corresponds to an Order of Operations error: not reading left-to-right for multiplication/division.
\item \( [54.92, 55.25] \)

 55.001, which corresponds to two Order of Operations errors.
\item \( [-42.81, -41.74] \)

* -42.525, this is the correct option
\item \( [55.29, 56.16] \)

 55.475, which corresponds to an Order of Operations error: multiplying by negative before squaring. For example: $(-3)^2 \neq -3^2$
\item \( \text{None of the above} \)

 You may have gotten this by making an unanticipated error. If you got a value that is not any of the others, please let the coordinator know so they can help you figure out what happened.
\end{enumerate}

\textbf{General Comment:} While you may remember (or were taught) PEMDAS is done in order, it is actually done as P/E/MD/AS. When we are at MD or AS, we read left to right.
}
\litem{
Choose the \textbf{smallest} set of Complex numbers that the number below belongs to.
\[ -\sqrt{\frac{25}{121}} + 64i^2 \]The solution is \( \text{Rational} \), which is option C.\begin{enumerate}[label=\Alph*.]
\item \( \text{Pure Imaginary} \)

This is a Complex number $(a+bi)$ that \textbf{only} has an imaginary part like $2i$.
\item \( \text{Irrational} \)

These cannot be written as a fraction of Integers. Remember: $\pi$ is not an Integer!
\item \( \text{Rational} \)

* This is the correct option!
\item \( \text{Not a Complex Number} \)

This is not a number. The only non-Complex number we know is dividing by 0 as this is not a number!
\item \( \text{Nonreal Complex} \)

This is a Complex number $(a+bi)$ that is not Real (has $i$ as part of the number).
\end{enumerate}

\textbf{General Comment:} Be sure to simplify $i^2 = -1$. This may remove the imaginary portion for your number. If you are having trouble, you may want to look at the \textit{Subgroups of the Real Numbers} section.
}
\litem{
Choose the \textbf{smallest} set of Real numbers that the number below belongs to.
\[ -\sqrt{\frac{6400}{64}} \]The solution is \( \text{Integer} \), which is option B.\begin{enumerate}[label=\Alph*.]
\item \( \text{Irrational} \)

These cannot be written as a fraction of Integers.
\item \( \text{Integer} \)

* This is the correct option!
\item \( \text{Rational} \)

These are numbers that can be written as fraction of Integers (e.g., -2/3)
\item \( \text{Whole} \)

These are the counting numbers with 0 (0, 1, 2, 3, ...)
\item \( \text{Not a Real number} \)

These are Nonreal Complex numbers \textbf{OR} things that are not numbers (e.g., dividing by 0).
\end{enumerate}

\textbf{General Comment:} First, you \textbf{NEED} to simplify the expression. This question simplifies to $-80$. 
 
 Be sure you look at the simplified fraction and not just the decimal expansion. Numbers such as 13, 17, and 19 provide \textbf{long but repeating/terminating decimal expansions!} 
 
 The only ways to *not* be a Real number are: dividing by 0 or taking the square root of a negative number. 
 
 Irrational numbers are more than just square root of 3: adding or subtracting values from square root of 3 is also irrational.
}
\litem{
Choose the \textbf{smallest} set of Real numbers that the number below belongs to.
\[ -\sqrt{\frac{17}{0}} \]The solution is \( \text{Not a Real number} \), which is option E.\begin{enumerate}[label=\Alph*.]
\item \( \text{Whole} \)

These are the counting numbers with 0 (0, 1, 2, 3, ...)
\item \( \text{Irrational} \)

These cannot be written as a fraction of Integers.
\item \( \text{Rational} \)

These are numbers that can be written as fraction of Integers (e.g., -2/3)
\item \( \text{Integer} \)

These are the negative and positive counting numbers (..., -3, -2, -1, 0, 1, 2, 3, ...)
\item \( \text{Not a Real number} \)

* This is the correct option!
\end{enumerate}

\textbf{General Comment:} First, you \textbf{NEED} to simplify the expression. This question simplifies to $-\sqrt{\frac{17}{0}}$. 
 
 Be sure you look at the simplified fraction and not just the decimal expansion. Numbers such as 13, 17, and 19 provide \textbf{long but repeating/terminating decimal expansions!} 
 
 The only ways to *not* be a Real number are: dividing by 0 or taking the square root of a negative number. 
 
 Irrational numbers are more than just square root of 3: adding or subtracting values from square root of 3 is also irrational.
}
\litem{
Simplify the expression below into the form $a+bi$. Then, choose the intervals that $a$ and $b$ belong to.
\[ \frac{-18 - 88 i}{3 + 5 i} \]The solution is \( -14.53  - 5.12 i \), which is option C.\begin{enumerate}[label=\Alph*.]
\item \( a \in [-7, -5.5] \text{ and } b \in [-18, -17] \)

 $-6.00  - 17.60 i$, which corresponds to just dividing the first term by the first term and the second by the second.
\item \( a \in [-494.5, -493.5] \text{ and } b \in [-5.5, -4.5] \)

 $-494.00  - 5.12 i$, which corresponds to forgetting to multiply the conjugate by the numerator and using a plus instead of a minus in the denominator.
\item \( a \in [-15, -14] \text{ and } b \in [-5.5, -4.5] \)

* $-14.53  - 5.12 i$, which is the correct option.
\item \( a \in [10.5, 13] \text{ and } b \in [-11, -9.5] \)

 $11.35  - 10.41 i$, which corresponds to forgetting to multiply the conjugate by the numerator and not computing the conjugate correctly.
\item \( a \in [-15, -14] \text{ and } b \in [-174.5, -172.5] \)

 $-14.53  - 174.00 i$, which corresponds to forgetting to multiply the conjugate by the numerator.
\end{enumerate}

\textbf{General Comment:} Multiply the numerator and denominator by the *conjugate* of the denominator, then simplify. For example, if we have $2+3i$, the conjugate is $2-3i$.
}
\litem{
Simplify the expression below and choose the interval the simplification is contained within.
\[ 20 - 17^2 + 18 \div 13 * 16 \div 15 \]The solution is \( -267.523 \), which is option D.\begin{enumerate}[label=\Alph*.]
\item \( [310.12, 310.92] \)

 310.477, which corresponds to an Order of Operations error: multiplying by negative before squaring. For example: $(-3)^2 \neq -3^2$
\item \( [-269.32, -268.77] \)

 -268.994, which corresponds to an Order of Operations error: not reading left-to-right for multiplication/division.
\item \( [308.47, 309.18] \)

 309.006, which corresponds to two Order of Operations errors.
\item \( [-268.2, -267.38] \)

* -267.523, this is the correct option
\item \( \text{None of the above} \)

 You may have gotten this by making an unanticipated error. If you got a value that is not any of the others, please let the coordinator know so they can help you figure out what happened.
\end{enumerate}

\textbf{General Comment:} While you may remember (or were taught) PEMDAS is done in order, it is actually done as P/E/MD/AS. When we are at MD or AS, we read left to right.
}
\litem{
Simplify the expression below into the form $a+bi$. Then, choose the intervals that $a$ and $b$ belong to.
\[ \frac{-72 - 11 i}{-6 + 5 i} \]The solution is \( 6.18  + 6.98 i \), which is option C.\begin{enumerate}[label=\Alph*.]
\item \( a \in [376, 378] \text{ and } b \in [6, 8] \)

 $377.00  + 6.98 i$, which corresponds to forgetting to multiply the conjugate by the numerator and using a plus instead of a minus in the denominator.
\item \( a \in [5.5, 6.5] \text{ and } b \in [425, 427.5] \)

 $6.18  + 426.00 i$, which corresponds to forgetting to multiply the conjugate by the numerator.
\item \( a \in [5.5, 6.5] \text{ and } b \in [6, 8] \)

* $6.18  + 6.98 i$, which is the correct option.
\item \( a \in [10.5, 13] \text{ and } b \in [-3, -1.5] \)

 $12.00  - 2.20 i$, which corresponds to just dividing the first term by the first term and the second by the second.
\item \( a \in [7.5, 8.5] \text{ and } b \in [-5, -3.5] \)

 $7.98  - 4.82 i$, which corresponds to forgetting to multiply the conjugate by the numerator and not computing the conjugate correctly.
\end{enumerate}

\textbf{General Comment:} Multiply the numerator and denominator by the *conjugate* of the denominator, then simplify. For example, if we have $2+3i$, the conjugate is $2-3i$.
}
\litem{
Simplify the expression below into the form $a+bi$. Then, choose the intervals that $a$ and $b$ belong to.
\[ (7 + 3 i)(2 - 9 i) \]The solution is \( 41 - 57 i \), which is option D.\begin{enumerate}[label=\Alph*.]
\item \( a \in [14, 20] \text{ and } b \in [-34, -20] \)

 $14 - 27 i$, which corresponds to just multiplying the real terms to get the real part of the solution and the coefficients in the complex terms to get the complex part.
\item \( a \in [38, 43] \text{ and } b \in [53, 59] \)

 $41 + 57 i$, which corresponds to adding a minus sign in both terms.
\item \( a \in [-19, -10] \text{ and } b \in [-72, -65] \)

 $-13 - 69 i$, which corresponds to adding a minus sign in the first term.
\item \( a \in [38, 43] \text{ and } b \in [-59, -50] \)

* $41 - 57 i$, which is the correct option.
\item \( a \in [-19, -10] \text{ and } b \in [66, 71] \)

 $-13 + 69 i$, which corresponds to adding a minus sign in the second term.
\end{enumerate}

\textbf{General Comment:} You can treat $i$ as a variable and distribute. Just remember that $i^2=-1$, so you can continue to reduce after you distribute.
}
\end{enumerate}

\end{document}