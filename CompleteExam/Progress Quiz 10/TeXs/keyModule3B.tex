\documentclass{extbook}[14pt]
\usepackage{multicol, enumerate, enumitem, hyperref, color, soul, setspace, parskip, fancyhdr, amssymb, amsthm, amsmath, latexsym, units, mathtools}
\everymath{\displaystyle}
\usepackage[headsep=0.5cm,headheight=0cm, left=1 in,right= 1 in,top= 1 in,bottom= 1 in]{geometry}
\usepackage{dashrule}  % Package to use the command below to create lines between items
\newcommand{\litem}[1]{\item #1

\rule{\textwidth}{0.4pt}}
\pagestyle{fancy}
\lhead{}
\chead{Answer Key for Progress Quiz 10 Version B}
\rhead{}
\lfoot{5170-5105}
\cfoot{}
\rfoot{Summer C 2021}
\begin{document}
\textbf{This key should allow you to understand why you choose the option you did (beyond just getting a question right or wrong). \href{https://xronos.clas.ufl.edu/mac1105spring2020/courseDescriptionAndMisc/Exams/LearningFromResults}{More instructions on how to use this key can be found here}.}

\textbf{If you have a suggestion to make the keys better, \href{https://forms.gle/CZkbZmPbC9XALEE88}{please fill out the short survey here}.}

\textit{Note: This key is auto-generated and may contain issues and/or errors. The keys are reviewed after each exam to ensure grading is done accurately. If there are issues (like duplicate options), they are noted in the offline gradebook. The keys are a work-in-progress to give students as many resources to improve as possible.}

\rule{\textwidth}{0.4pt}

\begin{enumerate}\litem{
Solve the linear inequality below. Then, choose the constant and interval combination that describes the solution set.
\[ -5 + 8 x \leq \frac{37 x - 3}{4} < -5 + 5 x \]The solution is \( [-3.40, -1.00) \), which is option A.\begin{enumerate}[label=\Alph*.]
\item \( [a, b), \text{ where } a \in [-6.75, 1.5] \text{ and } b \in [-1.65, 0.15] \)

$[-3.40, -1.00)$, which is the correct option.
\item \( (-\infty, a) \cup [b, \infty), \text{ where } a \in [-6, -0.75] \text{ and } b \in [-2.02, 0] \)

$(-\infty, -3.40) \cup [-1.00, \infty)$, which corresponds to displaying the and-inequality as an or-inequality AND flipping the inequality.
\item \( (a, b], \text{ where } a \in [-9, -0.75] \text{ and } b \in [-2.17, -0.45] \)

$(-3.40, -1.00]$, which corresponds to flipping the inequality.
\item \( (-\infty, a] \cup (b, \infty), \text{ where } a \in [-5.25, -1.5] \text{ and } b \in [-2.25, 0] \)

$(-\infty, -3.40] \cup (-1.00, \infty)$, which corresponds to displaying the and-inequality as an or-inequality.
\item \( \text{None of the above.} \)


\end{enumerate}

\textbf{General Comment:} To solve, you will need to break up the compound inequality into two inequalities. Be sure to keep track of the inequality! It may be best to draw a number line and graph your solution.
}
\litem{
Solve the linear inequality below. Then, choose the constant and interval combination that describes the solution set.
\[ \frac{-4}{3} - \frac{6}{9} x \leq \frac{-4}{2} x + \frac{7}{7} \]The solution is \( (-\infty, 1.75] \), which is option C.\begin{enumerate}[label=\Alph*.]
\item \( (-\infty, a], \text{ where } a \in [-6, 0.75] \)

 $(-\infty, -1.75]$, which corresponds to negating the endpoint of the solution.
\item \( [a, \infty), \text{ where } a \in [0.75, 4.5] \)

 $[1.75, \infty)$, which corresponds to switching the direction of the interval. You likely did this if you did not flip the inequality when dividing by a negative!
\item \( (-\infty, a], \text{ where } a \in [0.75, 4.5] \)

* $(-\infty, 1.75]$, which is the correct option.
\item \( [a, \infty), \text{ where } a \in [-2.25, 0.75] \)

 $[-1.75, \infty)$, which corresponds to switching the direction of the interval AND negating the endpoint. You likely did this if you did not flip the inequality when dividing by a negative as well as not moving values over to a side properly.
\item \( \text{None of the above}. \)

You may have chosen this if you thought the inequality did not match the ends of the intervals.
\end{enumerate}

\textbf{General Comment:} Remember that less/greater than or equal to includes the endpoint, while less/greater do not. Also, remember that you need to flip the inequality when you multiply or divide by a negative.
}
\litem{
Solve the linear inequality below. Then, choose the constant and interval combination that describes the solution set.
\[ 5x + 9 < 9x -8 \]The solution is \( (4.25, \infty) \), which is option D.\begin{enumerate}[label=\Alph*.]
\item \( (-\infty, a), \text{ where } a \in [3.25, 8.25] \)

 $(-\infty, 4.25)$, which corresponds to switching the direction of the interval. You likely did this if you did not flip the inequality when dividing by a negative!
\item \( (a, \infty), \text{ where } a \in [-4.25, -2.25] \)

 $(-4.25, \infty)$, which corresponds to negating the endpoint of the solution.
\item \( (-\infty, a), \text{ where } a \in [-4.25, 1.75] \)

 $(-\infty, -4.25)$, which corresponds to switching the direction of the interval AND negating the endpoint. You likely did this if you did not flip the inequality when dividing by a negative as well as not moving values over to a side properly.
\item \( (a, \infty), \text{ where } a \in [-1.75, 5.25] \)

* $(4.25, \infty)$, which is the correct option.
\item \( \text{None of the above}. \)

You may have chosen this if you thought the inequality did not match the ends of the intervals.
\end{enumerate}

\textbf{General Comment:} Remember that less/greater than or equal to includes the endpoint, while less/greater do not. Also, remember that you need to flip the inequality when you multiply or divide by a negative.
}
\litem{
Solve the linear inequality below. Then, choose the constant and interval combination that describes the solution set.
\[ 3x -6 \geq 7x + 3 \]The solution is \( (-\infty, -2.25] \), which is option A.\begin{enumerate}[label=\Alph*.]
\item \( (-\infty, a], \text{ where } a \in [-2.6, -0.1] \)

* $(-\infty, -2.25]$, which is the correct option.
\item \( (-\infty, a], \text{ where } a \in [0.2, 4.2] \)

 $(-\infty, 2.25]$, which corresponds to negating the endpoint of the solution.
\item \( [a, \infty), \text{ where } a \in [1.25, 8.25] \)

 $[2.25, \infty)$, which corresponds to switching the direction of the interval AND negating the endpoint. You likely did this if you did not flip the inequality when dividing by a negative as well as not moving values over to a side properly.
\item \( [a, \infty), \text{ where } a \in [-4.25, 0.75] \)

 $[-2.25, \infty)$, which corresponds to switching the direction of the interval. You likely did this if you did not flip the inequality when dividing by a negative!
\item \( \text{None of the above}. \)

You may have chosen this if you thought the inequality did not match the ends of the intervals.
\end{enumerate}

\textbf{General Comment:} Remember that less/greater than or equal to includes the endpoint, while less/greater do not. Also, remember that you need to flip the inequality when you multiply or divide by a negative.
}
\litem{
Using an interval or intervals, describe all the $x$-values within or including a distance of the given values.
\[ \text{ No less than } 4 \text{ units from the number } 7. \]The solution is \( (-\infty, 3] \cup [11, \infty) \), which is option A.\begin{enumerate}[label=\Alph*.]
\item \( (-\infty, 3] \cup [11, \infty) \)

This describes the values no less than 4 from 7
\item \( [3, 11] \)

This describes the values no more than 4 from 7
\item \( (3, 11) \)

This describes the values less than 4 from 7
\item \( (-\infty, 3) \cup (11, \infty) \)

This describes the values more than 4 from 7
\item \( \text{None of the above} \)

You likely thought the values in the interval were not correct.
\end{enumerate}

\textbf{General Comment:} When thinking about this language, it helps to draw a number line and try points.
}
\litem{
Solve the linear inequality below. Then, choose the constant and interval combination that describes the solution set.
\[ -7 - 4 x < \frac{20 x + 8}{4} \leq 8 + 4 x \]The solution is \( (-1.00, 6.00] \), which is option B.\begin{enumerate}[label=\Alph*.]
\item \( (-\infty, a] \cup (b, \infty), \text{ where } a \in [-3, 0.75] \text{ and } b \in [1.5, 9] \)

$(-\infty, -1.00] \cup (6.00, \infty)$, which corresponds to displaying the and-inequality as an or-inequality AND flipping the inequality.
\item \( (a, b], \text{ where } a \in [-2.55, -0.22] \text{ and } b \in [5.25, 8.25] \)

* $(-1.00, 6.00]$, which is the correct option.
\item \( [a, b), \text{ where } a \in [-4.5, -0.75] \text{ and } b \in [4.5, 11.25] \)

$[-1.00, 6.00)$, which corresponds to flipping the inequality.
\item \( (-\infty, a) \cup [b, \infty), \text{ where } a \in [-2.17, 0.3] \text{ and } b \in [2.25, 7.5] \)

$(-\infty, -1.00) \cup [6.00, \infty)$, which corresponds to displaying the and-inequality as an or-inequality.
\item \( \text{None of the above.} \)


\end{enumerate}

\textbf{General Comment:} To solve, you will need to break up the compound inequality into two inequalities. Be sure to keep track of the inequality! It may be best to draw a number line and graph your solution.
}
\litem{
Solve the linear inequality below. Then, choose the constant and interval combination that describes the solution set.
\[ -3 + 7 x > 9 x \text{ or } 4 + 5 x < 6 x \]The solution is \( (-\infty, -1.5) \text{ or } (4.0, \infty) \), which is option A.\begin{enumerate}[label=\Alph*.]
\item \( (-\infty, a) \cup (b, \infty), \text{ where } a \in [-3.75, -0.75] \text{ and } b \in [3.23, 4.42] \)

 * Correct option.
\item \( (-\infty, a] \cup [b, \infty), \text{ where } a \in [-4.27, -2.32] \text{ and } b \in [1.2, 1.65] \)

Corresponds to including the endpoints AND negating.
\item \( (-\infty, a) \cup (b, \infty), \text{ where } a \in [-7.5, -2.25] \text{ and } b \in [0.07, 2.7] \)

Corresponds to inverting the inequality and negating the solution.
\item \( (-\infty, a] \cup [b, \infty), \text{ where } a \in [-3.97, 1.12] \text{ and } b \in [2.32, 4.35] \)

Corresponds to including the endpoints (when they should be excluded).
\item \( (-\infty, \infty) \)

Corresponds to the variable canceling, which does not happen in this instance.
\end{enumerate}

\textbf{General Comment:} When multiplying or dividing by a negative, flip the sign.
}
\litem{
Solve the linear inequality below. Then, choose the constant and interval combination that describes the solution set.
\[ -5 + 4 x > 7 x \text{ or } 7 + 7 x < 9 x \]The solution is \( (-\infty, -1.667) \text{ or } (3.5, \infty) \), which is option C.\begin{enumerate}[label=\Alph*.]
\item \( (-\infty, a) \cup (b, \infty), \text{ where } a \in [-3.71, -3.04] \text{ and } b \in [0.97, 3.38] \)

Corresponds to inverting the inequality and negating the solution.
\item \( (-\infty, a] \cup [b, \infty), \text{ where } a \in [-4.5, -1.72] \text{ and } b \in [-1.35, 3.23] \)

Corresponds to including the endpoints AND negating.
\item \( (-\infty, a) \cup (b, \infty), \text{ where } a \in [-2.21, -0.17] \text{ and } b \in [1.8, 5.1] \)

 * Correct option.
\item \( (-\infty, a] \cup [b, \infty), \text{ where } a \in [-1.8, -0.82] \text{ and } b \in [3.3, 3.67] \)

Corresponds to including the endpoints (when they should be excluded).
\item \( (-\infty, \infty) \)

Corresponds to the variable canceling, which does not happen in this instance.
\end{enumerate}

\textbf{General Comment:} When multiplying or dividing by a negative, flip the sign.
}
\litem{
Solve the linear inequality below. Then, choose the constant and interval combination that describes the solution set.
\[ \frac{-9}{4} - \frac{6}{7} x < \frac{8}{8} x + \frac{7}{5} \]The solution is \( (-1.965, \infty) \), which is option D.\begin{enumerate}[label=\Alph*.]
\item \( (a, \infty), \text{ where } a \in [0, 4.5] \)

 $(1.965, \infty)$, which corresponds to negating the endpoint of the solution.
\item \( (-\infty, a), \text{ where } a \in [0, 4.5] \)

 $(-\infty, 1.965)$, which corresponds to switching the direction of the interval AND negating the endpoint. You likely did this if you did not flip the inequality when dividing by a negative as well as not moving values over to a side properly.
\item \( (-\infty, a), \text{ where } a \in [-2.25, 0] \)

 $(-\infty, -1.965)$, which corresponds to switching the direction of the interval. You likely did this if you did not flip the inequality when dividing by a negative!
\item \( (a, \infty), \text{ where } a \in [-6.75, -0.75] \)

* $(-1.965, \infty)$, which is the correct option.
\item \( \text{None of the above}. \)

You may have chosen this if you thought the inequality did not match the ends of the intervals.
\end{enumerate}

\textbf{General Comment:} Remember that less/greater than or equal to includes the endpoint, while less/greater do not. Also, remember that you need to flip the inequality when you multiply or divide by a negative.
}
\litem{
Using an interval or intervals, describe all the $x$-values within or including a distance of the given values.
\[ \text{ No more than } 10 \text{ units from the number } -7. \]The solution is \( [-17, 3] \), which is option B.\begin{enumerate}[label=\Alph*.]
\item \( (-17, 3) \)

This describes the values less than 10 from -7
\item \( [-17, 3] \)

This describes the values no more than 10 from -7
\item \( (-\infty, -17) \cup (3, \infty) \)

This describes the values more than 10 from -7
\item \( (-\infty, -17] \cup [3, \infty) \)

This describes the values no less than 10 from -7
\item \( \text{None of the above} \)

You likely thought the values in the interval were not correct.
\end{enumerate}

\textbf{General Comment:} When thinking about this language, it helps to draw a number line and try points.
}
\end{enumerate}

\end{document}