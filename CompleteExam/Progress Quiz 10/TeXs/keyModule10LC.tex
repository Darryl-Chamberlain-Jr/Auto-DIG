\documentclass{extbook}[14pt]
\usepackage{multicol, enumerate, enumitem, hyperref, color, soul, setspace, parskip, fancyhdr, amssymb, amsthm, amsmath, latexsym, units, mathtools}
\everymath{\displaystyle}
\usepackage[headsep=0.5cm,headheight=0cm, left=1 in,right= 1 in,top= 1 in,bottom= 1 in]{geometry}
\usepackage{dashrule}  % Package to use the command below to create lines between items
\newcommand{\litem}[1]{\item #1

\rule{\textwidth}{0.4pt}}
\pagestyle{fancy}
\lhead{}
\chead{Answer Key for Progress Quiz 10 Version C}
\rhead{}
\lfoot{5170-5105}
\cfoot{}
\rfoot{Summer C 2021}
\begin{document}
\textbf{This key should allow you to understand why you choose the option you did (beyond just getting a question right or wrong). \href{https://xronos.clas.ufl.edu/mac1105spring2020/courseDescriptionAndMisc/Exams/LearningFromResults}{More instructions on how to use this key can be found here}.}

\textbf{If you have a suggestion to make the keys better, \href{https://forms.gle/CZkbZmPbC9XALEE88}{please fill out the short survey here}.}

\textit{Note: This key is auto-generated and may contain issues and/or errors. The keys are reviewed after each exam to ensure grading is done accurately. If there are issues (like duplicate options), they are noted in the offline gradebook. The keys are a work-in-progress to give students as many resources to improve as possible.}

\rule{\textwidth}{0.4pt}

\begin{enumerate}\litem{
Factor the polynomial below completely. Then, choose the intervals the zeros of the polynomial belong to, where $z_1 \leq z_2 \leq z_3$. \textit{To make the problem easier, all zeros are between -5 and 5.}
\[ f(x) = 25x^{3} +95 x^{2} -142 x + 40 \]The solution is \( [-5, 0.4, 0.8] \), which is option D.\begin{enumerate}[label=\Alph*.]
\item \( z_1 \in [-0.9, -0.5], \text{   }  z_2 \in [-0.76, -0.27], \text{   and   } z_3 \in [4.8, 5.7] \)

 Distractor 1: Corresponds to negatives of all zeros.
\item \( z_1 \in [-5.3, -4.7], \text{   }  z_2 \in [1.02, 1.34], \text{   and   } z_3 \in [2.4, 2.9] \)

 Distractor 2: Corresponds to inversing rational roots.
\item \( z_1 \in [-3.7, -1.8], \text{   }  z_2 \in [-1.31, -1.16], \text{   and   } z_3 \in [4.8, 5.7] \)

 Distractor 3: Corresponds to negatives of all zeros AND inversing rational roots.
\item \( z_1 \in [-5.3, -4.7], \text{   }  z_2 \in [0.06, 0.68], \text{   and   } z_3 \in [0, 1.3] \)

* This is the solution!
\item \( z_1 \in [-4.5, -3.8], \text{   }  z_2 \in [-0.24, 0.01], \text{   and   } z_3 \in [4.8, 5.7] \)

 Distractor 4: Corresponds to moving factors from one rational to another.
\end{enumerate}

\textbf{General Comment:} Remember to try the middle-most integers first as these normally are the zeros. Also, once you get it to a quadratic, you can use your other factoring techniques to finish factoring.
}
\litem{
What are the \textit{possible Integer} roots of the polynomial below?
\[ f(x) = 3x^{4} +6 x^{3} +4 x^{2} +4 x + 4 \]The solution is \( \pm 1,\pm 2,\pm 4 \), which is option D.\begin{enumerate}[label=\Alph*.]
\item \( \pm 1,\pm 3 \)

 Distractor 1: Corresponds to the plus or minus factors of a1 only.
\item \( \text{ All combinations of: }\frac{\pm 1,\pm 3}{\pm 1,\pm 2,\pm 4} \)

 Distractor 3: Corresponds to the plus or minus of the inverse quotient (an/a0) of the factors. 
\item \( \text{ All combinations of: }\frac{\pm 1,\pm 2,\pm 4}{\pm 1,\pm 3} \)

This would have been the solution \textbf{if asked for the possible Rational roots}!
\item \( \pm 1,\pm 2,\pm 4 \)

* This is the solution \textbf{since we asked for the possible Integer roots}!
\item \( \text{There is no formula or theorem that tells us all possible Integer roots.} \)

 Distractor 4: Corresponds to not recognizing Integers as a subset of Rationals.
\end{enumerate}

\textbf{General Comment:} We have a way to find the possible Rational roots. The possible Integer roots are the Integers in this list.
}
\litem{
Factor the polynomial below completely. Then, choose the intervals the zeros of the polynomial belong to, where $z_1 \leq z_2 \leq z_3$. \textit{To make the problem easier, all zeros are between -5 and 5.}
\[ f(x) = 6x^{3} -7 x^{2} -43 x + 30 \]The solution is \( [-2.5, 0.67, 3] \), which is option D.\begin{enumerate}[label=\Alph*.]
\item \( z_1 \in [-4.9, -2.9], \text{   }  z_2 \in [-0.92, -0.59], \text{   and   } z_3 \in [1.97, 2.57] \)

 Distractor 1: Corresponds to negatives of all zeros.
\item \( z_1 \in [-1.6, 0.4], \text{   }  z_2 \in [1.39, 1.65], \text{   and   } z_3 \in [2.75, 3.19] \)

 Distractor 2: Corresponds to inversing rational roots.
\item \( z_1 \in [-4.9, -2.9], \text{   }  z_2 \in [-1.53, -1.44], \text{   and   } z_3 \in [0.34, 0.48] \)

 Distractor 3: Corresponds to negatives of all zeros AND inversing rational roots.
\item \( z_1 \in [-2.9, -1.4], \text{   }  z_2 \in [0.64, 0.91], \text{   and   } z_3 \in [2.75, 3.19] \)

* This is the solution!
\item \( z_1 \in [-4.9, -2.9], \text{   }  z_2 \in [-0.55, -0.24], \text{   and   } z_3 \in [4.43, 5.47] \)

 Distractor 4: Corresponds to moving factors from one rational to another.
\end{enumerate}

\textbf{General Comment:} Remember to try the middle-most integers first as these normally are the zeros. Also, once you get it to a quadratic, you can use your other factoring techniques to finish factoring.
}
\litem{
Perform the division below. Then, find the intervals that correspond to the quotient in the form $ax^2+bx+c$ and remainder $r$.
\[ \frac{4x^{3} -48 x + 62}{x + 4} \]The solution is \( 4x^{2} -16 x + 16 + \frac{-2}{x + 4} \), which is option B.\begin{enumerate}[label=\Alph*.]
\item \( a \in [0, 12], b \in [-20.8, -19.5], c \in [50, 56], \text{ and } r \in [-204, -197]. \)

 You multipled by the synthetic number and subtracted rather than adding during synthetic division.
\item \( a \in [0, 12], b \in [-16.8, -15.2], c \in [13, 21], \text{ and } r \in [-5, 2]. \)

* This is the solution!
\item \( a \in [-18, -9], b \in [63.5, 64.5], c \in [-305, -296], \text{ and } r \in [1278, 1285]. \)

 You multipled by the synthetic number rather than bringing the first factor down.
\item \( a \in [-18, -9], b \in [-64.7, -61.4], c \in [-305, -296], \text{ and } r \in [-1157, -1149]. \)

 You divided by the opposite of the factor AND multipled the first factor rather than just bringing it down.
\item \( a \in [0, 12], b \in [15, 17], c \in [13, 21], \text{ and } r \in [121, 133]. \)

 You divided by the opposite of the factor.
\end{enumerate}

\textbf{General Comment:} Be sure to synthetically divide by the zero of the denominator! Also, make sure to include 0 placeholders for missing terms.
}
\litem{
Perform the division below. Then, find the intervals that correspond to the quotient in the form $ax^2+bx+c$ and remainder $r$.
\[ \frac{10x^{3} -83 x^{2} +185 x -97}{x -5} \]The solution is \( 10x^{2} -33 x + 20 + \frac{3}{x -5} \), which is option B.\begin{enumerate}[label=\Alph*.]
\item \( a \in [5, 18], \text{   } b \in [-47, -38], \text{   } c \in [13, 15], \text{   and   } r \in [-50, -41]. \)

 You multiplied by the synthetic number and subtracted rather than adding during synthetic division.
\item \( a \in [5, 18], \text{   } b \in [-39, -28], \text{   } c \in [20, 28], \text{   and   } r \in [1, 4]. \)

* This is the solution!
\item \( a \in [49, 56], \text{   } b \in [163, 172], \text{   } c \in [1018, 1024], \text{   and   } r \in [4997, 5007]. \)

 You multiplied by the synthetic number rather than bringing the first factor down.
\item \( a \in [5, 18], \text{   } b \in [-136, -132], \text{   } c \in [847, 853], \text{   and   } r \in [-4349, -4342]. \)

 You divided by the opposite of the factor.
\item \( a \in [49, 56], \text{   } b \in [-333, -331], \text{   } c \in [1850, 1852], \text{   and   } r \in [-9348, -9342]. \)

 You divided by the opposite of the factor AND multiplied the first factor rather than just bringing it down.
\end{enumerate}

\textbf{General Comment:} Be sure to synthetically divide by the zero of the denominator!
}
\litem{
Perform the division below. Then, find the intervals that correspond to the quotient in the form $ax^2+bx+c$ and remainder $r$.
\[ \frac{8x^{3} -26 x -16}{x -2} \]The solution is \( 8x^{2} +16 x + 6 + \frac{-4}{x -2} \), which is option A.\begin{enumerate}[label=\Alph*.]
\item \( a \in [3, 12], b \in [15, 18], c \in [-2, 7], \text{ and } r \in [-5, -1]. \)

* This is the solution!
\item \( a \in [14, 18], b \in [-32, -28], c \in [33, 39], \text{ and } r \in [-94, -91]. \)

 You divided by the opposite of the factor AND multipled the first factor rather than just bringing it down.
\item \( a \in [3, 12], b \in [-16, -14], c \in [-2, 7], \text{ and } r \in [-28, -22]. \)

 You divided by the opposite of the factor.
\item \( a \in [14, 18], b \in [25, 38], c \in [33, 39], \text{ and } r \in [58, 66]. \)

 You multipled by the synthetic number rather than bringing the first factor down.
\item \( a \in [3, 12], b \in [4, 12], c \in [-19, -15], \text{ and } r \in [-34, -33]. \)

 You multipled by the synthetic number and subtracted rather than adding during synthetic division.
\end{enumerate}

\textbf{General Comment:} Be sure to synthetically divide by the zero of the denominator! Also, make sure to include 0 placeholders for missing terms.
}
\litem{
Factor the polynomial below completely, knowing that $x + 3$ is a factor. Then, choose the intervals the zeros of the polynomial belong to, where $z_1 \leq z_2 \leq z_3 \leq z_4$. \textit{To make the problem easier, all zeros are between -5 and 5.}
\[ f(x) = 20x^{4} -7 x^{3} -356 x^{2} -515 x -150 \]The solution is \( [-3, -1.25, -0.4, 5] \), which is option B.\begin{enumerate}[label=\Alph*.]
\item \( z_1 \in [-8, -4], \text{   }  z_2 \in [0.78, 1.7], z_3 \in [2.36, 2.74], \text{   and   } z_4 \in [3, 4] \)

 Distractor 3: Corresponds to negatives of all zeros AND inversing rational roots.
\item \( z_1 \in [-3, 1], \text{   }  z_2 \in [-1.93, -0.42], z_3 \in [-0.44, -0.28], \text{   and   } z_4 \in [5, 11] \)

* This is the solution!
\item \( z_1 \in [-3, 1], \text{   }  z_2 \in [-2.71, -2.11], z_3 \in [-0.87, -0.74], \text{   and   } z_4 \in [5, 11] \)

 Distractor 2: Corresponds to inversing rational roots.
\item \( z_1 \in [-8, -4], \text{   }  z_2 \in [-0.41, 0.22], z_3 \in [2.8, 3.08], \text{   and   } z_4 \in [5, 11] \)

 Distractor 4: Corresponds to moving factors from one rational to another.
\item \( z_1 \in [-8, -4], \text{   }  z_2 \in [0.23, 0.55], z_3 \in [1.17, 1.32], \text{   and   } z_4 \in [3, 4] \)

 Distractor 1: Corresponds to negatives of all zeros.
\end{enumerate}

\textbf{General Comment:} Remember to try the middle-most integers first as these normally are the zeros. Also, once you get it to a quadratic, you can use your other factoring techniques to finish factoring.
}
\litem{
Factor the polynomial below completely, knowing that $x + 2$ is a factor. Then, choose the intervals the zeros of the polynomial belong to, where $z_1 \leq z_2 \leq z_3 \leq z_4$. \textit{To make the problem easier, all zeros are between -5 and 5.}
\[ f(x) = 6x^{4} +7 x^{3} -44 x^{2} -28 x + 80 \]The solution is \( [-2.5, -2, 1.333, 2] \), which is option A.\begin{enumerate}[label=\Alph*.]
\item \( z_1 \in [-2.92, -2.27], \text{   }  z_2 \in [-2.01, -1.97], z_3 \in [1.29, 1.44], \text{   and   } z_4 \in [0.97, 2.24] \)

* This is the solution!
\item \( z_1 \in [-4.34, -3.25], \text{   }  z_2 \in [-2.01, -1.97], z_3 \in [0.81, 0.87], \text{   and   } z_4 \in [0.97, 2.24] \)

 Distractor 4: Corresponds to moving factors from one rational to another.
\item \( z_1 \in [-2.23, -1.75], \text{   }  z_2 \in [-1.36, -1.26], z_3 \in [1.96, 2.01], \text{   and   } z_4 \in [2.2, 2.56] \)

 Distractor 1: Corresponds to negatives of all zeros.
\item \( z_1 \in [-2.23, -1.75], \text{   }  z_2 \in [-0.64, -0.23], z_3 \in [0.71, 0.78], \text{   and   } z_4 \in [0.97, 2.24] \)

 Distractor 2: Corresponds to inversing rational roots.
\item \( z_1 \in [-2.23, -1.75], \text{   }  z_2 \in [-0.98, -0.48], z_3 \in [0.33, 0.41], \text{   and   } z_4 \in [0.97, 2.24] \)

 Distractor 3: Corresponds to negatives of all zeros AND inversing rational roots.
\end{enumerate}

\textbf{General Comment:} Remember to try the middle-most integers first as these normally are the zeros. Also, once you get it to a quadratic, you can use your other factoring techniques to finish factoring.
}
\litem{
What are the \textit{possible Integer} roots of the polynomial below?
\[ f(x) = 6x^{3} +6 x^{2} +6 x + 5 \]The solution is \( \pm 1,\pm 5 \), which is option A.\begin{enumerate}[label=\Alph*.]
\item \( \pm 1,\pm 5 \)

* This is the solution \textbf{since we asked for the possible Integer roots}!
\item \( \text{ All combinations of: }\frac{\pm 1,\pm 5}{\pm 1,\pm 2,\pm 3,\pm 6} \)

This would have been the solution \textbf{if asked for the possible Rational roots}!
\item \( \text{ All combinations of: }\frac{\pm 1,\pm 2,\pm 3,\pm 6}{\pm 1,\pm 5} \)

 Distractor 3: Corresponds to the plus or minus of the inverse quotient (an/a0) of the factors. 
\item \( \pm 1,\pm 2,\pm 3,\pm 6 \)

 Distractor 1: Corresponds to the plus or minus factors of a1 only.
\item \( \text{There is no formula or theorem that tells us all possible Integer roots.} \)

 Distractor 4: Corresponds to not recognizing Integers as a subset of Rationals.
\end{enumerate}

\textbf{General Comment:} We have a way to find the possible Rational roots. The possible Integer roots are the Integers in this list.
}
\litem{
Perform the division below. Then, find the intervals that correspond to the quotient in the form $ax^2+bx+c$ and remainder $r$.
\[ \frac{25x^{3} +80 x^{2} +9 x -15}{x + 3} \]The solution is \( 25x^{2} +5 x -6 + \frac{3}{x + 3} \), which is option B.\begin{enumerate}[label=\Alph*.]
\item \( a \in [23, 28], \text{   } b \in [-20, -14], \text{   } c \in [89, 90], \text{   and   } r \in [-372, -364]. \)

 You multiplied by the synthetic number and subtracted rather than adding during synthetic division.
\item \( a \in [23, 28], \text{   } b \in [1, 11], \text{   } c \in [-6, 3], \text{   and   } r \in [3, 6]. \)

* This is the solution!
\item \( a \in [-76, -72], \text{   } b \in [304, 309], \text{   } c \in [-908, -902], \text{   and   } r \in [2702, 2704]. \)

 You multiplied by the synthetic number rather than bringing the first factor down.
\item \( a \in [-76, -72], \text{   } b \in [-149, -135], \text{   } c \in [-429, -423], \text{   and   } r \in [-1293, -1287]. \)

 You divided by the opposite of the factor AND multiplied the first factor rather than just bringing it down.
\item \( a \in [23, 28], \text{   } b \in [154, 156], \text{   } c \in [465, 479], \text{   and   } r \in [1401, 1410]. \)

 You divided by the opposite of the factor.
\end{enumerate}

\textbf{General Comment:} Be sure to synthetically divide by the zero of the denominator!
}
\end{enumerate}

\end{document}