\documentclass{extbook}[14pt]
\usepackage{multicol, enumerate, enumitem, hyperref, color, soul, setspace, parskip, fancyhdr, amssymb, amsthm, amsmath, bbm, latexsym, units, mathtools}
\everymath{\displaystyle}
\usepackage[headsep=0.5cm,headheight=0cm, left=1 in,right= 1 in,top= 1 in,bottom= 1 in]{geometry}
\usepackage{dashrule}  % Package to use the command below to create lines between items
\newcommand{\litem}[1]{\item #1

\rule{\textwidth}{0.4pt}}
\pagestyle{fancy}
\lhead{}
\chead{Answer Key for Progress Quiz 10 Version C}
\rhead{}
\lfoot{6232-9639}
\cfoot{}
\rfoot{Fall 2020}
\begin{document}
\textbf{This key should allow you to understand why you choose the option you did (beyond just getting a question right or wrong). \href{https://xronos.clas.ufl.edu/mac1105spring2020/courseDescriptionAndMisc/Exams/LearningFromResults}{More instructions on how to use this key can be found here}.}

\textbf{If you have a suggestion to make the keys better, \href{https://forms.gle/CZkbZmPbC9XALEE88}{please fill out the short survey here}.}

\textit{Note: This key is auto-generated and may contain issues and/or errors. The keys are reviewed after each exam to ensure grading is done accurately. If there are issues (like duplicate options), they are noted in the offline gradebook. The keys are a work-in-progress to give students as many resources to improve as possible.}

\rule{\textwidth}{0.4pt}

\begin{enumerate}\litem{
Perform the division below. Then, find the intervals that correspond to the quotient in the form $ax^2+bx+c$ and remainder $r$.
\[ \frac{20x^{3} +87 x^{2} -80 x -72}{x + 5} \]

The solution is \( 20x^{2} -13 x -15 + \frac{3}{x + 5} \), which is option E.\begin{enumerate}[label=\Alph*.]
\item \( a \in [19, 26], \text{   } b \in [185, 190], \text{   } c \in [847, 860], \text{   and   } r \in [4200, 4206]. \)

 You divided by the opposite of the factor.
\item \( a \in [19, 26], \text{   } b \in [-42, -29], \text{   } c \in [112, 121], \text{   and   } r \in [-785, -778]. \)

 You multiplied by the synthetic number and subtracted rather than adding during synthetic division.
\item \( a \in [-101, -95], \text{   } b \in [-414, -407], \text{   } c \in [-2147, -2143], \text{   and   } r \in [-10799, -10794]. \)

 You divided by the opposite of the factor AND multiplied the first factor rather than just bringing it down.
\item \( a \in [-101, -95], \text{   } b \in [585, 592], \text{   } c \in [-3015, -3012], \text{   and   } r \in [14999, 15004]. \)

 You multiplied by the synthetic number rather than bringing the first factor down.
\item \( a \in [19, 26], \text{   } b \in [-16, -11], \text{   } c \in [-16, -4], \text{   and   } r \in [-2, 6]. \)

* This is the solution!
\end{enumerate}

\textbf{General Comment:} Be sure to synthetically divide by the zero of the denominator!
}
\litem{
What are the \textit{possible Rational} roots of the polynomial below?
\[ f(x) = 4x^{4} +6 x^{3} +7 x^{2} +7 x + 2 \]

The solution is \( \text{ All combinations of: }\frac{\pm 1,\pm 2}{\pm 1,\pm 2,\pm 4} \), which is option B.\begin{enumerate}[label=\Alph*.]
\item \( \pm 1,\pm 2 \)

This would have been the solution \textbf{if asked for the possible Integer roots}!
\item \( \text{ All combinations of: }\frac{\pm 1,\pm 2}{\pm 1,\pm 2,\pm 4} \)

* This is the solution \textbf{since we asked for the possible Rational roots}!
\item \( \pm 1,\pm 2,\pm 4 \)

 Distractor 1: Corresponds to the plus or minus factors of a1 only.
\item \( \text{ All combinations of: }\frac{\pm 1,\pm 2,\pm 4}{\pm 1,\pm 2} \)

 Distractor 3: Corresponds to the plus or minus of the inverse quotient (an/a0) of the factors. 
\item \( \text{ There is no formula or theorem that tells us all possible Rational roots.} \)

 Distractor 4: Corresponds to not recalling the theorem for rational roots of a polynomial.
\end{enumerate}

\textbf{General Comment:} We have a way to find the possible Rational roots. The possible Integer roots are the Integers in this list.
}
\litem{
Factor the polynomial below completely, knowing that $x+3$ is a factor. Then, choose the intervals the zeros of the polynomial belong to, where $z_1 \leq z_2 \leq z_3 \leq z_4$. \textit{To make the problem easier, all zeros are between -5 and 5.}
\[ f(x) = 20x^{4} +127 x^{3} +46 x^{2} -415 x + 150 \]

The solution is \( [-5, -3, 0.4, 1.25] \), which is option D.\begin{enumerate}[label=\Alph*.]
\item \( z_1 \in [-3.5, -1.5], \text{   }  z_2 \in [-1.11, -0.49], z_3 \in [2.9, 3.27], \text{   and   } z_4 \in [4.3, 6.2] \)

 Distractor 3: Corresponds to negatives of all zeros AND inversing rational roots.
\item \( z_1 \in [-5, -3], \text{   }  z_2 \in [-3.21, -2.66], z_3 \in [0.66, 0.81], \text{   and   } z_4 \in [2.3, 2.6] \)

 Distractor 2: Corresponds to inversing rational roots.
\item \( z_1 \in [-1.25, 2.75], \text{   }  z_2 \in [-0.65, -0.3], z_3 \in [2.9, 3.27], \text{   and   } z_4 \in [4.3, 6.2] \)

 Distractor 1: Corresponds to negatives of all zeros.
\item \( z_1 \in [-5, -3], \text{   }  z_2 \in [-3.21, -2.66], z_3 \in [-0.12, 0.4], \text{   and   } z_4 \in [-0.3, 1.8] \)

* This is the solution!
\item \( z_1 \in [-5, -3], \text{   }  z_2 \in [-0.2, 0.37], z_3 \in [2.9, 3.27], \text{   and   } z_4 \in [4.3, 6.2] \)

 Distractor 4: Corresponds to moving factors from one rational to another.
\end{enumerate}

\textbf{General Comment:} Remember to try the middle-most integers first as these normally are the zeros. Also, once you get it to a quadratic, you can use your other factoring techniques to finish factoring.
}
\litem{
Perform the division below. Then, find the intervals that correspond to the quotient in the form $ax^2+bx+c$ and remainder $r$.
\[ \frac{20x^{3} -76 x^{2} -32 x + 59}{x -4} \]

The solution is \( 20x^{2} +4 x -16 + \frac{-5}{x -4} \), which is option E.\begin{enumerate}[label=\Alph*.]
\item \( a \in [18, 25], \text{   } b \in [-158, -152], \text{   } c \in [589, 597], \text{   and   } r \in [-2311, -2305]. \)

 You divided by the opposite of the factor.
\item \( a \in [76, 85], \text{   } b \in [-399, -395], \text{   } c \in [1550, 1562], \text{   and   } r \in [-6154, -6144]. \)

 You divided by the opposite of the factor AND multiplied the first factor rather than just bringing it down.
\item \( a \in [76, 85], \text{   } b \in [242, 248], \text{   } c \in [943, 945], \text{   and   } r \in [3830, 3837]. \)

 You multiplied by the synthetic number rather than bringing the first factor down.
\item \( a \in [18, 25], \text{   } b \in [-21, -11], \text{   } c \in [-82, -77], \text{   and   } r \in [-181, -175]. \)

 You multiplied by the synthetic number and subtracted rather than adding during synthetic division.
\item \( a \in [18, 25], \text{   } b \in [2, 6], \text{   } c \in [-19, -12], \text{   and   } r \in [-7, -1]. \)

* This is the solution!
\end{enumerate}

\textbf{General Comment:} Be sure to synthetically divide by the zero of the denominator!
}
\litem{
Perform the division below. Then, find the intervals that correspond to the quotient in the form $ax^2+bx+c$ and remainder $r$.
\[ \frac{15x^{3} -35 x^{2} + 24}{x -2} \]

The solution is \( 15x^{2} -5 x -10 + \frac{4}{x -2} \), which is option C.\begin{enumerate}[label=\Alph*.]
\item \( a \in [28, 31], b \in [23, 32], c \in [47, 54], \text{ and } r \in [116, 130]. \)

 You multipled by the synthetic number rather than bringing the first factor down.
\item \( a \in [15, 19], b \in [-22, -14], c \in [-23, -16], \text{ and } r \in [0, 7]. \)

 You multipled by the synthetic number and subtracted rather than adding during synthetic division.
\item \( a \in [15, 19], b \in [-7, -1], c \in [-17, -8], \text{ and } r \in [0, 7]. \)

* This is the solution!
\item \( a \in [15, 19], b \in [-65, -59], c \in [129, 134], \text{ and } r \in [-241, -231]. \)

 You divided by the opposite of the factor.
\item \( a \in [28, 31], b \in [-104, -92], c \in [189, 194], \text{ and } r \in [-362, -355]. \)

 You divided by the opposite of the factor AND multipled the first factor rather than just bringing it down.
\end{enumerate}

\textbf{General Comment:} Be sure to synthetically divide by the zero of the denominator! Also, make sure to include 0 placeholders for missing terms.
}
\litem{
What are the \textit{possible Rational} roots of the polynomial below?
\[ f(x) = 6x^{3} +2 x^{2} +7 x + 7 \]

The solution is \( \text{ All combinations of: }\frac{\pm 1,\pm 7}{\pm 1,\pm 2,\pm 3,\pm 6} \), which is option A.\begin{enumerate}[label=\Alph*.]
\item \( \text{ All combinations of: }\frac{\pm 1,\pm 7}{\pm 1,\pm 2,\pm 3,\pm 6} \)

* This is the solution \textbf{since we asked for the possible Rational roots}!
\item \( \text{ All combinations of: }\frac{\pm 1,\pm 2,\pm 3,\pm 6}{\pm 1,\pm 7} \)

 Distractor 3: Corresponds to the plus or minus of the inverse quotient (an/a0) of the factors. 
\item \( \pm 1,\pm 7 \)

This would have been the solution \textbf{if asked for the possible Integer roots}!
\item \( \pm 1,\pm 2,\pm 3,\pm 6 \)

 Distractor 1: Corresponds to the plus or minus factors of a1 only.
\item \( \text{ There is no formula or theorem that tells us all possible Rational roots.} \)

 Distractor 4: Corresponds to not recalling the theorem for rational roots of a polynomial.
\end{enumerate}

\textbf{General Comment:} We have a way to find the possible Rational roots. The possible Integer roots are the Integers in this list.
}
\litem{
Perform the division below. Then, find the intervals that correspond to the quotient in the form $ax^2+bx+c$ and remainder $r$.
\[ \frac{6x^{3} -42 x + 38}{x + 3} \]

The solution is \( 6x^{2} -18 x + 12 + \frac{2}{x + 3} \), which is option A.\begin{enumerate}[label=\Alph*.]
\item \( a \in [1, 11], b \in [-18, -14], c \in [11, 20], \text{ and } r \in [2, 8]. \)

* This is the solution!
\item \( a \in [-21, -13], b \in [46, 58], c \in [-206, -203], \text{ and } r \in [644, 651]. \)

 You multipled by the synthetic number rather than bringing the first factor down.
\item \( a \in [1, 11], b \in [15, 25], c \in [11, 20], \text{ and } r \in [70, 78]. \)

 You divided by the opposite of the factor.
\item \( a \in [-21, -13], b \in [-59, -48], c \in [-206, -203], \text{ and } r \in [-576, -567]. \)

 You divided by the opposite of the factor AND multipled the first factor rather than just bringing it down.
\item \( a \in [1, 11], b \in [-25, -23], c \in [54, 59], \text{ and } r \in [-181, -170]. \)

 You multipled by the synthetic number and subtracted rather than adding during synthetic division.
\end{enumerate}

\textbf{General Comment:} Be sure to synthetically divide by the zero of the denominator! Also, make sure to include 0 placeholders for missing terms.
}
\litem{
Factor the polynomial below completely. Then, choose the intervals the zeros of the polynomial belong to, where $z_1 \leq z_2 \leq z_3$. \textit{To make the problem easier, all zeros are between -5 and 5.}
\[ f(x) = 6x^{3} +29 x^{2} -20 x -75 \]

The solution is \( [-5, -1.5, 1.6666666666666667] \), which is option D.\begin{enumerate}[label=\Alph*.]
\item \( z_1 \in [-0.76, -0.43], \text{   }  z_2 \in [0.2, 1.1], \text{   and   } z_3 \in [4, 8] \)

 Distractor 3: Corresponds to negatives of all zeros AND inversing rational roots.
\item \( z_1 \in [-1.72, -1.37], \text{   }  z_2 \in [1, 1.8], \text{   and   } z_3 \in [4, 8] \)

 Distractor 1: Corresponds to negatives of all zeros.
\item \( z_1 \in [-5.03, -4.73], \text{   }  z_2 \in [-1.4, 0.4], \text{   and   } z_3 \in [0.6, 1.6] \)

 Distractor 2: Corresponds to inversing rational roots.
\item \( z_1 \in [-5.03, -4.73], \text{   }  z_2 \in [-3.4, -0.7], \text{   and   } z_3 \in [0.67, 2.67] \)

* This is the solution!
\item \( z_1 \in [-1.13, -0.74], \text{   }  z_2 \in [2.5, 3.6], \text{   and   } z_3 \in [4, 8] \)

 Distractor 4: Corresponds to moving factors from one rational to another.
\end{enumerate}

\textbf{General Comment:} Remember to try the middle-most integers first as these normally are the zeros. Also, once you get it to a quadratic, you can use your other factoring techniques to finish factoring.
}
\litem{
Factor the polynomial below completely. Then, choose the intervals the zeros of the polynomial belong to, where $z_1 \leq z_2 \leq z_3$. \textit{To make the problem easier, all zeros are between -5 and 5.}
\[ f(x) = 4x^{3} -49 x -60 \]

The solution is \( [-2.5, -1.5, 4] \), which is option A.\begin{enumerate}[label=\Alph*.]
\item \( z_1 \in [-2.5, -1.5], \text{   }  z_2 \in [-1.97, -0.97], \text{   and   } z_3 \in [3.2, 4.3] \)

* This is the solution!
\item \( z_1 \in [-6, -3], \text{   }  z_2 \in [0.27, 0.52], \text{   and   } z_3 \in [-0.1, 0.8] \)

 Distractor 3: Corresponds to negatives of all zeros AND inversing rational roots.
\item \( z_1 \in [-0.67, 0.33], \text{   }  z_2 \in [-0.59, -0.2], \text{   and   } z_3 \in [3.2, 4.3] \)

 Distractor 2: Corresponds to inversing rational roots.
\item \( z_1 \in [-6, -3], \text{   }  z_2 \in [0.52, 0.99], \text{   and   } z_3 \in [4.2, 6.5] \)

 Distractor 4: Corresponds to moving factors from one rational to another.
\item \( z_1 \in [-6, -3], \text{   }  z_2 \in [1.43, 2.03], \text{   and   } z_3 \in [2, 3.6] \)

 Distractor 1: Corresponds to negatives of all zeros.
\end{enumerate}

\textbf{General Comment:} Remember to try the middle-most integers first as these normally are the zeros. Also, once you get it to a quadratic, you can use your other factoring techniques to finish factoring.
}
\litem{
Factor the polynomial below completely, knowing that $x-3$ is a factor. Then, choose the intervals the zeros of the polynomial belong to, where $z_1 \leq z_2 \leq z_3 \leq z_4$. \textit{To make the problem easier, all zeros are between -5 and 5.}
\[ f(x) = 6x^{4} -7 x^{3} -118 x^{2} +305 x -150 \]

The solution is \( [-5, 0.6666666666666666, 2.5, 3] \), which is option E.\begin{enumerate}[label=\Alph*.]
\item \( z_1 \in [-4.1, -2.1], \text{   }  z_2 \in [-1.61, -1.37], z_3 \in [-0.54, -0.06], \text{   and   } z_4 \in [4.9, 6.7] \)

 Distractor 3: Corresponds to negatives of all zeros AND inversing rational roots.
\item \( z_1 \in [-5.9, -3.7], \text{   }  z_2 \in [0.34, 0.44], z_3 \in [1.16, 1.7], \text{   and   } z_4 \in [2.1, 3.9] \)

 Distractor 2: Corresponds to inversing rational roots.
\item \( z_1 \in [-4.1, -2.1], \text{   }  z_2 \in [-2.02, -1.74], z_3 \in [-0.91, -0.7], \text{   and   } z_4 \in [4.9, 6.7] \)

 Distractor 4: Corresponds to moving factors from one rational to another.
\item \( z_1 \in [-4.1, -2.1], \text{   }  z_2 \in [-2.63, -2.41], z_3 \in [-0.68, -0.44], \text{   and   } z_4 \in [4.9, 6.7] \)

 Distractor 1: Corresponds to negatives of all zeros.
\item \( z_1 \in [-5.9, -3.7], \text{   }  z_2 \in [0.46, 0.72], z_3 \in [2.32, 2.9], \text{   and   } z_4 \in [2.1, 3.9] \)

* This is the solution!
\end{enumerate}

\textbf{General Comment:} Remember to try the middle-most integers first as these normally are the zeros. Also, once you get it to a quadratic, you can use your other factoring techniques to finish factoring.
}
\end{enumerate}

\end{document}