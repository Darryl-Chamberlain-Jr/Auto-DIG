\documentclass{extbook}[14pt]
\usepackage{multicol, enumerate, enumitem, hyperref, color, soul, setspace, parskip, fancyhdr, amssymb, amsthm, amsmath, latexsym, units, mathtools}
\everymath{\displaystyle}
\usepackage[headsep=0.5cm,headheight=0cm, left=1 in,right= 1 in,top= 1 in,bottom= 1 in]{geometry}
\usepackage{dashrule}  % Package to use the command below to create lines between items
\newcommand{\litem}[1]{\item #1

\rule{\textwidth}{0.4pt}}
\pagestyle{fancy}
\lhead{}
\chead{Answer Key for Progress Quiz 10 Version C}
\rhead{}
\lfoot{5170-5105}
\cfoot{}
\rfoot{Summer C 2021}
\begin{document}
\textbf{This key should allow you to understand why you choose the option you did (beyond just getting a question right or wrong). \href{https://xronos.clas.ufl.edu/mac1105spring2020/courseDescriptionAndMisc/Exams/LearningFromResults}{More instructions on how to use this key can be found here}.}

\textbf{If you have a suggestion to make the keys better, \href{https://forms.gle/CZkbZmPbC9XALEE88}{please fill out the short survey here}.}

\textit{Note: This key is auto-generated and may contain issues and/or errors. The keys are reviewed after each exam to ensure grading is done accurately. If there are issues (like duplicate options), they are noted in the offline gradebook. The keys are a work-in-progress to give students as many resources to improve as possible.}

\rule{\textwidth}{0.4pt}

\begin{enumerate}\litem{
Choose the \textbf{smallest} set of Complex numbers that the number below belongs to.
\[ \sqrt{\frac{0}{144}}+\sqrt{4}i \]The solution is \( \text{Pure Imaginary} \), which is option D.\begin{enumerate}[label=\Alph*.]
\item \( \text{Rational} \)

These are numbers that can be written as fraction of Integers (e.g., -2/3 + 5)
\item \( \text{Not a Complex Number} \)

This is not a number. The only non-Complex number we know is dividing by 0 as this is not a number!
\item \( \text{Nonreal Complex} \)

This is a Complex number $(a+bi)$ that is not Real (has $i$ as part of the number).
\item \( \text{Pure Imaginary} \)

* This is the correct option!
\item \( \text{Irrational} \)

These cannot be written as a fraction of Integers. Remember: $\pi$ is not an Integer!
\end{enumerate}

\textbf{General Comment:} Be sure to simplify $i^2 = -1$. This may remove the imaginary portion for your number. If you are having trouble, you may want to look at the \textit{Subgroups of the Real Numbers} section.
}
\litem{
Simplify the expression below and choose the interval the simplification is contained within.
\[ 19 - 11^2 + 20 \div 7 * 13 \div 17 \]The solution is \( -99.815 \), which is option D.\begin{enumerate}[label=\Alph*.]
\item \( [141, 142.7] \)

 142.185, which corresponds to an Order of Operations error: multiplying by negative before squaring. For example: $(-3)^2 \neq -3^2$
\item \( [-104.1, -101.6] \)

 -101.987, which corresponds to an Order of Operations error: not reading left-to-right for multiplication/division.
\item \( [136.5, 141.4] \)

 140.013, which corresponds to two Order of Operations errors.
\item \( [-100.3, -98.1] \)

* -99.815, this is the correct option
\item \( \text{None of the above} \)

 You may have gotten this by making an unanticipated error. If you got a value that is not any of the others, please let the coordinator know so they can help you figure out what happened.
\end{enumerate}

\textbf{General Comment:} While you may remember (or were taught) PEMDAS is done in order, it is actually done as P/E/MD/AS. When we are at MD or AS, we read left to right.
}
\litem{
Simplify the expression below into the form $a+bi$. Then, choose the intervals that $a$ and $b$ belong to.
\[ (8 + 6 i)(2 + 3 i) \]The solution is \( -2 + 36 i \), which is option E.\begin{enumerate}[label=\Alph*.]
\item \( a \in [-12, -1] \text{ and } b \in [-45, -33] \)

 $-2 - 36 i$, which corresponds to adding a minus sign in both terms.
\item \( a \in [33, 39] \text{ and } b \in [9, 15] \)

 $34 + 12 i$, which corresponds to adding a minus sign in the first term.
\item \( a \in [14, 21] \text{ and } b \in [17, 23] \)

 $16 + 18 i$, which corresponds to just multiplying the real terms to get the real part of the solution and the coefficients in the complex terms to get the complex part.
\item \( a \in [33, 39] \text{ and } b \in [-13, -3] \)

 $34 - 12 i$, which corresponds to adding a minus sign in the second term.
\item \( a \in [-12, -1] \text{ and } b \in [35, 40] \)

* $-2 + 36 i$, which is the correct option.
\end{enumerate}

\textbf{General Comment:} You can treat $i$ as a variable and distribute. Just remember that $i^2=-1$, so you can continue to reduce after you distribute.
}
\litem{
Simplify the expression below and choose the interval the simplification is contained within.
\[ 19 - 9^2 + 8 \div 5 * 4 \div 20 \]The solution is \( -61.680 \), which is option A.\begin{enumerate}[label=\Alph*.]
\item \( [-61.93, -61.53] \)

* -61.680, this is the correct option
\item \( [100, 100.05] \)

 100.020, which corresponds to two Order of Operations errors.
\item \( [-62.15, -61.94] \)

 -61.980, which corresponds to an Order of Operations error: not reading left-to-right for multiplication/division.
\item \( [100.18, 100.45] \)

 100.320, which corresponds to an Order of Operations error: multiplying by negative before squaring. For example: $(-3)^2 \neq -3^2$
\item \( \text{None of the above} \)

 You may have gotten this by making an unanticipated error. If you got a value that is not any of the others, please let the coordinator know so they can help you figure out what happened.
\end{enumerate}

\textbf{General Comment:} While you may remember (or were taught) PEMDAS is done in order, it is actually done as P/E/MD/AS. When we are at MD or AS, we read left to right.
}
\litem{
Choose the \textbf{smallest} set of Complex numbers that the number below belongs to.
\[ \frac{\sqrt{65}}{18}+\sqrt{-2}i \]The solution is \( \text{Irrational} \), which is option E.\begin{enumerate}[label=\Alph*.]
\item \( \text{Nonreal Complex} \)

This is a Complex number $(a+bi)$ that is not Real (has $i$ as part of the number).
\item \( \text{Not a Complex Number} \)

This is not a number. The only non-Complex number we know is dividing by 0 as this is not a number!
\item \( \text{Pure Imaginary} \)

This is a Complex number $(a+bi)$ that \textbf{only} has an imaginary part like $2i$.
\item \( \text{Rational} \)

These are numbers that can be written as fraction of Integers (e.g., -2/3 + 5)
\item \( \text{Irrational} \)

* This is the correct option!
\end{enumerate}

\textbf{General Comment:} Be sure to simplify $i^2 = -1$. This may remove the imaginary portion for your number. If you are having trouble, you may want to look at the \textit{Subgroups of the Real Numbers} section.
}
\litem{
Simplify the expression below into the form $a+bi$. Then, choose the intervals that $a$ and $b$ belong to.
\[ \frac{-72 - 55 i}{-7 + 3 i} \]The solution is \( 5.84  + 10.36 i \), which is option E.\begin{enumerate}[label=\Alph*.]
\item \( a \in [11, 12] \text{ and } b \in [1.5, 3] \)

 $11.53  + 2.91 i$, which corresponds to forgetting to multiply the conjugate by the numerator and not computing the conjugate correctly.
\item \( a \in [4.5, 7.5] \text{ and } b \in [600.5, 602.5] \)

 $5.84  + 601.00 i$, which corresponds to forgetting to multiply the conjugate by the numerator.
\item \( a \in [338.5, 339.5] \text{ and } b \in [9.5, 12] \)

 $339.00  + 10.36 i$, which corresponds to forgetting to multiply the conjugate by the numerator and using a plus instead of a minus in the denominator.
\item \( a \in [10, 11.5] \text{ and } b \in [-19, -17.5] \)

 $10.29  - 18.33 i$, which corresponds to just dividing the first term by the first term and the second by the second.
\item \( a \in [4.5, 7.5] \text{ and } b \in [9.5, 12] \)

* $5.84  + 10.36 i$, which is the correct option.
\end{enumerate}

\textbf{General Comment:} Multiply the numerator and denominator by the *conjugate* of the denominator, then simplify. For example, if we have $2+3i$, the conjugate is $2-3i$.
}
\litem{
Simplify the expression below into the form $a+bi$. Then, choose the intervals that $a$ and $b$ belong to.
\[ (3 - 10 i)(-5 + 8 i) \]The solution is \( 65 + 74 i \), which is option A.\begin{enumerate}[label=\Alph*.]
\item \( a \in [63, 66] \text{ and } b \in [72, 75] \)

* $65 + 74 i$, which is the correct option.
\item \( a \in [-17, -12] \text{ and } b \in [-84, -79] \)

 $-15 - 80 i$, which corresponds to just multiplying the real terms to get the real part of the solution and the coefficients in the complex terms to get the complex part.
\item \( a \in [63, 66] \text{ and } b \in [-74, -69] \)

 $65 - 74 i$, which corresponds to adding a minus sign in both terms.
\item \( a \in [-102, -92] \text{ and } b \in [24, 29] \)

 $-95 + 26 i$, which corresponds to adding a minus sign in the second term.
\item \( a \in [-102, -92] \text{ and } b \in [-31, -22] \)

 $-95 - 26 i$, which corresponds to adding a minus sign in the first term.
\end{enumerate}

\textbf{General Comment:} You can treat $i$ as a variable and distribute. Just remember that $i^2=-1$, so you can continue to reduce after you distribute.
}
\litem{
Choose the \textbf{smallest} set of Real numbers that the number below belongs to.
\[ \sqrt{\frac{121}{324}} \]The solution is \( \text{Rational} \), which is option C.\begin{enumerate}[label=\Alph*.]
\item \( \text{Integer} \)

These are the negative and positive counting numbers (..., -3, -2, -1, 0, 1, 2, 3, ...)
\item \( \text{Whole} \)

These are the counting numbers with 0 (0, 1, 2, 3, ...)
\item \( \text{Rational} \)

* This is the correct option!
\item \( \text{Not a Real number} \)

These are Nonreal Complex numbers \textbf{OR} things that are not numbers (e.g., dividing by 0).
\item \( \text{Irrational} \)

These cannot be written as a fraction of Integers.
\end{enumerate}

\textbf{General Comment:} First, you \textbf{NEED} to simplify the expression. This question simplifies to $\frac{11}{18}$. 
 
 Be sure you look at the simplified fraction and not just the decimal expansion. Numbers such as 13, 17, and 19 provide \textbf{long but repeating/terminating decimal expansions!} 
 
 The only ways to *not* be a Real number are: dividing by 0 or taking the square root of a negative number. 
 
 Irrational numbers are more than just square root of 3: adding or subtracting values from square root of 3 is also irrational.
}
\litem{
Simplify the expression below into the form $a+bi$. Then, choose the intervals that $a$ and $b$ belong to.
\[ \frac{36 - 88 i}{2 + i} \]The solution is \( -3.20  - 42.40 i \), which is option D.\begin{enumerate}[label=\Alph*.]
\item \( a \in [17.5, 18.5] \text{ and } b \in [-89, -87] \)

 $18.00  - 88.00 i$, which corresponds to just dividing the first term by the first term and the second by the second.
\item \( a \in [-4.5, -2] \text{ and } b \in [-213, -211] \)

 $-3.20  - 212.00 i$, which corresponds to forgetting to multiply the conjugate by the numerator.
\item \( a \in [31, 33.5] \text{ and } b \in [-29.5, -27.5] \)

 $32.00  - 28.00 i$, which corresponds to forgetting to multiply the conjugate by the numerator and not computing the conjugate correctly.
\item \( a \in [-4.5, -2] \text{ and } b \in [-43, -41.5] \)

* $-3.20  - 42.40 i$, which is the correct option.
\item \( a \in [-16.5, -15.5] \text{ and } b \in [-43, -41.5] \)

 $-16.00  - 42.40 i$, which corresponds to forgetting to multiply the conjugate by the numerator and using a plus instead of a minus in the denominator.
\end{enumerate}

\textbf{General Comment:} Multiply the numerator and denominator by the *conjugate* of the denominator, then simplify. For example, if we have $2+3i$, the conjugate is $2-3i$.
}
\litem{
Choose the \textbf{smallest} set of Real numbers that the number below belongs to.
\[ \sqrt{\frac{3600}{36}} \]The solution is \( \text{Whole} \), which is option B.\begin{enumerate}[label=\Alph*.]
\item \( \text{Irrational} \)

These cannot be written as a fraction of Integers.
\item \( \text{Whole} \)

* This is the correct option!
\item \( \text{Integer} \)

These are the negative and positive counting numbers (..., -3, -2, -1, 0, 1, 2, 3, ...)
\item \( \text{Not a Real number} \)

These are Nonreal Complex numbers \textbf{OR} things that are not numbers (e.g., dividing by 0).
\item \( \text{Rational} \)

These are numbers that can be written as fraction of Integers (e.g., -2/3)
\end{enumerate}

\textbf{General Comment:} First, you \textbf{NEED} to simplify the expression. This question simplifies to $60$. 
 
 Be sure you look at the simplified fraction and not just the decimal expansion. Numbers such as 13, 17, and 19 provide \textbf{long but repeating/terminating decimal expansions!} 
 
 The only ways to *not* be a Real number are: dividing by 0 or taking the square root of a negative number. 
 
 Irrational numbers are more than just square root of 3: adding or subtracting values from square root of 3 is also irrational.
}
\end{enumerate}

\end{document}