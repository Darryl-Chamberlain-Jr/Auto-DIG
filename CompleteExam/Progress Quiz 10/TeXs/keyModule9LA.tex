\documentclass{extbook}[14pt]
\usepackage{multicol, enumerate, enumitem, hyperref, color, soul, setspace, parskip, fancyhdr, amssymb, amsthm, amsmath, bbm, latexsym, units, mathtools}
\everymath{\displaystyle}
\usepackage[headsep=0.5cm,headheight=0cm, left=1 in,right= 1 in,top= 1 in,bottom= 1 in]{geometry}
\usepackage{dashrule}  % Package to use the command below to create lines between items
\newcommand{\litem}[1]{\item #1

\rule{\textwidth}{0.4pt}}
\pagestyle{fancy}
\lhead{}
\chead{Answer Key for Progress Quiz 10 Version A}
\rhead{}
\lfoot{6232-9639}
\cfoot{}
\rfoot{Fall 2020}
\begin{document}
\textbf{This key should allow you to understand why you choose the option you did (beyond just getting a question right or wrong). \href{https://xronos.clas.ufl.edu/mac1105spring2020/courseDescriptionAndMisc/Exams/LearningFromResults}{More instructions on how to use this key can be found here}.}

\textbf{If you have a suggestion to make the keys better, \href{https://forms.gle/CZkbZmPbC9XALEE88}{please fill out the short survey here}.}

\textit{Note: This key is auto-generated and may contain issues and/or errors. The keys are reviewed after each exam to ensure grading is done accurately. If there are issues (like duplicate options), they are noted in the offline gradebook. The keys are a work-in-progress to give students as many resources to improve as possible.}

\rule{\textwidth}{0.4pt}

\begin{enumerate}\litem{
Choose the interval below that $f$ composed with $g$ at $x=1$ is in.
\[ f(x) = x^{3} -1 x^{2} +3 x -3 \text{ and } g(x) = -x^{3} -3 x^{2} -x + 3 \]

The solution is \( -21.0 \), which is option A.\begin{enumerate}[label=\Alph*.]
\item \( (f \circ g)(1) \in [-22, -17] \)

* This is the correct solution
\item \( (f \circ g)(1) \in [-6, 0] \)

 Distractor 3: Corresponds to being slightly off from the solution.
\item \( (f \circ g)(1) \in [-31, -27] \)

 Distractor 2: Corresponds to being slightly off from the solution.
\item \( (f \circ g)(1) \in [3, 6] \)

 Distractor 1: Corresponds to reversing the composition.
\item \( \text{It is not possible to compose the two functions.} \)


\end{enumerate}

\textbf{General Comment:} $f$ composed with $g$ at $x$ means $f(g(x))$. The order matters!
}
\litem{
Determine whether the function below is 1-1.
\[ f(x) = 16 x^2 + 128 x + 256 \]

The solution is \( \text{no} \), which is option C.\begin{enumerate}[label=\Alph*.]
\item \( \text{Yes, the function is 1-1.} \)

Corresponds to believing the function passes the Horizontal Line test.
\item \( \text{No, because there is an $x$-value that goes to 2 different $y$-values.} \)

Corresponds to the Vertical Line test, which checks if an expression is a function.
\item \( \text{No, because there is a $y$-value that goes to 2 different $x$-values.} \)

* This is the solution.
\item \( \text{No, because the domain of the function is not $(-\infty, \infty)$.} \)

Corresponds to believing 1-1 means the domain is all Real numbers.
\item \( \text{No, because the range of the function is not $(-\infty, \infty)$.} \)

Corresponds to believing 1-1 means the range is all Real numbers.
\end{enumerate}

\textbf{General Comment:} There are only two valid options: The function is 1-1 OR No because there is a $y$-value that goes to 2 different $x$-values.
}
\litem{
Determine whether the function below is 1-1.
\[ f(x) = (6 x - 30)^3 \]

The solution is \( \text{yes} \), which is option A.\begin{enumerate}[label=\Alph*.]
\item \( \text{Yes, the function is 1-1.} \)

* This is the solution.
\item \( \text{No, because there is an $x$-value that goes to 2 different $y$-values.} \)

Corresponds to the Vertical Line test, which checks if an expression is a function.
\item \( \text{No, because the domain of the function is not $(-\infty, \infty)$.} \)

Corresponds to believing 1-1 means the domain is all Real numbers.
\item \( \text{No, because there is a $y$-value that goes to 2 different $x$-values.} \)

Corresponds to the Horizontal Line test, which this function passes.
\item \( \text{No, because the range of the function is not $(-\infty, \infty)$.} \)

Corresponds to believing 1-1 means the range is all Real numbers.
\end{enumerate}

\textbf{General Comment:} There are only two valid options: The function is 1-1 OR No because there is a $y$-value that goes to 2 different $x$-values.
}
\litem{
Multiply the following functions, then choose the domain of the resulting function from the list below.
\[ f(x) = 9x^{3} +6 x^{2} +9 x + 9 \text{ and } g(x) = \sqrt{-3x-7}  \]

The solution is \( \text{ The domain is all Real numbers less than or equal to} x = -2.3333333333333335. \), which is option C.\begin{enumerate}[label=\Alph*.]
\item \( \text{ The domain is all Real numbers except } x = a, \text{ where } a \in [-4.25, -0.25] \)


\item \( \text{ The domain is all Real numbers greater than or equal to } x = a, \text{ where } a \in [3.25, 4.25] \)


\item \( \text{ The domain is all Real numbers less than or equal to } x = a, \text{ where } a \in [-3.33, -1.33] \)


\item \( \text{ The domain is all Real numbers except } x = a \text{ and } x = b, \text{ where } a \in [6.2, 7.2] \text{ and } b \in [-12.2, -4.2] \)


\item \( \text{ The domain is all Real numbers. } \)


\end{enumerate}

\textbf{General Comment:} The new domain is the intersection of the previous domains.
}
\litem{
Find the inverse of the function below. Then, evaluate the inverse at $x = 10$ and choose the interval that $f^{-1}(10)$ belongs to.
\[ f(x) = \ln{(x-2)}+5 \]

The solution is \( f^{-1}(10) = 150.413 \), which is option D.\begin{enumerate}[label=\Alph*.]
\item \( f^{-1}(10) \in [142.41, 147.41] \)

 This solution corresponds to distractor 3.
\item \( f^{-1}(10) \in [3269015.37, 3269026.37] \)

 This solution corresponds to distractor 1.
\item \( f^{-1}(10) \in [2982.96, 2989.96] \)

 This solution corresponds to distractor 4.
\item \( f^{-1}(10) \in [149.41, 157.41] \)

 This is the solution.
\item \( f^{-1}(10) \in [162756.79, 162766.79] \)

 This solution corresponds to distractor 2.
\end{enumerate}

\textbf{General Comment:} Natural log and exponential functions always have an inverse. Once you switch the $x$ and $y$, use the conversion $ e^y = x \leftrightarrow y=\ln(x)$.
}
\litem{
Choose the interval below that $f$ composed with $g$ at $x=-1$ is in.
\[ f(x) = -2x^{3} -4 x^{2} -4 x \text{ and } g(x) = x^{3} +2 x^{2} -x -2 \]

The solution is \( 0.0 \), which is option C.\begin{enumerate}[label=\Alph*.]
\item \( (f \circ g)(-1) \in [12, 15] \)

 Distractor 1: Corresponds to reversing the composition.
\item \( (f \circ g)(-1) \in [17, 22] \)

 Distractor 3: Corresponds to being slightly off from the solution.
\item \( (f \circ g)(-1) \in [-2, 2] \)

* This is the correct solution
\item \( (f \circ g)(-1) \in [-9, -5] \)

 Distractor 2: Corresponds to being slightly off from the solution.
\item \( \text{It is not possible to compose the two functions.} \)


\end{enumerate}

\textbf{General Comment:} $f$ composed with $g$ at $x$ means $f(g(x))$. The order matters!
}
\litem{
Find the inverse of the function below. Then, evaluate the inverse at $x = 8$ and choose the interval that $f^{-1}(8)$ belongs to.
\[ f(x) = \ln{(x-4)}-2 \]

The solution is \( f^{-1}(8) = 22030.466 \), which is option C.\begin{enumerate}[label=\Alph*.]
\item \( f^{-1}(8) \in [162752.79, 162759.79] \)

 This solution corresponds to distractor 2.
\item \( f^{-1}(8) \in [400.43, 410.43] \)

 This solution corresponds to distractor 1.
\item \( f^{-1}(8) \in [22026.47, 22035.47] \)

 This is the solution.
\item \( f^{-1}(8) \in [51.6, 54.6] \)

 This solution corresponds to distractor 4.
\item \( f^{-1}(8) \in [22021.47, 22023.47] \)

 This solution corresponds to distractor 3.
\end{enumerate}

\textbf{General Comment:} Natural log and exponential functions always have an inverse. Once you switch the $x$ and $y$, use the conversion $ e^y = x \leftrightarrow y=\ln(x)$.
}
\litem{
Find the inverse of the function below (if it exists). Then, evaluate the inverse at $x = 15$ and choose the interval that $f^{-1}(15)$ belongs to.
\[ f(x) = 3 x^2 - 2 \]

The solution is \( \text{ The function is not invertible for all Real numbers. } \), which is option E.\begin{enumerate}[label=\Alph*.]
\item \( f^{-1}(15) \in [1.72, 2.16] \)

 Distractor 2: This corresponds to finding the (nonexistent) inverse and not subtracting by the vertical shift.
\item \( f^{-1}(15) \in [7.7, 8.51] \)

 Distractor 4: This corresponds to both distractors 2 and 3.
\item \( f^{-1}(15) \in [2.3, 2.85] \)

 Distractor 1: This corresponds to trying to find the inverse even though the function is not 1-1. 
\item \( f^{-1}(15) \in [4.85, 5.63] \)

 Distractor 3: This corresponds to finding the (nonexistent) inverse and dividing by a negative.
\item \( \text{ The function is not invertible for all Real numbers. } \)

* This is the correct option.
\end{enumerate}

\textbf{General Comment:} Be sure you check that the function is 1-1 before trying to find the inverse!
}
\litem{
Find the inverse of the function below (if it exists). Then, evaluate the inverse at $x = 12$ and choose the interval the $f^{-1}(12)$ belongs to.
\[ f(x) = \sqrt[3]{4 x - 3} \]

The solution is \( 432.75 \), which is option D.\begin{enumerate}[label=\Alph*.]
\item \( f^{-1}(12) \in [-433.14, -431.76] \)

 This solution corresponds to distractor 2.
\item \( f^{-1}(12) \in [430.43, 432.1] \)

 Distractor 1: This corresponds to 
\item \( f^{-1}(12) \in [-431.68, -430.43] \)

 This solution corresponds to distractor 3.
\item \( f^{-1}(12) \in [432.29, 434.48] \)

* This is the correct solution.
\item \( \text{ The function is not invertible for all Real numbers. } \)

 This solution corresponds to distractor 4.
\end{enumerate}

\textbf{General Comment:} Be sure you check that the function is 1-1 before trying to find the inverse!
}
\litem{
Add the following functions, then choose the domain of the resulting function from the list below.
\[ f(x) = \frac{4}{4x+21} \text{ and } g(x) = \frac{1}{3x-19} \]

The solution is \( \text{ The domain is all Real numbers except } x = -5.25 \text{ and } x = 6.333333333333333 \), which is option D.\begin{enumerate}[label=\Alph*.]
\item \( \text{ The domain is all Real numbers greater than or equal to } x = a, \text{ where } a \in [-10.4, -4.4] \)


\item \( \text{ The domain is all Real numbers except } x = a, \text{ where } a \in [-9.25, -0.25] \)


\item \( \text{ The domain is all Real numbers less than or equal to } x = a, \text{ where } a \in [3.75, 7.75] \)


\item \( \text{ The domain is all Real numbers except } x = a \text{ and } x = b, \text{ where } a \in [-9.25, -3.25] \text{ and } b \in [4.33, 15.33] \)


\item \( \text{ The domain is all Real numbers. } \)


\end{enumerate}

\textbf{General Comment:} The new domain is the intersection of the previous domains.
}
\end{enumerate}

\end{document}