\documentclass[14pt]{extbook}
\usepackage{multicol, enumerate, enumitem, hyperref, color, soul, setspace, parskip, fancyhdr} %General Packages
\usepackage{amssymb, amsthm, amsmath, latexsym, units, mathtools} %Math Packages
\everymath{\displaystyle} %All math in Display Style
% Packages with additional options
\usepackage[headsep=0.5cm,headheight=12pt, left=1 in,right= 1 in,top= 1 in,bottom= 1 in]{geometry}
\usepackage[usenames,dvipsnames]{xcolor}
\usepackage{dashrule}  % Package to use the command below to create lines between items
\newcommand{\litem}[1]{\item#1\hspace*{-1cm}\rule{\textwidth}{0.4pt}}
\pagestyle{fancy}
\lhead{Progress Quiz 10}
\chead{}
\rhead{Version A}
\lfoot{5170-5105}
\cfoot{}
\rfoot{Summer C 2021}
\begin{document}

\begin{enumerate}
\litem{
Factor the polynomial below completely. Then, choose the intervals the zeros of the polynomial belong to, where $z_1 \leq z_2 \leq z_3$. \textit{To make the problem easier, all zeros are between -5 and 5.}\[ f(x) = 8x^{3} -34 x^{2} -39 x + 45 \]\begin{enumerate}[label=\Alph*.]
\item \( z_1 \in [-5.9, -3.5], \text{   }  z_2 \in [-1.82, -1.31], \text{   and   } z_3 \in [0.6, 0.73] \)
\item \( z_1 \in [-1, -0.1], \text{   }  z_2 \in [0.87, 1.6], \text{   and   } z_3 \in [4.82, 5.23] \)
\item \( z_1 \in [-2.7, -0.7], \text{   }  z_2 \in [-0.01, 1.07], \text{   and   } z_3 \in [4.82, 5.23] \)
\item \( z_1 \in [-5.9, -3.5], \text{   }  z_2 \in [-1.01, -0.49], \text{   and   } z_3 \in [1.5, 1.58] \)
\item \( z_1 \in [-5.9, -3.5], \text{   }  z_2 \in [-3.25, -2.68], \text{   and   } z_3 \in [0.18, 0.44] \)

\end{enumerate} }
\litem{
What are the \textit{possible Integer} roots of the polynomial below?\[ f(x) = 4x^{3} +4 x^{2} +5 x + 5 \]\begin{enumerate}[label=\Alph*.]
\item \( \pm 1,\pm 2,\pm 4 \)
\item \( \text{ All combinations of: }\frac{\pm 1,\pm 5}{\pm 1,\pm 2,\pm 4} \)
\item \( \text{ All combinations of: }\frac{\pm 1,\pm 2,\pm 4}{\pm 1,\pm 5} \)
\item \( \pm 1,\pm 5 \)
\item \( \text{There is no formula or theorem that tells us all possible Integer roots.} \)

\end{enumerate} }
\litem{
Factor the polynomial below completely. Then, choose the intervals the zeros of the polynomial belong to, where $z_1 \leq z_2 \leq z_3$. \textit{To make the problem easier, all zeros are between -5 and 5.}\[ f(x) = 9x^{3} +27 x^{2} -82 x + 40 \]\begin{enumerate}[label=\Alph*.]
\item \( z_1 \in [-5.3, -4.79], \text{   }  z_2 \in [0.74, 0.79], \text{   and   } z_3 \in [1.47, 1.57] \)
\item \( z_1 \in [-1.7, -1.42], \text{   }  z_2 \in [-0.8, -0.68], \text{   and   } z_3 \in [4.86, 5.23] \)
\item \( z_1 \in [-1.36, -0.85], \text{   }  z_2 \in [-0.67, -0.56], \text{   and   } z_3 \in [4.86, 5.23] \)
\item \( z_1 \in [-4.43, -3.76], \text{   }  z_2 \in [-0.38, -0.13], \text{   and   } z_3 \in [4.86, 5.23] \)
\item \( z_1 \in [-5.3, -4.79], \text{   }  z_2 \in [0.65, 0.68], \text{   and   } z_3 \in [1.07, 1.37] \)

\end{enumerate} }
\litem{
Perform the division below. Then, find the intervals that correspond to the quotient in the form $ax^2+bx+c$ and remainder $r$.\[ \frac{25x^{3} -105 x^{2} + 83}{x -4} \]\begin{enumerate}[label=\Alph*.]
\item \( a \in [95, 103], b \in [-507, -499], c \in [2020, 2024], \text{ and } r \in [-8000, -7994]. \)
\item \( a \in [22, 30], b \in [-9, -4], c \in [-20, -19], \text{ and } r \in [2, 8]. \)
\item \( a \in [22, 30], b \in [-30, -28], c \in [-96, -87], \text{ and } r \in [-189, -184]. \)
\item \( a \in [22, 30], b \in [-207, -201], c \in [817, 822], \text{ and } r \in [-3198, -3194]. \)
\item \( a \in [95, 103], b \in [292, 297], c \in [1179, 1182], \text{ and } r \in [4802, 4808]. \)

\end{enumerate} }
\litem{
Perform the division below. Then, find the intervals that correspond to the quotient in the form $ax^2+bx+c$ and remainder $r$.\[ \frac{10x^{3} +41 x^{2} +51 x + 22}{x + 2} \]\begin{enumerate}[label=\Alph*.]
\item \( a \in [-22, -19], \text{   } b \in [79, 86], \text{   } c \in [-112, -106], \text{   and   } r \in [241, 249]. \)
\item \( a \in [9, 13], \text{   } b \in [17, 23], \text{   } c \in [6, 10], \text{   and   } r \in [-1, 9]. \)
\item \( a \in [9, 13], \text{   } b \in [9, 15], \text{   } c \in [16, 27], \text{   and   } r \in [-36, -29]. \)
\item \( a \in [9, 13], \text{   } b \in [57, 66], \text{   } c \in [173, 175], \text{   and   } r \in [362, 369]. \)
\item \( a \in [-22, -19], \text{   } b \in [0, 7], \text{   } c \in [52, 56], \text{   and   } r \in [126, 134]. \)

\end{enumerate} }
\litem{
Perform the division below. Then, find the intervals that correspond to the quotient in the form $ax^2+bx+c$ and remainder $r$.\[ \frac{8x^{3} -24 x + 14}{x + 2} \]\begin{enumerate}[label=\Alph*.]
\item \( a \in [6, 14], b \in [-28, -20], c \in [45, 53], \text{ and } r \in [-132, -124]. \)
\item \( a \in [-24, -15], b \in [28, 35], c \in [-88, -87], \text{ and } r \in [182, 199]. \)
\item \( a \in [6, 14], b \in [-19, -14], c \in [1, 14], \text{ and } r \in [-4, 0]. \)
\item \( a \in [6, 14], b \in [16, 23], c \in [1, 14], \text{ and } r \in [30, 36]. \)
\item \( a \in [-24, -15], b \in [-36, -28], c \in [-88, -87], \text{ and } r \in [-166, -160]. \)

\end{enumerate} }
\litem{
Factor the polynomial below completely, knowing that $x -5$ is a factor. Then, choose the intervals the zeros of the polynomial belong to, where $z_1 \leq z_2 \leq z_3 \leq z_4$. \textit{To make the problem easier, all zeros are between -5 and 5.}\[ f(x) = 12x^{4} -47 x^{3} -102 x^{2} +155 x + 150 \]\begin{enumerate}[label=\Alph*.]
\item \( z_1 \in [-2, -1], \text{   }  z_2 \in [-1.51, -1.18], z_3 \in [0.49, 0.63], \text{   and   } z_4 \in [4.6, 5.8] \)
\item \( z_1 \in [-2, -1], \text{   }  z_2 \in [-0.78, -0.61], z_3 \in [1.5, 1.69], \text{   and   } z_4 \in [4.6, 5.8] \)
\item \( z_1 \in [-7, -3], \text{   }  z_2 \in [-0.65, -0.53], z_3 \in [1.28, 1.5], \text{   and   } z_4 \in [1, 2.4] \)
\item \( z_1 \in [-7, -3], \text{   }  z_2 \in [-0.47, -0.33], z_3 \in [1.9, 2.12], \text{   and   } z_4 \in [2.9, 3.6] \)
\item \( z_1 \in [-7, -3], \text{   }  z_2 \in [-1.68, -1.5], z_3 \in [0.69, 0.99], \text{   and   } z_4 \in [1, 2.4] \)

\end{enumerate} }
\litem{
Factor the polynomial below completely, knowing that $x + 4$ is a factor. Then, choose the intervals the zeros of the polynomial belong to, where $z_1 \leq z_2 \leq z_3 \leq z_4$. \textit{To make the problem easier, all zeros are between -5 and 5.}\[ f(x) = 10x^{4} + x^{3} -133 x^{2} +122 x + 120 \]\begin{enumerate}[label=\Alph*.]
\item \( z_1 \in [-3.44, -2.2], \text{   }  z_2 \in [-2.14, -1.9], z_3 \in [0.55, 0.83], \text{   and   } z_4 \in [3.48, 4.06] \)
\item \( z_1 \in [-2.11, -1.78], \text{   }  z_2 \in [-0.44, -0.24], z_3 \in [1.31, 1.72], \text{   and   } z_4 \in [3.48, 4.06] \)
\item \( z_1 \in [-2.11, -1.78], \text{   }  z_2 \in [-0.56, -0.46], z_3 \in [2.94, 3.09], \text{   and   } z_4 \in [3.48, 4.06] \)
\item \( z_1 \in [-4.38, -3.92], \text{   }  z_2 \in [-0.61, -0.56], z_3 \in [1.78, 2.09], \text{   and   } z_4 \in [2.27, 3.18] \)
\item \( z_1 \in [-4.38, -3.92], \text{   }  z_2 \in [-1.79, -1.59], z_3 \in [0.23, 0.47], \text{   and   } z_4 \in [1.69, 2.08] \)

\end{enumerate} }
\litem{
What are the \textit{possible Integer} roots of the polynomial below?\[ f(x) = 5x^{3} +4 x^{2} +4 x + 2 \]\begin{enumerate}[label=\Alph*.]
\item \( \text{ All combinations of: }\frac{\pm 1,\pm 2}{\pm 1,\pm 5} \)
\item \( \pm 1,\pm 2 \)
\item \( \text{ All combinations of: }\frac{\pm 1,\pm 5}{\pm 1,\pm 2} \)
\item \( \pm 1,\pm 5 \)
\item \( \text{There is no formula or theorem that tells us all possible Integer roots.} \)

\end{enumerate} }
\litem{
Perform the division below. Then, find the intervals that correspond to the quotient in the form $ax^2+bx+c$ and remainder $r$.\[ \frac{6x^{3} +5 x^{2} -49 x -55}{x -3} \]\begin{enumerate}[label=\Alph*.]
\item \( a \in [18, 20], \text{   } b \in [57, 63], \text{   } c \in [126, 132], \text{   and   } r \in [328, 334]. \)
\item \( a \in [18, 20], \text{   } b \in [-51, -44], \text{   } c \in [95, 102], \text{   and   } r \in [-349, -348]. \)
\item \( a \in [6, 14], \text{   } b \in [-13, -12], \text{   } c \in [-13, -8], \text{   and   } r \in [-29, -20]. \)
\item \( a \in [6, 14], \text{   } b \in [13, 20], \text{   } c \in [-17, -14], \text{   and   } r \in [-86, -83]. \)
\item \( a \in [6, 14], \text{   } b \in [21, 24], \text{   } c \in [20, 23], \text{   and   } r \in [2, 10]. \)

\end{enumerate} }
\end{enumerate}

\end{document}