\documentclass[14pt]{extbook}
\usepackage{multicol, enumerate, enumitem, hyperref, color, soul, setspace, parskip, fancyhdr} %General Packages
\usepackage{amssymb, amsthm, amsmath, bbm, latexsym, units, mathtools} %Math Packages
\everymath{\displaystyle} %All math in Display Style
% Packages with additional options
\usepackage[headsep=0.5cm,headheight=12pt, left=1 in,right= 1 in,top= 1 in,bottom= 1 in]{geometry}
\usepackage[usenames,dvipsnames]{xcolor}
\usepackage{dashrule}  % Package to use the command below to create lines between items
\newcommand{\litem}[1]{\item#1\hspace*{-1cm}\rule{\textwidth}{0.4pt}}
\pagestyle{fancy}
\lhead{Progress Quiz 10}
\chead{}
\rhead{Version A}
\lfoot{6232-9639}
\cfoot{}
\rfoot{Fall 2020}
\begin{document}

\begin{enumerate}
\litem{
Perform the division below. Then, find the intervals that correspond to the quotient in the form $ax^2+bx+c$ and remainder $r$.\[ \frac{9x^{3} -6 x^{2} -51 x -39}{x -3} \]\begin{enumerate}[label=\Alph*.]
\item \( a \in [7, 16], \text{   } b \in [20, 28], \text{   } c \in [7, 14], \text{   and   } r \in [-5, 1]. \)
\item \( a \in [7, 16], \text{   } b \in [12, 13], \text{   } c \in [-29, -21], \text{   and   } r \in [-93, -91]. \)
\item \( a \in [22, 30], \text{   } b \in [74, 79], \text{   } c \in [174, 179], \text{   and   } r \in [482, 484]. \)
\item \( a \in [7, 16], \text{   } b \in [-41, -28], \text{   } c \in [47, 55], \text{   and   } r \in [-184, -176]. \)
\item \( a \in [22, 30], \text{   } b \in [-91, -84], \text{   } c \in [209, 214], \text{   and   } r \in [-673, -663]. \)

\end{enumerate} }
\litem{
What are the \textit{possible Integer} roots of the polynomial below?\[ f(x) = 2x^{4} +3 x^{3} +6 x^{2} +5 x + 6 \]\begin{enumerate}[label=\Alph*.]
\item \( \text{ All combinations of: }\frac{\pm 1,\pm 2}{\pm 1,\pm 2,\pm 3,\pm 6} \)
\item \( \pm 1,\pm 2,\pm 3,\pm 6 \)
\item \( \pm 1,\pm 2 \)
\item \( \text{ All combinations of: }\frac{\pm 1,\pm 2,\pm 3,\pm 6}{\pm 1,\pm 2} \)
\item \( \text{There is no formula or theorem that tells us all possible Integer roots.} \)

\end{enumerate} }
\litem{
Factor the polynomial below completely, knowing that $x-2$ is a factor. Then, choose the intervals the zeros of the polynomial belong to, where $z_1 \leq z_2 \leq z_3 \leq z_4$. \textit{To make the problem easier, all zeros are between -5 and 5.}\[ f(x) = 12x^{4} +35 x^{3} -23 x^{2} -140 x -100 \]\begin{enumerate}[label=\Alph*.]
\item \( z_1 \in [-6, -1], \text{   }  z_2 \in [0.2, 0.57], z_3 \in [1.92, 2.12], \text{   and   } z_4 \in [4, 9] \)
\item \( z_1 \in [-6, -1], \text{   }  z_2 \in [1.22, 1.29], z_3 \in [1.54, 1.77], \text{   and   } z_4 \in [0, 4] \)
\item \( z_1 \in [-6, -1], \text{   }  z_2 \in [0.59, 0.96], z_3 \in [0.61, 0.88], \text{   and   } z_4 \in [0, 4] \)
\item \( z_1 \in [-6, -1], \text{   }  z_2 \in [-1.42, -0.64], z_3 \in [-0.94, -0.45], \text{   and   } z_4 \in [0, 4] \)
\item \( z_1 \in [-6, -1], \text{   }  z_2 \in [-1.72, -1.33], z_3 \in [-1.28, -1.01], \text{   and   } z_4 \in [0, 4] \)

\end{enumerate} }
\litem{
Perform the division below. Then, find the intervals that correspond to the quotient in the form $ax^2+bx+c$ and remainder $r$.\[ \frac{15x^{3} +66 x^{2} +15 x -31}{x + 4} \]\begin{enumerate}[label=\Alph*.]
\item \( a \in [-61, -59], \text{   } b \in [302, 308], \text{   } c \in [-1214, -1208], \text{   and   } r \in [4805, 4810]. \)
\item \( a \in [14, 17], \text{   } b \in [120, 132], \text{   } c \in [519, 521], \text{   and   } r \in [2043, 2046]. \)
\item \( a \in [14, 17], \text{   } b \in [-1, 8], \text{   } c \in [-10, -4], \text{   and   } r \in [0, 10]. \)
\item \( a \in [-61, -59], \text{   } b \in [-174, -173], \text{   } c \in [-681, -677], \text{   and   } r \in [-2762, -2750]. \)
\item \( a \in [14, 17], \text{   } b \in [-15, -8], \text{   } c \in [60, 62], \text{   and   } r \in [-333, -330]. \)

\end{enumerate} }
\litem{
Perform the division below. Then, find the intervals that correspond to the quotient in the form $ax^2+bx+c$ and remainder $r$.\[ \frac{4x^{3} +12 x^{2} -11}{x + 2} \]\begin{enumerate}[label=\Alph*.]
\item \( a \in [2, 7], b \in [3.6, 4.5], c \in [-10, -6], \text{ and } r \in [3, 11]. \)
\item \( a \in [2, 7], b \in [-2, 1.1], c \in [-3, 3], \text{ and } r \in [-11, -9]. \)
\item \( a \in [-13, -7], b \in [26.7, 30.3], c \in [-56, -51], \text{ and } r \in [98, 104]. \)
\item \( a \in [-13, -7], b \in [-7.6, -3.8], c \in [-10, -6], \text{ and } r \in [-27, -25]. \)
\item \( a \in [2, 7], b \in [15.7, 20.2], c \in [39, 41], \text{ and } r \in [68, 73]. \)

\end{enumerate} }
\litem{
What are the \textit{possible Rational} roots of the polynomial below?\[ f(x) = 2x^{2} +3 x + 7 \]\begin{enumerate}[label=\Alph*.]
\item \( \pm 1,\pm 7 \)
\item \( \pm 1,\pm 2 \)
\item \( \text{ All combinations of: }\frac{\pm 1,\pm 7}{\pm 1,\pm 2} \)
\item \( \text{ All combinations of: }\frac{\pm 1,\pm 2}{\pm 1,\pm 7} \)
\item \( \text{ There is no formula or theorem that tells us all possible Rational roots.} \)

\end{enumerate} }
\litem{
Perform the division below. Then, find the intervals that correspond to the quotient in the form $ax^2+bx+c$ and remainder $r$.\[ \frac{12x^{3} +65 x^{2} -122}{x + 5} \]\begin{enumerate}[label=\Alph*.]
\item \( a \in [10, 14], b \in [5, 9], c \in [-28, -24], \text{ and } r \in [2, 5]. \)
\item \( a \in [-63, -57], b \in [363, 367], c \in [-1829, -1822], \text{ and } r \in [9003, 9008]. \)
\item \( a \in [10, 14], b \in [115, 126], c \in [624, 629], \text{ and } r \in [2998, 3004]. \)
\item \( a \in [-63, -57], b \in [-238, -228], c \in [-1181, -1172], \text{ and } r \in [-5999, -5996]. \)
\item \( a \in [10, 14], b \in [-9, -6], c \in [40, 45], \text{ and } r \in [-378, -373]. \)

\end{enumerate} }
\litem{
Factor the polynomial below completely. Then, choose the intervals the zeros of the polynomial belong to, where $z_1 \leq z_2 \leq z_3$. \textit{To make the problem easier, all zeros are between -5 and 5.}\[ f(x) = 15x^{3} +31 x^{2} -50 x -24 \]\begin{enumerate}[label=\Alph*.]
\item \( z_1 \in [-1.8, -1.17], \text{   }  z_2 \in [0.16, 0.41], \text{   and   } z_3 \in [2.57, 3.15] \)
\item \( z_1 \in [-1.24, -0.6], \text{   }  z_2 \in [2.46, 2.75], \text{   and   } z_3 \in [2.57, 3.15] \)
\item \( z_1 \in [-3.68, -2.89], \text{   }  z_2 \in [-3.01, -2.32], \text{   and   } z_3 \in [0.45, 1.14] \)
\item \( z_1 \in [-4.14, -3.63], \text{   }  z_2 \in [-0.24, 0.29], \text{   and   } z_3 \in [2.57, 3.15] \)
\item \( z_1 \in [-3.68, -2.89], \text{   }  z_2 \in [-0.43, -0.33], \text{   and   } z_3 \in [1.27, 1.74] \)

\end{enumerate} }
\litem{
Factor the polynomial below completely. Then, choose the intervals the zeros of the polynomial belong to, where $z_1 \leq z_2 \leq z_3$. \textit{To make the problem easier, all zeros are between -5 and 5.}\[ f(x) = 15x^{3} -64 x^{2} +12 x + 16 \]\begin{enumerate}[label=\Alph*.]
\item \( z_1 \in [-5, -3], \text{   }  z_2 \in [-0.78, -0.27], \text{   and   } z_3 \in [0.22, 0.71] \)
\item \( z_1 \in [-5, -3], \text{   }  z_2 \in [-1.61, -0.92], \text{   and   } z_3 \in [2.35, 2.71] \)
\item \( z_1 \in [-5, -3], \text{   }  z_2 \in [-2.24, -1.98], \text{   and   } z_3 \in [0.04, 0.16] \)
\item \( z_1 \in [-2.5, -1.5], \text{   }  z_2 \in [1.14, 1.56], \text{   and   } z_3 \in [3.73, 4.19] \)
\item \( z_1 \in [-1.4, 1.6], \text{   }  z_2 \in [0.54, 1.28], \text{   and   } z_3 \in [3.73, 4.19] \)

\end{enumerate} }
\litem{
Factor the polynomial below completely, knowing that $x+2$ is a factor. Then, choose the intervals the zeros of the polynomial belong to, where $z_1 \leq z_2 \leq z_3 \leq z_4$. \textit{To make the problem easier, all zeros are between -5 and 5.}\[ f(x) = 20x^{4} +129 x^{3} +194 x^{2} -48 x -160 \]\begin{enumerate}[label=\Alph*.]
\item \( z_1 \in [-4.33, -3.88], \text{   }  z_2 \in [0.19, 0.53], z_3 \in [1.37, 2.44], \text{   and   } z_4 \in [3.2, 5.3] \)
\item \( z_1 \in [-4.33, -3.88], \text{   }  z_2 \in [-2.12, -1.88], z_3 \in [-1.01, -0.29], \text{   and   } z_4 \in [1, 1.5] \)
\item \( z_1 \in [-4.33, -3.88], \text{   }  z_2 \in [-2.12, -1.88], z_3 \in [-1.98, -1.21], \text{   and   } z_4 \in [0.2, 0.9] \)
\item \( z_1 \in [-0.99, -0.08], \text{   }  z_2 \in [1.24, 1.6], z_3 \in [1.37, 2.44], \text{   and   } z_4 \in [3.2, 5.3] \)
\item \( z_1 \in [-1.61, -1.07], \text{   }  z_2 \in [0.76, 1.2], z_3 \in [1.37, 2.44], \text{   and   } z_4 \in [3.2, 5.3] \)

\end{enumerate} }
\end{enumerate}

\end{document}