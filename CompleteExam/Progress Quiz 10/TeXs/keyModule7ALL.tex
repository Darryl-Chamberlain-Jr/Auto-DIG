\documentclass{extbook}[14pt]
\usepackage{multicol, enumerate, enumitem, hyperref, color, soul, setspace, parskip, fancyhdr, amssymb, amsthm, amsmath, latexsym, units, mathtools}
\everymath{\displaystyle}
\usepackage[headsep=0.5cm,headheight=0cm, left=1 in,right= 1 in,top= 1 in,bottom= 1 in]{geometry}
\usepackage{dashrule}  % Package to use the command below to create lines between items
\newcommand{\litem}[1]{\item #1

\rule{\textwidth}{0.4pt}}
\pagestyle{fancy}
\lhead{}
\chead{Answer Key for Progress Quiz 10 Version ALL}
\rhead{}
\lfoot{5170-5105}
\cfoot{}
\rfoot{Summer C 2021}
\begin{document}
\textbf{This key should allow you to understand why you choose the option you did (beyond just getting a question right or wrong). \href{https://xronos.clas.ufl.edu/mac1105spring2020/courseDescriptionAndMisc/Exams/LearningFromResults}{More instructions on how to use this key can be found here}.}

\textbf{If you have a suggestion to make the keys better, \href{https://forms.gle/CZkbZmPbC9XALEE88}{please fill out the short survey here}.}

\textit{Note: This key is auto-generated and may contain issues and/or errors. The keys are reviewed after each exam to ensure grading is done accurately. If there are issues (like duplicate options), they are noted in the offline gradebook. The keys are a work-in-progress to give students as many resources to improve as possible.}

\rule{\textwidth}{0.4pt}

\begin{enumerate}\litem{
Choose the equation of the function graphed below.

\begin{center}
    \includegraphics[width=0.5\textwidth]{../Figures/rationalGraphToEquationCopyA.png}
\end{center}


The solution is \( f(x) = \frac{-1}{(x - 3)^2} + 2 \), which is option B.\begin{enumerate}[label=\Alph*.]
\item \( f(x) = \frac{1}{x + 3} + 2 \)

Corresponds to thinking the graph was a shifted version of $\frac{1}{x}$, using the general form $f(x) = \frac{a}{(x+h)^2}+k$, and the opposite leading coefficient.
\item \( f(x) = \frac{-1}{(x - 3)^2} + 2 \)

This is the correct option.
\item \( f(x) = \frac{-1}{x - 3} + 2 \)

Corresponds to thinking the graph was a shifted version of $\frac{1}{x}$.
\item \( f(x) = \frac{1}{(x + 3)^2} + 2 \)

Corresponds to using the general form $f(x) = \frac{a}{(x+h)^2}+k$ and the opposite leading coefficient.
\item \( \text{None of the above} \)

This corresponds to believing the vertex of the graph was not correct.
\end{enumerate}

\textbf{General Comment:} Remember that the general form of a basic rational equation is $ f(x) = \frac{a}{(x-h)^n} + k$, where $a$ is the leading coefficient (and in this case, we assume is either $1$ or $-1$), $n$ is the degree (in this case, either $1$ or $2$), and $(h, k)$ is the intersection of the asymptotes.
}
\litem{
Solve the rational equation below. Then, choose the interval(s) that the solution(s) belongs to.
\[ \frac{-88}{99x + 22} + 1 = \frac{-88}{99x + 22} \]The solution is \( \text{all solutions are invalid or lead to complex values in the equation.} \), which is option E.\begin{enumerate}[label=\Alph*.]
\item \( x_1 \in [-1.1, -0.1] \text{ and } x_2 \in [-0.34,-0.04] \)

$x = -0.222 \text{ and } x = -0.222$, which corresponds to getting the correct solution and believing there should be a second solution to the equation.
\item \( x_1 \in [-1.1, -0.1] \text{ and } x_2 \in [0.17,0.28] \)

$x = -0.222 \text{ and } x = 0.222$, which corresponds to getting the correct solution and believing there should be a second solution to the equation.
\item \( x \in [0.1,1.8] \)

$x = 0.222$, which corresponds to not distributing the factor $99x + 22$ correctly when trying to eliminate the fraction.
\item \( x \in [-0.22,2.78] \)

$x = -0.222$, which corresponds to not checking if this value leads to dividing by 0 in the original equation and thus is not a valid solution.
\item \( \text{All solutions lead to invalid or complex values in the equation.} \)

*$x = -0.222$ leads to dividing by 0 in the original equation and thus is not a valid solution, which is the correct option.
\end{enumerate}

\textbf{General Comment:} Distractors are different based on the number of solutions. Remember that after solving, we need to make sure our solution does not make the original equation divide by zero!
}
\litem{
Determine the domain of the function below.
\[ f(x) = \frac{6}{12x^{2} -12} \]The solution is \( \text{All Real numbers except } x = -1.000 \text{ and } x = 1.000. \), which is option E.\begin{enumerate}[label=\Alph*.]
\item \( \text{All Real numbers.} \)

This corresponds to thinking the denominator has complex roots or that rational functions have a domain of all Real numbers.
\item \( \text{All Real numbers except } x = a, \text{ where } a \in [-17.4, -14.6] \)

All Real numbers except $x = -16.000$, which corresponds to removing a distractor value from the denominator.
\item \( \text{All Real numbers except } x = a \text{ and } x = b, \text{ where } a \in [-17.4, -14.6] \text{ and } b \in [8.4, 9.8] \)

All Real numbers except $x = -16.000$ and $x = 9.000$, which corresponds to not factoring the denominator correctly.
\item \( \text{All Real numbers except } x = a, \text{ where } a \in [-3.6, -0.6] \)

All Real numbers except $x = -1.000$, which corresponds to removing only 1 value from the denominator.
\item \( \text{All Real numbers except } x = a \text{ and } x = b, \text{ where } a \in [-3.6, -0.6] \text{ and } b \in [0.1, 2.5] \)

All Real numbers except $x = -1.000$ and $x = 1.000$, which is the correct option.
\end{enumerate}

\textbf{General Comment:} Recall that dividing by zero is not a real number. Therefore the domain is all real numbers \textbf{except} those that make the denominator 0.
}
\litem{
Choose the graph of the equation below.
\[ f(x) = \frac{1}{(x - 1)^2} - 2 \]The solution is the graph below, which is option B.
    \begin{center}
        \includegraphics[width=0.3\textwidth]{../Figures/rationalEquationToGraphCopyBA.png}
    \end{center}\begin{enumerate}[label=\Alph*.]
\begin{multicols}{2}
\item \includegraphics[width = 0.3\textwidth]{../Figures/rationalEquationToGraphCopyAA.png}
\item \includegraphics[width = 0.3\textwidth]{../Figures/rationalEquationToGraphCopyBA.png}
\item \includegraphics[width = 0.3\textwidth]{../Figures/rationalEquationToGraphCopyCA.png}
\item \includegraphics[width = 0.3\textwidth]{../Figures/rationalEquationToGraphCopyDA.png}
\end{multicols}\item None of the above.\end{enumerate}
\textbf{General Comment:} Remember that the general form of a basic rational equation is $ f(x) = \frac{a}{(x-h)^n} + k$, where $a$ is the leading coefficient (and in this case, we assume is either $1$ or $-1$), $n$ is the degree (in this case, either $1$ or $2$), and $(h, k)$ is the intersection of the asymptotes.
}
\litem{
Solve the rational equation below. Then, choose the interval(s) that the solution(s) belongs to.
\[ \frac{4x}{-2x + 3} + \frac{-4x^{2}}{-8x^{2} +4 x + 12} = \frac{7}{4x + 4} \]The solution is \( \text{There are two solutions: } x = 0.570 \text{ and } x = -3.070 \), which is option C.\begin{enumerate}[label=\Alph*.]
\item \( x \in [-1.33,0.28] \)


\item \( x_1 \in [-0.84, 2.52] \text{ and } x_2 \in [-0.5,4.5] \)


\item \( x_1 \in [-0.84, 2.52] \text{ and } x_2 \in [-11.07,0.93] \)

* $x = 0.570 \text{ and } x = -3.070$, which is the correct option.
\item \( \text{All solutions lead to invalid or complex values in the equation.} \)


\item \( x \in [-3.83,-2.24] \)


\end{enumerate}

\textbf{General Comment:} Distractors are different based on the number of solutions. Remember that after solving, we need to make sure our solution does not make the original equation divide by zero!
}
\litem{
Choose the equation of the function graphed below.

\begin{center}
    \includegraphics[width=0.5\textwidth]{../Figures/rationalGraphToEquationA.png}
\end{center}


The solution is \( \text{None of the above as it should be } f(x) = \frac{1}{(x - 3)^2} + 1 \), which is option E.\begin{enumerate}[label=\Alph*.]
\item \( f(x) = \frac{-1}{x - 3} + 8 \)

Corresponds to thinking the graph was a shifted version of $\frac{1}{x}$, using the general form $f(x) = \frac{a}{(x-h)^2}+k$, the opposite leading coefficient, AND not noticing the $y$-value was wrong.
\item \( f(x) = \frac{-1}{(x - 3)^2} + 8 \)

Corresponds to using the general form $f(x) = \frac{a}{(x-h)^2}+k$, the opposite leading coefficient, AND not noticing the $y$-value was wrong.
\item \( f(x) = \frac{1}{(x + 3)^2} + 8 \)

The $x$- and $y$-value of the equation does not match the graph.
\item \( f(x) = \frac{1}{x + 3} + 8 \)

Corresponds to thinking the graph was a shifted version of $\frac{1}{x}$ AND not noticing the $y$-value was wrong.
\item \( \text{None of the above} \)

None of the equation options were the correct equation.
\end{enumerate}

\textbf{General Comment:} Remember that the general form of a basic rational equation is $ f(x) = \frac{a}{(x-h)^n} + k$, where $a$ is the leading coefficient (and in this case, we assume is either $1$ or $-1$), $n$ is the degree (in this case, either $1$ or $2$), and $(h, k)$ is the intersection of the asymptotes.
}
\litem{
Determine the domain of the function below.
\[ f(x) = \frac{6}{9x^{2} -27 x + 18} \]The solution is \( \text{All Real numbers except } x = 1.000 \text{ and } x = 2.000. \), which is option C.\begin{enumerate}[label=\Alph*.]
\item \( \text{All Real numbers.} \)

This corresponds to thinking the denominator has complex roots or that rational functions have a domain of all Real numbers.
\item \( \text{All Real numbers except } x = a, \text{ where } a \in [8.93, 9.88] \)

All Real numbers except $x = 9.000$, which corresponds to removing a distractor value from the denominator.
\item \( \text{All Real numbers except } x = a \text{ and } x = b, \text{ where } a \in [0.73, 1.14] \text{ and } b \in [1.29, 2.13] \)

All Real numbers except $x = 1.000$ and $x = 2.000$, which is the correct option.
\item \( \text{All Real numbers except } x = a, \text{ where } a \in [0.73, 1.14] \)

All Real numbers except $x = 1.000$, which corresponds to removing only 1 value from the denominator.
\item \( \text{All Real numbers except } x = a \text{ and } x = b, \text{ where } a \in [8.93, 9.88] \text{ and } b \in [17.21, 18.34] \)

All Real numbers except $x = 9.000$ and $x = 18.000$, which corresponds to not factoring the denominator correctly.
\end{enumerate}

\textbf{General Comment:} Recall that dividing by zero is not a real number. Therefore the domain is all real numbers \textbf{except} those that make the denominator 0.
}
\litem{
Choose the graph of the equation below.
\[ f(x) = \frac{-1}{x - 3} - 2 \]The solution is the graph below, which is option E.
    \begin{center}
        \includegraphics[width=0.3\textwidth]{../Figures/rationalEquationToGraphEA.png}
    \end{center}\begin{enumerate}[label=\Alph*.]
\begin{multicols}{2}
\item \includegraphics[width = 0.3\textwidth]{../Figures/rationalEquationToGraphAA.png}
\item \includegraphics[width = 0.3\textwidth]{../Figures/rationalEquationToGraphBA.png}
\item \includegraphics[width = 0.3\textwidth]{../Figures/rationalEquationToGraphCA.png}
\item \includegraphics[width = 0.3\textwidth]{../Figures/rationalEquationToGraphDA.png}
\end{multicols}\item None of the above.\end{enumerate}
\textbf{General Comment:} Remember that the general form of a basic rational equation is $ f(x) = \frac{a}{(x-h)^n} + k$, where $a$ is the leading coefficient (and in this case, we assume is either $1$ or $-1$), $n$ is the degree (in this case, either $1$ or $2$), and $(h, k)$ is the intersection of the asymptotes.
}
\litem{
Solve the rational equation below. Then, choose the interval(s) that the solution(s) belongs to.
\[ \frac{7x}{6x + 6} + \frac{-4x^{2}}{36x^{2} +60 x + 24} = \frac{-7}{6x + 4} \]The solution is \( \text{All solutions are invalid or lead to complex values in the equation.} \), which is option B.\begin{enumerate}[label=\Alph*.]
\item \( x_1 \in [-1.14, -0.86] \text{ and } x_2 \in [-0.7,-0.47] \)

$x = -1.000 \text{ and } x = -0.667$, which corresponds to solving $6x + 6 = 0$ and $6x + 4 = 0$ and treating them as solutions to the equation.
\item \( \text{All solutions lead to invalid or complex values in the equation.} \)

* The equation leads to solving $46x^{2} +70 x + 42=0$, which leads to complex solutions. This is the correct option.
\item \( x_1 \in [-1.47, -1.1] \text{ and } x_2 \in [-0.59,-0.06] \)

$x = -1.339 \text{ and } x = -0.183$, which corresponds to making the discriminant from the Quadratic Formula positive to avoid complex solutions.
\item \( x \in [-1.14,-0.86] \)

$x = -1.000$, which corresponds to solving $6x + 6 = 0$ and treating it as a solution to the equation.
\item \( x \in [-0.76,-0.61] \)

$x = -0.667$, which corresponds to solving $6x + 4 = 0$ and treating it as a solution to the equation.
\end{enumerate}

\textbf{General Comment:} Distractors are different based on the number of solutions. Remember that after solving, we need to make sure our solution does not make the original equation divide by zero!
}
\litem{
Solve the rational equation below. Then, choose the interval(s) that the solution(s) belongs to.
\[ \frac{20}{-40x + 20} + 1 = \frac{20}{-40x + 20} \]The solution is \( \text{all solutions are invalid or lead to complex values in the equation.} \), which is option E.\begin{enumerate}[label=\Alph*.]
\item \( x \in [0.5,2.5] \)

$x = 0.500$, which corresponds to not checking if this value leads to dividing by 0 in the original equation and thus is not a valid solution.
\item \( x_1 \in [-1, 0.1] \text{ and } x_2 \in [-0.5,2.5] \)

$x = -0.500 \text{ and } x = 0.500$, which corresponds to getting the correct solution and believing there should be a second solution to the equation.
\item \( x \in [-1,0.1] \)

$x = -0.500$, which corresponds to not distributing the factor $-40x + 20$ correctly when trying to eliminate the fraction.
\item \( x_1 \in [0, 1.1] \text{ and } x_2 \in [-0.5,2.5] \)

$x = 0.500 \text{ and } x = 0.500$, which corresponds to getting the correct solution and believing there should be a second solution to the equation.
\item \( \text{All solutions lead to invalid or complex values in the equation.} \)

*$x = 0.500$ leads to dividing by 0 in the original equation and thus is not a valid solution, which is the correct option.
\end{enumerate}

\textbf{General Comment:} Distractors are different based on the number of solutions. Remember that after solving, we need to make sure our solution does not make the original equation divide by zero!
}
\litem{
Choose the equation of the function graphed below.

\begin{center}
    \includegraphics[width=0.5\textwidth]{../Figures/rationalGraphToEquationCopyB.png}
\end{center}


The solution is \( f(x) = \frac{-1}{x + 2} + 3 \), which is option A.\begin{enumerate}[label=\Alph*.]
\item \( f(x) = \frac{-1}{x + 2} + 3 \)

This is the correct option.
\item \( f(x) = \frac{1}{(x - 2)^2} + 3 \)

Corresponds to thinking the graph was a shifted version of $\frac{1}{x^2}$, using the general form $f(x) = \frac{a}{x+h}+k$, and the opposite leading coefficient.
\item \( f(x) = \frac{-1}{(x + 2)^2} + 3 \)

Corresponds to thinking the graph was a shifted version of $\frac{1}{x^2}$.
\item \( f(x) = \frac{1}{x - 2} + 3 \)

Corresponds to using the general form $f(x) = \frac{a}{x+h}+k$ and the opposite leading coefficient.
\item \( \text{None of the above} \)

This corresponds to believing the vertex of the graph was not correct.
\end{enumerate}

\textbf{General Comment:} Remember that the general form of a basic rational equation is $ f(x) = \frac{a}{(x-h)^n} + k$, where $a$ is the leading coefficient (and in this case, we assume is either $1$ or $-1$), $n$ is the degree (in this case, either $1$ or $2$), and $(h, k)$ is the intersection of the asymptotes.
}
\litem{
Solve the rational equation below. Then, choose the interval(s) that the solution(s) belongs to.
\[ \frac{-84}{84x + 36} + 1 = \frac{-84}{84x + 36} \]The solution is \( \text{all solutions are invalid or lead to complex values in the equation.} \), which is option E.\begin{enumerate}[label=\Alph*.]
\item \( x \in [-1.43,0.57] \)

$x = -0.429$, which corresponds to not checking if this value leads to dividing by 0 in the original equation and thus is not a valid solution.
\item \( x \in [0,1.3] \)

$x = 0.429$, which corresponds to not distributing the factor $84x + 36$ correctly when trying to eliminate the fraction.
\item \( x_1 \in [-0.7, 0.2] \text{ and } x_2 \in [-0.4,1.7] \)

$x = -0.429 \text{ and } x = 0.429$, which corresponds to getting the correct solution and believing there should be a second solution to the equation.
\item \( x_1 \in [-0.7, 0.2] \text{ and } x_2 \in [-0.9,-0.2] \)

$x = -0.429 \text{ and } x = -0.429$, which corresponds to getting the correct solution and believing there should be a second solution to the equation.
\item \( \text{All solutions lead to invalid or complex values in the equation.} \)

*$x = -0.429$ leads to dividing by 0 in the original equation and thus is not a valid solution, which is the correct option.
\end{enumerate}

\textbf{General Comment:} Distractors are different based on the number of solutions. Remember that after solving, we need to make sure our solution does not make the original equation divide by zero!
}
\litem{
Determine the domain of the function below.
\[ f(x) = \frac{6}{25x^{2} +45 x + 18} \]The solution is \( \text{All Real numbers except } x = -1.200 \text{ and } x = -0.600. \), which is option A.\begin{enumerate}[label=\Alph*.]
\item \( \text{All Real numbers except } x = a \text{ and } x = b, \text{ where } a \in [-2.1, -0.7] \text{ and } b \in [-0.9, -0.2] \)

All Real numbers except $x = -1.200$ and $x = -0.600$, which is the correct option.
\item \( \text{All Real numbers except } x = a, \text{ where } a \in [-30.6, -29.5] \)

All Real numbers except $x = -30.000$, which corresponds to removing a distractor value from the denominator.
\item \( \text{All Real numbers except } x = a \text{ and } x = b, \text{ where } a \in [-30.6, -29.5] \text{ and } b \in [-15.9, -14.6] \)

All Real numbers except $x = -30.000$ and $x = -15.000$, which corresponds to not factoring the denominator correctly.
\item \( \text{All Real numbers.} \)

This corresponds to thinking the denominator has complex roots or that rational functions have a domain of all Real numbers.
\item \( \text{All Real numbers except } x = a, \text{ where } a \in [-2.1, -0.7] \)

All Real numbers except $x = -1.200$, which corresponds to removing only 1 value from the denominator.
\end{enumerate}

\textbf{General Comment:} Recall that dividing by zero is not a real number. Therefore the domain is all real numbers \textbf{except} those that make the denominator 0.
}
\litem{
Choose the graph of the equation below.
\[ f(x) = \frac{1}{(x + 3)^2} - 1 \]The solution is the graph below, which is option D.
    \begin{center}
        \includegraphics[width=0.3\textwidth]{../Figures/rationalEquationToGraphCopyDB.png}
    \end{center}\begin{enumerate}[label=\Alph*.]
\begin{multicols}{2}
\item \includegraphics[width = 0.3\textwidth]{../Figures/rationalEquationToGraphCopyAB.png}
\item \includegraphics[width = 0.3\textwidth]{../Figures/rationalEquationToGraphCopyBB.png}
\item \includegraphics[width = 0.3\textwidth]{../Figures/rationalEquationToGraphCopyCB.png}
\item \includegraphics[width = 0.3\textwidth]{../Figures/rationalEquationToGraphCopyDB.png}
\end{multicols}\item None of the above.\end{enumerate}
\textbf{General Comment:} Remember that the general form of a basic rational equation is $ f(x) = \frac{a}{(x-h)^n} + k$, where $a$ is the leading coefficient (and in this case, we assume is either $1$ or $-1$), $n$ is the degree (in this case, either $1$ or $2$), and $(h, k)$ is the intersection of the asymptotes.
}
\litem{
Solve the rational equation below. Then, choose the interval(s) that the solution(s) belongs to.
\[ \frac{-6x}{-7x -5} + \frac{-3x^{2}}{-14x^{2} -38 x -20} = \frac{-5}{2x + 4} \]The solution is \( \text{There are two solutions: } x = -3.450 \text{ and } x = -0.483 \), which is option D.\begin{enumerate}[label=\Alph*.]
\item \( \text{All solutions lead to invalid or complex values in the equation.} \)


\item \( x_1 \in [-3.93, -2.1] \text{ and } x_2 \in [-1.07,-0.51] \)


\item \( x \in [-2.81,-0.99] \)


\item \( x_1 \in [-3.93, -2.1] \text{ and } x_2 \in [-0.69,-0.27] \)

* $x = -3.450 \text{ and } x = -0.483$, which is the correct option.
\item \( x \in [-0.99,0.12] \)


\end{enumerate}

\textbf{General Comment:} Distractors are different based on the number of solutions. Remember that after solving, we need to make sure our solution does not make the original equation divide by zero!
}
\litem{
Choose the equation of the function graphed below.

\begin{center}
    \includegraphics[width=0.5\textwidth]{../Figures/rationalGraphToEquationB.png}
\end{center}


The solution is \( f(x) = \frac{1}{x + 1} + 1 \), which is option B.\begin{enumerate}[label=\Alph*.]
\item \( f(x) = \frac{-1}{x - 1} + 1 \)

Corresponds to using the general form $f(x) = \frac{a}{x+h}+k$ and the opposite leading coefficient.
\item \( f(x) = \frac{1}{x + 1} + 1 \)

This is the correct option.
\item \( f(x) = \frac{-1}{(x - 1)^2} + 1 \)

Corresponds to thinking the graph was a shifted version of $\frac{1}{x^2}$, using the general form $f(x) = \frac{a}{x+h}+k$, and the opposite leading coefficient.
\item \( f(x) = \frac{1}{(x + 1)^2} + 1 \)

Corresponds to thinking the graph was a shifted version of $\frac{1}{x^2}$.
\item \( \text{None of the above} \)

This corresponds to believing the vertex of the graph was not correct.
\end{enumerate}

\textbf{General Comment:} Remember that the general form of a basic rational equation is $ f(x) = \frac{a}{(x-h)^n} + k$, where $a$ is the leading coefficient (and in this case, we assume is either $1$ or $-1$), $n$ is the degree (in this case, either $1$ or $2$), and $(h, k)$ is the intersection of the asymptotes.
}
\litem{
Determine the domain of the function below.
\[ f(x) = \frac{4}{15x^{2} +24 x + 9} \]The solution is \( \text{All Real numbers except } x = -1.000 \text{ and } x = -0.600. \), which is option D.\begin{enumerate}[label=\Alph*.]
\item \( \text{All Real numbers.} \)

This corresponds to thinking the denominator has complex roots or that rational functions have a domain of all Real numbers.
\item \( \text{All Real numbers except } x = a, \text{ where } a \in [-15.49, -14.95] \)

All Real numbers except $x = -15.000$, which corresponds to removing a distractor value from the denominator.
\item \( \text{All Real numbers except } x = a \text{ and } x = b, \text{ where } a \in [-15.49, -14.95] \text{ and } b \in [-9.13, -8.74] \)

All Real numbers except $x = -15.000$ and $x = -9.000$, which corresponds to not factoring the denominator correctly.
\item \( \text{All Real numbers except } x = a \text{ and } x = b, \text{ where } a \in [-1.44, -0.83] \text{ and } b \in [-0.78, -0.34] \)

All Real numbers except $x = -1.000$ and $x = -0.600$, which is the correct option.
\item \( \text{All Real numbers except } x = a, \text{ where } a \in [-1.44, -0.83] \)

All Real numbers except $x = -1.000$, which corresponds to removing only 1 value from the denominator.
\end{enumerate}

\textbf{General Comment:} Recall that dividing by zero is not a real number. Therefore the domain is all real numbers \textbf{except} those that make the denominator 0.
}
\litem{
Choose the graph of the equation below.
\[ f(x) = \frac{1}{(x + 3)^2} + 3 \]The solution is the graph below, which is option E.
    \begin{center}
        \includegraphics[width=0.3\textwidth]{../Figures/rationalEquationToGraphEB.png}
    \end{center}\begin{enumerate}[label=\Alph*.]
\begin{multicols}{2}
\item \includegraphics[width = 0.3\textwidth]{../Figures/rationalEquationToGraphAB.png}
\item \includegraphics[width = 0.3\textwidth]{../Figures/rationalEquationToGraphBB.png}
\item \includegraphics[width = 0.3\textwidth]{../Figures/rationalEquationToGraphCB.png}
\item \includegraphics[width = 0.3\textwidth]{../Figures/rationalEquationToGraphDB.png}
\end{multicols}\item None of the above.\end{enumerate}
\textbf{General Comment:} Remember that the general form of a basic rational equation is $ f(x) = \frac{a}{(x-h)^n} + k$, where $a$ is the leading coefficient (and in this case, we assume is either $1$ or $-1$), $n$ is the degree (in this case, either $1$ or $2$), and $(h, k)$ is the intersection of the asymptotes.
}
\litem{
Solve the rational equation below. Then, choose the interval(s) that the solution(s) belongs to.
\[ \frac{-5x}{4x + 3} + \frac{-2x^{2}}{-12x^{2} +19 x + 21} = \frac{-7}{-3x + 7} \]The solution is \( \text{All solutions are invalid or lead to complex values in the equation.} \), which is option A.\begin{enumerate}[label=\Alph*.]
\item \( \text{All solutions lead to invalid or complex values in the equation.} \)

* The equation leads to solving $17x^{2} -7 x + 21=0$, which leads to complex solutions. This is the correct option.
\item \( x \in [2.32,2.42] \)

$x = 2.333$, which corresponds to solving $-3x + 7 = 0$ and treating it as a solution to the equation.
\item \( x \in [-0.88,-0.65] \)

$x = -0.750$, which corresponds to solving $4x + 3 = 0$ and treating it as a solution to the equation.
\item \( x_1 \in [-0.97, -0.81] \text{ and } x_2 \in [0.47,1.31] \)

$x = -0.886 \text{ and } x = 1.298$, which corresponds to making the discriminant from the Quadratic Formula positive to avoid complex solutions.
\item \( x_1 \in [-0.88, -0.65] \text{ and } x_2 \in [2.07,3.03] \)

$x = -0.750 \text{ and } x = 2.333$, which corresponds to solving $4x + 3 = 0$ and $-3x + 7 = 0$ and treating them as solutions to the equation.
\end{enumerate}

\textbf{General Comment:} Distractors are different based on the number of solutions. Remember that after solving, we need to make sure our solution does not make the original equation divide by zero!
}
\litem{
Solve the rational equation below. Then, choose the interval(s) that the solution(s) belongs to.
\[ \frac{-24}{60x -24} + 1 = \frac{-24}{60x -24} \]The solution is \( \text{all solutions are invalid or lead to complex values in the equation.} \), which is option D.\begin{enumerate}[label=\Alph*.]
\item \( x_1 \in [-0.5, -0.2] \text{ and } x_2 \in [0.4,2.4] \)

$x = -0.400 \text{ and } x = 0.400$, which corresponds to getting the correct solution and believing there should be a second solution to the equation.
\item \( x_1 \in [0.3, 0.8] \text{ and } x_2 \in [0.4,2.4] \)

$x = 0.400 \text{ and } x = 0.400$, which corresponds to getting the correct solution and believing there should be a second solution to the equation.
\item \( x \in [0.4,1.4] \)

$x = 0.400$, which corresponds to not checking if this value leads to dividing by 0 in the original equation and thus is not a valid solution.
\item \( \text{All solutions lead to invalid or complex values in the equation.} \)

*$x = 0.400$ leads to dividing by 0 in the original equation and thus is not a valid solution, which is the correct option.
\item \( x \in [-0.5,-0.2] \)

$x = -0.400$, which corresponds to not distributing the factor $60x -24$ correctly when trying to eliminate the fraction.
\end{enumerate}

\textbf{General Comment:} Distractors are different based on the number of solutions. Remember that after solving, we need to make sure our solution does not make the original equation divide by zero!
}
\litem{
Choose the equation of the function graphed below.

\begin{center}
    \includegraphics[width=0.5\textwidth]{../Figures/rationalGraphToEquationCopyC.png}
\end{center}


The solution is \( f(x) = \frac{-1}{(x + 2)^2} + 1 \), which is option C.\begin{enumerate}[label=\Alph*.]
\item \( f(x) = \frac{1}{(x - 2)^2} + 1 \)

Corresponds to using the general form $f(x) = \frac{a}{(x+h)^2}+k$ and the opposite leading coefficient.
\item \( f(x) = \frac{-1}{x + 2} + 1 \)

Corresponds to thinking the graph was a shifted version of $\frac{1}{x}$.
\item \( f(x) = \frac{-1}{(x + 2)^2} + 1 \)

This is the correct option.
\item \( f(x) = \frac{1}{x - 2} + 1 \)

Corresponds to thinking the graph was a shifted version of $\frac{1}{x}$, using the general form $f(x) = \frac{a}{(x+h)^2}+k$, and the opposite leading coefficient.
\item \( \text{None of the above} \)

This corresponds to believing the vertex of the graph was not correct.
\end{enumerate}

\textbf{General Comment:} Remember that the general form of a basic rational equation is $ f(x) = \frac{a}{(x-h)^n} + k$, where $a$ is the leading coefficient (and in this case, we assume is either $1$ or $-1$), $n$ is the degree (in this case, either $1$ or $2$), and $(h, k)$ is the intersection of the asymptotes.
}
\litem{
Solve the rational equation below. Then, choose the interval(s) that the solution(s) belongs to.
\[ \frac{8}{-4x + 2} + 2 = \frac{-3}{-32x + 16} \]The solution is \( x = 1.547 \), which is option D.\begin{enumerate}[label=\Alph*.]
\item \( x_1 \in [1, 2.3] \text{ and } x_2 \in [1.83,2.12] \)

$x = 1.547 \text{ and } x = 1.875$, which corresponds to getting the correct solution and believing there should be a second solution to the equation.
\item \( \text{All solutions lead to invalid or complex values in the equation.} \)

This corresponds to thinking $x = 1.547$ leads to dividing by zero in the original equation, which it does not.
\item \( x_1 \in [0.5, 1.2] \text{ and } x_2 \in [1.49,1.64] \)

$x = 0.547 \text{ and } x = 1.547$, which corresponds to getting the correct solution and believing there should be a second solution to the equation.
\item \( x \in [1.55,2.55] \)

* $x = 1.547$, which is the correct option.
\item \( x \in [0.5,1.2] \)

$x = 0.547$, which corresponds to not distributing the factor $-4x + 2$ correctly when trying to eliminate the fraction.
\end{enumerate}

\textbf{General Comment:} Distractors are different based on the number of solutions. Remember that after solving, we need to make sure our solution does not make the original equation divide by zero!
}
\litem{
Determine the domain of the function below.
\[ f(x) = \frac{3}{30x^{2} +10 x -20} \]The solution is \( \text{All Real numbers except } x = -1.000 \text{ and } x = 0.667. \), which is option A.\begin{enumerate}[label=\Alph*.]
\item \( \text{All Real numbers except } x = a \text{ and } x = b, \text{ where } a \in [-1.3, 0.2] \text{ and } b \in [0.2, 1.4] \)

All Real numbers except $x = -1.000$ and $x = 0.667$, which is the correct option.
\item \( \text{All Real numbers except } x = a, \text{ where } a \in [-1.3, 0.2] \)

All Real numbers except $x = -1.000$, which corresponds to removing only 1 value from the denominator.
\item \( \text{All Real numbers except } x = a, \text{ where } a \in [-25.6, -24.6] \)

All Real numbers except $x = -25.000$, which corresponds to removing a distractor value from the denominator.
\item \( \text{All Real numbers except } x = a \text{ and } x = b, \text{ where } a \in [-25.6, -24.6] \text{ and } b \in [22.6, 25.6] \)

All Real numbers except $x = -25.000$ and $x = 24.000$, which corresponds to not factoring the denominator correctly.
\item \( \text{All Real numbers.} \)

This corresponds to thinking the denominator has complex roots or that rational functions have a domain of all Real numbers.
\end{enumerate}

\textbf{General Comment:} Recall that dividing by zero is not a real number. Therefore the domain is all real numbers \textbf{except} those that make the denominator 0.
}
\litem{
Choose the graph of the equation below.
\[ f(x) = \frac{-1}{x - 3} + 1 \]The solution is the graph below, which is option D.
    \begin{center}
        \includegraphics[width=0.3\textwidth]{../Figures/rationalEquationToGraphCopyDC.png}
    \end{center}\begin{enumerate}[label=\Alph*.]
\begin{multicols}{2}
\item \includegraphics[width = 0.3\textwidth]{../Figures/rationalEquationToGraphCopyAC.png}
\item \includegraphics[width = 0.3\textwidth]{../Figures/rationalEquationToGraphCopyBC.png}
\item \includegraphics[width = 0.3\textwidth]{../Figures/rationalEquationToGraphCopyCC.png}
\item \includegraphics[width = 0.3\textwidth]{../Figures/rationalEquationToGraphCopyDC.png}
\end{multicols}\item None of the above.\end{enumerate}
\textbf{General Comment:} Remember that the general form of a basic rational equation is $ f(x) = \frac{a}{(x-h)^n} + k$, where $a$ is the leading coefficient (and in this case, we assume is either $1$ or $-1$), $n$ is the degree (in this case, either $1$ or $2$), and $(h, k)$ is the intersection of the asymptotes.
}
\litem{
Solve the rational equation below. Then, choose the interval(s) that the solution(s) belongs to.
\[ \frac{7x}{-3x -5} + \frac{-4x^{2}}{-9x^{2} -36 x -35} = \frac{7}{3x + 7} \]The solution is \( \text{There are two solutions: } x = -3.535 \text{ and } x = -0.582 \), which is option D.\begin{enumerate}[label=\Alph*.]
\item \( x \in [-1.9,5] \)


\item \( x_1 \in [-4.3, -2.9] \text{ and } x_2 \in [-2.45,-0.81] \)


\item \( x \in [-3.2,-0.7] \)


\item \( x_1 \in [-4.3, -2.9] \text{ and } x_2 \in [-0.67,1.13] \)

* $x = -3.535 \text{ and } x = -0.582$, which is the correct option.
\item \( \text{All solutions lead to invalid or complex values in the equation.} \)


\end{enumerate}

\textbf{General Comment:} Distractors are different based on the number of solutions. Remember that after solving, we need to make sure our solution does not make the original equation divide by zero!
}
\litem{
Choose the equation of the function graphed below.

\begin{center}
    \includegraphics[width=0.5\textwidth]{../Figures/rationalGraphToEquationC.png}
\end{center}


The solution is \( f(x) = \frac{1}{(x + 1)^2} - 2 \), which is option D.\begin{enumerate}[label=\Alph*.]
\item \( f(x) = \frac{-1}{(x - 1)^2} - 2 \)

Corresponds to using the general form $f(x) = \frac{a}{(x+h)^2}+k$ and the opposite leading coefficient.
\item \( f(x) = \frac{-1}{x - 1} - 2 \)

Corresponds to thinking the graph was a shifted version of $\frac{1}{x}$, using the general form $f(x) = \frac{a}{(x+h)^2}+k$, and the opposite leading coefficient.
\item \( f(x) = \frac{1}{x + 1} - 2 \)

Corresponds to thinking the graph was a shifted version of $\frac{1}{x}$.
\item \( f(x) = \frac{1}{(x + 1)^2} - 2 \)

This is the correct option.
\item \( \text{None of the above} \)

This corresponds to believing the vertex of the graph was not correct.
\end{enumerate}

\textbf{General Comment:} Remember that the general form of a basic rational equation is $ f(x) = \frac{a}{(x-h)^n} + k$, where $a$ is the leading coefficient (and in this case, we assume is either $1$ or $-1$), $n$ is the degree (in this case, either $1$ or $2$), and $(h, k)$ is the intersection of the asymptotes.
}
\litem{
Determine the domain of the function below.
\[ f(x) = \frac{3}{12x^{2} -29 x + 15} \]The solution is \( \text{All Real numbers except } x = 0.750 \text{ and } x = 1.667. \), which is option C.\begin{enumerate}[label=\Alph*.]
\item \( \text{All Real numbers except } x = a, \text{ where } a \in [11.1, 14] \)

All Real numbers except $x = 12.000$, which corresponds to removing a distractor value from the denominator.
\item \( \text{All Real numbers except } x = a, \text{ where } a \in [-1.9, 1.2] \)

All Real numbers except $x = 0.750$, which corresponds to removing only 1 value from the denominator.
\item \( \text{All Real numbers except } x = a \text{ and } x = b, \text{ where } a \in [-1.9, 1.2] \text{ and } b \in [1.3, 3.6] \)

All Real numbers except $x = 0.750$ and $x = 1.667$, which is the correct option.
\item \( \text{All Real numbers.} \)

This corresponds to thinking the denominator has complex roots or that rational functions have a domain of all Real numbers.
\item \( \text{All Real numbers except } x = a \text{ and } x = b, \text{ where } a \in [11.1, 14] \text{ and } b \in [12.8, 15.3] \)

All Real numbers except $x = 12.000$ and $x = 15.000$, which corresponds to not factoring the denominator correctly.
\end{enumerate}

\textbf{General Comment:} Recall that dividing by zero is not a real number. Therefore the domain is all real numbers \textbf{except} those that make the denominator 0.
}
\litem{
Choose the graph of the equation below.
\[ f(x) = \frac{1}{(x + 2)^2} + 1 \]The solution is the graph below, which is option E.
    \begin{center}
        \includegraphics[width=0.3\textwidth]{../Figures/rationalEquationToGraphEC.png}
    \end{center}\begin{enumerate}[label=\Alph*.]
\begin{multicols}{2}
\item \includegraphics[width = 0.3\textwidth]{../Figures/rationalEquationToGraphAC.png}
\item \includegraphics[width = 0.3\textwidth]{../Figures/rationalEquationToGraphBC.png}
\item \includegraphics[width = 0.3\textwidth]{../Figures/rationalEquationToGraphCC.png}
\item \includegraphics[width = 0.3\textwidth]{../Figures/rationalEquationToGraphDC.png}
\end{multicols}\item None of the above.\end{enumerate}
\textbf{General Comment:} Remember that the general form of a basic rational equation is $ f(x) = \frac{a}{(x-h)^n} + k$, where $a$ is the leading coefficient (and in this case, we assume is either $1$ or $-1$), $n$ is the degree (in this case, either $1$ or $2$), and $(h, k)$ is the intersection of the asymptotes.
}
\litem{
Solve the rational equation below. Then, choose the interval(s) that the solution(s) belongs to.
\[ \frac{2x}{-7x -7} + \frac{-2x^{2}}{49x^{2} +98 x + 49} = \frac{-5}{-7x -7} \]The solution is \( \text{There are two solutions: } x = -1.928 \text{ and } x = -1.135 \), which is option C.\begin{enumerate}[label=\Alph*.]
\item \( x_1 \in [-2.08, -1.92] \text{ and } x_2 \in [-1.06,-0.56] \)


\item \( x \in [-1.07,-0.9] \)


\item \( x_1 \in [-2.08, -1.92] \text{ and } x_2 \in [-1.23,-1.13] \)

* $x = -1.928 \text{ and } x = -1.135$, which is the correct option.
\item \( \text{All solutions lead to invalid or complex values in the equation.} \)


\item \( x \in [-1.4,-1.1] \)


\end{enumerate}

\textbf{General Comment:} Distractors are different based on the number of solutions. Remember that after solving, we need to make sure our solution does not make the original equation divide by zero!
}
\litem{
Solve the rational equation below. Then, choose the interval(s) that the solution(s) belongs to.
\[ \frac{-4}{8x -6} + 9 = \frac{9}{64x -48} \]The solution is \( x = 0.821 \), which is option A.\begin{enumerate}[label=\Alph*.]
\item \( x \in [-0.18,1.82] \)

* $x = 0.821$, which is the correct option.
\item \( x_1 \in [-2.68, 0.32] \text{ and } x_2 \in [0.55,0.83] \)

$x = -0.679 \text{ and } x = 0.821$, which corresponds to getting the correct solution and believing there should be a second solution to the equation.
\item \( x_1 \in [-0.18, 2.82] \text{ and } x_2 \in [0.84,1.26] \)

$x = 0.821 \text{ and } x = 0.931$, which corresponds to getting the correct solution and believing there should be a second solution to the equation.
\item \( \text{All solutions lead to invalid or complex values in the equation.} \)

This corresponds to thinking $x = 0.821$ leads to dividing by zero in the original equation, which it does not.
\item \( x \in [-2.68,0.32] \)

$x = -0.679$, which corresponds to not distributing the factor $8x -6$ correctly when trying to eliminate the fraction.
\end{enumerate}

\textbf{General Comment:} Distractors are different based on the number of solutions. Remember that after solving, we need to make sure our solution does not make the original equation divide by zero!
}
\end{enumerate}

\end{document}