\documentclass{extbook}[14pt]
\usepackage{multicol, enumerate, enumitem, hyperref, color, soul, setspace, parskip, fancyhdr, amssymb, amsthm, amsmath, latexsym, units, mathtools}
\everymath{\displaystyle}
\usepackage[headsep=0.5cm,headheight=0cm, left=1 in,right= 1 in,top= 1 in,bottom= 1 in]{geometry}
\usepackage{dashrule}  % Package to use the command below to create lines between items
\newcommand{\litem}[1]{\item #1

\rule{\textwidth}{0.4pt}}
\pagestyle{fancy}
\lhead{}
\chead{Answer Key for Progress Quiz 10 Version B}
\rhead{}
\lfoot{5170-5105}
\cfoot{}
\rfoot{Summer C 2021}
\begin{document}
\textbf{This key should allow you to understand why you choose the option you did (beyond just getting a question right or wrong). \href{https://xronos.clas.ufl.edu/mac1105spring2020/courseDescriptionAndMisc/Exams/LearningFromResults}{More instructions on how to use this key can be found here}.}

\textbf{If you have a suggestion to make the keys better, \href{https://forms.gle/CZkbZmPbC9XALEE88}{please fill out the short survey here}.}

\textit{Note: This key is auto-generated and may contain issues and/or errors. The keys are reviewed after each exam to ensure grading is done accurately. If there are issues (like duplicate options), they are noted in the offline gradebook. The keys are a work-in-progress to give students as many resources to improve as possible.}

\rule{\textwidth}{0.4pt}

\begin{enumerate}\litem{
Factor the polynomial below completely. Then, choose the intervals the zeros of the polynomial belong to, where $z_1 \leq z_2 \leq z_3$. \textit{To make the problem easier, all zeros are between -5 and 5.}
\[ f(x) = 20x^{3} -81 x^{2} +102 x -40 \]The solution is \( [0.8, 1.25, 2] \), which is option B.\begin{enumerate}[label=\Alph*.]
\item \( z_1 \in [-4.5, -3.6], \text{   }  z_2 \in [-3.3, -1.7], \text{   and   } z_3 \in [-0.39, 0.33] \)

 Distractor 4: Corresponds to moving factors from one rational to another.
\item \( z_1 \in [0.4, 1.6], \text{   }  z_2 \in [0.6, 2.1], \text{   and   } z_3 \in [1.36, 2.65] \)

* This is the solution!
\item \( z_1 \in [0.4, 1.6], \text{   }  z_2 \in [0.6, 2.1], \text{   and   } z_3 \in [1.36, 2.65] \)

 Distractor 2: Corresponds to inversing rational roots.
\item \( z_1 \in [-3.6, -1.7], \text{   }  z_2 \in [-1.4, -0.8], \text{   and   } z_3 \in [-0.98, -0.56] \)

 Distractor 3: Corresponds to negatives of all zeros AND inversing rational roots.
\item \( z_1 \in [-3.6, -1.7], \text{   }  z_2 \in [-1.4, -0.8], \text{   and   } z_3 \in [-0.98, -0.56] \)

 Distractor 1: Corresponds to negatives of all zeros.
\end{enumerate}

\textbf{General Comment:} Remember to try the middle-most integers first as these normally are the zeros. Also, once you get it to a quadratic, you can use your other factoring techniques to finish factoring.
}
\litem{
What are the \textit{possible Integer} roots of the polynomial below?
\[ f(x) = 2x^{4} +4 x^{3} +5 x^{2} +4 x + 5 \]The solution is \( \pm 1,\pm 5 \), which is option B.\begin{enumerate}[label=\Alph*.]
\item \( \pm 1,\pm 2 \)

 Distractor 1: Corresponds to the plus or minus factors of a1 only.
\item \( \pm 1,\pm 5 \)

* This is the solution \textbf{since we asked for the possible Integer roots}!
\item \( \text{ All combinations of: }\frac{\pm 1,\pm 2}{\pm 1,\pm 5} \)

 Distractor 3: Corresponds to the plus or minus of the inverse quotient (an/a0) of the factors. 
\item \( \text{ All combinations of: }\frac{\pm 1,\pm 5}{\pm 1,\pm 2} \)

This would have been the solution \textbf{if asked for the possible Rational roots}!
\item \( \text{There is no formula or theorem that tells us all possible Integer roots.} \)

 Distractor 4: Corresponds to not recognizing Integers as a subset of Rationals.
\end{enumerate}

\textbf{General Comment:} We have a way to find the possible Rational roots. The possible Integer roots are the Integers in this list.
}
\litem{
Factor the polynomial below completely. Then, choose the intervals the zeros of the polynomial belong to, where $z_1 \leq z_2 \leq z_3$. \textit{To make the problem easier, all zeros are between -5 and 5.}
\[ f(x) = 12x^{3} +41 x^{2} -40 x -48 \]The solution is \( [-4, -0.75, 1.33] \), which is option B.\begin{enumerate}[label=\Alph*.]
\item \( z_1 \in [-0.64, 0.17], \text{   }  z_2 \in [2.2, 3.67], \text{   and   } z_3 \in [3.19, 4.17] \)

 Distractor 4: Corresponds to moving factors from one rational to another.
\item \( z_1 \in [-4.17, -3.99], \text{   }  z_2 \in [-1.28, -0.53], \text{   and   } z_3 \in [1.15, 1.56] \)

* This is the solution!
\item \( z_1 \in [-2.06, -1.17], \text{   }  z_2 \in [0.41, 1.08], \text{   and   } z_3 \in [3.19, 4.17] \)

 Distractor 1: Corresponds to negatives of all zeros.
\item \( z_1 \in [-1.09, -0.54], \text{   }  z_2 \in [1.26, 1.72], \text{   and   } z_3 \in [3.19, 4.17] \)

 Distractor 3: Corresponds to negatives of all zeros AND inversing rational roots.
\item \( z_1 \in [-4.17, -3.99], \text{   }  z_2 \in [-1.92, -0.98], \text{   and   } z_3 \in [0.14, 0.89] \)

 Distractor 2: Corresponds to inversing rational roots.
\end{enumerate}

\textbf{General Comment:} Remember to try the middle-most integers first as these normally are the zeros. Also, once you get it to a quadratic, you can use your other factoring techniques to finish factoring.
}
\litem{
Perform the division below. Then, find the intervals that correspond to the quotient in the form $ax^2+bx+c$ and remainder $r$.
\[ \frac{20x^{3} +65 x^{2} -47}{x + 3} \]The solution is \( 20x^{2} +5 x -15 + \frac{-2}{x + 3} \), which is option A.\begin{enumerate}[label=\Alph*.]
\item \( a \in [16, 24], b \in [-3, 8], c \in [-18, -13], \text{ and } r \in [-3, 5]. \)

* This is the solution!
\item \( a \in [16, 24], b \in [-22, -12], c \in [55, 64], \text{ and } r \in [-290, -285]. \)

 You multipled by the synthetic number and subtracted rather than adding during synthetic division.
\item \( a \in [-64, -53], b \in [-118, -114], c \in [-346, -342], \text{ and } r \in [-1084, -1080]. \)

 You divided by the opposite of the factor AND multipled the first factor rather than just bringing it down.
\item \( a \in [-64, -53], b \in [244, 248], c \in [-736, -732], \text{ and } r \in [2158, 2159]. \)

 You multipled by the synthetic number rather than bringing the first factor down.
\item \( a \in [16, 24], b \in [125, 129], c \in [374, 381], \text{ and } r \in [1073, 1088]. \)

 You divided by the opposite of the factor.
\end{enumerate}

\textbf{General Comment:} Be sure to synthetically divide by the zero of the denominator! Also, make sure to include 0 placeholders for missing terms.
}
\litem{
Perform the division below. Then, find the intervals that correspond to the quotient in the form $ax^2+bx+c$ and remainder $r$.
\[ \frac{8x^{3} -20 x^{2} -56 x + 37}{x -4} \]The solution is \( 8x^{2} +12 x -8 + \frac{5}{x -4} \), which is option B.\begin{enumerate}[label=\Alph*.]
\item \( a \in [31, 33], \text{   } b \in [-150, -144], \text{   } c \in [535, 540], \text{   and   } r \in [-2111, -2106]. \)

 You divided by the opposite of the factor AND multiplied the first factor rather than just bringing it down.
\item \( a \in [4, 10], \text{   } b \in [12, 13], \text{   } c \in [-11, 1], \text{   and   } r \in [4, 11]. \)

* This is the solution!
\item \( a \in [4, 10], \text{   } b \in [-55, -50], \text{   } c \in [148, 155], \text{   and   } r \in [-573, -566]. \)

 You divided by the opposite of the factor.
\item \( a \in [31, 33], \text{   } b \in [107, 113], \text{   } c \in [370, 377], \text{   and   } r \in [1541, 1545]. \)

 You multiplied by the synthetic number rather than bringing the first factor down.
\item \( a \in [4, 10], \text{   } b \in [1, 5], \text{   } c \in [-48, -42], \text{   and   } r \in [-96, -92]. \)

 You multiplied by the synthetic number and subtracted rather than adding during synthetic division.
\end{enumerate}

\textbf{General Comment:} Be sure to synthetically divide by the zero of the denominator!
}
\litem{
Perform the division below. Then, find the intervals that correspond to the quotient in the form $ax^2+bx+c$ and remainder $r$.
\[ \frac{4x^{3} -75 x + 123}{x + 5} \]The solution is \( 4x^{2} -20 x + 25 + \frac{-2}{x + 5} \), which is option E.\begin{enumerate}[label=\Alph*.]
\item \( a \in [-1, 8], b \in [-26, -21], c \in [69, 70], \text{ and } r \in [-294, -289]. \)

 You multipled by the synthetic number and subtracted rather than adding during synthetic division.
\item \( a \in [-24, -19], b \in [99, 102], c \in [-584, -574], \text{ and } r \in [2991, 2999]. \)

 You multipled by the synthetic number rather than bringing the first factor down.
\item \( a \in [-24, -19], b \in [-102, -99], c \in [-584, -574], \text{ and } r \in [-2753, -2750]. \)

 You divided by the opposite of the factor AND multipled the first factor rather than just bringing it down.
\item \( a \in [-1, 8], b \in [19, 21], c \in [19, 27], \text{ and } r \in [245, 253]. \)

 You divided by the opposite of the factor.
\item \( a \in [-1, 8], b \in [-22, -14], c \in [19, 27], \text{ and } r \in [-5, -1]. \)

* This is the solution!
\end{enumerate}

\textbf{General Comment:} Be sure to synthetically divide by the zero of the denominator! Also, make sure to include 0 placeholders for missing terms.
}
\litem{
Factor the polynomial below completely, knowing that $x -4$ is a factor. Then, choose the intervals the zeros of the polynomial belong to, where $z_1 \leq z_2 \leq z_3 \leq z_4$. \textit{To make the problem easier, all zeros are between -5 and 5.}
\[ f(x) = 16x^{4} -16 x^{3} -217 x^{2} +25 x + 300 \]The solution is \( [-3, -1.25, 1.25, 4] \), which is option A.\begin{enumerate}[label=\Alph*.]
\item \( z_1 \in [-3.7, -2.8], \text{   }  z_2 \in [-1.39, -1.21], z_3 \in [1.02, 2.4], \text{   and   } z_4 \in [3.6, 4.4] \)

* This is the solution!
\item \( z_1 \in [-3.7, -2.8], \text{   }  z_2 \in [-1, -0.8], z_3 \in [0.52, 0.96], \text{   and   } z_4 \in [3.6, 4.4] \)

 Distractor 2: Corresponds to inversing rational roots.
\item \( z_1 \in [-5.5, -3.7], \text{   }  z_2 \in [-0.63, -0.22], z_3 \in [2.81, 3.19], \text{   and   } z_4 \in [4.2, 5.6] \)

 Distractor 4: Corresponds to moving factors from one rational to another.
\item \( z_1 \in [-5.5, -3.7], \text{   }  z_2 \in [-1, -0.8], z_3 \in [0.52, 0.96], \text{   and   } z_4 \in [2.2, 3.4] \)

 Distractor 3: Corresponds to negatives of all zeros AND inversing rational roots.
\item \( z_1 \in [-5.5, -3.7], \text{   }  z_2 \in [-1.39, -1.21], z_3 \in [1.02, 2.4], \text{   and   } z_4 \in [2.2, 3.4] \)

 Distractor 1: Corresponds to negatives of all zeros.
\end{enumerate}

\textbf{General Comment:} Remember to try the middle-most integers first as these normally are the zeros. Also, once you get it to a quadratic, you can use your other factoring techniques to finish factoring.
}
\litem{
Factor the polynomial below completely, knowing that $x + 2$ is a factor. Then, choose the intervals the zeros of the polynomial belong to, where $z_1 \leq z_2 \leq z_3 \leq z_4$. \textit{To make the problem easier, all zeros are between -5 and 5.}
\[ f(x) = 9x^{4} -12 x^{3} -92 x^{2} -32 x + 64 \]The solution is \( [-2, -1.333, 0.667, 4] \), which is option A.\begin{enumerate}[label=\Alph*.]
\item \( z_1 \in [-2.7, -1.7], \text{   }  z_2 \in [-1.36, -1.3], z_3 \in [0.54, 0.71], \text{   and   } z_4 \in [3.4, 4.8] \)

* This is the solution!
\item \( z_1 \in [-2.7, -1.7], \text{   }  z_2 \in [-0.78, -0.72], z_3 \in [1.49, 1.6], \text{   and   } z_4 \in [3.4, 4.8] \)

 Distractor 2: Corresponds to inversing rational roots.
\item \( z_1 \in [-4.7, -3.8], \text{   }  z_2 \in [-0.27, -0.17], z_3 \in [1.98, 2.01], \text{   and   } z_4 \in [3.4, 4.8] \)

 Distractor 4: Corresponds to moving factors from one rational to another.
\item \( z_1 \in [-4.7, -3.8], \text{   }  z_2 \in [-1.54, -1.42], z_3 \in [0.7, 0.81], \text{   and   } z_4 \in [0, 2.5] \)

 Distractor 3: Corresponds to negatives of all zeros AND inversing rational roots.
\item \( z_1 \in [-4.7, -3.8], \text{   }  z_2 \in [-0.74, -0.66], z_3 \in [1.28, 1.44], \text{   and   } z_4 \in [0, 2.5] \)

 Distractor 1: Corresponds to negatives of all zeros.
\end{enumerate}

\textbf{General Comment:} Remember to try the middle-most integers first as these normally are the zeros. Also, once you get it to a quadratic, you can use your other factoring techniques to finish factoring.
}
\litem{
What are the \textit{possible Integer} roots of the polynomial below?
\[ f(x) = 3x^{4} +2 x^{3} +5 x^{2} +7 x + 7 \]The solution is \( \pm 1,\pm 7 \), which is option A.\begin{enumerate}[label=\Alph*.]
\item \( \pm 1,\pm 7 \)

* This is the solution \textbf{since we asked for the possible Integer roots}!
\item \( \text{ All combinations of: }\frac{\pm 1,\pm 7}{\pm 1,\pm 3} \)

This would have been the solution \textbf{if asked for the possible Rational roots}!
\item \( \pm 1,\pm 3 \)

 Distractor 1: Corresponds to the plus or minus factors of a1 only.
\item \( \text{ All combinations of: }\frac{\pm 1,\pm 3}{\pm 1,\pm 7} \)

 Distractor 3: Corresponds to the plus or minus of the inverse quotient (an/a0) of the factors. 
\item \( \text{There is no formula or theorem that tells us all possible Integer roots.} \)

 Distractor 4: Corresponds to not recognizing Integers as a subset of Rationals.
\end{enumerate}

\textbf{General Comment:} We have a way to find the possible Rational roots. The possible Integer roots are the Integers in this list.
}
\litem{
Perform the division below. Then, find the intervals that correspond to the quotient in the form $ax^2+bx+c$ and remainder $r$.
\[ \frac{6x^{3} +39 x^{2} +78 x + 49}{x + 3} \]The solution is \( 6x^{2} +21 x + 15 + \frac{4}{x + 3} \), which is option B.\begin{enumerate}[label=\Alph*.]
\item \( a \in [-18, -13], \text{   } b \in [91, 98], \text{   } c \in [-202, -200.4], \text{   and   } r \in [648, 653]. \)

 You multiplied by the synthetic number rather than bringing the first factor down.
\item \( a \in [-2, 8], \text{   } b \in [21, 27], \text{   } c \in [13.7, 15.7], \text{   and   } r \in [2, 11]. \)

* This is the solution!
\item \( a \in [-2, 8], \text{   } b \in [15, 16], \text{   } c \in [17.7, 18.1], \text{   and   } r \in [-24, -16]. \)

 You multiplied by the synthetic number and subtracted rather than adding during synthetic division.
\item \( a \in [-2, 8], \text{   } b \in [57, 60], \text{   } c \in [248.6, 250.1], \text{   and   } r \in [793, 800]. \)

 You divided by the opposite of the factor.
\item \( a \in [-18, -13], \text{   } b \in [-16, -12], \text{   } c \in [32.1, 34.2], \text{   and   } r \in [144, 149]. \)

 You divided by the opposite of the factor AND multiplied the first factor rather than just bringing it down.
\end{enumerate}

\textbf{General Comment:} Be sure to synthetically divide by the zero of the denominator!
}
\end{enumerate}

\end{document}