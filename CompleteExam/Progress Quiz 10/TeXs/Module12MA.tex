\documentclass[14pt]{extbook}
\usepackage{multicol, enumerate, enumitem, hyperref, color, soul, setspace, parskip, fancyhdr} %General Packages
\usepackage{amssymb, amsthm, amsmath, latexsym, units, mathtools} %Math Packages
\everymath{\displaystyle} %All math in Display Style
% Packages with additional options
\usepackage[headsep=0.5cm,headheight=12pt, left=1 in,right= 1 in,top= 1 in,bottom= 1 in]{geometry}
\usepackage[usenames,dvipsnames]{xcolor}
\usepackage{dashrule}  % Package to use the command below to create lines between items
\newcommand{\litem}[1]{\item#1\hspace*{-1cm}\rule{\textwidth}{0.4pt}}
\pagestyle{fancy}
\lhead{Progress Quiz 10}
\chead{}
\rhead{Version A}
\lfoot{5170-5105}
\cfoot{}
\rfoot{Summer C 2021}
\begin{document}

\begin{enumerate}
\litem{
Determine the appropriate model for the graph of points below.
\begin{center}
    \includegraphics[width=0.5\textwidth]{../Figures/identifyModelGraph12A.png}
\end{center}
\begin{enumerate}[label=\Alph*.]
\item \( \text{Non-linear Power model} \)
\item \( \text{Exponential model} \)
\item \( \text{Logarithmic model} \)
\item \( \text{Linear model} \)
\item \( \text{None of the above} \)

\end{enumerate} }
\litem{
For the scenario below, use the model for the volume of a cylinder as $V = \pi r^2 h$.
\begin{center}
    \textit{ Pringles wants to add 24 \text{percent} more chips to their cylinder cans and minimize the design change of their cans. They've decided that the best way to minimize the design change is to increase the radius and height by the same percentage. What should this increase be? }
\end{center}
\begin{enumerate}[label=\Alph*.]
\item \( \text{About } 12 \text{ percent} \)
\item \( \text{About } 11 \text{ percent} \)
\item \( \text{About } 3 \text{ percent} \)
\item \( \text{About } 7 \text{ percent} \)
\item \( \text{None of the above} \)

\end{enumerate} }
\litem{
Solve the modeling problem below, if possible.
\begin{center}
    \textit{ A new virus is spreading throughout the world. There were initially 6 many cases reported, but the number of confirmed cases has tripled every 5 days. How long will it be until there are at least 1000000 confirmed cases? }
\end{center}
\begin{enumerate}[label=\Alph*.]
\item \( \text{About } 25 \text{ days} \)
\item \( \text{About } 61 \text{ days} \)
\item \( \text{About } 55 \text{ days} \)
\item \( \text{About } 24 \text{ days} \)
\item \( \text{There is not enough information to solve the problem.} \)

\end{enumerate} }
\litem{
Solve the modeling problem below, if possible.
\begin{center}
    \textit{ In CHM2045L, Brittany created a 22 liter 18 percent solution of chemical $\chi$ using two different solution percentages of chemical $\chi$. When she went to write her lab report, she realized she forgot to write the amount of each solution she used! If she remembers she used 18 percent and 32 percent solutions, what was the amount she used of the 32 percent solution? }
\end{center}
\begin{enumerate}[label=\Alph*.]
\item \( 22.00 liters \)
\item \( 11.00 liters \)
\item \( 0.00 liters \)
\item \( 11.84 liters \)
\item \( \text{There is not enough information to solve the problem.} \)

\end{enumerate} }
\litem{
Determine the appropriate model for the graph of points below.
\begin{center}
    \includegraphics[width=0.5\textwidth]{../Figures/identifyModelGraph12CopyA.png}
\end{center}
\begin{enumerate}[label=\Alph*.]
\item \( \text{Exponential model} \)
\item \( \text{Logarithmic model} \)
\item \( \text{Linear model} \)
\item \( \text{Non-linear Power model} \)
\item \( \text{None of the above} \)

\end{enumerate} }
\litem{
For the information provided below, construct a linear model that describes her total budget, $B$, as a function of the number of months, $x$ she is at UF.
\begin{center}
    \textit{ Aubrey is a college student going into her first year at UF. She will receive Bright Futures, which covers her tuition plus a \$400 educational expense each year. Before college, Aubrey saved up \$9000. She knows she will need to pay \$1000 in rent a month, \$40 for food a week, and \$64 in other weekly expenses. }
\end{center}
\begin{enumerate}[label=\Alph*.]
\item \( B(x) = 9000 x + 400 \)
\item \( B(x) = 9400 - 1416 x \)
\item \( B(x) = 400 x + 9000 \)
\item \( B(x) = 9400 - 1104 x \)
\item \( \text{None of the above.} \)

\end{enumerate} }
\litem{
Solve the modeling problem below, if possible.
\begin{center}
    \textit{ A new virus is spreading throughout the world. There were initially 4 many cases reported, but the number of confirmed cases has doubled every 3 days. How long will it be until there are at least 1000 confirmed cases? }
\end{center}
\begin{enumerate}[label=\Alph*.]
\item \( \text{About } 10 \text{ days} \)
\item \( \text{About } 24 \text{ days} \)
\item \( \text{About } 9 \text{ days} \)
\item \( \text{About } 17 \text{ days} \)
\item \( \text{There is not enough information to solve the problem.} \)

\end{enumerate} }
\litem{
The temperature of an object, $T$, in a different surrounding temperature $T_s$ will behave according to the formula $T(t) = Ae^{kt} + T_s$, where $t$ is minutes, $A$ is a constant, and k is a constant. Use this formula and the situation below to construct a model that describes the uranium's temperature, $T$, based on the amount of time t (in minutes) that have passed. Choose the correct constant $k$ from the options below.
\begin{center}
    \textit{ Uranium is taken out of the reactor with a temperature of $130^{\circ}$ C and is placed into a $20^{\circ}$ C bath to cool. After 29 minutes, the uranium has cooled to $64^{\circ}$ C. }
\end{center}
\begin{enumerate}[label=\Alph*.]
\item \( k = -0.03736 \)
\item \( k = -0.02224 \)
\item \( k = -0.02289 \)
\item \( k = -0.03266 \)
\item \( \text{None of the above} \)

\end{enumerate} }
\litem{
For the scenario below, use the model for the volume of a cylinder as $V = \pi r^2 h$.
\begin{center}
    \textit{ Pringles wants to add 34 \text{percent} more chips to their cylinder cans and minimize the design change of their cans. They've decided that the best way to minimize the design change is to increase the radius and height by the same percentage. What should this increase be? }
\end{center}
\begin{enumerate}[label=\Alph*.]
\item \( \text{About } 10 \text{ percent} \)
\item \( \text{About } 16 \text{ percent} \)
\item \( \text{About } 3 \text{ percent} \)
\item \( \text{About } 17 \text{ percent} \)
\item \( \text{None of the above} \)

\end{enumerate} }
\litem{
Solve the modeling problem below, if possible.
\begin{center}
    \textit{ In CHM2045L, Brittany created a 24 liter 24 percent solution of chemical $\chi$ using two different solution percentages of chemical $\chi$. When she went to write her lab report, she realized she forgot to write the amount of each solution she used! If she remembers she used 17 percent and 39 percent solutions, what was the amount she used of the 39 percent solution? }
\end{center}
\begin{enumerate}[label=\Alph*.]
\item \( 12.00 liters \)
\item \( 7.64 liters \)
\item \( 16.36 liters \)
\item \( 15.04 liters \)
\item \( \text{There is not enough information to solve the problem.} \)

\end{enumerate} }
\end{enumerate}

\end{document}