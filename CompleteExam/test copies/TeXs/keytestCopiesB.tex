\documentclass{extbook}[14pt]
\usepackage{multicol, enumerate, enumitem, hyperref, color, soul, setspace, parskip, fancyhdr, amssymb, amsthm, amsmath, latexsym, units, mathtools}
\everymath{\displaystyle}
\usepackage[headsep=0.5cm,headheight=0cm, left=1 in,right= 1 in,top= 1 in,bottom= 1 in]{geometry}
\usepackage{dashrule}  % Package to use the command below to create lines between items
\newcommand{\litem}[1]{\item #1

\rule{\textwidth}{0.4pt}}
\pagestyle{fancy}
\lhead{}
\chead{Answer Key for test copies Version B}
\rhead{}
\lfoot{5370-9939}
\cfoot{}
\rfoot{test}
\begin{document}
\textbf{This key should allow you to understand why you choose the option you did (beyond just getting a question right or wrong). \href{https://xronos.clas.ufl.edu/mac1105spring2020/courseDescriptionAndMisc/Exams/LearningFromResults}{More instructions on how to use this key can be found here}.}

\textbf{If you have a suggestion to make the keys better, \href{https://forms.gle/CZkbZmPbC9XALEE88}{please fill out the short survey here}.}

\textit{Note: This key is auto-generated and may contain issues and/or errors. The keys are reviewed after each exam to ensure grading is done accurately. If there are issues (like duplicate options), they are noted in the offline gradebook. The keys are a work-in-progress to give students as many resources to improve as possible.}

\rule{\textwidth}{0.4pt}

\begin{enumerate}\litem{
Simplify the expression below into the form $a+bi$. Then, choose the intervals that $a$ and $b$ belong to.
\[ (6 + 5 i)(-4 + 10 i) \]The solution is \( -74 + 40 i \), which is option B.\begin{enumerate}[label=\Alph*.]
\item \( a \in [22, 27] \text{ and } b \in [72, 87] \)

 $26 + 80 i$, which corresponds to adding a minus sign in the first term.
\item \( a \in [-79, -71] \text{ and } b \in [36, 43] \)

* $-74 + 40 i$, which is the correct option.
\item \( a \in [-79, -71] \text{ and } b \in [-44, -37] \)

 $-74 - 40 i$, which corresponds to adding a minus sign in both terms.
\item \( a \in [-26, -23] \text{ and } b \in [48, 55] \)

 $-24 + 50 i$, which corresponds to just multiplying the real terms to get the real part of the solution and the coefficients in the complex terms to get the complex part.
\item \( a \in [22, 27] \text{ and } b \in [-81, -76] \)

 $26 - 80 i$, which corresponds to adding a minus sign in the second term.
\end{enumerate}

\textbf{General Comment:} You can treat $i$ as a variable and distribute. Just remember that $i^2=-1$, so you can continue to reduce after you distribute.
}
\litem{
Simplify the expression below into the form $a+bi$. Then, choose the intervals that $a$ and $b$ belong to.
\[ \frac{63 - 55 i}{2 + 4 i} \]The solution is \( -4.70  - 18.10 i \), which is option A.\begin{enumerate}[label=\Alph*.]
\item \( a \in [-5, -3.5] \text{ and } b \in [-18.5, -17.5] \)

* $-4.70  - 18.10 i$, which is the correct option.
\item \( a \in [31, 32] \text{ and } b \in [-14.5, -13.5] \)

 $31.50  - 13.75 i$, which corresponds to just dividing the first term by the first term and the second by the second.
\item \( a \in [-5, -3.5] \text{ and } b \in [-363.5, -361.5] \)

 $-4.70  - 362.00 i$, which corresponds to forgetting to multiply the conjugate by the numerator.
\item \( a \in [16.5, 19] \text{ and } b \in [6.5, 7.5] \)

 $17.30  + 7.10 i$, which corresponds to forgetting to multiply the conjugate by the numerator and not computing the conjugate correctly.
\item \( a \in [-95, -93.5] \text{ and } b \in [-18.5, -17.5] \)

 $-94.00  - 18.10 i$, which corresponds to forgetting to multiply the conjugate by the numerator and using a plus instead of a minus in the denominator.
\end{enumerate}

\textbf{General Comment:} Multiply the numerator and denominator by the *conjugate* of the denominator, then simplify. For example, if we have $2+3i$, the conjugate is $2-3i$.
}
\litem{
Simplify the expression below into the form $a+bi$. Then, choose the intervals that $a$ and $b$ belong to.
\[ \frac{-63 + 33 i}{4 - 5 i} \]The solution is \( -10.17  - 4.46 i \), which is option C.\begin{enumerate}[label=\Alph*.]
\item \( a \in [-418.5, -416.5] \text{ and } b \in [-4.5, -2.5] \)

 $-417.00  - 4.46 i$, which corresponds to forgetting to multiply the conjugate by the numerator and using a plus instead of a minus in the denominator.
\item \( a \in [-3, -0.5] \text{ and } b \in [10, 12.5] \)

 $-2.12  + 10.90 i$, which corresponds to forgetting to multiply the conjugate by the numerator and not computing the conjugate correctly.
\item \( a \in [-10.5, -9.5] \text{ and } b \in [-4.5, -2.5] \)

* $-10.17  - 4.46 i$, which is the correct option.
\item \( a \in [-10.5, -9.5] \text{ and } b \in [-184, -181.5] \)

 $-10.17  - 183.00 i$, which corresponds to forgetting to multiply the conjugate by the numerator.
\item \( a \in [-16, -15.5] \text{ and } b \in [-8, -6] \)

 $-15.75  - 6.60 i$, which corresponds to just dividing the first term by the first term and the second by the second.
\end{enumerate}

\textbf{General Comment:} Multiply the numerator and denominator by the *conjugate* of the denominator, then simplify. For example, if we have $2+3i$, the conjugate is $2-3i$.
}
\litem{
Simplify the expression below into the form $a+bi$. Then, choose the intervals that $a$ and $b$ belong to.
\[ (-3 - 2 i)(-4 + 9 i) \]The solution is \( 30 - 19 i \), which is option A.\begin{enumerate}[label=\Alph*.]
\item \( a \in [28, 39] \text{ and } b \in [-20.3, -18.6] \)

* $30 - 19 i$, which is the correct option.
\item \( a \in [-7, -4] \text{ and } b \in [-38.2, -34.6] \)

 $-6 - 35 i$, which corresponds to adding a minus sign in the first term.
\item \( a \in [28, 39] \text{ and } b \in [16, 20.5] \)

 $30 + 19 i$, which corresponds to adding a minus sign in both terms.
\item \( a \in [4, 15] \text{ and } b \in [-18.1, -15.7] \)

 $12 - 18 i$, which corresponds to just multiplying the real terms to get the real part of the solution and the coefficients in the complex terms to get the complex part.
\item \( a \in [-7, -4] \text{ and } b \in [32.5, 35.6] \)

 $-6 + 35 i$, which corresponds to adding a minus sign in the second term.
\end{enumerate}

\textbf{General Comment:} You can treat $i$ as a variable and distribute. Just remember that $i^2=-1$, so you can continue to reduce after you distribute.
}
\end{enumerate}

\end{document}