\documentclass{extbook}[14pt]
\usepackage{multicol, enumerate, enumitem, hyperref, color, soul, setspace, parskip, fancyhdr, amssymb, amsthm, amsmath, latexsym, units, mathtools}
\everymath{\displaystyle}
\usepackage[headsep=0.5cm,headheight=0cm, left=1 in,right= 1 in,top= 1 in,bottom= 1 in]{geometry}
\usepackage{dashrule}  % Package to use the command below to create lines between items
\newcommand{\litem}[1]{\item #1

\rule{\textwidth}{0.4pt}}
\pagestyle{fancy}
\lhead{}
\chead{Answer Key for Progress Quiz 2 Version A}
\rhead{}
\lfoot{4389-3341}
\cfoot{}
\rfoot{Summer C 2021}
\begin{document}
\textbf{This key should allow you to understand why you choose the option you did (beyond just getting a question right or wrong). \href{https://xronos.clas.ufl.edu/mac1105spring2020/courseDescriptionAndMisc/Exams/LearningFromResults}{More instructions on how to use this key can be found here}.}

\textbf{If you have a suggestion to make the keys better, \href{https://forms.gle/CZkbZmPbC9XALEE88}{please fill out the short survey here}.}

\textit{Note: This key is auto-generated and may contain issues and/or errors. The keys are reviewed after each exam to ensure grading is done accurately. If there are issues (like duplicate options), they are noted in the offline gradebook. The keys are a work-in-progress to give students as many resources to improve as possible.}

\rule{\textwidth}{0.4pt}

\begin{enumerate}\litem{
Using an interval or intervals, describe all the $x$-values within or including a distance of the given values.
\[ \text{ Less than } 9 \text{ units from the number } 7. \]The solution is \( \text{None of the above} \), which is option E.\begin{enumerate}[label=\Alph*.]
\item \( (2, 16) \)

This describes the values less than 7 from 9
\item \( (-\infty, 2] \cup [16, \infty) \)

This describes the values no less than 7 from 9
\item \( (-\infty, 2) \cup (16, \infty) \)

This describes the values more than 7 from 9
\item \( [2, 16] \)

This describes the values no more than 7 from 9
\item \( \text{None of the above} \)

Options A-D described the values [more/less than] 7 units from 9, which is the reverse of what the question asked.
\end{enumerate}

\textbf{General Comment:} When thinking about this language, it helps to draw a number line and try points.
}
\litem{
Solve the linear inequality below. Then, choose the constant and interval combination that describes the solution set.
\[ -8x + 5 > -3x -8 \]The solution is \( (-\infty, 2.6) \), which is option C.\begin{enumerate}[label=\Alph*.]
\item \( (-\infty, a), \text{ where } a \in [-5.6, -1.6] \)

 $(-\infty, -2.6)$, which corresponds to negating the endpoint of the solution.
\item \( (a, \infty), \text{ where } a \in [-6.6, 0.4] \)

 $(-2.6, \infty)$, which corresponds to switching the direction of the interval AND negating the endpoint. You likely did this if you did not flip the inequality when dividing by a negative as well as not moving values over to a side properly.
\item \( (-\infty, a), \text{ where } a \in [-0.4, 8.6] \)

* $(-\infty, 2.6)$, which is the correct option.
\item \( (a, \infty), \text{ where } a \in [-2.4, 5.6] \)

 $(2.6, \infty)$, which corresponds to switching the direction of the interval. You likely did this if you did not flip the inequality when dividing by a negative!
\item \( \text{None of the above}. \)

You may have chosen this if you thought the inequality did not match the ends of the intervals.
\end{enumerate}

\textbf{General Comment:} Remember that less/greater than or equal to includes the endpoint, while less/greater do not. Also, remember that you need to flip the inequality when you multiply or divide by a negative.
}
\litem{
Solve the linear inequality below. Then, choose the constant and interval combination that describes the solution set.
\[ 5 - 4 x < \frac{-19 x - 6}{8} \leq 5 - 3 x \]The solution is \( (3.54, 9.20] \), which is option C.\begin{enumerate}[label=\Alph*.]
\item \( (-\infty, a] \cup (b, \infty), \text{ where } a \in [3, 5.25] \text{ and } b \in [8.25, 11.25] \)

$(-\infty, 3.54] \cup (9.20, \infty)$, which corresponds to displaying the and-inequality as an or-inequality AND flipping the inequality.
\item \( (-\infty, a) \cup [b, \infty), \text{ where } a \in [1.5, 7.5] \text{ and } b \in [8.25, 12.75] \)

$(-\infty, 3.54) \cup [9.20, \infty)$, which corresponds to displaying the and-inequality as an or-inequality.
\item \( (a, b], \text{ where } a \in [0.75, 7.5] \text{ and } b \in [9, 13.5] \)

* $(3.54, 9.20]$, which is the correct option.
\item \( [a, b), \text{ where } a \in [1.5, 7.5] \text{ and } b \in [5.25, 13.5] \)

$[3.54, 9.20)$, which corresponds to flipping the inequality.
\item \( \text{None of the above.} \)


\end{enumerate}

\textbf{General Comment:} To solve, you will need to break up the compound inequality into two inequalities. Be sure to keep track of the inequality! It may be best to draw a number line and graph your solution.
}
\litem{
Solve the linear inequality below. Then, choose the constant and interval combination that describes the solution set.
\[ -9 + 7 x \leq \frac{46 x + 5}{6} < -3 + 7 x \]The solution is \( \text{None of the above.} \), which is option E.\begin{enumerate}[label=\Alph*.]
\item \( (a, b], \text{ where } a \in [13.5, 17.25] \text{ and } b \in [-0.75, 10.5] \)

$(14.75, 5.75]$, which corresponds to flipping the inequality and getting negatives of the actual endpoints.
\item \( [a, b), \text{ where } a \in [12, 22.5] \text{ and } b \in [4.5, 6.75] \)

$[14.75, 5.75)$, which is the correct interval but negatives of the actual endpoints.
\item \( (-\infty, a) \cup [b, \infty), \text{ where } a \in [9.75, 15.75] \text{ and } b \in [3, 7.5] \)

$(-\infty, 14.75) \cup [5.75, \infty)$, which corresponds to displaying the and-inequality as an or-inequality AND flipping the inequality AND getting negatives of the actual endpoints.
\item \( (-\infty, a] \cup (b, \infty), \text{ where } a \in [9.75, 16.5] \text{ and } b \in [-1.5, 6.75] \)

$(-\infty, 14.75] \cup (5.75, \infty)$, which corresponds to displaying the and-inequality as an or-inequality and getting negatives of the actual endpoints.
\item \( \text{None of the above.} \)

* This is correct as the answer should be $[-14.75, -5.75)$.
\end{enumerate}

\textbf{General Comment:} To solve, you will need to break up the compound inequality into two inequalities. Be sure to keep track of the inequality! It may be best to draw a number line and graph your solution.
}
\litem{
Solve the linear inequality below. Then, choose the constant and interval combination that describes the solution set.
\[ \frac{5}{5} - \frac{6}{8} x > \frac{10}{4} x + \frac{9}{2} \]The solution is \( (-\infty, -1.077) \), which is option A.\begin{enumerate}[label=\Alph*.]
\item \( (-\infty, a), \text{ where } a \in [-3, -0.75] \)

* $(-\infty, -1.077)$, which is the correct option.
\item \( (-\infty, a), \text{ where } a \in [0, 3.75] \)

 $(-\infty, 1.077)$, which corresponds to negating the endpoint of the solution.
\item \( (a, \infty), \text{ where } a \in [-3.75, 0.75] \)

 $(-1.077, \infty)$, which corresponds to switching the direction of the interval. You likely did this if you did not flip the inequality when dividing by a negative!
\item \( (a, \infty), \text{ where } a \in [0, 3.75] \)

 $(1.077, \infty)$, which corresponds to switching the direction of the interval AND negating the endpoint. You likely did this if you did not flip the inequality when dividing by a negative as well as not moving values over to a side properly.
\item \( \text{None of the above}. \)

You may have chosen this if you thought the inequality did not match the ends of the intervals.
\end{enumerate}

\textbf{General Comment:} Remember that less/greater than or equal to includes the endpoint, while less/greater do not. Also, remember that you need to flip the inequality when you multiply or divide by a negative.
}
\litem{
Solve the linear inequality below. Then, choose the constant and interval combination that describes the solution set.
\[ -8 + 6 x > 8 x \text{ or } 4 + 6 x < 8 x \]The solution is \( (-\infty, -4.0) \text{ or } (2.0, \infty) \), which is option B.\begin{enumerate}[label=\Alph*.]
\item \( (-\infty, a) \cup (b, \infty), \text{ where } a \in [-2.25, 3] \text{ and } b \in [3, 4.35] \)

Corresponds to inverting the inequality and negating the solution.
\item \( (-\infty, a) \cup (b, \infty), \text{ where } a \in [-11.25, -3] \text{ and } b \in [1.88, 2.92] \)

 * Correct option.
\item \( (-\infty, a] \cup [b, \infty), \text{ where } a \in [-4.2, -2.17] \text{ and } b \in [0.6, 2.7] \)

Corresponds to including the endpoints (when they should be excluded).
\item \( (-\infty, a] \cup [b, \infty), \text{ where } a \in [-3.45, -0.22] \text{ and } b \in [2.4, 4.35] \)

Corresponds to including the endpoints AND negating.
\item \( (-\infty, \infty) \)

Corresponds to the variable canceling, which does not happen in this instance.
\end{enumerate}

\textbf{General Comment:} When multiplying or dividing by a negative, flip the sign.
}
\litem{
Solve the linear inequality below. Then, choose the constant and interval combination that describes the solution set.
\[ \frac{7}{7} + \frac{4}{9} x > \frac{5}{4} x - \frac{4}{6} \]The solution is \( (-\infty, 2.069) \), which is option C.\begin{enumerate}[label=\Alph*.]
\item \( (a, \infty), \text{ where } a \in [-5.25, 0] \)

 $(-2.069, \infty)$, which corresponds to switching the direction of the interval AND negating the endpoint. You likely did this if you did not flip the inequality when dividing by a negative as well as not moving values over to a side properly.
\item \( (a, \infty), \text{ where } a \in [1.5, 3.75] \)

 $(2.069, \infty)$, which corresponds to switching the direction of the interval. You likely did this if you did not flip the inequality when dividing by a negative!
\item \( (-\infty, a), \text{ where } a \in [0.75, 2.25] \)

* $(-\infty, 2.069)$, which is the correct option.
\item \( (-\infty, a), \text{ where } a \in [-5.25, 0] \)

 $(-\infty, -2.069)$, which corresponds to negating the endpoint of the solution.
\item \( \text{None of the above}. \)

You may have chosen this if you thought the inequality did not match the ends of the intervals.
\end{enumerate}

\textbf{General Comment:} Remember that less/greater than or equal to includes the endpoint, while less/greater do not. Also, remember that you need to flip the inequality when you multiply or divide by a negative.
}
\litem{
Solve the linear inequality below. Then, choose the constant and interval combination that describes the solution set.
\[ -3 + 4 x > 5 x \text{ or } -4 + 4 x < 7 x \]The solution is \( (-\infty, -3.0) \text{ or } (-1.333, \infty) \), which is option C.\begin{enumerate}[label=\Alph*.]
\item \( (-\infty, a) \cup (b, \infty), \text{ where } a \in [0, 2.25] \text{ and } b \in [0, 4.5] \)

Corresponds to inverting the inequality and negating the solution.
\item \( (-\infty, a] \cup [b, \infty), \text{ where } a \in [-4.5, -0.75] \text{ and } b \in [-1.43, 0.9] \)

Corresponds to including the endpoints (when they should be excluded).
\item \( (-\infty, a) \cup (b, \infty), \text{ where } a \in [-4.5, -2.25] \text{ and } b \in [-2.25, 1.5] \)

 * Correct option.
\item \( (-\infty, a] \cup [b, \infty), \text{ where } a \in [0, 7.5] \text{ and } b \in [0.15, 3.82] \)

Corresponds to including the endpoints AND negating.
\item \( (-\infty, \infty) \)

Corresponds to the variable canceling, which does not happen in this instance.
\end{enumerate}

\textbf{General Comment:} When multiplying or dividing by a negative, flip the sign.
}
\litem{
Using an interval or intervals, describe all the $x$-values within or including a distance of the given values.
\[ \text{ No more than } 4 \text{ units from the number } 8. \]The solution is \( \text{None of the above} \), which is option E.\begin{enumerate}[label=\Alph*.]
\item \( (-\infty, -4) \cup (12, \infty) \)

This describes the values more than 8 from 4
\item \( [-4, 12] \)

This describes the values no more than 8 from 4
\item \( (-\infty, -4] \cup [12, \infty) \)

This describes the values no less than 8 from 4
\item \( (-4, 12) \)

This describes the values less than 8 from 4
\item \( \text{None of the above} \)

Options A-D described the values [more/less than] 8 units from 4, which is the reverse of what the question asked.
\end{enumerate}

\textbf{General Comment:} When thinking about this language, it helps to draw a number line and try points.
}
\litem{
Solve the linear inequality below. Then, choose the constant and interval combination that describes the solution set.
\[ -4x + 9 < 3x -3 \]The solution is \( (1.714, \infty) \), which is option D.\begin{enumerate}[label=\Alph*.]
\item \( (-\infty, a), \text{ where } a \in [0.9, 2.2] \)

 $(-\infty, 1.714)$, which corresponds to switching the direction of the interval. You likely did this if you did not flip the inequality when dividing by a negative!
\item \( (a, \infty), \text{ where } a \in [-5.71, 1.29] \)

 $(-1.714, \infty)$, which corresponds to negating the endpoint of the solution.
\item \( (-\infty, a), \text{ where } a \in [-2.4, 0.2] \)

 $(-\infty, -1.714)$, which corresponds to switching the direction of the interval AND negating the endpoint. You likely did this if you did not flip the inequality when dividing by a negative as well as not moving values over to a side properly.
\item \( (a, \infty), \text{ where } a \in [1.71, 6.71] \)

* $(1.714, \infty)$, which is the correct option.
\item \( \text{None of the above}. \)

You may have chosen this if you thought the inequality did not match the ends of the intervals.
\end{enumerate}

\textbf{General Comment:} Remember that less/greater than or equal to includes the endpoint, while less/greater do not. Also, remember that you need to flip the inequality when you multiply or divide by a negative.
}
\end{enumerate}

\end{document}