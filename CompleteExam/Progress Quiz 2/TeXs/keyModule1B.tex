\documentclass{extbook}[14pt]
\usepackage{multicol, enumerate, enumitem, hyperref, color, soul, setspace, parskip, fancyhdr, amssymb, amsthm, amsmath, latexsym, units, mathtools}
\everymath{\displaystyle}
\usepackage[headsep=0.5cm,headheight=0cm, left=1 in,right= 1 in,top= 1 in,bottom= 1 in]{geometry}
\usepackage{dashrule}  % Package to use the command below to create lines between items
\newcommand{\litem}[1]{\item #1

\rule{\textwidth}{0.4pt}}
\pagestyle{fancy}
\lhead{}
\chead{Answer Key for Progress Quiz 2 Version B}
\rhead{}
\lfoot{4389-3341}
\cfoot{}
\rfoot{Summer C 2021}
\begin{document}
\textbf{This key should allow you to understand why you choose the option you did (beyond just getting a question right or wrong). \href{https://xronos.clas.ufl.edu/mac1105spring2020/courseDescriptionAndMisc/Exams/LearningFromResults}{More instructions on how to use this key can be found here}.}

\textbf{If you have a suggestion to make the keys better, \href{https://forms.gle/CZkbZmPbC9XALEE88}{please fill out the short survey here}.}

\textit{Note: This key is auto-generated and may contain issues and/or errors. The keys are reviewed after each exam to ensure grading is done accurately. If there are issues (like duplicate options), they are noted in the offline gradebook. The keys are a work-in-progress to give students as many resources to improve as possible.}

\rule{\textwidth}{0.4pt}

\begin{enumerate}\litem{
Choose the \textbf{smallest} set of Real numbers that the number below belongs to.
\[ \sqrt{\frac{1456}{13}} \]The solution is \( \text{Irrational} \), which is option B.\begin{enumerate}[label=\Alph*.]
\item \( \text{Rational} \)

These are numbers that can be written as fraction of Integers (e.g., -2/3)
\item \( \text{Irrational} \)

* This is the correct option!
\item \( \text{Integer} \)

These are the negative and positive counting numbers (..., -3, -2, -1, 0, 1, 2, 3, ...)
\item \( \text{Not a Real number} \)

These are Nonreal Complex numbers \textbf{OR} things that are not numbers (e.g., dividing by 0).
\item \( \text{Whole} \)

These are the counting numbers with 0 (0, 1, 2, 3, ...)
\end{enumerate}

\textbf{General Comment:} First, you \textbf{NEED} to simplify the expression. This question simplifies to $\sqrt{112}$. 
 
 Be sure you look at the simplified fraction and not just the decimal expansion. Numbers such as 13, 17, and 19 provide \textbf{long but repeating/terminating decimal expansions!} 
 
 The only ways to *not* be a Real number are: dividing by 0 or taking the square root of a negative number. 
 
 Irrational numbers are more than just square root of 3: adding or subtracting values from square root of 3 is also irrational.
}
\litem{
Simplify the expression below and choose the interval the simplification is contained within.
\[ 20 - 3^2 + 8 \div 5 * 10 \div 1 \]The solution is \( 27.000 \), which is option C.\begin{enumerate}[label=\Alph*.]
\item \( [11.16, 17.16] \)

 11.160, which corresponds to an Order of Operations error: not reading left-to-right for multiplication/division.
\item \( [44, 55] \)

 45.000, which corresponds to an Order of Operations error: multiplying by negative before squaring. For example: $(-3)^2 \neq -3^2$
\item \( [26, 29] \)

* 27.000, this is the correct option
\item \( [28.16, 35.16] \)

 29.160, which corresponds to two Order of Operations errors.
\item \( \text{None of the above} \)

 You may have gotten this by making an unanticipated error. If you got a value that is not any of the others, please let the coordinator know so they can help you figure out what happened.
\end{enumerate}

\textbf{General Comment:} While you may remember (or were taught) PEMDAS is done in order, it is actually done as P/E/MD/AS. When we are at MD or AS, we read left to right.
}
\litem{
Simplify the expression below into the form $a+bi$. Then, choose the intervals that $a$ and $b$ belong to.
\[ (-9 - 5 i)(-8 - 10 i) \]The solution is \( 22 + 130 i \), which is option E.\begin{enumerate}[label=\Alph*.]
\item \( a \in [114, 125] \text{ and } b \in [50, 53] \)

 $122 + 50 i$, which corresponds to adding a minus sign in the first term.
\item \( a \in [114, 125] \text{ and } b \in [-51, -49] \)

 $122 - 50 i$, which corresponds to adding a minus sign in the second term.
\item \( a \in [17, 23] \text{ and } b \in [-132, -126] \)

 $22 - 130 i$, which corresponds to adding a minus sign in both terms.
\item \( a \in [70, 74] \text{ and } b \in [50, 53] \)

 $72 + 50 i$, which corresponds to just multiplying the real terms to get the real part of the solution and the coefficients in the complex terms to get the complex part.
\item \( a \in [17, 23] \text{ and } b \in [129, 132] \)

* $22 + 130 i$, which is the correct option.
\end{enumerate}

\textbf{General Comment:} You can treat $i$ as a variable and distribute. Just remember that $i^2=-1$, so you can continue to reduce after you distribute.
}
\litem{
Simplify the expression below into the form $a+bi$. Then, choose the intervals that $a$ and $b$ belong to.
\[ (4 + 7 i)(-9 + 6 i) \]The solution is \( -78 - 39 i \), which is option C.\begin{enumerate}[label=\Alph*.]
\item \( a \in [-37, -30] \text{ and } b \in [39.9, 42.2] \)

 $-36 + 42 i$, which corresponds to just multiplying the real terms to get the real part of the solution and the coefficients in the complex terms to get the complex part.
\item \( a \in [4, 7] \text{ and } b \in [-89.3, -86.1] \)

 $6 - 87 i$, which corresponds to adding a minus sign in the second term.
\item \( a \in [-83, -74] \text{ and } b \in [-40.8, -38.1] \)

* $-78 - 39 i$, which is the correct option.
\item \( a \in [-83, -74] \text{ and } b \in [38.8, 41.8] \)

 $-78 + 39 i$, which corresponds to adding a minus sign in both terms.
\item \( a \in [4, 7] \text{ and } b \in [86.5, 89.6] \)

 $6 + 87 i$, which corresponds to adding a minus sign in the first term.
\end{enumerate}

\textbf{General Comment:} You can treat $i$ as a variable and distribute. Just remember that $i^2=-1$, so you can continue to reduce after you distribute.
}
\litem{
Choose the \textbf{smallest} set of Complex numbers that the number below belongs to.
\[ \sqrt{\frac{1040}{0}}+\sqrt{99} i \]The solution is \( \text{Not a Complex Number} \), which is option D.\begin{enumerate}[label=\Alph*.]
\item \( \text{Irrational} \)

These cannot be written as a fraction of Integers. Remember: $\pi$ is not an Integer!
\item \( \text{Pure Imaginary} \)

This is a Complex number $(a+bi)$ that \textbf{only} has an imaginary part like $2i$.
\item \( \text{Nonreal Complex} \)

This is a Complex number $(a+bi)$ that is not Real (has $i$ as part of the number).
\item \( \text{Not a Complex Number} \)

* This is the correct option!
\item \( \text{Rational} \)

These are numbers that can be written as fraction of Integers (e.g., -2/3 + 5)
\end{enumerate}

\textbf{General Comment:} Be sure to simplify $i^2 = -1$. This may remove the imaginary portion for your number. If you are having trouble, you may want to look at the \textit{Subgroups of the Real Numbers} section.
}
\litem{
Choose the \textbf{smallest} set of Complex numbers that the number below belongs to.
\[ \sqrt{\frac{0}{625}}+\sqrt{8}i \]The solution is \( \text{Pure Imaginary} \), which is option E.\begin{enumerate}[label=\Alph*.]
\item \( \text{Not a Complex Number} \)

This is not a number. The only non-Complex number we know is dividing by 0 as this is not a number!
\item \( \text{Irrational} \)

These cannot be written as a fraction of Integers. Remember: $\pi$ is not an Integer!
\item \( \text{Nonreal Complex} \)

This is a Complex number $(a+bi)$ that is not Real (has $i$ as part of the number).
\item \( \text{Rational} \)

These are numbers that can be written as fraction of Integers (e.g., -2/3 + 5)
\item \( \text{Pure Imaginary} \)

* This is the correct option!
\end{enumerate}

\textbf{General Comment:} Be sure to simplify $i^2 = -1$. This may remove the imaginary portion for your number. If you are having trouble, you may want to look at the \textit{Subgroups of the Real Numbers} section.
}
\litem{
Choose the \textbf{smallest} set of Real numbers that the number below belongs to.
\[ \sqrt{\frac{5}{0}} \]The solution is \( \text{Not a Real number} \), which is option E.\begin{enumerate}[label=\Alph*.]
\item \( \text{Whole} \)

These are the counting numbers with 0 (0, 1, 2, 3, ...)
\item \( \text{Rational} \)

These are numbers that can be written as fraction of Integers (e.g., -2/3)
\item \( \text{Irrational} \)

These cannot be written as a fraction of Integers.
\item \( \text{Integer} \)

These are the negative and positive counting numbers (..., -3, -2, -1, 0, 1, 2, 3, ...)
\item \( \text{Not a Real number} \)

* This is the correct option!
\end{enumerate}

\textbf{General Comment:} First, you \textbf{NEED} to simplify the expression. This question simplifies to $\sqrt{\frac{5}{0}}$. 
 
 Be sure you look at the simplified fraction and not just the decimal expansion. Numbers such as 13, 17, and 19 provide \textbf{long but repeating/terminating decimal expansions!} 
 
 The only ways to *not* be a Real number are: dividing by 0 or taking the square root of a negative number. 
 
 Irrational numbers are more than just square root of 3: adding or subtracting values from square root of 3 is also irrational.
}
\litem{
Simplify the expression below and choose the interval the simplification is contained within.
\[ 4 - 6 \div 19 * 5 - (20 * 13) \]The solution is \( -257.579 \), which is option D.\begin{enumerate}[label=\Alph*.]
\item \( [-257.17, -255.04] \)

 -256.063, which corresponds to an Order of Operations error: not reading left-to-right for multiplication/division.
\item \( [-228.8, -228.39] \)

 -228.526, which corresponds to not distributing a negative correctly.
\item \( [263.26, 264.85] \)

 263.937, which corresponds to not distributing addition and subtraction correctly.
\item \( [-258.68, -256.21] \)

* -257.579, which is the correct option.
\item \( \text{None of the above} \)

 You may have gotten this by making an unanticipated error. If you got a value that is not any of the others, please let the coordinator know so they can help you figure out what happened.
\end{enumerate}

\textbf{General Comment:} While you may remember (or were taught) PEMDAS is done in order, it is actually done as P/E/MD/AS. When we are at MD or AS, we read left to right.
}
\litem{
Simplify the expression below into the form $a+bi$. Then, choose the intervals that $a$ and $b$ belong to.
\[ \frac{9 + 55 i}{-7 + 6 i} \]The solution is \( 3.14  - 5.16 i \), which is option A.\begin{enumerate}[label=\Alph*.]
\item \( a \in [3, 5] \text{ and } b \in [-6, -5] \)

* $3.14  - 5.16 i$, which is the correct option.
\item \( a \in [3, 5] \text{ and } b \in [-440, -438] \)

 $3.14  - 439.00 i$, which corresponds to forgetting to multiply the conjugate by the numerator.
\item \( a \in [-5, -3.5] \text{ and } b \in [-4, -3] \)

 $-4.62  - 3.89 i$, which corresponds to forgetting to multiply the conjugate by the numerator and not computing the conjugate correctly.
\item \( a \in [-2, 0.5] \text{ and } b \in [8, 10] \)

 $-1.29  + 9.17 i$, which corresponds to just dividing the first term by the first term and the second by the second.
\item \( a \in [266.5, 269] \text{ and } b \in [-6, -5] \)

 $267.00  - 5.16 i$, which corresponds to forgetting to multiply the conjugate by the numerator and using a plus instead of a minus in the denominator.
\end{enumerate}

\textbf{General Comment:} Multiply the numerator and denominator by the *conjugate* of the denominator, then simplify. For example, if we have $2+3i$, the conjugate is $2-3i$.
}
\litem{
Simplify the expression below into the form $a+bi$. Then, choose the intervals that $a$ and $b$ belong to.
\[ \frac{-36 - 22 i}{5 + 6 i} \]The solution is \( -5.11  + 1.74 i \), which is option C.\begin{enumerate}[label=\Alph*.]
\item \( a \in [-312.5, -311.5] \text{ and } b \in [0, 2.5] \)

 $-312.00  + 1.74 i$, which corresponds to forgetting to multiply the conjugate by the numerator and using a plus instead of a minus in the denominator.
\item \( a \in [-5.5, -3.5] \text{ and } b \in [105, 106.5] \)

 $-5.11  + 106.00 i$, which corresponds to forgetting to multiply the conjugate by the numerator.
\item \( a \in [-5.5, -3.5] \text{ and } b \in [0, 2.5] \)

* $-5.11  + 1.74 i$, which is the correct option.
\item \( a \in [-8.5, -6.5] \text{ and } b \in [-5, -2.5] \)

 $-7.20  - 3.67 i$, which corresponds to just dividing the first term by the first term and the second by the second.
\item \( a \in [-2.5, 0] \text{ and } b \in [-6, -4.5] \)

 $-0.79  - 5.34 i$, which corresponds to forgetting to multiply the conjugate by the numerator and not computing the conjugate correctly.
\end{enumerate}

\textbf{General Comment:} Multiply the numerator and denominator by the *conjugate* of the denominator, then simplify. For example, if we have $2+3i$, the conjugate is $2-3i$.
}
\end{enumerate}

\end{document}