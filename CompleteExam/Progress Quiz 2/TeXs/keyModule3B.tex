\documentclass{extbook}[14pt]
\usepackage{multicol, enumerate, enumitem, hyperref, color, soul, setspace, parskip, fancyhdr, amssymb, amsthm, amsmath, latexsym, units, mathtools}
\everymath{\displaystyle}
\usepackage[headsep=0.5cm,headheight=0cm, left=1 in,right= 1 in,top= 1 in,bottom= 1 in]{geometry}
\usepackage{dashrule}  % Package to use the command below to create lines between items
\newcommand{\litem}[1]{\item #1

\rule{\textwidth}{0.4pt}}
\pagestyle{fancy}
\lhead{}
\chead{Answer Key for Progress Quiz 2 Version B}
\rhead{}
\lfoot{4389-3341}
\cfoot{}
\rfoot{Summer C 2021}
\begin{document}
\textbf{This key should allow you to understand why you choose the option you did (beyond just getting a question right or wrong). \href{https://xronos.clas.ufl.edu/mac1105spring2020/courseDescriptionAndMisc/Exams/LearningFromResults}{More instructions on how to use this key can be found here}.}

\textbf{If you have a suggestion to make the keys better, \href{https://forms.gle/CZkbZmPbC9XALEE88}{please fill out the short survey here}.}

\textit{Note: This key is auto-generated and may contain issues and/or errors. The keys are reviewed after each exam to ensure grading is done accurately. If there are issues (like duplicate options), they are noted in the offline gradebook. The keys are a work-in-progress to give students as many resources to improve as possible.}

\rule{\textwidth}{0.4pt}

\begin{enumerate}\litem{
Using an interval or intervals, describe all the $x$-values within or including a distance of the given values.
\[ \text{ No less than } 2 \text{ units from the number } 9. \]The solution is \( (-\infty, 7] \cup [11, \infty) \), which is option D.\begin{enumerate}[label=\Alph*.]
\item \( (7, 11) \)

This describes the values less than 2 from 9
\item \( [7, 11] \)

This describes the values no more than 2 from 9
\item \( (-\infty, 7) \cup (11, \infty) \)

This describes the values more than 2 from 9
\item \( (-\infty, 7] \cup [11, \infty) \)

This describes the values no less than 2 from 9
\item \( \text{None of the above} \)

You likely thought the values in the interval were not correct.
\end{enumerate}

\textbf{General Comment:} When thinking about this language, it helps to draw a number line and try points.
}
\litem{
Solve the linear inequality below. Then, choose the constant and interval combination that describes the solution set.
\[ 4x + 5 \leq 5x -8 \]The solution is \( [13.0, \infty) \), which is option A.\begin{enumerate}[label=\Alph*.]
\item \( [a, \infty), \text{ where } a \in [12, 14] \)

* $[13.0, \infty)$, which is the correct option.
\item \( [a, \infty), \text{ where } a \in [-16, -12] \)

 $[-13.0, \infty)$, which corresponds to negating the endpoint of the solution.
\item \( (-\infty, a], \text{ where } a \in [-15, -12] \)

 $(-\infty, -13.0]$, which corresponds to switching the direction of the interval AND negating the endpoint. You likely did this if you did not flip the inequality when dividing by a negative as well as not moving values over to a side properly.
\item \( (-\infty, a], \text{ where } a \in [6, 15] \)

 $(-\infty, 13.0]$, which corresponds to switching the direction of the interval. You likely did this if you did not flip the inequality when dividing by a negative!
\item \( \text{None of the above}. \)

You may have chosen this if you thought the inequality did not match the ends of the intervals.
\end{enumerate}

\textbf{General Comment:} Remember that less/greater than or equal to includes the endpoint, while less/greater do not. Also, remember that you need to flip the inequality when you multiply or divide by a negative.
}
\litem{
Solve the linear inequality below. Then, choose the constant and interval combination that describes the solution set.
\[ -4 - 6 x < \frac{-46 x + 8}{8} \leq -7 - 8 x \]The solution is \( (-20.00, -3.56] \), which is option D.\begin{enumerate}[label=\Alph*.]
\item \( (-\infty, a) \cup [b, \infty), \text{ where } a \in [-25.5, -17.25] \text{ and } b \in [-5.25, -1.5] \)

$(-\infty, -20.00) \cup [-3.56, \infty)$, which corresponds to displaying the and-inequality as an or-inequality.
\item \( [a, b), \text{ where } a \in [-21, -15.75] \text{ and } b \in [-4.5, 0.75] \)

$[-20.00, -3.56)$, which corresponds to flipping the inequality.
\item \( (-\infty, a] \cup (b, \infty), \text{ where } a \in [-21, -18] \text{ and } b \in [-6, -3] \)

$(-\infty, -20.00] \cup (-3.56, \infty)$, which corresponds to displaying the and-inequality as an or-inequality AND flipping the inequality.
\item \( (a, b], \text{ where } a \in [-21.75, -16.5] \text{ and } b \in [-6.75, -0.75] \)

* $(-20.00, -3.56]$, which is the correct option.
\item \( \text{None of the above.} \)


\end{enumerate}

\textbf{General Comment:} To solve, you will need to break up the compound inequality into two inequalities. Be sure to keep track of the inequality! It may be best to draw a number line and graph your solution.
}
\litem{
Solve the linear inequality below. Then, choose the constant and interval combination that describes the solution set.
\[ -6 + 8 x \leq \frac{26 x - 9}{3} < 4 + 4 x \]The solution is \( [-4.50, 1.50) \), which is option C.\begin{enumerate}[label=\Alph*.]
\item \( (-\infty, a) \cup [b, \infty), \text{ where } a \in [-9.75, -1.5] \text{ and } b \in [-0.75, 5.25] \)

$(-\infty, -4.50) \cup [1.50, \infty)$, which corresponds to displaying the and-inequality as an or-inequality AND flipping the inequality.
\item \( (a, b], \text{ where } a \in [-5.25, -0.75] \text{ and } b \in [-0.75, 9.75] \)

$(-4.50, 1.50]$, which corresponds to flipping the inequality.
\item \( [a, b), \text{ where } a \in [-5.25, -2.25] \text{ and } b \in [0.22, 1.72] \)

$[-4.50, 1.50)$, which is the correct option.
\item \( (-\infty, a] \cup (b, \infty), \text{ where } a \in [-7.5, -3] \text{ and } b \in [1.05, 1.8] \)

$(-\infty, -4.50] \cup (1.50, \infty)$, which corresponds to displaying the and-inequality as an or-inequality.
\item \( \text{None of the above.} \)


\end{enumerate}

\textbf{General Comment:} To solve, you will need to break up the compound inequality into two inequalities. Be sure to keep track of the inequality! It may be best to draw a number line and graph your solution.
}
\litem{
Solve the linear inequality below. Then, choose the constant and interval combination that describes the solution set.
\[ \frac{9}{8} + \frac{4}{7} x \leq \frac{7}{3} x - \frac{3}{4} \]The solution is \( [1.064, \infty) \), which is option A.\begin{enumerate}[label=\Alph*.]
\item \( [a, \infty), \text{ where } a \in [0, 2.25] \)

* $[1.064, \infty)$, which is the correct option.
\item \( [a, \infty), \text{ where } a \in [-4.5, 0] \)

 $[-1.064, \infty)$, which corresponds to negating the endpoint of the solution.
\item \( (-\infty, a], \text{ where } a \in [-4.5, 0.75] \)

 $(-\infty, -1.064]$, which corresponds to switching the direction of the interval AND negating the endpoint. You likely did this if you did not flip the inequality when dividing by a negative as well as not moving values over to a side properly.
\item \( (-\infty, a], \text{ where } a \in [0.75, 2.25] \)

 $(-\infty, 1.064]$, which corresponds to switching the direction of the interval. You likely did this if you did not flip the inequality when dividing by a negative!
\item \( \text{None of the above}. \)

You may have chosen this if you thought the inequality did not match the ends of the intervals.
\end{enumerate}

\textbf{General Comment:} Remember that less/greater than or equal to includes the endpoint, while less/greater do not. Also, remember that you need to flip the inequality when you multiply or divide by a negative.
}
\litem{
Solve the linear inequality below. Then, choose the constant and interval combination that describes the solution set.
\[ -7 + 3 x > 5 x \text{ or } 7 + 5 x < 8 x \]The solution is \( (-\infty, -3.5) \text{ or } (2.333, \infty) \), which is option B.\begin{enumerate}[label=\Alph*.]
\item \( (-\infty, a] \cup [b, \infty), \text{ where } a \in [-4.65, -2.4] \text{ and } b \in [1.95, 2.48] \)

Corresponds to including the endpoints (when they should be excluded).
\item \( (-\infty, a) \cup (b, \infty), \text{ where } a \in [-3.96, -3.48] \text{ and } b \in [1.74, 3.01] \)

 * Correct option.
\item \( (-\infty, a) \cup (b, \infty), \text{ where } a \in [-2.75, -2.24] \text{ and } b \in [3.05, 4.35] \)

Corresponds to inverting the inequality and negating the solution.
\item \( (-\infty, a] \cup [b, \infty), \text{ where } a \in [-3, -1.12] \text{ and } b \in [3.15, 5.77] \)

Corresponds to including the endpoints AND negating.
\item \( (-\infty, \infty) \)

Corresponds to the variable canceling, which does not happen in this instance.
\end{enumerate}

\textbf{General Comment:} When multiplying or dividing by a negative, flip the sign.
}
\litem{
Solve the linear inequality below. Then, choose the constant and interval combination that describes the solution set.
\[ \frac{-8}{7} - \frac{5}{8} x \leq \frac{5}{6} x + \frac{8}{9} \]The solution is \( [-1.393, \infty) \), which is option C.\begin{enumerate}[label=\Alph*.]
\item \( (-\infty, a], \text{ where } a \in [0, 4.5] \)

 $(-\infty, 1.393]$, which corresponds to switching the direction of the interval AND negating the endpoint. You likely did this if you did not flip the inequality when dividing by a negative as well as not moving values over to a side properly.
\item \( [a, \infty), \text{ where } a \in [0.75, 2.25] \)

 $[1.393, \infty)$, which corresponds to negating the endpoint of the solution.
\item \( [a, \infty), \text{ where } a \in [-2.25, 0.75] \)

* $[-1.393, \infty)$, which is the correct option.
\item \( (-\infty, a], \text{ where } a \in [-2.25, 0.75] \)

 $(-\infty, -1.393]$, which corresponds to switching the direction of the interval. You likely did this if you did not flip the inequality when dividing by a negative!
\item \( \text{None of the above}. \)

You may have chosen this if you thought the inequality did not match the ends of the intervals.
\end{enumerate}

\textbf{General Comment:} Remember that less/greater than or equal to includes the endpoint, while less/greater do not. Also, remember that you need to flip the inequality when you multiply or divide by a negative.
}
\litem{
Solve the linear inequality below. Then, choose the constant and interval combination that describes the solution set.
\[ -7 - 3 x > 4 x \text{ or } 6 + 9 x < 10 x \]The solution is \( (-\infty, -1.0) \text{ or } (6.0, \infty) \), which is option B.\begin{enumerate}[label=\Alph*.]
\item \( (-\infty, a) \cup (b, \infty), \text{ where } a \in [-8.25, -4.5] \text{ and } b \in [-0.75, 3.75] \)

Corresponds to inverting the inequality and negating the solution.
\item \( (-\infty, a) \cup (b, \infty), \text{ where } a \in [-3.75, 0] \text{ and } b \in [1.5, 8.25] \)

 * Correct option.
\item \( (-\infty, a] \cup [b, \infty), \text{ where } a \in [-9.75, -3] \text{ and } b \in [0, 4.5] \)

Corresponds to including the endpoints AND negating.
\item \( (-\infty, a] \cup [b, \infty), \text{ where } a \in [-5.25, 2.25] \text{ and } b \in [3.75, 9] \)

Corresponds to including the endpoints (when they should be excluded).
\item \( (-\infty, \infty) \)

Corresponds to the variable canceling, which does not happen in this instance.
\end{enumerate}

\textbf{General Comment:} When multiplying or dividing by a negative, flip the sign.
}
\litem{
Using an interval or intervals, describe all the $x$-values within or including a distance of the given values.
\[ \text{ No more than } 5 \text{ units from the number } 4. \]The solution is \( [-1, 9] \), which is option A.\begin{enumerate}[label=\Alph*.]
\item \( [-1, 9] \)

This describes the values no more than 5 from 4
\item \( (-1, 9) \)

This describes the values less than 5 from 4
\item \( (-\infty, -1) \cup (9, \infty) \)

This describes the values more than 5 from 4
\item \( (-\infty, -1] \cup [9, \infty) \)

This describes the values no less than 5 from 4
\item \( \text{None of the above} \)

You likely thought the values in the interval were not correct.
\end{enumerate}

\textbf{General Comment:} When thinking about this language, it helps to draw a number line and try points.
}
\litem{
Solve the linear inequality below. Then, choose the constant and interval combination that describes the solution set.
\[ -8x -3 \leq 5x + 4 \]The solution is \( [-0.538, \infty) \), which is option B.\begin{enumerate}[label=\Alph*.]
\item \( [a, \infty), \text{ where } a \in [0.06, 1.26] \)

 $[0.538, \infty)$, which corresponds to negating the endpoint of the solution.
\item \( [a, \infty), \text{ where } a \in [-1.23, -0.26] \)

* $[-0.538, \infty)$, which is the correct option.
\item \( (-\infty, a], \text{ where } a \in [-1.54, 0.46] \)

 $(-\infty, -0.538]$, which corresponds to switching the direction of the interval. You likely did this if you did not flip the inequality when dividing by a negative!
\item \( (-\infty, a], \text{ where } a \in [-0.46, 5.54] \)

 $(-\infty, 0.538]$, which corresponds to switching the direction of the interval AND negating the endpoint. You likely did this if you did not flip the inequality when dividing by a negative as well as not moving values over to a side properly.
\item \( \text{None of the above}. \)

You may have chosen this if you thought the inequality did not match the ends of the intervals.
\end{enumerate}

\textbf{General Comment:} Remember that less/greater than or equal to includes the endpoint, while less/greater do not. Also, remember that you need to flip the inequality when you multiply or divide by a negative.
}
\end{enumerate}

\end{document}