\documentclass{extbook}[14pt]
\usepackage{multicol, enumerate, enumitem, hyperref, color, soul, setspace, parskip, fancyhdr, amssymb, amsthm, amsmath, bbm, latexsym, units, mathtools}
\everymath{\displaystyle}
\usepackage[headsep=0.5cm,headheight=0cm, left=1 in,right= 1 in,top= 1 in,bottom= 1 in]{geometry}
\usepackage{dashrule}  % Package to use the command below to create lines between items
\newcommand{\litem}[1]{\item #1

\rule{\textwidth}{0.4pt}}
\pagestyle{fancy}
\lhead{}
\chead{Answer Key for Progress Quiz 2 Version A}
\rhead{}
\lfoot{7862-5421}
\cfoot{}
\rfoot{Spring 2021}
\begin{document}
\textbf{This key should allow you to understand why you choose the option you did (beyond just getting a question right or wrong). \href{https://xronos.clas.ufl.edu/mac1105spring2020/courseDescriptionAndMisc/Exams/LearningFromResults}{More instructions on how to use this key can be found here}.}

\textbf{If you have a suggestion to make the keys better, \href{https://forms.gle/CZkbZmPbC9XALEE88}{please fill out the short survey here}.}

\textit{Note: This key is auto-generated and may contain issues and/or errors. The keys are reviewed after each exam to ensure grading is done accurately. If there are issues (like duplicate options), they are noted in the offline gradebook. The keys are a work-in-progress to give students as many resources to improve as possible.}

\rule{\textwidth}{0.4pt}

\begin{enumerate}\litem{
Choose the \textbf{smallest} set of Real numbers that the number below belongs to.
\[ -\sqrt{\frac{250000}{625}} \]

The solution is \( \text{Integer} \), which is option E.\begin{enumerate}[label=\Alph*.]
\item \( \text{Rational} \)

These are numbers that can be written as fraction of Integers (e.g., -2/3)
\item \( \text{Not a Real number} \)

These are Nonreal Complex numbers \textbf{OR} things that are not numbers (e.g., dividing by 0).
\item \( \text{Whole} \)

These are the counting numbers with 0 (0, 1, 2, 3, ...)
\item \( \text{Irrational} \)

These cannot be written as a fraction of Integers.
\item \( \text{Integer} \)

* This is the correct option!
\end{enumerate}

\textbf{General Comment:} First, you \textbf{NEED} to simplify the expression. This question simplifies to $-500$. 
 
 Be sure you look at the simplified fraction and not just the decimal expansion. Numbers such as 13, 17, and 19 provide \textbf{long but repeating/terminating decimal expansions!} 
 
 The only ways to *not* be a Real number are: dividing by 0 or taking the square root of a negative number. 
 
 Irrational numbers are more than just square root of 3: adding or subtracting values from square root of 3 is also irrational.
}
\litem{
Simplify the expression below and choose the interval the simplification is contained within.
\[ 14 - 5 \div 15 * 20 - (19 * 9) \]

The solution is \( -163.667 \), which is option B.\begin{enumerate}[label=\Alph*.]
\item \( [-159.02, -153.02] \)

 -157.017, which corresponds to an Order of Operations error: not reading left-to-right for multiplication/division.
\item \( [-167.67, -159.67] \)

* -163.667, which is the correct option.
\item \( [-105, -104] \)

 -105.000, which corresponds to not distributing a negative correctly.
\item \( [184.98, 185.98] \)

 184.983, which corresponds to not distributing addition and subtraction correctly.
\item \( \text{None of the above} \)

 You may have gotten this by making an unanticipated error. If you got a value that is not any of the others, please let the coordinator know so they can help you figure out what happened.
\end{enumerate}

\textbf{General Comment:} While you may remember (or were taught) PEMDAS is done in order, it is actually done as P/E/MD/AS. When we are at MD or AS, we read left to right.
}
\litem{
Choose the \textbf{smallest} set of Real numbers that the number below belongs to.
\[ \sqrt{\frac{32400}{81}} \]

The solution is \( \text{Whole} \), which is option A.\begin{enumerate}[label=\Alph*.]
\item \( \text{Whole} \)

* This is the correct option!
\item \( \text{Rational} \)

These are numbers that can be written as fraction of Integers (e.g., -2/3)
\item \( \text{Not a Real number} \)

These are Nonreal Complex numbers \textbf{OR} things that are not numbers (e.g., dividing by 0).
\item \( \text{Integer} \)

These are the negative and positive counting numbers (..., -3, -2, -1, 0, 1, 2, 3, ...)
\item \( \text{Irrational} \)

These cannot be written as a fraction of Integers.
\end{enumerate}

\textbf{General Comment:} First, you \textbf{NEED} to simplify the expression. This question simplifies to $180$. 
 
 Be sure you look at the simplified fraction and not just the decimal expansion. Numbers such as 13, 17, and 19 provide \textbf{long but repeating/terminating decimal expansions!} 
 
 The only ways to *not* be a Real number are: dividing by 0 or taking the square root of a negative number. 
 
 Irrational numbers are more than just square root of 3: adding or subtracting values from square root of 3 is also irrational.
}
\litem{
Simplify the expression below into the form $a+bi$. Then, choose the intervals that $a$ and $b$ belong to.
\[ \frac{-63 + 11 i}{3 - 5 i} \]

The solution is \( -7.18  - 8.29 i \), which is option D.\begin{enumerate}[label=\Alph*.]
\item \( a \in [-22, -20.5] \text{ and } b \in [-2.5, -0.5] \)

 $-21.00  - 2.20 i$, which corresponds to just dividing the first term by the first term and the second by the second.
\item \( a \in [-5, -3] \text{ and } b \in [9.5, 11] \)

 $-3.94  + 10.24 i$, which corresponds to forgetting to multiply the conjugate by the numerator and not computing the conjugate correctly.
\item \( a \in [-7.5, -7] \text{ and } b \in [-282.5, -281] \)

 $-7.18  - 282.00 i$, which corresponds to forgetting to multiply the conjugate by the numerator.
\item \( a \in [-7.5, -7] \text{ and } b \in [-10, -7] \)

* $-7.18  - 8.29 i$, which is the correct option.
\item \( a \in [-245.5, -243] \text{ and } b \in [-10, -7] \)

 $-244.00  - 8.29 i$, which corresponds to forgetting to multiply the conjugate by the numerator and using a plus instead of a minus in the denominator.
\end{enumerate}

\textbf{General Comment:} Multiply the numerator and denominator by the *conjugate* of the denominator, then simplify. For example, if we have $2+3i$, the conjugate is $2-3i$.
}
\litem{
Simplify the expression below into the form $a+bi$. Then, choose the intervals that $a$ and $b$ belong to.
\[ \frac{27 + 44 i}{6 - 2 i} \]

The solution is \( 1.85  + 7.95 i \), which is option E.\begin{enumerate}[label=\Alph*.]
\item \( a \in [73, 74.5] \text{ and } b \in [7, 8.5] \)

 $74.00  + 7.95 i$, which corresponds to forgetting to multiply the conjugate by the numerator and using a plus instead of a minus in the denominator.
\item \( a \in [5.5, 7] \text{ and } b \in [3.5, 6.5] \)

 $6.25  + 5.25 i$, which corresponds to forgetting to multiply the conjugate by the numerator and not computing the conjugate correctly.
\item \( a \in [0.5, 2.5] \text{ and } b \in [317.5, 319.5] \)

 $1.85  + 318.00 i$, which corresponds to forgetting to multiply the conjugate by the numerator.
\item \( a \in [3, 5] \text{ and } b \in [-22.5, -20.5] \)

 $4.50  - 22.00 i$, which corresponds to just dividing the first term by the first term and the second by the second.
\item \( a \in [0.5, 2.5] \text{ and } b \in [7, 8.5] \)

* $1.85  + 7.95 i$, which is the correct option.
\end{enumerate}

\textbf{General Comment:} Multiply the numerator and denominator by the *conjugate* of the denominator, then simplify. For example, if we have $2+3i$, the conjugate is $2-3i$.
}
\litem{
Simplify the expression below and choose the interval the simplification is contained within.
\[ 8 - 1^2 + 12 \div 15 * 18 \div 14 \]

The solution is \( 8.029 \), which is option A.\begin{enumerate}[label=\Alph*.]
\item \( [7.42, 8.89] \)

* 8.029, this is the correct option
\item \( [8.79, 9.57] \)

 9.003, which corresponds to two Order of Operations errors.
\item \( [6.4, 7.59] \)

 7.003, which corresponds to an Order of Operations error: not reading left-to-right for multiplication/division.
\item \( [9.85, 10.53] \)

 10.029, which corresponds to an Order of Operations error: multiplying by negative before squaring. For example: $(-3)^2 \neq -3^2$
\item \( \text{None of the above} \)

 You may have gotten this by making an unanticipated error. If you got a value that is not any of the others, please let the coordinator know so they can help you figure out what happened.
\end{enumerate}

\textbf{General Comment:} While you may remember (or were taught) PEMDAS is done in order, it is actually done as P/E/MD/AS. When we are at MD or AS, we read left to right.
}
\litem{
Choose the \textbf{smallest} set of Complex numbers that the number below belongs to.
\[ \frac{\sqrt{221}}{10}+\sqrt{-5}i \]

The solution is \( \text{Irrational} \), which is option D.\begin{enumerate}[label=\Alph*.]
\item \( \text{Rational} \)

These are numbers that can be written as fraction of Integers (e.g., -2/3 + 5)
\item \( \text{Nonreal Complex} \)

This is a Complex number $(a+bi)$ that is not Real (has $i$ as part of the number).
\item \( \text{Pure Imaginary} \)

This is a Complex number $(a+bi)$ that \textbf{only} has an imaginary part like $2i$.
\item \( \text{Irrational} \)

* This is the correct option!
\item \( \text{Not a Complex Number} \)

This is not a number. The only non-Complex number we know is dividing by 0 as this is not a number!
\end{enumerate}

\textbf{General Comment:} Be sure to simplify $i^2 = -1$. This may remove the imaginary portion for your number. If you are having trouble, you may want to look at the \textit{Subgroups of the Real Numbers} section.
}
\litem{
Simplify the expression below into the form $a+bi$. Then, choose the intervals that $a$ and $b$ belong to.
\[ (5 - 9 i)(-7 + 10 i) \]

The solution is \( 55 + 113 i \), which is option D.\begin{enumerate}[label=\Alph*.]
\item \( a \in [54, 61] \text{ and } b \in [-119, -111] \)

 $55 - 113 i$, which corresponds to adding a minus sign in both terms.
\item \( a \in [-128, -122] \text{ and } b \in [11, 19] \)

 $-125 + 13 i$, which corresponds to adding a minus sign in the second term.
\item \( a \in [-128, -122] \text{ and } b \in [-13, -7] \)

 $-125 - 13 i$, which corresponds to adding a minus sign in the first term.
\item \( a \in [54, 61] \text{ and } b \in [106, 114] \)

* $55 + 113 i$, which is the correct option.
\item \( a \in [-38, -33] \text{ and } b \in [-93, -85] \)

 $-35 - 90 i$, which corresponds to just multiplying the real terms to get the real part of the solution and the coefficients in the complex terms to get the complex part.
\end{enumerate}

\textbf{General Comment:} You can treat $i$ as a variable and distribute. Just remember that $i^2=-1$, so you can continue to reduce after you distribute.
}
\litem{
Choose the \textbf{smallest} set of Complex numbers that the number below belongs to.
\[ \sqrt{\frac{-720}{0}} i+\sqrt{208}i \]

The solution is \( \text{Not a Complex Number} \), which is option D.\begin{enumerate}[label=\Alph*.]
\item \( \text{Rational} \)

These are numbers that can be written as fraction of Integers (e.g., -2/3 + 5)
\item \( \text{Nonreal Complex} \)

This is a Complex number $(a+bi)$ that is not Real (has $i$ as part of the number).
\item \( \text{Irrational} \)

These cannot be written as a fraction of Integers. Remember: $\pi$ is not an Integer!
\item \( \text{Not a Complex Number} \)

* This is the correct option!
\item \( \text{Pure Imaginary} \)

This is a Complex number $(a+bi)$ that \textbf{only} has an imaginary part like $2i$.
\end{enumerate}

\textbf{General Comment:} Be sure to simplify $i^2 = -1$. This may remove the imaginary portion for your number. If you are having trouble, you may want to look at the \textit{Subgroups of the Real Numbers} section.
}
\litem{
Simplify the expression below into the form $a+bi$. Then, choose the intervals that $a$ and $b$ belong to.
\[ (-2 - 9 i)(-4 + 8 i) \]

The solution is \( 80 + 20 i \), which is option A.\begin{enumerate}[label=\Alph*.]
\item \( a \in [76, 85] \text{ and } b \in [10, 22] \)

* $80 + 20 i$, which is the correct option.
\item \( a \in [-64, -61] \text{ and } b \in [48, 56] \)

 $-64 + 52 i$, which corresponds to adding a minus sign in the second term.
\item \( a \in [76, 85] \text{ and } b \in [-23, -13] \)

 $80 - 20 i$, which corresponds to adding a minus sign in both terms.
\item \( a \in [3, 14] \text{ and } b \in [-76, -70] \)

 $8 - 72 i$, which corresponds to just multiplying the real terms to get the real part of the solution and the coefficients in the complex terms to get the complex part.
\item \( a \in [-64, -61] \text{ and } b \in [-52, -47] \)

 $-64 - 52 i$, which corresponds to adding a minus sign in the first term.
\end{enumerate}

\textbf{General Comment:} You can treat $i$ as a variable and distribute. Just remember that $i^2=-1$, so you can continue to reduce after you distribute.
}
\end{enumerate}

\end{document}