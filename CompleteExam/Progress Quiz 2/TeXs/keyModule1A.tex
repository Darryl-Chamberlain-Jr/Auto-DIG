\documentclass{extbook}[14pt]
\usepackage{multicol, enumerate, enumitem, hyperref, color, soul, setspace, parskip, fancyhdr, amssymb, amsthm, amsmath, latexsym, units, mathtools}
\everymath{\displaystyle}
\usepackage[headsep=0.5cm,headheight=0cm, left=1 in,right= 1 in,top= 1 in,bottom= 1 in]{geometry}
\usepackage{dashrule}  % Package to use the command below to create lines between items
\newcommand{\litem}[1]{\item #1

\rule{\textwidth}{0.4pt}}
\pagestyle{fancy}
\lhead{}
\chead{Answer Key for Progress Quiz 2 Version A}
\rhead{}
\lfoot{4389-3341}
\cfoot{}
\rfoot{Summer C 2021}
\begin{document}
\textbf{This key should allow you to understand why you choose the option you did (beyond just getting a question right or wrong). \href{https://xronos.clas.ufl.edu/mac1105spring2020/courseDescriptionAndMisc/Exams/LearningFromResults}{More instructions on how to use this key can be found here}.}

\textbf{If you have a suggestion to make the keys better, \href{https://forms.gle/CZkbZmPbC9XALEE88}{please fill out the short survey here}.}

\textit{Note: This key is auto-generated and may contain issues and/or errors. The keys are reviewed after each exam to ensure grading is done accurately. If there are issues (like duplicate options), they are noted in the offline gradebook. The keys are a work-in-progress to give students as many resources to improve as possible.}

\rule{\textwidth}{0.4pt}

\begin{enumerate}\litem{
Choose the \textbf{smallest} set of Real numbers that the number below belongs to.
\[ \sqrt{\frac{-1386}{14}} \]The solution is \( \text{Not a Real number} \), which is option A.\begin{enumerate}[label=\Alph*.]
\item \( \text{Not a Real number} \)

* This is the correct option!
\item \( \text{Integer} \)

These are the negative and positive counting numbers (..., -3, -2, -1, 0, 1, 2, 3, ...)
\item \( \text{Irrational} \)

These cannot be written as a fraction of Integers.
\item \( \text{Whole} \)

These are the counting numbers with 0 (0, 1, 2, 3, ...)
\item \( \text{Rational} \)

These are numbers that can be written as fraction of Integers (e.g., -2/3)
\end{enumerate}

\textbf{General Comment:} First, you \textbf{NEED} to simplify the expression. This question simplifies to $\sqrt{99} i$. 
 
 Be sure you look at the simplified fraction and not just the decimal expansion. Numbers such as 13, 17, and 19 provide \textbf{long but repeating/terminating decimal expansions!} 
 
 The only ways to *not* be a Real number are: dividing by 0 or taking the square root of a negative number. 
 
 Irrational numbers are more than just square root of 3: adding or subtracting values from square root of 3 is also irrational.
}
\litem{
Simplify the expression below and choose the interval the simplification is contained within.
\[ 1 - 15 \div 2 * 20 - (9 * 18) \]The solution is \( -311.000 \), which is option D.\begin{enumerate}[label=\Alph*.]
\item \( [-164.38, -155.38] \)

 -161.375, which corresponds to an Order of Operations error: not reading left-to-right for multiplication/division.
\item \( [158.62, 165.62] \)

 162.625, which corresponds to not distributing addition and subtraction correctly.
\item \( [-2846, -2839] \)

 -2844.000, which corresponds to not distributing a negative correctly.
\item \( [-311, -310] \)

* -311.000, which is the correct option.
\item \( \text{None of the above} \)

 You may have gotten this by making an unanticipated error. If you got a value that is not any of the others, please let the coordinator know so they can help you figure out what happened.
\end{enumerate}

\textbf{General Comment:} While you may remember (or were taught) PEMDAS is done in order, it is actually done as P/E/MD/AS. When we are at MD or AS, we read left to right.
}
\litem{
Simplify the expression below into the form $a+bi$. Then, choose the intervals that $a$ and $b$ belong to.
\[ (2 - 7 i)(-6 - 8 i) \]The solution is \( -68 + 26 i \), which is option D.\begin{enumerate}[label=\Alph*.]
\item \( a \in [43, 50] \text{ and } b \in [57.1, 59.8] \)

 $44 + 58 i$, which corresponds to adding a minus sign in the second term.
\item \( a \in [-13, -10] \text{ and } b \in [55.7, 56.3] \)

 $-12 + 56 i$, which corresponds to just multiplying the real terms to get the real part of the solution and the coefficients in the complex terms to get the complex part.
\item \( a \in [-75, -67] \text{ and } b \in [-27, -24] \)

 $-68 - 26 i$, which corresponds to adding a minus sign in both terms.
\item \( a \in [-75, -67] \text{ and } b \in [24.2, 26.9] \)

* $-68 + 26 i$, which is the correct option.
\item \( a \in [43, 50] \text{ and } b \in [-59.7, -57.9] \)

 $44 - 58 i$, which corresponds to adding a minus sign in the first term.
\end{enumerate}

\textbf{General Comment:} You can treat $i$ as a variable and distribute. Just remember that $i^2=-1$, so you can continue to reduce after you distribute.
}
\litem{
Simplify the expression below into the form $a+bi$. Then, choose the intervals that $a$ and $b$ belong to.
\[ (-8 + 4 i)(3 + 10 i) \]The solution is \( -64 - 68 i \), which is option E.\begin{enumerate}[label=\Alph*.]
\item \( a \in [12, 22] \text{ and } b \in [92, 98] \)

 $16 + 92 i$, which corresponds to adding a minus sign in the second term.
\item \( a \in [-26, -22] \text{ and } b \in [37, 44] \)

 $-24 + 40 i$, which corresponds to just multiplying the real terms to get the real part of the solution and the coefficients in the complex terms to get the complex part.
\item \( a \in [-66, -58] \text{ and } b \in [65, 73] \)

 $-64 + 68 i$, which corresponds to adding a minus sign in both terms.
\item \( a \in [12, 22] \text{ and } b \in [-98, -89] \)

 $16 - 92 i$, which corresponds to adding a minus sign in the first term.
\item \( a \in [-66, -58] \text{ and } b \in [-74, -67] \)

* $-64 - 68 i$, which is the correct option.
\end{enumerate}

\textbf{General Comment:} You can treat $i$ as a variable and distribute. Just remember that $i^2=-1$, so you can continue to reduce after you distribute.
}
\litem{
Choose the \textbf{smallest} set of Complex numbers that the number below belongs to.
\[ \sqrt{\frac{2730}{15}}+\sqrt{110} i \]The solution is \( \text{Nonreal Complex} \), which is option E.\begin{enumerate}[label=\Alph*.]
\item \( \text{Not a Complex Number} \)

This is not a number. The only non-Complex number we know is dividing by 0 as this is not a number!
\item \( \text{Rational} \)

These are numbers that can be written as fraction of Integers (e.g., -2/3 + 5)
\item \( \text{Pure Imaginary} \)

This is a Complex number $(a+bi)$ that \textbf{only} has an imaginary part like $2i$.
\item \( \text{Irrational} \)

These cannot be written as a fraction of Integers. Remember: $\pi$ is not an Integer!
\item \( \text{Nonreal Complex} \)

* This is the correct option!
\end{enumerate}

\textbf{General Comment:} Be sure to simplify $i^2 = -1$. This may remove the imaginary portion for your number. If you are having trouble, you may want to look at the \textit{Subgroups of the Real Numbers} section.
}
\litem{
Choose the \textbf{smallest} set of Complex numbers that the number below belongs to.
\[ \sqrt{\frac{0}{15}}+\sqrt{6}i \]The solution is \( \text{Pure Imaginary} \), which is option B.\begin{enumerate}[label=\Alph*.]
\item \( \text{Irrational} \)

These cannot be written as a fraction of Integers. Remember: $\pi$ is not an Integer!
\item \( \text{Pure Imaginary} \)

* This is the correct option!
\item \( \text{Not a Complex Number} \)

This is not a number. The only non-Complex number we know is dividing by 0 as this is not a number!
\item \( \text{Rational} \)

These are numbers that can be written as fraction of Integers (e.g., -2/3 + 5)
\item \( \text{Nonreal Complex} \)

This is a Complex number $(a+bi)$ that is not Real (has $i$ as part of the number).
\end{enumerate}

\textbf{General Comment:} Be sure to simplify $i^2 = -1$. This may remove the imaginary portion for your number. If you are having trouble, you may want to look at the \textit{Subgroups of the Real Numbers} section.
}
\litem{
Choose the \textbf{smallest} set of Real numbers that the number below belongs to.
\[ -\sqrt{\frac{9216}{36}} \]The solution is \( \text{Integer} \), which is option D.\begin{enumerate}[label=\Alph*.]
\item \( \text{Whole} \)

These are the counting numbers with 0 (0, 1, 2, 3, ...)
\item \( \text{Rational} \)

These are numbers that can be written as fraction of Integers (e.g., -2/3)
\item \( \text{Irrational} \)

These cannot be written as a fraction of Integers.
\item \( \text{Integer} \)

* This is the correct option!
\item \( \text{Not a Real number} \)

These are Nonreal Complex numbers \textbf{OR} things that are not numbers (e.g., dividing by 0).
\end{enumerate}

\textbf{General Comment:} First, you \textbf{NEED} to simplify the expression. This question simplifies to $-96$. 
 
 Be sure you look at the simplified fraction and not just the decimal expansion. Numbers such as 13, 17, and 19 provide \textbf{long but repeating/terminating decimal expansions!} 
 
 The only ways to *not* be a Real number are: dividing by 0 or taking the square root of a negative number. 
 
 Irrational numbers are more than just square root of 3: adding or subtracting values from square root of 3 is also irrational.
}
\litem{
Simplify the expression below and choose the interval the simplification is contained within.
\[ 12 - 16^2 + 8 \div 4 * 9 \div 6 \]The solution is \( -241.000 \), which is option B.\begin{enumerate}[label=\Alph*.]
\item \( [268.9, 274.3] \)

 271.000, which corresponds to an Order of Operations error: multiplying by negative before squaring. For example: $(-3)^2 \neq -3^2$
\item \( [-243.4, -239.4] \)

* -241.000, this is the correct option
\item \( [265.8, 270.5] \)

 268.037, which corresponds to two Order of Operations errors.
\item \( [-248.1, -243.2] \)

 -243.963, which corresponds to an Order of Operations error: not reading left-to-right for multiplication/division.
\item \( \text{None of the above} \)

 You may have gotten this by making an unanticipated error. If you got a value that is not any of the others, please let the coordinator know so they can help you figure out what happened.
\end{enumerate}

\textbf{General Comment:} While you may remember (or were taught) PEMDAS is done in order, it is actually done as P/E/MD/AS. When we are at MD or AS, we read left to right.
}
\litem{
Simplify the expression below into the form $a+bi$. Then, choose the intervals that $a$ and $b$ belong to.
\[ \frac{9 - 44 i}{-3 - 7 i} \]The solution is \( 4.84  + 3.36 i \), which is option E.\begin{enumerate}[label=\Alph*.]
\item \( a \in [-7, -5] \text{ and } b \in [0.5, 1.5] \)

 $-5.78  + 1.19 i$, which corresponds to forgetting to multiply the conjugate by the numerator and not computing the conjugate correctly.
\item \( a \in [280.5, 282] \text{ and } b \in [2.5, 4.5] \)

 $281.00  + 3.36 i$, which corresponds to forgetting to multiply the conjugate by the numerator and using a plus instead of a minus in the denominator.
\item \( a \in [-4.5, -2] \text{ and } b \in [4.5, 7.5] \)

 $-3.00  + 6.29 i$, which corresponds to just dividing the first term by the first term and the second by the second.
\item \( a \in [4, 5.5] \text{ and } b \in [194.5, 195.5] \)

 $4.84  + 195.00 i$, which corresponds to forgetting to multiply the conjugate by the numerator.
\item \( a \in [4, 5.5] \text{ and } b \in [2.5, 4.5] \)

* $4.84  + 3.36 i$, which is the correct option.
\end{enumerate}

\textbf{General Comment:} Multiply the numerator and denominator by the *conjugate* of the denominator, then simplify. For example, if we have $2+3i$, the conjugate is $2-3i$.
}
\litem{
Simplify the expression below into the form $a+bi$. Then, choose the intervals that $a$ and $b$ belong to.
\[ \frac{-9 - 55 i}{2 - 7 i} \]The solution is \( 6.92  - 3.26 i \), which is option A.\begin{enumerate}[label=\Alph*.]
\item \( a \in [6, 8] \text{ and } b \in [-5, -2.5] \)

* $6.92  - 3.26 i$, which is the correct option.
\item \( a \in [-5.5, -3.5] \text{ and } b \in [7.5, 8.5] \)

 $-4.50  + 7.86 i$, which corresponds to just dividing the first term by the first term and the second by the second.
\item \( a \in [366.5, 367.5] \text{ and } b \in [-5, -2.5] \)

 $367.00  - 3.26 i$, which corresponds to forgetting to multiply the conjugate by the numerator and using a plus instead of a minus in the denominator.
\item \( a \in [-9, -6.5] \text{ and } b \in [-2, 1] \)

 $-7.60  - 0.89 i$, which corresponds to forgetting to multiply the conjugate by the numerator and not computing the conjugate correctly.
\item \( a \in [6, 8] \text{ and } b \in [-174.5, -172.5] \)

 $6.92  - 173.00 i$, which corresponds to forgetting to multiply the conjugate by the numerator.
\end{enumerate}

\textbf{General Comment:} Multiply the numerator and denominator by the *conjugate* of the denominator, then simplify. For example, if we have $2+3i$, the conjugate is $2-3i$.
}
\end{enumerate}

\end{document}