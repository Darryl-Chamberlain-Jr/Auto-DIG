\documentclass[14pt]{extbook}
\usepackage{multicol, enumerate, enumitem, hyperref, color, soul, setspace, parskip, fancyhdr} %General Packages
\usepackage{amssymb, amsthm, amsmath, latexsym, units, mathtools} %Math Packages
\everymath{\displaystyle} %All math in Display Style
% Packages with additional options
\usepackage[headsep=0.5cm,headheight=12pt, left=1 in,right= 1 in,top= 1 in,bottom= 1 in]{geometry}
\usepackage[usenames,dvipsnames]{xcolor}
\usepackage{dashrule}  % Package to use the command below to create lines between items
\newcommand{\litem}[1]{\item#1\hspace*{-1cm}\rule{\textwidth}{0.4pt}}
\pagestyle{fancy}
\lhead{Progress Quiz 9}
\chead{}
\rhead{Version ALL}
\lfoot{9541-5764}
\cfoot{}
\rfoot{Summer C 2021}
\begin{document}

\begin{enumerate}
\litem{
What are the \textit{possible Rational} roots of the polynomial below?\[ f(x) = 6x^{4} +2 x^{3} +4 x^{2} +5 x + 4 \]\begin{enumerate}[label=\Alph*.]
\item \( \pm 1,\pm 2,\pm 4 \)
\item \( \text{ All combinations of: }\frac{\pm 1,\pm 2,\pm 4}{\pm 1,\pm 2,\pm 3,\pm 6} \)
\item \( \pm 1,\pm 2,\pm 3,\pm 6 \)
\item \( \text{ All combinations of: }\frac{\pm 1,\pm 2,\pm 3,\pm 6}{\pm 1,\pm 2,\pm 4} \)
\item \( \text{ There is no formula or theorem that tells us all possible Rational roots.} \)

\end{enumerate} }
\litem{
Perform the division below. Then, find the intervals that correspond to the quotient in the form $ax^2+bx+c$ and remainder $r$.\[ \frac{12x^{3} -36 x -28}{x -2} \]\begin{enumerate}[label=\Alph*.]
\item \( a \in [11, 16], b \in [10, 17], c \in [-27, -21], \text{ and } r \in [-55, -47]. \)
\item \( a \in [24, 25], b \in [-49, -46], c \in [60, 64], \text{ and } r \in [-150, -142]. \)
\item \( a \in [11, 16], b \in [-26, -23], c \in [11, 17], \text{ and } r \in [-55, -47]. \)
\item \( a \in [24, 25], b \in [47, 50], c \in [60, 64], \text{ and } r \in [91, 96]. \)
\item \( a \in [11, 16], b \in [24, 26], c \in [11, 17], \text{ and } r \in [-5, -1]. \)

\end{enumerate} }
\litem{
Factor the polynomial below completely, knowing that $x -4$ is a factor. Then, choose the intervals the zeros of the polynomial belong to, where $z_1 \leq z_2 \leq z_3 \leq z_4$. \textit{To make the problem easier, all zeros are between -5 and 5.}\[ f(x) = 12x^{4} -61 x^{3} +15 x^{2} +178 x -120 \]\begin{enumerate}[label=\Alph*.]
\item \( z_1 \in [-0.96, -0.32], \text{   }  z_2 \in [1.3, 2.27], z_3 \in [1.63, 2.06], \text{   and   } z_4 \in [3.61, 4.69] \)
\item \( z_1 \in [-5.22, -3.24], \text{   }  z_2 \in [-2.02, -0.96], z_3 \in [-0.28, -0.21], \text{   and   } z_4 \in [4.73, 5.53] \)
\item \( z_1 \in [-2.16, -1.43], \text{   }  z_2 \in [0.41, 0.76], z_3 \in [1.63, 2.06], \text{   and   } z_4 \in [3.61, 4.69] \)
\item \( z_1 \in [-5.22, -3.24], \text{   }  z_2 \in [-2.02, -0.96], z_3 \in [-0.87, -0.32], \text{   and   } z_4 \in [1.55, 1.79] \)
\item \( z_1 \in [-5.22, -3.24], \text{   }  z_2 \in [-2.02, -0.96], z_3 \in [-1.42, -1.27], \text{   and   } z_4 \in [-0.02, 0.92] \)

\end{enumerate} }
\litem{
Factor the polynomial below completely. Then, choose the intervals the zeros of the polynomial belong to, where $z_1 \leq z_2 \leq z_3$. \textit{To make the problem easier, all zeros are between -5 and 5.}\[ f(x) = 10x^{3} -49 x^{2} +42 x + 45 \]\begin{enumerate}[label=\Alph*.]
\item \( z_1 \in [-1.8, -1.5], \text{   }  z_2 \in [0.33, 0.84], \text{   and   } z_3 \in [2.97, 3.01] \)
\item \( z_1 \in [-5.1, -4.2], \text{   }  z_2 \in [-3.32, -2.82], \text{   and   } z_3 \in [0.28, 0.38] \)
\item \( z_1 \in [-3.1, -2.9], \text{   }  z_2 \in [-1.21, -0.2], \text{   and   } z_3 \in [1.64, 1.76] \)
\item \( z_1 \in [-0.7, 0.4], \text{   }  z_2 \in [2.4, 2.99], \text{   and   } z_3 \in [2.97, 3.01] \)
\item \( z_1 \in [-3.1, -2.9], \text{   }  z_2 \in [-2.57, -2.28], \text{   and   } z_3 \in [0.47, 0.78] \)

\end{enumerate} }
\litem{
Factor the polynomial below completely. Then, choose the intervals the zeros of the polynomial belong to, where $z_1 \leq z_2 \leq z_3$. \textit{To make the problem easier, all zeros are between -5 and 5.}\[ f(x) = 15x^{3} +94 x^{2} +101 x + 30 \]\begin{enumerate}[label=\Alph*.]
\item \( z_1 \in [-5.29, -4.77], \text{   }  z_2 \in [-2, -1.62], \text{   and   } z_3 \in [-1.7, -1.3] \)
\item \( z_1 \in [-0.77, 0.55], \text{   }  z_2 \in [1.75, 2.15], \text{   and   } z_3 \in [4.3, 5.2] \)
\item \( z_1 \in [0.52, 0.97], \text{   }  z_2 \in [0.63, 0.76], \text{   and   } z_3 \in [4.3, 5.2] \)
\item \( z_1 \in [1.45, 1.92], \text{   }  z_2 \in [1.29, 1.9], \text{   and   } z_3 \in [4.3, 5.2] \)
\item \( z_1 \in [-5.29, -4.77], \text{   }  z_2 \in [-0.83, -0.09], \text{   and   } z_3 \in [-1.1, 1.1] \)

\end{enumerate} }
\litem{
Perform the division below. Then, find the intervals that correspond to the quotient in the form $ax^2+bx+c$ and remainder $r$.\[ \frac{6x^{3} -18 x^{2} -36 x + 43}{x -4} \]\begin{enumerate}[label=\Alph*.]
\item \( a \in [5, 7], \text{   } b \in [5, 11], \text{   } c \in [-12, -10], \text{   and   } r \in [-10, -3]. \)
\item \( a \in [24, 28], \text{   } b \in [-117, -109], \text{   } c \in [414, 423], \text{   and   } r \in [-1638, -1632]. \)
\item \( a \in [24, 28], \text{   } b \in [73, 79], \text{   } c \in [272, 279], \text{   and   } r \in [1145, 1153]. \)
\item \( a \in [5, 7], \text{   } b \in [-45, -36], \text{   } c \in [132, 137], \text{   and   } r \in [-490, -482]. \)
\item \( a \in [5, 7], \text{   } b \in [-4, 4], \text{   } c \in [-38, -31], \text{   and   } r \in [-68, -63]. \)

\end{enumerate} }
\litem{
Perform the division below. Then, find the intervals that correspond to the quotient in the form $ax^2+bx+c$ and remainder $r$.\[ \frac{20x^{3} -65 x^{2} -160 x -72}{x -5} \]\begin{enumerate}[label=\Alph*.]
\item \( a \in [15, 23], \text{   } b \in [32, 39], \text{   } c \in [13, 23], \text{   and   } r \in [2, 4]. \)
\item \( a \in [96, 103], \text{   } b \in [427, 441], \text{   } c \in [2012, 2022], \text{   and   } r \in [9997, 10006]. \)
\item \( a \in [15, 23], \text{   } b \in [14, 17], \text{   } c \in [-104, -98], \text{   and   } r \in [-479, -467]. \)
\item \( a \in [15, 23], \text{   } b \in [-172, -161], \text{   } c \in [663, 670], \text{   and   } r \in [-3400, -3394]. \)
\item \( a \in [96, 103], \text{   } b \in [-565, -560], \text{   } c \in [2665, 2671], \text{   and   } r \in [-13401, -13395]. \)

\end{enumerate} }
\litem{
Perform the division below. Then, find the intervals that correspond to the quotient in the form $ax^2+bx+c$ and remainder $r$.\[ \frac{12x^{3} +52 x^{2} -62}{x + 4} \]\begin{enumerate}[label=\Alph*.]
\item \( a \in [10, 14], b \in [3, 9], c \in [-21, -13], \text{ and } r \in [1, 8]. \)
\item \( a \in [-55, -46], b \in [-141, -136], c \in [-569, -557], \text{ and } r \in [-2302, -2299]. \)
\item \( a \in [-55, -46], b \in [244, 248], c \in [-976, -972], \text{ and } r \in [3838, 3846]. \)
\item \( a \in [10, 14], b \in [94, 102], c \in [394, 403], \text{ and } r \in [1537, 1539]. \)
\item \( a \in [10, 14], b \in [-13, -5], c \in [36, 41], \text{ and } r \in [-267, -260]. \)

\end{enumerate} }
\litem{
Factor the polynomial below completely, knowing that $x + 4$ is a factor. Then, choose the intervals the zeros of the polynomial belong to, where $z_1 \leq z_2 \leq z_3 \leq z_4$. \textit{To make the problem easier, all zeros are between -5 and 5.}\[ f(x) = 12x^{4} -23 x^{3} -244 x^{2} +235 x + 300 \]\begin{enumerate}[label=\Alph*.]
\item \( z_1 \in [-4.25, -3.89], \text{   }  z_2 \in [-0.85, -0.62], z_3 \in [1.41, 1.73], \text{   and   } z_4 \in [4.1, 5.5] \)
\item \( z_1 \in [-6.08, -4.94], \text{   }  z_2 \in [-0.65, -0.56], z_3 \in [1.32, 1.36], \text{   and   } z_4 \in [3.9, 4.7] \)
\item \( z_1 \in [-6.08, -4.94], \text{   }  z_2 \in [-0.49, -0.34], z_3 \in [2.96, 3.22], \text{   and   } z_4 \in [3.9, 4.7] \)
\item \( z_1 \in [-4.25, -3.89], \text{   }  z_2 \in [-1.37, -1.11], z_3 \in [0.45, 0.66], \text{   and   } z_4 \in [4.1, 5.5] \)
\item \( z_1 \in [-6.08, -4.94], \text{   }  z_2 \in [-1.84, -1.45], z_3 \in [0.7, 0.83], \text{   and   } z_4 \in [3.9, 4.7] \)

\end{enumerate} }
\litem{
What are the \textit{possible Integer} roots of the polynomial below?\[ f(x) = 4x^{4} +2 x^{3} +4 x^{2} +6 x + 6 \]\begin{enumerate}[label=\Alph*.]
\item \( \pm 1,\pm 2,\pm 3,\pm 6 \)
\item \( \pm 1,\pm 2,\pm 4 \)
\item \( \text{ All combinations of: }\frac{\pm 1,\pm 2,\pm 3,\pm 6}{\pm 1,\pm 2,\pm 4} \)
\item \( \text{ All combinations of: }\frac{\pm 1,\pm 2,\pm 4}{\pm 1,\pm 2,\pm 3,\pm 6} \)
\item \( \text{There is no formula or theorem that tells us all possible Integer roots.} \)

\end{enumerate} }
\litem{
What are the \textit{possible Integer} roots of the polynomial below?\[ f(x) = 6x^{3} +4 x^{2} +3 x + 3 \]\begin{enumerate}[label=\Alph*.]
\item \( \text{ All combinations of: }\frac{\pm 1,\pm 2,\pm 3,\pm 6}{\pm 1,\pm 3} \)
\item \( \pm 1,\pm 2,\pm 3,\pm 6 \)
\item \( \pm 1,\pm 3 \)
\item \( \text{ All combinations of: }\frac{\pm 1,\pm 3}{\pm 1,\pm 2,\pm 3,\pm 6} \)
\item \( \text{There is no formula or theorem that tells us all possible Integer roots.} \)

\end{enumerate} }
\litem{
Perform the division below. Then, find the intervals that correspond to the quotient in the form $ax^2+bx+c$ and remainder $r$.\[ \frac{15x^{3} -38 x^{2} + 34}{x -2} \]\begin{enumerate}[label=\Alph*.]
\item \( a \in [27, 31], b \in [21, 25], c \in [41, 48], \text{ and } r \in [122, 126]. \)
\item \( a \in [27, 31], b \in [-100, -94], c \in [196, 202], \text{ and } r \in [-362, -355]. \)
\item \( a \in [13, 19], b \in [-8, -7], c \in [-18, -15], \text{ and } r \in [0, 8]. \)
\item \( a \in [13, 19], b \in [-72, -66], c \in [135, 142], \text{ and } r \in [-244, -232]. \)
\item \( a \in [13, 19], b \in [-23, -20], c \in [-27, -20], \text{ and } r \in [10, 15]. \)

\end{enumerate} }
\litem{
Factor the polynomial below completely, knowing that $x -5$ is a factor. Then, choose the intervals the zeros of the polynomial belong to, where $z_1 \leq z_2 \leq z_3 \leq z_4$. \textit{To make the problem easier, all zeros are between -5 and 5.}\[ f(x) = 12x^{4} -1 x^{3} -266 x^{2} -205 x + 300 \]\begin{enumerate}[label=\Alph*.]
\item \( z_1 \in [-4.9, -3.1], \text{   }  z_2 \in [-1.79, -1.55], z_3 \in [0.73, 0.83], \text{   and   } z_4 \in [4.82, 5.32] \)
\item \( z_1 \in [-5.2, -4.6], \text{   }  z_2 \in [-1.39, -1.31], z_3 \in [0.52, 0.73], \text{   and   } z_4 \in [3.73, 4.71] \)
\item \( z_1 \in [-5.2, -4.6], \text{   }  z_2 \in [-0.91, -0.61], z_3 \in [1.58, 1.83], \text{   and   } z_4 \in [3.73, 4.71] \)
\item \( z_1 \in [-4.9, -3.1], \text{   }  z_2 \in [-0.71, -0.54], z_3 \in [1.24, 1.38], \text{   and   } z_4 \in [4.82, 5.32] \)
\item \( z_1 \in [-5.2, -4.6], \text{   }  z_2 \in [-3.17, -2.81], z_3 \in [0.4, 0.57], \text{   and   } z_4 \in [3.73, 4.71] \)

\end{enumerate} }
\litem{
Factor the polynomial below completely. Then, choose the intervals the zeros of the polynomial belong to, where $z_1 \leq z_2 \leq z_3$. \textit{To make the problem easier, all zeros are between -5 and 5.}\[ f(x) = 25x^{3} -50 x^{2} -69 x -18 \]\begin{enumerate}[label=\Alph*.]
\item \( z_1 \in [-3.9, -2.8], \text{   }  z_2 \in [-0.07, 0.38], \text{   and   } z_3 \in [2.7, 4.2] \)
\item \( z_1 \in [-2.6, -1.3], \text{   }  z_2 \in [-1.75, -1.59], \text{   and   } z_3 \in [2.7, 4.2] \)
\item \( z_1 \in [-3.9, -2.8], \text{   }  z_2 \in [0.23, 0.59], \text{   and   } z_3 \in [-0.3, 1.1] \)
\item \( z_1 \in [-3.9, -2.8], \text{   }  z_2 \in [1.62, 1.74], \text{   and   } z_3 \in [2.3, 2.9] \)
\item \( z_1 \in [-1.6, 0.1], \text{   }  z_2 \in [-0.46, -0.26], \text{   and   } z_3 \in [2.7, 4.2] \)

\end{enumerate} }
\litem{
Factor the polynomial below completely. Then, choose the intervals the zeros of the polynomial belong to, where $z_1 \leq z_2 \leq z_3$. \textit{To make the problem easier, all zeros are between -5 and 5.}\[ f(x) = 12x^{3} -77 x^{2} +131 x -60 \]\begin{enumerate}[label=\Alph*.]
\item \( z_1 \in [0.63, 0.78], \text{   }  z_2 \in [1.57, 1.81], \text{   and   } z_3 \in [4, 4.07] \)
\item \( z_1 \in [0.6, 0.71], \text{   }  z_2 \in [1.23, 1.4], \text{   and   } z_3 \in [4, 4.07] \)
\item \( z_1 \in [-4.1, -3.95], \text{   }  z_2 \in [-1.92, -1.56], \text{   and   } z_3 \in [-0.81, -0.65] \)
\item \( z_1 \in [-5.12, -4.95], \text{   }  z_2 \in [-4.14, -3.6], \text{   and   } z_3 \in [-0.28, -0.19] \)
\item \( z_1 \in [-4.1, -3.95], \text{   }  z_2 \in [-1.56, -1.04], \text{   and   } z_3 \in [-0.74, -0.46] \)

\end{enumerate} }
\litem{
Perform the division below. Then, find the intervals that correspond to the quotient in the form $ax^2+bx+c$ and remainder $r$.\[ \frac{16x^{3} -24 x^{2} -31 x + 35}{x -2} \]\begin{enumerate}[label=\Alph*.]
\item \( a \in [30, 39], \text{   } b \in [-89, -86], \text{   } c \in [145, 149], \text{   and   } r \in [-257, -250]. \)
\item \( a \in [13, 17], \text{   } b \in [1, 9], \text{   } c \in [-15, -11], \text{   and   } r \in [2, 13]. \)
\item \( a \in [13, 17], \text{   } b \in [-59, -55], \text{   } c \in [74, 85], \text{   and   } r \in [-127, -124]. \)
\item \( a \in [30, 39], \text{   } b \in [39, 43], \text{   } c \in [46, 50], \text{   and   } r \in [127, 135]. \)
\item \( a \in [13, 17], \text{   } b \in [-11, -7], \text{   } c \in [-43, -34], \text{   and   } r \in [-4, 2]. \)

\end{enumerate} }
\litem{
Perform the division below. Then, find the intervals that correspond to the quotient in the form $ax^2+bx+c$ and remainder $r$.\[ \frac{10x^{3} -40 x^{2} -10 x + 37}{x -4} \]\begin{enumerate}[label=\Alph*.]
\item \( a \in [7, 13], \text{   } b \in [-2, 7], \text{   } c \in [-10, -6], \text{   and   } r \in [-5, 0]. \)
\item \( a \in [7, 13], \text{   } b \in [-11, -3], \text{   } c \in [-42, -38], \text{   and   } r \in [-88, -79]. \)
\item \( a \in [35, 44], \text{   } b \in [-201, -193], \text{   } c \in [788, 794], \text{   and   } r \in [-3124, -3119]. \)
\item \( a \in [7, 13], \text{   } b \in [-83, -73], \text{   } c \in [306, 313], \text{   and   } r \in [-1203, -1196]. \)
\item \( a \in [35, 44], \text{   } b \in [119, 129], \text{   } c \in [464, 476], \text{   and   } r \in [1916, 1918]. \)

\end{enumerate} }
\litem{
Perform the division below. Then, find the intervals that correspond to the quotient in the form $ax^2+bx+c$ and remainder $r$.\[ \frac{8x^{3} -62 x + 35}{x + 3} \]\begin{enumerate}[label=\Alph*.]
\item \( a \in [-29, -20], b \in [-76, -71], c \in [-280, -275], \text{ and } r \in [-800, -798]. \)
\item \( a \in [-29, -20], b \in [69, 75], c \in [-280, -275], \text{ and } r \in [869, 873]. \)
\item \( a \in [8, 14], b \in [-35, -31], c \in [66, 67], \text{ and } r \in [-233, -227]. \)
\item \( a \in [8, 14], b \in [22, 30], c \in [9, 22], \text{ and } r \in [60, 72]. \)
\item \( a \in [8, 14], b \in [-26, -21], c \in [9, 22], \text{ and } r \in [4, 8]. \)

\end{enumerate} }
\litem{
Factor the polynomial below completely, knowing that $x -5$ is a factor. Then, choose the intervals the zeros of the polynomial belong to, where $z_1 \leq z_2 \leq z_3 \leq z_4$. \textit{To make the problem easier, all zeros are between -5 and 5.}\[ f(x) = 15x^{4} -44 x^{3} -159 x^{2} +8 x + 60 \]\begin{enumerate}[label=\Alph*.]
\item \( z_1 \in [-10, -4], \text{   }  z_2 \in [-0.26, -0.18], z_3 \in [1.99, 2.03], \text{   and   } z_4 \in [1, 4] \)
\item \( z_1 \in [-10, -4], \text{   }  z_2 \in [-0.66, -0.58], z_3 \in [0.65, 0.68], \text{   and   } z_4 \in [1, 4] \)
\item \( z_1 \in [-10, -4], \text{   }  z_2 \in [-1.68, -1.51], z_3 \in [1.46, 1.52], \text{   and   } z_4 \in [1, 4] \)
\item \( z_1 \in [-2, 1], \text{   }  z_2 \in [-0.73, -0.64], z_3 \in [0.58, 0.61], \text{   and   } z_4 \in [5, 6] \)
\item \( z_1 \in [-2, 1], \text{   }  z_2 \in [-1.52, -1.49], z_3 \in [1.63, 1.71], \text{   and   } z_4 \in [5, 6] \)

\end{enumerate} }
\litem{
What are the \textit{possible Rational} roots of the polynomial below?\[ f(x) = 3x^{2} +6 x + 7 \]\begin{enumerate}[label=\Alph*.]
\item \( \pm 1,\pm 3 \)
\item \( \text{ All combinations of: }\frac{\pm 1,\pm 3}{\pm 1,\pm 7} \)
\item \( \pm 1,\pm 7 \)
\item \( \text{ All combinations of: }\frac{\pm 1,\pm 7}{\pm 1,\pm 3} \)
\item \( \text{ There is no formula or theorem that tells us all possible Rational roots.} \)

\end{enumerate} }
\litem{
What are the \textit{possible Rational} roots of the polynomial below?\[ f(x) = 4x^{4} +6 x^{3} +5 x^{2} +4 x + 3 \]\begin{enumerate}[label=\Alph*.]
\item \( \text{ All combinations of: }\frac{\pm 1,\pm 3}{\pm 1,\pm 2,\pm 4} \)
\item \( \pm 1,\pm 3 \)
\item \( \text{ All combinations of: }\frac{\pm 1,\pm 2,\pm 4}{\pm 1,\pm 3} \)
\item \( \pm 1,\pm 2,\pm 4 \)
\item \( \text{ There is no formula or theorem that tells us all possible Rational roots.} \)

\end{enumerate} }
\litem{
Perform the division below. Then, find the intervals that correspond to the quotient in the form $ax^2+bx+c$ and remainder $r$.\[ \frac{20x^{3} -60 x + 44}{x + 2} \]\begin{enumerate}[label=\Alph*.]
\item \( a \in [19, 22], b \in [-62, -58], c \in [115, 129], \text{ and } r \in [-316, -315]. \)
\item \( a \in [19, 22], b \in [-41, -38], c \in [16, 27], \text{ and } r \in [2, 5]. \)
\item \( a \in [-41, -34], b \in [-82, -79], c \in [-222, -212], \text{ and } r \in [-396, -391]. \)
\item \( a \in [-41, -34], b \in [74, 87], c \in [-222, -212], \text{ and } r \in [478, 485]. \)
\item \( a \in [19, 22], b \in [37, 41], c \in [16, 27], \text{ and } r \in [81, 89]. \)

\end{enumerate} }
\litem{
Factor the polynomial below completely, knowing that $x + 5$ is a factor. Then, choose the intervals the zeros of the polynomial belong to, where $z_1 \leq z_2 \leq z_3 \leq z_4$. \textit{To make the problem easier, all zeros are between -5 and 5.}\[ f(x) = 15x^{4} +151 x^{3} +429 x^{2} +185 x -300 \]\begin{enumerate}[label=\Alph*.]
\item \( z_1 \in [-0.52, 0.32], \text{   }  z_2 \in [3.14, 4.93], z_3 \in [4.85, 5.74], \text{   and   } z_4 \in [4.91, 6.54] \)
\item \( z_1 \in [-5.08, -4.82], \text{   }  z_2 \in [-4.33, -2.95], z_3 \in [-1.8, -1.17], \text{   and   } z_4 \in [0.44, 1.16] \)
\item \( z_1 \in [-1.83, -1.54], \text{   }  z_2 \in [0.34, 0.99], z_3 \in [3.09, 4.44], \text{   and   } z_4 \in [4.91, 6.54] \)
\item \( z_1 \in [-5.08, -4.82], \text{   }  z_2 \in [-4.33, -2.95], z_3 \in [-0.77, -0.17], \text{   and   } z_4 \in [0.89, 2.04] \)
\item \( z_1 \in [-1.07, -0.25], \text{   }  z_2 \in [1.41, 2.22], z_3 \in [3.09, 4.44], \text{   and   } z_4 \in [4.91, 6.54] \)

\end{enumerate} }
\litem{
Factor the polynomial below completely. Then, choose the intervals the zeros of the polynomial belong to, where $z_1 \leq z_2 \leq z_3$. \textit{To make the problem easier, all zeros are between -5 and 5.}\[ f(x) = 12x^{3} +11 x^{2} -45 x -50 \]\begin{enumerate}[label=\Alph*.]
\item \( z_1 \in [-0.89, -0.61], \text{   }  z_2 \in [-0.68, -0.46], \text{   and   } z_3 \in [1.96, 2.53] \)
\item \( z_1 \in [-1.84, -1.27], \text{   }  z_2 \in [-1.31, -1.2], \text{   and   } z_3 \in [1.96, 2.53] \)
\item \( z_1 \in [-2.26, -1.88], \text{   }  z_2 \in [0.35, 0.54], \text{   and   } z_3 \in [4.65, 5.07] \)
\item \( z_1 \in [-2.26, -1.88], \text{   }  z_2 \in [0.45, 0.67], \text{   and   } z_3 \in [0.65, 0.85] \)
\item \( z_1 \in [-2.26, -1.88], \text{   }  z_2 \in [1.12, 1.42], \text{   and   } z_3 \in [1.03, 1.93] \)

\end{enumerate} }
\litem{
Factor the polynomial below completely. Then, choose the intervals the zeros of the polynomial belong to, where $z_1 \leq z_2 \leq z_3$. \textit{To make the problem easier, all zeros are between -5 and 5.}\[ f(x) = 10x^{3} + x^{2} -77 x + 30 \]\begin{enumerate}[label=\Alph*.]
\item \( z_1 \in [-2.06, -1.86], \text{   }  z_2 \in [-0.7, -0.5], \text{   and   } z_3 \in [2.63, 3.17] \)
\item \( z_1 \in [-3.19, -2.73], \text{   }  z_2 \in [0.32, 0.41], \text{   and   } z_3 \in [2.14, 2.85] \)
\item \( z_1 \in [-3.19, -2.73], \text{   }  z_2 \in [0.32, 0.41], \text{   and   } z_3 \in [2.14, 2.85] \)
\item \( z_1 \in [-2.57, -2.12], \text{   }  z_2 \in [-0.48, -0.3], \text{   and   } z_3 \in [2.63, 3.17] \)
\item \( z_1 \in [-2.57, -2.12], \text{   }  z_2 \in [-0.48, -0.3], \text{   and   } z_3 \in [2.63, 3.17] \)

\end{enumerate} }
\litem{
Perform the division below. Then, find the intervals that correspond to the quotient in the form $ax^2+bx+c$ and remainder $r$.\[ \frac{15x^{3} +65 x^{2} +90 x + 37}{x + 2} \]\begin{enumerate}[label=\Alph*.]
\item \( a \in [15, 20], \text{   } b \in [19, 21], \text{   } c \in [30, 31], \text{   and   } r \in [-56, -46]. \)
\item \( a \in [15, 20], \text{   } b \in [35, 38], \text{   } c \in [16, 22], \text{   and   } r \in [-4, -2]. \)
\item \( a \in [15, 20], \text{   } b \in [90, 97], \text{   } c \in [280, 281], \text{   and   } r \in [588, 607]. \)
\item \( a \in [-32, -28], \text{   } b \in [124, 127], \text{   } c \in [-163, -157], \text{   and   } r \in [356, 359]. \)
\item \( a \in [-32, -28], \text{   } b \in [4, 6], \text{   } c \in [96, 103], \text{   and   } r \in [227, 239]. \)

\end{enumerate} }
\litem{
Perform the division below. Then, find the intervals that correspond to the quotient in the form $ax^2+bx+c$ and remainder $r$.\[ \frac{6x^{3} +27 x^{2} +39 x + 23}{x + 2} \]\begin{enumerate}[label=\Alph*.]
\item \( a \in [-14, -8], \text{   } b \in [46, 55], \text{   } c \in [-64, -57], \text{   and   } r \in [149, 153]. \)
\item \( a \in [1, 10], \text{   } b \in [39, 40], \text{   } c \in [116, 119], \text{   and   } r \in [253, 263]. \)
\item \( a \in [-14, -8], \text{   } b \in [3, 5], \text{   } c \in [41, 48], \text{   and   } r \in [111, 118]. \)
\item \( a \in [1, 10], \text{   } b \in [9, 13], \text{   } c \in [12, 14], \text{   and   } r \in [-14, -7]. \)
\item \( a \in [1, 10], \text{   } b \in [15, 24], \text{   } c \in [3, 10], \text{   and   } r \in [2, 12]. \)

\end{enumerate} }
\litem{
Perform the division below. Then, find the intervals that correspond to the quotient in the form $ax^2+bx+c$ and remainder $r$.\[ \frac{12x^{3} -65 x^{2} + 120}{x -5} \]\begin{enumerate}[label=\Alph*.]
\item \( a \in [11, 16], b \in [-8, -1], c \in [-27, -21], \text{ and } r \in [-7, -1]. \)
\item \( a \in [11, 16], b \in [-125, -124], c \in [617, 628], \text{ and } r \in [-3013, -3003]. \)
\item \( a \in [60, 65], b \in [-369, -364], c \in [1819, 1828], \text{ and } r \in [-9010, -9000]. \)
\item \( a \in [11, 16], b \in [-19, -16], c \in [-69, -67], \text{ and } r \in [-152, -149]. \)
\item \( a \in [60, 65], b \in [235, 241], c \in [1174, 1176], \text{ and } r \in [5995, 5996]. \)

\end{enumerate} }
\litem{
Factor the polynomial below completely, knowing that $x -3$ is a factor. Then, choose the intervals the zeros of the polynomial belong to, where $z_1 \leq z_2 \leq z_3 \leq z_4$. \textit{To make the problem easier, all zeros are between -5 and 5.}\[ f(x) = 16x^{4} -112 x^{3} +167 x^{2} +175 x -300 \]\begin{enumerate}[label=\Alph*.]
\item \( z_1 \in [-4.63, -3.96], \text{   }  z_2 \in [-3.23, -2.25], z_3 \in [-1.08, -0.54], \text{   and   } z_4 \in [0.18, 1.04] \)
\item \( z_1 \in [-0.98, -0.53], \text{   }  z_2 \in [-0.07, 0.93], z_3 \in [2.84, 3.28], \text{   and   } z_4 \in [3.16, 4.57] \)
\item \( z_1 \in [-4.63, -3.96], \text{   }  z_2 \in [-3.23, -2.25], z_3 \in [-0.48, 0.1], \text{   and   } z_4 \in [4.77, 5.57] \)
\item \( z_1 \in [-4.63, -3.96], \text{   }  z_2 \in [-3.23, -2.25], z_3 \in [-1.68, -1.22], \text{   and   } z_4 \in [0.94, 1.37] \)
\item \( z_1 \in [-1.71, -1.18], \text{   }  z_2 \in [1.2, 2.35], z_3 \in [2.84, 3.28], \text{   and   } z_4 \in [3.16, 4.57] \)

\end{enumerate} }
\litem{
What are the \textit{possible Rational} roots of the polynomial below?\[ f(x) = 2x^{3} +7 x^{2} +7 x + 4 \]\begin{enumerate}[label=\Alph*.]
\item \( \pm 1,\pm 2,\pm 4 \)
\item \( \text{ All combinations of: }\frac{\pm 1,\pm 2}{\pm 1,\pm 2,\pm 4} \)
\item \( \text{ All combinations of: }\frac{\pm 1,\pm 2,\pm 4}{\pm 1,\pm 2} \)
\item \( \pm 1,\pm 2 \)
\item \( \text{ There is no formula or theorem that tells us all possible Rational roots.} \)

\end{enumerate} }
\end{enumerate}

\end{document}