\documentclass{extbook}[14pt]
\usepackage{multicol, enumerate, enumitem, hyperref, color, soul, setspace, parskip, fancyhdr, amssymb, amsthm, amsmath, latexsym, units, mathtools}
\everymath{\displaystyle}
\usepackage[headsep=0.5cm,headheight=0cm, left=1 in,right= 1 in,top= 1 in,bottom= 1 in]{geometry}
\usepackage{dashrule}  % Package to use the command below to create lines between items
\newcommand{\litem}[1]{\item #1

\rule{\textwidth}{0.4pt}}
\pagestyle{fancy}
\lhead{}
\chead{Answer Key for Progress Quiz 9 Version ALL}
\rhead{}
\lfoot{9541-5764}
\cfoot{}
\rfoot{Summer C 2021}
\begin{document}
\textbf{This key should allow you to understand why you choose the option you did (beyond just getting a question right or wrong). \href{https://xronos.clas.ufl.edu/mac1105spring2020/courseDescriptionAndMisc/Exams/LearningFromResults}{More instructions on how to use this key can be found here}.}

\textbf{If you have a suggestion to make the keys better, \href{https://forms.gle/CZkbZmPbC9XALEE88}{please fill out the short survey here}.}

\textit{Note: This key is auto-generated and may contain issues and/or errors. The keys are reviewed after each exam to ensure grading is done accurately. If there are issues (like duplicate options), they are noted in the offline gradebook. The keys are a work-in-progress to give students as many resources to improve as possible.}

\rule{\textwidth}{0.4pt}

\begin{enumerate}\litem{
Simplify the expression below and choose the interval the simplification is contained within.
\[ 13 - 4^2 + 11 \div 17 * 15 \div 10 \]The solution is \( -2.029 \), which is option B.\begin{enumerate}[label=\Alph*.]
\item \( [29.43, 30.3] \)

 29.971, which corresponds to an Order of Operations error: multiplying by negative before squaring. For example: $(-3)^2 \neq -3^2$
\item \( [-2.83, -1.86] \)

* -2.029, this is the correct option
\item \( [-3.87, -2.33] \)

 -2.996, which corresponds to an Order of Operations error: not reading left-to-right for multiplication/division.
\item \( [28.66, 29.69] \)

 29.004, which corresponds to two Order of Operations errors.
\item \( \text{None of the above} \)

 You may have gotten this by making an unanticipated error. If you got a value that is not any of the others, please let the coordinator know so they can help you figure out what happened.
\end{enumerate}

\textbf{General Comment:} While you may remember (or were taught) PEMDAS is done in order, it is actually done as P/E/MD/AS. When we are at MD or AS, we read left to right.
}
\litem{
Choose the \textbf{smallest} set of Real numbers that the number below belongs to.
\[ -\sqrt{\frac{-2160}{9}} \]The solution is \( \text{Not a Real number} \), which is option B.\begin{enumerate}[label=\Alph*.]
\item \( \text{Rational} \)

These are numbers that can be written as fraction of Integers (e.g., -2/3)
\item \( \text{Not a Real number} \)

* This is the correct option!
\item \( \text{Irrational} \)

These cannot be written as a fraction of Integers.
\item \( \text{Whole} \)

These are the counting numbers with 0 (0, 1, 2, 3, ...)
\item \( \text{Integer} \)

These are the negative and positive counting numbers (..., -3, -2, -1, 0, 1, 2, 3, ...)
\end{enumerate}

\textbf{General Comment:} First, you \textbf{NEED} to simplify the expression. This question simplifies to $-\sqrt{240} i$. 
 
 Be sure you look at the simplified fraction and not just the decimal expansion. Numbers such as 13, 17, and 19 provide \textbf{long but repeating/terminating decimal expansions!} 
 
 The only ways to *not* be a Real number are: dividing by 0 or taking the square root of a negative number. 
 
 Irrational numbers are more than just square root of 3: adding or subtracting values from square root of 3 is also irrational.
}
\litem{
Simplify the expression below into the form $a+bi$. Then, choose the intervals that $a$ and $b$ belong to.
\[ \frac{9 - 77 i}{8 - 6 i} \]The solution is \( 5.34  - 5.62 i \), which is option C.\begin{enumerate}[label=\Alph*.]
\item \( a \in [533, 535] \text{ and } b \in [-6.5, -4.5] \)

 $534.00  - 5.62 i$, which corresponds to forgetting to multiply the conjugate by the numerator and using a plus instead of a minus in the denominator.
\item \( a \in [-4.5, -2.5] \text{ and } b \in [-8, -6] \)

 $-3.90  - 6.70 i$, which corresponds to forgetting to multiply the conjugate by the numerator and not computing the conjugate correctly.
\item \( a \in [4.5, 6.5] \text{ and } b \in [-6.5, -4.5] \)

* $5.34  - 5.62 i$, which is the correct option.
\item \( a \in [1, 2.5] \text{ and } b \in [12, 14] \)

 $1.12  + 12.83 i$, which corresponds to just dividing the first term by the first term and the second by the second.
\item \( a \in [4.5, 6.5] \text{ and } b \in [-562.5, -561] \)

 $5.34  - 562.00 i$, which corresponds to forgetting to multiply the conjugate by the numerator.
\end{enumerate}

\textbf{General Comment:} Multiply the numerator and denominator by the *conjugate* of the denominator, then simplify. For example, if we have $2+3i$, the conjugate is $2-3i$.
}
\litem{
Simplify the expression below into the form $a+bi$. Then, choose the intervals that $a$ and $b$ belong to.
\[ (4 - 9 i)(-2 - 10 i) \]The solution is \( -98 - 22 i \), which is option D.\begin{enumerate}[label=\Alph*.]
\item \( a \in [79, 87] \text{ and } b \in [56, 62] \)

 $82 + 58 i$, which corresponds to adding a minus sign in the second term.
\item \( a \in [-16, -7] \text{ and } b \in [88, 93] \)

 $-8 + 90 i$, which corresponds to just multiplying the real terms to get the real part of the solution and the coefficients in the complex terms to get the complex part.
\item \( a \in [-99, -94] \text{ and } b \in [19, 23] \)

 $-98 + 22 i$, which corresponds to adding a minus sign in both terms.
\item \( a \in [-99, -94] \text{ and } b \in [-24, -21] \)

* $-98 - 22 i$, which is the correct option.
\item \( a \in [79, 87] \text{ and } b \in [-64, -57] \)

 $82 - 58 i$, which corresponds to adding a minus sign in the first term.
\end{enumerate}

\textbf{General Comment:} You can treat $i$ as a variable and distribute. Just remember that $i^2=-1$, so you can continue to reduce after you distribute.
}
\litem{
Choose the \textbf{smallest} set of Complex numbers that the number below belongs to.
\[ \sqrt{\frac{0}{15}}+\sqrt{4}i \]The solution is \( \text{Pure Imaginary} \), which is option D.\begin{enumerate}[label=\Alph*.]
\item \( \text{Rational} \)

These are numbers that can be written as fraction of Integers (e.g., -2/3 + 5)
\item \( \text{Irrational} \)

These cannot be written as a fraction of Integers. Remember: $\pi$ is not an Integer!
\item \( \text{Not a Complex Number} \)

This is not a number. The only non-Complex number we know is dividing by 0 as this is not a number!
\item \( \text{Pure Imaginary} \)

* This is the correct option!
\item \( \text{Nonreal Complex} \)

This is a Complex number $(a+bi)$ that is not Real (has $i$ as part of the number).
\end{enumerate}

\textbf{General Comment:} Be sure to simplify $i^2 = -1$. This may remove the imaginary portion for your number. If you are having trouble, you may want to look at the \textit{Subgroups of the Real Numbers} section.
}
\litem{
Simplify the expression below into the form $a+bi$. Then, choose the intervals that $a$ and $b$ belong to.
\[ (-8 + 5 i)(-9 - 7 i) \]The solution is \( 107 + 11 i \), which is option C.\begin{enumerate}[label=\Alph*.]
\item \( a \in [31, 44] \text{ and } b \in [98, 104] \)

 $37 + 101 i$, which corresponds to adding a minus sign in the first term.
\item \( a \in [65, 78] \text{ and } b \in [-37, -29] \)

 $72 - 35 i$, which corresponds to just multiplying the real terms to get the real part of the solution and the coefficients in the complex terms to get the complex part.
\item \( a \in [107, 110] \text{ and } b \in [11, 15] \)

* $107 + 11 i$, which is the correct option.
\item \( a \in [107, 110] \text{ and } b \in [-15, -7] \)

 $107 - 11 i$, which corresponds to adding a minus sign in both terms.
\item \( a \in [31, 44] \text{ and } b \in [-101, -99] \)

 $37 - 101 i$, which corresponds to adding a minus sign in the second term.
\end{enumerate}

\textbf{General Comment:} You can treat $i$ as a variable and distribute. Just remember that $i^2=-1$, so you can continue to reduce after you distribute.
}
\litem{
Simplify the expression below and choose the interval the simplification is contained within.
\[ 5 - 13 \div 19 * 7 - (20 * 4) \]The solution is \( -79.789 \), which is option D.\begin{enumerate}[label=\Alph*.]
\item \( [84.7, 85.14] \)

 84.902, which corresponds to not distributing addition and subtraction correctly.
\item \( [-79.27, -77.74] \)

 -79.158, which corresponds to not distributing a negative correctly.
\item \( [-75.97, -73.49] \)

 -75.098, which corresponds to an Order of Operations error: not reading left-to-right for multiplication/division.
\item \( [-80.42, -79.52] \)

* -79.789, which is the correct option.
\item \( \text{None of the above} \)

 You may have gotten this by making an unanticipated error. If you got a value that is not any of the others, please let the coordinator know so they can help you figure out what happened.
\end{enumerate}

\textbf{General Comment:} While you may remember (or were taught) PEMDAS is done in order, it is actually done as P/E/MD/AS. When we are at MD or AS, we read left to right.
}
\litem{
Choose the \textbf{smallest} set of Complex numbers that the number below belongs to.
\[ \sqrt{\frac{-1575}{15}}+\sqrt{60} \]The solution is \( \text{Nonreal Complex} \), which is option B.\begin{enumerate}[label=\Alph*.]
\item \( \text{Rational} \)

These are numbers that can be written as fraction of Integers (e.g., -2/3 + 5)
\item \( \text{Nonreal Complex} \)

* This is the correct option!
\item \( \text{Not a Complex Number} \)

This is not a number. The only non-Complex number we know is dividing by 0 as this is not a number!
\item \( \text{Pure Imaginary} \)

This is a Complex number $(a+bi)$ that \textbf{only} has an imaginary part like $2i$.
\item \( \text{Irrational} \)

These cannot be written as a fraction of Integers. Remember: $\pi$ is not an Integer!
\end{enumerate}

\textbf{General Comment:} Be sure to simplify $i^2 = -1$. This may remove the imaginary portion for your number. If you are having trouble, you may want to look at the \textit{Subgroups of the Real Numbers} section.
}
\litem{
Simplify the expression below into the form $a+bi$. Then, choose the intervals that $a$ and $b$ belong to.
\[ \frac{-36 - 77 i}{-1 + 6 i} \]The solution is \( -11.51  + 7.92 i \), which is option E.\begin{enumerate}[label=\Alph*.]
\item \( a \in [-428, -425.5] \text{ and } b \in [6, 8.5] \)

 $-426.00  + 7.92 i$, which corresponds to forgetting to multiply the conjugate by the numerator and using a plus instead of a minus in the denominator.
\item \( a \in [13, 14.5] \text{ and } b \in [-4.5, -3] \)

 $13.46  - 3.76 i$, which corresponds to forgetting to multiply the conjugate by the numerator and not computing the conjugate correctly.
\item \( a \in [-12.5, -11] \text{ and } b \in [292.5, 294] \)

 $-11.51  + 293.00 i$, which corresponds to forgetting to multiply the conjugate by the numerator.
\item \( a \in [35, 37.5] \text{ and } b \in [-13.5, -12.5] \)

 $36.00  - 12.83 i$, which corresponds to just dividing the first term by the first term and the second by the second.
\item \( a \in [-12.5, -11] \text{ and } b \in [6, 8.5] \)

* $-11.51  + 7.92 i$, which is the correct option.
\end{enumerate}

\textbf{General Comment:} Multiply the numerator and denominator by the *conjugate* of the denominator, then simplify. For example, if we have $2+3i$, the conjugate is $2-3i$.
}
\litem{
Choose the \textbf{smallest} set of Real numbers that the number below belongs to.
\[ \sqrt{\frac{-525}{5}} \]The solution is \( \text{Not a Real number} \), which is option B.\begin{enumerate}[label=\Alph*.]
\item \( \text{Whole} \)

These are the counting numbers with 0 (0, 1, 2, 3, ...)
\item \( \text{Not a Real number} \)

* This is the correct option!
\item \( \text{Rational} \)

These are numbers that can be written as fraction of Integers (e.g., -2/3)
\item \( \text{Integer} \)

These are the negative and positive counting numbers (..., -3, -2, -1, 0, 1, 2, 3, ...)
\item \( \text{Irrational} \)

These cannot be written as a fraction of Integers.
\end{enumerate}

\textbf{General Comment:} First, you \textbf{NEED} to simplify the expression. This question simplifies to $\sqrt{105} i$. 
 
 Be sure you look at the simplified fraction and not just the decimal expansion. Numbers such as 13, 17, and 19 provide \textbf{long but repeating/terminating decimal expansions!} 
 
 The only ways to *not* be a Real number are: dividing by 0 or taking the square root of a negative number. 
 
 Irrational numbers are more than just square root of 3: adding or subtracting values from square root of 3 is also irrational.
}
\litem{
Simplify the expression below and choose the interval the simplification is contained within.
\[ 4 - 20^2 + 2 \div 6 * 16 \div 7 \]The solution is \( -395.238 \), which is option A.\begin{enumerate}[label=\Alph*.]
\item \( [-395.45, -394.63] \)

* -395.238, this is the correct option
\item \( [403.57, 404.34] \)

 404.003, which corresponds to two Order of Operations errors.
\item \( [-396.71, -395.71] \)

 -395.997, which corresponds to an Order of Operations error: not reading left-to-right for multiplication/division.
\item \( [404.69, 404.85] \)

 404.762, which corresponds to an Order of Operations error: multiplying by negative before squaring. For example: $(-3)^2 \neq -3^2$
\item \( \text{None of the above} \)

 You may have gotten this by making an unanticipated error. If you got a value that is not any of the others, please let the coordinator know so they can help you figure out what happened.
\end{enumerate}

\textbf{General Comment:} While you may remember (or were taught) PEMDAS is done in order, it is actually done as P/E/MD/AS. When we are at MD or AS, we read left to right.
}
\litem{
Choose the \textbf{smallest} set of Real numbers that the number below belongs to.
\[ -\sqrt{\frac{576}{49}} \]The solution is \( \text{Rational} \), which is option A.\begin{enumerate}[label=\Alph*.]
\item \( \text{Rational} \)

* This is the correct option!
\item \( \text{Whole} \)

These are the counting numbers with 0 (0, 1, 2, 3, ...)
\item \( \text{Irrational} \)

These cannot be written as a fraction of Integers.
\item \( \text{Not a Real number} \)

These are Nonreal Complex numbers \textbf{OR} things that are not numbers (e.g., dividing by 0).
\item \( \text{Integer} \)

These are the negative and positive counting numbers (..., -3, -2, -1, 0, 1, 2, 3, ...)
\end{enumerate}

\textbf{General Comment:} First, you \textbf{NEED} to simplify the expression. This question simplifies to $-\frac{24}{7}$. 
 
 Be sure you look at the simplified fraction and not just the decimal expansion. Numbers such as 13, 17, and 19 provide \textbf{long but repeating/terminating decimal expansions!} 
 
 The only ways to *not* be a Real number are: dividing by 0 or taking the square root of a negative number. 
 
 Irrational numbers are more than just square root of 3: adding or subtracting values from square root of 3 is also irrational.
}
\litem{
Simplify the expression below into the form $a+bi$. Then, choose the intervals that $a$ and $b$ belong to.
\[ \frac{54 + 44 i}{-1 + 5 i} \]The solution is \( 6.38  - 12.08 i \), which is option D.\begin{enumerate}[label=\Alph*.]
\item \( a \in [-55.5, -53] \text{ and } b \in [8.74, 8.84] \)

 $-54.00  + 8.80 i$, which corresponds to just dividing the first term by the first term and the second by the second.
\item \( a \in [6, 7.5] \text{ and } b \in [-314.01, -313.96] \)

 $6.38  - 314.00 i$, which corresponds to forgetting to multiply the conjugate by the numerator.
\item \( a \in [165, 167] \text{ and } b \in [-12.08, -12.05] \)

 $166.00  - 12.08 i$, which corresponds to forgetting to multiply the conjugate by the numerator and using a plus instead of a minus in the denominator.
\item \( a \in [6, 7.5] \text{ and } b \in [-12.08, -12.05] \)

* $6.38  - 12.08 i$, which is the correct option.
\item \( a \in [-11, -10] \text{ and } b \in [8.65, 8.71] \)

 $-10.54  + 8.69 i$, which corresponds to forgetting to multiply the conjugate by the numerator and not computing the conjugate correctly.
\end{enumerate}

\textbf{General Comment:} Multiply the numerator and denominator by the *conjugate* of the denominator, then simplify. For example, if we have $2+3i$, the conjugate is $2-3i$.
}
\litem{
Simplify the expression below into the form $a+bi$. Then, choose the intervals that $a$ and $b$ belong to.
\[ (-10 + 2 i)(-4 + 3 i) \]The solution is \( 34 - 38 i \), which is option A.\begin{enumerate}[label=\Alph*.]
\item \( a \in [34, 36] \text{ and } b \in [-44, -34] \)

* $34 - 38 i$, which is the correct option.
\item \( a \in [42, 50] \text{ and } b \in [-25, -21] \)

 $46 - 22 i$, which corresponds to adding a minus sign in the first term.
\item \( a \in [34, 36] \text{ and } b \in [34, 39] \)

 $34 + 38 i$, which corresponds to adding a minus sign in both terms.
\item \( a \in [38, 42] \text{ and } b \in [-9, 10] \)

 $40 + 6 i$, which corresponds to just multiplying the real terms to get the real part of the solution and the coefficients in the complex terms to get the complex part.
\item \( a \in [42, 50] \text{ and } b \in [17, 27] \)

 $46 + 22 i$, which corresponds to adding a minus sign in the second term.
\end{enumerate}

\textbf{General Comment:} You can treat $i$ as a variable and distribute. Just remember that $i^2=-1$, so you can continue to reduce after you distribute.
}
\litem{
Choose the \textbf{smallest} set of Complex numbers that the number below belongs to.
\[ \sqrt{\frac{-1815}{11}}+\sqrt{0}i \]The solution is \( \text{Pure Imaginary} \), which is option B.\begin{enumerate}[label=\Alph*.]
\item \( \text{Irrational} \)

These cannot be written as a fraction of Integers. Remember: $\pi$ is not an Integer!
\item \( \text{Pure Imaginary} \)

* This is the correct option!
\item \( \text{Nonreal Complex} \)

This is a Complex number $(a+bi)$ that is not Real (has $i$ as part of the number).
\item \( \text{Not a Complex Number} \)

This is not a number. The only non-Complex number we know is dividing by 0 as this is not a number!
\item \( \text{Rational} \)

These are numbers that can be written as fraction of Integers (e.g., -2/3 + 5)
\end{enumerate}

\textbf{General Comment:} Be sure to simplify $i^2 = -1$. This may remove the imaginary portion for your number. If you are having trouble, you may want to look at the \textit{Subgroups of the Real Numbers} section.
}
\litem{
Simplify the expression below into the form $a+bi$. Then, choose the intervals that $a$ and $b$ belong to.
\[ (-2 + 8 i)(5 + 7 i) \]The solution is \( -66 + 26 i \), which is option C.\begin{enumerate}[label=\Alph*.]
\item \( a \in [-68, -63] \text{ and } b \in [-26.15, -24.4] \)

 $-66 - 26 i$, which corresponds to adding a minus sign in both terms.
\item \( a \in [46, 49] \text{ and } b \in [53.91, 54.51] \)

 $46 + 54 i$, which corresponds to adding a minus sign in the second term.
\item \( a \in [-68, -63] \text{ and } b \in [25.99, 27.39] \)

* $-66 + 26 i$, which is the correct option.
\item \( a \in [-18, -5] \text{ and } b \in [55.28, 56.57] \)

 $-10 + 56 i$, which corresponds to just multiplying the real terms to get the real part of the solution and the coefficients in the complex terms to get the complex part.
\item \( a \in [46, 49] \text{ and } b \in [-54.41, -53.76] \)

 $46 - 54 i$, which corresponds to adding a minus sign in the first term.
\end{enumerate}

\textbf{General Comment:} You can treat $i$ as a variable and distribute. Just remember that $i^2=-1$, so you can continue to reduce after you distribute.
}
\litem{
Simplify the expression below and choose the interval the simplification is contained within.
\[ 18 - 12^2 + 7 \div 8 * 13 \div 16 \]The solution is \( -125.289 \), which is option A.\begin{enumerate}[label=\Alph*.]
\item \( [-125.62, -124.54] \)

* -125.289, this is the correct option
\item \( [-126.26, -125.96] \)

 -125.996, which corresponds to an Order of Operations error: not reading left-to-right for multiplication/division.
\item \( [161.86, 162.42] \)

 162.004, which corresponds to two Order of Operations errors.
\item \( [162.05, 162.91] \)

 162.711, which corresponds to an Order of Operations error: multiplying by negative before squaring. For example: $(-3)^2 \neq -3^2$
\item \( \text{None of the above} \)

 You may have gotten this by making an unanticipated error. If you got a value that is not any of the others, please let the coordinator know so they can help you figure out what happened.
\end{enumerate}

\textbf{General Comment:} While you may remember (or were taught) PEMDAS is done in order, it is actually done as P/E/MD/AS. When we are at MD or AS, we read left to right.
}
\litem{
Choose the \textbf{smallest} set of Complex numbers that the number below belongs to.
\[ \frac{2}{-11}+81i^2 \]The solution is \( \text{Rational} \), which is option C.\begin{enumerate}[label=\Alph*.]
\item \( \text{Pure Imaginary} \)

This is a Complex number $(a+bi)$ that \textbf{only} has an imaginary part like $2i$.
\item \( \text{Nonreal Complex} \)

This is a Complex number $(a+bi)$ that is not Real (has $i$ as part of the number).
\item \( \text{Rational} \)

* This is the correct option!
\item \( \text{Irrational} \)

These cannot be written as a fraction of Integers. Remember: $\pi$ is not an Integer!
\item \( \text{Not a Complex Number} \)

This is not a number. The only non-Complex number we know is dividing by 0 as this is not a number!
\end{enumerate}

\textbf{General Comment:} Be sure to simplify $i^2 = -1$. This may remove the imaginary portion for your number. If you are having trouble, you may want to look at the \textit{Subgroups of the Real Numbers} section.
}
\litem{
Simplify the expression below into the form $a+bi$. Then, choose the intervals that $a$ and $b$ belong to.
\[ \frac{63 + 22 i}{-5 - i} \]The solution is \( -12.96  - 1.81 i \), which is option A.\begin{enumerate}[label=\Alph*.]
\item \( a \in [-13.39, -12.83] \text{ and } b \in [-3, -1] \)

* $-12.96  - 1.81 i$, which is the correct option.
\item \( a \in [-13.39, -12.83] \text{ and } b \in [-48, -45] \)

 $-12.96  - 47.00 i$, which corresponds to forgetting to multiply the conjugate by the numerator.
\item \( a \in [-12.9, -12.46] \text{ and } b \in [-23.5, -21.5] \)

 $-12.60  - 22.00 i$, which corresponds to just dividing the first term by the first term and the second by the second.
\item \( a \in [-337.01, -336.76] \text{ and } b \in [-3, -1] \)

 $-337.00  - 1.81 i$, which corresponds to forgetting to multiply the conjugate by the numerator and using a plus instead of a minus in the denominator.
\item \( a \in [-11.33, -11.17] \text{ and } b \in [-8.5, -6] \)

 $-11.27  - 6.65 i$, which corresponds to forgetting to multiply the conjugate by the numerator and not computing the conjugate correctly.
\end{enumerate}

\textbf{General Comment:} Multiply the numerator and denominator by the *conjugate* of the denominator, then simplify. For example, if we have $2+3i$, the conjugate is $2-3i$.
}
\litem{
Choose the \textbf{smallest} set of Real numbers that the number below belongs to.
\[ -\sqrt{\frac{-990}{5}} \]The solution is \( \text{Not a Real number} \), which is option E.\begin{enumerate}[label=\Alph*.]
\item \( \text{Whole} \)

These are the counting numbers with 0 (0, 1, 2, 3, ...)
\item \( \text{Integer} \)

These are the negative and positive counting numbers (..., -3, -2, -1, 0, 1, 2, 3, ...)
\item \( \text{Irrational} \)

These cannot be written as a fraction of Integers.
\item \( \text{Rational} \)

These are numbers that can be written as fraction of Integers (e.g., -2/3)
\item \( \text{Not a Real number} \)

* This is the correct option!
\end{enumerate}

\textbf{General Comment:} First, you \textbf{NEED} to simplify the expression. This question simplifies to $-\sqrt{198} i$. 
 
 Be sure you look at the simplified fraction and not just the decimal expansion. Numbers such as 13, 17, and 19 provide \textbf{long but repeating/terminating decimal expansions!} 
 
 The only ways to *not* be a Real number are: dividing by 0 or taking the square root of a negative number. 
 
 Irrational numbers are more than just square root of 3: adding or subtracting values from square root of 3 is also irrational.
}
\litem{
Simplify the expression below and choose the interval the simplification is contained within.
\[ 15 - 20 \div 7 * 13 - (11 * 14) \]The solution is \( -176.143 \), which is option B.\begin{enumerate}[label=\Alph*.]
\item \( [-140.22, -134.22] \)

 -139.220, which corresponds to an Order of Operations error: not reading left-to-right for multiplication/division.
\item \( [-181.14, -174.14] \)

* -176.143, which is the correct option.
\item \( [168.78, 170.78] \)

 168.780, which corresponds to not distributing addition and subtraction correctly.
\item \( [-466, -460] \)

 -464.000, which corresponds to not distributing a negative correctly.
\item \( \text{None of the above} \)

 You may have gotten this by making an unanticipated error. If you got a value that is not any of the others, please let the coordinator know so they can help you figure out what happened.
\end{enumerate}

\textbf{General Comment:} While you may remember (or were taught) PEMDAS is done in order, it is actually done as P/E/MD/AS. When we are at MD or AS, we read left to right.
}
\litem{
Choose the \textbf{smallest} set of Real numbers that the number below belongs to.
\[ -\sqrt{\frac{9025}{361}} \]The solution is \( \text{Integer} \), which is option B.\begin{enumerate}[label=\Alph*.]
\item \( \text{Whole} \)

These are the counting numbers with 0 (0, 1, 2, 3, ...)
\item \( \text{Integer} \)

* This is the correct option!
\item \( \text{Irrational} \)

These cannot be written as a fraction of Integers.
\item \( \text{Not a Real number} \)

These are Nonreal Complex numbers \textbf{OR} things that are not numbers (e.g., dividing by 0).
\item \( \text{Rational} \)

These are numbers that can be written as fraction of Integers (e.g., -2/3)
\end{enumerate}

\textbf{General Comment:} First, you \textbf{NEED} to simplify the expression. This question simplifies to $-95$. 
 
 Be sure you look at the simplified fraction and not just the decimal expansion. Numbers such as 13, 17, and 19 provide \textbf{long but repeating/terminating decimal expansions!} 
 
 The only ways to *not* be a Real number are: dividing by 0 or taking the square root of a negative number. 
 
 Irrational numbers are more than just square root of 3: adding or subtracting values from square root of 3 is also irrational.
}
\litem{
Simplify the expression below into the form $a+bi$. Then, choose the intervals that $a$ and $b$ belong to.
\[ \frac{-54 + 88 i}{3 - 4 i} \]The solution is \( -20.56  + 1.92 i \), which is option E.\begin{enumerate}[label=\Alph*.]
\item \( a \in [-514.5, -513.5] \text{ and } b \in [1.5, 2.5] \)

 $-514.00  + 1.92 i$, which corresponds to forgetting to multiply the conjugate by the numerator and using a plus instead of a minus in the denominator.
\item \( a \in [7, 8.5] \text{ and } b \in [19, 20] \)

 $7.60  + 19.20 i$, which corresponds to forgetting to multiply the conjugate by the numerator and not computing the conjugate correctly.
\item \( a \in [-19, -17] \text{ and } b \in [-22.5, -21.5] \)

 $-18.00  - 22.00 i$, which corresponds to just dividing the first term by the first term and the second by the second.
\item \( a \in [-21.5, -19.5] \text{ and } b \in [47.5, 48.5] \)

 $-20.56  + 48.00 i$, which corresponds to forgetting to multiply the conjugate by the numerator.
\item \( a \in [-21.5, -19.5] \text{ and } b \in [1.5, 2.5] \)

* $-20.56  + 1.92 i$, which is the correct option.
\end{enumerate}

\textbf{General Comment:} Multiply the numerator and denominator by the *conjugate* of the denominator, then simplify. For example, if we have $2+3i$, the conjugate is $2-3i$.
}
\litem{
Simplify the expression below into the form $a+bi$. Then, choose the intervals that $a$ and $b$ belong to.
\[ (9 - 8 i)(-4 + 3 i) \]The solution is \( -12 + 59 i \), which is option C.\begin{enumerate}[label=\Alph*.]
\item \( a \in [-65, -55] \text{ and } b \in [-5, -2] \)

 $-60 - 5 i$, which corresponds to adding a minus sign in the first term.
\item \( a \in [-40, -27] \text{ and } b \in [-28, -19] \)

 $-36 - 24 i$, which corresponds to just multiplying the real terms to get the real part of the solution and the coefficients in the complex terms to get the complex part.
\item \( a \in [-15, -5] \text{ and } b \in [57, 61] \)

* $-12 + 59 i$, which is the correct option.
\item \( a \in [-65, -55] \text{ and } b \in [5, 6] \)

 $-60 + 5 i$, which corresponds to adding a minus sign in the second term.
\item \( a \in [-15, -5] \text{ and } b \in [-59, -56] \)

 $-12 - 59 i$, which corresponds to adding a minus sign in both terms.
\end{enumerate}

\textbf{General Comment:} You can treat $i$ as a variable and distribute. Just remember that $i^2=-1$, so you can continue to reduce after you distribute.
}
\litem{
Choose the \textbf{smallest} set of Complex numbers that the number below belongs to.
\[ \frac{-20}{2}+49i^2 \]The solution is \( \text{Rational} \), which is option A.\begin{enumerate}[label=\Alph*.]
\item \( \text{Rational} \)

* This is the correct option!
\item \( \text{Nonreal Complex} \)

This is a Complex number $(a+bi)$ that is not Real (has $i$ as part of the number).
\item \( \text{Not a Complex Number} \)

This is not a number. The only non-Complex number we know is dividing by 0 as this is not a number!
\item \( \text{Irrational} \)

These cannot be written as a fraction of Integers. Remember: $\pi$ is not an Integer!
\item \( \text{Pure Imaginary} \)

This is a Complex number $(a+bi)$ that \textbf{only} has an imaginary part like $2i$.
\end{enumerate}

\textbf{General Comment:} Be sure to simplify $i^2 = -1$. This may remove the imaginary portion for your number. If you are having trouble, you may want to look at the \textit{Subgroups of the Real Numbers} section.
}
\litem{
Simplify the expression below into the form $a+bi$. Then, choose the intervals that $a$ and $b$ belong to.
\[ (10 - 6 i)(2 + 4 i) \]The solution is \( 44 + 28 i \), which is option C.\begin{enumerate}[label=\Alph*.]
\item \( a \in [41, 47] \text{ and } b \in [-29, -25] \)

 $44 - 28 i$, which corresponds to adding a minus sign in both terms.
\item \( a \in [17, 23] \text{ and } b \in [-25, -18] \)

 $20 - 24 i$, which corresponds to just multiplying the real terms to get the real part of the solution and the coefficients in the complex terms to get the complex part.
\item \( a \in [41, 47] \text{ and } b \in [24, 29] \)

* $44 + 28 i$, which is the correct option.
\item \( a \in [-6, -1] \text{ and } b \in [50, 53] \)

 $-4 + 52 i$, which corresponds to adding a minus sign in the first term.
\item \( a \in [-6, -1] \text{ and } b \in [-53, -51] \)

 $-4 - 52 i$, which corresponds to adding a minus sign in the second term.
\end{enumerate}

\textbf{General Comment:} You can treat $i$ as a variable and distribute. Just remember that $i^2=-1$, so you can continue to reduce after you distribute.
}
\litem{
Simplify the expression below and choose the interval the simplification is contained within.
\[ 6 - 12^2 + 2 \div 16 * 17 \div 13 \]The solution is \( -137.837 \), which is option A.\begin{enumerate}[label=\Alph*.]
\item \( [-137.91, -137.54] \)

* -137.837, this is the correct option
\item \( [149.66, 150.07] \)

 150.001, which corresponds to two Order of Operations errors.
\item \( [150.15, 150.24] \)

 150.163, which corresponds to an Order of Operations error: multiplying by negative before squaring. For example: $(-3)^2 \neq -3^2$
\item \( [-138.24, -137.98] \)

 -137.999, which corresponds to an Order of Operations error: not reading left-to-right for multiplication/division.
\item \( \text{None of the above} \)

 You may have gotten this by making an unanticipated error. If you got a value that is not any of the others, please let the coordinator know so they can help you figure out what happened.
\end{enumerate}

\textbf{General Comment:} While you may remember (or were taught) PEMDAS is done in order, it is actually done as P/E/MD/AS. When we are at MD or AS, we read left to right.
}
\litem{
Choose the \textbf{smallest} set of Complex numbers that the number below belongs to.
\[ -\sqrt{\frac{324}{625}} + 25i^2 \]The solution is \( \text{Rational} \), which is option E.\begin{enumerate}[label=\Alph*.]
\item \( \text{Pure Imaginary} \)

This is a Complex number $(a+bi)$ that \textbf{only} has an imaginary part like $2i$.
\item \( \text{Irrational} \)

These cannot be written as a fraction of Integers. Remember: $\pi$ is not an Integer!
\item \( \text{Not a Complex Number} \)

This is not a number. The only non-Complex number we know is dividing by 0 as this is not a number!
\item \( \text{Nonreal Complex} \)

This is a Complex number $(a+bi)$ that is not Real (has $i$ as part of the number).
\item \( \text{Rational} \)

* This is the correct option!
\end{enumerate}

\textbf{General Comment:} Be sure to simplify $i^2 = -1$. This may remove the imaginary portion for your number. If you are having trouble, you may want to look at the \textit{Subgroups of the Real Numbers} section.
}
\litem{
Simplify the expression below into the form $a+bi$. Then, choose the intervals that $a$ and $b$ belong to.
\[ \frac{27 - 11 i}{4 + 8 i} \]The solution is \( 0.25  - 3.25 i \), which is option E.\begin{enumerate}[label=\Alph*.]
\item \( a \in [6.5, 8] \text{ and } b \in [-2, 0.5] \)

 $6.75  - 1.38 i$, which corresponds to just dividing the first term by the first term and the second by the second.
\item \( a \in [-0.5, 0.5] \text{ and } b \in [-261.5, -258.5] \)

 $0.25  - 260.00 i$, which corresponds to forgetting to multiply the conjugate by the numerator.
\item \( a \in [18.5, 20.5] \text{ and } b \in [-4, -2.5] \)

 $20.00  - 3.25 i$, which corresponds to forgetting to multiply the conjugate by the numerator and using a plus instead of a minus in the denominator.
\item \( a \in [1.5, 3.5] \text{ and } b \in [1, 4.5] \)

 $2.45  + 2.15 i$, which corresponds to forgetting to multiply the conjugate by the numerator and not computing the conjugate correctly.
\item \( a \in [-0.5, 0.5] \text{ and } b \in [-4, -2.5] \)

* $0.25  - 3.25 i$, which is the correct option.
\end{enumerate}

\textbf{General Comment:} Multiply the numerator and denominator by the *conjugate* of the denominator, then simplify. For example, if we have $2+3i$, the conjugate is $2-3i$.
}
\litem{
Choose the \textbf{smallest} set of Real numbers that the number below belongs to.
\[ \sqrt{\frac{36}{529}} \]The solution is \( \text{Rational} \), which is option B.\begin{enumerate}[label=\Alph*.]
\item \( \text{Whole} \)

These are the counting numbers with 0 (0, 1, 2, 3, ...)
\item \( \text{Rational} \)

* This is the correct option!
\item \( \text{Irrational} \)

These cannot be written as a fraction of Integers.
\item \( \text{Not a Real number} \)

These are Nonreal Complex numbers \textbf{OR} things that are not numbers (e.g., dividing by 0).
\item \( \text{Integer} \)

These are the negative and positive counting numbers (..., -3, -2, -1, 0, 1, 2, 3, ...)
\end{enumerate}

\textbf{General Comment:} First, you \textbf{NEED} to simplify the expression. This question simplifies to $\frac{6}{23}$. 
 
 Be sure you look at the simplified fraction and not just the decimal expansion. Numbers such as 13, 17, and 19 provide \textbf{long but repeating/terminating decimal expansions!} 
 
 The only ways to *not* be a Real number are: dividing by 0 or taking the square root of a negative number. 
 
 Irrational numbers are more than just square root of 3: adding or subtracting values from square root of 3 is also irrational.
}
\end{enumerate}

\end{document}