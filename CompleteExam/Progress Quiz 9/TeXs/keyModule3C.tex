\documentclass{extbook}[14pt]
\usepackage{multicol, enumerate, enumitem, hyperref, color, soul, setspace, parskip, fancyhdr, amssymb, amsthm, amsmath, latexsym, units, mathtools}
\everymath{\displaystyle}
\usepackage[headsep=0.5cm,headheight=0cm, left=1 in,right= 1 in,top= 1 in,bottom= 1 in]{geometry}
\usepackage{dashrule}  % Package to use the command below to create lines between items
\newcommand{\litem}[1]{\item #1

\rule{\textwidth}{0.4pt}}
\pagestyle{fancy}
\lhead{}
\chead{Answer Key for Progress Quiz 9 Version C}
\rhead{}
\lfoot{9541-5764}
\cfoot{}
\rfoot{Summer C 2021}
\begin{document}
\textbf{This key should allow you to understand why you choose the option you did (beyond just getting a question right or wrong). \href{https://xronos.clas.ufl.edu/mac1105spring2020/courseDescriptionAndMisc/Exams/LearningFromResults}{More instructions on how to use this key can be found here}.}

\textbf{If you have a suggestion to make the keys better, \href{https://forms.gle/CZkbZmPbC9XALEE88}{please fill out the short survey here}.}

\textit{Note: This key is auto-generated and may contain issues and/or errors. The keys are reviewed after each exam to ensure grading is done accurately. If there are issues (like duplicate options), they are noted in the offline gradebook. The keys are a work-in-progress to give students as many resources to improve as possible.}

\rule{\textwidth}{0.4pt}

\begin{enumerate}\litem{
Solve the linear inequality below. Then, choose the constant and interval combination that describes the solution set.
\[ -3 + 6 x > 8 x \text{ or } 6 + 3 x < 4 x \]The solution is \( (-\infty, -1.5) \text{ or } (6.0, \infty) \), which is option C.\begin{enumerate}[label=\Alph*.]
\item \( (-\infty, a) \cup (b, \infty), \text{ where } a \in [-9, -3] \text{ and } b \in [-3, 2.25] \)

Corresponds to inverting the inequality and negating the solution.
\item \( (-\infty, a] \cup [b, \infty), \text{ where } a \in [-9, -4.5] \text{ and } b \in [0.75, 2.25] \)

Corresponds to including the endpoints AND negating.
\item \( (-\infty, a) \cup (b, \infty), \text{ where } a \in [-3, 2.25] \text{ and } b \in [3, 9] \)

 * Correct option.
\item \( (-\infty, a] \cup [b, \infty), \text{ where } a \in [-4.5, -0.75] \text{ and } b \in [3.75, 9.75] \)

Corresponds to including the endpoints (when they should be excluded).
\item \( (-\infty, \infty) \)

Corresponds to the variable canceling, which does not happen in this instance.
\end{enumerate}

\textbf{General Comment:} When multiplying or dividing by a negative, flip the sign.
}
\litem{
Solve the linear inequality below. Then, choose the constant and interval combination that describes the solution set.
\[ \frac{9}{4} - \frac{4}{7} x \leq \frac{3}{5} x - \frac{4}{8} \]The solution is \( [2.348, \infty) \), which is option D.\begin{enumerate}[label=\Alph*.]
\item \( [a, \infty), \text{ where } a \in [-3.75, 0.75] \)

 $[-2.348, \infty)$, which corresponds to negating the endpoint of the solution.
\item \( (-\infty, a], \text{ where } a \in [-0.75, 8.25] \)

 $(-\infty, 2.348]$, which corresponds to switching the direction of the interval. You likely did this if you did not flip the inequality when dividing by a negative!
\item \( (-\infty, a], \text{ where } a \in [-3, -1.5] \)

 $(-\infty, -2.348]$, which corresponds to switching the direction of the interval AND negating the endpoint. You likely did this if you did not flip the inequality when dividing by a negative as well as not moving values over to a side properly.
\item \( [a, \infty), \text{ where } a \in [1.5, 3.75] \)

* $[2.348, \infty)$, which is the correct option.
\item \( \text{None of the above}. \)

You may have chosen this if you thought the inequality did not match the ends of the intervals.
\end{enumerate}

\textbf{General Comment:} Remember that less/greater than or equal to includes the endpoint, while less/greater do not. Also, remember that you need to flip the inequality when you multiply or divide by a negative.
}
\litem{
Solve the linear inequality below. Then, choose the constant and interval combination that describes the solution set.
\[ \frac{-4}{7} - \frac{10}{4} x > \frac{-3}{5} x + \frac{8}{6} \]The solution is \( (-\infty, -1.003) \), which is option B.\begin{enumerate}[label=\Alph*.]
\item \( (a, \infty), \text{ where } a \in [0.3, 1.2] \)

 $(1.003, \infty)$, which corresponds to switching the direction of the interval AND negating the endpoint. You likely did this if you did not flip the inequality when dividing by a negative as well as not moving values over to a side properly.
\item \( (-\infty, a), \text{ where } a \in [-1.88, 0.6] \)

* $(-\infty, -1.003)$, which is the correct option.
\item \( (a, \infty), \text{ where } a \in [-2.4, 0.22] \)

 $(-1.003, \infty)$, which corresponds to switching the direction of the interval. You likely did this if you did not flip the inequality when dividing by a negative!
\item \( (-\infty, a), \text{ where } a \in [0.67, 1.12] \)

 $(-\infty, 1.003)$, which corresponds to negating the endpoint of the solution.
\item \( \text{None of the above}. \)

You may have chosen this if you thought the inequality did not match the ends of the intervals.
\end{enumerate}

\textbf{General Comment:} Remember that less/greater than or equal to includes the endpoint, while less/greater do not. Also, remember that you need to flip the inequality when you multiply or divide by a negative.
}
\litem{
Using an interval or intervals, describe all the $x$-values within or including a distance of the given values.
\[ \text{ More than } 2 \text{ units from the number } 6. \]The solution is \( (-\infty, 4) \cup (8, \infty) \), which is option A.\begin{enumerate}[label=\Alph*.]
\item \( (-\infty, 4) \cup (8, \infty) \)

This describes the values more than 2 from 6
\item \( (-\infty, 4] \cup [8, \infty) \)

This describes the values no less than 2 from 6
\item \( [4, 8] \)

This describes the values no more than 2 from 6
\item \( (4, 8) \)

This describes the values less than 2 from 6
\item \( \text{None of the above} \)

You likely thought the values in the interval were not correct.
\end{enumerate}

\textbf{General Comment:} When thinking about this language, it helps to draw a number line and try points.
}
\litem{
Solve the linear inequality below. Then, choose the constant and interval combination that describes the solution set.
\[ 5 + 8 x \leq \frac{53 x + 4}{6} < 9 + 8 x \]The solution is \( \text{None of the above.} \), which is option E.\begin{enumerate}[label=\Alph*.]
\item \( (-\infty, a) \cup [b, \infty), \text{ where } a \in [-9, -1.5] \text{ and } b \in [-11.25, -9] \)

$(-\infty, -5.20) \cup [-10.00, \infty)$, which corresponds to displaying the and-inequality as an or-inequality AND flipping the inequality AND getting negatives of the actual endpoints.
\item \( (a, b], \text{ where } a \in [-6, 0] \text{ and } b \in [-15, -6.75] \)

$(-5.20, -10.00]$, which corresponds to flipping the inequality and getting negatives of the actual endpoints.
\item \( (-\infty, a] \cup (b, \infty), \text{ where } a \in [-9.75, -3] \text{ and } b \in [-12.75, -5.25] \)

$(-\infty, -5.20] \cup (-10.00, \infty)$, which corresponds to displaying the and-inequality as an or-inequality and getting negatives of the actual endpoints.
\item \( [a, b), \text{ where } a \in [-8.25, -0.75] \text{ and } b \in [-11.25, -8.25] \)

$[-5.20, -10.00)$, which is the correct interval but negatives of the actual endpoints.
\item \( \text{None of the above.} \)

* This is correct as the answer should be $[5.20, 10.00)$.
\end{enumerate}

\textbf{General Comment:} To solve, you will need to break up the compound inequality into two inequalities. Be sure to keep track of the inequality! It may be best to draw a number line and graph your solution.
}
\litem{
Solve the linear inequality below. Then, choose the constant and interval combination that describes the solution set.
\[ -10x + 5 < -9x + 9 \]The solution is \( (-4.0, \infty) \), which is option D.\begin{enumerate}[label=\Alph*.]
\item \( (-\infty, a), \text{ where } a \in [-7, 0] \)

 $(-\infty, -4.0)$, which corresponds to switching the direction of the interval. You likely did this if you did not flip the inequality when dividing by a negative!
\item \( (-\infty, a), \text{ where } a \in [2, 8] \)

 $(-\infty, 4.0)$, which corresponds to switching the direction of the interval AND negating the endpoint. You likely did this if you did not flip the inequality when dividing by a negative as well as not moving values over to a side properly.
\item \( (a, \infty), \text{ where } a \in [-1, 8] \)

 $(4.0, \infty)$, which corresponds to negating the endpoint of the solution.
\item \( (a, \infty), \text{ where } a \in [-11, -1] \)

* $(-4.0, \infty)$, which is the correct option.
\item \( \text{None of the above}. \)

You may have chosen this if you thought the inequality did not match the ends of the intervals.
\end{enumerate}

\textbf{General Comment:} Remember that less/greater than or equal to includes the endpoint, while less/greater do not. Also, remember that you need to flip the inequality when you multiply or divide by a negative.
}
\litem{
Solve the linear inequality below. Then, choose the constant and interval combination that describes the solution set.
\[ 7 + 8 x > 11 x \text{ or } 9 + 7 x < 9 x \]The solution is \( (-\infty, 2.333) \text{ or } (4.5, \infty) \), which is option B.\begin{enumerate}[label=\Alph*.]
\item \( (-\infty, a] \cup [b, \infty), \text{ where } a \in [0.75, 7.5] \text{ and } b \in [3.75, 9] \)

Corresponds to including the endpoints (when they should be excluded).
\item \( (-\infty, a) \cup (b, \infty), \text{ where } a \in [0, 6.75] \text{ and } b \in [0, 6.75] \)

 * Correct option.
\item \( (-\infty, a) \cup (b, \infty), \text{ where } a \in [-6, -3.75] \text{ and } b \in [-5.25, -0.75] \)

Corresponds to inverting the inequality and negating the solution.
\item \( (-\infty, a] \cup [b, \infty), \text{ where } a \in [-6.75, 0] \text{ and } b \in [-6, 3.75] \)

Corresponds to including the endpoints AND negating.
\item \( (-\infty, \infty) \)

Corresponds to the variable canceling, which does not happen in this instance.
\end{enumerate}

\textbf{General Comment:} When multiplying or dividing by a negative, flip the sign.
}
\litem{
Using an interval or intervals, describe all the $x$-values within or including a distance of the given values.
\[ \text{ More than } 9 \text{ units from the number } 8. \]The solution is \( (-\infty, -1) \cup (17, \infty) \), which is option C.\begin{enumerate}[label=\Alph*.]
\item \( (-\infty, -1] \cup [17, \infty) \)

This describes the values no less than 9 from 8
\item \( (-1, 17) \)

This describes the values less than 9 from 8
\item \( (-\infty, -1) \cup (17, \infty) \)

This describes the values more than 9 from 8
\item \( [-1, 17] \)

This describes the values no more than 9 from 8
\item \( \text{None of the above} \)

You likely thought the values in the interval were not correct.
\end{enumerate}

\textbf{General Comment:} When thinking about this language, it helps to draw a number line and try points.
}
\litem{
Solve the linear inequality below. Then, choose the constant and interval combination that describes the solution set.
\[ 6 - 6 x \leq \frac{-7 x - 8}{3} < 5 - 4 x \]The solution is \( [2.36, 4.60) \), which is option C.\begin{enumerate}[label=\Alph*.]
\item \( (-\infty, a) \cup [b, \infty), \text{ where } a \in [-2.25, 7.5] \text{ and } b \in [-1.5, 6.75] \)

$(-\infty, 2.36) \cup [4.60, \infty)$, which corresponds to displaying the and-inequality as an or-inequality AND flipping the inequality.
\item \( (-\infty, a] \cup (b, \infty), \text{ where } a \in [1.5, 6.75] \text{ and } b \in [3.75, 6] \)

$(-\infty, 2.36] \cup (4.60, \infty)$, which corresponds to displaying the and-inequality as an or-inequality.
\item \( [a, b), \text{ where } a \in [-2.25, 5.25] \text{ and } b \in [2.25, 9] \)

$[2.36, 4.60)$, which is the correct option.
\item \( (a, b], \text{ where } a \in [-2.25, 3] \text{ and } b \in [2.25, 9] \)

$(2.36, 4.60]$, which corresponds to flipping the inequality.
\item \( \text{None of the above.} \)


\end{enumerate}

\textbf{General Comment:} To solve, you will need to break up the compound inequality into two inequalities. Be sure to keep track of the inequality! It may be best to draw a number line and graph your solution.
}
\litem{
Solve the linear inequality below. Then, choose the constant and interval combination that describes the solution set.
\[ -5x + 7 \geq 9x -6 \]The solution is \( (-\infty, 0.929] \), which is option D.\begin{enumerate}[label=\Alph*.]
\item \( (-\infty, a], \text{ where } a \in [-3.8, -0.4] \)

 $(-\infty, -0.929]$, which corresponds to negating the endpoint of the solution.
\item \( [a, \infty), \text{ where } a \in [-1.6, -0.3] \)

 $[-0.929, \infty)$, which corresponds to switching the direction of the interval AND negating the endpoint. You likely did this if you did not flip the inequality when dividing by a negative as well as not moving values over to a side properly.
\item \( [a, \infty), \text{ where } a \in [-0.6, 2.5] \)

 $[0.929, \infty)$, which corresponds to switching the direction of the interval. You likely did this if you did not flip the inequality when dividing by a negative!
\item \( (-\infty, a], \text{ where } a \in [-0.7, 1.9] \)

* $(-\infty, 0.929]$, which is the correct option.
\item \( \text{None of the above}. \)

You may have chosen this if you thought the inequality did not match the ends of the intervals.
\end{enumerate}

\textbf{General Comment:} Remember that less/greater than or equal to includes the endpoint, while less/greater do not. Also, remember that you need to flip the inequality when you multiply or divide by a negative.
}
\end{enumerate}

\end{document}