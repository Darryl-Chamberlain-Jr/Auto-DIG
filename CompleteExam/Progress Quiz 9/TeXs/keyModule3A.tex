\documentclass{extbook}[14pt]
\usepackage{multicol, enumerate, enumitem, hyperref, color, soul, setspace, parskip, fancyhdr, amssymb, amsthm, amsmath, latexsym, units, mathtools}
\everymath{\displaystyle}
\usepackage[headsep=0.5cm,headheight=0cm, left=1 in,right= 1 in,top= 1 in,bottom= 1 in]{geometry}
\usepackage{dashrule}  % Package to use the command below to create lines between items
\newcommand{\litem}[1]{\item #1

\rule{\textwidth}{0.4pt}}
\pagestyle{fancy}
\lhead{}
\chead{Answer Key for Progress Quiz 9 Version A}
\rhead{}
\lfoot{9541-5764}
\cfoot{}
\rfoot{Summer C 2021}
\begin{document}
\textbf{This key should allow you to understand why you choose the option you did (beyond just getting a question right or wrong). \href{https://xronos.clas.ufl.edu/mac1105spring2020/courseDescriptionAndMisc/Exams/LearningFromResults}{More instructions on how to use this key can be found here}.}

\textbf{If you have a suggestion to make the keys better, \href{https://forms.gle/CZkbZmPbC9XALEE88}{please fill out the short survey here}.}

\textit{Note: This key is auto-generated and may contain issues and/or errors. The keys are reviewed after each exam to ensure grading is done accurately. If there are issues (like duplicate options), they are noted in the offline gradebook. The keys are a work-in-progress to give students as many resources to improve as possible.}

\rule{\textwidth}{0.4pt}

\begin{enumerate}\litem{
Solve the linear inequality below. Then, choose the constant and interval combination that describes the solution set.
\[ -6 + 5 x > 8 x \text{ or } 3 + 5 x < 6 x \]The solution is \( (-\infty, -2.0) \text{ or } (3.0, \infty) \), which is option D.\begin{enumerate}[label=\Alph*.]
\item \( (-\infty, a] \cup [b, \infty), \text{ where } a \in [-5.33, -2.1] \text{ and } b \in [0.07, 2.55] \)

Corresponds to including the endpoints AND negating.
\item \( (-\infty, a) \cup (b, \infty), \text{ where } a \in [-3.2, -2.9] \text{ and } b \in [0.72, 2.54] \)

Corresponds to inverting the inequality and negating the solution.
\item \( (-\infty, a] \cup [b, \infty), \text{ where } a \in [-2.1, -1.12] \text{ and } b \in [2.92, 3.67] \)

Corresponds to including the endpoints (when they should be excluded).
\item \( (-\infty, a) \cup (b, \infty), \text{ where } a \in [-2.86, -1.35] \text{ and } b \in [2.86, 3.55] \)

 * Correct option.
\item \( (-\infty, \infty) \)

Corresponds to the variable canceling, which does not happen in this instance.
\end{enumerate}

\textbf{General Comment:} When multiplying or dividing by a negative, flip the sign.
}
\litem{
Solve the linear inequality below. Then, choose the constant and interval combination that describes the solution set.
\[ \frac{-3}{5} - \frac{10}{7} x \geq \frac{-5}{3} x - \frac{9}{6} \]The solution is \( [-3.78, \infty) \), which is option A.\begin{enumerate}[label=\Alph*.]
\item \( [a, \infty), \text{ where } a \in [-6, 0] \)

* $[-3.78, \infty)$, which is the correct option.
\item \( (-\infty, a], \text{ where } a \in [-8.25, -3] \)

 $(-\infty, -3.78]$, which corresponds to switching the direction of the interval. You likely did this if you did not flip the inequality when dividing by a negative!
\item \( [a, \infty), \text{ where } a \in [-0.75, 6] \)

 $[3.78, \infty)$, which corresponds to negating the endpoint of the solution.
\item \( (-\infty, a], \text{ where } a \in [2.25, 5.25] \)

 $(-\infty, 3.78]$, which corresponds to switching the direction of the interval AND negating the endpoint. You likely did this if you did not flip the inequality when dividing by a negative as well as not moving values over to a side properly.
\item \( \text{None of the above}. \)

You may have chosen this if you thought the inequality did not match the ends of the intervals.
\end{enumerate}

\textbf{General Comment:} Remember that less/greater than or equal to includes the endpoint, while less/greater do not. Also, remember that you need to flip the inequality when you multiply or divide by a negative.
}
\litem{
Solve the linear inequality below. Then, choose the constant and interval combination that describes the solution set.
\[ \frac{9}{7} - \frac{7}{6} x > \frac{-3}{3} x + \frac{6}{8} \]The solution is \( (-\infty, 3.214) \), which is option A.\begin{enumerate}[label=\Alph*.]
\item \( (-\infty, a), \text{ where } a \in [1.5, 3.75] \)

* $(-\infty, 3.214)$, which is the correct option.
\item \( (-\infty, a), \text{ where } a \in [-4.5, -0.75] \)

 $(-\infty, -3.214)$, which corresponds to negating the endpoint of the solution.
\item \( (a, \infty), \text{ where } a \in [0, 4.5] \)

 $(3.214, \infty)$, which corresponds to switching the direction of the interval. You likely did this if you did not flip the inequality when dividing by a negative!
\item \( (a, \infty), \text{ where } a \in [-3.75, -1.5] \)

 $(-3.214, \infty)$, which corresponds to switching the direction of the interval AND negating the endpoint. You likely did this if you did not flip the inequality when dividing by a negative as well as not moving values over to a side properly.
\item \( \text{None of the above}. \)

You may have chosen this if you thought the inequality did not match the ends of the intervals.
\end{enumerate}

\textbf{General Comment:} Remember that less/greater than or equal to includes the endpoint, while less/greater do not. Also, remember that you need to flip the inequality when you multiply or divide by a negative.
}
\litem{
Using an interval or intervals, describe all the $x$-values within or including a distance of the given values.
\[ \text{ Less than } 2 \text{ units from the number } 7. \]The solution is \( \text{None of the above} \), which is option E.\begin{enumerate}[label=\Alph*.]
\item \( (-\infty, -5) \cup (9, \infty) \)

This describes the values more than 7 from 2
\item \( (-\infty, -5] \cup [9, \infty) \)

This describes the values no less than 7 from 2
\item \( (-5, 9) \)

This describes the values less than 7 from 2
\item \( [-5, 9] \)

This describes the values no more than 7 from 2
\item \( \text{None of the above} \)

Options A-D described the values [more/less than] 7 units from 2, which is the reverse of what the question asked.
\end{enumerate}

\textbf{General Comment:} When thinking about this language, it helps to draw a number line and try points.
}
\litem{
Solve the linear inequality below. Then, choose the constant and interval combination that describes the solution set.
\[ -4 + 8 x < \frac{74 x - 5}{9} \leq -6 + 7 x \]The solution is \( \text{None of the above.} \), which is option E.\begin{enumerate}[label=\Alph*.]
\item \( (-\infty, a) \cup [b, \infty), \text{ where } a \in [12.75, 15.75] \text{ and } b \in [1.5, 9] \)

$(-\infty, 15.50) \cup [4.45, \infty)$, which corresponds to displaying the and-inequality as an or-inequality and getting negatives of the actual endpoints.
\item \( (a, b], \text{ where } a \in [12, 16.5] \text{ and } b \in [2.25, 6] \)

$(15.50, 4.45]$, which is the correct interval but negatives of the actual endpoints.
\item \( (-\infty, a] \cup (b, \infty), \text{ where } a \in [10.5, 18.75] \text{ and } b \in [2.25, 5.25] \)

$(-\infty, 15.50] \cup (4.45, \infty)$, which corresponds to displaying the and-inequality as an or-inequality AND flipping the inequality AND getting negatives of the actual endpoints.
\item \( [a, b), \text{ where } a \in [10.5, 16.5] \text{ and } b \in [3.75, 12] \)

$[15.50, 4.45)$, which corresponds to flipping the inequality and getting negatives of the actual endpoints.
\item \( \text{None of the above.} \)

* This is correct as the answer should be $(-15.50, -4.45]$.
\end{enumerate}

\textbf{General Comment:} To solve, you will need to break up the compound inequality into two inequalities. Be sure to keep track of the inequality! It may be best to draw a number line and graph your solution.
}
\litem{
Solve the linear inequality below. Then, choose the constant and interval combination that describes the solution set.
\[ -5x + 5 > 8x + 8 \]The solution is \( (-\infty, -0.231) \), which is option D.\begin{enumerate}[label=\Alph*.]
\item \( (a, \infty), \text{ where } a \in [-0.23, 0.79] \)

 $(0.231, \infty)$, which corresponds to switching the direction of the interval AND negating the endpoint. You likely did this if you did not flip the inequality when dividing by a negative as well as not moving values over to a side properly.
\item \( (a, \infty), \text{ where } a \in [-1, -0.15] \)

 $(-0.231, \infty)$, which corresponds to switching the direction of the interval. You likely did this if you did not flip the inequality when dividing by a negative!
\item \( (-\infty, a), \text{ where } a \in [-0.1, 1.1] \)

 $(-\infty, 0.231)$, which corresponds to negating the endpoint of the solution.
\item \( (-\infty, a), \text{ where } a \in [-2.6, 0.2] \)

* $(-\infty, -0.231)$, which is the correct option.
\item \( \text{None of the above}. \)

You may have chosen this if you thought the inequality did not match the ends of the intervals.
\end{enumerate}

\textbf{General Comment:} Remember that less/greater than or equal to includes the endpoint, while less/greater do not. Also, remember that you need to flip the inequality when you multiply or divide by a negative.
}
\litem{
Solve the linear inequality below. Then, choose the constant and interval combination that describes the solution set.
\[ 3 + 7 x > 10 x \text{ or } 3 + 4 x < 5 x \]The solution is \( (-\infty, 1.0) \text{ or } (3.0, \infty) \), which is option D.\begin{enumerate}[label=\Alph*.]
\item \( (-\infty, a] \cup [b, \infty), \text{ where } a \in [-4.5, 0.75] \text{ and } b \in [-3.75, -0.75] \)

Corresponds to including the endpoints AND negating.
\item \( (-\infty, a) \cup (b, \infty), \text{ where } a \in [-4.05, -1.35] \text{ and } b \in [-3, 0.75] \)

Corresponds to inverting the inequality and negating the solution.
\item \( (-\infty, a] \cup [b, \infty), \text{ where } a \in [-0.75, 3.75] \text{ and } b \in [0, 6] \)

Corresponds to including the endpoints (when they should be excluded).
\item \( (-\infty, a) \cup (b, \infty), \text{ where } a \in [-0.75, 1.12] \text{ and } b \in [0.75, 7.5] \)

 * Correct option.
\item \( (-\infty, \infty) \)

Corresponds to the variable canceling, which does not happen in this instance.
\end{enumerate}

\textbf{General Comment:} When multiplying or dividing by a negative, flip the sign.
}
\litem{
Using an interval or intervals, describe all the $x$-values within or including a distance of the given values.
\[ \text{ Less than } 3 \text{ units from the number } 8. \]The solution is \( \text{None of the above} \), which is option E.\begin{enumerate}[label=\Alph*.]
\item \( [-5, 11] \)

This describes the values no more than 8 from 3
\item \( (-\infty, -5] \cup [11, \infty) \)

This describes the values no less than 8 from 3
\item \( (-\infty, -5) \cup (11, \infty) \)

This describes the values more than 8 from 3
\item \( (-5, 11) \)

This describes the values less than 8 from 3
\item \( \text{None of the above} \)

Options A-D described the values [more/less than] 8 units from 3, which is the reverse of what the question asked.
\end{enumerate}

\textbf{General Comment:} When thinking about this language, it helps to draw a number line and try points.
}
\litem{
Solve the linear inequality below. Then, choose the constant and interval combination that describes the solution set.
\[ -9 + 8 x \leq \frac{35 x + 3}{4} < -6 + 5 x \]The solution is \( [-13.00, -1.80) \), which is option B.\begin{enumerate}[label=\Alph*.]
\item \( (-\infty, a) \cup [b, \infty), \text{ where } a \in [-22.5, -7.5] \text{ and } b \in [-6.75, 1.5] \)

$(-\infty, -13.00) \cup [-1.80, \infty)$, which corresponds to displaying the and-inequality as an or-inequality AND flipping the inequality.
\item \( [a, b), \text{ where } a \in [-15.75, -12] \text{ and } b \in [-3, 0.75] \)

$[-13.00, -1.80)$, which is the correct option.
\item \( (a, b], \text{ where } a \in [-20.25, -11.25] \text{ and } b \in [-2.4, -0.07] \)

$(-13.00, -1.80]$, which corresponds to flipping the inequality.
\item \( (-\infty, a] \cup (b, \infty), \text{ where } a \in [-16.5, -6.75] \text{ and } b \in [-5.62, 1.35] \)

$(-\infty, -13.00] \cup (-1.80, \infty)$, which corresponds to displaying the and-inequality as an or-inequality.
\item \( \text{None of the above.} \)


\end{enumerate}

\textbf{General Comment:} To solve, you will need to break up the compound inequality into two inequalities. Be sure to keep track of the inequality! It may be best to draw a number line and graph your solution.
}
\litem{
Solve the linear inequality below. Then, choose the constant and interval combination that describes the solution set.
\[ -7x -4 \leq 10x + 3 \]The solution is \( [-0.412, \infty) \), which is option D.\begin{enumerate}[label=\Alph*.]
\item \( (-\infty, a], \text{ where } a \in [0.29, 0.5] \)

 $(-\infty, 0.412]$, which corresponds to switching the direction of the interval AND negating the endpoint. You likely did this if you did not flip the inequality when dividing by a negative as well as not moving values over to a side properly.
\item \( (-\infty, a], \text{ where } a \in [-0.82, 0.24] \)

 $(-\infty, -0.412]$, which corresponds to switching the direction of the interval. You likely did this if you did not flip the inequality when dividing by a negative!
\item \( [a, \infty), \text{ where } a \in [-0.26, 0.47] \)

 $[0.412, \infty)$, which corresponds to negating the endpoint of the solution.
\item \( [a, \infty), \text{ where } a \in [-0.77, -0.2] \)

* $[-0.412, \infty)$, which is the correct option.
\item \( \text{None of the above}. \)

You may have chosen this if you thought the inequality did not match the ends of the intervals.
\end{enumerate}

\textbf{General Comment:} Remember that less/greater than or equal to includes the endpoint, while less/greater do not. Also, remember that you need to flip the inequality when you multiply or divide by a negative.
}
\end{enumerate}

\end{document}