\documentclass[14pt]{extbook}
\usepackage{multicol, enumerate, enumitem, hyperref, color, soul, setspace, parskip, fancyhdr} %General Packages
\usepackage{amssymb, amsthm, amsmath, latexsym, units, mathtools} %Math Packages
\everymath{\displaystyle} %All math in Display Style
% Packages with additional options
\usepackage[headsep=0.5cm,headheight=12pt, left=1 in,right= 1 in,top= 1 in,bottom= 1 in]{geometry}
\usepackage[usenames,dvipsnames]{xcolor}
\usepackage{dashrule}  % Package to use the command below to create lines between items
\newcommand{\litem}[1]{\item#1\hspace*{-1cm}\rule{\textwidth}{0.4pt}}
\pagestyle{fancy}
\lhead{Progress Quiz 9}
\chead{}
\rhead{Version A}
\lfoot{9541-5764}
\cfoot{}
\rfoot{Summer C 2021}
\begin{document}

\begin{enumerate}
\litem{
What are the \textit{possible Rational} roots of the polynomial below?\[ f(x) = 6x^{4} +2 x^{3} +4 x^{2} +5 x + 4 \]\begin{enumerate}[label=\Alph*.]
\item \( \pm 1,\pm 2,\pm 4 \)
\item \( \text{ All combinations of: }\frac{\pm 1,\pm 2,\pm 4}{\pm 1,\pm 2,\pm 3,\pm 6} \)
\item \( \pm 1,\pm 2,\pm 3,\pm 6 \)
\item \( \text{ All combinations of: }\frac{\pm 1,\pm 2,\pm 3,\pm 6}{\pm 1,\pm 2,\pm 4} \)
\item \( \text{ There is no formula or theorem that tells us all possible Rational roots.} \)

\end{enumerate} }
\litem{
Perform the division below. Then, find the intervals that correspond to the quotient in the form $ax^2+bx+c$ and remainder $r$.\[ \frac{12x^{3} -36 x -28}{x -2} \]\begin{enumerate}[label=\Alph*.]
\item \( a \in [11, 16], b \in [10, 17], c \in [-27, -21], \text{ and } r \in [-55, -47]. \)
\item \( a \in [24, 25], b \in [-49, -46], c \in [60, 64], \text{ and } r \in [-150, -142]. \)
\item \( a \in [11, 16], b \in [-26, -23], c \in [11, 17], \text{ and } r \in [-55, -47]. \)
\item \( a \in [24, 25], b \in [47, 50], c \in [60, 64], \text{ and } r \in [91, 96]. \)
\item \( a \in [11, 16], b \in [24, 26], c \in [11, 17], \text{ and } r \in [-5, -1]. \)

\end{enumerate} }
\litem{
Factor the polynomial below completely, knowing that $x -4$ is a factor. Then, choose the intervals the zeros of the polynomial belong to, where $z_1 \leq z_2 \leq z_3 \leq z_4$. \textit{To make the problem easier, all zeros are between -5 and 5.}\[ f(x) = 12x^{4} -61 x^{3} +15 x^{2} +178 x -120 \]\begin{enumerate}[label=\Alph*.]
\item \( z_1 \in [-0.96, -0.32], \text{   }  z_2 \in [1.3, 2.27], z_3 \in [1.63, 2.06], \text{   and   } z_4 \in [3.61, 4.69] \)
\item \( z_1 \in [-5.22, -3.24], \text{   }  z_2 \in [-2.02, -0.96], z_3 \in [-0.28, -0.21], \text{   and   } z_4 \in [4.73, 5.53] \)
\item \( z_1 \in [-2.16, -1.43], \text{   }  z_2 \in [0.41, 0.76], z_3 \in [1.63, 2.06], \text{   and   } z_4 \in [3.61, 4.69] \)
\item \( z_1 \in [-5.22, -3.24], \text{   }  z_2 \in [-2.02, -0.96], z_3 \in [-0.87, -0.32], \text{   and   } z_4 \in [1.55, 1.79] \)
\item \( z_1 \in [-5.22, -3.24], \text{   }  z_2 \in [-2.02, -0.96], z_3 \in [-1.42, -1.27], \text{   and   } z_4 \in [-0.02, 0.92] \)

\end{enumerate} }
\litem{
Factor the polynomial below completely. Then, choose the intervals the zeros of the polynomial belong to, where $z_1 \leq z_2 \leq z_3$. \textit{To make the problem easier, all zeros are between -5 and 5.}\[ f(x) = 10x^{3} -49 x^{2} +42 x + 45 \]\begin{enumerate}[label=\Alph*.]
\item \( z_1 \in [-1.8, -1.5], \text{   }  z_2 \in [0.33, 0.84], \text{   and   } z_3 \in [2.97, 3.01] \)
\item \( z_1 \in [-5.1, -4.2], \text{   }  z_2 \in [-3.32, -2.82], \text{   and   } z_3 \in [0.28, 0.38] \)
\item \( z_1 \in [-3.1, -2.9], \text{   }  z_2 \in [-1.21, -0.2], \text{   and   } z_3 \in [1.64, 1.76] \)
\item \( z_1 \in [-0.7, 0.4], \text{   }  z_2 \in [2.4, 2.99], \text{   and   } z_3 \in [2.97, 3.01] \)
\item \( z_1 \in [-3.1, -2.9], \text{   }  z_2 \in [-2.57, -2.28], \text{   and   } z_3 \in [0.47, 0.78] \)

\end{enumerate} }
\litem{
Factor the polynomial below completely. Then, choose the intervals the zeros of the polynomial belong to, where $z_1 \leq z_2 \leq z_3$. \textit{To make the problem easier, all zeros are between -5 and 5.}\[ f(x) = 15x^{3} +94 x^{2} +101 x + 30 \]\begin{enumerate}[label=\Alph*.]
\item \( z_1 \in [-5.29, -4.77], \text{   }  z_2 \in [-2, -1.62], \text{   and   } z_3 \in [-1.7, -1.3] \)
\item \( z_1 \in [-0.77, 0.55], \text{   }  z_2 \in [1.75, 2.15], \text{   and   } z_3 \in [4.3, 5.2] \)
\item \( z_1 \in [0.52, 0.97], \text{   }  z_2 \in [0.63, 0.76], \text{   and   } z_3 \in [4.3, 5.2] \)
\item \( z_1 \in [1.45, 1.92], \text{   }  z_2 \in [1.29, 1.9], \text{   and   } z_3 \in [4.3, 5.2] \)
\item \( z_1 \in [-5.29, -4.77], \text{   }  z_2 \in [-0.83, -0.09], \text{   and   } z_3 \in [-1.1, 1.1] \)

\end{enumerate} }
\litem{
Perform the division below. Then, find the intervals that correspond to the quotient in the form $ax^2+bx+c$ and remainder $r$.\[ \frac{6x^{3} -18 x^{2} -36 x + 43}{x -4} \]\begin{enumerate}[label=\Alph*.]
\item \( a \in [5, 7], \text{   } b \in [5, 11], \text{   } c \in [-12, -10], \text{   and   } r \in [-10, -3]. \)
\item \( a \in [24, 28], \text{   } b \in [-117, -109], \text{   } c \in [414, 423], \text{   and   } r \in [-1638, -1632]. \)
\item \( a \in [24, 28], \text{   } b \in [73, 79], \text{   } c \in [272, 279], \text{   and   } r \in [1145, 1153]. \)
\item \( a \in [5, 7], \text{   } b \in [-45, -36], \text{   } c \in [132, 137], \text{   and   } r \in [-490, -482]. \)
\item \( a \in [5, 7], \text{   } b \in [-4, 4], \text{   } c \in [-38, -31], \text{   and   } r \in [-68, -63]. \)

\end{enumerate} }
\litem{
Perform the division below. Then, find the intervals that correspond to the quotient in the form $ax^2+bx+c$ and remainder $r$.\[ \frac{20x^{3} -65 x^{2} -160 x -72}{x -5} \]\begin{enumerate}[label=\Alph*.]
\item \( a \in [15, 23], \text{   } b \in [32, 39], \text{   } c \in [13, 23], \text{   and   } r \in [2, 4]. \)
\item \( a \in [96, 103], \text{   } b \in [427, 441], \text{   } c \in [2012, 2022], \text{   and   } r \in [9997, 10006]. \)
\item \( a \in [15, 23], \text{   } b \in [14, 17], \text{   } c \in [-104, -98], \text{   and   } r \in [-479, -467]. \)
\item \( a \in [15, 23], \text{   } b \in [-172, -161], \text{   } c \in [663, 670], \text{   and   } r \in [-3400, -3394]. \)
\item \( a \in [96, 103], \text{   } b \in [-565, -560], \text{   } c \in [2665, 2671], \text{   and   } r \in [-13401, -13395]. \)

\end{enumerate} }
\litem{
Perform the division below. Then, find the intervals that correspond to the quotient in the form $ax^2+bx+c$ and remainder $r$.\[ \frac{12x^{3} +52 x^{2} -62}{x + 4} \]\begin{enumerate}[label=\Alph*.]
\item \( a \in [10, 14], b \in [3, 9], c \in [-21, -13], \text{ and } r \in [1, 8]. \)
\item \( a \in [-55, -46], b \in [-141, -136], c \in [-569, -557], \text{ and } r \in [-2302, -2299]. \)
\item \( a \in [-55, -46], b \in [244, 248], c \in [-976, -972], \text{ and } r \in [3838, 3846]. \)
\item \( a \in [10, 14], b \in [94, 102], c \in [394, 403], \text{ and } r \in [1537, 1539]. \)
\item \( a \in [10, 14], b \in [-13, -5], c \in [36, 41], \text{ and } r \in [-267, -260]. \)

\end{enumerate} }
\litem{
Factor the polynomial below completely, knowing that $x + 4$ is a factor. Then, choose the intervals the zeros of the polynomial belong to, where $z_1 \leq z_2 \leq z_3 \leq z_4$. \textit{To make the problem easier, all zeros are between -5 and 5.}\[ f(x) = 12x^{4} -23 x^{3} -244 x^{2} +235 x + 300 \]\begin{enumerate}[label=\Alph*.]
\item \( z_1 \in [-4.25, -3.89], \text{   }  z_2 \in [-0.85, -0.62], z_3 \in [1.41, 1.73], \text{   and   } z_4 \in [4.1, 5.5] \)
\item \( z_1 \in [-6.08, -4.94], \text{   }  z_2 \in [-0.65, -0.56], z_3 \in [1.32, 1.36], \text{   and   } z_4 \in [3.9, 4.7] \)
\item \( z_1 \in [-6.08, -4.94], \text{   }  z_2 \in [-0.49, -0.34], z_3 \in [2.96, 3.22], \text{   and   } z_4 \in [3.9, 4.7] \)
\item \( z_1 \in [-4.25, -3.89], \text{   }  z_2 \in [-1.37, -1.11], z_3 \in [0.45, 0.66], \text{   and   } z_4 \in [4.1, 5.5] \)
\item \( z_1 \in [-6.08, -4.94], \text{   }  z_2 \in [-1.84, -1.45], z_3 \in [0.7, 0.83], \text{   and   } z_4 \in [3.9, 4.7] \)

\end{enumerate} }
\litem{
What are the \textit{possible Integer} roots of the polynomial below?\[ f(x) = 4x^{4} +2 x^{3} +4 x^{2} +6 x + 6 \]\begin{enumerate}[label=\Alph*.]
\item \( \pm 1,\pm 2,\pm 3,\pm 6 \)
\item \( \pm 1,\pm 2,\pm 4 \)
\item \( \text{ All combinations of: }\frac{\pm 1,\pm 2,\pm 3,\pm 6}{\pm 1,\pm 2,\pm 4} \)
\item \( \text{ All combinations of: }\frac{\pm 1,\pm 2,\pm 4}{\pm 1,\pm 2,\pm 3,\pm 6} \)
\item \( \text{There is no formula or theorem that tells us all possible Integer roots.} \)

\end{enumerate} }
\end{enumerate}

\end{document}