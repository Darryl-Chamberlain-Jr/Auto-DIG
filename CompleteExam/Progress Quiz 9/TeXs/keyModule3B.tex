\documentclass{extbook}[14pt]
\usepackage{multicol, enumerate, enumitem, hyperref, color, soul, setspace, parskip, fancyhdr, amssymb, amsthm, amsmath, latexsym, units, mathtools}
\everymath{\displaystyle}
\usepackage[headsep=0.5cm,headheight=0cm, left=1 in,right= 1 in,top= 1 in,bottom= 1 in]{geometry}
\usepackage{dashrule}  % Package to use the command below to create lines between items
\newcommand{\litem}[1]{\item #1

\rule{\textwidth}{0.4pt}}
\pagestyle{fancy}
\lhead{}
\chead{Answer Key for Progress Quiz 9 Version B}
\rhead{}
\lfoot{9541-5764}
\cfoot{}
\rfoot{Summer C 2021}
\begin{document}
\textbf{This key should allow you to understand why you choose the option you did (beyond just getting a question right or wrong). \href{https://xronos.clas.ufl.edu/mac1105spring2020/courseDescriptionAndMisc/Exams/LearningFromResults}{More instructions on how to use this key can be found here}.}

\textbf{If you have a suggestion to make the keys better, \href{https://forms.gle/CZkbZmPbC9XALEE88}{please fill out the short survey here}.}

\textit{Note: This key is auto-generated and may contain issues and/or errors. The keys are reviewed after each exam to ensure grading is done accurately. If there are issues (like duplicate options), they are noted in the offline gradebook. The keys are a work-in-progress to give students as many resources to improve as possible.}

\rule{\textwidth}{0.4pt}

\begin{enumerate}\litem{
Solve the linear inequality below. Then, choose the constant and interval combination that describes the solution set.
\[ -7 + 6 x > 7 x \text{ or } -7 + 8 x < 10 x \]The solution is \( (-\infty, -7.0) \text{ or } (-3.5, \infty) \), which is option A.\begin{enumerate}[label=\Alph*.]
\item \( (-\infty, a) \cup (b, \infty), \text{ where } a \in [-9.75, -3.75] \text{ and } b \in [-4.5, -2.25] \)

 * Correct option.
\item \( (-\infty, a] \cup [b, \infty), \text{ where } a \in [-8.25, -5.25] \text{ and } b \in [-6, 1.5] \)

Corresponds to including the endpoints (when they should be excluded).
\item \( (-\infty, a] \cup [b, \infty), \text{ where } a \in [-0.75, 5.25] \text{ and } b \in [4.5, 7.5] \)

Corresponds to including the endpoints AND negating.
\item \( (-\infty, a) \cup (b, \infty), \text{ where } a \in [0.75, 6.75] \text{ and } b \in [3.75, 8.25] \)

Corresponds to inverting the inequality and negating the solution.
\item \( (-\infty, \infty) \)

Corresponds to the variable canceling, which does not happen in this instance.
\end{enumerate}

\textbf{General Comment:} When multiplying or dividing by a negative, flip the sign.
}
\litem{
Solve the linear inequality below. Then, choose the constant and interval combination that describes the solution set.
\[ \frac{3}{9} + \frac{3}{5} x \geq \frac{6}{6} x - \frac{8}{4} \]The solution is \( (-\infty, 5.833] \), which is option B.\begin{enumerate}[label=\Alph*.]
\item \( (-\infty, a], \text{ where } a \in [-9.75, -3.75] \)

 $(-\infty, -5.833]$, which corresponds to negating the endpoint of the solution.
\item \( (-\infty, a], \text{ where } a \in [5.25, 8.25] \)

* $(-\infty, 5.833]$, which is the correct option.
\item \( [a, \infty), \text{ where } a \in [4.5, 6.75] \)

 $[5.833, \infty)$, which corresponds to switching the direction of the interval. You likely did this if you did not flip the inequality when dividing by a negative!
\item \( [a, \infty), \text{ where } a \in [-7.5, -4.5] \)

 $[-5.833, \infty)$, which corresponds to switching the direction of the interval AND negating the endpoint. You likely did this if you did not flip the inequality when dividing by a negative as well as not moving values over to a side properly.
\item \( \text{None of the above}. \)

You may have chosen this if you thought the inequality did not match the ends of the intervals.
\end{enumerate}

\textbf{General Comment:} Remember that less/greater than or equal to includes the endpoint, while less/greater do not. Also, remember that you need to flip the inequality when you multiply or divide by a negative.
}
\litem{
Solve the linear inequality below. Then, choose the constant and interval combination that describes the solution set.
\[ \frac{3}{7} - \frac{5}{4} x < \frac{9}{8} x - \frac{10}{2} \]The solution is \( (2.286, \infty) \), which is option B.\begin{enumerate}[label=\Alph*.]
\item \( (a, \infty), \text{ where } a \in [-5.25, -1.5] \)

 $(-2.286, \infty)$, which corresponds to negating the endpoint of the solution.
\item \( (a, \infty), \text{ where } a \in [0, 5.25] \)

* $(2.286, \infty)$, which is the correct option.
\item \( (-\infty, a), \text{ where } a \in [-2.25, 5.25] \)

 $(-\infty, 2.286)$, which corresponds to switching the direction of the interval. You likely did this if you did not flip the inequality when dividing by a negative!
\item \( (-\infty, a), \text{ where } a \in [-3.75, 0.75] \)

 $(-\infty, -2.286)$, which corresponds to switching the direction of the interval AND negating the endpoint. You likely did this if you did not flip the inequality when dividing by a negative as well as not moving values over to a side properly.
\item \( \text{None of the above}. \)

You may have chosen this if you thought the inequality did not match the ends of the intervals.
\end{enumerate}

\textbf{General Comment:} Remember that less/greater than or equal to includes the endpoint, while less/greater do not. Also, remember that you need to flip the inequality when you multiply or divide by a negative.
}
\litem{
Using an interval or intervals, describe all the $x$-values within or including a distance of the given values.
\[ \text{ More than } 7 \text{ units from the number } 9. \]The solution is \( (-\infty, 2) \cup (16, \infty) \), which is option B.\begin{enumerate}[label=\Alph*.]
\item \( (2, 16) \)

This describes the values less than 7 from 9
\item \( (-\infty, 2) \cup (16, \infty) \)

This describes the values more than 7 from 9
\item \( [2, 16] \)

This describes the values no more than 7 from 9
\item \( (-\infty, 2] \cup [16, \infty) \)

This describes the values no less than 7 from 9
\item \( \text{None of the above} \)

You likely thought the values in the interval were not correct.
\end{enumerate}

\textbf{General Comment:} When thinking about this language, it helps to draw a number line and try points.
}
\litem{
Solve the linear inequality below. Then, choose the constant and interval combination that describes the solution set.
\[ -6 - 9 x \leq \frac{-36 x - 7}{8} < 7 - 5 x \]The solution is \( [-1.14, 15.75) \), which is option B.\begin{enumerate}[label=\Alph*.]
\item \( (a, b], \text{ where } a \in [-1.72, -0.9] \text{ and } b \in [13.5, 18] \)

$(-1.14, 15.75]$, which corresponds to flipping the inequality.
\item \( [a, b), \text{ where } a \in [-4.12, 0.53] \text{ and } b \in [13.5, 20.25] \)

$[-1.14, 15.75)$, which is the correct option.
\item \( (-\infty, a) \cup [b, \infty), \text{ where } a \in [-2.62, -0.22] \text{ and } b \in [14.25, 21] \)

$(-\infty, -1.14) \cup [15.75, \infty)$, which corresponds to displaying the and-inequality as an or-inequality AND flipping the inequality.
\item \( (-\infty, a] \cup (b, \infty), \text{ where } a \in [-3.38, 0] \text{ and } b \in [13.5, 19.5] \)

$(-\infty, -1.14] \cup (15.75, \infty)$, which corresponds to displaying the and-inequality as an or-inequality.
\item \( \text{None of the above.} \)


\end{enumerate}

\textbf{General Comment:} To solve, you will need to break up the compound inequality into two inequalities. Be sure to keep track of the inequality! It may be best to draw a number line and graph your solution.
}
\litem{
Solve the linear inequality below. Then, choose the constant and interval combination that describes the solution set.
\[ -6x + 7 > -4x -3 \]The solution is \( (-\infty, 5.0) \), which is option D.\begin{enumerate}[label=\Alph*.]
\item \( (-\infty, a), \text{ where } a \in [-7, 1] \)

 $(-\infty, -5.0)$, which corresponds to negating the endpoint of the solution.
\item \( (a, \infty), \text{ where } a \in [-6, -1] \)

 $(-5.0, \infty)$, which corresponds to switching the direction of the interval AND negating the endpoint. You likely did this if you did not flip the inequality when dividing by a negative as well as not moving values over to a side properly.
\item \( (a, \infty), \text{ where } a \in [4, 8] \)

 $(5.0, \infty)$, which corresponds to switching the direction of the interval. You likely did this if you did not flip the inequality when dividing by a negative!
\item \( (-\infty, a), \text{ where } a \in [1, 10] \)

* $(-\infty, 5.0)$, which is the correct option.
\item \( \text{None of the above}. \)

You may have chosen this if you thought the inequality did not match the ends of the intervals.
\end{enumerate}

\textbf{General Comment:} Remember that less/greater than or equal to includes the endpoint, while less/greater do not. Also, remember that you need to flip the inequality when you multiply or divide by a negative.
}
\litem{
Solve the linear inequality below. Then, choose the constant and interval combination that describes the solution set.
\[ -5 + 5 x > 7 x \text{ or } -4 + 7 x < 10 x \]The solution is \( (-\infty, -2.5) \text{ or } (-1.333, \infty) \), which is option B.\begin{enumerate}[label=\Alph*.]
\item \( (-\infty, a] \cup [b, \infty), \text{ where } a \in [-5.25, -1.5] \text{ and } b \in [-6.75, 0.75] \)

Corresponds to including the endpoints (when they should be excluded).
\item \( (-\infty, a) \cup (b, \infty), \text{ where } a \in [-3.97, -0.67] \text{ and } b \in [-3.75, -0.75] \)

 * Correct option.
\item \( (-\infty, a) \cup (b, \infty), \text{ where } a \in [1.05, 3] \text{ and } b \in [1.5, 3] \)

Corresponds to inverting the inequality and negating the solution.
\item \( (-\infty, a] \cup [b, \infty), \text{ where } a \in [-0.75, 9] \text{ and } b \in [0.75, 3.75] \)

Corresponds to including the endpoints AND negating.
\item \( (-\infty, \infty) \)

Corresponds to the variable canceling, which does not happen in this instance.
\end{enumerate}

\textbf{General Comment:} When multiplying or dividing by a negative, flip the sign.
}
\litem{
Using an interval or intervals, describe all the $x$-values within or including a distance of the given values.
\[ \text{ No less than } 8 \text{ units from the number } 4. \]The solution is \( (-\infty, -4] \cup [12, \infty) \), which is option A.\begin{enumerate}[label=\Alph*.]
\item \( (-\infty, -4] \cup [12, \infty) \)

This describes the values no less than 8 from 4
\item \( (-\infty, -4) \cup (12, \infty) \)

This describes the values more than 8 from 4
\item \( [-4, 12] \)

This describes the values no more than 8 from 4
\item \( (-4, 12) \)

This describes the values less than 8 from 4
\item \( \text{None of the above} \)

You likely thought the values in the interval were not correct.
\end{enumerate}

\textbf{General Comment:} When thinking about this language, it helps to draw a number line and try points.
}
\litem{
Solve the linear inequality below. Then, choose the constant and interval combination that describes the solution set.
\[ -8 - 7 x < \frac{-42 x + 8}{9} \leq 5 - 5 x \]The solution is \( (-3.81, 12.33] \), which is option C.\begin{enumerate}[label=\Alph*.]
\item \( [a, b), \text{ where } a \in [-6.75, 2.25] \text{ and } b \in [9, 13.5] \)

$[-3.81, 12.33)$, which corresponds to flipping the inequality.
\item \( (-\infty, a] \cup (b, \infty), \text{ where } a \in [-9.75, 0] \text{ and } b \in [12, 17.25] \)

$(-\infty, -3.81] \cup (12.33, \infty)$, which corresponds to displaying the and-inequality as an or-inequality AND flipping the inequality.
\item \( (a, b], \text{ where } a \in [-6, 0] \text{ and } b \in [10.5, 14.25] \)

* $(-3.81, 12.33]$, which is the correct option.
\item \( (-\infty, a) \cup [b, \infty), \text{ where } a \in [-5.25, -3] \text{ and } b \in [10.5, 15] \)

$(-\infty, -3.81) \cup [12.33, \infty)$, which corresponds to displaying the and-inequality as an or-inequality.
\item \( \text{None of the above.} \)


\end{enumerate}

\textbf{General Comment:} To solve, you will need to break up the compound inequality into two inequalities. Be sure to keep track of the inequality! It may be best to draw a number line and graph your solution.
}
\litem{
Solve the linear inequality below. Then, choose the constant and interval combination that describes the solution set.
\[ 5x + 10 < 10x + 7 \]The solution is \( (0.6, \infty) \), which is option D.\begin{enumerate}[label=\Alph*.]
\item \( (-\infty, a), \text{ where } a \in [-0.83, -0.34] \)

 $(-\infty, -0.6)$, which corresponds to switching the direction of the interval AND negating the endpoint. You likely did this if you did not flip the inequality when dividing by a negative as well as not moving values over to a side properly.
\item \( (a, \infty), \text{ where } a \in [-2.1, -0.1] \)

 $(-0.6, \infty)$, which corresponds to negating the endpoint of the solution.
\item \( (-\infty, a), \text{ where } a \in [-0.35, 2.13] \)

 $(-\infty, 0.6)$, which corresponds to switching the direction of the interval. You likely did this if you did not flip the inequality when dividing by a negative!
\item \( (a, \infty), \text{ where } a \in [-0.4, 4.9] \)

* $(0.6, \infty)$, which is the correct option.
\item \( \text{None of the above}. \)

You may have chosen this if you thought the inequality did not match the ends of the intervals.
\end{enumerate}

\textbf{General Comment:} Remember that less/greater than or equal to includes the endpoint, while less/greater do not. Also, remember that you need to flip the inequality when you multiply or divide by a negative.
}
\end{enumerate}

\end{document}