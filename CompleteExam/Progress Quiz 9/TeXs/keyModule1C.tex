\documentclass{extbook}[14pt]
\usepackage{multicol, enumerate, enumitem, hyperref, color, soul, setspace, parskip, fancyhdr, amssymb, amsthm, amsmath, latexsym, units, mathtools}
\everymath{\displaystyle}
\usepackage[headsep=0.5cm,headheight=0cm, left=1 in,right= 1 in,top= 1 in,bottom= 1 in]{geometry}
\usepackage{dashrule}  % Package to use the command below to create lines between items
\newcommand{\litem}[1]{\item #1

\rule{\textwidth}{0.4pt}}
\pagestyle{fancy}
\lhead{}
\chead{Answer Key for Progress Quiz 9 Version C}
\rhead{}
\lfoot{9541-5764}
\cfoot{}
\rfoot{Summer C 2021}
\begin{document}
\textbf{This key should allow you to understand why you choose the option you did (beyond just getting a question right or wrong). \href{https://xronos.clas.ufl.edu/mac1105spring2020/courseDescriptionAndMisc/Exams/LearningFromResults}{More instructions on how to use this key can be found here}.}

\textbf{If you have a suggestion to make the keys better, \href{https://forms.gle/CZkbZmPbC9XALEE88}{please fill out the short survey here}.}

\textit{Note: This key is auto-generated and may contain issues and/or errors. The keys are reviewed after each exam to ensure grading is done accurately. If there are issues (like duplicate options), they are noted in the offline gradebook. The keys are a work-in-progress to give students as many resources to improve as possible.}

\rule{\textwidth}{0.4pt}

\begin{enumerate}\litem{
Simplify the expression below and choose the interval the simplification is contained within.
\[ 15 - 20 \div 7 * 13 - (11 * 14) \]The solution is \( -176.143 \), which is option B.\begin{enumerate}[label=\Alph*.]
\item \( [-140.22, -134.22] \)

 -139.220, which corresponds to an Order of Operations error: not reading left-to-right for multiplication/division.
\item \( [-181.14, -174.14] \)

* -176.143, which is the correct option.
\item \( [168.78, 170.78] \)

 168.780, which corresponds to not distributing addition and subtraction correctly.
\item \( [-466, -460] \)

 -464.000, which corresponds to not distributing a negative correctly.
\item \( \text{None of the above} \)

 You may have gotten this by making an unanticipated error. If you got a value that is not any of the others, please let the coordinator know so they can help you figure out what happened.
\end{enumerate}

\textbf{General Comment:} While you may remember (or were taught) PEMDAS is done in order, it is actually done as P/E/MD/AS. When we are at MD or AS, we read left to right.
}
\litem{
Choose the \textbf{smallest} set of Real numbers that the number below belongs to.
\[ -\sqrt{\frac{9025}{361}} \]The solution is \( \text{Integer} \), which is option B.\begin{enumerate}[label=\Alph*.]
\item \( \text{Whole} \)

These are the counting numbers with 0 (0, 1, 2, 3, ...)
\item \( \text{Integer} \)

* This is the correct option!
\item \( \text{Irrational} \)

These cannot be written as a fraction of Integers.
\item \( \text{Not a Real number} \)

These are Nonreal Complex numbers \textbf{OR} things that are not numbers (e.g., dividing by 0).
\item \( \text{Rational} \)

These are numbers that can be written as fraction of Integers (e.g., -2/3)
\end{enumerate}

\textbf{General Comment:} First, you \textbf{NEED} to simplify the expression. This question simplifies to $-95$. 
 
 Be sure you look at the simplified fraction and not just the decimal expansion. Numbers such as 13, 17, and 19 provide \textbf{long but repeating/terminating decimal expansions!} 
 
 The only ways to *not* be a Real number are: dividing by 0 or taking the square root of a negative number. 
 
 Irrational numbers are more than just square root of 3: adding or subtracting values from square root of 3 is also irrational.
}
\litem{
Simplify the expression below into the form $a+bi$. Then, choose the intervals that $a$ and $b$ belong to.
\[ \frac{-54 + 88 i}{3 - 4 i} \]The solution is \( -20.56  + 1.92 i \), which is option E.\begin{enumerate}[label=\Alph*.]
\item \( a \in [-514.5, -513.5] \text{ and } b \in [1.5, 2.5] \)

 $-514.00  + 1.92 i$, which corresponds to forgetting to multiply the conjugate by the numerator and using a plus instead of a minus in the denominator.
\item \( a \in [7, 8.5] \text{ and } b \in [19, 20] \)

 $7.60  + 19.20 i$, which corresponds to forgetting to multiply the conjugate by the numerator and not computing the conjugate correctly.
\item \( a \in [-19, -17] \text{ and } b \in [-22.5, -21.5] \)

 $-18.00  - 22.00 i$, which corresponds to just dividing the first term by the first term and the second by the second.
\item \( a \in [-21.5, -19.5] \text{ and } b \in [47.5, 48.5] \)

 $-20.56  + 48.00 i$, which corresponds to forgetting to multiply the conjugate by the numerator.
\item \( a \in [-21.5, -19.5] \text{ and } b \in [1.5, 2.5] \)

* $-20.56  + 1.92 i$, which is the correct option.
\end{enumerate}

\textbf{General Comment:} Multiply the numerator and denominator by the *conjugate* of the denominator, then simplify. For example, if we have $2+3i$, the conjugate is $2-3i$.
}
\litem{
Simplify the expression below into the form $a+bi$. Then, choose the intervals that $a$ and $b$ belong to.
\[ (9 - 8 i)(-4 + 3 i) \]The solution is \( -12 + 59 i \), which is option C.\begin{enumerate}[label=\Alph*.]
\item \( a \in [-65, -55] \text{ and } b \in [-5, -2] \)

 $-60 - 5 i$, which corresponds to adding a minus sign in the first term.
\item \( a \in [-40, -27] \text{ and } b \in [-28, -19] \)

 $-36 - 24 i$, which corresponds to just multiplying the real terms to get the real part of the solution and the coefficients in the complex terms to get the complex part.
\item \( a \in [-15, -5] \text{ and } b \in [57, 61] \)

* $-12 + 59 i$, which is the correct option.
\item \( a \in [-65, -55] \text{ and } b \in [5, 6] \)

 $-60 + 5 i$, which corresponds to adding a minus sign in the second term.
\item \( a \in [-15, -5] \text{ and } b \in [-59, -56] \)

 $-12 - 59 i$, which corresponds to adding a minus sign in both terms.
\end{enumerate}

\textbf{General Comment:} You can treat $i$ as a variable and distribute. Just remember that $i^2=-1$, so you can continue to reduce after you distribute.
}
\litem{
Choose the \textbf{smallest} set of Complex numbers that the number below belongs to.
\[ \frac{-20}{2}+49i^2 \]The solution is \( \text{Rational} \), which is option A.\begin{enumerate}[label=\Alph*.]
\item \( \text{Rational} \)

* This is the correct option!
\item \( \text{Nonreal Complex} \)

This is a Complex number $(a+bi)$ that is not Real (has $i$ as part of the number).
\item \( \text{Not a Complex Number} \)

This is not a number. The only non-Complex number we know is dividing by 0 as this is not a number!
\item \( \text{Irrational} \)

These cannot be written as a fraction of Integers. Remember: $\pi$ is not an Integer!
\item \( \text{Pure Imaginary} \)

This is a Complex number $(a+bi)$ that \textbf{only} has an imaginary part like $2i$.
\end{enumerate}

\textbf{General Comment:} Be sure to simplify $i^2 = -1$. This may remove the imaginary portion for your number. If you are having trouble, you may want to look at the \textit{Subgroups of the Real Numbers} section.
}
\litem{
Simplify the expression below into the form $a+bi$. Then, choose the intervals that $a$ and $b$ belong to.
\[ (10 - 6 i)(2 + 4 i) \]The solution is \( 44 + 28 i \), which is option C.\begin{enumerate}[label=\Alph*.]
\item \( a \in [41, 47] \text{ and } b \in [-29, -25] \)

 $44 - 28 i$, which corresponds to adding a minus sign in both terms.
\item \( a \in [17, 23] \text{ and } b \in [-25, -18] \)

 $20 - 24 i$, which corresponds to just multiplying the real terms to get the real part of the solution and the coefficients in the complex terms to get the complex part.
\item \( a \in [41, 47] \text{ and } b \in [24, 29] \)

* $44 + 28 i$, which is the correct option.
\item \( a \in [-6, -1] \text{ and } b \in [50, 53] \)

 $-4 + 52 i$, which corresponds to adding a minus sign in the first term.
\item \( a \in [-6, -1] \text{ and } b \in [-53, -51] \)

 $-4 - 52 i$, which corresponds to adding a minus sign in the second term.
\end{enumerate}

\textbf{General Comment:} You can treat $i$ as a variable and distribute. Just remember that $i^2=-1$, so you can continue to reduce after you distribute.
}
\litem{
Simplify the expression below and choose the interval the simplification is contained within.
\[ 6 - 12^2 + 2 \div 16 * 17 \div 13 \]The solution is \( -137.837 \), which is option A.\begin{enumerate}[label=\Alph*.]
\item \( [-137.91, -137.54] \)

* -137.837, this is the correct option
\item \( [149.66, 150.07] \)

 150.001, which corresponds to two Order of Operations errors.
\item \( [150.15, 150.24] \)

 150.163, which corresponds to an Order of Operations error: multiplying by negative before squaring. For example: $(-3)^2 \neq -3^2$
\item \( [-138.24, -137.98] \)

 -137.999, which corresponds to an Order of Operations error: not reading left-to-right for multiplication/division.
\item \( \text{None of the above} \)

 You may have gotten this by making an unanticipated error. If you got a value that is not any of the others, please let the coordinator know so they can help you figure out what happened.
\end{enumerate}

\textbf{General Comment:} While you may remember (or were taught) PEMDAS is done in order, it is actually done as P/E/MD/AS. When we are at MD or AS, we read left to right.
}
\litem{
Choose the \textbf{smallest} set of Complex numbers that the number below belongs to.
\[ -\sqrt{\frac{324}{625}} + 25i^2 \]The solution is \( \text{Rational} \), which is option E.\begin{enumerate}[label=\Alph*.]
\item \( \text{Pure Imaginary} \)

This is a Complex number $(a+bi)$ that \textbf{only} has an imaginary part like $2i$.
\item \( \text{Irrational} \)

These cannot be written as a fraction of Integers. Remember: $\pi$ is not an Integer!
\item \( \text{Not a Complex Number} \)

This is not a number. The only non-Complex number we know is dividing by 0 as this is not a number!
\item \( \text{Nonreal Complex} \)

This is a Complex number $(a+bi)$ that is not Real (has $i$ as part of the number).
\item \( \text{Rational} \)

* This is the correct option!
\end{enumerate}

\textbf{General Comment:} Be sure to simplify $i^2 = -1$. This may remove the imaginary portion for your number. If you are having trouble, you may want to look at the \textit{Subgroups of the Real Numbers} section.
}
\litem{
Simplify the expression below into the form $a+bi$. Then, choose the intervals that $a$ and $b$ belong to.
\[ \frac{27 - 11 i}{4 + 8 i} \]The solution is \( 0.25  - 3.25 i \), which is option E.\begin{enumerate}[label=\Alph*.]
\item \( a \in [6.5, 8] \text{ and } b \in [-2, 0.5] \)

 $6.75  - 1.38 i$, which corresponds to just dividing the first term by the first term and the second by the second.
\item \( a \in [-0.5, 0.5] \text{ and } b \in [-261.5, -258.5] \)

 $0.25  - 260.00 i$, which corresponds to forgetting to multiply the conjugate by the numerator.
\item \( a \in [18.5, 20.5] \text{ and } b \in [-4, -2.5] \)

 $20.00  - 3.25 i$, which corresponds to forgetting to multiply the conjugate by the numerator and using a plus instead of a minus in the denominator.
\item \( a \in [1.5, 3.5] \text{ and } b \in [1, 4.5] \)

 $2.45  + 2.15 i$, which corresponds to forgetting to multiply the conjugate by the numerator and not computing the conjugate correctly.
\item \( a \in [-0.5, 0.5] \text{ and } b \in [-4, -2.5] \)

* $0.25  - 3.25 i$, which is the correct option.
\end{enumerate}

\textbf{General Comment:} Multiply the numerator and denominator by the *conjugate* of the denominator, then simplify. For example, if we have $2+3i$, the conjugate is $2-3i$.
}
\litem{
Choose the \textbf{smallest} set of Real numbers that the number below belongs to.
\[ \sqrt{\frac{36}{529}} \]The solution is \( \text{Rational} \), which is option B.\begin{enumerate}[label=\Alph*.]
\item \( \text{Whole} \)

These are the counting numbers with 0 (0, 1, 2, 3, ...)
\item \( \text{Rational} \)

* This is the correct option!
\item \( \text{Irrational} \)

These cannot be written as a fraction of Integers.
\item \( \text{Not a Real number} \)

These are Nonreal Complex numbers \textbf{OR} things that are not numbers (e.g., dividing by 0).
\item \( \text{Integer} \)

These are the negative and positive counting numbers (..., -3, -2, -1, 0, 1, 2, 3, ...)
\end{enumerate}

\textbf{General Comment:} First, you \textbf{NEED} to simplify the expression. This question simplifies to $\frac{6}{23}$. 
 
 Be sure you look at the simplified fraction and not just the decimal expansion. Numbers such as 13, 17, and 19 provide \textbf{long but repeating/terminating decimal expansions!} 
 
 The only ways to *not* be a Real number are: dividing by 0 or taking the square root of a negative number. 
 
 Irrational numbers are more than just square root of 3: adding or subtracting values from square root of 3 is also irrational.
}
\end{enumerate}

\end{document}