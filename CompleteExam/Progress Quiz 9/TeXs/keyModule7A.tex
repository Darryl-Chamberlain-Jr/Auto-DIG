\documentclass{extbook}[14pt]
\usepackage{multicol, enumerate, enumitem, hyperref, color, soul, setspace, parskip, fancyhdr, amssymb, amsthm, amsmath, latexsym, units, mathtools}
\everymath{\displaystyle}
\usepackage[headsep=0.5cm,headheight=0cm, left=1 in,right= 1 in,top= 1 in,bottom= 1 in]{geometry}
\usepackage{dashrule}  % Package to use the command below to create lines between items
\newcommand{\litem}[1]{\item #1

\rule{\textwidth}{0.4pt}}
\pagestyle{fancy}
\lhead{}
\chead{Answer Key for Progress Quiz 9 Version A}
\rhead{}
\lfoot{9541-5764}
\cfoot{}
\rfoot{Summer C 2021}
\begin{document}
\textbf{This key should allow you to understand why you choose the option you did (beyond just getting a question right or wrong). \href{https://xronos.clas.ufl.edu/mac1105spring2020/courseDescriptionAndMisc/Exams/LearningFromResults}{More instructions on how to use this key can be found here}.}

\textbf{If you have a suggestion to make the keys better, \href{https://forms.gle/CZkbZmPbC9XALEE88}{please fill out the short survey here}.}

\textit{Note: This key is auto-generated and may contain issues and/or errors. The keys are reviewed after each exam to ensure grading is done accurately. If there are issues (like duplicate options), they are noted in the offline gradebook. The keys are a work-in-progress to give students as many resources to improve as possible.}

\rule{\textwidth}{0.4pt}

\begin{enumerate}\litem{
Solve the rational equation below. Then, choose the interval(s) that the solution(s) belongs to.
\[ \frac{-30}{35x -10} + 1 = \frac{-30}{35x -10} \]The solution is \( \text{all solutions are invalid or lead to complex values in the equation.} \), which is option B.\begin{enumerate}[label=\Alph*.]
\item \( x \in [0.29,2.29] \)

$x = 0.286$, which corresponds to not checking if this value leads to dividing by 0 in the original equation and thus is not a valid solution.
\item \( \text{All solutions lead to invalid or complex values in the equation.} \)

*$x = 0.286$ leads to dividing by 0 in the original equation and thus is not a valid solution, which is the correct option.
\item \( x \in [-0.4,0.2] \)

$x = -0.286$, which corresponds to not distributing the factor $35x -10$ correctly when trying to eliminate the fraction.
\item \( x_1 \in [-0.4, 0.2] \text{ and } x_2 \in [-0.71,2.29] \)

$x = -0.286 \text{ and } x = 0.286$, which corresponds to getting the correct solution and believing there should be a second solution to the equation.
\item \( x_1 \in [0.2, 0.8] \text{ and } x_2 \in [-0.71,2.29] \)

$x = 0.286 \text{ and } x = 0.286$, which corresponds to getting the correct solution and believing there should be a second solution to the equation.
\end{enumerate}

\textbf{General Comment:} Distractors are different based on the number of solutions. Remember that after solving, we need to make sure our solution does not make the original equation divide by zero!
}
\litem{
Choose the graph of the equation below.
\[ f(x) = \frac{1}{(x - 2)^2} - 2 \]The solution is the graph below, which is option B.
    \begin{center}
        \includegraphics[width=0.3\textwidth]{../Figures/rationalEquationToGraphCopyBA.png}
    \end{center}\begin{enumerate}[label=\Alph*.]
\begin{multicols}{2}
\item \includegraphics[width = 0.3\textwidth]{../Figures/rationalEquationToGraphCopyAA.png}
\item \includegraphics[width = 0.3\textwidth]{../Figures/rationalEquationToGraphCopyBA.png}
\item \includegraphics[width = 0.3\textwidth]{../Figures/rationalEquationToGraphCopyCA.png}
\item \includegraphics[width = 0.3\textwidth]{../Figures/rationalEquationToGraphCopyDA.png}
\end{multicols}\item None of the above.\end{enumerate}
\textbf{General Comment:} Remember that the general form of a basic rational equation is $ f(x) = \frac{a}{(x-h)^n} + k$, where $a$ is the leading coefficient (and in this case, we assume is either $1$ or $-1$), $n$ is the degree (in this case, either $1$ or $2$), and $(h, k)$ is the intersection of the asymptotes.
}
\litem{
Solve the rational equation below. Then, choose the interval(s) that the solution(s) belongs to.
\[ \frac{2x}{7x + 2} + \frac{-5x^{2}}{-35x^{2} -31 x -6} = \frac{4}{-5x -3} \]The solution is \( \text{There are two solutions: } x = -2.000 \text{ and } x = -0.267 \), which is option C.\begin{enumerate}[label=\Alph*.]
\item \( x \in [-0.53,0.55] \)


\item \( \text{All solutions lead to invalid or complex values in the equation.} \)


\item \( x_1 \in [-2.69, -1.04] \text{ and } x_2 \in [-0.27,-0.24] \)

* $x = -2.000 \text{ and } x = -0.267$, which is the correct option.
\item \( x \in [-0.88,-0.31] \)


\item \( x_1 \in [-2.69, -1.04] \text{ and } x_2 \in [-0.3,-0.28] \)


\end{enumerate}

\textbf{General Comment:} Distractors are different based on the number of solutions. Remember that after solving, we need to make sure our solution does not make the original equation divide by zero!
}
\litem{
Determine the domain of the function below.
\[ f(x) = \frac{6}{36x^{2} +54 x + 20} \]The solution is \( \text{All Real numbers except } x = -0.833 \text{ and } x = -0.667. \), which is option C.\begin{enumerate}[label=\Alph*.]
\item \( \text{All Real numbers except } x = a, \text{ where } a \in [-30.08, -29.84] \)

All Real numbers except $x = -30.000$, which corresponds to removing a distractor value from the denominator.
\item \( \text{All Real numbers.} \)

This corresponds to thinking the denominator has complex roots or that rational functions have a domain of all Real numbers.
\item \( \text{All Real numbers except } x = a \text{ and } x = b, \text{ where } a \in [-0.86, -0.73] \text{ and } b \in [-0.73, -0.5] \)

All Real numbers except $x = -0.833$ and $x = -0.667$, which is the correct option.
\item \( \text{All Real numbers except } x = a, \text{ where } a \in [-0.86, -0.73] \)

All Real numbers except $x = -0.833$, which corresponds to removing only 1 value from the denominator.
\item \( \text{All Real numbers except } x = a \text{ and } x = b, \text{ where } a \in [-30.08, -29.84] \text{ and } b \in [-24.13, -23.8] \)

All Real numbers except $x = -30.000$ and $x = -24.000$, which corresponds to not factoring the denominator correctly.
\end{enumerate}

\textbf{General Comment:} Recall that dividing by zero is not a real number. Therefore the domain is all real numbers \textbf{except} those that make the denominator 0.
}
\litem{
Solve the rational equation below. Then, choose the interval(s) that the solution(s) belongs to.
\[ \frac{-9}{4x -4} + -6 = \frac{-6}{-20x + 20} \]The solution is \( x = 0.575 \), which is option E.\begin{enumerate}[label=\Alph*.]
\item \( x_1 \in [-3.42, 0.57] \text{ and } x_2 \in [0.49,0.7] \)

$x = -1.425 \text{ and } x = 0.575$, which corresponds to getting the correct solution and believing there should be a second solution to the equation.
\item \( x \in [-3.42,0.57] \)

$x = -1.425$, which corresponds to not distributing the factor $4x -4$ correctly when trying to eliminate the fraction.
\item \( x_1 \in [0.57, 3.58] \text{ and } x_2 \in [0.74,1.39] \)

$x = 0.575 \text{ and } x = 0.875$, which corresponds to getting the correct solution and believing there should be a second solution to the equation.
\item \( \text{All solutions lead to invalid or complex values in the equation.} \)

This corresponds to thinking $x = 0.575$ leads to dividing by zero in the original equation, which it does not.
\item \( x \in [0.57,1.57] \)

* $x = 0.575$, which is the correct option.
\end{enumerate}

\textbf{General Comment:} Distractors are different based on the number of solutions. Remember that after solving, we need to make sure our solution does not make the original equation divide by zero!
}
\litem{
Choose the equation of the function graphed below.

\begin{center}
    \includegraphics[width=0.5\textwidth]{../Figures/rationalGraphToEquationA.png}
\end{center}


The solution is \( f(x) = \frac{-1}{(x - 3)^2} - 1 \), which is option A.\begin{enumerate}[label=\Alph*.]
\item \( f(x) = \frac{-1}{(x - 3)^2} - 1 \)

This is the correct option.
\item \( f(x) = \frac{1}{(x + 3)^2} - 1 \)

Corresponds to using the general form $f(x) = \frac{a}{(x+h)^2}+k$ and the opposite leading coefficient.
\item \( f(x) = \frac{1}{x + 3} - 1 \)

Corresponds to thinking the graph was a shifted version of $\frac{1}{x}$, using the general form $f(x) = \frac{a}{(x+h)^2}+k$, and the opposite leading coefficient.
\item \( f(x) = \frac{-1}{x - 3} - 1 \)

Corresponds to thinking the graph was a shifted version of $\frac{1}{x}$.
\item \( \text{None of the above} \)

This corresponds to believing the vertex of the graph was not correct.
\end{enumerate}

\textbf{General Comment:} Remember that the general form of a basic rational equation is $ f(x) = \frac{a}{(x-h)^n} + k$, where $a$ is the leading coefficient (and in this case, we assume is either $1$ or $-1$), $n$ is the degree (in this case, either $1$ or $2$), and $(h, k)$ is the intersection of the asymptotes.
}
\litem{
Solve the rational equation below. Then, choose the interval(s) that the solution(s) belongs to.
\[ \frac{-3x}{-6x -7} + \frac{-6x^{2}}{-30x^{2} -47 x -14} = \frac{-3}{5x + 2} \]The solution is \( \text{All solutions are invalid or lead to complex values in the equation.} \), which is option D.\begin{enumerate}[label=\Alph*.]
\item \( x_1 \in [-1.64, -0.93] \text{ and } x_2 \in [-0.7,1.1] \)

$x = -1.167 \text{ and } x = -0.400$, which corresponds to solving $-6x -7 = 0$ and $5x + 2 = 0$ and treating them as solutions to the equation.
\item \( x \in [-1.64,-0.93] \)

$x = -1.167$, which corresponds to solving $-6x -7 = 0$ and treating it as a solution to the equation.
\item \( x \in [-0.54,-0.38] \)

$x = -0.400$, which corresponds to solving $5x + 2 = 0$ and treating it as a solution to the equation.
\item \( \text{All solutions lead to invalid or complex values in the equation.} \)

* The equation leads to solving $-9x^{2} -24 x -21=0$, which leads to complex solutions. This is the correct option.
\item \( x_1 \in [-0.6, -0.44] \text{ and } x_2 \in [-3.7,-1.3] \)

$x = -0.588 \text{ and } x = -2.079$, which corresponds to making the discriminant from the Quadratic Formula positive to avoid complex solutions.
\end{enumerate}

\textbf{General Comment:} Distractors are different based on the number of solutions. Remember that after solving, we need to make sure our solution does not make the original equation divide by zero!
}
\litem{
Choose the graph of the equation below.
\[ f(x) = \frac{-1}{(x - 3)^2} - 2 \]The solution is the graph below, which is option E.
    \begin{center}
        \includegraphics[width=0.3\textwidth]{../Figures/rationalEquationToGraphEA.png}
    \end{center}\begin{enumerate}[label=\Alph*.]
\begin{multicols}{2}
\item \includegraphics[width = 0.3\textwidth]{../Figures/rationalEquationToGraphAA.png}
\item \includegraphics[width = 0.3\textwidth]{../Figures/rationalEquationToGraphBA.png}
\item \includegraphics[width = 0.3\textwidth]{../Figures/rationalEquationToGraphCA.png}
\item \includegraphics[width = 0.3\textwidth]{../Figures/rationalEquationToGraphDA.png}
\end{multicols}\item None of the above.\end{enumerate}
\textbf{General Comment:} Remember that the general form of a basic rational equation is $ f(x) = \frac{a}{(x-h)^n} + k$, where $a$ is the leading coefficient (and in this case, we assume is either $1$ or $-1$), $n$ is the degree (in this case, either $1$ or $2$), and $(h, k)$ is the intersection of the asymptotes.
}
\litem{
Choose the equation of the function graphed below.

\begin{center}
    \includegraphics[width=0.5\textwidth]{../Figures/rationalGraphToEquationCopyA.png}
\end{center}


The solution is \( f(x) = \frac{-1}{(x + 2)^2} + 3 \), which is option A.\begin{enumerate}[label=\Alph*.]
\item \( f(x) = \frac{-1}{(x + 2)^2} + 3 \)

This is the correct option.
\item \( f(x) = \frac{1}{x - 2} + 3 \)

Corresponds to thinking the graph was a shifted version of $\frac{1}{x}$, using the general form $f(x) = \frac{a}{(x+h)^2}+k$, and the opposite leading coefficient.
\item \( f(x) = \frac{1}{(x - 2)^2} + 3 \)

Corresponds to using the general form $f(x) = \frac{a}{(x+h)^2}+k$ and the opposite leading coefficient.
\item \( f(x) = \frac{-1}{x + 2} + 3 \)

Corresponds to thinking the graph was a shifted version of $\frac{1}{x}$.
\item \( \text{None of the above} \)

This corresponds to believing the vertex of the graph was not correct.
\end{enumerate}

\textbf{General Comment:} Remember that the general form of a basic rational equation is $ f(x) = \frac{a}{(x-h)^n} + k$, where $a$ is the leading coefficient (and in this case, we assume is either $1$ or $-1$), $n$ is the degree (in this case, either $1$ or $2$), and $(h, k)$ is the intersection of the asymptotes.
}
\litem{
Determine the domain of the function below.
\[ f(x) = \frac{5}{15x^{2} +42 x + 24} \]The solution is \( \text{All Real numbers except } x = -2.000 \text{ and } x = -0.800. \), which is option E.\begin{enumerate}[label=\Alph*.]
\item \( \text{All Real numbers except } x = a \text{ and } x = b, \text{ where } a \in [-20.36, -19.15] \text{ and } b \in [-18.44, -17.43] \)

All Real numbers except $x = -20.000$ and $x = -18.000$, which corresponds to not factoring the denominator correctly.
\item \( \text{All Real numbers except } x = a, \text{ where } a \in [-2.3, -1.32] \)

All Real numbers except $x = -2.000$, which corresponds to removing only 1 value from the denominator.
\item \( \text{All Real numbers.} \)

This corresponds to thinking the denominator has complex roots or that rational functions have a domain of all Real numbers.
\item \( \text{All Real numbers except } x = a, \text{ where } a \in [-20.36, -19.15] \)

All Real numbers except $x = -20.000$, which corresponds to removing a distractor value from the denominator.
\item \( \text{All Real numbers except } x = a \text{ and } x = b, \text{ where } a \in [-2.3, -1.32] \text{ and } b \in [-1, -0.34] \)

All Real numbers except $x = -2.000$ and $x = -0.800$, which is the correct option.
\end{enumerate}

\textbf{General Comment:} Recall that dividing by zero is not a real number. Therefore the domain is all real numbers \textbf{except} those that make the denominator 0.
}
\end{enumerate}

\end{document}