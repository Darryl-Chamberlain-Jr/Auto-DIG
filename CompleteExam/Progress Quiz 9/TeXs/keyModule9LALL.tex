\documentclass{extbook}[14pt]
\usepackage{multicol, enumerate, enumitem, hyperref, color, soul, setspace, parskip, fancyhdr, amssymb, amsthm, amsmath, latexsym, units, mathtools}
\everymath{\displaystyle}
\usepackage[headsep=0.5cm,headheight=0cm, left=1 in,right= 1 in,top= 1 in,bottom= 1 in]{geometry}
\usepackage{dashrule}  % Package to use the command below to create lines between items
\newcommand{\litem}[1]{\item #1

\rule{\textwidth}{0.4pt}}
\pagestyle{fancy}
\lhead{}
\chead{Answer Key for Progress Quiz 9 Version ALL}
\rhead{}
\lfoot{9541-5764}
\cfoot{}
\rfoot{Summer C 2021}
\begin{document}
\textbf{This key should allow you to understand why you choose the option you did (beyond just getting a question right or wrong). \href{https://xronos.clas.ufl.edu/mac1105spring2020/courseDescriptionAndMisc/Exams/LearningFromResults}{More instructions on how to use this key can be found here}.}

\textbf{If you have a suggestion to make the keys better, \href{https://forms.gle/CZkbZmPbC9XALEE88}{please fill out the short survey here}.}

\textit{Note: This key is auto-generated and may contain issues and/or errors. The keys are reviewed after each exam to ensure grading is done accurately. If there are issues (like duplicate options), they are noted in the offline gradebook. The keys are a work-in-progress to give students as many resources to improve as possible.}

\rule{\textwidth}{0.4pt}

\begin{enumerate}\litem{
Find the inverse of the function below. Then, evaluate the inverse at $x = 7$ and choose the interval that $f^-1(7)$ belongs to.
\[ f(x) = \ln{(x-5)}+3 \]The solution is \( f^{-1}(7) = 59.598 \), which is option C.\begin{enumerate}[label=\Alph*.]
\item \( f^{-1}(7) \in [162751.79, 162762.79] \)

 This solution corresponds to distractor 2.
\item \( f^{-1}(7) \in [45.6, 50.6] \)

 This solution corresponds to distractor 3.
\item \( f^{-1}(7) \in [55.6, 61.6] \)

 This is the solution.
\item \( f^{-1}(7) \in [9.39, 11.39] \)

 This solution corresponds to distractor 4.
\item \( f^{-1}(7) \in [22030.47, 22034.47] \)

 This solution corresponds to distractor 1.
\end{enumerate}

\textbf{General Comment:} Natural log and exponential functions always have an inverse. Once you switch the $x$ and $y$, use the conversion $ e^y = x \leftrightarrow y=\ln(x)$.
}
\litem{
Multiply the following functions, then choose the domain of the resulting function from the list below.
\[ f(x) = \frac{4}{4x-19} \text{ and } g(x) = \frac{2}{6x-29} \]The solution is \( \text{ The domain is all Real numbers except } x = 4.75 \text{ and } x = 4.83 \), which is option D.\begin{enumerate}[label=\Alph*.]
\item \( \text{ The domain is all Real numbers greater than or equal to } x = a, \text{ where } a \in [-8.67, -3.67] \)


\item \( \text{ The domain is all Real numbers less than or equal to } x = a, \text{ where } a \in [2, 8] \)


\item \( \text{ The domain is all Real numbers except } x = a, \text{ where } a \in [-9.2, -2.2] \)


\item \( \text{ The domain is all Real numbers except } x = a \text{ and } x = b, \text{ where } a \in [-0.25, 8.75] \text{ and } b \in [2.83, 6.83] \)


\item \( \text{ The domain is all Real numbers. } \)


\end{enumerate}

\textbf{General Comment:} The new domain is the intersection of the previous domains.
}
\litem{
Find the inverse of the function below (if it exists). Then, evaluate the inverse at $x = -11$ and choose the interval that $f^-1(-11)$ belongs to.
\[ f(x) = \sqrt[3]{2 x - 3} \]The solution is \( -664.0 \), which is option B.\begin{enumerate}[label=\Alph*.]
\item \( f^{-1}(-11) \in [661.2, 664.3] \)

 This solution corresponds to distractor 2.
\item \( f^{-1}(-11) \in [-666.5, -663.8] \)

* This is the correct solution.
\item \( f^{-1}(-11) \in [-668.5, -665] \)

 Distractor 1: This corresponds to 
\item \( f^{-1}(-11) \in [666.6, 669.2] \)

 This solution corresponds to distractor 3.
\item \( \text{ The function is not invertible for all Real numbers. } \)

 This solution corresponds to distractor 4.
\end{enumerate}

\textbf{General Comment:} Be sure you check that the function is 1-1 before trying to find the inverse!
}
\litem{
Subtract the following functions, then choose the domain of the resulting function from the list below.
\[ f(x) = \frac{3}{4x-17} \text{ and } g(x) = \frac{5}{5x+34} \]The solution is \( \text{ The domain is all Real numbers except } x = 4.25 \text{ and } x = -6.8 \), which is option D.\begin{enumerate}[label=\Alph*.]
\item \( \text{ The domain is all Real numbers greater than or equal to } x = a, \text{ where } a \in [0.67, 10.67] \)


\item \( \text{ The domain is all Real numbers less than or equal to } x = a, \text{ where } a \in [0.5, 7.5] \)


\item \( \text{ The domain is all Real numbers except } x = a, \text{ where } a \in [-6.6, -1.6] \)


\item \( \text{ The domain is all Real numbers except } x = a \text{ and } x = b, \text{ where } a \in [2.25, 11.25] \text{ and } b \in [-9.8, -4.8] \)


\item \( \text{ The domain is all Real numbers. } \)


\end{enumerate}

\textbf{General Comment:} The new domain is the intersection of the previous domains.
}
\litem{
Choose the interval below that $f$ composed with $g$ at $x=-1$ is in.
\[ f(x) = 4x^{3} -2 x^{2} -3 x + 1 \text{ and } g(x) = -3x^{3} -4 x^{2} +4 x + 4 \]The solution is \( -2.0 \), which is option D.\begin{enumerate}[label=\Alph*.]
\item \( (f \circ g)(-1) \in [2.9, 5.2] \)

 Distractor 1: Corresponds to reversing the composition.
\item \( (f \circ g)(-1) \in [-13.3, -11.7] \)

 Distractor 2: Corresponds to being slightly off from the solution.
\item \( (f \circ g)(-1) \in [-9.3, -5.2] \)

 Distractor 3: Corresponds to being slightly off from the solution.
\item \( (f \circ g)(-1) \in [-3.7, 1.5] \)

* This is the correct solution
\item \( \text{It is not possible to compose the two functions.} \)


\end{enumerate}

\textbf{General Comment:} $f$ composed with $g$ at $x$ means $f(g(x))$. The order matters!
}
\litem{
Choose the interval below that $f$ composed with $g$ at $x=1$ is in.
\[ f(x) = 4x^{3} -3 x^{2} -4 x \text{ and } g(x) = x^{3} -2 x^{2} -x \]The solution is \( -36.0 \), which is option A.\begin{enumerate}[label=\Alph*.]
\item \( (f \circ g)(1) \in [-37, -33.6] \)

* This is the correct solution
\item \( (f \circ g)(1) \in [-43.1, -38.7] \)

 Distractor 1: Corresponds to reversing the composition.
\item \( (f \circ g)(1) \in [-35.4, -32.2] \)

 Distractor 3: Corresponds to being slightly off from the solution.
\item \( (f \circ g)(1) \in [-46.5, -43] \)

 Distractor 2: Corresponds to being slightly off from the solution.
\item \( \text{It is not possible to compose the two functions.} \)


\end{enumerate}

\textbf{General Comment:} $f$ composed with $g$ at $x$ means $f(g(x))$. The order matters!
}
\litem{
Determine whether the function below is 1-1.
\[ f(x) = -18 x^2 + 30 x + 408 \]The solution is \( \text{no} \), which is option A.\begin{enumerate}[label=\Alph*.]
\item \( \text{No, because there is a $y$-value that goes to 2 different $x$-values.} \)

* This is the solution.
\item \( \text{No, because there is an $x$-value that goes to 2 different $y$-values.} \)

Corresponds to the Vertical Line test, which checks if an expression is a function.
\item \( \text{No, because the domain of the function is not $(-\infty, \infty)$.} \)

Corresponds to believing 1-1 means the domain is all Real numbers.
\item \( \text{No, because the range of the function is not $(-\infty, \infty)$.} \)

Corresponds to believing 1-1 means the range is all Real numbers.
\item \( \text{Yes, the function is 1-1.} \)

Corresponds to believing the function passes the Horizontal Line test.
\end{enumerate}

\textbf{General Comment:} There are only two valid options: The function is 1-1 OR No because there is a $y$-value that goes to 2 different $x$-values.
}
\litem{
Determine whether the function below is 1-1.
\[ f(x) = -18 x^2 - 27 x + 551 \]The solution is \( \text{no} \), which is option C.\begin{enumerate}[label=\Alph*.]
\item \( \text{Yes, the function is 1-1.} \)

Corresponds to believing the function passes the Horizontal Line test.
\item \( \text{No, because there is an $x$-value that goes to 2 different $y$-values.} \)

Corresponds to the Vertical Line test, which checks if an expression is a function.
\item \( \text{No, because there is a $y$-value that goes to 2 different $x$-values.} \)

* This is the solution.
\item \( \text{No, because the range of the function is not $(-\infty, \infty)$.} \)

Corresponds to believing 1-1 means the range is all Real numbers.
\item \( \text{No, because the domain of the function is not $(-\infty, \infty)$.} \)

Corresponds to believing 1-1 means the domain is all Real numbers.
\end{enumerate}

\textbf{General Comment:} There are only two valid options: The function is 1-1 OR No because there is a $y$-value that goes to 2 different $x$-values.
}
\litem{
Find the inverse of the function below. Then, evaluate the inverse at $x = 9$ and choose the interval that $f^-1(9)$ belongs to.
\[ f(x) = \ln{(x-5)}-2 \]The solution is \( f^{-1}(9) = 59879.142 \), which is option E.\begin{enumerate}[label=\Alph*.]
\item \( f^{-1}(9) \in [1099.63, 1108.63] \)

 This solution corresponds to distractor 1.
\item \( f^{-1}(9) \in [59866.14, 59873.14] \)

 This solution corresponds to distractor 3.
\item \( f^{-1}(9) \in [51.6, 54.6] \)

 This solution corresponds to distractor 4.
\item \( f^{-1}(9) \in [1202602.28, 1202607.28] \)

 This solution corresponds to distractor 2.
\item \( f^{-1}(9) \in [59877.14, 59881.14] \)

 This is the solution.
\end{enumerate}

\textbf{General Comment:} Natural log and exponential functions always have an inverse. Once you switch the $x$ and $y$, use the conversion $ e^y = x \leftrightarrow y=\ln(x)$.
}
\litem{
Find the inverse of the function below (if it exists). Then, evaluate the inverse at $x = -13$ and choose the interval that $f^-1(-13)$ belongs to.
\[ f(x) = \sqrt[3]{5 x - 3} \]The solution is \( -438.8 \), which is option B.\begin{enumerate}[label=\Alph*.]
\item \( f^{-1}(-13) \in [439.58, 440.16] \)

 This solution corresponds to distractor 3.
\item \( f^{-1}(-13) \in [-439.47, -438.01] \)

* This is the correct solution.
\item \( f^{-1}(-13) \in [438.07, 439.06] \)

 This solution corresponds to distractor 2.
\item \( f^{-1}(-13) \in [-441.24, -439.27] \)

 Distractor 1: This corresponds to 
\item \( \text{ The function is not invertible for all Real numbers. } \)

 This solution corresponds to distractor 4.
\end{enumerate}

\textbf{General Comment:} Be sure you check that the function is 1-1 before trying to find the inverse!
}
\litem{
Find the inverse of the function below. Then, evaluate the inverse at $x = 8$ and choose the interval that $f^-1(8)$ belongs to.
\[ f(x) = \ln{(x-4)}+2 \]The solution is \( f^{-1}(8) = 407.429 \), which is option C.\begin{enumerate}[label=\Alph*.]
\item \( f^{-1}(8) \in [50.6, 61.6] \)

 This solution corresponds to distractor 4.
\item \( f^{-1}(8) \in [162749.79, 162758.79] \)

 This solution corresponds to distractor 2.
\item \( f^{-1}(8) \in [406.43, 412.43] \)

 This is the solution.
\item \( f^{-1}(8) \in [22026.47, 22035.47] \)

 This solution corresponds to distractor 1.
\item \( f^{-1}(8) \in [398.43, 402.43] \)

 This solution corresponds to distractor 3.
\end{enumerate}

\textbf{General Comment:} Natural log and exponential functions always have an inverse. Once you switch the $x$ and $y$, use the conversion $ e^y = x \leftrightarrow y=\ln(x)$.
}
\litem{
Multiply the following functions, then choose the domain of the resulting function from the list below.
\[ f(x) = 9x^{3} +7 x^{2} +8 x + 4 \text{ and } g(x) = 7x^{3} +4 x^{2} +9 x + 7 \]The solution is \( (-\infty, \infty) \), which is option E.\begin{enumerate}[label=\Alph*.]
\item \( \text{ The domain is all Real numbers greater than or equal to } x = a, \text{ where } a \in [-7.5, -5.5] \)


\item \( \text{ The domain is all Real numbers less than or equal to } x = a, \text{ where } a \in [3.4, 6.4] \)


\item \( \text{ The domain is all Real numbers except } x = a, \text{ where } a \in [2.8, 9.8] \)


\item \( \text{ The domain is all Real numbers except } x = a \text{ and } x = b, \text{ where } a \in [6.25, 10.25] \text{ and } b \in [5.8, 8.8] \)


\item \( \text{ The domain is all Real numbers. } \)


\end{enumerate}

\textbf{General Comment:} The new domain is the intersection of the previous domains.
}
\litem{
Find the inverse of the function below (if it exists). Then, evaluate the inverse at $x = 11$ and choose the interval that $f^-1(11)$ belongs to.
\[ f(x) = 5 x^2 + 3 \]The solution is \( \text{ The function is not invertible for all Real numbers. } \), which is option E.\begin{enumerate}[label=\Alph*.]
\item \( f^{-1}(11) \in [1.47, 1.75] \)

 Distractor 2: This corresponds to finding the (nonexistent) inverse and not subtracting by the vertical shift.
\item \( f^{-1}(11) \in [5.09, 5.31] \)

 Distractor 4: This corresponds to both distractors 2 and 3.
\item \( f^{-1}(11) \in [2.19, 2.55] \)

 Distractor 3: This corresponds to finding the (nonexistent) inverse and dividing by a negative.
\item \( f^{-1}(11) \in [1.13, 1.27] \)

 Distractor 1: This corresponds to trying to find the inverse even though the function is not 1-1. 
\item \( \text{ The function is not invertible for all Real numbers. } \)

* This is the correct option.
\end{enumerate}

\textbf{General Comment:} Be sure you check that the function is 1-1 before trying to find the inverse!
}
\litem{
Multiply the following functions, then choose the domain of the resulting function from the list below.
\[ f(x) = \sqrt{-4x+14}  \text{ and } g(x) = 8x^{2} +8 x + 5 \]The solution is \( \text{ The domain is all Real numbers less than or equal to} x = 3.5. \), which is option C.\begin{enumerate}[label=\Alph*.]
\item \( \text{ The domain is all Real numbers except } x = a, \text{ where } a \in [4.33, 13.33] \)


\item \( \text{ The domain is all Real numbers greater than or equal to } x = a, \text{ where } a \in [-7.25, 1.75] \)


\item \( \text{ The domain is all Real numbers less than or equal to } x = a, \text{ where } a \in [-2.5, 8.5] \)


\item \( \text{ The domain is all Real numbers except } x = a \text{ and } x = b, \text{ where } a \in [-0.67, 5.33] \text{ and } b \in [5.4, 10.4] \)


\item \( \text{ The domain is all Real numbers. } \)


\end{enumerate}

\textbf{General Comment:} The new domain is the intersection of the previous domains.
}
\litem{
Choose the interval below that $f$ composed with $g$ at $x=1$ is in.
\[ f(x) = -4x^{3} +3 x^{2} +x -2 \text{ and } g(x) = -x^{3} -2 x^{2} +3 x \]The solution is \( -2.0 \), which is option C.\begin{enumerate}[label=\Alph*.]
\item \( (f \circ g)(1) \in [-7.5, -5.4] \)

 Distractor 1: Corresponds to reversing the composition.
\item \( (f \circ g)(1) \in [-1.8, -0.2] \)

 Distractor 3: Corresponds to being slightly off from the solution.
\item \( (f \circ g)(1) \in [-3.8, -1.6] \)

* This is the correct solution
\item \( (f \circ g)(1) \in [-10, -7.5] \)

 Distractor 2: Corresponds to being slightly off from the solution.
\item \( \text{It is not possible to compose the two functions.} \)


\end{enumerate}

\textbf{General Comment:} $f$ composed with $g$ at $x$ means $f(g(x))$. The order matters!
}
\litem{
Choose the interval below that $f$ composed with $g$ at $x=1$ is in.
\[ f(x) = -2x^{3} +4 x^{2} -4 x \text{ and } g(x) = -x^{3} +2 x^{2} -x + 3 \]The solution is \( -30.0 \), which is option B.\begin{enumerate}[label=\Alph*.]
\item \( (f \circ g)(1) \in [27, 35] \)

 Distractor 3: Corresponds to being slightly off from the solution.
\item \( (f \circ g)(1) \in [-35, -28] \)

* This is the correct solution
\item \( (f \circ g)(1) \in [20, 23] \)

 Distractor 1: Corresponds to reversing the composition.
\item \( (f \circ g)(1) \in [-23, -22] \)

 Distractor 2: Corresponds to being slightly off from the solution.
\item \( \text{It is not possible to compose the two functions.} \)


\end{enumerate}

\textbf{General Comment:} $f$ composed with $g$ at $x$ means $f(g(x))$. The order matters!
}
\litem{
Determine whether the function below is 1-1.
\[ f(x) = -12 x^2 - 99 x - 195 \]The solution is \( \text{no} \), which is option D.\begin{enumerate}[label=\Alph*.]
\item \( \text{No, because the domain of the function is not $(-\infty, \infty)$.} \)

Corresponds to believing 1-1 means the domain is all Real numbers.
\item \( \text{No, because there is an $x$-value that goes to 2 different $y$-values.} \)

Corresponds to the Vertical Line test, which checks if an expression is a function.
\item \( \text{No, because the range of the function is not $(-\infty, \infty)$.} \)

Corresponds to believing 1-1 means the range is all Real numbers.
\item \( \text{No, because there is a $y$-value that goes to 2 different $x$-values.} \)

* This is the solution.
\item \( \text{Yes, the function is 1-1.} \)

Corresponds to believing the function passes the Horizontal Line test.
\end{enumerate}

\textbf{General Comment:} There are only two valid options: The function is 1-1 OR No because there is a $y$-value that goes to 2 different $x$-values.
}
\litem{
Determine whether the function below is 1-1.
\[ f(x) = (5 x - 36)^3 \]The solution is \( \text{yes} \), which is option C.\begin{enumerate}[label=\Alph*.]
\item \( \text{No, because there is an $x$-value that goes to 2 different $y$-values.} \)

Corresponds to the Vertical Line test, which checks if an expression is a function.
\item \( \text{No, because the domain of the function is not $(-\infty, \infty)$.} \)

Corresponds to believing 1-1 means the domain is all Real numbers.
\item \( \text{Yes, the function is 1-1.} \)

* This is the solution.
\item \( \text{No, because the range of the function is not $(-\infty, \infty)$.} \)

Corresponds to believing 1-1 means the range is all Real numbers.
\item \( \text{No, because there is a $y$-value that goes to 2 different $x$-values.} \)

Corresponds to the Horizontal Line test, which this function passes.
\end{enumerate}

\textbf{General Comment:} There are only two valid options: The function is 1-1 OR No because there is a $y$-value that goes to 2 different $x$-values.
}
\litem{
Find the inverse of the function below. Then, evaluate the inverse at $x = 7$ and choose the interval that $f^-1(7)$ belongs to.
\[ f(x) = \ln{(x-5)}-4 \]The solution is \( f^{-1}(7) = 59879.142 \), which is option E.\begin{enumerate}[label=\Alph*.]
\item \( f^{-1}(7) \in [59862.14, 59872.14] \)

 This solution corresponds to distractor 3.
\item \( f^{-1}(7) \in [24.09, 28.09] \)

 This solution corresponds to distractor 1.
\item \( f^{-1}(7) \in [162746.79, 162754.79] \)

 This solution corresponds to distractor 2.
\item \( f^{-1}(7) \in [0.39, 10.39] \)

 This solution corresponds to distractor 4.
\item \( f^{-1}(7) \in [59879.14, 59883.14] \)

 This is the solution.
\end{enumerate}

\textbf{General Comment:} Natural log and exponential functions always have an inverse. Once you switch the $x$ and $y$, use the conversion $ e^y = x \leftrightarrow y=\ln(x)$.
}
\litem{
Find the inverse of the function below (if it exists). Then, evaluate the inverse at $x = -14$ and choose the interval that $f^-1(-14)$ belongs to.
\[ f(x) = 4 x^2 + 3 \]The solution is \( \text{ The function is not invertible for all Real numbers. } \), which is option E.\begin{enumerate}[label=\Alph*.]
\item \( f^{-1}(-14) \in [1.53, 1.84] \)

 Distractor 2: This corresponds to finding the (nonexistent) inverse and not subtracting by the vertical shift.
\item \( f^{-1}(-14) \in [2.88, 3.62] \)

 Distractor 3: This corresponds to finding the (nonexistent) inverse and dividing by a negative.
\item \( f^{-1}(-14) \in [1.89, 2.14] \)

 Distractor 1: This corresponds to trying to find the inverse even though the function is not 1-1. 
\item \( f^{-1}(-14) \in [3.81, 4.19] \)

 Distractor 4: This corresponds to both distractors 2 and 3.
\item \( \text{ The function is not invertible for all Real numbers. } \)

* This is the correct option.
\end{enumerate}

\textbf{General Comment:} Be sure you check that the function is 1-1 before trying to find the inverse!
}
\litem{
Find the inverse of the function below. Then, evaluate the inverse at $x = 6$ and choose the interval that $f^-1(6)$ belongs to.
\[ f(x) = e^{x-4}-4 \]The solution is \( f^{-1}(6) = 6.303 \), which is option D.\begin{enumerate}[label=\Alph*.]
\item \( f^{-1}(6) \in [-7.31, -2.31] \)

 This solution corresponds to distractor 2.
\item \( f^{-1}(6) \in [-1.7, 3.3] \)

 This solution corresponds to distractor 3.
\item \( f^{-1}(6) \in [-7.31, -2.31] \)

 This solution corresponds to distractor 4.
\item \( f^{-1}(6) \in [6.3, 9.3] \)

 This is the solution.
\item \( f^{-1}(6) \in [-1.7, 3.3] \)

 This solution corresponds to distractor 1.
\end{enumerate}

\textbf{General Comment:} Natural log and exponential functions always have an inverse. Once you switch the $x$ and $y$, use the conversion $ e^y = x \leftrightarrow y=\ln(x)$.
}
\litem{
Multiply the following functions, then choose the domain of the resulting function from the list below.
\[ f(x) = 8x + 2 \text{ and } g(x) = \sqrt{3x+14}  \]The solution is \( \text{ The domain is all Real numbers greater than or equal to} x = -4.67. \), which is option B.\begin{enumerate}[label=\Alph*.]
\item \( \text{ The domain is all Real numbers less than or equal to } x = a, \text{ where } a \in [-2.17, 1.83] \)


\item \( \text{ The domain is all Real numbers greater than or equal to } x = a, \text{ where } a \in [-5.67, -2.67] \)


\item \( \text{ The domain is all Real numbers except } x = a, \text{ where } a \in [2.25, 7.25] \)


\item \( \text{ The domain is all Real numbers except } x = a \text{ and } x = b, \text{ where } a \in [4.67, 12.67] \text{ and } b \in [6.67, 10.67] \)


\item \( \text{ The domain is all Real numbers. } \)


\end{enumerate}

\textbf{General Comment:} The new domain is the intersection of the previous domains.
}
\litem{
Find the inverse of the function below (if it exists). Then, evaluate the inverse at $x = -14$ and choose the interval that $f^-1(-14)$ belongs to.
\[ f(x) = 5 x^2 - 4 \]The solution is \( \text{ The function is not invertible for all Real numbers. } \), which is option E.\begin{enumerate}[label=\Alph*.]
\item \( f^{-1}(-14) \in [4.27, 4.55] \)

 Distractor 3: This corresponds to finding the (nonexistent) inverse and dividing by a negative.
\item \( f^{-1}(-14) \in [1.76, 2.08] \)

 Distractor 2: This corresponds to finding the (nonexistent) inverse and not subtracting by the vertical shift.
\item \( f^{-1}(-14) \in [1.22, 1.83] \)

 Distractor 1: This corresponds to trying to find the inverse even though the function is not 1-1. 
\item \( f^{-1}(-14) \in [6.66, 7.48] \)

 Distractor 4: This corresponds to both distractors 2 and 3.
\item \( \text{ The function is not invertible for all Real numbers. } \)

* This is the correct option.
\end{enumerate}

\textbf{General Comment:} Be sure you check that the function is 1-1 before trying to find the inverse!
}
\litem{
Subtract the following functions, then choose the domain of the resulting function from the list below.
\[ f(x) = 7x + 8 \text{ and } g(x) = \sqrt{-4x+11}  \]The solution is \( \text{ The domain is all Real numbers less than or equal to} x = 2.75. \), which is option B.\begin{enumerate}[label=\Alph*.]
\item \( \text{ The domain is all Real numbers greater than or equal to } x = a, \text{ where } a \in [-5.6, -3.6] \)


\item \( \text{ The domain is all Real numbers less than or equal to } x = a, \text{ where } a \in [-0.25, 3.75] \)


\item \( \text{ The domain is all Real numbers except } x = a, \text{ where } a \in [-11.2, -6.2] \)


\item \( \text{ The domain is all Real numbers except } x = a \text{ and } x = b, \text{ where } a \in [-9.8, -4.8] \text{ and } b \in [-4.2, 1.8] \)


\item \( \text{ The domain is all Real numbers. } \)


\end{enumerate}

\textbf{General Comment:} The new domain is the intersection of the previous domains.
}
\litem{
Choose the interval below that $f$ composed with $g$ at $x=1$ is in.
\[ f(x) = -x^{3} +2 x^{2} +x -3 \text{ and } g(x) = 2x^{3} +2 x^{2} -2 x \]The solution is \( -1.0 \), which is option B.\begin{enumerate}[label=\Alph*.]
\item \( (f \circ g)(1) \in [1.69, 3.52] \)

 Distractor 1: Corresponds to reversing the composition.
\item \( (f \circ g)(1) \in [-2.82, -0.76] \)

* This is the correct solution
\item \( (f \circ g)(1) \in [3.04, 5.12] \)

 Distractor 2: Corresponds to being slightly off from the solution.
\item \( (f \circ g)(1) \in [7.85, 8.59] \)

 Distractor 3: Corresponds to being slightly off from the solution.
\item \( \text{It is not possible to compose the two functions.} \)


\end{enumerate}

\textbf{General Comment:} $f$ composed with $g$ at $x$ means $f(g(x))$. The order matters!
}
\litem{
Choose the interval below that $f$ composed with $g$ at $x=2$ is in.
\[ f(x) = -2x^{3} +3 x^{2} +2 x \text{ and } g(x) = -2x^{3} +2 x^{2} +4 x \]The solution is \( 0.0 \), which is option B.\begin{enumerate}[label=\Alph*.]
\item \( (f \circ g)(2) \in [-0.5, 0.1] \)

 Distractor 1: Corresponds to reversing the composition.
\item \( (f \circ g)(2) \in [-0.5, 0.1] \)

* This is the correct solution
\item \( (f \circ g)(2) \in [9.4, 11] \)

 Distractor 2: Corresponds to being slightly off from the solution.
\item \( (f \circ g)(2) \in [5.2, 9.2] \)

 Distractor 3: Corresponds to being slightly off from the solution.
\item \( \text{It is not possible to compose the two functions.} \)


\end{enumerate}

\textbf{General Comment:} $f$ composed with $g$ at $x$ means $f(g(x))$. The order matters!
}
\litem{
Determine whether the function below is 1-1.
\[ f(x) = -20 x^2 - 247 x - 713 \]The solution is \( \text{no} \), which is option A.\begin{enumerate}[label=\Alph*.]
\item \( \text{No, because there is a $y$-value that goes to 2 different $x$-values.} \)

* This is the solution.
\item \( \text{Yes, the function is 1-1.} \)

Corresponds to believing the function passes the Horizontal Line test.
\item \( \text{No, because the domain of the function is not $(-\infty, \infty)$.} \)

Corresponds to believing 1-1 means the domain is all Real numbers.
\item \( \text{No, because there is an $x$-value that goes to 2 different $y$-values.} \)

Corresponds to the Vertical Line test, which checks if an expression is a function.
\item \( \text{No, because the range of the function is not $(-\infty, \infty)$.} \)

Corresponds to believing 1-1 means the range is all Real numbers.
\end{enumerate}

\textbf{General Comment:} There are only two valid options: The function is 1-1 OR No because there is a $y$-value that goes to 2 different $x$-values.
}
\litem{
Determine whether the function below is 1-1.
\[ f(x) = 36 x^2 + 300 x + 625 \]The solution is \( \text{no} \), which is option D.\begin{enumerate}[label=\Alph*.]
\item \( \text{No, because the domain of the function is not $(-\infty, \infty)$.} \)

Corresponds to believing 1-1 means the domain is all Real numbers.
\item \( \text{No, because the range of the function is not $(-\infty, \infty)$.} \)

Corresponds to believing 1-1 means the range is all Real numbers.
\item \( \text{Yes, the function is 1-1.} \)

Corresponds to believing the function passes the Horizontal Line test.
\item \( \text{No, because there is a $y$-value that goes to 2 different $x$-values.} \)

* This is the solution.
\item \( \text{No, because there is an $x$-value that goes to 2 different $y$-values.} \)

Corresponds to the Vertical Line test, which checks if an expression is a function.
\end{enumerate}

\textbf{General Comment:} There are only two valid options: The function is 1-1 OR No because there is a $y$-value that goes to 2 different $x$-values.
}
\litem{
Find the inverse of the function below. Then, evaluate the inverse at $x = 10$ and choose the interval that $f^-1(10)$ belongs to.
\[ f(x) = \ln{(x-3)}+5 \]The solution is \( f^{-1}(10) = 151.413 \), which is option B.\begin{enumerate}[label=\Alph*.]
\item \( f^{-1}(10) \in [442417.39, 442419.39] \)

 This solution corresponds to distractor 2.
\item \( f^{-1}(10) \in [150.41, 156.41] \)

 This is the solution.
\item \( f^{-1}(10) \in [140.41, 151.41] \)

 This solution corresponds to distractor 3.
\item \( f^{-1}(10) \in [3269014.37, 3269023.37] \)

 This solution corresponds to distractor 1.
\item \( f^{-1}(10) \in [1097.63, 1104.63] \)

 This solution corresponds to distractor 4.
\end{enumerate}

\textbf{General Comment:} Natural log and exponential functions always have an inverse. Once you switch the $x$ and $y$, use the conversion $ e^y = x \leftrightarrow y=\ln(x)$.
}
\litem{
Find the inverse of the function below (if it exists). Then, evaluate the inverse at $x = 10$ and choose the interval that $f^-1(10)$ belongs to.
\[ f(x) = \sqrt[3]{5 x + 4} \]The solution is \( 199.2 \), which is option B.\begin{enumerate}[label=\Alph*.]
\item \( f^{-1}(10) \in [-199.5, -199.1] \)

 This solution corresponds to distractor 2.
\item \( f^{-1}(10) \in [198.4, 199.8] \)

* This is the correct solution.
\item \( f^{-1}(10) \in [199.5, 202.3] \)

 Distractor 1: This corresponds to 
\item \( f^{-1}(10) \in [-203.4, -200.2] \)

 This solution corresponds to distractor 3.
\item \( \text{ The function is not invertible for all Real numbers. } \)

 This solution corresponds to distractor 4.
\end{enumerate}

\textbf{General Comment:} Be sure you check that the function is 1-1 before trying to find the inverse!
}
\end{enumerate}

\end{document}