\documentclass{extbook}[14pt]
\usepackage{multicol, enumerate, enumitem, hyperref, color, soul, setspace, parskip, fancyhdr, amssymb, amsthm, amsmath, latexsym, units, mathtools}
\everymath{\displaystyle}
\usepackage[headsep=0.5cm,headheight=0cm, left=1 in,right= 1 in,top= 1 in,bottom= 1 in]{geometry}
\usepackage{dashrule}  % Package to use the command below to create lines between items
\newcommand{\litem}[1]{\item #1

\rule{\textwidth}{0.4pt}}
\pagestyle{fancy}
\lhead{}
\chead{Answer Key for Progress Quiz 9 Version B}
\rhead{}
\lfoot{9541-5764}
\cfoot{}
\rfoot{Summer C 2021}
\begin{document}
\textbf{This key should allow you to understand why you choose the option you did (beyond just getting a question right or wrong). \href{https://xronos.clas.ufl.edu/mac1105spring2020/courseDescriptionAndMisc/Exams/LearningFromResults}{More instructions on how to use this key can be found here}.}

\textbf{If you have a suggestion to make the keys better, \href{https://forms.gle/CZkbZmPbC9XALEE88}{please fill out the short survey here}.}

\textit{Note: This key is auto-generated and may contain issues and/or errors. The keys are reviewed after each exam to ensure grading is done accurately. If there are issues (like duplicate options), they are noted in the offline gradebook. The keys are a work-in-progress to give students as many resources to improve as possible.}

\rule{\textwidth}{0.4pt}

\begin{enumerate}\litem{
What are the \textit{possible Integer} roots of the polynomial below?
\[ f(x) = 6x^{3} +4 x^{2} +3 x + 3 \]The solution is \( \pm 1,\pm 3 \), which is option C.\begin{enumerate}[label=\Alph*.]
\item \( \text{ All combinations of: }\frac{\pm 1,\pm 2,\pm 3,\pm 6}{\pm 1,\pm 3} \)

 Distractor 3: Corresponds to the plus or minus of the inverse quotient (an/a0) of the factors. 
\item \( \pm 1,\pm 2,\pm 3,\pm 6 \)

 Distractor 1: Corresponds to the plus or minus factors of a1 only.
\item \( \pm 1,\pm 3 \)

* This is the solution \textbf{since we asked for the possible Integer roots}!
\item \( \text{ All combinations of: }\frac{\pm 1,\pm 3}{\pm 1,\pm 2,\pm 3,\pm 6} \)

This would have been the solution \textbf{if asked for the possible Rational roots}!
\item \( \text{There is no formula or theorem that tells us all possible Integer roots.} \)

 Distractor 4: Corresponds to not recognizing Integers as a subset of Rationals.
\end{enumerate}

\textbf{General Comment:} We have a way to find the possible Rational roots. The possible Integer roots are the Integers in this list.
}
\litem{
Perform the division below. Then, find the intervals that correspond to the quotient in the form $ax^2+bx+c$ and remainder $r$.
\[ \frac{15x^{3} -38 x^{2} + 34}{x -2} \]The solution is \( 15x^{2} -8 x -16 + \frac{2}{x -2} \), which is option C.\begin{enumerate}[label=\Alph*.]
\item \( a \in [27, 31], b \in [21, 25], c \in [41, 48], \text{ and } r \in [122, 126]. \)

 You multipled by the synthetic number rather than bringing the first factor down.
\item \( a \in [27, 31], b \in [-100, -94], c \in [196, 202], \text{ and } r \in [-362, -355]. \)

 You divided by the opposite of the factor AND multipled the first factor rather than just bringing it down.
\item \( a \in [13, 19], b \in [-8, -7], c \in [-18, -15], \text{ and } r \in [0, 8]. \)

* This is the solution!
\item \( a \in [13, 19], b \in [-72, -66], c \in [135, 142], \text{ and } r \in [-244, -232]. \)

 You divided by the opposite of the factor.
\item \( a \in [13, 19], b \in [-23, -20], c \in [-27, -20], \text{ and } r \in [10, 15]. \)

 You multipled by the synthetic number and subtracted rather than adding during synthetic division.
\end{enumerate}

\textbf{General Comment:} Be sure to synthetically divide by the zero of the denominator! Also, make sure to include 0 placeholders for missing terms.
}
\litem{
Factor the polynomial below completely, knowing that $x -5$ is a factor. Then, choose the intervals the zeros of the polynomial belong to, where $z_1 \leq z_2 \leq z_3 \leq z_4$. \textit{To make the problem easier, all zeros are between -5 and 5.}
\[ f(x) = 12x^{4} -1 x^{3} -266 x^{2} -205 x + 300 \]The solution is \( [-4, -1.667, 0.75, 5] \), which is option A.\begin{enumerate}[label=\Alph*.]
\item \( z_1 \in [-4.9, -3.1], \text{   }  z_2 \in [-1.79, -1.55], z_3 \in [0.73, 0.83], \text{   and   } z_4 \in [4.82, 5.32] \)

* This is the solution!
\item \( z_1 \in [-5.2, -4.6], \text{   }  z_2 \in [-1.39, -1.31], z_3 \in [0.52, 0.73], \text{   and   } z_4 \in [3.73, 4.71] \)

 Distractor 3: Corresponds to negatives of all zeros AND inversing rational roots.
\item \( z_1 \in [-5.2, -4.6], \text{   }  z_2 \in [-0.91, -0.61], z_3 \in [1.58, 1.83], \text{   and   } z_4 \in [3.73, 4.71] \)

 Distractor 1: Corresponds to negatives of all zeros.
\item \( z_1 \in [-4.9, -3.1], \text{   }  z_2 \in [-0.71, -0.54], z_3 \in [1.24, 1.38], \text{   and   } z_4 \in [4.82, 5.32] \)

 Distractor 2: Corresponds to inversing rational roots.
\item \( z_1 \in [-5.2, -4.6], \text{   }  z_2 \in [-3.17, -2.81], z_3 \in [0.4, 0.57], \text{   and   } z_4 \in [3.73, 4.71] \)

 Distractor 4: Corresponds to moving factors from one rational to another.
\end{enumerate}

\textbf{General Comment:} Remember to try the middle-most integers first as these normally are the zeros. Also, once you get it to a quadratic, you can use your other factoring techniques to finish factoring.
}
\litem{
Factor the polynomial below completely. Then, choose the intervals the zeros of the polynomial belong to, where $z_1 \leq z_2 \leq z_3$. \textit{To make the problem easier, all zeros are between -5 and 5.}
\[ f(x) = 25x^{3} -50 x^{2} -69 x -18 \]The solution is \( [-0.6, -0.4, 3] \), which is option E.\begin{enumerate}[label=\Alph*.]
\item \( z_1 \in [-3.9, -2.8], \text{   }  z_2 \in [-0.07, 0.38], \text{   and   } z_3 \in [2.7, 4.2] \)

 Distractor 4: Corresponds to moving factors from one rational to another.
\item \( z_1 \in [-2.6, -1.3], \text{   }  z_2 \in [-1.75, -1.59], \text{   and   } z_3 \in [2.7, 4.2] \)

 Distractor 2: Corresponds to inversing rational roots.
\item \( z_1 \in [-3.9, -2.8], \text{   }  z_2 \in [0.23, 0.59], \text{   and   } z_3 \in [-0.3, 1.1] \)

 Distractor 1: Corresponds to negatives of all zeros.
\item \( z_1 \in [-3.9, -2.8], \text{   }  z_2 \in [1.62, 1.74], \text{   and   } z_3 \in [2.3, 2.9] \)

 Distractor 3: Corresponds to negatives of all zeros AND inversing rational roots.
\item \( z_1 \in [-1.6, 0.1], \text{   }  z_2 \in [-0.46, -0.26], \text{   and   } z_3 \in [2.7, 4.2] \)

* This is the solution!
\end{enumerate}

\textbf{General Comment:} Remember to try the middle-most integers first as these normally are the zeros. Also, once you get it to a quadratic, you can use your other factoring techniques to finish factoring.
}
\litem{
Factor the polynomial below completely. Then, choose the intervals the zeros of the polynomial belong to, where $z_1 \leq z_2 \leq z_3$. \textit{To make the problem easier, all zeros are between -5 and 5.}
\[ f(x) = 12x^{3} -77 x^{2} +131 x -60 \]The solution is \( [0.75, 1.67, 4] \), which is option A.\begin{enumerate}[label=\Alph*.]
\item \( z_1 \in [0.63, 0.78], \text{   }  z_2 \in [1.57, 1.81], \text{   and   } z_3 \in [4, 4.07] \)

* This is the solution!
\item \( z_1 \in [0.6, 0.71], \text{   }  z_2 \in [1.23, 1.4], \text{   and   } z_3 \in [4, 4.07] \)

 Distractor 2: Corresponds to inversing rational roots.
\item \( z_1 \in [-4.1, -3.95], \text{   }  z_2 \in [-1.92, -1.56], \text{   and   } z_3 \in [-0.81, -0.65] \)

 Distractor 1: Corresponds to negatives of all zeros.
\item \( z_1 \in [-5.12, -4.95], \text{   }  z_2 \in [-4.14, -3.6], \text{   and   } z_3 \in [-0.28, -0.19] \)

 Distractor 4: Corresponds to moving factors from one rational to another.
\item \( z_1 \in [-4.1, -3.95], \text{   }  z_2 \in [-1.56, -1.04], \text{   and   } z_3 \in [-0.74, -0.46] \)

 Distractor 3: Corresponds to negatives of all zeros AND inversing rational roots.
\end{enumerate}

\textbf{General Comment:} Remember to try the middle-most integers first as these normally are the zeros. Also, once you get it to a quadratic, you can use your other factoring techniques to finish factoring.
}
\litem{
Perform the division below. Then, find the intervals that correspond to the quotient in the form $ax^2+bx+c$ and remainder $r$.
\[ \frac{16x^{3} -24 x^{2} -31 x + 35}{x -2} \]The solution is \( 16x^{2} +8 x -15 + \frac{5}{x -2} \), which is option B.\begin{enumerate}[label=\Alph*.]
\item \( a \in [30, 39], \text{   } b \in [-89, -86], \text{   } c \in [145, 149], \text{   and   } r \in [-257, -250]. \)

 You divided by the opposite of the factor AND multiplied the first factor rather than just bringing it down.
\item \( a \in [13, 17], \text{   } b \in [1, 9], \text{   } c \in [-15, -11], \text{   and   } r \in [2, 13]. \)

* This is the solution!
\item \( a \in [13, 17], \text{   } b \in [-59, -55], \text{   } c \in [74, 85], \text{   and   } r \in [-127, -124]. \)

 You divided by the opposite of the factor.
\item \( a \in [30, 39], \text{   } b \in [39, 43], \text{   } c \in [46, 50], \text{   and   } r \in [127, 135]. \)

 You multiplied by the synthetic number rather than bringing the first factor down.
\item \( a \in [13, 17], \text{   } b \in [-11, -7], \text{   } c \in [-43, -34], \text{   and   } r \in [-4, 2]. \)

 You multiplied by the synthetic number and subtracted rather than adding during synthetic division.
\end{enumerate}

\textbf{General Comment:} Be sure to synthetically divide by the zero of the denominator!
}
\litem{
Perform the division below. Then, find the intervals that correspond to the quotient in the form $ax^2+bx+c$ and remainder $r$.
\[ \frac{10x^{3} -40 x^{2} -10 x + 37}{x -4} \]The solution is \( 10x^{2} -10 + \frac{-3}{x -4} \), which is option A.\begin{enumerate}[label=\Alph*.]
\item \( a \in [7, 13], \text{   } b \in [-2, 7], \text{   } c \in [-10, -6], \text{   and   } r \in [-5, 0]. \)

* This is the solution!
\item \( a \in [7, 13], \text{   } b \in [-11, -3], \text{   } c \in [-42, -38], \text{   and   } r \in [-88, -79]. \)

 You multiplied by the synthetic number and subtracted rather than adding during synthetic division.
\item \( a \in [35, 44], \text{   } b \in [-201, -193], \text{   } c \in [788, 794], \text{   and   } r \in [-3124, -3119]. \)

 You divided by the opposite of the factor AND multiplied the first factor rather than just bringing it down.
\item \( a \in [7, 13], \text{   } b \in [-83, -73], \text{   } c \in [306, 313], \text{   and   } r \in [-1203, -1196]. \)

 You divided by the opposite of the factor.
\item \( a \in [35, 44], \text{   } b \in [119, 129], \text{   } c \in [464, 476], \text{   and   } r \in [1916, 1918]. \)

 You multiplied by the synthetic number rather than bringing the first factor down.
\end{enumerate}

\textbf{General Comment:} Be sure to synthetically divide by the zero of the denominator!
}
\litem{
Perform the division below. Then, find the intervals that correspond to the quotient in the form $ax^2+bx+c$ and remainder $r$.
\[ \frac{8x^{3} -62 x + 35}{x + 3} \]The solution is \( 8x^{2} -24 x + 10 + \frac{5}{x + 3} \), which is option E.\begin{enumerate}[label=\Alph*.]
\item \( a \in [-29, -20], b \in [-76, -71], c \in [-280, -275], \text{ and } r \in [-800, -798]. \)

 You divided by the opposite of the factor AND multipled the first factor rather than just bringing it down.
\item \( a \in [-29, -20], b \in [69, 75], c \in [-280, -275], \text{ and } r \in [869, 873]. \)

 You multipled by the synthetic number rather than bringing the first factor down.
\item \( a \in [8, 14], b \in [-35, -31], c \in [66, 67], \text{ and } r \in [-233, -227]. \)

 You multipled by the synthetic number and subtracted rather than adding during synthetic division.
\item \( a \in [8, 14], b \in [22, 30], c \in [9, 22], \text{ and } r \in [60, 72]. \)

 You divided by the opposite of the factor.
\item \( a \in [8, 14], b \in [-26, -21], c \in [9, 22], \text{ and } r \in [4, 8]. \)

* This is the solution!
\end{enumerate}

\textbf{General Comment:} Be sure to synthetically divide by the zero of the denominator! Also, make sure to include 0 placeholders for missing terms.
}
\litem{
Factor the polynomial below completely, knowing that $x -5$ is a factor. Then, choose the intervals the zeros of the polynomial belong to, where $z_1 \leq z_2 \leq z_3 \leq z_4$. \textit{To make the problem easier, all zeros are between -5 and 5.}
\[ f(x) = 15x^{4} -44 x^{3} -159 x^{2} +8 x + 60 \]The solution is \( [-2, -0.667, 0.6, 5] \), which is option D.\begin{enumerate}[label=\Alph*.]
\item \( z_1 \in [-10, -4], \text{   }  z_2 \in [-0.26, -0.18], z_3 \in [1.99, 2.03], \text{   and   } z_4 \in [1, 4] \)

 Distractor 4: Corresponds to moving factors from one rational to another.
\item \( z_1 \in [-10, -4], \text{   }  z_2 \in [-0.66, -0.58], z_3 \in [0.65, 0.68], \text{   and   } z_4 \in [1, 4] \)

 Distractor 1: Corresponds to negatives of all zeros.
\item \( z_1 \in [-10, -4], \text{   }  z_2 \in [-1.68, -1.51], z_3 \in [1.46, 1.52], \text{   and   } z_4 \in [1, 4] \)

 Distractor 3: Corresponds to negatives of all zeros AND inversing rational roots.
\item \( z_1 \in [-2, 1], \text{   }  z_2 \in [-0.73, -0.64], z_3 \in [0.58, 0.61], \text{   and   } z_4 \in [5, 6] \)

* This is the solution!
\item \( z_1 \in [-2, 1], \text{   }  z_2 \in [-1.52, -1.49], z_3 \in [1.63, 1.71], \text{   and   } z_4 \in [5, 6] \)

 Distractor 2: Corresponds to inversing rational roots.
\end{enumerate}

\textbf{General Comment:} Remember to try the middle-most integers first as these normally are the zeros. Also, once you get it to a quadratic, you can use your other factoring techniques to finish factoring.
}
\litem{
What are the \textit{possible Rational} roots of the polynomial below?
\[ f(x) = 3x^{2} +6 x + 7 \]The solution is \( \text{ All combinations of: }\frac{\pm 1,\pm 7}{\pm 1,\pm 3} \), which is option D.\begin{enumerate}[label=\Alph*.]
\item \( \pm 1,\pm 3 \)

 Distractor 1: Corresponds to the plus or minus factors of a1 only.
\item \( \text{ All combinations of: }\frac{\pm 1,\pm 3}{\pm 1,\pm 7} \)

 Distractor 3: Corresponds to the plus or minus of the inverse quotient (an/a0) of the factors. 
\item \( \pm 1,\pm 7 \)

This would have been the solution \textbf{if asked for the possible Integer roots}!
\item \( \text{ All combinations of: }\frac{\pm 1,\pm 7}{\pm 1,\pm 3} \)

* This is the solution \textbf{since we asked for the possible Rational roots}!
\item \( \text{ There is no formula or theorem that tells us all possible Rational roots.} \)

 Distractor 4: Corresponds to not recalling the theorem for rational roots of a polynomial.
\end{enumerate}

\textbf{General Comment:} We have a way to find the possible Rational roots. The possible Integer roots are the Integers in this list.
}
\end{enumerate}

\end{document}