\documentclass[14pt]{extbook}
\usepackage{multicol, enumerate, enumitem, hyperref, color, soul, setspace, parskip, fancyhdr} %General Packages
\usepackage{amssymb, amsthm, amsmath, latexsym, units, mathtools} %Math Packages
\everymath{\displaystyle} %All math in Display Style
% Packages with additional options
\usepackage[headsep=0.5cm,headheight=12pt, left=1 in,right= 1 in,top= 1 in,bottom= 1 in]{geometry}
\usepackage[usenames,dvipsnames]{xcolor}
\usepackage{dashrule}  % Package to use the command below to create lines between items
\newcommand{\litem}[1]{\item#1\hspace*{-1cm}\rule{\textwidth}{0.4pt}}
\pagestyle{fancy}
\lhead{Progress Quiz 9}
\chead{}
\rhead{Version C}
\lfoot{9541-5764}
\cfoot{}
\rfoot{Summer C 2021}
\begin{document}

\begin{enumerate}
\litem{
Find the inverse of the function below. Then, evaluate the inverse at $x = 6$ and choose the interval that $f^-1(6)$ belongs to.\[ f(x) = e^{x-4}-4 \]\begin{enumerate}[label=\Alph*.]
\item \( f^{-1}(6) \in [-7.31, -2.31] \)
\item \( f^{-1}(6) \in [-1.7, 3.3] \)
\item \( f^{-1}(6) \in [-7.31, -2.31] \)
\item \( f^{-1}(6) \in [6.3, 9.3] \)
\item \( f^{-1}(6) \in [-1.7, 3.3] \)

\end{enumerate} }
\litem{
Multiply the following functions, then choose the domain of the resulting function from the list below.\[ f(x) = 8x + 2 \text{ and } g(x) = \sqrt{3x+14}  \]\begin{enumerate}[label=\Alph*.]
\item \( \text{ The domain is all Real numbers less than or equal to } x = a, \text{ where } a \in [-2.17, 1.83] \)
\item \( \text{ The domain is all Real numbers greater than or equal to } x = a, \text{ where } a \in [-5.67, -2.67] \)
\item \( \text{ The domain is all Real numbers except } x = a, \text{ where } a \in [2.25, 7.25] \)
\item \( \text{ The domain is all Real numbers except } x = a \text{ and } x = b, \text{ where } a \in [4.67, 12.67] \text{ and } b \in [6.67, 10.67] \)
\item \( \text{ The domain is all Real numbers. } \)

\end{enumerate} }
\litem{
Find the inverse of the function below (if it exists). Then, evaluate the inverse at $x = -14$ and choose the interval that $f^-1(-14)$ belongs to.\[ f(x) = 5 x^2 - 4 \]\begin{enumerate}[label=\Alph*.]
\item \( f^{-1}(-14) \in [4.27, 4.55] \)
\item \( f^{-1}(-14) \in [1.76, 2.08] \)
\item \( f^{-1}(-14) \in [1.22, 1.83] \)
\item \( f^{-1}(-14) \in [6.66, 7.48] \)
\item \( \text{ The function is not invertible for all Real numbers. } \)

\end{enumerate} }
\litem{
Subtract the following functions, then choose the domain of the resulting function from the list below.\[ f(x) = 7x + 8 \text{ and } g(x) = \sqrt{-4x+11}  \]\begin{enumerate}[label=\Alph*.]
\item \( \text{ The domain is all Real numbers greater than or equal to } x = a, \text{ where } a \in [-5.6, -3.6] \)
\item \( \text{ The domain is all Real numbers less than or equal to } x = a, \text{ where } a \in [-0.25, 3.75] \)
\item \( \text{ The domain is all Real numbers except } x = a, \text{ where } a \in [-11.2, -6.2] \)
\item \( \text{ The domain is all Real numbers except } x = a \text{ and } x = b, \text{ where } a \in [-9.8, -4.8] \text{ and } b \in [-4.2, 1.8] \)
\item \( \text{ The domain is all Real numbers. } \)

\end{enumerate} }
\litem{
Choose the interval below that $f$ composed with $g$ at $x=1$ is in.\[ f(x) = -x^{3} +2 x^{2} +x -3 \text{ and } g(x) = 2x^{3} +2 x^{2} -2 x \]\begin{enumerate}[label=\Alph*.]
\item \( (f \circ g)(1) \in [1.69, 3.52] \)
\item \( (f \circ g)(1) \in [-2.82, -0.76] \)
\item \( (f \circ g)(1) \in [3.04, 5.12] \)
\item \( (f \circ g)(1) \in [7.85, 8.59] \)
\item \( \text{It is not possible to compose the two functions.} \)

\end{enumerate} }
\litem{
Choose the interval below that $f$ composed with $g$ at $x=2$ is in.\[ f(x) = -2x^{3} +3 x^{2} +2 x \text{ and } g(x) = -2x^{3} +2 x^{2} +4 x \]\begin{enumerate}[label=\Alph*.]
\item \( (f \circ g)(2) \in [-0.5, 0.1] \)
\item \( (f \circ g)(2) \in [-0.5, 0.1] \)
\item \( (f \circ g)(2) \in [9.4, 11] \)
\item \( (f \circ g)(2) \in [5.2, 9.2] \)
\item \( \text{It is not possible to compose the two functions.} \)

\end{enumerate} }
\litem{
Determine whether the function below is 1-1.\[ f(x) = -20 x^2 - 247 x - 713 \]\begin{enumerate}[label=\Alph*.]
\item \( \text{No, because there is a $y$-value that goes to 2 different $x$-values.} \)
\item \( \text{Yes, the function is 1-1.} \)
\item \( \text{No, because the domain of the function is not $(-\infty, \infty)$.} \)
\item \( \text{No, because there is an $x$-value that goes to 2 different $y$-values.} \)
\item \( \text{No, because the range of the function is not $(-\infty, \infty)$.} \)

\end{enumerate} }
\litem{
Determine whether the function below is 1-1.\[ f(x) = 36 x^2 + 300 x + 625 \]\begin{enumerate}[label=\Alph*.]
\item \( \text{No, because the domain of the function is not $(-\infty, \infty)$.} \)
\item \( \text{No, because the range of the function is not $(-\infty, \infty)$.} \)
\item \( \text{Yes, the function is 1-1.} \)
\item \( \text{No, because there is a $y$-value that goes to 2 different $x$-values.} \)
\item \( \text{No, because there is an $x$-value that goes to 2 different $y$-values.} \)

\end{enumerate} }
\litem{
Find the inverse of the function below. Then, evaluate the inverse at $x = 10$ and choose the interval that $f^-1(10)$ belongs to.\[ f(x) = \ln{(x-3)}+5 \]\begin{enumerate}[label=\Alph*.]
\item \( f^{-1}(10) \in [442417.39, 442419.39] \)
\item \( f^{-1}(10) \in [150.41, 156.41] \)
\item \( f^{-1}(10) \in [140.41, 151.41] \)
\item \( f^{-1}(10) \in [3269014.37, 3269023.37] \)
\item \( f^{-1}(10) \in [1097.63, 1104.63] \)

\end{enumerate} }
\litem{
Find the inverse of the function below (if it exists). Then, evaluate the inverse at $x = 10$ and choose the interval that $f^-1(10)$ belongs to.\[ f(x) = \sqrt[3]{5 x + 4} \]\begin{enumerate}[label=\Alph*.]
\item \( f^{-1}(10) \in [-199.5, -199.1] \)
\item \( f^{-1}(10) \in [198.4, 199.8] \)
\item \( f^{-1}(10) \in [199.5, 202.3] \)
\item \( f^{-1}(10) \in [-203.4, -200.2] \)
\item \( \text{ The function is not invertible for all Real numbers. } \)

\end{enumerate} }
\end{enumerate}

\end{document}