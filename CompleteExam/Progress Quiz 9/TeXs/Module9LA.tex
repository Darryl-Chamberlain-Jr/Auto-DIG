\documentclass[14pt]{extbook}
\usepackage{multicol, enumerate, enumitem, hyperref, color, soul, setspace, parskip, fancyhdr} %General Packages
\usepackage{amssymb, amsthm, amsmath, latexsym, units, mathtools} %Math Packages
\everymath{\displaystyle} %All math in Display Style
% Packages with additional options
\usepackage[headsep=0.5cm,headheight=12pt, left=1 in,right= 1 in,top= 1 in,bottom= 1 in]{geometry}
\usepackage[usenames,dvipsnames]{xcolor}
\usepackage{dashrule}  % Package to use the command below to create lines between items
\newcommand{\litem}[1]{\item#1\hspace*{-1cm}\rule{\textwidth}{0.4pt}}
\pagestyle{fancy}
\lhead{Progress Quiz 9}
\chead{}
\rhead{Version A}
\lfoot{9541-5764}
\cfoot{}
\rfoot{Summer C 2021}
\begin{document}

\begin{enumerate}
\litem{
Find the inverse of the function below. Then, evaluate the inverse at $x = 7$ and choose the interval that $f^-1(7)$ belongs to.\[ f(x) = \ln{(x-5)}+3 \]\begin{enumerate}[label=\Alph*.]
\item \( f^{-1}(7) \in [162751.79, 162762.79] \)
\item \( f^{-1}(7) \in [45.6, 50.6] \)
\item \( f^{-1}(7) \in [55.6, 61.6] \)
\item \( f^{-1}(7) \in [9.39, 11.39] \)
\item \( f^{-1}(7) \in [22030.47, 22034.47] \)

\end{enumerate} }
\litem{
Multiply the following functions, then choose the domain of the resulting function from the list below.\[ f(x) = \frac{4}{4x-19} \text{ and } g(x) = \frac{2}{6x-29} \]\begin{enumerate}[label=\Alph*.]
\item \( \text{ The domain is all Real numbers greater than or equal to } x = a, \text{ where } a \in [-8.67, -3.67] \)
\item \( \text{ The domain is all Real numbers less than or equal to } x = a, \text{ where } a \in [2, 8] \)
\item \( \text{ The domain is all Real numbers except } x = a, \text{ where } a \in [-9.2, -2.2] \)
\item \( \text{ The domain is all Real numbers except } x = a \text{ and } x = b, \text{ where } a \in [-0.25, 8.75] \text{ and } b \in [2.83, 6.83] \)
\item \( \text{ The domain is all Real numbers. } \)

\end{enumerate} }
\litem{
Find the inverse of the function below (if it exists). Then, evaluate the inverse at $x = -11$ and choose the interval that $f^-1(-11)$ belongs to.\[ f(x) = \sqrt[3]{2 x - 3} \]\begin{enumerate}[label=\Alph*.]
\item \( f^{-1}(-11) \in [661.2, 664.3] \)
\item \( f^{-1}(-11) \in [-666.5, -663.8] \)
\item \( f^{-1}(-11) \in [-668.5, -665] \)
\item \( f^{-1}(-11) \in [666.6, 669.2] \)
\item \( \text{ The function is not invertible for all Real numbers. } \)

\end{enumerate} }
\litem{
Subtract the following functions, then choose the domain of the resulting function from the list below.\[ f(x) = \frac{3}{4x-17} \text{ and } g(x) = \frac{5}{5x+34} \]\begin{enumerate}[label=\Alph*.]
\item \( \text{ The domain is all Real numbers greater than or equal to } x = a, \text{ where } a \in [0.67, 10.67] \)
\item \( \text{ The domain is all Real numbers less than or equal to } x = a, \text{ where } a \in [0.5, 7.5] \)
\item \( \text{ The domain is all Real numbers except } x = a, \text{ where } a \in [-6.6, -1.6] \)
\item \( \text{ The domain is all Real numbers except } x = a \text{ and } x = b, \text{ where } a \in [2.25, 11.25] \text{ and } b \in [-9.8, -4.8] \)
\item \( \text{ The domain is all Real numbers. } \)

\end{enumerate} }
\litem{
Choose the interval below that $f$ composed with $g$ at $x=-1$ is in.\[ f(x) = 4x^{3} -2 x^{2} -3 x + 1 \text{ and } g(x) = -3x^{3} -4 x^{2} +4 x + 4 \]\begin{enumerate}[label=\Alph*.]
\item \( (f \circ g)(-1) \in [2.9, 5.2] \)
\item \( (f \circ g)(-1) \in [-13.3, -11.7] \)
\item \( (f \circ g)(-1) \in [-9.3, -5.2] \)
\item \( (f \circ g)(-1) \in [-3.7, 1.5] \)
\item \( \text{It is not possible to compose the two functions.} \)

\end{enumerate} }
\litem{
Choose the interval below that $f$ composed with $g$ at $x=1$ is in.\[ f(x) = 4x^{3} -3 x^{2} -4 x \text{ and } g(x) = x^{3} -2 x^{2} -x \]\begin{enumerate}[label=\Alph*.]
\item \( (f \circ g)(1) \in [-37, -33.6] \)
\item \( (f \circ g)(1) \in [-43.1, -38.7] \)
\item \( (f \circ g)(1) \in [-35.4, -32.2] \)
\item \( (f \circ g)(1) \in [-46.5, -43] \)
\item \( \text{It is not possible to compose the two functions.} \)

\end{enumerate} }
\litem{
Determine whether the function below is 1-1.\[ f(x) = -18 x^2 + 30 x + 408 \]\begin{enumerate}[label=\Alph*.]
\item \( \text{No, because there is a $y$-value that goes to 2 different $x$-values.} \)
\item \( \text{No, because there is an $x$-value that goes to 2 different $y$-values.} \)
\item \( \text{No, because the domain of the function is not $(-\infty, \infty)$.} \)
\item \( \text{No, because the range of the function is not $(-\infty, \infty)$.} \)
\item \( \text{Yes, the function is 1-1.} \)

\end{enumerate} }
\litem{
Determine whether the function below is 1-1.\[ f(x) = -18 x^2 - 27 x + 551 \]\begin{enumerate}[label=\Alph*.]
\item \( \text{Yes, the function is 1-1.} \)
\item \( \text{No, because there is an $x$-value that goes to 2 different $y$-values.} \)
\item \( \text{No, because there is a $y$-value that goes to 2 different $x$-values.} \)
\item \( \text{No, because the range of the function is not $(-\infty, \infty)$.} \)
\item \( \text{No, because the domain of the function is not $(-\infty, \infty)$.} \)

\end{enumerate} }
\litem{
Find the inverse of the function below. Then, evaluate the inverse at $x = 9$ and choose the interval that $f^-1(9)$ belongs to.\[ f(x) = \ln{(x-5)}-2 \]\begin{enumerate}[label=\Alph*.]
\item \( f^{-1}(9) \in [1099.63, 1108.63] \)
\item \( f^{-1}(9) \in [59866.14, 59873.14] \)
\item \( f^{-1}(9) \in [51.6, 54.6] \)
\item \( f^{-1}(9) \in [1202602.28, 1202607.28] \)
\item \( f^{-1}(9) \in [59877.14, 59881.14] \)

\end{enumerate} }
\litem{
Find the inverse of the function below (if it exists). Then, evaluate the inverse at $x = -13$ and choose the interval that $f^-1(-13)$ belongs to.\[ f(x) = \sqrt[3]{5 x - 3} \]\begin{enumerate}[label=\Alph*.]
\item \( f^{-1}(-13) \in [439.58, 440.16] \)
\item \( f^{-1}(-13) \in [-439.47, -438.01] \)
\item \( f^{-1}(-13) \in [438.07, 439.06] \)
\item \( f^{-1}(-13) \in [-441.24, -439.27] \)
\item \( \text{ The function is not invertible for all Real numbers. } \)

\end{enumerate} }
\end{enumerate}

\end{document}