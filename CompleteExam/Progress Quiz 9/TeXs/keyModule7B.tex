\documentclass{extbook}[14pt]
\usepackage{multicol, enumerate, enumitem, hyperref, color, soul, setspace, parskip, fancyhdr, amssymb, amsthm, amsmath, latexsym, units, mathtools}
\everymath{\displaystyle}
\usepackage[headsep=0.5cm,headheight=0cm, left=1 in,right= 1 in,top= 1 in,bottom= 1 in]{geometry}
\usepackage{dashrule}  % Package to use the command below to create lines between items
\newcommand{\litem}[1]{\item #1

\rule{\textwidth}{0.4pt}}
\pagestyle{fancy}
\lhead{}
\chead{Answer Key for Progress Quiz 9 Version B}
\rhead{}
\lfoot{9541-5764}
\cfoot{}
\rfoot{Summer C 2021}
\begin{document}
\textbf{This key should allow you to understand why you choose the option you did (beyond just getting a question right or wrong). \href{https://xronos.clas.ufl.edu/mac1105spring2020/courseDescriptionAndMisc/Exams/LearningFromResults}{More instructions on how to use this key can be found here}.}

\textbf{If you have a suggestion to make the keys better, \href{https://forms.gle/CZkbZmPbC9XALEE88}{please fill out the short survey here}.}

\textit{Note: This key is auto-generated and may contain issues and/or errors. The keys are reviewed after each exam to ensure grading is done accurately. If there are issues (like duplicate options), they are noted in the offline gradebook. The keys are a work-in-progress to give students as many resources to improve as possible.}

\rule{\textwidth}{0.4pt}

\begin{enumerate}\litem{
Solve the rational equation below. Then, choose the interval(s) that the solution(s) belongs to.
\[ \frac{4}{7x -5} + 3 = \frac{-3}{42x -30} \]The solution is \( x = 0.500 \), which is option E.\begin{enumerate}[label=\Alph*.]
\item \( x_1 \in [0.18, 0.4] \text{ and } x_2 \in [-0.5,2.5] \)

$x = 0.381 \text{ and } x = 0.500$, which corresponds to getting the correct solution and believing there should be a second solution to the equation.
\item \( x \in [-1.11,-0.8] \)

$x = -0.929$, which corresponds to not distributing the factor $7x -5$ correctly when trying to eliminate the fraction.
\item \( \text{All solutions lead to invalid or complex values in the equation.} \)

This corresponds to thinking $x = 0.500$ leads to dividing by zero in the original equation, which it does not.
\item \( x_1 \in [-1.11, -0.8] \text{ and } x_2 \in [-0.5,2.5] \)

$x = -0.929 \text{ and } x = 0.500$, which corresponds to getting the correct solution and believing there should be a second solution to the equation.
\item \( x \in [-1.5,1.5] \)

* $x = 0.500$, which is the correct option.
\end{enumerate}

\textbf{General Comment:} Distractors are different based on the number of solutions. Remember that after solving, we need to make sure our solution does not make the original equation divide by zero!
}
\litem{
Choose the graph of the equation below.
\[ f(x) = \frac{1}{(x + 3)^2} - 3 \]The solution is the graph below, which is option A.
    \begin{center}
        \includegraphics[width=0.3\textwidth]{../Figures/rationalEquationToGraphCopyAB.png}
    \end{center}\begin{enumerate}[label=\Alph*.]
\begin{multicols}{2}
\item \includegraphics[width = 0.3\textwidth]{../Figures/rationalEquationToGraphCopyAB.png}
\item \includegraphics[width = 0.3\textwidth]{../Figures/rationalEquationToGraphCopyBB.png}
\item \includegraphics[width = 0.3\textwidth]{../Figures/rationalEquationToGraphCopyCB.png}
\item \includegraphics[width = 0.3\textwidth]{../Figures/rationalEquationToGraphCopyDB.png}
\end{multicols}\item None of the above.\end{enumerate}
\textbf{General Comment:} Remember that the general form of a basic rational equation is $ f(x) = \frac{a}{(x-h)^n} + k$, where $a$ is the leading coefficient (and in this case, we assume is either $1$ or $-1$), $n$ is the degree (in this case, either $1$ or $2$), and $(h, k)$ is the intersection of the asymptotes.
}
\litem{
Solve the rational equation below. Then, choose the interval(s) that the solution(s) belongs to.
\[ \frac{3x}{-2x + 6} + \frac{-4x^{2}}{-8x^{2} +28 x -12} = \frac{-4}{4x -2} \]The solution is \( \text{All solutions are invalid or lead to complex values in the equation.} \), which is option B.\begin{enumerate}[label=\Alph*.]
\item \( x_1 \in [-1.06, -0.31] \text{ and } x_2 \in [1.54,2.91] \)

$x = -0.706 \text{ and } x = 1.581$, which corresponds to making the discriminant from the Quadratic Formula positive to avoid complex solutions.
\item \( \text{All solutions lead to invalid or complex values in the equation.} \)

* The equation leads to solving $16x^{2} -14 x + 24=0$, which leads to complex solutions. This is the correct option.
\item \( x_1 \in [2.69, 3.2] \text{ and } x_2 \in [0.49,1.23] \)

$x = 3.000 \text{ and } x = 0.500$, which corresponds to solving $-2x + 6 = 0$ and $4x -2 = 0$ and treating them as solutions to the equation.
\item \( x \in [0.38,1.49] \)

$x = 0.500$, which corresponds to solving $4x -2 = 0$ and treating it as a solution to the equation.
\item \( x \in [2.69,3.2] \)

$x = 3.000$, which corresponds to solving $-2x + 6 = 0$ and treating it as a solution to the equation.
\end{enumerate}

\textbf{General Comment:} Distractors are different based on the number of solutions. Remember that after solving, we need to make sure our solution does not make the original equation divide by zero!
}
\litem{
Determine the domain of the function below.
\[ f(x) = \frac{5}{12x^{2} +2 x -24} \]The solution is \( \text{All Real numbers except } x = -1.500 \text{ and } x = 1.333. \), which is option C.\begin{enumerate}[label=\Alph*.]
\item \( \text{All Real numbers except } x = a, \text{ where } a \in [-1.5, -0.5] \)

All Real numbers except $x = -1.500$, which corresponds to removing only 1 value from the denominator.
\item \( \text{All Real numbers except } x = a, \text{ where } a \in [-24, -20] \)

All Real numbers except $x = -24.000$, which corresponds to removing a distractor value from the denominator.
\item \( \text{All Real numbers except } x = a \text{ and } x = b, \text{ where } a \in [-1.5, -0.5] \text{ and } b \in [-0.67, 2.33] \)

All Real numbers except $x = -1.500$ and $x = 1.333$, which is the correct option.
\item \( \text{All Real numbers except } x = a \text{ and } x = b, \text{ where } a \in [-24, -20] \text{ and } b \in [11, 13] \)

All Real numbers except $x = -24.000$ and $x = 12.000$, which corresponds to not factoring the denominator correctly.
\item \( \text{All Real numbers.} \)

This corresponds to thinking the denominator has complex roots or that rational functions have a domain of all Real numbers.
\end{enumerate}

\textbf{General Comment:} Recall that dividing by zero is not a real number. Therefore the domain is all real numbers \textbf{except} those that make the denominator 0.
}
\litem{
Solve the rational equation below. Then, choose the interval(s) that the solution(s) belongs to.
\[ \frac{-4}{-6x -5} + -8 = \frac{5}{-12x -10} \]The solution is \( x = -0.698 \), which is option E.\begin{enumerate}[label=\Alph*.]
\item \( x_1 \in [-0.7, 0.3] \text{ and } x_2 \in [-0.03,2.97] \)

$x = -0.698 \text{ and } x = 0.969$, which corresponds to getting the correct solution and believing there should be a second solution to the equation.
\item \( x_1 \in [-0.7, 0.3] \text{ and } x_2 \in [-0.65,0.35] \)

$x = -0.698 \text{ and } x = -0.646$, which corresponds to getting the correct solution and believing there should be a second solution to the equation.
\item \( \text{All solutions lead to invalid or complex values in the equation.} \)

This corresponds to thinking $x = -0.698$ leads to dividing by zero in the original equation, which it does not.
\item \( x \in [-0.03,3.97] \)

$x = 0.969$, which corresponds to not distributing the factor $-6x -5$ correctly when trying to eliminate the fraction.
\item \( x \in [-0.7,0.3] \)

* $x = -0.698$, which is the correct option.
\end{enumerate}

\textbf{General Comment:} Distractors are different based on the number of solutions. Remember that after solving, we need to make sure our solution does not make the original equation divide by zero!
}
\litem{
Choose the equation of the function graphed below.

\begin{center}
    \includegraphics[width=0.5\textwidth]{../Figures/rationalGraphToEquationB.png}
\end{center}


The solution is \( f(x) = \frac{-1}{(x - 2)^2} + 3 \), which is option C.\begin{enumerate}[label=\Alph*.]
\item \( f(x) = \frac{-1}{x - 2} + 3 \)

Corresponds to thinking the graph was a shifted version of $\frac{1}{x}$.
\item \( f(x) = \frac{1}{(x + 2)^2} + 3 \)

Corresponds to using the general form $f(x) = \frac{a}{(x+h)^2}+k$ and the opposite leading coefficient.
\item \( f(x) = \frac{-1}{(x - 2)^2} + 3 \)

This is the correct option.
\item \( f(x) = \frac{1}{x + 2} + 3 \)

Corresponds to thinking the graph was a shifted version of $\frac{1}{x}$, using the general form $f(x) = \frac{a}{(x+h)^2}+k$, and the opposite leading coefficient.
\item \( \text{None of the above} \)

This corresponds to believing the vertex of the graph was not correct.
\end{enumerate}

\textbf{General Comment:} Remember that the general form of a basic rational equation is $ f(x) = \frac{a}{(x-h)^n} + k$, where $a$ is the leading coefficient (and in this case, we assume is either $1$ or $-1$), $n$ is the degree (in this case, either $1$ or $2$), and $(h, k)$ is the intersection of the asymptotes.
}
\litem{
Solve the rational equation below. Then, choose the interval(s) that the solution(s) belongs to.
\[ \frac{-2x}{-4x -4} + \frac{-4x^{2}}{-20x^{2} -12 x + 8} = \frac{5}{5x -2} \]The solution is \( \text{There are two solutions: } x = -0.614 \text{ and } x = 2.328 \), which is option D.\begin{enumerate}[label=\Alph*.]
\item \( x \in [-0.13,0.5] \)


\item \( \text{All solutions lead to invalid or complex values in the equation.} \)


\item \( x_1 \in [-0.87, -0.46] \text{ and } x_2 \in [-3.5,-0.9] \)


\item \( x_1 \in [-0.87, -0.46] \text{ and } x_2 \in [1.8,3.3] \)

* $x = -0.614 \text{ and } x = 2.328$, which is the correct option.
\item \( x \in [2.16,3.28] \)


\end{enumerate}

\textbf{General Comment:} Distractors are different based on the number of solutions. Remember that after solving, we need to make sure our solution does not make the original equation divide by zero!
}
\litem{
Choose the graph of the equation below.
\[ f(x) = \frac{-1}{x + 1} + 3 \]The solution is the graph below, which is option D.
    \begin{center}
        \includegraphics[width=0.3\textwidth]{../Figures/rationalEquationToGraphDB.png}
    \end{center}\begin{enumerate}[label=\Alph*.]
\begin{multicols}{2}
\item \includegraphics[width = 0.3\textwidth]{../Figures/rationalEquationToGraphAB.png}
\item \includegraphics[width = 0.3\textwidth]{../Figures/rationalEquationToGraphBB.png}
\item \includegraphics[width = 0.3\textwidth]{../Figures/rationalEquationToGraphCB.png}
\item \includegraphics[width = 0.3\textwidth]{../Figures/rationalEquationToGraphDB.png}
\end{multicols}\item None of the above.\end{enumerate}
\textbf{General Comment:} Remember that the general form of a basic rational equation is $ f(x) = \frac{a}{(x-h)^n} + k$, where $a$ is the leading coefficient (and in this case, we assume is either $1$ or $-1$), $n$ is the degree (in this case, either $1$ or $2$), and $(h, k)$ is the intersection of the asymptotes.
}
\litem{
Choose the equation of the function graphed below.

\begin{center}
    \includegraphics[width=0.5\textwidth]{../Figures/rationalGraphToEquationCopyB.png}
\end{center}


The solution is \( f(x) = \frac{1}{(x + 1)^2} + 1 \), which is option A.\begin{enumerate}[label=\Alph*.]
\item \( f(x) = \frac{1}{(x + 1)^2} + 1 \)

This is the correct option.
\item \( f(x) = \frac{1}{x + 1} + 1 \)

Corresponds to thinking the graph was a shifted version of $\frac{1}{x}$.
\item \( f(x) = \frac{-1}{x - 1} + 1 \)

Corresponds to thinking the graph was a shifted version of $\frac{1}{x}$, using the general form $f(x) = \frac{a}{(x+h)^2}+k$, and the opposite leading coefficient.
\item \( f(x) = \frac{-1}{(x - 1)^2} + 1 \)

Corresponds to using the general form $f(x) = \frac{a}{(x+h)^2}+k$ and the opposite leading coefficient.
\item \( \text{None of the above} \)

This corresponds to believing the vertex of the graph was not correct.
\end{enumerate}

\textbf{General Comment:} Remember that the general form of a basic rational equation is $ f(x) = \frac{a}{(x-h)^n} + k$, where $a$ is the leading coefficient (and in this case, we assume is either $1$ or $-1$), $n$ is the degree (in this case, either $1$ or $2$), and $(h, k)$ is the intersection of the asymptotes.
}
\litem{
Determine the domain of the function below.
\[ f(x) = \frac{6}{9x^{2} -25} \]The solution is \( \text{All Real numbers except } x = -1.667 \text{ and } x = 1.667. \), which is option C.\begin{enumerate}[label=\Alph*.]
\item \( \text{All Real numbers except } x = a, \text{ where } a \in [-4.67, 1.33] \)

All Real numbers except $x = -1.667$, which corresponds to removing only 1 value from the denominator.
\item \( \text{All Real numbers.} \)

This corresponds to thinking the denominator has complex roots or that rational functions have a domain of all Real numbers.
\item \( \text{All Real numbers except } x = a \text{ and } x = b, \text{ where } a \in [-4.67, 1.33] \text{ and } b \in [-0.33, 3.67] \)

All Real numbers except $x = -1.667$ and $x = 1.667$, which is the correct option.
\item \( \text{All Real numbers except } x = a \text{ and } x = b, \text{ where } a \in [-16, -11] \text{ and } b \in [15, 20] \)

All Real numbers except $x = -15.000$ and $x = 15.000$, which corresponds to not factoring the denominator correctly.
\item \( \text{All Real numbers except } x = a, \text{ where } a \in [-16, -11] \)

All Real numbers except $x = -15.000$, which corresponds to removing a distractor value from the denominator.
\end{enumerate}

\textbf{General Comment:} Recall that dividing by zero is not a real number. Therefore the domain is all real numbers \textbf{except} those that make the denominator 0.
}
\end{enumerate}

\end{document}