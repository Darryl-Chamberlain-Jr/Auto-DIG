\documentclass{extbook}[14pt]
\usepackage{multicol, enumerate, enumitem, hyperref, color, soul, setspace, parskip, fancyhdr, amssymb, amsthm, amsmath, latexsym, units, mathtools}
\everymath{\displaystyle}
\usepackage[headsep=0.5cm,headheight=0cm, left=1 in,right= 1 in,top= 1 in,bottom= 1 in]{geometry}
\usepackage{dashrule}  % Package to use the command below to create lines between items
\newcommand{\litem}[1]{\item #1

\rule{\textwidth}{0.4pt}}
\pagestyle{fancy}
\lhead{}
\chead{Answer Key for Progress Quiz 9 Version B}
\rhead{}
\lfoot{9541-5764}
\cfoot{}
\rfoot{Summer C 2021}
\begin{document}
\textbf{This key should allow you to understand why you choose the option you did (beyond just getting a question right or wrong). \href{https://xronos.clas.ufl.edu/mac1105spring2020/courseDescriptionAndMisc/Exams/LearningFromResults}{More instructions on how to use this key can be found here}.}

\textbf{If you have a suggestion to make the keys better, \href{https://forms.gle/CZkbZmPbC9XALEE88}{please fill out the short survey here}.}

\textit{Note: This key is auto-generated and may contain issues and/or errors. The keys are reviewed after each exam to ensure grading is done accurately. If there are issues (like duplicate options), they are noted in the offline gradebook. The keys are a work-in-progress to give students as many resources to improve as possible.}

\rule{\textwidth}{0.4pt}

\begin{enumerate}\litem{
 Solve the equation for $x$ and choose the interval that contains $x$ (if it exists).
\[  6 = \sqrt[4]{\frac{21}{e^{8x}}} \]The solution is \( x = -0.515 \), which is option A.\begin{enumerate}[label=\Alph*.]
\item \( x \in [-0.55, -0.4] \)

* $x = -0.515$, which is the correct option.
\item \( x \in [-0.19, 0.29] \)

$x = -0.067$, which corresponds to treating any root as a square root.
\item \( x \in [-3.6, -3.21] \)

$x = -3.381$, which corresponds to thinking you don't need to take the natural log of both sides before reducing, as if the equation already had a natural log on the right side.
\item \( \text{There is no Real solution to the equation.} \)

This corresponds to believing you cannot solve the equation.
\item \( \text{None of the above.} \)

This corresponds to making an unexpected error.
\end{enumerate}

\textbf{General Comment:} \textbf{General Comments}: After using the properties of logarithmic functions to break up the right-hand side, use $\ln(e) = 1$ to reduce the question to a linear function to solve. You can put $\ln(21)$ into a calculator if you are having trouble.
}
\litem{
Solve the equation for $x$ and choose the interval that contains the solution (if it exists).
\[ 5^{-2x+2} = \left(\frac{1}{343}\right)^{-3x-5} \]The solution is \( x = -1.253 \), which is option B.\begin{enumerate}[label=\Alph*.]
\item \( x \in [-8.2, -6] \)

$x = -7.000$, which corresponds to solving the numerators as equal while ignoring the bases are different.
\item \( x \in [-1.5, -1] \)

* $x = -1.253$, which is the correct option.
\item \( x \in [24.7, 26.8] \)

$x = 25.970$, which corresponds to distributing the $\ln(base)$ to the second term of the exponent only.
\item \( x \in [-1.2, 0.6] \)

$x = 0.338$, which corresponds to distributing the $\ln(base)$ to the first term of the exponent only.
\item \( \text{There is no Real solution to the equation.} \)

This corresponds to believing there is no solution since the bases are not powers of each other.
\end{enumerate}

\textbf{General Comment:} \textbf{General Comments:} This question was written so that the bases could not be written the same. You will need to take the log of both sides.
}
\litem{
Which of the following intervals describes the Domain of the function below?
\[ f(x) = e^{x+2}+1 \]The solution is \( (-\infty, \infty) \), which is option E.\begin{enumerate}[label=\Alph*.]
\item \( [a, \infty), a \in [-2.2, -0.3] \)

$[-1, \infty)$, which corresponds to using the negative vertical shift AND flipping the Range interval AND including the endpoint.
\item \( (-\infty, a), a \in [-0.9, 1.2] \)

$(-\infty, 1)$, which corresponds to using the correct vertical shift *if we wanted the Range*.
\item \( (-\infty, a], a \in [-0.9, 1.2] \)

$(-\infty, 1]$, which corresponds to using the correct vertical shift *if we wanted the Range* AND including the endpoint.
\item \( (a, \infty), a \in [-2.2, -0.3] \)

$(-1, \infty)$, which corresponds to using the negative vertical shift AND flipping the Range interval.
\item \( (-\infty, \infty) \)

* This is the correct option.
\end{enumerate}

\textbf{General Comment:} \textbf{General Comments}: Domain of a basic exponential function is $(-\infty, \infty)$ while the Range is $(0, \infty)$. We can shift these intervals [and even flip when $a<0$!] to find the new Domain/Range.
}
\litem{
Solve the equation for $x$ and choose the interval that contains the solution (if it exists).
\[ 5^{5x-3} = 64^{2x+2} \]The solution is \( x = -48.585 \), which is option C.\begin{enumerate}[label=\Alph*.]
\item \( x \in [0.67, 3.67] \)

$x = 1.667$, which corresponds to solving the numerators as equal while ignoring the bases are different.
\item \( x \in [3.38, 6.38] \)

$x = 4.382$, which corresponds to distributing the $\ln(base)$ to the second term of the exponent only.
\item \( x \in [-55.59, -46.59] \)

* $x = -48.585$, which is the correct option.
\item \( x \in [-21.48, -16.48] \)

$x = -18.479$, which corresponds to distributing the $\ln(base)$ to the first term of the exponent only.
\item \( \text{There is no Real solution to the equation.} \)

This corresponds to believing there is no solution since the bases are not powers of each other.
\end{enumerate}

\textbf{General Comment:} \textbf{General Comments:} This question was written so that the bases could not be written the same. You will need to take the log of both sides.
}
\litem{
Which of the following intervals describes the Domain of the function below?
\[ f(x) = \log_2{(x-4)}-9 \]The solution is \( (4, \infty) \), which is option B.\begin{enumerate}[label=\Alph*.]
\item \( (-\infty, a], a \in [6, 17] \)

$(-\infty, 9]$, which corresponds to using the negative vertical shift AND including the endpoint AND flipping the domain.
\item \( (a, \infty), a \in [3, 8] \)

* $(4, \infty)$, which is the correct option.
\item \( [a, \infty), a \in [-11, -8] \)

$[-9, \infty)$, which corresponds to using the vertical shift when shifting the Domain AND including the endpoint.
\item \( (-\infty, a), a \in [-8, -2] \)

$(-\infty, -4)$, which corresponds to flipping the Domain. Remember: the general for is $a*\log(x-h)+k$, \textbf{where $a$ does not affect the domain}.
\item \( (-\infty, \infty) \)

This corresponds to thinking of the range of the log function (or the domain of the exponential function).
\end{enumerate}

\textbf{General Comment:} \textbf{General Comments}: The domain of a basic logarithmic function is $(0, \infty)$ and the Range is $(-\infty, \infty)$. We can use shifts when finding the Domain, but the Range will always be all Real numbers.
}
\litem{
Which of the following intervals describes the Domain of the function below?
\[ f(x) = \log_2{(x+8)}+8 \]The solution is \( (-8, \infty) \), which is option D.\begin{enumerate}[label=\Alph*.]
\item \( (-\infty, a], a \in [-9, -1] \)

$(-\infty, -8]$, which corresponds to using the negative vertical shift AND including the endpoint AND flipping the domain.
\item \( (-\infty, a), a \in [5, 12] \)

$(-\infty, 8)$, which corresponds to flipping the Domain. Remember: the general for is $a*\log(x-h)+k$, \textbf{where $a$ does not affect the domain}.
\item \( [a, \infty), a \in [5, 12] \)

$[8, \infty)$, which corresponds to using the vertical shift when shifting the Domain AND including the endpoint.
\item \( (a, \infty), a \in [-9, -1] \)

* $(-8, \infty)$, which is the correct option.
\item \( (-\infty, \infty) \)

This corresponds to thinking of the range of the log function (or the domain of the exponential function).
\end{enumerate}

\textbf{General Comment:} \textbf{General Comments}: The domain of a basic logarithmic function is $(0, \infty)$ and the Range is $(-\infty, \infty)$. We can use shifts when finding the Domain, but the Range will always be all Real numbers.
}
\litem{
 Solve the equation for $x$ and choose the interval that contains $x$ (if it exists).
\[  24 = \sqrt[3]{\frac{10}{e^{3x}}} \]The solution is \( x = -2.411 \), which is option B.\begin{enumerate}[label=\Alph*.]
\item \( x \in [-24.82, -24.05] \)

$x = -24.768$, which corresponds to thinking you don't need to take the natural log of both sides before reducing, as if the equation already had a natural log on the right side.
\item \( x \in [-2.95, -2.19] \)

* $x = -2.411$, which is the correct option.
\item \( x \in [-1.73, -1.18] \)

$x = -1.351$, which corresponds to treating any root as a square root.
\item \( \text{There is no Real solution to the equation.} \)

This corresponds to believing you cannot solve the equation.
\item \( \text{None of the above.} \)

This corresponds to making an unexpected error.
\end{enumerate}

\textbf{General Comment:} \textbf{General Comments}: After using the properties of logarithmic functions to break up the right-hand side, use $\ln(e) = 1$ to reduce the question to a linear function to solve. You can put $\ln(10)$ into a calculator if you are having trouble.
}
\litem{
Solve the equation for $x$ and choose the interval that contains the solution (if it exists).
\[ \log_{3}{(-2x+6)}+4 = 3 \]The solution is \( x = 2.833 \), which is option B.\begin{enumerate}[label=\Alph*.]
\item \( x \in [3.3, 3.56] \)

$x = 3.500$, which corresponds to reversing the base and exponent when converting.
\item \( x \in [2.61, 3.43] \)

* $x = 2.833$, which is the correct option.
\item \( x \in [-10.57, -9.06] \)

$x = -10.500$, which corresponds to ignoring the vertical shift when converting to exponential form.
\item \( x \in [-2.51, -2.22] \)

$x = -2.500$, which corresponds to reversing the base and exponent when converting and reversing the value with $x$.
\item \( \text{There is no Real solution to the equation.} \)

Corresponds to believing a negative coefficient within the log equation means there is no Real solution.
\end{enumerate}

\textbf{General Comment:} \textbf{General Comments:} First, get the equation in the form $\log_b{(cx+d)} = a$. Then, convert to $b^a = cx+d$ and solve.
}
\litem{
Solve the equation for $x$ and choose the interval that contains the solution (if it exists).
\[ \log_{5}{(-4x+6)}+5 = 2 \]The solution is \( x = 1.498 \), which is option A.\begin{enumerate}[label=\Alph*.]
\item \( x \in [-2.5, 6.5] \)

* $x = 1.498$, which is the correct option.
\item \( x \in [-5.75, -2.75] \)

$x = -4.750$, which corresponds to ignoring the vertical shift when converting to exponential form.
\item \( x \in [56.25, 60.25] \)

$x = 59.250$, which corresponds to reversing the base and exponent when converting and reversing the value with $x$.
\item \( x \in [61.25, 66.25] \)

$x = 62.250$, which corresponds to reversing the base and exponent when converting.
\item \( \text{There is no Real solution to the equation.} \)

Corresponds to believing a negative coefficient within the log equation means there is no Real solution.
\end{enumerate}

\textbf{General Comment:} \textbf{General Comments:} First, get the equation in the form $\log_b{(cx+d)} = a$. Then, convert to $b^a = cx+d$ and solve.
}
\litem{
Which of the following intervals describes the Range of the function below?
\[ f(x) = -e^{x-4}-8 \]The solution is \( (-\infty, -8) \), which is option A.\begin{enumerate}[label=\Alph*.]
\item \( (-\infty, a), a \in [-8, -7] \)

* $(-\infty, -8)$, which is the correct option.
\item \( (a, \infty), a \in [7, 12] \)

$(8, \infty)$, which corresponds to using the negative vertical shift AND flipping the Range interval.
\item \( (-\infty, a], a \in [-8, -7] \)

$(-\infty, -8]$, which corresponds to including the endpoint.
\item \( [a, \infty), a \in [7, 12] \)

$[8, \infty)$, which corresponds to using the negative vertical shift AND flipping the Range interval AND including the endpoint.
\item \( (-\infty, \infty) \)

This corresponds to confusing range of an exponential function with the domain of an exponential function.
\end{enumerate}

\textbf{General Comment:} \textbf{General Comments}: Domain of a basic exponential function is $(-\infty, \infty)$ while the Range is $(0, \infty)$. We can shift these intervals [and even flip when $a<0$!] to find the new Domain/Range.
}
\end{enumerate}

\end{document}