\documentclass{extbook}[14pt]
\usepackage{multicol, enumerate, enumitem, hyperref, color, soul, setspace, parskip, fancyhdr, amssymb, amsthm, amsmath, latexsym, units, mathtools}
\everymath{\displaystyle}
\usepackage[headsep=0.5cm,headheight=0cm, left=1 in,right= 1 in,top= 1 in,bottom= 1 in]{geometry}
\usepackage{dashrule}  % Package to use the command below to create lines between items
\newcommand{\litem}[1]{\item #1

\rule{\textwidth}{0.4pt}}
\pagestyle{fancy}
\lhead{}
\chead{Answer Key for Progress Quiz 9 Version C}
\rhead{}
\lfoot{9541-5764}
\cfoot{}
\rfoot{Summer C 2021}
\begin{document}
\textbf{This key should allow you to understand why you choose the option you did (beyond just getting a question right or wrong). \href{https://xronos.clas.ufl.edu/mac1105spring2020/courseDescriptionAndMisc/Exams/LearningFromResults}{More instructions on how to use this key can be found here}.}

\textbf{If you have a suggestion to make the keys better, \href{https://forms.gle/CZkbZmPbC9XALEE88}{please fill out the short survey here}.}

\textit{Note: This key is auto-generated and may contain issues and/or errors. The keys are reviewed after each exam to ensure grading is done accurately. If there are issues (like duplicate options), they are noted in the offline gradebook. The keys are a work-in-progress to give students as many resources to improve as possible.}

\rule{\textwidth}{0.4pt}

\begin{enumerate}\litem{
 Solve the equation for $x$ and choose the interval that contains $x$ (if it exists).
\[  13 = \sqrt[7]{\frac{28}{e^{5x}}} \]The solution is \( x = -2.924, \text{ which does not fit in any of the interval options.} \), which is option E.\begin{enumerate}[label=\Alph*.]
\item \( x \in [2.92, 5.92] \)

$x = 2.924$, which is the negative of the correct solution.
\item \( x \in [-21.87, -17.87] \)

$x = -18.866$, which corresponds to thinking you don't need to take the natural log of both sides before reducing, as if the right side already has a natural log.
\item \( x \in [-2.36, 0.64] \)

$x = -0.360$, which corresponds to treating any root as a square root.
\item \( \text{There is no Real solution to the equation.} \)

This corresponds to believing you cannot solve the equation.
\item \( \text{None of the above.} \)

* $x = -2.924$ is the correct solution and does not fit in any of the other intervals.
\end{enumerate}

\textbf{General Comment:} \textbf{General Comments}: After using the properties of logarithmic functions to break up the right-hand side, use $\ln(e) = 1$ to reduce the question to a linear function to solve. You can put $\ln(28)$ into a calculator if you are having trouble.
}
\litem{
Solve the equation for $x$ and choose the interval that contains the solution (if it exists).
\[ 2^{4x-2} = 9^{3x+5} \]The solution is \( x = -3.240 \), which is option A.\begin{enumerate}[label=\Alph*.]
\item \( x \in [-3.24, -2.24] \)

* $x = -3.240$, which is the correct option.
\item \( x \in [4, 11] \)

$x = 7.000$, which corresponds to solving the numerators as equal while ignoring the bases are different.
\item \( x \in [10.37, 14.37] \)

$x = 12.372$, which corresponds to distributing the $\ln(base)$ to the second term of the exponent only.
\item \( x \in [-2.83, 0.17] \)

$x = -1.833$, which corresponds to distributing the $\ln(base)$ to the first term of the exponent only.
\item \( \text{There is no Real solution to the equation.} \)

This corresponds to believing there is no solution since the bases are not powers of each other.
\end{enumerate}

\textbf{General Comment:} \textbf{General Comments:} This question was written so that the bases could not be written the same. You will need to take the log of both sides.
}
\litem{
Which of the following intervals describes the Range of the function below?
\[ f(x) = -e^{x+4}+2 \]The solution is \( (-\infty, 2) \), which is option B.\begin{enumerate}[label=\Alph*.]
\item \( [a, \infty), a \in [-5, 1] \)

$[-2, \infty)$, which corresponds to using the negative vertical shift AND flipping the Range interval AND including the endpoint.
\item \( (-\infty, a), a \in [-1, 10] \)

* $(-\infty, 2)$, which is the correct option.
\item \( (-\infty, a], a \in [-1, 10] \)

$(-\infty, 2]$, which corresponds to including the endpoint.
\item \( (a, \infty), a \in [-5, 1] \)

$(-2, \infty)$, which corresponds to using the negative vertical shift AND flipping the Range interval.
\item \( (-\infty, \infty) \)

This corresponds to confusing range of an exponential function with the domain of an exponential function.
\end{enumerate}

\textbf{General Comment:} \textbf{General Comments}: Domain of a basic exponential function is $(-\infty, \infty)$ while the Range is $(0, \infty)$. We can shift these intervals [and even flip when $a<0$!] to find the new Domain/Range.
}
\litem{
Solve the equation for $x$ and choose the interval that contains the solution (if it exists).
\[ 2^{4x+2} = 343^{2x+5} \]The solution is \( x = -3.123 \), which is option A.\begin{enumerate}[label=\Alph*.]
\item \( x \in [-4.5, -2.9] \)

* $x = -3.123$, which is the correct option.
\item \( x \in [12.5, 15.1] \)

$x = 13.901$, which corresponds to distributing the $\ln(base)$ to the second term of the exponent only.
\item \( x \in [-1, 0.6] \)

$x = -0.337$, which corresponds to distributing the $\ln(base)$ to the first term of the exponent only.
\item \( x \in [1.4, 3] \)

$x = 1.500$, which corresponds to solving the numerators as equal while ignoring the bases are different.
\item \( \text{There is no Real solution to the equation.} \)

This corresponds to believing there is no solution since the bases are not powers of each other.
\end{enumerate}

\textbf{General Comment:} \textbf{General Comments:} This question was written so that the bases could not be written the same. You will need to take the log of both sides.
}
\litem{
Which of the following intervals describes the Range of the function below?
\[ f(x) = \log_2{(x+3)}-8 \]The solution is \( (\infty, \infty) \), which is option E.\begin{enumerate}[label=\Alph*.]
\item \( (-\infty, a), a \in [-10, -6] \)

$(-\infty, -8)$, which corresponds to using the vertical shift while the Range is $(-\infty, \infty)$.
\item \( (-\infty, a), a \in [7, 11] \)

$(-\infty, 8)$, which corresponds to using the using the negative of vertical shift on $(0, \infty)$.
\item \( [a, \infty), a \in [-6, -2] \)

$[-8, \infty)$, which corresponds to using the flipped Domain AND including the endpoint.
\item \( [a, \infty), a \in [3, 5] \)

$[3, \infty)$, which corresponds to using the negative of the horizontal shift AND including the endpoint.
\item \( (-\infty, \infty) \)

*This is the correct option.
\end{enumerate}

\textbf{General Comment:} \textbf{General Comments}: The domain of a basic logarithmic function is $(0, \infty)$ and the Range is $(-\infty, \infty)$. We can use shifts when finding the Domain, but the Range will always be all Real numbers.
}
\litem{
Which of the following intervals describes the Domain of the function below?
\[ f(x) = -\log_2{(x+8)}+4 \]The solution is \( (-8, \infty) \), which is option B.\begin{enumerate}[label=\Alph*.]
\item \( (-\infty, a), a \in [5.8, 8.9] \)

$(-\infty, 8)$, which corresponds to flipping the Domain. Remember: the general for is $a*\log(x-h)+k$, \textbf{where $a$ does not affect the domain}.
\item \( (a, \infty), a \in [-9.3, -6] \)

* $(-8, \infty)$, which is the correct option.
\item \( [a, \infty), a \in [3.8, 7.1] \)

$[4, \infty)$, which corresponds to using the vertical shift when shifting the Domain AND including the endpoint.
\item \( (-\infty, a], a \in [-6.6, -2.7] \)

$(-\infty, -4]$, which corresponds to using the negative vertical shift AND including the endpoint AND flipping the domain.
\item \( (-\infty, \infty) \)

This corresponds to thinking of the range of the log function (or the domain of the exponential function).
\end{enumerate}

\textbf{General Comment:} \textbf{General Comments}: The domain of a basic logarithmic function is $(0, \infty)$ and the Range is $(-\infty, \infty)$. We can use shifts when finding the Domain, but the Range will always be all Real numbers.
}
\litem{
 Solve the equation for $x$ and choose the interval that contains $x$ (if it exists).
\[  12 = \sqrt[5]{\frac{27}{e^{3x}}} \]The solution is \( x = -3.043 \), which is option B.\begin{enumerate}[label=\Alph*.]
\item \( x \in [-24.1, -19.1] \)

$x = -21.099$, which corresponds to thinking you don't need to take the natural log of both sides before reducing, as if the equation already had a natural log on the right side.
\item \( x \in [-6.04, -2.04] \)

* $x = -3.043$, which is the correct option.
\item \( x \in [-1.56, 0.44] \)

$x = -0.558$, which corresponds to treating any root as a square root.
\item \( \text{There is no Real solution to the equation.} \)

This corresponds to believing you cannot solve the equation.
\item \( \text{None of the above.} \)

This corresponds to making an unexpected error.
\end{enumerate}

\textbf{General Comment:} \textbf{General Comments}: After using the properties of logarithmic functions to break up the right-hand side, use $\ln(e) = 1$ to reduce the question to a linear function to solve. You can put $\ln(27)$ into a calculator if you are having trouble.
}
\litem{
Solve the equation for $x$ and choose the interval that contains the solution (if it exists).
\[ \log_{5}{(2x+8)}+5 = 2 \]The solution is \( x = -3.996 \), which is option A.\begin{enumerate}[label=\Alph*.]
\item \( x \in [-4, -3] \)

* $x = -3.996$, which is the correct option.
\item \( x \in [5.5, 12.5] \)

$x = 8.500$, which corresponds to ignoring the vertical shift when converting to exponential form.
\item \( x \in [-122.5, -114.5] \)

$x = -117.500$, which corresponds to reversing the base and exponent when converting and reversing the value with $x$.
\item \( x \in [-129.5, -123.5] \)

$x = -125.500$, which corresponds to reversing the base and exponent when converting.
\item \( \text{There is no Real solution to the equation.} \)

Corresponds to believing a negative coefficient within the log equation means there is no Real solution.
\end{enumerate}

\textbf{General Comment:} \textbf{General Comments:} First, get the equation in the form $\log_b{(cx+d)} = a$. Then, convert to $b^a = cx+d$ and solve.
}
\litem{
Solve the equation for $x$ and choose the interval that contains the solution (if it exists).
\[ \log_{4}{(2x+7)}+4 = 2 \]The solution is \( x = -3.469 \), which is option C.\begin{enumerate}[label=\Alph*.]
\item \( x \in [0.5, 7.5] \)

$x = 4.500$, which corresponds to ignoring the vertical shift when converting to exponential form.
\item \( x \in [0.5, 7.5] \)

$x = 4.500$, which corresponds to reversing the base and exponent when converting.
\item \( x \in [-7.47, -2.47] \)

* $x = -3.469$, which is the correct option.
\item \( x \in [11.5, 19.5] \)

$x = 11.500$, which corresponds to reversing the base and exponent when converting and reversing the value with $x$.
\item \( \text{There is no Real solution to the equation.} \)

Corresponds to believing a negative coefficient within the log equation means there is no Real solution.
\end{enumerate}

\textbf{General Comment:} \textbf{General Comments:} First, get the equation in the form $\log_b{(cx+d)} = a$. Then, convert to $b^a = cx+d$ and solve.
}
\litem{
Which of the following intervals describes the Domain of the function below?
\[ f(x) = -e^{x+7}+7 \]The solution is \( (-\infty, \infty) \), which is option E.\begin{enumerate}[label=\Alph*.]
\item \( (a, \infty), a \in [-11, 0] \)

$(-7, \infty)$, which corresponds to using the negative vertical shift AND flipping the Range interval.
\item \( (-\infty, a], a \in [6, 9] \)

$(-\infty, 7]$, which corresponds to using the correct vertical shift *if we wanted the Range* AND including the endpoint.
\item \( [a, \infty), a \in [-11, 0] \)

$[-7, \infty)$, which corresponds to using the negative vertical shift AND flipping the Range interval AND including the endpoint.
\item \( (-\infty, a), a \in [6, 9] \)

$(-\infty, 7)$, which corresponds to using the correct vertical shift *if we wanted the Range*.
\item \( (-\infty, \infty) \)

* This is the correct option.
\end{enumerate}

\textbf{General Comment:} \textbf{General Comments}: Domain of a basic exponential function is $(-\infty, \infty)$ while the Range is $(0, \infty)$. We can shift these intervals [and even flip when $a<0$!] to find the new Domain/Range.
}
\end{enumerate}

\end{document}