\documentclass[14pt]{extbook}
\usepackage{multicol, enumerate, enumitem, hyperref, color, soul, setspace, parskip, fancyhdr} %General Packages
\usepackage{amssymb, amsthm, amsmath, latexsym, units, mathtools} %Math Packages
\everymath{\displaystyle} %All math in Display Style
% Packages with additional options
\usepackage[headsep=0.5cm,headheight=12pt, left=1 in,right= 1 in,top= 1 in,bottom= 1 in]{geometry}
\usepackage[usenames,dvipsnames]{xcolor}
\usepackage{dashrule}  % Package to use the command below to create lines between items
\newcommand{\litem}[1]{\item#1\hspace*{-1cm}\rule{\textwidth}{0.4pt}}
\pagestyle{fancy}
\lhead{Progress Quiz 9}
\chead{}
\rhead{Version C}
\lfoot{9541-5764}
\cfoot{}
\rfoot{Summer C 2021}
\begin{document}

\begin{enumerate}
\litem{
Choose the equation of the function graphed below.
\begin{center}
    \includegraphics[width=0.5\textwidth]{../Figures/radicalGraphToEquationC.png}
\end{center}
\begin{enumerate}[label=\Alph*.]
\item \( f(x) = - \sqrt[3]{x + 14} - 3 \)
\item \( f(x) = \sqrt[3]{x - 14} - 3 \)
\item \( f(x) = - \sqrt[3]{x - 14} - 3 \)
\item \( f(x) = \sqrt[3]{x + 14} - 3 \)
\item \( \text{None of the above} \)

\end{enumerate} }
\litem{
Choose the equation of the function graphed below.
\begin{center}
    \includegraphics[width=0.5\textwidth]{../Figures/radicalGraphToEquationCopyC.png}
\end{center}
\begin{enumerate}[label=\Alph*.]
\item \( f(x) = - \sqrt[3]{x + 14} - 4 \)
\item \( f(x) = - \sqrt[3]{x - 14} - 4 \)
\item \( f(x) = \sqrt[3]{x + 14} - 4 \)
\item \( f(x) = \sqrt[3]{x - 14} - 4 \)
\item \( \text{None of the above} \)

\end{enumerate} }
\litem{
Solve the radical equation below. Then, choose the interval(s) that the solution(s) belongs to.\[ \sqrt{-15 x^2 + 63} - \sqrt{24 x} = 0 \]\begin{enumerate}[label=\Alph*.]
\item \( x_1 \in [1.4, 4.4] \text{ and } x_2 \in [1.6,3.4] \)
\item \( x \in [-3,0] \)
\item \( x_1 \in [-3, 0] \text{ and } x_2 \in [-0.5,2.2] \)
\item \( \text{All solutions lead to invalid or complex values in the equation.} \)
\item \( x \in [1.4,4.4] \)

\end{enumerate} }
\litem{
Solve the radical equation below. Then, choose the interval(s) that the solution(s) belongs to.\[ \sqrt{-5 x + 2} - \sqrt{8 x + 8} = 0 \]\begin{enumerate}[label=\Alph*.]
\item \( x_1 \in [-0.55, -0.31] \text{ and } x_2 \in [0.4,7.4] \)
\item \( x \in [0,1.15] \)
\item \( \text{All solutions lead to invalid or complex values in the equation.} \)
\item \( x_1 \in [-1.33, -0.99] \text{ and } x_2 \in [0.4,7.4] \)
\item \( x \in [-0.55,-0.31] \)

\end{enumerate} }
\litem{
What is the domain of the function below?\[ f(x) = \sqrt[5]{-6 x + 7} \]\begin{enumerate}[label=\Alph*.]
\item \( (-\infty, \infty) \)
\item \( \text{The domain is } (-\infty, a], \text{   where } a \in [1.02, 1.48] \)
\item \( \text{The domain is } (-\infty, a], \text{   where } a \in [0.74, 0.96] \)
\item \( \text{The domain is } [a, \infty), \text{   where } a \in [0.21, 1.13] \)
\item \( \text{The domain is } [a, \infty), \text{   where } a \in [1.03, 1.43] \)

\end{enumerate} }
\litem{
Choose the graph of the equation below.\[ f(x) = \sqrt[3]{x + 12} - 7 \]\begin{enumerate}[label=\Alph*.]
\begin{multicols}{2}\item \includegraphics[width = 0.3\textwidth]{../Figures/radicalEquationToGraphAC.png}\item \includegraphics[width = 0.3\textwidth]{../Figures/radicalEquationToGraphBC.png}\item \includegraphics[width = 0.3\textwidth]{../Figures/radicalEquationToGraphCC.png}\item \includegraphics[width = 0.3\textwidth]{../Figures/radicalEquationToGraphDC.png}\end{multicols}\item None of the above.
\end{enumerate} }
\litem{
What is the domain of the function below?\[ f(x) = \sqrt[7]{-9 x + 3} \]\begin{enumerate}[label=\Alph*.]
\item \( \text{The domain is } [a, \infty), \text{   where } a \in [2, 5] \)
\item \( \text{The domain is } (-\infty, a], \text{   where } a \in [0.3, 1.1] \)
\item \( \text{The domain is } (-\infty, a], \text{   where } a \in [2.8, 4.1] \)
\item \( \text{The domain is } [a, \infty), \text{   where } a \in [-5.67, 2.33] \)
\item \( (-\infty, \infty) \)

\end{enumerate} }
\litem{
Solve the radical equation below. Then, choose the interval(s) that the solution(s) belongs to.\[ \sqrt{-36 x^2 - 10} - \sqrt{-42 x} = 0 \]\begin{enumerate}[label=\Alph*.]
\item \( x_1 \in [0.15, 0.65] \text{ and } x_2 \in [0.6,1.6] \)
\item \( x \in [0.7,0.85] \)
\item \( x \in [0.15,0.65] \)
\item \( \text{All solutions lead to invalid or complex values in the equation.} \)
\item \( x_1 \in [-0.47, -0.2] \text{ and } x_2 \in [-2.6,-0.7] \)

\end{enumerate} }
\litem{
Choose the graph of the equation below.\[ f(x) = \sqrt{x - 14} + 7 \]\begin{enumerate}[label=\Alph*.]
\begin{multicols}{2}\item \includegraphics[width = 0.3\textwidth]{../Figures/radicalEquationToGraphCopyAC.png}\item \includegraphics[width = 0.3\textwidth]{../Figures/radicalEquationToGraphCopyBC.png}\item \includegraphics[width = 0.3\textwidth]{../Figures/radicalEquationToGraphCopyCC.png}\item \includegraphics[width = 0.3\textwidth]{../Figures/radicalEquationToGraphCopyDC.png}\end{multicols}\item None of the above.
\end{enumerate} }
\litem{
Solve the radical equation below. Then, choose the interval(s) that the solution(s) belongs to.\[ \sqrt{9 x - 9} - \sqrt{-5 x - 3} = 0 \]\begin{enumerate}[label=\Alph*.]
\item \( x_1 \in [-0.87, 0.4] \text{ and } x_2 \in [1,5] \)
\item \( x \in [-0.11,0.48] \)
\item \( \text{All solutions lead to invalid or complex values in the equation.} \)
\item \( x \in [0.78,1.87] \)
\item \( x_1 \in [-0.11, 0.48] \text{ and } x_2 \in [1,5] \)

\end{enumerate} }
\end{enumerate}

\end{document}