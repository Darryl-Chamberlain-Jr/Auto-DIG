\documentclass[14pt]{extbook}
\usepackage{multicol, enumerate, enumitem, hyperref, color, soul, setspace, parskip, fancyhdr} %General Packages
\usepackage{amssymb, amsthm, amsmath, latexsym, units, mathtools} %Math Packages
\everymath{\displaystyle} %All math in Display Style
% Packages with additional options
\usepackage[headsep=0.5cm,headheight=12pt, left=1 in,right= 1 in,top= 1 in,bottom= 1 in]{geometry}
\usepackage[usenames,dvipsnames]{xcolor}
\usepackage{dashrule}  % Package to use the command below to create lines between items
\newcommand{\litem}[1]{\item#1\hspace*{-1cm}\rule{\textwidth}{0.4pt}}
\pagestyle{fancy}
\lhead{Progress Quiz 9}
\chead{}
\rhead{Version B}
\lfoot{9541-5764}
\cfoot{}
\rfoot{Summer C 2021}
\begin{document}

\begin{enumerate}
\litem{
Find the inverse of the function below. Then, evaluate the inverse at $x = 8$ and choose the interval that $f^-1(8)$ belongs to.\[ f(x) = \ln{(x-4)}+2 \]\begin{enumerate}[label=\Alph*.]
\item \( f^{-1}(8) \in [50.6, 61.6] \)
\item \( f^{-1}(8) \in [162749.79, 162758.79] \)
\item \( f^{-1}(8) \in [406.43, 412.43] \)
\item \( f^{-1}(8) \in [22026.47, 22035.47] \)
\item \( f^{-1}(8) \in [398.43, 402.43] \)

\end{enumerate} }
\litem{
Multiply the following functions, then choose the domain of the resulting function from the list below.\[ f(x) = 9x^{3} +7 x^{2} +8 x + 4 \text{ and } g(x) = 7x^{3} +4 x^{2} +9 x + 7 \]\begin{enumerate}[label=\Alph*.]
\item \( \text{ The domain is all Real numbers greater than or equal to } x = a, \text{ where } a \in [-7.5, -5.5] \)
\item \( \text{ The domain is all Real numbers less than or equal to } x = a, \text{ where } a \in [3.4, 6.4] \)
\item \( \text{ The domain is all Real numbers except } x = a, \text{ where } a \in [2.8, 9.8] \)
\item \( \text{ The domain is all Real numbers except } x = a \text{ and } x = b, \text{ where } a \in [6.25, 10.25] \text{ and } b \in [5.8, 8.8] \)
\item \( \text{ The domain is all Real numbers. } \)

\end{enumerate} }
\litem{
Find the inverse of the function below (if it exists). Then, evaluate the inverse at $x = 11$ and choose the interval that $f^-1(11)$ belongs to.\[ f(x) = 5 x^2 + 3 \]\begin{enumerate}[label=\Alph*.]
\item \( f^{-1}(11) \in [1.47, 1.75] \)
\item \( f^{-1}(11) \in [5.09, 5.31] \)
\item \( f^{-1}(11) \in [2.19, 2.55] \)
\item \( f^{-1}(11) \in [1.13, 1.27] \)
\item \( \text{ The function is not invertible for all Real numbers. } \)

\end{enumerate} }
\litem{
Multiply the following functions, then choose the domain of the resulting function from the list below.\[ f(x) = \sqrt{-4x+14}  \text{ and } g(x) = 8x^{2} +8 x + 5 \]\begin{enumerate}[label=\Alph*.]
\item \( \text{ The domain is all Real numbers except } x = a, \text{ where } a \in [4.33, 13.33] \)
\item \( \text{ The domain is all Real numbers greater than or equal to } x = a, \text{ where } a \in [-7.25, 1.75] \)
\item \( \text{ The domain is all Real numbers less than or equal to } x = a, \text{ where } a \in [-2.5, 8.5] \)
\item \( \text{ The domain is all Real numbers except } x = a \text{ and } x = b, \text{ where } a \in [-0.67, 5.33] \text{ and } b \in [5.4, 10.4] \)
\item \( \text{ The domain is all Real numbers. } \)

\end{enumerate} }
\litem{
Choose the interval below that $f$ composed with $g$ at $x=1$ is in.\[ f(x) = -4x^{3} +3 x^{2} +x -2 \text{ and } g(x) = -x^{3} -2 x^{2} +3 x \]\begin{enumerate}[label=\Alph*.]
\item \( (f \circ g)(1) \in [-7.5, -5.4] \)
\item \( (f \circ g)(1) \in [-1.8, -0.2] \)
\item \( (f \circ g)(1) \in [-3.8, -1.6] \)
\item \( (f \circ g)(1) \in [-10, -7.5] \)
\item \( \text{It is not possible to compose the two functions.} \)

\end{enumerate} }
\litem{
Choose the interval below that $f$ composed with $g$ at $x=1$ is in.\[ f(x) = -2x^{3} +4 x^{2} -4 x \text{ and } g(x) = -x^{3} +2 x^{2} -x + 3 \]\begin{enumerate}[label=\Alph*.]
\item \( (f \circ g)(1) \in [27, 35] \)
\item \( (f \circ g)(1) \in [-35, -28] \)
\item \( (f \circ g)(1) \in [20, 23] \)
\item \( (f \circ g)(1) \in [-23, -22] \)
\item \( \text{It is not possible to compose the two functions.} \)

\end{enumerate} }
\litem{
Determine whether the function below is 1-1.\[ f(x) = -12 x^2 - 99 x - 195 \]\begin{enumerate}[label=\Alph*.]
\item \( \text{No, because the domain of the function is not $(-\infty, \infty)$.} \)
\item \( \text{No, because there is an $x$-value that goes to 2 different $y$-values.} \)
\item \( \text{No, because the range of the function is not $(-\infty, \infty)$.} \)
\item \( \text{No, because there is a $y$-value that goes to 2 different $x$-values.} \)
\item \( \text{Yes, the function is 1-1.} \)

\end{enumerate} }
\litem{
Determine whether the function below is 1-1.\[ f(x) = (5 x - 36)^3 \]\begin{enumerate}[label=\Alph*.]
\item \( \text{No, because there is an $x$-value that goes to 2 different $y$-values.} \)
\item \( \text{No, because the domain of the function is not $(-\infty, \infty)$.} \)
\item \( \text{Yes, the function is 1-1.} \)
\item \( \text{No, because the range of the function is not $(-\infty, \infty)$.} \)
\item \( \text{No, because there is a $y$-value that goes to 2 different $x$-values.} \)

\end{enumerate} }
\litem{
Find the inverse of the function below. Then, evaluate the inverse at $x = 7$ and choose the interval that $f^-1(7)$ belongs to.\[ f(x) = \ln{(x-5)}-4 \]\begin{enumerate}[label=\Alph*.]
\item \( f^{-1}(7) \in [59862.14, 59872.14] \)
\item \( f^{-1}(7) \in [24.09, 28.09] \)
\item \( f^{-1}(7) \in [162746.79, 162754.79] \)
\item \( f^{-1}(7) \in [0.39, 10.39] \)
\item \( f^{-1}(7) \in [59879.14, 59883.14] \)

\end{enumerate} }
\litem{
Find the inverse of the function below (if it exists). Then, evaluate the inverse at $x = -14$ and choose the interval that $f^-1(-14)$ belongs to.\[ f(x) = 4 x^2 + 3 \]\begin{enumerate}[label=\Alph*.]
\item \( f^{-1}(-14) \in [1.53, 1.84] \)
\item \( f^{-1}(-14) \in [2.88, 3.62] \)
\item \( f^{-1}(-14) \in [1.89, 2.14] \)
\item \( f^{-1}(-14) \in [3.81, 4.19] \)
\item \( \text{ The function is not invertible for all Real numbers. } \)

\end{enumerate} }
\end{enumerate}

\end{document}