\documentclass{extbook}[14pt]
\usepackage{multicol, enumerate, enumitem, hyperref, color, soul, setspace, parskip, fancyhdr, amssymb, amsthm, amsmath, latexsym, units, mathtools}
\everymath{\displaystyle}
\usepackage[headsep=0.5cm,headheight=0cm, left=1 in,right= 1 in,top= 1 in,bottom= 1 in]{geometry}
\usepackage{dashrule}  % Package to use the command below to create lines between items
\newcommand{\litem}[1]{\item #1

\rule{\textwidth}{0.4pt}}
\pagestyle{fancy}
\lhead{}
\chead{Answer Key for Progress Quiz 9 Version C}
\rhead{}
\lfoot{9541-5764}
\cfoot{}
\rfoot{Summer C 2021}
\begin{document}
\textbf{This key should allow you to understand why you choose the option you did (beyond just getting a question right or wrong). \href{https://xronos.clas.ufl.edu/mac1105spring2020/courseDescriptionAndMisc/Exams/LearningFromResults}{More instructions on how to use this key can be found here}.}

\textbf{If you have a suggestion to make the keys better, \href{https://forms.gle/CZkbZmPbC9XALEE88}{please fill out the short survey here}.}

\textit{Note: This key is auto-generated and may contain issues and/or errors. The keys are reviewed after each exam to ensure grading is done accurately. If there are issues (like duplicate options), they are noted in the offline gradebook. The keys are a work-in-progress to give students as many resources to improve as possible.}

\rule{\textwidth}{0.4pt}

\begin{enumerate}\litem{
What are the \textit{possible Rational} roots of the polynomial below?
\[ f(x) = 4x^{4} +6 x^{3} +5 x^{2} +4 x + 3 \]The solution is \( \text{ All combinations of: }\frac{\pm 1,\pm 3}{\pm 1,\pm 2,\pm 4} \), which is option A.\begin{enumerate}[label=\Alph*.]
\item \( \text{ All combinations of: }\frac{\pm 1,\pm 3}{\pm 1,\pm 2,\pm 4} \)

* This is the solution \textbf{since we asked for the possible Rational roots}!
\item \( \pm 1,\pm 3 \)

This would have been the solution \textbf{if asked for the possible Integer roots}!
\item \( \text{ All combinations of: }\frac{\pm 1,\pm 2,\pm 4}{\pm 1,\pm 3} \)

 Distractor 3: Corresponds to the plus or minus of the inverse quotient (an/a0) of the factors. 
\item \( \pm 1,\pm 2,\pm 4 \)

 Distractor 1: Corresponds to the plus or minus factors of a1 only.
\item \( \text{ There is no formula or theorem that tells us all possible Rational roots.} \)

 Distractor 4: Corresponds to not recalling the theorem for rational roots of a polynomial.
\end{enumerate}

\textbf{General Comment:} We have a way to find the possible Rational roots. The possible Integer roots are the Integers in this list.
}
\litem{
Perform the division below. Then, find the intervals that correspond to the quotient in the form $ax^2+bx+c$ and remainder $r$.
\[ \frac{20x^{3} -60 x + 44}{x + 2} \]The solution is \( 20x^{2} -40 x + 20 + \frac{4}{x + 2} \), which is option B.\begin{enumerate}[label=\Alph*.]
\item \( a \in [19, 22], b \in [-62, -58], c \in [115, 129], \text{ and } r \in [-316, -315]. \)

 You multipled by the synthetic number and subtracted rather than adding during synthetic division.
\item \( a \in [19, 22], b \in [-41, -38], c \in [16, 27], \text{ and } r \in [2, 5]. \)

* This is the solution!
\item \( a \in [-41, -34], b \in [-82, -79], c \in [-222, -212], \text{ and } r \in [-396, -391]. \)

 You divided by the opposite of the factor AND multipled the first factor rather than just bringing it down.
\item \( a \in [-41, -34], b \in [74, 87], c \in [-222, -212], \text{ and } r \in [478, 485]. \)

 You multipled by the synthetic number rather than bringing the first factor down.
\item \( a \in [19, 22], b \in [37, 41], c \in [16, 27], \text{ and } r \in [81, 89]. \)

 You divided by the opposite of the factor.
\end{enumerate}

\textbf{General Comment:} Be sure to synthetically divide by the zero of the denominator! Also, make sure to include 0 placeholders for missing terms.
}
\litem{
Factor the polynomial below completely, knowing that $x + 5$ is a factor. Then, choose the intervals the zeros of the polynomial belong to, where $z_1 \leq z_2 \leq z_3 \leq z_4$. \textit{To make the problem easier, all zeros are between -5 and 5.}
\[ f(x) = 15x^{4} +151 x^{3} +429 x^{2} +185 x -300 \]The solution is \( [-5, -4, -1.667, 0.6] \), which is option B.\begin{enumerate}[label=\Alph*.]
\item \( z_1 \in [-0.52, 0.32], \text{   }  z_2 \in [3.14, 4.93], z_3 \in [4.85, 5.74], \text{   and   } z_4 \in [4.91, 6.54] \)

 Distractor 4: Corresponds to moving factors from one rational to another.
\item \( z_1 \in [-5.08, -4.82], \text{   }  z_2 \in [-4.33, -2.95], z_3 \in [-1.8, -1.17], \text{   and   } z_4 \in [0.44, 1.16] \)

* This is the solution!
\item \( z_1 \in [-1.83, -1.54], \text{   }  z_2 \in [0.34, 0.99], z_3 \in [3.09, 4.44], \text{   and   } z_4 \in [4.91, 6.54] \)

 Distractor 3: Corresponds to negatives of all zeros AND inversing rational roots.
\item \( z_1 \in [-5.08, -4.82], \text{   }  z_2 \in [-4.33, -2.95], z_3 \in [-0.77, -0.17], \text{   and   } z_4 \in [0.89, 2.04] \)

 Distractor 2: Corresponds to inversing rational roots.
\item \( z_1 \in [-1.07, -0.25], \text{   }  z_2 \in [1.41, 2.22], z_3 \in [3.09, 4.44], \text{   and   } z_4 \in [4.91, 6.54] \)

 Distractor 1: Corresponds to negatives of all zeros.
\end{enumerate}

\textbf{General Comment:} Remember to try the middle-most integers first as these normally are the zeros. Also, once you get it to a quadratic, you can use your other factoring techniques to finish factoring.
}
\litem{
Factor the polynomial below completely. Then, choose the intervals the zeros of the polynomial belong to, where $z_1 \leq z_2 \leq z_3$. \textit{To make the problem easier, all zeros are between -5 and 5.}
\[ f(x) = 12x^{3} +11 x^{2} -45 x -50 \]The solution is \( [-1.67, -1.25, 2] \), which is option B.\begin{enumerate}[label=\Alph*.]
\item \( z_1 \in [-0.89, -0.61], \text{   }  z_2 \in [-0.68, -0.46], \text{   and   } z_3 \in [1.96, 2.53] \)

 Distractor 2: Corresponds to inversing rational roots.
\item \( z_1 \in [-1.84, -1.27], \text{   }  z_2 \in [-1.31, -1.2], \text{   and   } z_3 \in [1.96, 2.53] \)

* This is the solution!
\item \( z_1 \in [-2.26, -1.88], \text{   }  z_2 \in [0.35, 0.54], \text{   and   } z_3 \in [4.65, 5.07] \)

 Distractor 4: Corresponds to moving factors from one rational to another.
\item \( z_1 \in [-2.26, -1.88], \text{   }  z_2 \in [0.45, 0.67], \text{   and   } z_3 \in [0.65, 0.85] \)

 Distractor 3: Corresponds to negatives of all zeros AND inversing rational roots.
\item \( z_1 \in [-2.26, -1.88], \text{   }  z_2 \in [1.12, 1.42], \text{   and   } z_3 \in [1.03, 1.93] \)

 Distractor 1: Corresponds to negatives of all zeros.
\end{enumerate}

\textbf{General Comment:} Remember to try the middle-most integers first as these normally are the zeros. Also, once you get it to a quadratic, you can use your other factoring techniques to finish factoring.
}
\litem{
Factor the polynomial below completely. Then, choose the intervals the zeros of the polynomial belong to, where $z_1 \leq z_2 \leq z_3$. \textit{To make the problem easier, all zeros are between -5 and 5.}
\[ f(x) = 10x^{3} + x^{2} -77 x + 30 \]The solution is \( [-3, 0.4, 2.5] \), which is option B.\begin{enumerate}[label=\Alph*.]
\item \( z_1 \in [-2.06, -1.86], \text{   }  z_2 \in [-0.7, -0.5], \text{   and   } z_3 \in [2.63, 3.17] \)

 Distractor 4: Corresponds to moving factors from one rational to another.
\item \( z_1 \in [-3.19, -2.73], \text{   }  z_2 \in [0.32, 0.41], \text{   and   } z_3 \in [2.14, 2.85] \)

* This is the solution!
\item \( z_1 \in [-3.19, -2.73], \text{   }  z_2 \in [0.32, 0.41], \text{   and   } z_3 \in [2.14, 2.85] \)

 Distractor 2: Corresponds to inversing rational roots.
\item \( z_1 \in [-2.57, -2.12], \text{   }  z_2 \in [-0.48, -0.3], \text{   and   } z_3 \in [2.63, 3.17] \)

 Distractor 3: Corresponds to negatives of all zeros AND inversing rational roots.
\item \( z_1 \in [-2.57, -2.12], \text{   }  z_2 \in [-0.48, -0.3], \text{   and   } z_3 \in [2.63, 3.17] \)

 Distractor 1: Corresponds to negatives of all zeros.
\end{enumerate}

\textbf{General Comment:} Remember to try the middle-most integers first as these normally are the zeros. Also, once you get it to a quadratic, you can use your other factoring techniques to finish factoring.
}
\litem{
Perform the division below. Then, find the intervals that correspond to the quotient in the form $ax^2+bx+c$ and remainder $r$.
\[ \frac{15x^{3} +65 x^{2} +90 x + 37}{x + 2} \]The solution is \( 15x^{2} +35 x + 20 + \frac{-3}{x + 2} \), which is option B.\begin{enumerate}[label=\Alph*.]
\item \( a \in [15, 20], \text{   } b \in [19, 21], \text{   } c \in [30, 31], \text{   and   } r \in [-56, -46]. \)

 You multiplied by the synthetic number and subtracted rather than adding during synthetic division.
\item \( a \in [15, 20], \text{   } b \in [35, 38], \text{   } c \in [16, 22], \text{   and   } r \in [-4, -2]. \)

* This is the solution!
\item \( a \in [15, 20], \text{   } b \in [90, 97], \text{   } c \in [280, 281], \text{   and   } r \in [588, 607]. \)

 You divided by the opposite of the factor.
\item \( a \in [-32, -28], \text{   } b \in [124, 127], \text{   } c \in [-163, -157], \text{   and   } r \in [356, 359]. \)

 You multiplied by the synthetic number rather than bringing the first factor down.
\item \( a \in [-32, -28], \text{   } b \in [4, 6], \text{   } c \in [96, 103], \text{   and   } r \in [227, 239]. \)

 You divided by the opposite of the factor AND multiplied the first factor rather than just bringing it down.
\end{enumerate}

\textbf{General Comment:} Be sure to synthetically divide by the zero of the denominator!
}
\litem{
Perform the division below. Then, find the intervals that correspond to the quotient in the form $ax^2+bx+c$ and remainder $r$.
\[ \frac{6x^{3} +27 x^{2} +39 x + 23}{x + 2} \]The solution is \( 6x^{2} +15 x + 9 + \frac{5}{x + 2} \), which is option E.\begin{enumerate}[label=\Alph*.]
\item \( a \in [-14, -8], \text{   } b \in [46, 55], \text{   } c \in [-64, -57], \text{   and   } r \in [149, 153]. \)

 You multiplied by the synthetic number rather than bringing the first factor down.
\item \( a \in [1, 10], \text{   } b \in [39, 40], \text{   } c \in [116, 119], \text{   and   } r \in [253, 263]. \)

 You divided by the opposite of the factor.
\item \( a \in [-14, -8], \text{   } b \in [3, 5], \text{   } c \in [41, 48], \text{   and   } r \in [111, 118]. \)

 You divided by the opposite of the factor AND multiplied the first factor rather than just bringing it down.
\item \( a \in [1, 10], \text{   } b \in [9, 13], \text{   } c \in [12, 14], \text{   and   } r \in [-14, -7]. \)

 You multiplied by the synthetic number and subtracted rather than adding during synthetic division.
\item \( a \in [1, 10], \text{   } b \in [15, 24], \text{   } c \in [3, 10], \text{   and   } r \in [2, 12]. \)

* This is the solution!
\end{enumerate}

\textbf{General Comment:} Be sure to synthetically divide by the zero of the denominator!
}
\litem{
Perform the division below. Then, find the intervals that correspond to the quotient in the form $ax^2+bx+c$ and remainder $r$.
\[ \frac{12x^{3} -65 x^{2} + 120}{x -5} \]The solution is \( 12x^{2} -5 x -25 + \frac{-5}{x -5} \), which is option A.\begin{enumerate}[label=\Alph*.]
\item \( a \in [11, 16], b \in [-8, -1], c \in [-27, -21], \text{ and } r \in [-7, -1]. \)

* This is the solution!
\item \( a \in [11, 16], b \in [-125, -124], c \in [617, 628], \text{ and } r \in [-3013, -3003]. \)

 You divided by the opposite of the factor.
\item \( a \in [60, 65], b \in [-369, -364], c \in [1819, 1828], \text{ and } r \in [-9010, -9000]. \)

 You divided by the opposite of the factor AND multipled the first factor rather than just bringing it down.
\item \( a \in [11, 16], b \in [-19, -16], c \in [-69, -67], \text{ and } r \in [-152, -149]. \)

 You multipled by the synthetic number and subtracted rather than adding during synthetic division.
\item \( a \in [60, 65], b \in [235, 241], c \in [1174, 1176], \text{ and } r \in [5995, 5996]. \)

 You multipled by the synthetic number rather than bringing the first factor down.
\end{enumerate}

\textbf{General Comment:} Be sure to synthetically divide by the zero of the denominator! Also, make sure to include 0 placeholders for missing terms.
}
\litem{
Factor the polynomial below completely, knowing that $x -3$ is a factor. Then, choose the intervals the zeros of the polynomial belong to, where $z_1 \leq z_2 \leq z_3 \leq z_4$. \textit{To make the problem easier, all zeros are between -5 and 5.}
\[ f(x) = 16x^{4} -112 x^{3} +167 x^{2} +175 x -300 \]The solution is \( [-1.25, 1.25, 3, 4] \), which is option E.\begin{enumerate}[label=\Alph*.]
\item \( z_1 \in [-4.63, -3.96], \text{   }  z_2 \in [-3.23, -2.25], z_3 \in [-1.08, -0.54], \text{   and   } z_4 \in [0.18, 1.04] \)

 Distractor 3: Corresponds to negatives of all zeros AND inversing rational roots.
\item \( z_1 \in [-0.98, -0.53], \text{   }  z_2 \in [-0.07, 0.93], z_3 \in [2.84, 3.28], \text{   and   } z_4 \in [3.16, 4.57] \)

 Distractor 2: Corresponds to inversing rational roots.
\item \( z_1 \in [-4.63, -3.96], \text{   }  z_2 \in [-3.23, -2.25], z_3 \in [-0.48, 0.1], \text{   and   } z_4 \in [4.77, 5.57] \)

 Distractor 4: Corresponds to moving factors from one rational to another.
\item \( z_1 \in [-4.63, -3.96], \text{   }  z_2 \in [-3.23, -2.25], z_3 \in [-1.68, -1.22], \text{   and   } z_4 \in [0.94, 1.37] \)

 Distractor 1: Corresponds to negatives of all zeros.
\item \( z_1 \in [-1.71, -1.18], \text{   }  z_2 \in [1.2, 2.35], z_3 \in [2.84, 3.28], \text{   and   } z_4 \in [3.16, 4.57] \)

* This is the solution!
\end{enumerate}

\textbf{General Comment:} Remember to try the middle-most integers first as these normally are the zeros. Also, once you get it to a quadratic, you can use your other factoring techniques to finish factoring.
}
\litem{
What are the \textit{possible Rational} roots of the polynomial below?
\[ f(x) = 2x^{3} +7 x^{2} +7 x + 4 \]The solution is \( \text{ All combinations of: }\frac{\pm 1,\pm 2,\pm 4}{\pm 1,\pm 2} \), which is option C.\begin{enumerate}[label=\Alph*.]
\item \( \pm 1,\pm 2,\pm 4 \)

This would have been the solution \textbf{if asked for the possible Integer roots}!
\item \( \text{ All combinations of: }\frac{\pm 1,\pm 2}{\pm 1,\pm 2,\pm 4} \)

 Distractor 3: Corresponds to the plus or minus of the inverse quotient (an/a0) of the factors. 
\item \( \text{ All combinations of: }\frac{\pm 1,\pm 2,\pm 4}{\pm 1,\pm 2} \)

* This is the solution \textbf{since we asked for the possible Rational roots}!
\item \( \pm 1,\pm 2 \)

 Distractor 1: Corresponds to the plus or minus factors of a1 only.
\item \( \text{ There is no formula or theorem that tells us all possible Rational roots.} \)

 Distractor 4: Corresponds to not recalling the theorem for rational roots of a polynomial.
\end{enumerate}

\textbf{General Comment:} We have a way to find the possible Rational roots. The possible Integer roots are the Integers in this list.
}
\end{enumerate}

\end{document}