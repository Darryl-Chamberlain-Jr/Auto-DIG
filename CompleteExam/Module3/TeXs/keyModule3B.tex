\documentclass{extbook}[14pt]
\usepackage{multicol, enumerate, enumitem, hyperref, color, soul, setspace, parskip, fancyhdr, amssymb, amsthm, amsmath, latexsym, units, mathtools}
\everymath{\displaystyle}
\usepackage[headsep=0.5cm,headheight=0cm, left=1 in,right= 1 in,top= 1 in,bottom= 1 in]{geometry}
\usepackage{dashrule}  % Package to use the command below to create lines between items
\newcommand{\litem}[1]{\item #1

\rule{\textwidth}{0.4pt}}
\pagestyle{fancy}
\lhead{}
\chead{Answer Key for Module3 Version B}
\rhead{}
\lfoot{9428-8599}
\cfoot{}
\rfoot{test}
\begin{document}
\textbf{This key should allow you to understand why you choose the option you did (beyond just getting a question right or wrong). \href{https://xronos.clas.ufl.edu/mac1105spring2020/courseDescriptionAndMisc/Exams/LearningFromResults}{More instructions on how to use this key can be found here}.}

\textbf{If you have a suggestion to make the keys better, \href{https://forms.gle/CZkbZmPbC9XALEE88}{please fill out the short survey here}.}

\textit{Note: This key is auto-generated and may contain issues and/or errors. The keys are reviewed after each exam to ensure grading is done accurately. If there are issues (like duplicate options), they are noted in the offline gradebook. The keys are a work-in-progress to give students as many resources to improve as possible.}

\rule{\textwidth}{0.4pt}

\begin{enumerate}\litem{
Using an interval or intervals, describe all the $x$-values within or including a distance of the given values.
\[ \text{ More than } 9 \text{ units from the number } -5. \]The solution is \( (-\infty, -14) \cup (4, \infty) \).\begin{enumerate}[label=\Alph*.]
\textbf{Plausible alternative answers include:}This describes the values no more than 9 from -5
This describes the values more than 9 from -5
This describes the values less than 9 from -5
This describes the values no less than 9 from -5
You likely thought the values in the interval were not correct.
\end{enumerate}

\textbf{General Comment:} When thinking about this language, it helps to draw a number line and try points.
}
\litem{
Solve the linear inequality below.
\[ -9 - 7 x \leq \frac{-20 x + 4}{8} < 9 - 3 x \]The solution is \( [-2.11, 17.00) \).\begin{enumerate}[label=\Alph*.]
\textbf{Plausible alternative answers include:}$(-\infty, -2.11] \cup (17.00, \infty)$, which corresponds to displaying the and-inequality as an or-inequality.
$(-\infty, -2.11) \cup [17.00, \infty)$, which corresponds to displaying the and-inequality as an or-inequality AND flipping the inequality.
$[-2.11, 17.00)$, which is the correct option.
$(-2.11, 17.00]$, which corresponds to flipping the inequality.

\end{enumerate}

\textbf{General Comment:} To solve, you will need to break up the compound inequality into two inequalities. Be sure to keep track of the inequality! It may be best to draw a number line and graph your solution.
}
\litem{
Solve the linear inequality below.
\[ -9x + 8 \geq -6x -4 \]The solution is \( (-\infty, 4.0] \).\begin{enumerate}[label=\Alph*.]
\textbf{Plausible alternative answers include:}* $(-\infty, 4.0]$, which is the correct option.
 $[4.0, \infty)$, which corresponds to switching the direction of the interval. You likely did this if you did not flip the inequality when dividing by a negative!
 $[-4.0, \infty)$, which corresponds to switching the direction of the interval AND negating the endpoint. You likely did this if you did not flip the inequality when dividing by a negative as well as not moving values over to a side properly.
 $(-\infty, -4.0]$, which corresponds to negating the endpoint of the solution.
You may have chosen this if you thought the inequality did not match the ends of the intervals.
\end{enumerate}

\textbf{General Comment:} Remember that less/greater than or equal to includes the endpoint, while less/greater do not. Also, remember that you need to flip the inequality when you multiply or divide by a negative.
}
\litem{
Solve the linear inequality below.
\[ -8 + 3 x > 5 x \text{ or } -3 + 6 x < 9 x \]The solution is \( (-\infty, -4.0) \text{ or } (-1.0, \infty) \).\begin{enumerate}[label=\Alph*.]
\textbf{Plausible alternative answers include:}Corresponds to including the endpoints (when they should be excluded).
Corresponds to inverting the inequality and negating the solution.
Corresponds to including the endpoints AND negating.
 * Correct option.
Corresponds to the variable canceling, which does not happen in this instance.
\end{enumerate}

\textbf{General Comment:} When multiplying or dividing by a negative, flip the sign.
}
\litem{
Solve the linear inequality below.
\[ -9x + 8 \geq 5x -4 \]The solution is \( (-\infty, 0.857] \).\begin{enumerate}[label=\Alph*.]
\textbf{Plausible alternative answers include:} $(-\infty, -0.857]$, which corresponds to negating the endpoint of the solution.
 $[-0.857, \infty)$, which corresponds to switching the direction of the interval AND negating the endpoint. You likely did this if you did not flip the inequality when dividing by a negative as well as not moving values over to a side properly.
 $[0.857, \infty)$, which corresponds to switching the direction of the interval. You likely did this if you did not flip the inequality when dividing by a negative!
* $(-\infty, 0.857]$, which is the correct option.
You may have chosen this if you thought the inequality did not match the ends of the intervals.
\end{enumerate}

\textbf{General Comment:} Remember that less/greater than or equal to includes the endpoint, while less/greater do not. Also, remember that you need to flip the inequality when you multiply or divide by a negative.
}
\litem{
Solve the linear inequality below.
\[ \frac{8}{6} + \frac{7}{4} x < \frac{8}{7} x - \frac{8}{5} \]The solution is \( (-\infty, -4.831) \).\begin{enumerate}[label=\Alph*.]
\textbf{Plausible alternative answers include:}* $(-\infty, -4.831)$, which is the correct option.
 $(-\infty, 4.831)$, which corresponds to negating the endpoint of the solution.
 $(-4.831, \infty)$, which corresponds to switching the direction of the interval. You likely did this if you did not flip the inequality when dividing by a negative!
 $(4.831, \infty)$, which corresponds to switching the direction of the interval AND negating the endpoint. You likely did this if you did not flip the inequality when dividing by a negative as well as not moving values over to a side properly.
You may have chosen this if you thought the inequality did not match the ends of the intervals.
\end{enumerate}

\textbf{General Comment:} Remember that less/greater than or equal to includes the endpoint, while less/greater do not. Also, remember that you need to flip the inequality when you multiply or divide by a negative.
}
\litem{
Solve the linear inequality below.
\[ -4 + 3 x > 4 x \text{ or } -3 + 6 x < 8 x \]The solution is \( (-\infty, -4.0) \text{ or } (-1.5, \infty) \).\begin{enumerate}[label=\Alph*.]
\textbf{Plausible alternative answers include:}Corresponds to inverting the inequality and negating the solution.
Corresponds to including the endpoints AND negating.
 * Correct option.
Corresponds to including the endpoints (when they should be excluded).
Corresponds to the variable canceling, which does not happen in this instance.
\end{enumerate}

\textbf{General Comment:} When multiplying or dividing by a negative, flip the sign.
}
\litem{
Solve the linear inequality below.
\[ -9 - 9 x \leq \frac{-50 x + 5}{9} < 6 - 6 x \]The solution is \( [-2.77, 12.25) \).\begin{enumerate}[label=\Alph*.]
\textbf{Plausible alternative answers include:}$(-\infty, -2.77] \cup (12.25, \infty)$, which corresponds to displaying the and-inequality as an or-inequality.
$[-2.77, 12.25)$, which is the correct option.
$(-\infty, -2.77) \cup [12.25, \infty)$, which corresponds to displaying the and-inequality as an or-inequality AND flipping the inequality.
$(-2.77, 12.25]$, which corresponds to flipping the inequality.

\end{enumerate}

\textbf{General Comment:} To solve, you will need to break up the compound inequality into two inequalities. Be sure to keep track of the inequality! It may be best to draw a number line and graph your solution.
}
\litem{
Solve the linear inequality below.
\[ \frac{-4}{2} - \frac{3}{9} x < \frac{3}{3} x + \frac{10}{8} \]The solution is \( (-2.438, \infty) \).\begin{enumerate}[label=\Alph*.]
\textbf{Plausible alternative answers include:} $(-\infty, -2.438)$, which corresponds to switching the direction of the interval. You likely did this if you did not flip the inequality when dividing by a negative!
 $(-\infty, 2.438)$, which corresponds to switching the direction of the interval AND negating the endpoint. You likely did this if you did not flip the inequality when dividing by a negative as well as not moving values over to a side properly.
 $(2.438, \infty)$, which corresponds to negating the endpoint of the solution.
* $(-2.438, \infty)$, which is the correct option.
You may have chosen this if you thought the inequality did not match the ends of the intervals.
\end{enumerate}

\textbf{General Comment:} Remember that less/greater than or equal to includes the endpoint, while less/greater do not. Also, remember that you need to flip the inequality when you multiply or divide by a negative.
}
\litem{
Using an interval or intervals, describe all the $x$-values within or including a distance of the given values.
\[ \text{ Less than } 4 \text{ units from the number } -3. \]The solution is \( (-7, 1) \).\begin{enumerate}[label=\Alph*.]
\textbf{Plausible alternative answers include:}This describes the values no more than 4 from -3
This describes the values less than 4 from -3
This describes the values no less than 4 from -3
This describes the values more than 4 from -3
You likely thought the values in the interval were not correct.
\end{enumerate}

\textbf{General Comment:} When thinking about this language, it helps to draw a number line and try points.
}
\end{enumerate}

\end{document}