\documentclass{extbook}[14pt]
\usepackage{multicol, enumerate, enumitem, hyperref, color, soul, setspace, parskip, fancyhdr, amssymb, amsthm, amsmath, latexsym, units, mathtools}
\everymath{\displaystyle}
\usepackage[headsep=0.5cm,headheight=0cm, left=1 in,right= 1 in,top= 1 in,bottom= 1 in]{geometry}
\usepackage{dashrule}  % Package to use the command below to create lines between items
\newcommand{\litem}[1]{\item #1

\rule{\textwidth}{0.4pt}}
\pagestyle{fancy}
\lhead{}
\chead{Answer Key for Module3 Version C}
\rhead{}
\lfoot{9428-8599}
\cfoot{}
\rfoot{test}
\begin{document}
\textbf{This key should allow you to understand why you choose the option you did (beyond just getting a question right or wrong). \href{https://xronos.clas.ufl.edu/mac1105spring2020/courseDescriptionAndMisc/Exams/LearningFromResults}{More instructions on how to use this key can be found here}.}

\textbf{If you have a suggestion to make the keys better, \href{https://forms.gle/CZkbZmPbC9XALEE88}{please fill out the short survey here}.}

\textit{Note: This key is auto-generated and may contain issues and/or errors. The keys are reviewed after each exam to ensure grading is done accurately. If there are issues (like duplicate options), they are noted in the offline gradebook. The keys are a work-in-progress to give students as many resources to improve as possible.}

\rule{\textwidth}{0.4pt}

\begin{enumerate}\litem{
Using an interval or intervals, describe all the $x$-values within or including a distance of the given values.
\[ \text{ No less than } 8 \text{ units from the number } 4. \]The solution is \( (-\infty, -4] \cup [12, \infty) \).\begin{enumerate}[label=\Alph*.]
\textbf{Plausible alternative answers include:}This describes the values less than 8 from 4
This describes the values no less than 8 from 4
This describes the values more than 8 from 4
This describes the values no more than 8 from 4
You likely thought the values in the interval were not correct.
\end{enumerate}

\textbf{General Comment:} When thinking about this language, it helps to draw a number line and try points.
}
\litem{
Solve the linear inequality below.
\[ -8 + 6 x \leq \frac{59 x - 9}{9} < 8 + 3 x \]The solution is \( [-12.60, 2.53) \).\begin{enumerate}[label=\Alph*.]
\textbf{Plausible alternative answers include:}$[-12.60, 2.53)$, which is the correct option.
$(-12.60, 2.53]$, which corresponds to flipping the inequality.
$(-\infty, -12.60] \cup (2.53, \infty)$, which corresponds to displaying the and-inequality as an or-inequality.
$(-\infty, -12.60) \cup [2.53, \infty)$, which corresponds to displaying the and-inequality as an or-inequality AND flipping the inequality.

\end{enumerate}

\textbf{General Comment:} To solve, you will need to break up the compound inequality into two inequalities. Be sure to keep track of the inequality! It may be best to draw a number line and graph your solution.
}
\litem{
Solve the linear inequality below.
\[ -5x + 7 \geq 3x -4 \]The solution is \( (-\infty, 1.375] \).\begin{enumerate}[label=\Alph*.]
\textbf{Plausible alternative answers include:} $[1.375, \infty)$, which corresponds to switching the direction of the interval. You likely did this if you did not flip the inequality when dividing by a negative!
* $(-\infty, 1.375]$, which is the correct option.
 $[-1.375, \infty)$, which corresponds to switching the direction of the interval AND negating the endpoint. You likely did this if you did not flip the inequality when dividing by a negative as well as not moving values over to a side properly.
 $(-\infty, -1.375]$, which corresponds to negating the endpoint of the solution.
You may have chosen this if you thought the inequality did not match the ends of the intervals.
\end{enumerate}

\textbf{General Comment:} Remember that less/greater than or equal to includes the endpoint, while less/greater do not. Also, remember that you need to flip the inequality when you multiply or divide by a negative.
}
\litem{
Solve the linear inequality below.
\[ 9 - 4 x > 5 x \text{ or } 3 + 4 x < 5 x \]The solution is \( (-\infty, 1.0) \text{ or } (3.0, \infty) \).\begin{enumerate}[label=\Alph*.]
\textbf{Plausible alternative answers include:}Corresponds to including the endpoints AND negating.
Corresponds to including the endpoints (when they should be excluded).
 * Correct option.
Corresponds to inverting the inequality and negating the solution.
Corresponds to the variable canceling, which does not happen in this instance.
\end{enumerate}

\textbf{General Comment:} When multiplying or dividing by a negative, flip the sign.
}
\litem{
Solve the linear inequality below.
\[ -7x -7 \leq 10x + 10 \]The solution is \( [-1.0, \infty) \).\begin{enumerate}[label=\Alph*.]
\textbf{Plausible alternative answers include:}* $[-1.0, \infty)$, which is the correct option.
 $(-\infty, 1.0]$, which corresponds to switching the direction of the interval AND negating the endpoint. You likely did this if you did not flip the inequality when dividing by a negative as well as not moving values over to a side properly.
 $(-\infty, -1.0]$, which corresponds to switching the direction of the interval. You likely did this if you did not flip the inequality when dividing by a negative!
 $[1.0, \infty)$, which corresponds to negating the endpoint of the solution.
You may have chosen this if you thought the inequality did not match the ends of the intervals.
\end{enumerate}

\textbf{General Comment:} Remember that less/greater than or equal to includes the endpoint, while less/greater do not. Also, remember that you need to flip the inequality when you multiply or divide by a negative.
}
\litem{
Solve the linear inequality below.
\[ \frac{-3}{6} + \frac{5}{4} x \geq \frac{10}{9} x - \frac{6}{8} \]The solution is \( [-1.8, \infty) \).\begin{enumerate}[label=\Alph*.]
\textbf{Plausible alternative answers include:} $(-\infty, -1.8]$, which corresponds to switching the direction of the interval. You likely did this if you did not flip the inequality when dividing by a negative!
 $[1.8, \infty)$, which corresponds to negating the endpoint of the solution.
* $[-1.8, \infty)$, which is the correct option.
 $(-\infty, 1.8]$, which corresponds to switching the direction of the interval AND negating the endpoint. You likely did this if you did not flip the inequality when dividing by a negative as well as not moving values over to a side properly.
You may have chosen this if you thought the inequality did not match the ends of the intervals.
\end{enumerate}

\textbf{General Comment:} Remember that less/greater than or equal to includes the endpoint, while less/greater do not. Also, remember that you need to flip the inequality when you multiply or divide by a negative.
}
\litem{
Solve the linear inequality below.
\[ -8 + 8 x > 9 x \text{ or } -7 + 8 x < 11 x \]The solution is \( (-\infty, -8.0) \text{ or } (-2.333, \infty) \).\begin{enumerate}[label=\Alph*.]
\textbf{Plausible alternative answers include:}Corresponds to inverting the inequality and negating the solution.
Corresponds to including the endpoints (when they should be excluded).
Corresponds to including the endpoints AND negating.
 * Correct option.
Corresponds to the variable canceling, which does not happen in this instance.
\end{enumerate}

\textbf{General Comment:} When multiplying or dividing by a negative, flip the sign.
}
\litem{
Solve the linear inequality below.
\[ 4 - 4 x < \frac{-8 x + 3}{3} \leq 6 - 4 x \]The solution is \( (2.25, 3.75] \).\begin{enumerate}[label=\Alph*.]
\textbf{Plausible alternative answers include:}$[2.25, 3.75)$, which corresponds to flipping the inequality.
$(-\infty, 2.25) \cup [3.75, \infty)$, which corresponds to displaying the and-inequality as an or-inequality.
$(-\infty, 2.25] \cup (3.75, \infty)$, which corresponds to displaying the and-inequality as an or-inequality AND flipping the inequality.
* $(2.25, 3.75]$, which is the correct option.

\end{enumerate}

\textbf{General Comment:} To solve, you will need to break up the compound inequality into two inequalities. Be sure to keep track of the inequality! It may be best to draw a number line and graph your solution.
}
\litem{
Solve the linear inequality below.
\[ \frac{-5}{4} - \frac{8}{7} x \geq \frac{4}{9} x + \frac{9}{5} \]The solution is \( (-\infty, -1.921] \).\begin{enumerate}[label=\Alph*.]
\textbf{Plausible alternative answers include:} $[1.921, \infty)$, which corresponds to switching the direction of the interval AND negating the endpoint. You likely did this if you did not flip the inequality when dividing by a negative as well as not moving values over to a side properly.
 $[-1.921, \infty)$, which corresponds to switching the direction of the interval. You likely did this if you did not flip the inequality when dividing by a negative!
* $(-\infty, -1.921]$, which is the correct option.
 $(-\infty, 1.921]$, which corresponds to negating the endpoint of the solution.
You may have chosen this if you thought the inequality did not match the ends of the intervals.
\end{enumerate}

\textbf{General Comment:} Remember that less/greater than or equal to includes the endpoint, while less/greater do not. Also, remember that you need to flip the inequality when you multiply or divide by a negative.
}
\litem{
Using an interval or intervals, describe all the $x$-values within or including a distance of the given values.
\[ \text{ No less than } 2 \text{ units from the number } 7. \]The solution is \( (-\infty, 5] \cup [9, \infty) \).\begin{enumerate}[label=\Alph*.]
\textbf{Plausible alternative answers include:}This describes the values no more than 2 from 7
This describes the values no less than 2 from 7
This describes the values less than 2 from 7
This describes the values more than 2 from 7
You likely thought the values in the interval were not correct.
\end{enumerate}

\textbf{General Comment:} When thinking about this language, it helps to draw a number line and try points.
}
\end{enumerate}

\end{document}