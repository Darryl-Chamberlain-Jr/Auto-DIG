\documentclass{extbook}[14pt]
\usepackage{multicol, enumerate, enumitem, hyperref, color, soul, setspace, parskip, fancyhdr, amssymb, amsthm, amsmath, latexsym, units, mathtools}
\everymath{\displaystyle}
\usepackage[headsep=0.5cm,headheight=0cm, left=1 in,right= 1 in,top= 1 in,bottom= 1 in]{geometry}
\usepackage{dashrule}  % Package to use the command below to create lines between items
\newcommand{\litem}[1]{\item #1

\rule{\textwidth}{0.4pt}}
\pagestyle{fancy}
\lhead{}
\chead{Answer Key for Module3 Version A}
\rhead{}
\lfoot{9428-8599}
\cfoot{}
\rfoot{test}
\begin{document}
\textbf{This key should allow you to understand why you choose the option you did (beyond just getting a question right or wrong). \href{https://xronos.clas.ufl.edu/mac1105spring2020/courseDescriptionAndMisc/Exams/LearningFromResults}{More instructions on how to use this key can be found here}.}

\textbf{If you have a suggestion to make the keys better, \href{https://forms.gle/CZkbZmPbC9XALEE88}{please fill out the short survey here}.}

\textit{Note: This key is auto-generated and may contain issues and/or errors. The keys are reviewed after each exam to ensure grading is done accurately. If there are issues (like duplicate options), they are noted in the offline gradebook. The keys are a work-in-progress to give students as many resources to improve as possible.}

\rule{\textwidth}{0.4pt}

\begin{enumerate}\litem{
Using an interval or intervals, describe all the $x$-values within or including a distance of the given values.
\[ \text{ More than } 7 \text{ units from the number } 8. \]The solution is \( \text{None of the above} \).\begin{enumerate}[label=\Alph*.]
\textbf{Plausible alternative answers include:}This describes the values no less than 8 from 7
This describes the values no more than 8 from 7
This describes the values more than 8 from 7
This describes the values less than 8 from 7
Options A-D described the values [more/less than] 8 units from 7, which is the reverse of what the question asked.
\end{enumerate}

\textbf{General Comment:} When thinking about this language, it helps to draw a number line and try points.
}
\litem{
Solve the linear inequality below.
\[ -3 - 8 x < \frac{-70 x + 4}{9} \leq -4 - 9 x \]The solution is \( \text{None of the above.} \).\begin{enumerate}[label=\Alph*.]
\textbf{Plausible alternative answers include:}$(15.50, 3.64]$, which is the correct interval but negatives of the actual endpoints.
$(-\infty, 15.50] \cup (3.64, \infty)$, which corresponds to displaying the and-inequality as an or-inequality AND flipping the inequality AND getting negatives of the actual endpoints.
$(-\infty, 15.50) \cup [3.64, \infty)$, which corresponds to displaying the and-inequality as an or-inequality and getting negatives of the actual endpoints.
$[15.50, 3.64)$, which corresponds to flipping the inequality and getting negatives of the actual endpoints.
* This is correct as the answer should be $(-15.50, -3.64]$.
\end{enumerate}

\textbf{General Comment:} To solve, you will need to break up the compound inequality into two inequalities. Be sure to keep track of the inequality! It may be best to draw a number line and graph your solution.
}
\litem{
Solve the linear inequality below.
\[ -9x + 5 > 6x -8 \]The solution is \( (-\infty, 0.867) \).\begin{enumerate}[label=\Alph*.]
\textbf{Plausible alternative answers include:} $(-\infty, -0.867)$, which corresponds to negating the endpoint of the solution.
 $(0.867, \infty)$, which corresponds to switching the direction of the interval. You likely did this if you did not flip the inequality when dividing by a negative!
* $(-\infty, 0.867)$, which is the correct option.
 $(-0.867, \infty)$, which corresponds to switching the direction of the interval AND negating the endpoint. You likely did this if you did not flip the inequality when dividing by a negative as well as not moving values over to a side properly.
You may have chosen this if you thought the inequality did not match the ends of the intervals.
\end{enumerate}

\textbf{General Comment:} Remember that less/greater than or equal to includes the endpoint, while less/greater do not. Also, remember that you need to flip the inequality when you multiply or divide by a negative.
}
\litem{
Solve the linear inequality below.
\[ -9 + 7 x > 8 x \text{ or } -6 + 3 x < 6 x \]The solution is \( (-\infty, -9.0) \text{ or } (-2.0, \infty) \).\begin{enumerate}[label=\Alph*.]
\textbf{Plausible alternative answers include:} * Correct option.
Corresponds to including the endpoints AND negating.
Corresponds to inverting the inequality and negating the solution.
Corresponds to including the endpoints (when they should be excluded).
Corresponds to the variable canceling, which does not happen in this instance.
\end{enumerate}

\textbf{General Comment:} When multiplying or dividing by a negative, flip the sign.
}
\litem{
Solve the linear inequality below.
\[ -10x + 7 > 10x -5 \]The solution is \( (-\infty, 0.6) \).\begin{enumerate}[label=\Alph*.]
\textbf{Plausible alternative answers include:} $(-0.6, \infty)$, which corresponds to switching the direction of the interval AND negating the endpoint. You likely did this if you did not flip the inequality when dividing by a negative as well as not moving values over to a side properly.
 $(0.6, \infty)$, which corresponds to switching the direction of the interval. You likely did this if you did not flip the inequality when dividing by a negative!
 $(-\infty, -0.6)$, which corresponds to negating the endpoint of the solution.
* $(-\infty, 0.6)$, which is the correct option.
You may have chosen this if you thought the inequality did not match the ends of the intervals.
\end{enumerate}

\textbf{General Comment:} Remember that less/greater than or equal to includes the endpoint, while less/greater do not. Also, remember that you need to flip the inequality when you multiply or divide by a negative.
}
\litem{
Solve the linear inequality below.
\[ \frac{3}{4} - \frac{10}{5} x \geq \frac{-8}{3} x - \frac{3}{8} \]The solution is \( [-1.688, \infty) \).\begin{enumerate}[label=\Alph*.]
\textbf{Plausible alternative answers include:}* $[-1.688, \infty)$, which is the correct option.
 $[1.688, \infty)$, which corresponds to negating the endpoint of the solution.
 $(-\infty, 1.688]$, which corresponds to switching the direction of the interval AND negating the endpoint. You likely did this if you did not flip the inequality when dividing by a negative as well as not moving values over to a side properly.
 $(-\infty, -1.688]$, which corresponds to switching the direction of the interval. You likely did this if you did not flip the inequality when dividing by a negative!
You may have chosen this if you thought the inequality did not match the ends of the intervals.
\end{enumerate}

\textbf{General Comment:} Remember that less/greater than or equal to includes the endpoint, while less/greater do not. Also, remember that you need to flip the inequality when you multiply or divide by a negative.
}
\litem{
Solve the linear inequality below.
\[ 4 + 9 x > 12 x \text{ or } 6 + 3 x < 5 x \]The solution is \( (-\infty, 1.333) \text{ or } (3.0, \infty) \).\begin{enumerate}[label=\Alph*.]
\textbf{Plausible alternative answers include:} * Correct option.
Corresponds to including the endpoints (when they should be excluded).
Corresponds to including the endpoints AND negating.
Corresponds to inverting the inequality and negating the solution.
Corresponds to the variable canceling, which does not happen in this instance.
\end{enumerate}

\textbf{General Comment:} When multiplying or dividing by a negative, flip the sign.
}
\litem{
Solve the linear inequality below.
\[ -4 + 6 x < \frac{29 x - 9}{4} \leq 4 + 5 x \]The solution is \( \text{None of the above.} \).\begin{enumerate}[label=\Alph*.]
\textbf{Plausible alternative answers include:}$(-\infty, 1.40] \cup (-2.78, \infty)$, which corresponds to displaying the and-inequality as an or-inequality AND flipping the inequality AND getting negatives of the actual endpoints.
$(1.40, -2.78]$, which is the correct interval but negatives of the actual endpoints.
$(-\infty, 1.40) \cup [-2.78, \infty)$, which corresponds to displaying the and-inequality as an or-inequality and getting negatives of the actual endpoints.
$[1.40, -2.78)$, which corresponds to flipping the inequality and getting negatives of the actual endpoints.
* This is correct as the answer should be $(-1.40, 2.78]$.
\end{enumerate}

\textbf{General Comment:} To solve, you will need to break up the compound inequality into two inequalities. Be sure to keep track of the inequality! It may be best to draw a number line and graph your solution.
}
\litem{
Solve the linear inequality below.
\[ \frac{9}{5} + \frac{3}{6} x \geq \frac{9}{8} x + \frac{3}{7} \]The solution is \( (-\infty, 2.194] \).\begin{enumerate}[label=\Alph*.]
\textbf{Plausible alternative answers include:}* $(-\infty, 2.194]$, which is the correct option.
 $[-2.194, \infty)$, which corresponds to switching the direction of the interval AND negating the endpoint. You likely did this if you did not flip the inequality when dividing by a negative as well as not moving values over to a side properly.
 $(-\infty, -2.194]$, which corresponds to negating the endpoint of the solution.
 $[2.194, \infty)$, which corresponds to switching the direction of the interval. You likely did this if you did not flip the inequality when dividing by a negative!
You may have chosen this if you thought the inequality did not match the ends of the intervals.
\end{enumerate}

\textbf{General Comment:} Remember that less/greater than or equal to includes the endpoint, while less/greater do not. Also, remember that you need to flip the inequality when you multiply or divide by a negative.
}
\litem{
Using an interval or intervals, describe all the $x$-values within or including a distance of the given values.
\[ \text{ No more than } 4 \text{ units from the number } 2. \]The solution is \( [-2, 6] \).\begin{enumerate}[label=\Alph*.]
\textbf{Plausible alternative answers include:}This describes the values less than 4 from 2
This describes the values no more than 4 from 2
This describes the values no less than 4 from 2
This describes the values more than 4 from 2
You likely thought the values in the interval were not correct.
\end{enumerate}

\textbf{General Comment:} When thinking about this language, it helps to draw a number line and try points.
}
\end{enumerate}

\end{document}