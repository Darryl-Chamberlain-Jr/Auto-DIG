\documentclass{extbook}[14pt]
\usepackage{multicol, enumerate, enumitem, hyperref, color, soul, setspace, parskip, fancyhdr, amssymb, amsthm, amsmath, latexsym, units, mathtools}
\everymath{\displaystyle}
\usepackage[headsep=0.5cm,headheight=0cm, left=1 in,right= 1 in,top= 1 in,bottom= 1 in]{geometry}
\usepackage{dashrule}  % Package to use the command below to create lines between items
\newcommand{\litem}[1]{\item #1

\rule{\textwidth}{0.4pt}}
\pagestyle{fancy}
\lhead{}
\chead{Answer Key for Progress Quiz 3 Version C}
\rhead{}
\lfoot{3012-8528}
\cfoot{}
\rfoot{Summer C 2021}
\begin{document}
\textbf{This key should allow you to understand why you choose the option you did (beyond just getting a question right or wrong). \href{https://xronos.clas.ufl.edu/mac1105spring2020/courseDescriptionAndMisc/Exams/LearningFromResults}{More instructions on how to use this key can be found here}.}

\textbf{If you have a suggestion to make the keys better, \href{https://forms.gle/CZkbZmPbC9XALEE88}{please fill out the short survey here}.}

\textit{Note: This key is auto-generated and may contain issues and/or errors. The keys are reviewed after each exam to ensure grading is done accurately. If there are issues (like duplicate options), they are noted in the offline gradebook. The keys are a work-in-progress to give students as many resources to improve as possible.}

\rule{\textwidth}{0.4pt}

\begin{enumerate}\litem{
Simplify the expression below into the form $a+bi$. Then, choose the intervals that $a$ and $b$ belong to.
\[ (-6 + 4 i)(9 - 3 i) \]The solution is \( -42 + 54 i \), which is option C.\begin{enumerate}[label=\Alph*.]
\item \( a \in [-42, -38] \text{ and } b \in [-56, -50] \)

 $-42 - 54 i$, which corresponds to adding a minus sign in both terms.
\item \( a \in [-68, -58] \text{ and } b \in [16, 19] \)

 $-66 + 18 i$, which corresponds to adding a minus sign in the second term.
\item \( a \in [-42, -38] \text{ and } b \in [52, 59] \)

* $-42 + 54 i$, which is the correct option.
\item \( a \in [-68, -58] \text{ and } b \in [-18, -16] \)

 $-66 - 18 i$, which corresponds to adding a minus sign in the first term.
\item \( a \in [-55, -49] \text{ and } b \in [-16, -6] \)

 $-54 - 12 i$, which corresponds to just multiplying the real terms to get the real part of the solution and the coefficients in the complex terms to get the complex part.
\end{enumerate}

\textbf{General Comment:} You can treat $i$ as a variable and distribute. Just remember that $i^2=-1$, so you can continue to reduce after you distribute.
}
\litem{
Simplify the expression below and choose the interval the simplification is contained within.
\[ 20 - 17^2 + 7 \div 18 * 19 \div 8 \]The solution is \( -268.076 \), which is option D.\begin{enumerate}[label=\Alph*.]
\item \( [-269.87, -268.41] \)

 -268.997, which corresponds to an Order of Operations error: not reading left-to-right for multiplication/division.
\item \( [309.82, 310.51] \)

 309.924, which corresponds to an Order of Operations error: multiplying by negative before squaring. For example: $(-3)^2 \neq -3^2$
\item \( [308.54, 309.06] \)

 309.003, which corresponds to two Order of Operations errors.
\item \( [-268.96, -267.02] \)

* -268.076, this is the correct option
\item \( \text{None of the above} \)

 You may have gotten this by making an unanticipated error. If you got a value that is not any of the others, please let the coordinator know so they can help you figure out what happened.
\end{enumerate}

\textbf{General Comment:} While you may remember (or were taught) PEMDAS is done in order, it is actually done as P/E/MD/AS. When we are at MD or AS, we read left to right.
}
\litem{
Choose the \textbf{smallest} set of Complex numbers that the number below belongs to.
\[ \frac{12}{14}+\sqrt{-9}i \]The solution is \( \text{Rational} \), which is option D.\begin{enumerate}[label=\Alph*.]
\item \( \text{Irrational} \)

These cannot be written as a fraction of Integers. Remember: $\pi$ is not an Integer!
\item \( \text{Not a Complex Number} \)

This is not a number. The only non-Complex number we know is dividing by 0 as this is not a number!
\item \( \text{Nonreal Complex} \)

This is a Complex number $(a+bi)$ that is not Real (has $i$ as part of the number).
\item \( \text{Rational} \)

* This is the correct option!
\item \( \text{Pure Imaginary} \)

This is a Complex number $(a+bi)$ that \textbf{only} has an imaginary part like $2i$.
\end{enumerate}

\textbf{General Comment:} Be sure to simplify $i^2 = -1$. This may remove the imaginary portion for your number. If you are having trouble, you may want to look at the \textit{Subgroups of the Real Numbers} section.
}
\litem{
Choose the \textbf{smallest} set of Complex numbers that the number below belongs to.
\[ -\sqrt{\frac{225}{121}} + 16i^2 \]The solution is \( \text{Rational} \), which is option D.\begin{enumerate}[label=\Alph*.]
\item \( \text{Nonreal Complex} \)

This is a Complex number $(a+bi)$ that is not Real (has $i$ as part of the number).
\item \( \text{Pure Imaginary} \)

This is a Complex number $(a+bi)$ that \textbf{only} has an imaginary part like $2i$.
\item \( \text{Irrational} \)

These cannot be written as a fraction of Integers. Remember: $\pi$ is not an Integer!
\item \( \text{Rational} \)

* This is the correct option!
\item \( \text{Not a Complex Number} \)

This is not a number. The only non-Complex number we know is dividing by 0 as this is not a number!
\end{enumerate}

\textbf{General Comment:} Be sure to simplify $i^2 = -1$. This may remove the imaginary portion for your number. If you are having trouble, you may want to look at the \textit{Subgroups of the Real Numbers} section.
}
\litem{
Simplify the expression below into the form $a+bi$. Then, choose the intervals that $a$ and $b$ belong to.
\[ \frac{63 - 22 i}{-1 + 3 i} \]The solution is \( -12.90  - 16.70 i \), which is option A.\begin{enumerate}[label=\Alph*.]
\item \( a \in [-13, -12.5] \text{ and } b \in [-17.5, -16] \)

* $-12.90  - 16.70 i$, which is the correct option.
\item \( a \in [-64.5, -62.5] \text{ and } b \in [-9, -5.5] \)

 $-63.00  - 7.33 i$, which corresponds to just dividing the first term by the first term and the second by the second.
\item \( a \in [-13, -12.5] \text{ and } b \in [-169, -166] \)

 $-12.90  - 167.00 i$, which corresponds to forgetting to multiply the conjugate by the numerator.
\item \( a \in [-130, -128] \text{ and } b \in [-17.5, -16] \)

 $-129.00  - 16.70 i$, which corresponds to forgetting to multiply the conjugate by the numerator and using a plus instead of a minus in the denominator.
\item \( a \in [-0.5, 1.5] \text{ and } b \in [20, 22.5] \)

 $0.30  + 21.10 i$, which corresponds to forgetting to multiply the conjugate by the numerator and not computing the conjugate correctly.
\end{enumerate}

\textbf{General Comment:} Multiply the numerator and denominator by the *conjugate* of the denominator, then simplify. For example, if we have $2+3i$, the conjugate is $2-3i$.
}
\litem{
Simplify the expression below and choose the interval the simplification is contained within.
\[ 20 - 19 \div 14 * 6 - (16 * 17) \]The solution is \( -260.143 \), which is option D.\begin{enumerate}[label=\Alph*.]
\item \( [-70.43, -63.43] \)

 -70.429, which corresponds to not distributing a negative correctly.
\item \( [-255.23, -247.23] \)

 -252.226, which corresponds to an Order of Operations error: not reading left-to-right for multiplication/division.
\item \( [291.77, 293.77] \)

 291.774, which corresponds to not distributing addition and subtraction correctly.
\item \( [-262.14, -258.14] \)

* -260.143, which is the correct option.
\item \( \text{None of the above} \)

 You may have gotten this by making an unanticipated error. If you got a value that is not any of the others, please let the coordinator know so they can help you figure out what happened.
\end{enumerate}

\textbf{General Comment:} While you may remember (or were taught) PEMDAS is done in order, it is actually done as P/E/MD/AS. When we are at MD or AS, we read left to right.
}
\litem{
Simplify the expression below into the form $a+bi$. Then, choose the intervals that $a$ and $b$ belong to.
\[ (9 - 6 i)(2 - 8 i) \]The solution is \( -30 - 84 i \), which is option B.\begin{enumerate}[label=\Alph*.]
\item \( a \in [-34, -28] \text{ and } b \in [81, 88] \)

 $-30 + 84 i$, which corresponds to adding a minus sign in both terms.
\item \( a \in [-34, -28] \text{ and } b \in [-86, -81] \)

* $-30 - 84 i$, which is the correct option.
\item \( a \in [66, 68] \text{ and } b \in [60, 67] \)

 $66 + 60 i$, which corresponds to adding a minus sign in the second term.
\item \( a \in [66, 68] \text{ and } b \in [-64, -55] \)

 $66 - 60 i$, which corresponds to adding a minus sign in the first term.
\item \( a \in [17, 23] \text{ and } b \in [48, 55] \)

 $18 + 48 i$, which corresponds to just multiplying the real terms to get the real part of the solution and the coefficients in the complex terms to get the complex part.
\end{enumerate}

\textbf{General Comment:} You can treat $i$ as a variable and distribute. Just remember that $i^2=-1$, so you can continue to reduce after you distribute.
}
\litem{
Simplify the expression below into the form $a+bi$. Then, choose the intervals that $a$ and $b$ belong to.
\[ \frac{45 - 66 i}{7 - i} \]The solution is \( 7.62  - 8.34 i \), which is option C.\begin{enumerate}[label=\Alph*.]
\item \( a \in [4, 6] \text{ and } b \in [-10.5, -9.5] \)

 $4.98  - 10.14 i$, which corresponds to forgetting to multiply the conjugate by the numerator and not computing the conjugate correctly.
\item \( a \in [5.5, 7] \text{ and } b \in [65, 66.5] \)

 $6.43  + 66.00 i$, which corresponds to just dividing the first term by the first term and the second by the second.
\item \( a \in [6.5, 8] \text{ and } b \in [-9, -6.5] \)

* $7.62  - 8.34 i$, which is the correct option.
\item \( a \in [380, 382] \text{ and } b \in [-9, -6.5] \)

 $381.00  - 8.34 i$, which corresponds to forgetting to multiply the conjugate by the numerator and using a plus instead of a minus in the denominator.
\item \( a \in [6.5, 8] \text{ and } b \in [-417.5, -416] \)

 $7.62  - 417.00 i$, which corresponds to forgetting to multiply the conjugate by the numerator.
\end{enumerate}

\textbf{General Comment:} Multiply the numerator and denominator by the *conjugate* of the denominator, then simplify. For example, if we have $2+3i$, the conjugate is $2-3i$.
}
\litem{
Choose the \textbf{smallest} set of Real numbers that the number below belongs to.
\[ \sqrt{\frac{-715}{5}} \]The solution is \( \text{Not a Real number} \), which is option D.\begin{enumerate}[label=\Alph*.]
\item \( \text{Integer} \)

These are the negative and positive counting numbers (..., -3, -2, -1, 0, 1, 2, 3, ...)
\item \( \text{Irrational} \)

These cannot be written as a fraction of Integers.
\item \( \text{Rational} \)

These are numbers that can be written as fraction of Integers (e.g., -2/3)
\item \( \text{Not a Real number} \)

* This is the correct option!
\item \( \text{Whole} \)

These are the counting numbers with 0 (0, 1, 2, 3, ...)
\end{enumerate}

\textbf{General Comment:} First, you \textbf{NEED} to simplify the expression. This question simplifies to $\sqrt{143} i$. 
 
 Be sure you look at the simplified fraction and not just the decimal expansion. Numbers such as 13, 17, and 19 provide \textbf{long but repeating/terminating decimal expansions!} 
 
 The only ways to *not* be a Real number are: dividing by 0 or taking the square root of a negative number. 
 
 Irrational numbers are more than just square root of 3: adding or subtracting values from square root of 3 is also irrational.
}
\litem{
Choose the \textbf{smallest} set of Real numbers that the number below belongs to.
\[ \sqrt{\frac{38025}{169}} \]The solution is \( \text{Whole} \), which is option E.\begin{enumerate}[label=\Alph*.]
\item \( \text{Integer} \)

These are the negative and positive counting numbers (..., -3, -2, -1, 0, 1, 2, 3, ...)
\item \( \text{Not a Real number} \)

These are Nonreal Complex numbers \textbf{OR} things that are not numbers (e.g., dividing by 0).
\item \( \text{Rational} \)

These are numbers that can be written as fraction of Integers (e.g., -2/3)
\item \( \text{Irrational} \)

These cannot be written as a fraction of Integers.
\item \( \text{Whole} \)

* This is the correct option!
\end{enumerate}

\textbf{General Comment:} First, you \textbf{NEED} to simplify the expression. This question simplifies to $195$. 
 
 Be sure you look at the simplified fraction and not just the decimal expansion. Numbers such as 13, 17, and 19 provide \textbf{long but repeating/terminating decimal expansions!} 
 
 The only ways to *not* be a Real number are: dividing by 0 or taking the square root of a negative number. 
 
 Irrational numbers are more than just square root of 3: adding or subtracting values from square root of 3 is also irrational.
}
\end{enumerate}

\end{document}