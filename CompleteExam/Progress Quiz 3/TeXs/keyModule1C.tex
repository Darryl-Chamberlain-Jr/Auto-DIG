\documentclass{extbook}[14pt]
\usepackage{multicol, enumerate, enumitem, hyperref, color, soul, setspace, parskip, fancyhdr, amssymb, amsthm, amsmath, bbm, latexsym, units, mathtools}
\everymath{\displaystyle}
\usepackage[headsep=0.5cm,headheight=0cm, left=1 in,right= 1 in,top= 1 in,bottom= 1 in]{geometry}
\usepackage{dashrule}  % Package to use the command below to create lines between items
\newcommand{\litem}[1]{\item #1

\rule{\textwidth}{0.4pt}}
\pagestyle{fancy}
\lhead{}
\chead{Answer Key for Progress Quiz 3 Version C}
\rhead{}
\lfoot{}
\cfoot{}
\rfoot{Fall 2020}
\begin{document}
\textbf{This key should allow you to understand why you choose the option you did (beyond just getting a question right or wrong). \href{https://xronos.clas.ufl.edu/mac1105spring2020/courseDescriptionAndMisc/Exams/LearningFromResults}{More instructions on how to use this key can be found here}.}

\textbf{If you have a suggestion to make the keys better, \href{https://forms.gle/CZkbZmPbC9XALEE88}{please fill out the short survey here}.}

\textit{Note: This key is auto-generated and may contain issues and/or errors. The keys are reviewed after each exam to ensure grading is done accurately. If there are issues (like duplicate options), they are noted in the offline gradebook. The keys are a work-in-progress to give students as many resources to improve as possible.}

\rule{\textwidth}{0.4pt}

\begin{enumerate}\litem{
Simplify the expression below and choose the interval the simplification is contained within.
\[ 17 - 16 \div 15 * 9 - (6 * 20) \]
The solution is \( -112.600 \), which is option D.\begin{enumerate}[label=\Alph*.]
\item \( [-110.12, -97.12] \)

 -103.119, which corresponds to an Order of Operations error: not reading left-to-right for multiplication/division.
\item \( [27, 31] \)

 28.000, which corresponds to not distributing a negative correctly.
\item \( [136.88, 141.88] \)

 136.881, which corresponds to not distributing addition and subtraction correctly.
\item \( [-117.6, -110.6] \)

* -112.600, which is the correct option.
\item \( \text{None of the above} \)

 You may have gotten this by making an unanticipated error. If you got a value that is not any of the others, please let the coordinator know so they can help you figure out what happened.
\end{enumerate}

\textbf{General Comment:} While you may remember (or were taught) PEMDAS is done in order, it is actually done as P/E/MD/AS. When we are at MD or AS, we read left to right.
}
\litem{
Choose the \textbf{smallest} set of Real numbers that the number below belongs to.
\[ \sqrt{\frac{74529}{441}} \]
The solution is \( \text{Whole} \), which is option E.\begin{enumerate}[label=\Alph*.]
\item \( \text{Not a Real number} \)

These are Nonreal Complex numbers \textbf{OR} things that are not numbers (e.g., dividing by 0).
\item \( \text{Irrational} \)

These cannot be written as a fraction of Integers.
\item \( \text{Integer} \)

These are the negative and positive counting numbers (..., -3, -2, -1, 0, 1, 2, 3, ...)
\item \( \text{Rational} \)

These are numbers that can be written as fraction of Integers (e.g., -2/3)
\item \( \text{Whole} \)

* This is the correct option!
\end{enumerate}

\textbf{General Comment:} First, you \textbf{NEED} to simplify the expression. This question simplifies to $273$. 
 
 Be sure you look at the simplified fraction and not just the decimal expansion. Numbers such as 13, 17, and 19 provide \textbf{long but repeating/terminating decimal expansions!} 
 
 The only ways to *not* be a Real number are: dividing by 0 or taking the square root of a negative number. 
 
 Irrational numbers are more than just square root of 3: adding or subtracting values from square root of 3 is also irrational.
}
\litem{
Simplify the expression below into the form $a+bi$. Then, choose the intervals that $a$ and $b$ belong to.
\[ (3 + 8 i)(2 + 5 i) \]
The solution is \( -34 + 31 i \), which is option A.\begin{enumerate}[label=\Alph*.]
\item \( a \in [-38, -29] \text{ and } b \in [29.6, 33.4] \)

* $-34 + 31 i$, which is the correct option.
\item \( a \in [46, 47] \text{ and } b \in [-1.1, 0.5] \)

 $46 - i$, which corresponds to adding a minus sign in the first term.
\item \( a \in [5, 8] \text{ and } b \in [38, 41] \)

 $6 + 40 i$, which corresponds to just multiplying the real terms to get the real part of the solution and the coefficients in the complex terms to get the complex part.
\item \( a \in [-38, -29] \text{ and } b \in [-32.6, -30.4] \)

 $-34 - 31 i$, which corresponds to adding a minus sign in both terms.
\item \( a \in [46, 47] \text{ and } b \in [-0.1, 6.2] \)

 $46 + i$, which corresponds to adding a minus sign in the second term.
\end{enumerate}

\textbf{General Comment:} You can treat $i$ as a variable and distribute. Just remember that $i^2=-1$, so you can continue to reduce after you distribute.
}
\litem{
Choose the \textbf{smallest} set of Real numbers that the number below belongs to.
\[ -\sqrt{\frac{484}{81}} \]
The solution is \( \text{Rational} \), which is option B.\begin{enumerate}[label=\Alph*.]
\item \( \text{Integer} \)

These are the negative and positive counting numbers (..., -3, -2, -1, 0, 1, 2, 3, ...)
\item \( \text{Rational} \)

* This is the correct option!
\item \( \text{Whole} \)

These are the counting numbers with 0 (0, 1, 2, 3, ...)
\item \( \text{Irrational} \)

These cannot be written as a fraction of Integers.
\item \( \text{Not a Real number} \)

These are Nonreal Complex numbers \textbf{OR} things that are not numbers (e.g., dividing by 0).
\end{enumerate}

\textbf{General Comment:} First, you \textbf{NEED} to simplify the expression. This question simplifies to $-\frac{22}{9}$. 
 
 Be sure you look at the simplified fraction and not just the decimal expansion. Numbers such as 13, 17, and 19 provide \textbf{long but repeating/terminating decimal expansions!} 
 
 The only ways to *not* be a Real number are: dividing by 0 or taking the square root of a negative number. 
 
 Irrational numbers are more than just square root of 3: adding or subtracting values from square root of 3 is also irrational.
}
\litem{
Simplify the expression below and choose the interval the simplification is contained within.
\[ 3 - 11 \div 16 * 15 - (10 * 19) \]
The solution is \( -197.312 \), which is option D.\begin{enumerate}[label=\Alph*.]
\item \( [190.95, 194.95] \)

 192.954, which corresponds to not distributing addition and subtraction correctly.
\item \( [-190.05, -181.05] \)

 -187.046, which corresponds to an Order of Operations error: not reading left-to-right for multiplication/division.
\item \( [-330.94, -325.94] \)

 -328.938, which corresponds to not distributing a negative correctly.
\item \( [-200.31, -190.31] \)

* -197.312, which is the correct option.
\item \( \text{None of the above} \)

 You may have gotten this by making an unanticipated error. If you got a value that is not any of the others, please let the coordinator know so they can help you figure out what happened.
\end{enumerate}

\textbf{General Comment:} While you may remember (or were taught) PEMDAS is done in order, it is actually done as P/E/MD/AS. When we are at MD or AS, we read left to right.
}
\litem{
Choose the \textbf{smallest} set of Complex numbers that the number below belongs to.
\[ \frac{0}{-8 \pi}+\sqrt{6}i \]
The solution is \( \text{Pure Imaginary} \), which is option A.\begin{enumerate}[label=\Alph*.]
\item \( \text{Pure Imaginary} \)

* This is the correct option!
\item \( \text{Rational} \)

These are numbers that can be written as fraction of Integers (e.g., -2/3 + 5)
\item \( \text{Nonreal Complex} \)

This is a Complex number $(a+bi)$ that is not Real (has $i$ as part of the number).
\item \( \text{Irrational} \)

These cannot be written as a fraction of Integers. Remember: $\pi$ is not an Integer!
\item \( \text{Not a Complex Number} \)

This is not a number. The only non-Complex number we know is dividing by 0 as this is not a number!
\end{enumerate}

\textbf{General Comment:} Be sure to simplify $i^2 = -1$. This may remove the imaginary portion for your number. If you are having trouble, you may want to look at the \textit{Subgroups of the Real Numbers} section.
}
\litem{
Simplify the expression below into the form $a+bi$. Then, choose the intervals that $a$ and $b$ belong to.
\[ (6 + 10 i)(-9 + 2 i) \]
The solution is \( -74 - 78 i \), which is option E.\begin{enumerate}[label=\Alph*.]
\item \( a \in [-80, -72] \text{ and } b \in [78, 82] \)

 $-74 + 78 i$, which corresponds to adding a minus sign in both terms.
\item \( a \in [-58, -50] \text{ and } b \in [20, 25] \)

 $-54 + 20 i$, which corresponds to just multiplying the real terms to get the real part of the solution and the coefficients in the complex terms to get the complex part.
\item \( a \in [-36, -31] \text{ and } b \in [102, 105] \)

 $-34 + 102 i$, which corresponds to adding a minus sign in the first term.
\item \( a \in [-36, -31] \text{ and } b \in [-103, -98] \)

 $-34 - 102 i$, which corresponds to adding a minus sign in the second term.
\item \( a \in [-80, -72] \text{ and } b \in [-82, -77] \)

* $-74 - 78 i$, which is the correct option.
\end{enumerate}

\textbf{General Comment:} You can treat $i$ as a variable and distribute. Just remember that $i^2=-1$, so you can continue to reduce after you distribute.
}
\litem{
Simplify the expression below into the form $a+bi$. Then, choose the intervals that $a$ and $b$ belong to.
\[ \frac{-9 - 44 i}{-2 - 7 i} \]
The solution is \( 6.15  + 0.47 i \), which is option B.\begin{enumerate}[label=\Alph*.]
\item \( a \in [5.5, 6.5] \text{ and } b \in [24, 26] \)

 $6.15  + 25.00 i$, which corresponds to forgetting to multiply the conjugate by the numerator.
\item \( a \in [5.5, 6.5] \text{ and } b \in [0, 2] \)

* $6.15  + 0.47 i$, which is the correct option.
\item \( a \in [-6, -4.5] \text{ and } b \in [2, 4] \)

 $-5.47  + 2.85 i$, which corresponds to forgetting to multiply the conjugate by the numerator and not computing the conjugate correctly.
\item \( a \in [4, 5.5] \text{ and } b \in [5.5, 7.5] \)

 $4.50  + 6.29 i$, which corresponds to just dividing the first term by the first term and the second by the second.
\item \( a \in [325.5, 326.5] \text{ and } b \in [0, 2] \)

 $326.00  + 0.47 i$, which corresponds to forgetting to multiply the conjugate by the numerator and using a plus instead of a minus in the denominator.
\end{enumerate}

\textbf{General Comment:} Multiply the numerator and denominator by the *conjugate* of the denominator, then simplify. For example, if we have $2+3i$, the conjugate is $2-3i$.
}
\litem{
Choose the \textbf{smallest} set of Complex numbers that the number below belongs to.
\[ \sqrt{\frac{0}{36}}+\sqrt{5}i \]
The solution is \( \text{Pure Imaginary} \), which is option C.\begin{enumerate}[label=\Alph*.]
\item \( \text{Not a Complex Number} \)

This is not a number. The only non-Complex number we know is dividing by 0 as this is not a number!
\item \( \text{Nonreal Complex} \)

This is a Complex number $(a+bi)$ that is not Real (has $i$ as part of the number).
\item \( \text{Pure Imaginary} \)

* This is the correct option!
\item \( \text{Rational} \)

These are numbers that can be written as fraction of Integers (e.g., -2/3 + 5)
\item \( \text{Irrational} \)

These cannot be written as a fraction of Integers. Remember: $\pi$ is not an Integer!
\end{enumerate}

\textbf{General Comment:} Be sure to simplify $i^2 = -1$. This may remove the imaginary portion for your number. If you are having trouble, you may want to look at the \textit{Subgroups of the Real Numbers} section.
}
\litem{
Simplify the expression below into the form $a+bi$. Then, choose the intervals that $a$ and $b$ belong to.
\[ \frac{45 + 11 i}{-3 - 4 i} \]
The solution is \( -7.16  + 5.88 i \), which is option A.\begin{enumerate}[label=\Alph*.]
\item \( a \in [-7.5, -6.5] \text{ and } b \in [4.5, 6] \)

* $-7.16  + 5.88 i$, which is the correct option.
\item \( a \in [-180, -178] \text{ and } b \in [4.5, 6] \)

 $-179.00  + 5.88 i$, which corresponds to forgetting to multiply the conjugate by the numerator and using a plus instead of a minus in the denominator.
\item \( a \in [-4, -2.5] \text{ and } b \in [-10.5, -8] \)

 $-3.64  - 8.52 i$, which corresponds to forgetting to multiply the conjugate by the numerator and not computing the conjugate correctly.
\item \( a \in [-7.5, -6.5] \text{ and } b \in [145, 148] \)

 $-7.16  + 147.00 i$, which corresponds to forgetting to multiply the conjugate by the numerator.
\item \( a \in [-16, -14.5] \text{ and } b \in [-3.5, -2] \)

 $-15.00  - 2.75 i$, which corresponds to just dividing the first term by the first term and the second by the second.
\end{enumerate}

\textbf{General Comment:} Multiply the numerator and denominator by the *conjugate* of the denominator, then simplify. For example, if we have $2+3i$, the conjugate is $2-3i$.
}
\end{enumerate}

\end{document}