\documentclass{extbook}[14pt]
\usepackage{multicol, enumerate, enumitem, hyperref, color, soul, setspace, parskip, fancyhdr, amssymb, amsthm, amsmath, latexsym, units, mathtools}
\everymath{\displaystyle}
\usepackage[headsep=0.5cm,headheight=0cm, left=1 in,right= 1 in,top= 1 in,bottom= 1 in]{geometry}
\usepackage{dashrule}  % Package to use the command below to create lines between items
\newcommand{\litem}[1]{\item #1

\rule{\textwidth}{0.4pt}}
\pagestyle{fancy}
\lhead{}
\chead{Answer Key for Progress Quiz 3 Version B}
\rhead{}
\lfoot{3012-8528}
\cfoot{}
\rfoot{Summer C 2021}
\begin{document}
\textbf{This key should allow you to understand why you choose the option you did (beyond just getting a question right or wrong). \href{https://xronos.clas.ufl.edu/mac1105spring2020/courseDescriptionAndMisc/Exams/LearningFromResults}{More instructions on how to use this key can be found here}.}

\textbf{If you have a suggestion to make the keys better, \href{https://forms.gle/CZkbZmPbC9XALEE88}{please fill out the short survey here}.}

\textit{Note: This key is auto-generated and may contain issues and/or errors. The keys are reviewed after each exam to ensure grading is done accurately. If there are issues (like duplicate options), they are noted in the offline gradebook. The keys are a work-in-progress to give students as many resources to improve as possible.}

\rule{\textwidth}{0.4pt}

\begin{enumerate}\litem{
Simplify the expression below into the form $a+bi$. Then, choose the intervals that $a$ and $b$ belong to.
\[ (10 - 4 i)(-6 - 8 i) \]The solution is \( -92 - 56 i \), which is option D.\begin{enumerate}[label=\Alph*.]
\item \( a \in [-38, -25] \text{ and } b \in [-104, -102] \)

 $-28 - 104 i$, which corresponds to adding a minus sign in the first term.
\item \( a \in [-38, -25] \text{ and } b \in [104, 108] \)

 $-28 + 104 i$, which corresponds to adding a minus sign in the second term.
\item \( a \in [-92, -87] \text{ and } b \in [52, 57] \)

 $-92 + 56 i$, which corresponds to adding a minus sign in both terms.
\item \( a \in [-92, -87] \text{ and } b \in [-56, -54] \)

* $-92 - 56 i$, which is the correct option.
\item \( a \in [-63, -58] \text{ and } b \in [28, 35] \)

 $-60 + 32 i$, which corresponds to just multiplying the real terms to get the real part of the solution and the coefficients in the complex terms to get the complex part.
\end{enumerate}

\textbf{General Comment:} You can treat $i$ as a variable and distribute. Just remember that $i^2=-1$, so you can continue to reduce after you distribute.
}
\litem{
Simplify the expression below and choose the interval the simplification is contained within.
\[ 12 - 8 \div 13 * 16 - (15 * 14) \]The solution is \( -207.846 \), which is option C.\begin{enumerate}[label=\Alph*.]
\item \( [-202.04, -193.04] \)

 -198.038, which corresponds to an Order of Operations error: not reading left-to-right for multiplication/division.
\item \( [-182.85, -176.85] \)

 -179.846, which corresponds to not distributing a negative correctly.
\item \( [-213.85, -203.85] \)

* -207.846, which is the correct option.
\item \( [219.96, 223.96] \)

 221.962, which corresponds to not distributing addition and subtraction correctly.
\item \( \text{None of the above} \)

 You may have gotten this by making an unanticipated error. If you got a value that is not any of the others, please let the coordinator know so they can help you figure out what happened.
\end{enumerate}

\textbf{General Comment:} While you may remember (or were taught) PEMDAS is done in order, it is actually done as P/E/MD/AS. When we are at MD or AS, we read left to right.
}
\litem{
Choose the \textbf{smallest} set of Complex numbers that the number below belongs to.
\[ \sqrt{\frac{-567}{9}} i+\sqrt{55}i \]The solution is \( \text{Nonreal Complex} \), which is option C.\begin{enumerate}[label=\Alph*.]
\item \( \text{Not a Complex Number} \)

This is not a number. The only non-Complex number we know is dividing by 0 as this is not a number!
\item \( \text{Pure Imaginary} \)

This is a Complex number $(a+bi)$ that \textbf{only} has an imaginary part like $2i$.
\item \( \text{Nonreal Complex} \)

* This is the correct option!
\item \( \text{Rational} \)

These are numbers that can be written as fraction of Integers (e.g., -2/3 + 5)
\item \( \text{Irrational} \)

These cannot be written as a fraction of Integers. Remember: $\pi$ is not an Integer!
\end{enumerate}

\textbf{General Comment:} Be sure to simplify $i^2 = -1$. This may remove the imaginary portion for your number. If you are having trouble, you may want to look at the \textit{Subgroups of the Real Numbers} section.
}
\litem{
Choose the \textbf{smallest} set of Complex numbers that the number below belongs to.
\[ \frac{\sqrt{119}}{20}+\sqrt{-6}i \]The solution is \( \text{Irrational} \), which is option C.\begin{enumerate}[label=\Alph*.]
\item \( \text{Pure Imaginary} \)

This is a Complex number $(a+bi)$ that \textbf{only} has an imaginary part like $2i$.
\item \( \text{Nonreal Complex} \)

This is a Complex number $(a+bi)$ that is not Real (has $i$ as part of the number).
\item \( \text{Irrational} \)

* This is the correct option!
\item \( \text{Not a Complex Number} \)

This is not a number. The only non-Complex number we know is dividing by 0 as this is not a number!
\item \( \text{Rational} \)

These are numbers that can be written as fraction of Integers (e.g., -2/3 + 5)
\end{enumerate}

\textbf{General Comment:} Be sure to simplify $i^2 = -1$. This may remove the imaginary portion for your number. If you are having trouble, you may want to look at the \textit{Subgroups of the Real Numbers} section.
}
\litem{
Simplify the expression below into the form $a+bi$. Then, choose the intervals that $a$ and $b$ belong to.
\[ \frac{45 - 22 i}{7 - 4 i} \]The solution is \( 6.20  + 0.40 i \), which is option A.\begin{enumerate}[label=\Alph*.]
\item \( a \in [6.04, 6.3] \text{ and } b \in [-0.5, 1] \)

* $6.20  + 0.40 i$, which is the correct option.
\item \( a \in [6.04, 6.3] \text{ and } b \in [25.5, 28] \)

 $6.20  + 26.00 i$, which corresponds to forgetting to multiply the conjugate by the numerator.
\item \( a \in [402.98, 403.23] \text{ and } b \in [-0.5, 1] \)

 $403.00  + 0.40 i$, which corresponds to forgetting to multiply the conjugate by the numerator and using a plus instead of a minus in the denominator.
\item \( a \in [3.36, 3.52] \text{ and } b \in [-6, -4.5] \)

 $3.49  - 5.14 i$, which corresponds to forgetting to multiply the conjugate by the numerator and not computing the conjugate correctly.
\item \( a \in [6.28, 6.6] \text{ and } b \in [5, 6] \)

 $6.43  + 5.50 i$, which corresponds to just dividing the first term by the first term and the second by the second.
\end{enumerate}

\textbf{General Comment:} Multiply the numerator and denominator by the *conjugate* of the denominator, then simplify. For example, if we have $2+3i$, the conjugate is $2-3i$.
}
\litem{
Simplify the expression below and choose the interval the simplification is contained within.
\[ 9 - 4^2 + 16 \div 18 * 7 \div 17 \]The solution is \( -6.634 \), which is option D.\begin{enumerate}[label=\Alph*.]
\item \( [25.34, 26.01] \)

 25.366, which corresponds to an Order of Operations error: multiplying by negative before squaring. For example: $(-3)^2 \neq -3^2$
\item \( [-7.31, -6.88] \)

 -6.993, which corresponds to an Order of Operations error: not reading left-to-right for multiplication/division.
\item \( [24.16, 25.14] \)

 25.007, which corresponds to two Order of Operations errors.
\item \( [-6.66, -6.5] \)

* -6.634, this is the correct option
\item \( \text{None of the above} \)

 You may have gotten this by making an unanticipated error. If you got a value that is not any of the others, please let the coordinator know so they can help you figure out what happened.
\end{enumerate}

\textbf{General Comment:} While you may remember (or were taught) PEMDAS is done in order, it is actually done as P/E/MD/AS. When we are at MD or AS, we read left to right.
}
\litem{
Simplify the expression below into the form $a+bi$. Then, choose the intervals that $a$ and $b$ belong to.
\[ (6 - 3 i)(7 - 10 i) \]The solution is \( 12 - 81 i \), which is option E.\begin{enumerate}[label=\Alph*.]
\item \( a \in [36, 44] \text{ and } b \in [28, 35] \)

 $42 + 30 i$, which corresponds to just multiplying the real terms to get the real part of the solution and the coefficients in the complex terms to get the complex part.
\item \( a \in [9, 14] \text{ and } b \in [80, 82] \)

 $12 + 81 i$, which corresponds to adding a minus sign in both terms.
\item \( a \in [66, 75] \text{ and } b \in [-40, -38] \)

 $72 - 39 i$, which corresponds to adding a minus sign in the first term.
\item \( a \in [66, 75] \text{ and } b \in [38, 47] \)

 $72 + 39 i$, which corresponds to adding a minus sign in the second term.
\item \( a \in [9, 14] \text{ and } b \in [-81, -77] \)

* $12 - 81 i$, which is the correct option.
\end{enumerate}

\textbf{General Comment:} You can treat $i$ as a variable and distribute. Just remember that $i^2=-1$, so you can continue to reduce after you distribute.
}
\litem{
Simplify the expression below into the form $a+bi$. Then, choose the intervals that $a$ and $b$ belong to.
\[ \frac{-63 + 33 i}{-6 + 2 i} \]The solution is \( 11.10  - 1.80 i \), which is option C.\begin{enumerate}[label=\Alph*.]
\item \( a \in [7.5, 8.5] \text{ and } b \in [-9, -7] \)

 $7.80  - 8.10 i$, which corresponds to forgetting to multiply the conjugate by the numerator and not computing the conjugate correctly.
\item \( a \in [10, 11] \text{ and } b \in [15.5, 17] \)

 $10.50  + 16.50 i$, which corresponds to just dividing the first term by the first term and the second by the second.
\item \( a \in [11, 12] \text{ and } b \in [-2.5, -0.5] \)

* $11.10  - 1.80 i$, which is the correct option.
\item \( a \in [443, 445] \text{ and } b \in [-2.5, -0.5] \)

 $444.00  - 1.80 i$, which corresponds to forgetting to multiply the conjugate by the numerator and using a plus instead of a minus in the denominator.
\item \( a \in [11, 12] \text{ and } b \in [-73.5, -71.5] \)

 $11.10  - 72.00 i$, which corresponds to forgetting to multiply the conjugate by the numerator.
\end{enumerate}

\textbf{General Comment:} Multiply the numerator and denominator by the *conjugate* of the denominator, then simplify. For example, if we have $2+3i$, the conjugate is $2-3i$.
}
\litem{
Choose the \textbf{smallest} set of Real numbers that the number below belongs to.
\[ -\sqrt{\frac{102400}{256}} \]The solution is \( \text{Integer} \), which is option B.\begin{enumerate}[label=\Alph*.]
\item \( \text{Rational} \)

These are numbers that can be written as fraction of Integers (e.g., -2/3)
\item \( \text{Integer} \)

* This is the correct option!
\item \( \text{Irrational} \)

These cannot be written as a fraction of Integers.
\item \( \text{Whole} \)

These are the counting numbers with 0 (0, 1, 2, 3, ...)
\item \( \text{Not a Real number} \)

These are Nonreal Complex numbers \textbf{OR} things that are not numbers (e.g., dividing by 0).
\end{enumerate}

\textbf{General Comment:} First, you \textbf{NEED} to simplify the expression. This question simplifies to $-320$. 
 
 Be sure you look at the simplified fraction and not just the decimal expansion. Numbers such as 13, 17, and 19 provide \textbf{long but repeating/terminating decimal expansions!} 
 
 The only ways to *not* be a Real number are: dividing by 0 or taking the square root of a negative number. 
 
 Irrational numbers are more than just square root of 3: adding or subtracting values from square root of 3 is also irrational.
}
\litem{
Choose the \textbf{smallest} set of Real numbers that the number below belongs to.
\[ -\sqrt{\frac{14}{0}} \]The solution is \( \text{Not a Real number} \), which is option C.\begin{enumerate}[label=\Alph*.]
\item \( \text{Rational} \)

These are numbers that can be written as fraction of Integers (e.g., -2/3)
\item \( \text{Irrational} \)

These cannot be written as a fraction of Integers.
\item \( \text{Not a Real number} \)

* This is the correct option!
\item \( \text{Whole} \)

These are the counting numbers with 0 (0, 1, 2, 3, ...)
\item \( \text{Integer} \)

These are the negative and positive counting numbers (..., -3, -2, -1, 0, 1, 2, 3, ...)
\end{enumerate}

\textbf{General Comment:} First, you \textbf{NEED} to simplify the expression. This question simplifies to $-\sqrt{\frac{14}{0}}$. 
 
 Be sure you look at the simplified fraction and not just the decimal expansion. Numbers such as 13, 17, and 19 provide \textbf{long but repeating/terminating decimal expansions!} 
 
 The only ways to *not* be a Real number are: dividing by 0 or taking the square root of a negative number. 
 
 Irrational numbers are more than just square root of 3: adding or subtracting values from square root of 3 is also irrational.
}
\end{enumerate}

\end{document}