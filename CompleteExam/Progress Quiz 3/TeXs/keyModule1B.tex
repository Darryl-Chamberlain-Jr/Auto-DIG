\documentclass{extbook}[14pt]
\usepackage{multicol, enumerate, enumitem, hyperref, color, soul, setspace, parskip, fancyhdr, amssymb, amsthm, amsmath, bbm, latexsym, units, mathtools}
\everymath{\displaystyle}
\usepackage[headsep=0.5cm,headheight=0cm, left=1 in,right= 1 in,top= 1 in,bottom= 1 in]{geometry}
\usepackage{dashrule}  % Package to use the command below to create lines between items
\newcommand{\litem}[1]{\item #1

\rule{\textwidth}{0.4pt}}
\pagestyle{fancy}
\lhead{}
\chead{Answer Key for Progress Quiz 3 Version B}
\rhead{}
\lfoot{3148-2249}
\cfoot{}
\rfoot{Spring 2021}
\begin{document}
\textbf{This key should allow you to understand why you choose the option you did (beyond just getting a question right or wrong). \href{https://xronos.clas.ufl.edu/mac1105spring2020/courseDescriptionAndMisc/Exams/LearningFromResults}{More instructions on how to use this key can be found here}.}

\textbf{If you have a suggestion to make the keys better, \href{https://forms.gle/CZkbZmPbC9XALEE88}{please fill out the short survey here}.}

\textit{Note: This key is auto-generated and may contain issues and/or errors. The keys are reviewed after each exam to ensure grading is done accurately. If there are issues (like duplicate options), they are noted in the offline gradebook. The keys are a work-in-progress to give students as many resources to improve as possible.}

\rule{\textwidth}{0.4pt}

\begin{enumerate}\litem{
Choose the \textbf{smallest} set of Complex numbers that the number below belongs to.
\[ \frac{0}{-2 \pi}+\sqrt{4}i \]

The solution is \( \text{Pure Imaginary} \), which is option E.\begin{enumerate}[label=\Alph*.]
\item \( \text{Rational} \)

These are numbers that can be written as fraction of Integers (e.g., -2/3 + 5)
\item \( \text{Nonreal Complex} \)

This is a Complex number $(a+bi)$ that is not Real (has $i$ as part of the number).
\item \( \text{Not a Complex Number} \)

This is not a number. The only non-Complex number we know is dividing by 0 as this is not a number!
\item \( \text{Irrational} \)

These cannot be written as a fraction of Integers. Remember: $\pi$ is not an Integer!
\item \( \text{Pure Imaginary} \)

* This is the correct option!
\end{enumerate}

\textbf{General Comment:} Be sure to simplify $i^2 = -1$. This may remove the imaginary portion for your number. If you are having trouble, you may want to look at the \textit{Subgroups of the Real Numbers} section.
}
\litem{
Simplify the expression below and choose the interval the simplification is contained within.
\[ 4 - 15 \div 6 * 14 - (3 * 19) \]

The solution is \( -88.000 \), which is option B.\begin{enumerate}[label=\Alph*.]
\item \( [-53.18, -50.18] \)

 -53.179, which corresponds to an Order of Operations error: not reading left-to-right for multiplication/division.
\item \( [-92, -83] \)

* -88.000, which is the correct option.
\item \( [56.82, 66.82] \)

 60.821, which corresponds to not distributing addition and subtraction correctly.
\item \( [-653, -641] \)

 -646.000, which corresponds to not distributing a negative correctly.
\item \( \text{None of the above} \)

 You may have gotten this by making an unanticipated error. If you got a value that is not any of the others, please let the coordinator know so they can help you figure out what happened.
\end{enumerate}

\textbf{General Comment:} While you may remember (or were taught) PEMDAS is done in order, it is actually done as P/E/MD/AS. When we are at MD or AS, we read left to right.
}
\litem{
Simplify the expression below and choose the interval the simplification is contained within.
\[ 11 - 4^2 + 5 \div 3 * 16 \div 18 \]

The solution is \( -3.519 \), which is option D.\begin{enumerate}[label=\Alph*.]
\item \( [-7, -4.2] \)

 -4.994, which corresponds to an Order of Operations error: not reading left-to-right for multiplication/division.
\item \( [27.7, 29.2] \)

 28.481, which corresponds to an Order of Operations error: multiplying by negative before squaring. For example: $(-3)^2 \neq -3^2$
\item \( [24, 27.1] \)

 27.006, which corresponds to two Order of Operations errors.
\item \( [-3.9, -1.8] \)

* -3.519, this is the correct option
\item \( \text{None of the above} \)

 You may have gotten this by making an unanticipated error. If you got a value that is not any of the others, please let the coordinator know so they can help you figure out what happened.
\end{enumerate}

\textbf{General Comment:} While you may remember (or were taught) PEMDAS is done in order, it is actually done as P/E/MD/AS. When we are at MD or AS, we read left to right.
}
\litem{
Choose the \textbf{smallest} set of Real numbers that the number below belongs to.
\[ -\sqrt{\frac{14}{0}} \]

The solution is \( \text{Not a Real number} \), which is option D.\begin{enumerate}[label=\Alph*.]
\item \( \text{Rational} \)

These are numbers that can be written as fraction of Integers (e.g., -2/3)
\item \( \text{Irrational} \)

These cannot be written as a fraction of Integers.
\item \( \text{Integer} \)

These are the negative and positive counting numbers (..., -3, -2, -1, 0, 1, 2, 3, ...)
\item \( \text{Not a Real number} \)

* This is the correct option!
\item \( \text{Whole} \)

These are the counting numbers with 0 (0, 1, 2, 3, ...)
\end{enumerate}

\textbf{General Comment:} First, you \textbf{NEED} to simplify the expression. This question simplifies to $-\sqrt{\frac{14}{0}}$. 
 
 Be sure you look at the simplified fraction and not just the decimal expansion. Numbers such as 13, 17, and 19 provide \textbf{long but repeating/terminating decimal expansions!} 
 
 The only ways to *not* be a Real number are: dividing by 0 or taking the square root of a negative number. 
 
 Irrational numbers are more than just square root of 3: adding or subtracting values from square root of 3 is also irrational.
}
\litem{
Simplify the expression below into the form $a+bi$. Then, choose the intervals that $a$ and $b$ belong to.
\[ \frac{36 - 22 i}{8 - i} \]

The solution is \( 4.77  - 2.15 i \), which is option C.\begin{enumerate}[label=\Alph*.]
\item \( a \in [309.95, 310.2] \text{ and } b \in [-2.5, -1] \)

 $310.00  - 2.15 i$, which corresponds to forgetting to multiply the conjugate by the numerator and using a plus instead of a minus in the denominator.
\item \( a \in [4.55, 5.35] \text{ and } b \in [-140.5, -138.5] \)

 $4.77  - 140.00 i$, which corresponds to forgetting to multiply the conjugate by the numerator.
\item \( a \in [4.55, 5.35] \text{ and } b \in [-2.5, -1] \)

* $4.77  - 2.15 i$, which is the correct option.
\item \( a \in [4.15, 4.65] \text{ and } b \in [21.5, 22.5] \)

 $4.50  + 22.00 i$, which corresponds to just dividing the first term by the first term and the second by the second.
\item \( a \in [3.65, 4.25] \text{ and } b \in [-3.5, -3] \)

 $4.09  - 3.26 i$, which corresponds to forgetting to multiply the conjugate by the numerator and not computing the conjugate correctly.
\end{enumerate}

\textbf{General Comment:} Multiply the numerator and denominator by the *conjugate* of the denominator, then simplify. For example, if we have $2+3i$, the conjugate is $2-3i$.
}
\litem{
Choose the \textbf{smallest} set of Complex numbers that the number below belongs to.
\[ \frac{-21}{0}+\sqrt{99} i \]

The solution is \( \text{Not a Complex Number} \), which is option B.\begin{enumerate}[label=\Alph*.]
\item \( \text{Nonreal Complex} \)

This is a Complex number $(a+bi)$ that is not Real (has $i$ as part of the number).
\item \( \text{Not a Complex Number} \)

* This is the correct option!
\item \( \text{Rational} \)

These are numbers that can be written as fraction of Integers (e.g., -2/3 + 5)
\item \( \text{Pure Imaginary} \)

This is a Complex number $(a+bi)$ that \textbf{only} has an imaginary part like $2i$.
\item \( \text{Irrational} \)

These cannot be written as a fraction of Integers. Remember: $\pi$ is not an Integer!
\end{enumerate}

\textbf{General Comment:} Be sure to simplify $i^2 = -1$. This may remove the imaginary portion for your number. If you are having trouble, you may want to look at the \textit{Subgroups of the Real Numbers} section.
}
\litem{
Simplify the expression below into the form $a+bi$. Then, choose the intervals that $a$ and $b$ belong to.
\[ (9 - 7 i)(-5 + 4 i) \]

The solution is \( -17 + 71 i \), which is option C.\begin{enumerate}[label=\Alph*.]
\item \( a \in [-20, -13] \text{ and } b \in [-73, -67] \)

 $-17 - 71 i$, which corresponds to adding a minus sign in both terms.
\item \( a \in [-56, -44] \text{ and } b \in [-32, -24] \)

 $-45 - 28 i$, which corresponds to just multiplying the real terms to get the real part of the solution and the coefficients in the complex terms to get the complex part.
\item \( a \in [-20, -13] \text{ and } b \in [71, 74] \)

* $-17 + 71 i$, which is the correct option.
\item \( a \in [-74, -72] \text{ and } b \in [1, 6] \)

 $-73 + i$, which corresponds to adding a minus sign in the first term.
\item \( a \in [-74, -72] \text{ and } b \in [-2, 0] \)

 $-73 - i$, which corresponds to adding a minus sign in the second term.
\end{enumerate}

\textbf{General Comment:} You can treat $i$ as a variable and distribute. Just remember that $i^2=-1$, so you can continue to reduce after you distribute.
}
\litem{
Simplify the expression below into the form $a+bi$. Then, choose the intervals that $a$ and $b$ belong to.
\[ \frac{-27 - 77 i}{4 - i} \]

The solution is \( -1.82  - 19.71 i \), which is option D.\begin{enumerate}[label=\Alph*.]
\item \( a \in [-11.5, -9.5] \text{ and } b \in [-17, -16] \)

 $-10.88  - 16.53 i$, which corresponds to forgetting to multiply the conjugate by the numerator and not computing the conjugate correctly.
\item \( a \in [-31.5, -29.5] \text{ and } b \in [-21, -19] \)

 $-31.00  - 19.71 i$, which corresponds to forgetting to multiply the conjugate by the numerator and using a plus instead of a minus in the denominator.
\item \( a \in [-8.5, -6] \text{ and } b \in [76.5, 78] \)

 $-6.75  + 77.00 i$, which corresponds to just dividing the first term by the first term and the second by the second.
\item \( a \in [-3, -1.5] \text{ and } b \in [-21, -19] \)

* $-1.82  - 19.71 i$, which is the correct option.
\item \( a \in [-3, -1.5] \text{ and } b \in [-336, -333] \)

 $-1.82  - 335.00 i$, which corresponds to forgetting to multiply the conjugate by the numerator.
\end{enumerate}

\textbf{General Comment:} Multiply the numerator and denominator by the *conjugate* of the denominator, then simplify. For example, if we have $2+3i$, the conjugate is $2-3i$.
}
\litem{
Simplify the expression below into the form $a+bi$. Then, choose the intervals that $a$ and $b$ belong to.
\[ (8 - 2 i)(6 + 3 i) \]

The solution is \( 54 + 12 i \), which is option E.\begin{enumerate}[label=\Alph*.]
\item \( a \in [53, 55] \text{ and } b \in [-12, -7] \)

 $54 - 12 i$, which corresponds to adding a minus sign in both terms.
\item \( a \in [38, 43] \text{ and } b \in [33, 38] \)

 $42 + 36 i$, which corresponds to adding a minus sign in the first term.
\item \( a \in [38, 43] \text{ and } b \in [-40, -30] \)

 $42 - 36 i$, which corresponds to adding a minus sign in the second term.
\item \( a \in [46, 52] \text{ and } b \in [-7, -4] \)

 $48 - 6 i$, which corresponds to just multiplying the real terms to get the real part of the solution and the coefficients in the complex terms to get the complex part.
\item \( a \in [53, 55] \text{ and } b \in [5, 14] \)

* $54 + 12 i$, which is the correct option.
\end{enumerate}

\textbf{General Comment:} You can treat $i$ as a variable and distribute. Just remember that $i^2=-1$, so you can continue to reduce after you distribute.
}
\litem{
Choose the \textbf{smallest} set of Real numbers that the number below belongs to.
\[ -\sqrt{\frac{8100}{36}} \]

The solution is \( \text{Integer} \), which is option C.\begin{enumerate}[label=\Alph*.]
\item \( \text{Irrational} \)

These cannot be written as a fraction of Integers.
\item \( \text{Not a Real number} \)

These are Nonreal Complex numbers \textbf{OR} things that are not numbers (e.g., dividing by 0).
\item \( \text{Integer} \)

* This is the correct option!
\item \( \text{Whole} \)

These are the counting numbers with 0 (0, 1, 2, 3, ...)
\item \( \text{Rational} \)

These are numbers that can be written as fraction of Integers (e.g., -2/3)
\end{enumerate}

\textbf{General Comment:} First, you \textbf{NEED} to simplify the expression. This question simplifies to $-90$. 
 
 Be sure you look at the simplified fraction and not just the decimal expansion. Numbers such as 13, 17, and 19 provide \textbf{long but repeating/terminating decimal expansions!} 
 
 The only ways to *not* be a Real number are: dividing by 0 or taking the square root of a negative number. 
 
 Irrational numbers are more than just square root of 3: adding or subtracting values from square root of 3 is also irrational.
}
\end{enumerate}

\end{document}