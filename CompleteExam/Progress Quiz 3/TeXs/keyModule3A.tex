\documentclass{extbook}[14pt]
\usepackage{multicol, enumerate, enumitem, hyperref, color, soul, setspace, parskip, fancyhdr, amssymb, amsthm, amsmath, latexsym, units, mathtools}
\everymath{\displaystyle}
\usepackage[headsep=0.5cm,headheight=0cm, left=1 in,right= 1 in,top= 1 in,bottom= 1 in]{geometry}
\usepackage{dashrule}  % Package to use the command below to create lines between items
\newcommand{\litem}[1]{\item #1

\rule{\textwidth}{0.4pt}}
\pagestyle{fancy}
\lhead{}
\chead{Answer Key for Progress Quiz 3 Version A}
\rhead{}
\lfoot{3012-8528}
\cfoot{}
\rfoot{Summer C 2021}
\begin{document}
\textbf{This key should allow you to understand why you choose the option you did (beyond just getting a question right or wrong). \href{https://xronos.clas.ufl.edu/mac1105spring2020/courseDescriptionAndMisc/Exams/LearningFromResults}{More instructions on how to use this key can be found here}.}

\textbf{If you have a suggestion to make the keys better, \href{https://forms.gle/CZkbZmPbC9XALEE88}{please fill out the short survey here}.}

\textit{Note: This key is auto-generated and may contain issues and/or errors. The keys are reviewed after each exam to ensure grading is done accurately. If there are issues (like duplicate options), they are noted in the offline gradebook. The keys are a work-in-progress to give students as many resources to improve as possible.}

\rule{\textwidth}{0.4pt}

\begin{enumerate}\litem{
Solve the linear inequality below. Then, choose the constant and interval combination that describes the solution set.
\[ -8 + 5 x > 6 x \text{ or } -3 + 6 x < 9 x \]The solution is \( (-\infty, -8.0) \text{ or } (-1.0, \infty) \), which is option A.\begin{enumerate}[label=\Alph*.]
\item \( (-\infty, a) \cup (b, \infty), \text{ where } a \in [-12, -6] \text{ and } b \in [-6, 0] \)

 * Correct option.
\item \( (-\infty, a) \cup (b, \infty), \text{ where } a \in [0, 3.75] \text{ and } b \in [3.75, 15.75] \)

Corresponds to inverting the inequality and negating the solution.
\item \( (-\infty, a] \cup [b, \infty), \text{ where } a \in [-9.75, -3.75] \text{ and } b \in [-8.25, 2.25] \)

Corresponds to including the endpoints (when they should be excluded).
\item \( (-\infty, a] \cup [b, \infty), \text{ where } a \in [0, 3] \text{ and } b \in [5.25, 13.5] \)

Corresponds to including the endpoints AND negating.
\item \( (-\infty, \infty) \)

Corresponds to the variable canceling, which does not happen in this instance.
\end{enumerate}

\textbf{General Comment:} When multiplying or dividing by a negative, flip the sign.
}
\litem{
Solve the linear inequality below. Then, choose the constant and interval combination that describes the solution set.
\[ \frac{7}{5} - \frac{7}{8} x \geq \frac{3}{3} x - \frac{6}{2} \]The solution is \( (-\infty, 2.347] \), which is option C.\begin{enumerate}[label=\Alph*.]
\item \( [a, \infty), \text{ where } a \in [-0.75, 3] \)

 $[2.347, \infty)$, which corresponds to switching the direction of the interval. You likely did this if you did not flip the inequality when dividing by a negative!
\item \( [a, \infty), \text{ where } a \in [-6.75, 0] \)

 $[-2.347, \infty)$, which corresponds to switching the direction of the interval AND negating the endpoint. You likely did this if you did not flip the inequality when dividing by a negative as well as not moving values over to a side properly.
\item \( (-\infty, a], \text{ where } a \in [0.75, 6] \)

* $(-\infty, 2.347]$, which is the correct option.
\item \( (-\infty, a], \text{ where } a \in [-3, 1.5] \)

 $(-\infty, -2.347]$, which corresponds to negating the endpoint of the solution.
\item \( \text{None of the above}. \)

You may have chosen this if you thought the inequality did not match the ends of the intervals.
\end{enumerate}

\textbf{General Comment:} Remember that less/greater than or equal to includes the endpoint, while less/greater do not. Also, remember that you need to flip the inequality when you multiply or divide by a negative.
}
\litem{
Using an interval or intervals, describe all the $x$-values within or including a distance of the given values.
\[ \text{ No less than } 10 \text{ units from the number } 9. \]The solution is \( (-\infty, -1] \cup [19, \infty) \), which is option B.\begin{enumerate}[label=\Alph*.]
\item \( (-\infty, -1) \cup (19, \infty) \)

This describes the values more than 10 from 9
\item \( (-\infty, -1] \cup [19, \infty) \)

This describes the values no less than 10 from 9
\item \( [-1, 19] \)

This describes the values no more than 10 from 9
\item \( (-1, 19) \)

This describes the values less than 10 from 9
\item \( \text{None of the above} \)

You likely thought the values in the interval were not correct.
\end{enumerate}

\textbf{General Comment:} When thinking about this language, it helps to draw a number line and try points.
}
\litem{
Solve the linear inequality below. Then, choose the constant and interval combination that describes the solution set.
\[ -4 - 6 x < \frac{-32 x + 8}{6} \leq -3 - 7 x \]The solution is \( \text{None of the above.} \), which is option E.\begin{enumerate}[label=\Alph*.]
\item \( (a, b], \text{ where } a \in [3.75, 9.75] \text{ and } b \in [-2.25, 4.5] \)

$(8.00, 2.60]$, which is the correct interval but negatives of the actual endpoints.
\item \( (-\infty, a) \cup [b, \infty), \text{ where } a \in [6.75, 9.75] \text{ and } b \in [-1.5, 6] \)

$(-\infty, 8.00) \cup [2.60, \infty)$, which corresponds to displaying the and-inequality as an or-inequality and getting negatives of the actual endpoints.
\item \( [a, b), \text{ where } a \in [5.25, 12.75] \text{ and } b \in [0, 6.75] \)

$[8.00, 2.60)$, which corresponds to flipping the inequality and getting negatives of the actual endpoints.
\item \( (-\infty, a] \cup (b, \infty), \text{ where } a \in [7.5, 11.25] \text{ and } b \in [0.75, 4.5] \)

$(-\infty, 8.00] \cup (2.60, \infty)$, which corresponds to displaying the and-inequality as an or-inequality AND flipping the inequality AND getting negatives of the actual endpoints.
\item \( \text{None of the above.} \)

* This is correct as the answer should be $(-8.00, -2.60]$.
\end{enumerate}

\textbf{General Comment:} To solve, you will need to break up the compound inequality into two inequalities. Be sure to keep track of the inequality! It may be best to draw a number line and graph your solution.
}
\litem{
Solve the linear inequality below. Then, choose the constant and interval combination that describes the solution set.
\[ -4 + 4 x > 5 x \text{ or } 5 + 6 x < 8 x \]The solution is \( (-\infty, -4.0) \text{ or } (2.5, \infty) \), which is option D.\begin{enumerate}[label=\Alph*.]
\item \( (-\infty, a] \cup [b, \infty), \text{ where } a \in [-5.02, -3.97] \text{ and } b \in [1.73, 2.55] \)

Corresponds to including the endpoints (when they should be excluded).
\item \( (-\infty, a] \cup [b, \infty), \text{ where } a \in [-3.52, -1.57] \text{ and } b \in [3.89, 4.51] \)

Corresponds to including the endpoints AND negating.
\item \( (-\infty, a) \cup (b, \infty), \text{ where } a \in [-2.85, -1.88] \text{ and } b \in [3, 6] \)

Corresponds to inverting the inequality and negating the solution.
\item \( (-\infty, a) \cup (b, \infty), \text{ where } a \in [-4.65, -3.3] \text{ and } b \in [0, 3] \)

 * Correct option.
\item \( (-\infty, \infty) \)

Corresponds to the variable canceling, which does not happen in this instance.
\end{enumerate}

\textbf{General Comment:} When multiplying or dividing by a negative, flip the sign.
}
\litem{
Solve the linear inequality below. Then, choose the constant and interval combination that describes the solution set.
\[ -5x + 4 < -4x + 10 \]The solution is \( (-6.0, \infty) \), which is option A.\begin{enumerate}[label=\Alph*.]
\item \( (a, \infty), \text{ where } a \in [-15, -3] \)

* $(-6.0, \infty)$, which is the correct option.
\item \( (-\infty, a), \text{ where } a \in [-7, -4] \)

 $(-\infty, -6.0)$, which corresponds to switching the direction of the interval. You likely did this if you did not flip the inequality when dividing by a negative!
\item \( (-\infty, a), \text{ where } a \in [6, 8] \)

 $(-\infty, 6.0)$, which corresponds to switching the direction of the interval AND negating the endpoint. You likely did this if you did not flip the inequality when dividing by a negative as well as not moving values over to a side properly.
\item \( (a, \infty), \text{ where } a \in [2, 11] \)

 $(6.0, \infty)$, which corresponds to negating the endpoint of the solution.
\item \( \text{None of the above}. \)

You may have chosen this if you thought the inequality did not match the ends of the intervals.
\end{enumerate}

\textbf{General Comment:} Remember that less/greater than or equal to includes the endpoint, while less/greater do not. Also, remember that you need to flip the inequality when you multiply or divide by a negative.
}
\litem{
Solve the linear inequality below. Then, choose the constant and interval combination that describes the solution set.
\[ 4x -5 > 10x -8 \]The solution is \( (-\infty, 0.5) \), which is option B.\begin{enumerate}[label=\Alph*.]
\item \( (-\infty, a), \text{ where } a \in [-0.58, -0.47] \)

 $(-\infty, -0.5)$, which corresponds to negating the endpoint of the solution.
\item \( (-\infty, a), \text{ where } a \in [-0.23, 0.73] \)

* $(-\infty, 0.5)$, which is the correct option.
\item \( (a, \infty), \text{ where } a \in [0.25, 1.38] \)

 $(0.5, \infty)$, which corresponds to switching the direction of the interval. You likely did this if you did not flip the inequality when dividing by a negative!
\item \( (a, \infty), \text{ where } a \in [-1.27, -0.08] \)

 $(-0.5, \infty)$, which corresponds to switching the direction of the interval AND negating the endpoint. You likely did this if you did not flip the inequality when dividing by a negative as well as not moving values over to a side properly.
\item \( \text{None of the above}. \)

You may have chosen this if you thought the inequality did not match the ends of the intervals.
\end{enumerate}

\textbf{General Comment:} Remember that less/greater than or equal to includes the endpoint, while less/greater do not. Also, remember that you need to flip the inequality when you multiply or divide by a negative.
}
\litem{
Solve the linear inequality below. Then, choose the constant and interval combination that describes the solution set.
\[ -5 + 6 x < \frac{77 x - 8}{9} \leq 7 + 8 x \]The solution is \( (-1.61, 14.20] \), which is option D.\begin{enumerate}[label=\Alph*.]
\item \( [a, b), \text{ where } a \in [-6.75, 0.75] \text{ and } b \in [8.25, 18] \)

$[-1.61, 14.20)$, which corresponds to flipping the inequality.
\item \( (-\infty, a) \cup [b, \infty), \text{ where } a \in [-5.25, 0] \text{ and } b \in [12, 19.5] \)

$(-\infty, -1.61) \cup [14.20, \infty)$, which corresponds to displaying the and-inequality as an or-inequality.
\item \( (-\infty, a] \cup (b, \infty), \text{ where } a \in [-4.27, 0.82] \text{ and } b \in [13.5, 15] \)

$(-\infty, -1.61] \cup (14.20, \infty)$, which corresponds to displaying the and-inequality as an or-inequality AND flipping the inequality.
\item \( (a, b], \text{ where } a \in [-4.5, -0.75] \text{ and } b \in [9, 15] \)

* $(-1.61, 14.20]$, which is the correct option.
\item \( \text{None of the above.} \)


\end{enumerate}

\textbf{General Comment:} To solve, you will need to break up the compound inequality into two inequalities. Be sure to keep track of the inequality! It may be best to draw a number line and graph your solution.
}
\litem{
Using an interval or intervals, describe all the $x$-values within or including a distance of the given values.
\[ \text{ No more than } 10 \text{ units from the number } 6. \]The solution is \( [-4, 16] \), which is option A.\begin{enumerate}[label=\Alph*.]
\item \( [-4, 16] \)

This describes the values no more than 10 from 6
\item \( (-\infty, -4] \cup [16, \infty) \)

This describes the values no less than 10 from 6
\item \( (-\infty, -4) \cup (16, \infty) \)

This describes the values more than 10 from 6
\item \( (-4, 16) \)

This describes the values less than 10 from 6
\item \( \text{None of the above} \)

You likely thought the values in the interval were not correct.
\end{enumerate}

\textbf{General Comment:} When thinking about this language, it helps to draw a number line and try points.
}
\litem{
Solve the linear inequality below. Then, choose the constant and interval combination that describes the solution set.
\[ \frac{-5}{2} - \frac{6}{7} x > \frac{6}{3} x + \frac{8}{9} \]The solution is \( (-\infty, -1.186) \), which is option D.\begin{enumerate}[label=\Alph*.]
\item \( (a, \infty), \text{ where } a \in [-2.02, -1.12] \)

 $(-1.186, \infty)$, which corresponds to switching the direction of the interval. You likely did this if you did not flip the inequality when dividing by a negative!
\item \( (a, \infty), \text{ where } a \in [-0.15, 2.1] \)

 $(1.186, \infty)$, which corresponds to switching the direction of the interval AND negating the endpoint. You likely did this if you did not flip the inequality when dividing by a negative as well as not moving values over to a side properly.
\item \( (-\infty, a), \text{ where } a \in [0.45, 3.52] \)

 $(-\infty, 1.186)$, which corresponds to negating the endpoint of the solution.
\item \( (-\infty, a), \text{ where } a \in [-1.5, -0.15] \)

* $(-\infty, -1.186)$, which is the correct option.
\item \( \text{None of the above}. \)

You may have chosen this if you thought the inequality did not match the ends of the intervals.
\end{enumerate}

\textbf{General Comment:} Remember that less/greater than or equal to includes the endpoint, while less/greater do not. Also, remember that you need to flip the inequality when you multiply or divide by a negative.
}
\end{enumerate}

\end{document}