\documentclass{extbook}[14pt]
\usepackage{multicol, enumerate, enumitem, hyperref, color, soul, setspace, parskip, fancyhdr, amssymb, amsthm, amsmath, bbm, latexsym, units, mathtools}
\everymath{\displaystyle}
\usepackage[headsep=0.5cm,headheight=0cm, left=1 in,right= 1 in,top= 1 in,bottom= 1 in]{geometry}
\usepackage{dashrule}  % Package to use the command below to create lines between items
\newcommand{\litem}[1]{\item #1

\rule{\textwidth}{0.4pt}}
\pagestyle{fancy}
\lhead{}
\chead{Answer Key for Progress Quiz 3 Version B}
\rhead{}
\lfoot{3148-2249}
\cfoot{}
\rfoot{Spring 2021}
\begin{document}
\textbf{This key should allow you to understand why you choose the option you did (beyond just getting a question right or wrong). \href{https://xronos.clas.ufl.edu/mac1105spring2020/courseDescriptionAndMisc/Exams/LearningFromResults}{More instructions on how to use this key can be found here}.}

\textbf{If you have a suggestion to make the keys better, \href{https://forms.gle/CZkbZmPbC9XALEE88}{please fill out the short survey here}.}

\textit{Note: This key is auto-generated and may contain issues and/or errors. The keys are reviewed after each exam to ensure grading is done accurately. If there are issues (like duplicate options), they are noted in the offline gradebook. The keys are a work-in-progress to give students as many resources to improve as possible.}

\rule{\textwidth}{0.4pt}

\begin{enumerate}\litem{
Solve the linear inequality below. Then, choose the constant and interval combination that describes the solution set.
\[ -9 + 6 x > 7 x \text{ or } -8 + 3 x < 5 x \]

The solution is \( (-\infty, -9.0) \text{ or } (-4.0, \infty) \), which is option B.\begin{enumerate}[label=\Alph*.]
\item \( (-\infty, a] \cup [b, \infty), \text{ where } a \in [-14, -6] \text{ and } b \in [-4, -3] \)

Corresponds to including the endpoints (when they should be excluded).
\item \( (-\infty, a) \cup (b, \infty), \text{ where } a \in [-10, -5] \text{ and } b \in [-4, -2] \)

 * Correct option.
\item \( (-\infty, a] \cup [b, \infty), \text{ where } a \in [3, 6] \text{ and } b \in [9, 13] \)

Corresponds to including the endpoints AND negating.
\item \( (-\infty, a) \cup (b, \infty), \text{ where } a \in [4, 6] \text{ and } b \in [7, 13] \)

Corresponds to inverting the inequality and negating the solution.
\item \( (-\infty, \infty) \)

Corresponds to the variable canceling, which does not happen in this instance.
\end{enumerate}

\textbf{General Comment:} When multiplying or dividing by a negative, flip the sign.
}
\litem{
Using an interval or intervals, describe all the $x$-values within or including a distance of the given values.
\[ \text{ No less than } 7 \text{ units from the number } 1. \]

The solution is \( \text{None of the above} \), which is option E.\begin{enumerate}[label=\Alph*.]
\item \( (-\infty, 6) \cup (8, \infty) \)

This describes the values more than 1 from 7
\item \( (6, 8) \)

This describes the values less than 1 from 7
\item \( (-\infty, 6] \cup [8, \infty) \)

This describes the values no less than 1 from 7
\item \( [6, 8] \)

This describes the values no more than 1 from 7
\item \( \text{None of the above} \)

Options A-D described the values [more/less than] 1 units from 7, which is the reverse of what the question asked.
\end{enumerate}

\textbf{General Comment:} When thinking about this language, it helps to draw a number line and try points.
}
\litem{
Using an interval or intervals, describe all the $x$-values within or including a distance of the given values.
\[ \text{ Less than } 2 \text{ units from the number } 10. \]

The solution is \( (8, 12) \), which is option C.\begin{enumerate}[label=\Alph*.]
\item \( (-\infty, 8) \cup (12, \infty) \)

This describes the values more than 2 from 10
\item \( [8, 12] \)

This describes the values no more than 2 from 10
\item \( (8, 12) \)

This describes the values less than 2 from 10
\item \( (-\infty, 8] \cup [12, \infty) \)

This describes the values no less than 2 from 10
\item \( \text{None of the above} \)

You likely thought the values in the interval were not correct.
\end{enumerate}

\textbf{General Comment:} When thinking about this language, it helps to draw a number line and try points.
}
\litem{
Solve the linear inequality below. Then, choose the constant and interval combination that describes the solution set.
\[ -5 - 8 x < \frac{-44 x - 5}{6} \leq 4 - 8 x \]

The solution is \( (-6.25, 7.25] \), which is option B.\begin{enumerate}[label=\Alph*.]
\item \( (-\infty, a) \cup [b, \infty), \text{ where } a \in [-7.25, -5.25] \text{ and } b \in [6.25, 9.25] \)

$(-\infty, -6.25) \cup [7.25, \infty)$, which corresponds to displaying the and-inequality as an or-inequality.
\item \( (a, b], \text{ where } a \in [-6.25, -5.25] \text{ and } b \in [4.25, 12.25] \)

* $(-6.25, 7.25]$, which is the correct option.
\item \( (-\infty, a] \cup (b, \infty), \text{ where } a \in [-8.25, -1.25] \text{ and } b \in [5.25, 13.25] \)

$(-\infty, -6.25] \cup (7.25, \infty)$, which corresponds to displaying the and-inequality as an or-inequality AND flipping the inequality.
\item \( [a, b), \text{ where } a \in [-6.25, -2.25] \text{ and } b \in [6.25, 10.25] \)

$[-6.25, 7.25)$, which corresponds to flipping the inequality.
\item \( \text{None of the above.} \)


\end{enumerate}

\textbf{General Comment:} To solve, you will need to break up the compound inequality into two inequalities. Be sure to keep track of the inequality! It may be best to draw a number line and graph your solution.
}
\litem{
Solve the linear inequality below. Then, choose the constant and interval combination that describes the solution set.
\[ \frac{-3}{3} + \frac{6}{4} x > \frac{7}{7} x - \frac{4}{8} \]

The solution is \( (1.0, \infty) \), which is option D.\begin{enumerate}[label=\Alph*.]
\item \( (-\infty, a), \text{ where } a \in [-0.6, 2.2] \)

 $(-\infty, 1.0)$, which corresponds to switching the direction of the interval. You likely did this if you did not flip the inequality when dividing by a negative!
\item \( (-\infty, a), \text{ where } a \in [-1.5, -0.6] \)

 $(-\infty, -1.0)$, which corresponds to switching the direction of the interval AND negating the endpoint. You likely did this if you did not flip the inequality when dividing by a negative as well as not moving values over to a side properly.
\item \( (a, \infty), \text{ where } a \in [-1.6, -0.3] \)

 $(-1.0, \infty)$, which corresponds to negating the endpoint of the solution.
\item \( (a, \infty), \text{ where } a \in [0.8, 1.9] \)

* $(1.0, \infty)$, which is the correct option.
\item \( \text{None of the above}. \)

You may have chosen this if you thought the inequality did not match the ends of the intervals.
\end{enumerate}

\textbf{General Comment:} Remember that less/greater than or equal to includes the endpoint, while less/greater do not. Also, remember that you need to flip the inequality when you multiply or divide by a negative.
}
\litem{
Solve the linear inequality below. Then, choose the constant and interval combination that describes the solution set.
\[ -7 + 8 x < \frac{76 x - 6}{9} \leq 4 + 7 x \]

The solution is \( (-14.25, 3.23] \), which is option B.\begin{enumerate}[label=\Alph*.]
\item \( (-\infty, a) \cup [b, \infty), \text{ where } a \in [-19.25, -7.25] \text{ and } b \in [3.23, 8.23] \)

$(-\infty, -14.25) \cup [3.23, \infty)$, which corresponds to displaying the and-inequality as an or-inequality.
\item \( (a, b], \text{ where } a \in [-18.25, -12.25] \text{ and } b \in [3.23, 5.23] \)

* $(-14.25, 3.23]$, which is the correct option.
\item \( (-\infty, a] \cup (b, \infty), \text{ where } a \in [-14.25, -13.25] \text{ and } b \in [2.23, 5.23] \)

$(-\infty, -14.25] \cup (3.23, \infty)$, which corresponds to displaying the and-inequality as an or-inequality AND flipping the inequality.
\item \( [a, b), \text{ where } a \in [-18.25, -11.25] \text{ and } b \in [0.23, 4.23] \)

$[-14.25, 3.23)$, which corresponds to flipping the inequality.
\item \( \text{None of the above.} \)


\end{enumerate}

\textbf{General Comment:} To solve, you will need to break up the compound inequality into two inequalities. Be sure to keep track of the inequality! It may be best to draw a number line and graph your solution.
}
\litem{
Solve the linear inequality below. Then, choose the constant and interval combination that describes the solution set.
\[ \frac{-10}{2} - \frac{6}{8} x \leq \frac{10}{4} x + \frac{8}{3} \]

The solution is \( [-2.359, \infty) \), which is option B.\begin{enumerate}[label=\Alph*.]
\item \( (-\infty, a], \text{ where } a \in [-0.64, 5.36] \)

 $(-\infty, 2.359]$, which corresponds to switching the direction of the interval AND negating the endpoint. You likely did this if you did not flip the inequality when dividing by a negative as well as not moving values over to a side properly.
\item \( [a, \infty), \text{ where } a \in [-6.36, 0.64] \)

* $[-2.359, \infty)$, which is the correct option.
\item \( [a, \infty), \text{ where } a \in [1.36, 3.36] \)

 $[2.359, \infty)$, which corresponds to negating the endpoint of the solution.
\item \( (-\infty, a], \text{ where } a \in [-5.36, 0.64] \)

 $(-\infty, -2.359]$, which corresponds to switching the direction of the interval. You likely did this if you did not flip the inequality when dividing by a negative!
\item \( \text{None of the above}. \)

You may have chosen this if you thought the inequality did not match the ends of the intervals.
\end{enumerate}

\textbf{General Comment:} Remember that less/greater than or equal to includes the endpoint, while less/greater do not. Also, remember that you need to flip the inequality when you multiply or divide by a negative.
}
\litem{
Solve the linear inequality below. Then, choose the constant and interval combination that describes the solution set.
\[ -8 + 8 x > 9 x \text{ or } -7 - 3 x < 4 x \]

The solution is \( (-\infty, -8.0) \text{ or } (-1.0, \infty) \), which is option A.\begin{enumerate}[label=\Alph*.]
\item \( (-\infty, a) \cup (b, \infty), \text{ where } a \in [-11, -7] \text{ and } b \in [-8, 2] \)

 * Correct option.
\item \( (-\infty, a] \cup [b, \infty), \text{ where } a \in [-11, -6] \text{ and } b \in [-1, 1] \)

Corresponds to including the endpoints (when they should be excluded).
\item \( (-\infty, a] \cup [b, \infty), \text{ where } a \in [-1, 2] \text{ and } b \in [7, 14] \)

Corresponds to including the endpoints AND negating.
\item \( (-\infty, a) \cup (b, \infty), \text{ where } a \in [-1, 2] \text{ and } b \in [6, 14] \)

Corresponds to inverting the inequality and negating the solution.
\item \( (-\infty, \infty) \)

Corresponds to the variable canceling, which does not happen in this instance.
\end{enumerate}

\textbf{General Comment:} When multiplying or dividing by a negative, flip the sign.
}
\litem{
Solve the linear inequality below. Then, choose the constant and interval combination that describes the solution set.
\[ -4x + 9 < 3x -3 \]

The solution is \( (1.714, \infty) \), which is option C.\begin{enumerate}[label=\Alph*.]
\item \( (-\infty, a), \text{ where } a \in [-1.29, 6.71] \)

 $(-\infty, 1.714)$, which corresponds to switching the direction of the interval. You likely did this if you did not flip the inequality when dividing by a negative!
\item \( (-\infty, a), \text{ where } a \in [-3.71, 1.29] \)

 $(-\infty, -1.714)$, which corresponds to switching the direction of the interval AND negating the endpoint. You likely did this if you did not flip the inequality when dividing by a negative as well as not moving values over to a side properly.
\item \( (a, \infty), \text{ where } a \in [-0.29, 6.71] \)

* $(1.714, \infty)$, which is the correct option.
\item \( (a, \infty), \text{ where } a \in [-2.71, -0.71] \)

 $(-1.714, \infty)$, which corresponds to negating the endpoint of the solution.
\item \( \text{None of the above}. \)

You may have chosen this if you thought the inequality did not match the ends of the intervals.
\end{enumerate}

\textbf{General Comment:} Remember that less/greater than or equal to includes the endpoint, while less/greater do not. Also, remember that you need to flip the inequality when you multiply or divide by a negative.
}
\litem{
Solve the linear inequality below. Then, choose the constant and interval combination that describes the solution set.
\[ -5x -3 \leq 5x + 7 \]

The solution is \( [-1.0, \infty) \), which is option C.\begin{enumerate}[label=\Alph*.]
\item \( (-\infty, a], \text{ where } a \in [-0.1, 1.1] \)

 $(-\infty, 1.0]$, which corresponds to switching the direction of the interval AND negating the endpoint. You likely did this if you did not flip the inequality when dividing by a negative as well as not moving values over to a side properly.
\item \( (-\infty, a], \text{ where } a \in [-4.2, 0.8] \)

 $(-\infty, -1.0]$, which corresponds to switching the direction of the interval. You likely did this if you did not flip the inequality when dividing by a negative!
\item \( [a, \infty), \text{ where } a \in [-1, 0] \)

* $[-1.0, \infty)$, which is the correct option.
\item \( [a, \infty), \text{ where } a \in [1, 2] \)

 $[1.0, \infty)$, which corresponds to negating the endpoint of the solution.
\item \( \text{None of the above}. \)

You may have chosen this if you thought the inequality did not match the ends of the intervals.
\end{enumerate}

\textbf{General Comment:} Remember that less/greater than or equal to includes the endpoint, while less/greater do not. Also, remember that you need to flip the inequality when you multiply or divide by a negative.
}
\end{enumerate}

\end{document}