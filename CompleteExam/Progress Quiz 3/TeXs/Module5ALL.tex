\documentclass[14pt]{extbook}
\usepackage{multicol, enumerate, enumitem, hyperref, color, soul, setspace, parskip, fancyhdr} %General Packages
\usepackage{amssymb, amsthm, amsmath, latexsym, units, mathtools} %Math Packages
\everymath{\displaystyle} %All math in Display Style
% Packages with additional options
\usepackage[headsep=0.5cm,headheight=12pt, left=1 in,right= 1 in,top= 1 in,bottom= 1 in]{geometry}
\usepackage[usenames,dvipsnames]{xcolor}
\usepackage{dashrule}  % Package to use the command below to create lines between items
\newcommand{\litem}[1]{\item#1\hspace*{-1cm}\rule{\textwidth}{0.4pt}}
\pagestyle{fancy}
\lhead{Progress Quiz 3}
\chead{}
\rhead{Version ALL}
\lfoot{3012-8528}
\cfoot{}
\rfoot{Summer C 2021}
\begin{document}

\begin{enumerate}
\litem{
Choose the equation of the function graphed below.
\begin{center}
    \includegraphics[width=0.5\textwidth]{../Figures/radicalGraphToEquationA.png}
\end{center}
\begin{enumerate}[label=\Alph*.]
\item \( f(x) = \sqrt[3]{x + 6} + 4 \)
\item \( f(x) = \sqrt[3]{x - 6} + 4 \)
\item \( f(x) = - \sqrt[3]{x - 6} + 4 \)
\item \( f(x) = - \sqrt[3]{x + 6} + 4 \)
\item \( \text{None of the above} \)

\end{enumerate} }
\litem{
What is the domain of the function below?\[ f(x) = \sqrt[4]{-7 x - 8} \]\begin{enumerate}[label=\Alph*.]
\item \( (-\infty, \infty) \)
\item \( [a, \infty), \text{where } a \in [-1.2, -1.07] \)
\item \( (-\infty, a], \text{where } a \in [-1.1, -0.39] \)
\item \( (-\infty, a], \text{ where } a \in [-1.24, -1.06] \)
\item \( [a, \infty), \text{where } a \in [-0.89, -0.69] \)

\end{enumerate} }
\litem{
Solve the radical equation below. Then, choose the interval(s) that the solution(s) belongs to.\[ \sqrt{-7 x - 6} - \sqrt{-4 x + 6} = 0 \]\begin{enumerate}[label=\Alph*.]
\item \( x_1 \in [-1.95, -0.73] \text{ and } x_2 \in [1.2,1.6] \)
\item \( \text{All solutions lead to invalid or complex values in the equation.} \)
\item \( x \in [-4.73,-3.97] \)
\item \( x_1 \in [-4.73, -3.97] \text{ and } x_2 \in [-3.2,0.8] \)
\item \( x \in [-0.65,0.93] \)

\end{enumerate} }
\litem{
Solve the radical equation below. Then, choose the interval(s) that the solution(s) belongs to.\[ \sqrt{-2 x + 6} - \sqrt{-3 x - 7} = 0 \]\begin{enumerate}[label=\Alph*.]
\item \( x \in [-16,-8] \)
\item \( \text{All solutions lead to invalid or complex values in the equation.} \)
\item \( x_1 \in [-5.33, -1.33] \text{ and } x_2 \in [1,5] \)
\item \( x_1 \in [-16, -8] \text{ and } x_2 \in [1,5] \)
\item \( x \in [0,4] \)

\end{enumerate} }
\litem{
Choose the equation of the function graphed below.
\begin{center}
    \includegraphics[width=0.5\textwidth]{../Figures/radicalGraphToEquationCopyA.png}
\end{center}
\begin{enumerate}[label=\Alph*.]
\item \( f(x) = \sqrt[3]{x - 8} - 4 \)
\item \( f(x) = - \sqrt[3]{x - 8} - 4 \)
\item \( f(x) = - \sqrt[3]{x + 8} - 4 \)
\item \( f(x) = \sqrt[3]{x + 8} - 4 \)
\item \( \text{None of the above} \)

\end{enumerate} }
\litem{
Solve the radical equation below. Then, choose the interval(s) that the solution(s) belongs to.\[ \sqrt{12 x^2 - 32} - \sqrt{8 x} = 0 \]\begin{enumerate}[label=\Alph*.]
\item \( x_1 \in [-1.73, -0.76] \text{ and } x_2 \in [0,6] \)
\item \( \text{All solutions lead to invalid or complex values in the equation.} \)
\item \( x_1 \in [0.82, 1.39] \text{ and } x_2 \in [0,6] \)
\item \( x \in [-1.73,-0.76] \)
\item \( x \in [1.89,2.04] \)

\end{enumerate} }
\litem{
Choose the graph of the equation below.\[ f(x) = - \sqrt[3]{x - 8} - 5 \]\begin{enumerate}[label=\Alph*.]
\begin{multicols}{2}\item \includegraphics[width = 0.3\textwidth]{../Figures/radicalEquationToGraphAA.png}\item \includegraphics[width = 0.3\textwidth]{../Figures/radicalEquationToGraphBA.png}\item \includegraphics[width = 0.3\textwidth]{../Figures/radicalEquationToGraphCA.png}\item \includegraphics[width = 0.3\textwidth]{../Figures/radicalEquationToGraphDA.png}\end{multicols}\item None of the above.
\end{enumerate} }
\litem{
Solve the radical equation below. Then, choose the interval(s) that the solution(s) belongs to.\[ \sqrt{-27 x^2 + 35} - \sqrt{-24 x} = 0 \]\begin{enumerate}[label=\Alph*.]
\item \( x \in [1.6,2.1] \)
\item \( x_1 \in [-3.5, 0.1] \text{ and } x_2 \in [0.67,4.67] \)
\item \( \text{All solutions lead to invalid or complex values in the equation.} \)
\item \( x_1 \in [0.7, 1.4] \text{ and } x_2 \in [0.67,4.67] \)
\item \( x \in [-3.5,0.1] \)

\end{enumerate} }
\litem{
Choose the graph of the equation below.\[ f(x) = \sqrt{x - 12} - 3 \]\begin{enumerate}[label=\Alph*.]
\begin{multicols}{2}\item \includegraphics[width = 0.3\textwidth]{../Figures/radicalEquationToGraphCopyAA.png}\item \includegraphics[width = 0.3\textwidth]{../Figures/radicalEquationToGraphCopyBA.png}\item \includegraphics[width = 0.3\textwidth]{../Figures/radicalEquationToGraphCopyCA.png}\item \includegraphics[width = 0.3\textwidth]{../Figures/radicalEquationToGraphCopyDA.png}\end{multicols}\item None of the above.
\end{enumerate} }
\litem{
What is the domain of the function below?\[ f(x) = \sqrt[3]{8 x + 9} \]\begin{enumerate}[label=\Alph*.]
\item \( \text{The domain is } (-\infty, a], \text{   where } a \in [-1.45, -1] \)
\item \( \text{The domain is } (-\infty, a], \text{   where } a \in [-0.96, -0.54] \)
\item \( (-\infty, \infty) \)
\item \( \text{The domain is } [a, \infty), \text{   where } a \in [-0.92, -0.86] \)
\item \( \text{The domain is } [a, \infty), \text{   where } a \in [-1.64, -1.01] \)

\end{enumerate} }
\litem{
Choose the equation of the function graphed below.
\begin{center}
    \includegraphics[width=0.5\textwidth]{../Figures/radicalGraphToEquationB.png}
\end{center}
\begin{enumerate}[label=\Alph*.]
\item \( f(x) = \sqrt{x + 6} + 4 \)
\item \( f(x) = - \sqrt{x - 6} + 4 \)
\item \( f(x) = - \sqrt{x + 6} + 4 \)
\item \( f(x) = \sqrt{x - 6} + 4 \)
\item \( \text{None of the above} \)

\end{enumerate} }
\litem{
What is the domain of the function below?\[ f(x) = \sqrt[7]{-4 x + 9} \]\begin{enumerate}[label=\Alph*.]
\item \( \text{The domain is } (-\infty, a], \text{   where } a \in [1.3, 3.6] \)
\item \( \text{The domain is } (-\infty, a], \text{   where } a \in [0, 1.7] \)
\item \( \text{The domain is } [a, \infty), \text{   where } a \in [-0.6, 1.3] \)
\item \( \text{The domain is } [a, \infty), \text{   where } a \in [2.2, 2.4] \)
\item \( (-\infty, \infty) \)

\end{enumerate} }
\litem{
Solve the radical equation below. Then, choose the interval(s) that the solution(s) belongs to.\[ \sqrt{-3 x + 9} - \sqrt{4 x - 7} = 0 \]\begin{enumerate}[label=\Alph*.]
\item \( x_1 \in [1.78, 3.14] \text{ and } x_2 \in [0,10] \)
\item \( \text{All solutions lead to invalid or complex values in the equation.} \)
\item \( x \in [-0.11,0.57] \)
\item \( x \in [1.78,3.14] \)
\item \( x_1 \in [0.34, 2.25] \text{ and } x_2 \in [0,10] \)

\end{enumerate} }
\litem{
Solve the radical equation below. Then, choose the interval(s) that the solution(s) belongs to.\[ \sqrt{-5 x + 9} - \sqrt{5 x - 9} = 0 \]\begin{enumerate}[label=\Alph*.]
\item \( \text{All solutions lead to invalid or complex values in the equation.} \)
\item \( x \in [1,4] \)
\item \( x \in [-0.9,1.1] \)
\item \( x_1 \in [1, 4] \text{ and } x_2 \in [-2.2,4.8] \)
\item \( x_1 \in [1, 4] \text{ and } x_2 \in [-2.2,4.8] \)

\end{enumerate} }
\litem{
Choose the equation of the function graphed below.
\begin{center}
    \includegraphics[width=0.5\textwidth]{../Figures/radicalGraphToEquationCopyB.png}
\end{center}
\begin{enumerate}[label=\Alph*.]
\item \( f(x) = - \sqrt{x - 14} + 7 \)
\item \( f(x) = \sqrt{x - 14} + 7 \)
\item \( f(x) = - \sqrt{x + 14} + 7 \)
\item \( f(x) = \sqrt{x + 14} + 7 \)
\item \( \text{None of the above} \)

\end{enumerate} }
\litem{
Solve the radical equation below. Then, choose the interval(s) that the solution(s) belongs to.\[ \sqrt{45 x^2 + 42} - \sqrt{-89 x} = 0 \]\begin{enumerate}[label=\Alph*.]
\item \( x \in [-0.91,-0.66] \)
\item \( \text{All solutions lead to invalid or complex values in the equation.} \)
\item \( x_1 \in [-1.81, -0.85] \text{ and } x_2 \in [-0.78,0.22] \)
\item \( x_1 \in [-0.13, 2.15] \text{ and } x_2 \in [0.2,4.2] \)
\item \( x \in [-1.81,-0.85] \)

\end{enumerate} }
\litem{
Choose the graph of the equation below.\[ f(x) = - \sqrt[3]{x - 6} - 4 \]\begin{enumerate}[label=\Alph*.]
\begin{multicols}{2}\item \includegraphics[width = 0.3\textwidth]{../Figures/radicalEquationToGraphAB.png}\item \includegraphics[width = 0.3\textwidth]{../Figures/radicalEquationToGraphBB.png}\item \includegraphics[width = 0.3\textwidth]{../Figures/radicalEquationToGraphCB.png}\item \includegraphics[width = 0.3\textwidth]{../Figures/radicalEquationToGraphDB.png}\end{multicols}\item None of the above.
\end{enumerate} }
\litem{
Solve the radical equation below. Then, choose the interval(s) that the solution(s) belongs to.\[ \sqrt{-36 x^2 - 56} - \sqrt{-95 x} = 0 \]\begin{enumerate}[label=\Alph*.]
\item \( x_1 \in [0.79, 1.17] \text{ and } x_2 \in [1.75,2.75] \)
\item \( x_1 \in [-1.03, -0.47] \text{ and } x_2 \in [-4.75,0.25] \)
\item \( \text{All solutions lead to invalid or complex values in the equation.} \)
\item \( x \in [1.45,2.74] \)
\item \( x \in [0.79,1.17] \)

\end{enumerate} }
\litem{
Choose the graph of the equation below.\[ f(x) = - \sqrt[3]{x + 12} - 6 \]\begin{enumerate}[label=\Alph*.]
\begin{multicols}{2}\item \includegraphics[width = 0.3\textwidth]{../Figures/radicalEquationToGraphCopyAB.png}\item \includegraphics[width = 0.3\textwidth]{../Figures/radicalEquationToGraphCopyBB.png}\item \includegraphics[width = 0.3\textwidth]{../Figures/radicalEquationToGraphCopyCB.png}\item \includegraphics[width = 0.3\textwidth]{../Figures/radicalEquationToGraphCopyDB.png}\end{multicols}\item None of the above.
\end{enumerate} }
\litem{
What is the domain of the function below?\[ f(x) = \sqrt[4]{8 x - 3} \]\begin{enumerate}[label=\Alph*.]
\item \( (-\infty, a], \text{where } a \in [1.67, 6.67] \)
\item \( [a, \infty), \text{ where } a \in [-2.62, 2.38] \)
\item \( [a, \infty), \text{where } a \in [1.67, 4.67] \)
\item \( (-\infty, \infty) \)
\item \( (-\infty, a], \text{where } a \in [-0.62, 2.38] \)

\end{enumerate} }
\litem{
Choose the equation of the function graphed below.
\begin{center}
    \includegraphics[width=0.5\textwidth]{../Figures/radicalGraphToEquationC.png}
\end{center}
\begin{enumerate}[label=\Alph*.]
\item \( f(x) = - \sqrt[3]{x - 12} + 4 \)
\item \( f(x) = \sqrt[3]{x - 12} + 4 \)
\item \( f(x) = \sqrt[3]{x + 12} + 4 \)
\item \( f(x) = - \sqrt[3]{x + 12} + 4 \)
\item \( \text{None of the above} \)

\end{enumerate} }
\litem{
What is the domain of the function below?\[ f(x) = \sqrt[8]{7 x + 5} \]\begin{enumerate}[label=\Alph*.]
\item \( (-\infty, \infty) \)
\item \( [a, \infty), \text{where } a \in [-1.88, -1.02] \)
\item \( [a, \infty), \text{ where } a \in [-0.82, -0.36] \)
\item \( (-\infty, a], \text{where } a \in [-2.1, -0.83] \)
\item \( (-\infty, a], \text{where } a \in [-0.95, 0.49] \)

\end{enumerate} }
\litem{
Solve the radical equation below. Then, choose the interval(s) that the solution(s) belongs to.\[ \sqrt{9 x - 7} - \sqrt{2 x - 5} = 0 \]\begin{enumerate}[label=\Alph*.]
\item \( x \in [-0.19,0.29] \)
\item \( \text{All solutions lead to invalid or complex values in the equation.} \)
\item \( x \in [1.42,1.95] \)
\item \( x_1 \in [0.42, 1.49] \text{ and } x_2 \in [2.01,2.77] \)
\item \( x_1 \in [-0.19, 0.29] \text{ and } x_2 \in [0.03,0.82] \)

\end{enumerate} }
\litem{
Solve the radical equation below. Then, choose the interval(s) that the solution(s) belongs to.\[ \sqrt{4 x + 6} - \sqrt{6 x + 8} = 0 \]\begin{enumerate}[label=\Alph*.]
\item \( \text{All solutions lead to invalid or complex values in the equation.} \)
\item \( x_1 \in [-2.85, -1.21] \text{ and } x_2 \in [-1.2,-0.66] \)
\item \( x \in [6.06,7.44] \)
\item \( x_1 \in [-2.85, -1.21] \text{ and } x_2 \in [-1.75,-1.28] \)
\item \( x \in [-1.13,-0.94] \)

\end{enumerate} }
\litem{
Choose the equation of the function graphed below.
\begin{center}
    \includegraphics[width=0.5\textwidth]{../Figures/radicalGraphToEquationCopyC.png}
\end{center}
\begin{enumerate}[label=\Alph*.]
\item \( f(x) = \sqrt{x + 8} + 4 \)
\item \( f(x) = - \sqrt{x + 8} + 4 \)
\item \( f(x) = \sqrt{x - 8} + 4 \)
\item \( f(x) = - \sqrt{x - 8} + 4 \)
\item \( \text{None of the above} \)

\end{enumerate} }
\litem{
Solve the radical equation below. Then, choose the interval(s) that the solution(s) belongs to.\[ \sqrt{8 x^2 - 56} - \sqrt{18 x} = 0 \]\begin{enumerate}[label=\Alph*.]
\item \( x \in [-3,-1.6] \)
\item \( \text{All solutions lead to invalid or complex values in the equation.} \)
\item \( x_1 \in [0.8, 2.8] \text{ and } x_2 \in [2,9] \)
\item \( x_1 \in [-3, -1.6] \text{ and } x_2 \in [2,9] \)
\item \( x \in [2.6,4.7] \)

\end{enumerate} }
\litem{
Choose the graph of the equation below.\[ f(x) = - \sqrt{x - 10} + 5 \]\begin{enumerate}[label=\Alph*.]
\begin{multicols}{2}\item \includegraphics[width = 0.3\textwidth]{../Figures/radicalEquationToGraphAC.png}\item \includegraphics[width = 0.3\textwidth]{../Figures/radicalEquationToGraphBC.png}\item \includegraphics[width = 0.3\textwidth]{../Figures/radicalEquationToGraphCC.png}\item \includegraphics[width = 0.3\textwidth]{../Figures/radicalEquationToGraphDC.png}\end{multicols}\item None of the above.
\end{enumerate} }
\litem{
Solve the radical equation below. Then, choose the interval(s) that the solution(s) belongs to.\[ \sqrt{-56 x^2 - 35} - \sqrt{91 x} = 0 \]\begin{enumerate}[label=\Alph*.]
\item \( x \in [-1.03,-0.83] \)
\item \( x_1 \in [-1.03, -0.83] \text{ and } x_2 \in [-1.25,0.13] \)
\item \( \text{All solutions lead to invalid or complex values in the equation.} \)
\item \( x \in [-0.66,-0.32] \)
\item \( x_1 \in [0.94, 1.32] \text{ and } x_2 \in [0.06,1.18] \)

\end{enumerate} }
\litem{
Choose the graph of the equation below.\[ f(x) = \sqrt{x + 12} + 6 \]\begin{enumerate}[label=\Alph*.]
\begin{multicols}{2}\item \includegraphics[width = 0.3\textwidth]{../Figures/radicalEquationToGraphCopyAC.png}\item \includegraphics[width = 0.3\textwidth]{../Figures/radicalEquationToGraphCopyBC.png}\item \includegraphics[width = 0.3\textwidth]{../Figures/radicalEquationToGraphCopyCC.png}\item \includegraphics[width = 0.3\textwidth]{../Figures/radicalEquationToGraphCopyDC.png}\end{multicols}\item None of the above.
\end{enumerate} }
\litem{
What is the domain of the function below?\[ f(x) = \sqrt[8]{-8 x + 7} \]\begin{enumerate}[label=\Alph*.]
\item \( (-\infty, a], \text{ where } a \in [0.68, 1.07] \)
\item \( [a, \infty), \text{where } a \in [0.92, 1.43] \)
\item \( [a, \infty), \text{where } a \in [0.73, 0.91] \)
\item \( (-\infty, a], \text{where } a \in [0.99, 1.58] \)
\item \( (-\infty, \infty) \)

\end{enumerate} }
\end{enumerate}

\end{document}