\documentclass[14pt]{extbook}
\usepackage{multicol, enumerate, enumitem, hyperref, color, soul, setspace, parskip, fancyhdr} %General Packages
\usepackage{amssymb, amsthm, amsmath, latexsym, units, mathtools} %Math Packages
\everymath{\displaystyle} %All math in Display Style
% Packages with additional options
\usepackage[headsep=0.5cm,headheight=12pt, left=1 in,right= 1 in,top= 1 in,bottom= 1 in]{geometry}
\usepackage[usenames,dvipsnames]{xcolor}
\usepackage{dashrule}  % Package to use the command below to create lines between items
\newcommand{\litem}[1]{\item#1\hspace*{-1cm}\rule{\textwidth}{0.4pt}}
\pagestyle{fancy}
\lhead{Progress Quiz 3}
\chead{}
\rhead{Version ALL}
\lfoot{3012-8528}
\cfoot{}
\rfoot{Summer C 2021}
\begin{document}

\begin{enumerate}
\litem{
Describe the end behavior of the polynomial below.\[ f(x) = -3(x + 5)^{4}(x - 5)^{7}(x - 7)^{4}(x + 7)^{6} \]\begin{enumerate}[label=\Alph*.]
\begin{multicols}{2}\item \includegraphics[width = 0.3\textwidth]{../Figures/polyEndBehaviorCopyAA.png}\item \includegraphics[width = 0.3\textwidth]{../Figures/polyEndBehaviorCopyBA.png}\item \includegraphics[width = 0.3\textwidth]{../Figures/polyEndBehaviorCopyCA.png}\item \includegraphics[width = 0.3\textwidth]{../Figures/polyEndBehaviorCopyDA.png}\end{multicols}\item None of the above.
\end{enumerate} }
\litem{
Describe the zero behavior of the zero $x = 3$ of the polynomial below.\[ f(x) = 6(x - 3)^{8}(x + 3)^{13}(x - 4)^{9}(x + 4)^{12} \]\begin{enumerate}[label=\Alph*.]
\begin{multicols}{2}\item \includegraphics[width = 0.3\textwidth]{../Figures/polyZeroBehaviorAA.png}\item \includegraphics[width = 0.3\textwidth]{../Figures/polyZeroBehaviorBA.png}\item \includegraphics[width = 0.3\textwidth]{../Figures/polyZeroBehaviorCA.png}\item \includegraphics[width = 0.3\textwidth]{../Figures/polyZeroBehaviorDA.png}\end{multicols}\item None of the above.
\end{enumerate} }
\litem{
Which of the following equations \textit{could} be of the graph presented below?
\begin{center}
    \includegraphics[width=0.5\textwidth]{../Figures/polyGraphToFunctionA.png}
\end{center}
\begin{enumerate}[label=\Alph*.]
\item \( 4x^{7} (x - 2)^{10} (x - 1)^{5} \)
\item \( -8x^{9} (x - 2)^{6} (x - 1)^{5} \)
\item \( 18x^{7} (x - 2)^{7} (x - 1)^{5} \)
\item \( 7x^{10} (x - 2)^{8} (x - 1)^{11} \)
\item \( -12x^{9} (x - 2)^{5} (x - 1)^{5} \)

\end{enumerate} }
\litem{
Construct the lowest-degree polynomial given the zeros below. Then, choose the intervals that contain the coefficients of the polynomial in the form $x^3+bx^2+cx+d$.\[ 4 + 5 i \text{ and } 2 \]\begin{enumerate}[label=\Alph*.]
\item \( b \in [7, 16], c \in [56, 57.3], \text{ and } d \in [81, 86.3] \)
\item \( b \in [0, 2], c \in [-6.8, -1.8], \text{ and } d \in [3.8, 8.5] \)
\item \( b \in [0, 2], c \in [-8.8, -6.5], \text{ and } d \in [8.4, 13.1] \)
\item \( b \in [-13, -4], c \in [56, 57.3], \text{ and } d \in [-84.9, -79.5] \)
\item \( \text{None of the above.} \)

\end{enumerate} }
\litem{
Which of the following equations \textit{could} be of the graph presented below?
\begin{center}
    \includegraphics[width=0.5\textwidth]{../Figures/polyGraphToFunctionCopyA.png}
\end{center}
\begin{enumerate}[label=\Alph*.]
\item \( -18x^{9} (x - 2)^{10} (x + 4)^{8} \)
\item \( 19x^{6} (x - 2)^{9} (x + 4)^{9} \)
\item \( -14x^{9} (x - 2)^{6} (x + 4)^{5} \)
\item \( 11x^{9} (x - 2)^{8} (x + 4)^{7} \)
\item \( 19x^{6} (x - 2)^{10} (x + 4)^{9} \)

\end{enumerate} }
\litem{
Describe the end behavior of the polynomial below.\[ f(x) = -3(x + 2)^{3}(x - 2)^{8}(x + 6)^{5}(x - 6)^{5} \]\begin{enumerate}[label=\Alph*.]
\begin{multicols}{2}\item \includegraphics[width = 0.3\textwidth]{../Figures/polyEndBehaviorAA.png}\item \includegraphics[width = 0.3\textwidth]{../Figures/polyEndBehaviorBA.png}\item \includegraphics[width = 0.3\textwidth]{../Figures/polyEndBehaviorCA.png}\item \includegraphics[width = 0.3\textwidth]{../Figures/polyEndBehaviorDA.png}\end{multicols}\item None of the above.
\end{enumerate} }
\litem{
Describe the zero behavior of the zero $x = 7$ of the polynomial below.\[ f(x) = 9(x + 8)^{9}(x - 8)^{7}(x - 7)^{7}(x + 7)^{2} \]\begin{enumerate}[label=\Alph*.]
\begin{multicols}{2}\item \includegraphics[width = 0.3\textwidth]{../Figures/polyZeroBehaviorCopyAA.png}\item \includegraphics[width = 0.3\textwidth]{../Figures/polyZeroBehaviorCopyBA.png}\item \includegraphics[width = 0.3\textwidth]{../Figures/polyZeroBehaviorCopyCA.png}\item \includegraphics[width = 0.3\textwidth]{../Figures/polyZeroBehaviorCopyDA.png}\end{multicols}\item None of the above.
\end{enumerate} }
\litem{
Construct the lowest-degree polynomial given the zeros below. Then, choose the intervals that contain the coefficients of the polynomial in the form $ax^3+bx^2+cx+d$.\[ \frac{-1}{5}, \frac{-1}{4}, \text{ and } 6 \]\begin{enumerate}[label=\Alph*.]
\item \( a \in [16, 21], b \in [107, 116], c \in [-55, -47], \text{ and } d \in [-2, 13] \)
\item \( a \in [16, 21], b \in [-137, -123], c \in [51, 61], \text{ and } d \in [-9, -5] \)
\item \( a \in [16, 21], b \in [-113, -106], c \in [-55, -47], \text{ and } d \in [-2, 13] \)
\item \( a \in [16, 21], b \in [-120, -116], c \in [-19, -4], \text{ and } d \in [-2, 13] \)
\item \( a \in [16, 21], b \in [-113, -106], c \in [-55, -47], \text{ and } d \in [-9, -5] \)

\end{enumerate} }
\litem{
Construct the lowest-degree polynomial given the zeros below. Then, choose the intervals that contain the coefficients of the polynomial in the form $x^3+bx^2+cx+d$.\[ -5 + 4 i \text{ and } -2 \]\begin{enumerate}[label=\Alph*.]
\item \( b \in [-14, -10], c \in [53, 67], \text{ and } d \in [-88, -77] \)
\item \( b \in [1, 6], c \in [-5, -1], \text{ and } d \in [-8, 0] \)
\item \( b \in [1, 6], c \in [7, 8], \text{ and } d \in [9, 17] \)
\item \( b \in [9, 25], c \in [53, 67], \text{ and } d \in [82, 90] \)
\item \( \text{None of the above.} \)

\end{enumerate} }
\litem{
Construct the lowest-degree polynomial given the zeros below. Then, choose the intervals that contain the coefficients of the polynomial in the form $ax^3+bx^2+cx+d$.\[ \frac{-5}{4}, \frac{-3}{4}, \text{ and } -5 \]\begin{enumerate}[label=\Alph*.]
\item \( a \in [12, 18], b \in [44, 51], c \in [-145, -143], \text{ and } d \in [73, 83] \)
\item \( a \in [12, 18], b \in [104, 114], c \in [171, 181], \text{ and } d \in [-76, -74] \)
\item \( a \in [12, 18], b \in [72, 73], c \in [-60, -50], \text{ and } d \in [-76, -74] \)
\item \( a \in [12, 18], b \in [-114, -109], c \in [171, 181], \text{ and } d \in [-76, -74] \)
\item \( a \in [12, 18], b \in [104, 114], c \in [171, 181], \text{ and } d \in [73, 83] \)

\end{enumerate} }
\litem{
Describe the end behavior of the polynomial below.\[ f(x) = 4(x + 3)^{4}(x - 3)^{9}(x + 2)^{4}(x - 2)^{6} \]\begin{enumerate}[label=\Alph*.]
\begin{multicols}{2}\item \includegraphics[width = 0.3\textwidth]{../Figures/polyEndBehaviorCopyAB.png}\item \includegraphics[width = 0.3\textwidth]{../Figures/polyEndBehaviorCopyBB.png}\item \includegraphics[width = 0.3\textwidth]{../Figures/polyEndBehaviorCopyCB.png}\item \includegraphics[width = 0.3\textwidth]{../Figures/polyEndBehaviorCopyDB.png}\end{multicols}\item None of the above.
\end{enumerate} }
\litem{
Describe the zero behavior of the zero $x = -8$ of the polynomial below.\[ f(x) = -3(x + 9)^{6}(x - 9)^{5}(x + 8)^{14}(x - 8)^{9} \]\begin{enumerate}[label=\Alph*.]
\begin{multicols}{2}\item \includegraphics[width = 0.3\textwidth]{../Figures/polyZeroBehaviorAB.png}\item \includegraphics[width = 0.3\textwidth]{../Figures/polyZeroBehaviorBB.png}\item \includegraphics[width = 0.3\textwidth]{../Figures/polyZeroBehaviorCB.png}\item \includegraphics[width = 0.3\textwidth]{../Figures/polyZeroBehaviorDB.png}\end{multicols}\item None of the above.
\end{enumerate} }
\litem{
Which of the following equations \textit{could} be of the graph presented below?
\begin{center}
    \includegraphics[width=0.5\textwidth]{../Figures/polyGraphToFunctionB.png}
\end{center}
\begin{enumerate}[label=\Alph*.]
\item \( 17(x + 2)^{10} (x - 2)^{7} (x + 4)^{5} \)
\item \( -9(x + 2)^{10} (x - 2)^{9} (x + 4)^{11} \)
\item \( -18(x + 2)^{10} (x - 2)^{11} (x + 4)^{10} \)
\item \( 8(x + 2)^{7} (x - 2)^{4} (x + 4)^{5} \)
\item \( 20(x + 2)^{4} (x - 2)^{4} (x + 4)^{9} \)

\end{enumerate} }
\litem{
Construct the lowest-degree polynomial given the zeros below. Then, choose the intervals that contain the coefficients of the polynomial in the form $x^3+bx^2+cx+d$.\[ 3 + 2 i \text{ and } 4 \]\begin{enumerate}[label=\Alph*.]
\item \( b \in [1, 8], c \in [-7.52, -6.31], \text{ and } d \in [11, 13] \)
\item \( b \in [5, 14], c \in [36.76, 37.91], \text{ and } d \in [49, 57] \)
\item \( b \in [-15, -7], c \in [36.76, 37.91], \text{ and } d \in [-52, -48] \)
\item \( b \in [1, 8], c \in [-6.57, -5.83], \text{ and } d \in [6, 9] \)
\item \( \text{None of the above.} \)

\end{enumerate} }
\litem{
Which of the following equations \textit{could} be of the graph presented below?
\begin{center}
    \includegraphics[width=0.5\textwidth]{../Figures/polyGraphToFunctionCopyB.png}
\end{center}
\begin{enumerate}[label=\Alph*.]
\item \( 4(x - 2)^{4} (x + 3)^{6} (x - 1)^{5} \)
\item \( 16(x - 2)^{5} (x + 3)^{5} (x - 1)^{5} \)
\item \( -9(x - 2)^{10} (x + 3)^{9} (x - 1)^{7} \)
\item \( 4(x - 2)^{6} (x + 3)^{7} (x - 1)^{11} \)
\item \( -10(x - 2)^{11} (x + 3)^{5} (x - 1)^{7} \)

\end{enumerate} }
\litem{
Describe the end behavior of the polynomial below.\[ f(x) = -9(x + 8)^{4}(x - 8)^{5}(x - 6)^{4}(x + 6)^{5} \]\begin{enumerate}[label=\Alph*.]
\begin{multicols}{2}\item \includegraphics[width = 0.3\textwidth]{../Figures/polyEndBehaviorAB.png}\item \includegraphics[width = 0.3\textwidth]{../Figures/polyEndBehaviorBB.png}\item \includegraphics[width = 0.3\textwidth]{../Figures/polyEndBehaviorCB.png}\item \includegraphics[width = 0.3\textwidth]{../Figures/polyEndBehaviorDB.png}\end{multicols}\item None of the above.
\end{enumerate} }
\litem{
Describe the zero behavior of the zero $x = -5$ of the polynomial below.\[ f(x) = -9(x - 5)^{4}(x + 5)^{7}(x - 9)^{4}(x + 9)^{8} \]\begin{enumerate}[label=\Alph*.]
\begin{multicols}{2}\item \includegraphics[width = 0.3\textwidth]{../Figures/polyZeroBehaviorCopyAB.png}\item \includegraphics[width = 0.3\textwidth]{../Figures/polyZeroBehaviorCopyBB.png}\item \includegraphics[width = 0.3\textwidth]{../Figures/polyZeroBehaviorCopyCB.png}\item \includegraphics[width = 0.3\textwidth]{../Figures/polyZeroBehaviorCopyDB.png}\end{multicols}\item None of the above.
\end{enumerate} }
\litem{
Construct the lowest-degree polynomial given the zeros below. Then, choose the intervals that contain the coefficients of the polynomial in the form $ax^3+bx^2+cx+d$.\[ 1, \frac{-3}{4}, \text{ and } \frac{6}{5} \]\begin{enumerate}[label=\Alph*.]
\item \( a \in [20, 21], b \in [-35, -25], c \in [-9, 1], \text{ and } d \in [15, 24] \)
\item \( a \in [20, 21], b \in [-22, -14], c \in [-21, -16], \text{ and } d \in [15, 24] \)
\item \( a \in [20, 21], b \in [27, 36], c \in [-9, 1], \text{ and } d \in [-26, -17] \)
\item \( a \in [20, 21], b \in [-35, -25], c \in [-9, 1], \text{ and } d \in [-26, -17] \)
\item \( a \in [20, 21], b \in [10, 12], c \in [-33, -25], \text{ and } d \in [-26, -17] \)

\end{enumerate} }
\litem{
Construct the lowest-degree polynomial given the zeros below. Then, choose the intervals that contain the coefficients of the polynomial in the form $x^3+bx^2+cx+d$.\[ 3 + 4 i \text{ and } 4 \]\begin{enumerate}[label=\Alph*.]
\item \( b \in [-5, 7], c \in [-7.9, -4.2], \text{ and } d \in [11, 13] \)
\item \( b \in [-5, 7], c \in [-8.4, -7.9], \text{ and } d \in [13, 18] \)
\item \( b \in [7, 19], c \in [48.6, 51.5], \text{ and } d \in [98, 101] \)
\item \( b \in [-10, -4], c \in [48.6, 51.5], \text{ and } d \in [-102, -94] \)
\item \( \text{None of the above.} \)

\end{enumerate} }
\litem{
Construct the lowest-degree polynomial given the zeros below. Then, choose the intervals that contain the coefficients of the polynomial in the form $ax^3+bx^2+cx+d$.\[ \frac{3}{4}, \frac{5}{2}, \text{ and } -4 \]\begin{enumerate}[label=\Alph*.]
\item \( a \in [5, 9], b \in [16, 29], c \in [-71, -68], \text{ and } d \in [-63, -58] \)
\item \( a \in [5, 9], b \in [-13, -5], c \in [-95, -76], \text{ and } d \in [-63, -58] \)
\item \( a \in [5, 9], b \in [2, 9], c \in [-95, -76], \text{ and } d \in [55, 66] \)
\item \( a \in [5, 9], b \in [2, 9], c \in [-95, -76], \text{ and } d \in [-63, -58] \)
\item \( a \in [5, 9], b \in [57, 64], c \in [115, 125], \text{ and } d \in [55, 66] \)

\end{enumerate} }
\litem{
Describe the end behavior of the polynomial below.\[ f(x) = 7(x + 5)^{4}(x - 5)^{7}(x - 9)^{3}(x + 9)^{3} \]\begin{enumerate}[label=\Alph*.]
\begin{multicols}{2}\item \includegraphics[width = 0.3\textwidth]{../Figures/polyEndBehaviorCopyAC.png}\item \includegraphics[width = 0.3\textwidth]{../Figures/polyEndBehaviorCopyBC.png}\item \includegraphics[width = 0.3\textwidth]{../Figures/polyEndBehaviorCopyCC.png}\item \includegraphics[width = 0.3\textwidth]{../Figures/polyEndBehaviorCopyDC.png}\end{multicols}\item None of the above.
\end{enumerate} }
\litem{
Describe the zero behavior of the zero $x = -5$ of the polynomial below.\[ f(x) = 3(x - 3)^{9}(x + 3)^{7}(x + 5)^{4}(x - 5)^{3} \]\begin{enumerate}[label=\Alph*.]
\begin{multicols}{2}\item \includegraphics[width = 0.3\textwidth]{../Figures/polyZeroBehaviorAC.png}\item \includegraphics[width = 0.3\textwidth]{../Figures/polyZeroBehaviorBC.png}\item \includegraphics[width = 0.3\textwidth]{../Figures/polyZeroBehaviorCC.png}\item \includegraphics[width = 0.3\textwidth]{../Figures/polyZeroBehaviorDC.png}\end{multicols}\item None of the above.
\end{enumerate} }
\litem{
Which of the following equations \textit{could} be of the graph presented below?
\begin{center}
    \includegraphics[width=0.5\textwidth]{../Figures/polyGraphToFunctionC.png}
\end{center}
\begin{enumerate}[label=\Alph*.]
\item \( 9(x - 1)^{10} (x - 2)^{8} (x - 3)^{5} \)
\item \( 19(x - 1)^{7} (x - 2)^{10} (x - 3)^{5} \)
\item \( -11(x - 1)^{4} (x - 2)^{11} (x - 3)^{4} \)
\item \( 7(x - 1)^{6} (x - 2)^{11} (x - 3)^{7} \)
\item \( -6(x - 1)^{10} (x - 2)^{11} (x - 3)^{5} \)

\end{enumerate} }
\litem{
Construct the lowest-degree polynomial given the zeros below. Then, choose the intervals that contain the coefficients of the polynomial in the form $x^3+bx^2+cx+d$.\[ 5 + 2 i \text{ and } 4 \]\begin{enumerate}[label=\Alph*.]
\item \( b \in [13, 15], c \in [64, 73], \text{ and } d \in [114, 122] \)
\item \( b \in [-4, 5], c \in [-19, -7], \text{ and } d \in [17, 23] \)
\item \( b \in [-4, 5], c \in [-8, -3], \text{ and } d \in [8, 14] \)
\item \( b \in [-16, -13], c \in [64, 73], \text{ and } d \in [-125, -112] \)
\item \( \text{None of the above.} \)

\end{enumerate} }
\litem{
Which of the following equations \textit{could} be of the graph presented below?
\begin{center}
    \includegraphics[width=0.5\textwidth]{../Figures/polyGraphToFunctionCopyC.png}
\end{center}
\begin{enumerate}[label=\Alph*.]
\item \( 19x^{5} (x + 1)^{6} (x + 2)^{7} \)
\item \( -14x^{7} (x + 1)^{10} (x + 2)^{4} \)
\item \( 14x^{6} (x + 1)^{10} (x + 2)^{9} \)
\item \( -11x^{4} (x + 1)^{10} (x + 2)^{4} \)
\item \( 2x^{9} (x + 1)^{8} (x + 2)^{10} \)

\end{enumerate} }
\litem{
Describe the end behavior of the polynomial below.\[ f(x) = -4(x - 4)^{3}(x + 4)^{6}(x + 8)^{5}(x - 8)^{5} \]\begin{enumerate}[label=\Alph*.]
\begin{multicols}{2}\item \includegraphics[width = 0.3\textwidth]{../Figures/polyEndBehaviorAC.png}\item \includegraphics[width = 0.3\textwidth]{../Figures/polyEndBehaviorBC.png}\item \includegraphics[width = 0.3\textwidth]{../Figures/polyEndBehaviorCC.png}\item \includegraphics[width = 0.3\textwidth]{../Figures/polyEndBehaviorDC.png}\end{multicols}\item None of the above.
\end{enumerate} }
\litem{
Describe the zero behavior of the zero $x = 9$ of the polynomial below.\[ f(x) = -7(x - 9)^{4}(x + 9)^{7}(x + 2)^{6}(x - 2)^{7} \]\begin{enumerate}[label=\Alph*.]
\begin{multicols}{2}\item \includegraphics[width = 0.3\textwidth]{../Figures/polyZeroBehaviorCopyAC.png}\item \includegraphics[width = 0.3\textwidth]{../Figures/polyZeroBehaviorCopyBC.png}\item \includegraphics[width = 0.3\textwidth]{../Figures/polyZeroBehaviorCopyCC.png}\item \includegraphics[width = 0.3\textwidth]{../Figures/polyZeroBehaviorCopyDC.png}\end{multicols}\item None of the above.
\end{enumerate} }
\litem{
Construct the lowest-degree polynomial given the zeros below. Then, choose the intervals that contain the coefficients of the polynomial in the form $ax^3+bx^2+cx+d$.\[ \frac{1}{4}, 4, \text{ and } \frac{3}{5} \]\begin{enumerate}[label=\Alph*.]
\item \( a \in [12, 21], b \in [-101, -93], c \in [61, 75], \text{ and } d \in [-15, -8] \)
\item \( a \in [12, 21], b \in [-101, -93], c \in [61, 75], \text{ and } d \in [8, 15] \)
\item \( a \in [12, 21], b \in [-88, -82], c \in [23, 26], \text{ and } d \in [8, 15] \)
\item \( a \in [12, 21], b \in [95, 101], c \in [61, 75], \text{ and } d \in [8, 15] \)
\item \( a \in [12, 21], b \in [69, 77], c \in [-36, -28], \text{ and } d \in [-15, -8] \)

\end{enumerate} }
\litem{
Construct the lowest-degree polynomial given the zeros below. Then, choose the intervals that contain the coefficients of the polynomial in the form $x^3+bx^2+cx+d$.\[ -3 + 2 i \text{ and } 1 \]\begin{enumerate}[label=\Alph*.]
\item \( b \in [-8, -3], c \in [7, 8], \text{ and } d \in [11, 14] \)
\item \( b \in [2, 8], c \in [7, 8], \text{ and } d \in [-16, -8] \)
\item \( b \in [1, 2], c \in [0, 4], \text{ and } d \in [-5, -2] \)
\item \( b \in [1, 2], c \in [-7, 1], \text{ and } d \in [0, 5] \)
\item \( \text{None of the above.} \)

\end{enumerate} }
\litem{
Construct the lowest-degree polynomial given the zeros below. Then, choose the intervals that contain the coefficients of the polynomial in the form $ax^3+bx^2+cx+d$.\[ 4, \frac{-3}{5}, \text{ and } \frac{3}{4} \]\begin{enumerate}[label=\Alph*.]
\item \( a \in [19, 26], b \in [81, 88], c \in [3, 6], \text{ and } d \in [-37, -33] \)
\item \( a \in [19, 26], b \in [76, 78], c \in [-22, -18], \text{ and } d \in [-37, -33] \)
\item \( a \in [19, 26], b \in [-85, -80], c \in [3, 6], \text{ and } d \in [29, 39] \)
\item \( a \in [19, 26], b \in [46, 63], c \in [-101, -94], \text{ and } d \in [29, 39] \)
\item \( a \in [19, 26], b \in [-85, -80], c \in [3, 6], \text{ and } d \in [-37, -33] \)

\end{enumerate} }
\end{enumerate}

\end{document}