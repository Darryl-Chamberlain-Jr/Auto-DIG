\documentclass[14pt]{extbook}
\usepackage{multicol, enumerate, enumitem, hyperref, color, soul, setspace, parskip, fancyhdr} %General Packages
\usepackage{amssymb, amsthm, amsmath, latexsym, units, mathtools} %Math Packages
\everymath{\displaystyle} %All math in Display Style
% Packages with additional options
\usepackage[headsep=0.5cm,headheight=12pt, left=1 in,right= 1 in,top= 1 in,bottom= 1 in]{geometry}
\usepackage[usenames,dvipsnames]{xcolor}
\usepackage{dashrule}  % Package to use the command below to create lines between items
\newcommand{\litem}[1]{\item#1\hspace*{-1cm}\rule{\textwidth}{0.4pt}}
\pagestyle{fancy}
\lhead{Progress Quiz 3}
\chead{}
\rhead{Version ALL}
\lfoot{3012-8528}
\cfoot{}
\rfoot{Summer C 2021}
\begin{document}

\begin{enumerate}
\litem{
Simplify the expression below into the form $a+bi$. Then, choose the intervals that $a$ and $b$ belong to.\[ (8 + 2 i)(-9 + 7 i) \]\begin{enumerate}[label=\Alph*.]
\item \( a \in [-59, -55] \text{ and } b \in [-78, -72] \)
\item \( a \in [-73, -63] \text{ and } b \in [11, 16] \)
\item \( a \in [-87, -85] \text{ and } b \in [-44, -36] \)
\item \( a \in [-59, -55] \text{ and } b \in [74, 77] \)
\item \( a \in [-87, -85] \text{ and } b \in [34, 41] \)

\end{enumerate} }
\litem{
Simplify the expression below and choose the interval the simplification is contained within.\[ 3 - 2^2 + 1 \div 10 * 18 \div 11 \]\begin{enumerate}[label=\Alph*.]
\item \( [-1.12, -0.9] \)
\item \( [7.09, 7.26] \)
\item \( [6.71, 7.11] \)
\item \( [-0.85, -0.3] \)
\item \( \text{None of the above} \)

\end{enumerate} }
\litem{
Choose the \textbf{smallest} set of Complex numbers that the number below belongs to.\[ \sqrt{\frac{1188}{9}}+\sqrt{45} i \]\begin{enumerate}[label=\Alph*.]
\item \( \text{Rational} \)
\item \( \text{Nonreal Complex} \)
\item \( \text{Pure Imaginary} \)
\item \( \text{Not a Complex Number} \)
\item \( \text{Irrational} \)

\end{enumerate} }
\litem{
Choose the \textbf{smallest} set of Complex numbers that the number below belongs to.\[ \sqrt{\frac{625}{0}}+\sqrt{45} i \]\begin{enumerate}[label=\Alph*.]
\item \( \text{Irrational} \)
\item \( \text{Pure Imaginary} \)
\item \( \text{Rational} \)
\item \( \text{Not a Complex Number} \)
\item \( \text{Nonreal Complex} \)

\end{enumerate} }
\litem{
Simplify the expression below into the form $a+bi$. Then, choose the intervals that $a$ and $b$ belong to.\[ \frac{-9 - 33 i}{-7 + 5 i} \]\begin{enumerate}[label=\Alph*.]
\item \( a \in [-103.5, -101] \text{ and } b \in [3, 4.5] \)
\item \( a \in [-3, -1] \text{ and } b \in [3, 4.5] \)
\item \( a \in [-3, -1] \text{ and } b \in [275.5, 276.5] \)
\item \( a \in [1.5, 4] \text{ and } b \in [2, 3] \)
\item \( a \in [0.5, 1.5] \text{ and } b \in [-8, -6.5] \)

\end{enumerate} }
\litem{
Simplify the expression below and choose the interval the simplification is contained within.\[ 6 - 3^2 + 19 \div 5 * 10 \div 2 \]\begin{enumerate}[label=\Alph*.]
\item \( [33.31, 34.53] \)
\item \( [15.92, 16.33] \)
\item \( [14.78, 15.26] \)
\item \( [-2.93, -1.92] \)
\item \( \text{None of the above} \)

\end{enumerate} }
\litem{
Simplify the expression below into the form $a+bi$. Then, choose the intervals that $a$ and $b$ belong to.\[ (-7 - 8 i)(3 + 10 i) \]\begin{enumerate}[label=\Alph*.]
\item \( a \in [56, 63] \text{ and } b \in [93, 97] \)
\item \( a \in [56, 63] \text{ and } b \in [-96, -92] \)
\item \( a \in [-103, -100] \text{ and } b \in [-46, -40] \)
\item \( a \in [-24, -16] \text{ and } b \in [-85, -73] \)
\item \( a \in [-103, -100] \text{ and } b \in [46, 47] \)

\end{enumerate} }
\litem{
Simplify the expression below into the form $a+bi$. Then, choose the intervals that $a$ and $b$ belong to.\[ \frac{-72 - 66 i}{3 + 4 i} \]\begin{enumerate}[label=\Alph*.]
\item \( a \in [-25, -23.5] \text{ and } b \in [-17.5, -15.5] \)
\item \( a \in [-20.5, -19] \text{ and } b \in [89.5, 91] \)
\item \( a \in [-20.5, -19] \text{ and } b \in [3, 5] \)
\item \( a \in [1.5, 2] \text{ and } b \in [-20, -19] \)
\item \( a \in [-481, -479] \text{ and } b \in [3, 5] \)

\end{enumerate} }
\litem{
Choose the \textbf{smallest} set of Real numbers that the number below belongs to.\[ \sqrt{\frac{23}{0}} \]\begin{enumerate}[label=\Alph*.]
\item \( \text{Not a Real number} \)
\item \( \text{Whole} \)
\item \( \text{Rational} \)
\item \( \text{Irrational} \)
\item \( \text{Integer} \)

\end{enumerate} }
\litem{
Choose the \textbf{smallest} set of Real numbers that the number below belongs to.\[ -\sqrt{\frac{256}{625}} \]\begin{enumerate}[label=\Alph*.]
\item \( \text{Integer} \)
\item \( \text{Rational} \)
\item \( \text{Not a Real number} \)
\item \( \text{Whole} \)
\item \( \text{Irrational} \)

\end{enumerate} }
\litem{
Simplify the expression below into the form $a+bi$. Then, choose the intervals that $a$ and $b$ belong to.\[ (10 - 4 i)(-6 - 8 i) \]\begin{enumerate}[label=\Alph*.]
\item \( a \in [-38, -25] \text{ and } b \in [-104, -102] \)
\item \( a \in [-38, -25] \text{ and } b \in [104, 108] \)
\item \( a \in [-92, -87] \text{ and } b \in [52, 57] \)
\item \( a \in [-92, -87] \text{ and } b \in [-56, -54] \)
\item \( a \in [-63, -58] \text{ and } b \in [28, 35] \)

\end{enumerate} }
\litem{
Simplify the expression below and choose the interval the simplification is contained within.\[ 12 - 8 \div 13 * 16 - (15 * 14) \]\begin{enumerate}[label=\Alph*.]
\item \( [-202.04, -193.04] \)
\item \( [-182.85, -176.85] \)
\item \( [-213.85, -203.85] \)
\item \( [219.96, 223.96] \)
\item \( \text{None of the above} \)

\end{enumerate} }
\litem{
Choose the \textbf{smallest} set of Complex numbers that the number below belongs to.\[ \sqrt{\frac{-567}{9}} i+\sqrt{55}i \]\begin{enumerate}[label=\Alph*.]
\item \( \text{Not a Complex Number} \)
\item \( \text{Pure Imaginary} \)
\item \( \text{Nonreal Complex} \)
\item \( \text{Rational} \)
\item \( \text{Irrational} \)

\end{enumerate} }
\litem{
Choose the \textbf{smallest} set of Complex numbers that the number below belongs to.\[ \frac{\sqrt{119}}{20}+\sqrt{-6}i \]\begin{enumerate}[label=\Alph*.]
\item \( \text{Pure Imaginary} \)
\item \( \text{Nonreal Complex} \)
\item \( \text{Irrational} \)
\item \( \text{Not a Complex Number} \)
\item \( \text{Rational} \)

\end{enumerate} }
\litem{
Simplify the expression below into the form $a+bi$. Then, choose the intervals that $a$ and $b$ belong to.\[ \frac{45 - 22 i}{7 - 4 i} \]\begin{enumerate}[label=\Alph*.]
\item \( a \in [6.04, 6.3] \text{ and } b \in [-0.5, 1] \)
\item \( a \in [6.04, 6.3] \text{ and } b \in [25.5, 28] \)
\item \( a \in [402.98, 403.23] \text{ and } b \in [-0.5, 1] \)
\item \( a \in [3.36, 3.52] \text{ and } b \in [-6, -4.5] \)
\item \( a \in [6.28, 6.6] \text{ and } b \in [5, 6] \)

\end{enumerate} }
\litem{
Simplify the expression below and choose the interval the simplification is contained within.\[ 9 - 4^2 + 16 \div 18 * 7 \div 17 \]\begin{enumerate}[label=\Alph*.]
\item \( [25.34, 26.01] \)
\item \( [-7.31, -6.88] \)
\item \( [24.16, 25.14] \)
\item \( [-6.66, -6.5] \)
\item \( \text{None of the above} \)

\end{enumerate} }
\litem{
Simplify the expression below into the form $a+bi$. Then, choose the intervals that $a$ and $b$ belong to.\[ (6 - 3 i)(7 - 10 i) \]\begin{enumerate}[label=\Alph*.]
\item \( a \in [36, 44] \text{ and } b \in [28, 35] \)
\item \( a \in [9, 14] \text{ and } b \in [80, 82] \)
\item \( a \in [66, 75] \text{ and } b \in [-40, -38] \)
\item \( a \in [66, 75] \text{ and } b \in [38, 47] \)
\item \( a \in [9, 14] \text{ and } b \in [-81, -77] \)

\end{enumerate} }
\litem{
Simplify the expression below into the form $a+bi$. Then, choose the intervals that $a$ and $b$ belong to.\[ \frac{-63 + 33 i}{-6 + 2 i} \]\begin{enumerate}[label=\Alph*.]
\item \( a \in [7.5, 8.5] \text{ and } b \in [-9, -7] \)
\item \( a \in [10, 11] \text{ and } b \in [15.5, 17] \)
\item \( a \in [11, 12] \text{ and } b \in [-2.5, -0.5] \)
\item \( a \in [443, 445] \text{ and } b \in [-2.5, -0.5] \)
\item \( a \in [11, 12] \text{ and } b \in [-73.5, -71.5] \)

\end{enumerate} }
\litem{
Choose the \textbf{smallest} set of Real numbers that the number below belongs to.\[ -\sqrt{\frac{102400}{256}} \]\begin{enumerate}[label=\Alph*.]
\item \( \text{Rational} \)
\item \( \text{Integer} \)
\item \( \text{Irrational} \)
\item \( \text{Whole} \)
\item \( \text{Not a Real number} \)

\end{enumerate} }
\litem{
Choose the \textbf{smallest} set of Real numbers that the number below belongs to.\[ -\sqrt{\frac{14}{0}} \]\begin{enumerate}[label=\Alph*.]
\item \( \text{Rational} \)
\item \( \text{Irrational} \)
\item \( \text{Not a Real number} \)
\item \( \text{Whole} \)
\item \( \text{Integer} \)

\end{enumerate} }
\litem{
Simplify the expression below into the form $a+bi$. Then, choose the intervals that $a$ and $b$ belong to.\[ (-6 + 4 i)(9 - 3 i) \]\begin{enumerate}[label=\Alph*.]
\item \( a \in [-42, -38] \text{ and } b \in [-56, -50] \)
\item \( a \in [-68, -58] \text{ and } b \in [16, 19] \)
\item \( a \in [-42, -38] \text{ and } b \in [52, 59] \)
\item \( a \in [-68, -58] \text{ and } b \in [-18, -16] \)
\item \( a \in [-55, -49] \text{ and } b \in [-16, -6] \)

\end{enumerate} }
\litem{
Simplify the expression below and choose the interval the simplification is contained within.\[ 20 - 17^2 + 7 \div 18 * 19 \div 8 \]\begin{enumerate}[label=\Alph*.]
\item \( [-269.87, -268.41] \)
\item \( [309.82, 310.51] \)
\item \( [308.54, 309.06] \)
\item \( [-268.96, -267.02] \)
\item \( \text{None of the above} \)

\end{enumerate} }
\litem{
Choose the \textbf{smallest} set of Complex numbers that the number below belongs to.\[ \frac{12}{14}+\sqrt{-9}i \]\begin{enumerate}[label=\Alph*.]
\item \( \text{Irrational} \)
\item \( \text{Not a Complex Number} \)
\item \( \text{Nonreal Complex} \)
\item \( \text{Rational} \)
\item \( \text{Pure Imaginary} \)

\end{enumerate} }
\litem{
Choose the \textbf{smallest} set of Complex numbers that the number below belongs to.\[ -\sqrt{\frac{225}{121}} + 16i^2 \]\begin{enumerate}[label=\Alph*.]
\item \( \text{Nonreal Complex} \)
\item \( \text{Pure Imaginary} \)
\item \( \text{Irrational} \)
\item \( \text{Rational} \)
\item \( \text{Not a Complex Number} \)

\end{enumerate} }
\litem{
Simplify the expression below into the form $a+bi$. Then, choose the intervals that $a$ and $b$ belong to.\[ \frac{63 - 22 i}{-1 + 3 i} \]\begin{enumerate}[label=\Alph*.]
\item \( a \in [-13, -12.5] \text{ and } b \in [-17.5, -16] \)
\item \( a \in [-64.5, -62.5] \text{ and } b \in [-9, -5.5] \)
\item \( a \in [-13, -12.5] \text{ and } b \in [-169, -166] \)
\item \( a \in [-130, -128] \text{ and } b \in [-17.5, -16] \)
\item \( a \in [-0.5, 1.5] \text{ and } b \in [20, 22.5] \)

\end{enumerate} }
\litem{
Simplify the expression below and choose the interval the simplification is contained within.\[ 20 - 19 \div 14 * 6 - (16 * 17) \]\begin{enumerate}[label=\Alph*.]
\item \( [-70.43, -63.43] \)
\item \( [-255.23, -247.23] \)
\item \( [291.77, 293.77] \)
\item \( [-262.14, -258.14] \)
\item \( \text{None of the above} \)

\end{enumerate} }
\litem{
Simplify the expression below into the form $a+bi$. Then, choose the intervals that $a$ and $b$ belong to.\[ (9 - 6 i)(2 - 8 i) \]\begin{enumerate}[label=\Alph*.]
\item \( a \in [-34, -28] \text{ and } b \in [81, 88] \)
\item \( a \in [-34, -28] \text{ and } b \in [-86, -81] \)
\item \( a \in [66, 68] \text{ and } b \in [60, 67] \)
\item \( a \in [66, 68] \text{ and } b \in [-64, -55] \)
\item \( a \in [17, 23] \text{ and } b \in [48, 55] \)

\end{enumerate} }
\litem{
Simplify the expression below into the form $a+bi$. Then, choose the intervals that $a$ and $b$ belong to.\[ \frac{45 - 66 i}{7 - i} \]\begin{enumerate}[label=\Alph*.]
\item \( a \in [4, 6] \text{ and } b \in [-10.5, -9.5] \)
\item \( a \in [5.5, 7] \text{ and } b \in [65, 66.5] \)
\item \( a \in [6.5, 8] \text{ and } b \in [-9, -6.5] \)
\item \( a \in [380, 382] \text{ and } b \in [-9, -6.5] \)
\item \( a \in [6.5, 8] \text{ and } b \in [-417.5, -416] \)

\end{enumerate} }
\litem{
Choose the \textbf{smallest} set of Real numbers that the number below belongs to.\[ \sqrt{\frac{-715}{5}} \]\begin{enumerate}[label=\Alph*.]
\item \( \text{Integer} \)
\item \( \text{Irrational} \)
\item \( \text{Rational} \)
\item \( \text{Not a Real number} \)
\item \( \text{Whole} \)

\end{enumerate} }
\litem{
Choose the \textbf{smallest} set of Real numbers that the number below belongs to.\[ \sqrt{\frac{38025}{169}} \]\begin{enumerate}[label=\Alph*.]
\item \( \text{Integer} \)
\item \( \text{Not a Real number} \)
\item \( \text{Rational} \)
\item \( \text{Irrational} \)
\item \( \text{Whole} \)

\end{enumerate} }
\end{enumerate}

\end{document}