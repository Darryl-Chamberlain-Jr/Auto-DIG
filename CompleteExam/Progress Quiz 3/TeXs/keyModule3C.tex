\documentclass{extbook}[14pt]
\usepackage{multicol, enumerate, enumitem, hyperref, color, soul, setspace, parskip, fancyhdr, amssymb, amsthm, amsmath, bbm, latexsym, units, mathtools}
\everymath{\displaystyle}
\usepackage[headsep=0.5cm,headheight=0cm, left=1 in,right= 1 in,top= 1 in,bottom= 1 in]{geometry}
\usepackage{dashrule}  % Package to use the command below to create lines between items
\newcommand{\litem}[1]{\item #1

\rule{\textwidth}{0.4pt}}
\pagestyle{fancy}
\lhead{}
\chead{Answer Key for Progress Quiz 3 Version C}
\rhead{}
\lfoot{}
\cfoot{}
\rfoot{Fall 2020}
\begin{document}
\textbf{This key should allow you to understand why you choose the option you did (beyond just getting a question right or wrong). \href{https://xronos.clas.ufl.edu/mac1105spring2020/courseDescriptionAndMisc/Exams/LearningFromResults}{More instructions on how to use this key can be found here}.}

\textbf{If you have a suggestion to make the keys better, \href{https://forms.gle/CZkbZmPbC9XALEE88}{please fill out the short survey here}.}

\textit{Note: This key is auto-generated and may contain issues and/or errors. The keys are reviewed after each exam to ensure grading is done accurately. If there are issues (like duplicate options), they are noted in the offline gradebook. The keys are a work-in-progress to give students as many resources to improve as possible.}

\rule{\textwidth}{0.4pt}

\begin{enumerate}\litem{
Solve the linear inequality below. Then, choose the constant and interval combination that describes the solution set.
\[ -3 + 6 x > 8 x \text{ or } 6 + 8 x < 9 x \]
The solution is \( (-\infty, -1.5) \text{ or } (6.0, \infty) \), which is option D.\begin{enumerate}[label=\Alph*.]
\item \( (-\infty, a] \cup [b, \infty), \text{ where } a \in [-3.5, 1.5] \text{ and } b \in [6, 8] \)

Corresponds to including the endpoints (when they should be excluded).
\item \( (-\infty, a] \cup [b, \infty), \text{ where } a \in [-7, -3] \text{ and } b \in [1.5, 2.5] \)

Corresponds to including the endpoints AND negating.
\item \( (-\infty, a) \cup (b, \infty), \text{ where } a \in [-9, -4] \text{ and } b \in [-0.5, 3.5] \)

Corresponds to inverting the inequality and negating the solution.
\item \( (-\infty, a) \cup (b, \infty), \text{ where } a \in [-5.5, -0.5] \text{ and } b \in [5, 12] \)

 * Correct option.
\item \( (-\infty, \infty) \)

Corresponds to the variable canceling, which does not happen in this instance.
\end{enumerate}

\textbf{General Comment:} When multiplying or dividing by a negative, flip the sign.
}
\litem{
Using an interval or intervals, describe all the $x$-values within or including a distance of the given values.
\[ \text{ More than } 2 \text{ units from the number } 10. \]
The solution is \( \text{None of the above} \), which is option E.\begin{enumerate}[label=\Alph*.]
\item \( [-8, 12] \)

This describes the values no more than 10 from 2
\item \( (-\infty, -8) \cup (12, \infty) \)

This describes the values more than 10 from 2
\item \( (-\infty, -8] \cup [12, \infty) \)

This describes the values no less than 10 from 2
\item \( (-8, 12) \)

This describes the values less than 10 from 2
\item \( \text{None of the above} \)

Options A-D described the values [more/less than] 10 units from 2, which is the reverse of what the question asked.
\end{enumerate}

\textbf{General Comment:} When thinking about this language, it helps to draw a number line and try points.
}
\litem{
Solve the linear inequality below. Then, choose the constant and interval combination that describes the solution set.
\[ \frac{4}{3} - \frac{4}{4} x \geq \frac{8}{9} x + \frac{9}{2} \]
The solution is \( (-\infty, -1.676] \), which is option D.\begin{enumerate}[label=\Alph*.]
\item \( (-\infty, a], \text{ where } a \in [0.68, 3.68] \)

 $(-\infty, 1.676]$, which corresponds to negating the endpoint of the solution.
\item \( [a, \infty), \text{ where } a \in [-1.68, 0.32] \)

 $[-1.676, \infty)$, which corresponds to switching the direction of the interval. You likely did this if you did not flip the inequality when dividing by a negative!
\item \( [a, \infty), \text{ where } a \in [-1.32, 4.68] \)

 $[1.676, \infty)$, which corresponds to switching the direction of the interval AND negating the endpoint. You likely did this if you did not flip the inequality when dividing by a negative as well as not moving values over to a side properly.
\item \( (-\infty, a], \text{ where } a \in [-1.68, 1.32] \)

* $(-\infty, -1.676]$, which is the correct option.
\item \( \text{None of the above}. \)

You may have chosen this if you thought the inequality did not match the ends of the intervals.
\end{enumerate}

\textbf{General Comment:} Remember that less/greater than or equal to includes the endpoint, while less/greater do not. Also, remember that you need to flip the inequality when you multiply or divide by a negative.
}
\litem{
Solve the linear inequality below. Then, choose the constant and interval combination that describes the solution set.
\[ -9x + 3 < -5x + 9 \]
The solution is \( (-1.5, \infty) \), which is option C.\begin{enumerate}[label=\Alph*.]
\item \( (-\infty, a), \text{ where } a \in [-5.5, -0.5] \)

 $(-\infty, -1.5)$, which corresponds to switching the direction of the interval. You likely did this if you did not flip the inequality when dividing by a negative!
\item \( (a, \infty), \text{ where } a \in [-0.9, 2.1] \)

 $(1.5, \infty)$, which corresponds to negating the endpoint of the solution.
\item \( (a, \infty), \text{ where } a \in [-2.4, -0.3] \)

* $(-1.5, \infty)$, which is the correct option.
\item \( (-\infty, a), \text{ where } a \in [0.5, 5.5] \)

 $(-\infty, 1.5)$, which corresponds to switching the direction of the interval AND negating the endpoint. You likely did this if you did not flip the inequality when dividing by a negative as well as not moving values over to a side properly.
\item \( \text{None of the above}. \)

You may have chosen this if you thought the inequality did not match the ends of the intervals.
\end{enumerate}

\textbf{General Comment:} Remember that less/greater than or equal to includes the endpoint, while less/greater do not. Also, remember that you need to flip the inequality when you multiply or divide by a negative.
}
\litem{
Using an interval or intervals, describe all the $x$-values within or including a distance of the given values.
\[ \text{ No less than } 3 \text{ units from the number } 2. \]
The solution is \( (-\infty, -1] \cup [5, \infty) \), which is option C.\begin{enumerate}[label=\Alph*.]
\item \( [-1, 5] \)

This describes the values no more than 3 from 2
\item \( (-\infty, -1) \cup (5, \infty) \)

This describes the values more than 3 from 2
\item \( (-\infty, -1] \cup [5, \infty) \)

This describes the values no less than 3 from 2
\item \( (-1, 5) \)

This describes the values less than 3 from 2
\item \( \text{None of the above} \)

You likely thought the values in the interval were not correct.
\end{enumerate}

\textbf{General Comment:} When thinking about this language, it helps to draw a number line and try points.
}
\litem{
Solve the linear inequality below. Then, choose the constant and interval combination that describes the solution set.
\[ \frac{10}{8} + \frac{4}{5} x \geq \frac{5}{6} x + \frac{6}{9} \]
The solution is \( (-\infty, 17.5] \), which is option B.\begin{enumerate}[label=\Alph*.]
\item \( (-\infty, a], \text{ where } a \in [-17.5, -15.5] \)

 $(-\infty, -17.5]$, which corresponds to negating the endpoint of the solution.
\item \( (-\infty, a], \text{ where } a \in [16.5, 19.5] \)

* $(-\infty, 17.5]$, which is the correct option.
\item \( [a, \infty), \text{ where } a \in [15.5, 23.5] \)

 $[17.5, \infty)$, which corresponds to switching the direction of the interval. You likely did this if you did not flip the inequality when dividing by a negative!
\item \( [a, \infty), \text{ where } a \in [-19.5, -13.5] \)

 $[-17.5, \infty)$, which corresponds to switching the direction of the interval AND negating the endpoint. You likely did this if you did not flip the inequality when dividing by a negative as well as not moving values over to a side properly.
\item \( \text{None of the above}. \)

You may have chosen this if you thought the inequality did not match the ends of the intervals.
\end{enumerate}

\textbf{General Comment:} Remember that less/greater than or equal to includes the endpoint, while less/greater do not. Also, remember that you need to flip the inequality when you multiply or divide by a negative.
}
\litem{
Solve the linear inequality below. Then, choose the constant and interval combination that describes the solution set.
\[ -7x + 10 \leq 9x -6 \]
The solution is \( [1.0, \infty) \), which is option A.\begin{enumerate}[label=\Alph*.]
\item \( [a, \infty), \text{ where } a \in [0.6, 1.6] \)

* $[1.0, \infty)$, which is the correct option.
\item \( (-\infty, a], \text{ where } a \in [0.5, 2] \)

 $(-\infty, 1.0]$, which corresponds to switching the direction of the interval. You likely did this if you did not flip the inequality when dividing by a negative!
\item \( (-\infty, a], \text{ where } a \in [-5.6, -0.9] \)

 $(-\infty, -1.0]$, which corresponds to switching the direction of the interval AND negating the endpoint. You likely did this if you did not flip the inequality when dividing by a negative as well as not moving values over to a side properly.
\item \( [a, \infty), \text{ where } a \in [-4.6, -0.6] \)

 $[-1.0, \infty)$, which corresponds to negating the endpoint of the solution.
\item \( \text{None of the above}. \)

You may have chosen this if you thought the inequality did not match the ends of the intervals.
\end{enumerate}

\textbf{General Comment:} Remember that less/greater than or equal to includes the endpoint, while less/greater do not. Also, remember that you need to flip the inequality when you multiply or divide by a negative.
}
\litem{
Solve the linear inequality below. Then, choose the constant and interval combination that describes the solution set.
\[ -5 + 6 x \leq \frac{50 x - 5}{5} < 4 + 9 x \]
The solution is \( \text{None of the above.} \), which is option E.\begin{enumerate}[label=\Alph*.]
\item \( [a, b), \text{ where } a \in [0, 1.5] \text{ and } b \in [-5, 1] \)

$[1.00, -5.00)$, which is the correct interval but negatives of the actual endpoints.
\item \( (-\infty, a] \cup (b, \infty), \text{ where } a \in [0.9, 1.7] \text{ and } b \in [-6, -3] \)

$(-\infty, 1.00] \cup (-5.00, \infty)$, which corresponds to displaying the and-inequality as an or-inequality and getting negatives of the actual endpoints.
\item \( (a, b], \text{ where } a \in [0, 1.43] \text{ and } b \in [-9, -2] \)

$(1.00, -5.00]$, which corresponds to flipping the inequality and getting negatives of the actual endpoints.
\item \( (-\infty, a) \cup [b, \infty), \text{ where } a \in [0.4, 2] \text{ and } b \in [-8, -2] \)

$(-\infty, 1.00) \cup [-5.00, \infty)$, which corresponds to displaying the and-inequality as an or-inequality AND flipping the inequality AND getting negatives of the actual endpoints.
\item \( \text{None of the above.} \)

* This is correct as the answer should be $[-1.00, 5.00)$.
\end{enumerate}

\textbf{General Comment:} To solve, you will need to break up the compound inequality into two inequalities. Be sure to keep track of the inequality! It may be best to draw a number line and graph your solution.
}
\litem{
Solve the linear inequality below. Then, choose the constant and interval combination that describes the solution set.
\[ 8 + 8 x > 10 x \text{ or } 5 + 7 x < 8 x \]
The solution is \( (-\infty, 4.0) \text{ or } (5.0, \infty) \), which is option A.\begin{enumerate}[label=\Alph*.]
\item \( (-\infty, a) \cup (b, \infty), \text{ where } a \in [1, 10] \text{ and } b \in [1, 6] \)

 * Correct option.
\item \( (-\infty, a] \cup [b, \infty), \text{ where } a \in [-5, 0] \text{ and } b \in [-10, -1] \)

Corresponds to including the endpoints AND negating.
\item \( (-\infty, a] \cup [b, \infty), \text{ where } a \in [0, 8] \text{ and } b \in [5, 6] \)

Corresponds to including the endpoints (when they should be excluded).
\item \( (-\infty, a) \cup (b, \infty), \text{ where } a \in [-6, -2] \text{ and } b \in [-5, 0] \)

Corresponds to inverting the inequality and negating the solution.
\item \( (-\infty, \infty) \)

Corresponds to the variable canceling, which does not happen in this instance.
\end{enumerate}

\textbf{General Comment:} When multiplying or dividing by a negative, flip the sign.
}
\litem{
Solve the linear inequality below. Then, choose the constant and interval combination that describes the solution set.
\[ -4 - 7 x \leq \frac{-61 x - 9}{9} < 3 - 7 x \]
The solution is \( \text{None of the above.} \), which is option E.\begin{enumerate}[label=\Alph*.]
\item \( [a, b), \text{ where } a \in [12.5, 20.5] \text{ and } b \in [-19, -14] \)

$[13.50, -18.00)$, which is the correct interval but negatives of the actual endpoints.
\item \( (-\infty, a) \cup [b, \infty), \text{ where } a \in [9.5, 16.5] \text{ and } b \in [-21, -14] \)

$(-\infty, 13.50) \cup [-18.00, \infty)$, which corresponds to displaying the and-inequality as an or-inequality AND flipping the inequality AND getting negatives of the actual endpoints.
\item \( (a, b], \text{ where } a \in [10.5, 15.5] \text{ and } b \in [-21, -15] \)

$(13.50, -18.00]$, which corresponds to flipping the inequality and getting negatives of the actual endpoints.
\item \( (-\infty, a] \cup (b, \infty), \text{ where } a \in [10.5, 14.5] \text{ and } b \in [-21, -17] \)

$(-\infty, 13.50] \cup (-18.00, \infty)$, which corresponds to displaying the and-inequality as an or-inequality and getting negatives of the actual endpoints.
\item \( \text{None of the above.} \)

* This is correct as the answer should be $[-13.50, 18.00)$.
\end{enumerate}

\textbf{General Comment:} To solve, you will need to break up the compound inequality into two inequalities. Be sure to keep track of the inequality! It may be best to draw a number line and graph your solution.
}
\end{enumerate}

\end{document}