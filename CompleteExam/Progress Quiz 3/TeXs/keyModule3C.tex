\documentclass{extbook}[14pt]
\usepackage{multicol, enumerate, enumitem, hyperref, color, soul, setspace, parskip, fancyhdr, amssymb, amsthm, amsmath, bbm, latexsym, units, mathtools}
\everymath{\displaystyle}
\usepackage[headsep=0.5cm,headheight=0cm, left=1 in,right= 1 in,top= 1 in,bottom= 1 in]{geometry}
\usepackage{dashrule}  % Package to use the command below to create lines between items
\newcommand{\litem}[1]{\item #1

\rule{\textwidth}{0.4pt}}
\pagestyle{fancy}
\lhead{}
\chead{Answer Key for Progress Quiz 3 Version C}
\rhead{}
\lfoot{3148-2249}
\cfoot{}
\rfoot{Spring 2021}
\begin{document}
\textbf{This key should allow you to understand why you choose the option you did (beyond just getting a question right or wrong). \href{https://xronos.clas.ufl.edu/mac1105spring2020/courseDescriptionAndMisc/Exams/LearningFromResults}{More instructions on how to use this key can be found here}.}

\textbf{If you have a suggestion to make the keys better, \href{https://forms.gle/CZkbZmPbC9XALEE88}{please fill out the short survey here}.}

\textit{Note: This key is auto-generated and may contain issues and/or errors. The keys are reviewed after each exam to ensure grading is done accurately. If there are issues (like duplicate options), they are noted in the offline gradebook. The keys are a work-in-progress to give students as many resources to improve as possible.}

\rule{\textwidth}{0.4pt}

\begin{enumerate}\litem{
Solve the linear inequality below. Then, choose the constant and interval combination that describes the solution set.
\[ -9 + 3 x > 5 x \text{ or } 7 + 7 x < 9 x \]

The solution is \( (-\infty, -4.5) \text{ or } (3.5, \infty) \), which is option D.\begin{enumerate}[label=\Alph*.]
\item \( (-\infty, a] \cup [b, \infty), \text{ where } a \in [-5.15, -4] \text{ and } b \in [2.7, 3.9] \)

Corresponds to including the endpoints (when they should be excluded).
\item \( (-\infty, a] \cup [b, \infty), \text{ where } a \in [-4.24, -2.41] \text{ and } b \in [4.2, 5.7] \)

Corresponds to including the endpoints AND negating.
\item \( (-\infty, a) \cup (b, \infty), \text{ where } a \in [-3.7, -1.8] \text{ and } b \in [4.45, 5.54] \)

Corresponds to inverting the inequality and negating the solution.
\item \( (-\infty, a) \cup (b, \infty), \text{ where } a \in [-4.7, -4] \text{ and } b \in [2.64, 4.03] \)

 * Correct option.
\item \( (-\infty, \infty) \)

Corresponds to the variable canceling, which does not happen in this instance.
\end{enumerate}

\textbf{General Comment:} When multiplying or dividing by a negative, flip the sign.
}
\litem{
Using an interval or intervals, describe all the $x$-values within or including a distance of the given values.
\[ \text{ No more than } 3 \text{ units from the number } 4. \]

The solution is \( [1, 7] \), which is option C.\begin{enumerate}[label=\Alph*.]
\item \( (-\infty, 1) \cup (7, \infty) \)

This describes the values more than 3 from 4
\item \( (-\infty, 1] \cup [7, \infty) \)

This describes the values no less than 3 from 4
\item \( [1, 7] \)

This describes the values no more than 3 from 4
\item \( (1, 7) \)

This describes the values less than 3 from 4
\item \( \text{None of the above} \)

You likely thought the values in the interval were not correct.
\end{enumerate}

\textbf{General Comment:} When thinking about this language, it helps to draw a number line and try points.
}
\litem{
Using an interval or intervals, describe all the $x$-values within or including a distance of the given values.
\[ \text{ No less than } 6 \text{ units from the number } -10. \]

The solution is \( (-\infty, -16] \cup [-4, \infty) \), which is option C.\begin{enumerate}[label=\Alph*.]
\item \( (-16, -4) \)

This describes the values less than 6 from -10
\item \( (-\infty, -16) \cup (-4, \infty) \)

This describes the values more than 6 from -10
\item \( (-\infty, -16] \cup [-4, \infty) \)

This describes the values no less than 6 from -10
\item \( [-16, -4] \)

This describes the values no more than 6 from -10
\item \( \text{None of the above} \)

You likely thought the values in the interval were not correct.
\end{enumerate}

\textbf{General Comment:} When thinking about this language, it helps to draw a number line and try points.
}
\litem{
Solve the linear inequality below. Then, choose the constant and interval combination that describes the solution set.
\[ -5 - 3 x \leq \frac{-13 x - 6}{6} < 8 - 3 x \]

The solution is \( [-4.80, 10.80) \), which is option C.\begin{enumerate}[label=\Alph*.]
\item \( (-\infty, a) \cup [b, \infty), \text{ where } a \in [-8.8, -3.8] \text{ and } b \in [9.8, 13.8] \)

$(-\infty, -4.80) \cup [10.80, \infty)$, which corresponds to displaying the and-inequality as an or-inequality AND flipping the inequality.
\item \( (a, b], \text{ where } a \in [-7.8, 1.2] \text{ and } b \in [7.8, 11.8] \)

$(-4.80, 10.80]$, which corresponds to flipping the inequality.
\item \( [a, b), \text{ where } a \in [-6.8, -2.8] \text{ and } b \in [8.8, 12.8] \)

$[-4.80, 10.80)$, which is the correct option.
\item \( (-\infty, a] \cup (b, \infty), \text{ where } a \in [-7.8, -3.8] \text{ and } b \in [9.8, 12.8] \)

$(-\infty, -4.80] \cup (10.80, \infty)$, which corresponds to displaying the and-inequality as an or-inequality.
\item \( \text{None of the above.} \)


\end{enumerate}

\textbf{General Comment:} To solve, you will need to break up the compound inequality into two inequalities. Be sure to keep track of the inequality! It may be best to draw a number line and graph your solution.
}
\litem{
Solve the linear inequality below. Then, choose the constant and interval combination that describes the solution set.
\[ \frac{-6}{3} - \frac{10}{8} x \leq \frac{-4}{5} x + \frac{5}{6} \]

The solution is \( [-6.296, \infty) \), which is option B.\begin{enumerate}[label=\Alph*.]
\item \( [a, \infty), \text{ where } a \in [5.3, 7.3] \)

 $[6.296, \infty)$, which corresponds to negating the endpoint of the solution.
\item \( [a, \infty), \text{ where } a \in [-7.3, -4.3] \)

* $[-6.296, \infty)$, which is the correct option.
\item \( (-\infty, a], \text{ where } a \in [3.3, 10.3] \)

 $(-\infty, 6.296]$, which corresponds to switching the direction of the interval AND negating the endpoint. You likely did this if you did not flip the inequality when dividing by a negative as well as not moving values over to a side properly.
\item \( (-\infty, a], \text{ where } a \in [-9.3, -5.3] \)

 $(-\infty, -6.296]$, which corresponds to switching the direction of the interval. You likely did this if you did not flip the inequality when dividing by a negative!
\item \( \text{None of the above}. \)

You may have chosen this if you thought the inequality did not match the ends of the intervals.
\end{enumerate}

\textbf{General Comment:} Remember that less/greater than or equal to includes the endpoint, while less/greater do not. Also, remember that you need to flip the inequality when you multiply or divide by a negative.
}
\litem{
Solve the linear inequality below. Then, choose the constant and interval combination that describes the solution set.
\[ 9 + 4 x < \frac{74 x + 5}{9} \leq 9 + 8 x \]

The solution is \( \text{None of the above.} \), which is option E.\begin{enumerate}[label=\Alph*.]
\item \( (-\infty, a) \cup [b, \infty), \text{ where } a \in [-2, -1] \text{ and } b \in [-39, -37] \)

$(-\infty, -2.00) \cup [-38.00, \infty)$, which corresponds to displaying the and-inequality as an or-inequality and getting negatives of the actual endpoints.
\item \( (-\infty, a] \cup (b, \infty), \text{ where } a \in [-3, -1] \text{ and } b \in [-41, -35] \)

$(-\infty, -2.00] \cup (-38.00, \infty)$, which corresponds to displaying the and-inequality as an or-inequality AND flipping the inequality AND getting negatives of the actual endpoints.
\item \( [a, b), \text{ where } a \in [-2, 0] \text{ and } b \in [-39, -33] \)

$[-2.00, -38.00)$, which corresponds to flipping the inequality and getting negatives of the actual endpoints.
\item \( (a, b], \text{ where } a \in [-2, -1] \text{ and } b \in [-39, -37] \)

$(-2.00, -38.00]$, which is the correct interval but negatives of the actual endpoints.
\item \( \text{None of the above.} \)

* This is correct as the answer should be $(2.00, 38.00]$.
\end{enumerate}

\textbf{General Comment:} To solve, you will need to break up the compound inequality into two inequalities. Be sure to keep track of the inequality! It may be best to draw a number line and graph your solution.
}
\litem{
Solve the linear inequality below. Then, choose the constant and interval combination that describes the solution set.
\[ \frac{9}{3} - \frac{6}{8} x \geq \frac{6}{7} x - \frac{8}{6} \]

The solution is \( (-\infty, 2.696] \), which is option A.\begin{enumerate}[label=\Alph*.]
\item \( (-\infty, a], \text{ where } a \in [-0.3, 4.7] \)

* $(-\infty, 2.696]$, which is the correct option.
\item \( (-\infty, a], \text{ where } a \in [-3.7, -0.7] \)

 $(-\infty, -2.696]$, which corresponds to negating the endpoint of the solution.
\item \( [a, \infty), \text{ where } a \in [-0.3, 3.7] \)

 $[2.696, \infty)$, which corresponds to switching the direction of the interval. You likely did this if you did not flip the inequality when dividing by a negative!
\item \( [a, \infty), \text{ where } a \in [-2.7, -1.7] \)

 $[-2.696, \infty)$, which corresponds to switching the direction of the interval AND negating the endpoint. You likely did this if you did not flip the inequality when dividing by a negative as well as not moving values over to a side properly.
\item \( \text{None of the above}. \)

You may have chosen this if you thought the inequality did not match the ends of the intervals.
\end{enumerate}

\textbf{General Comment:} Remember that less/greater than or equal to includes the endpoint, while less/greater do not. Also, remember that you need to flip the inequality when you multiply or divide by a negative.
}
\litem{
Solve the linear inequality below. Then, choose the constant and interval combination that describes the solution set.
\[ -8 + 5 x > 6 x \text{ or } 7 + 3 x < 4 x \]

The solution is \( (-\infty, -8.0) \text{ or } (7.0, \infty) \), which is option D.\begin{enumerate}[label=\Alph*.]
\item \( (-\infty, a] \cup [b, \infty), \text{ where } a \in [-8.35, -7.78] \text{ and } b \in [4.9, 7.7] \)

Corresponds to including the endpoints (when they should be excluded).
\item \( (-\infty, a) \cup (b, \infty), \text{ where } a \in [-7.74, -6.53] \text{ and } b \in [7.15, 8.15] \)

Corresponds to inverting the inequality and negating the solution.
\item \( (-\infty, a] \cup [b, \infty), \text{ where } a \in [-7.42, -6.53] \text{ and } b \in [7.9, 8.6] \)

Corresponds to including the endpoints AND negating.
\item \( (-\infty, a) \cup (b, \infty), \text{ where } a \in [-8.84, -7.14] \text{ and } b \in [6.55, 7.11] \)

 * Correct option.
\item \( (-\infty, \infty) \)

Corresponds to the variable canceling, which does not happen in this instance.
\end{enumerate}

\textbf{General Comment:} When multiplying or dividing by a negative, flip the sign.
}
\litem{
Solve the linear inequality below. Then, choose the constant and interval combination that describes the solution set.
\[ 6x -7 < 9x + 5 \]

The solution is \( (-4.0, \infty) \), which is option A.\begin{enumerate}[label=\Alph*.]
\item \( (a, \infty), \text{ where } a \in [-14, 1] \)

* $(-4.0, \infty)$, which is the correct option.
\item \( (-\infty, a), \text{ where } a \in [-6, -2] \)

 $(-\infty, -4.0)$, which corresponds to switching the direction of the interval. You likely did this if you did not flip the inequality when dividing by a negative!
\item \( (a, \infty), \text{ where } a \in [4, 7] \)

 $(4.0, \infty)$, which corresponds to negating the endpoint of the solution.
\item \( (-\infty, a), \text{ where } a \in [3, 6] \)

 $(-\infty, 4.0)$, which corresponds to switching the direction of the interval AND negating the endpoint. You likely did this if you did not flip the inequality when dividing by a negative as well as not moving values over to a side properly.
\item \( \text{None of the above}. \)

You may have chosen this if you thought the inequality did not match the ends of the intervals.
\end{enumerate}

\textbf{General Comment:} Remember that less/greater than or equal to includes the endpoint, while less/greater do not. Also, remember that you need to flip the inequality when you multiply or divide by a negative.
}
\litem{
Solve the linear inequality below. Then, choose the constant and interval combination that describes the solution set.
\[ -8x + 4 \leq 7x + 5 \]

The solution is \( [-0.067, \infty) \), which is option D.\begin{enumerate}[label=\Alph*.]
\item \( [a, \infty), \text{ where } a \in [0.06, 0.38] \)

 $[0.067, \infty)$, which corresponds to negating the endpoint of the solution.
\item \( (-\infty, a], \text{ where } a \in [0.03, 0.1] \)

 $(-\infty, 0.067]$, which corresponds to switching the direction of the interval AND negating the endpoint. You likely did this if you did not flip the inequality when dividing by a negative as well as not moving values over to a side properly.
\item \( (-\infty, a], \text{ where } a \in [-0.12, -0.05] \)

 $(-\infty, -0.067]$, which corresponds to switching the direction of the interval. You likely did this if you did not flip the inequality when dividing by a negative!
\item \( [a, \infty), \text{ where } a \in [-0.27, 0.03] \)

* $[-0.067, \infty)$, which is the correct option.
\item \( \text{None of the above}. \)

You may have chosen this if you thought the inequality did not match the ends of the intervals.
\end{enumerate}

\textbf{General Comment:} Remember that less/greater than or equal to includes the endpoint, while less/greater do not. Also, remember that you need to flip the inequality when you multiply or divide by a negative.
}
\end{enumerate}

\end{document}