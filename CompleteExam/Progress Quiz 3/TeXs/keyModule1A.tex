\documentclass{extbook}[14pt]
\usepackage{multicol, enumerate, enumitem, hyperref, color, soul, setspace, parskip, fancyhdr, amssymb, amsthm, amsmath, bbm, latexsym, units, mathtools}
\everymath{\displaystyle}
\usepackage[headsep=0.5cm,headheight=0cm, left=1 in,right= 1 in,top= 1 in,bottom= 1 in]{geometry}
\usepackage{dashrule}  % Package to use the command below to create lines between items
\newcommand{\litem}[1]{\item #1

\rule{\textwidth}{0.4pt}}
\pagestyle{fancy}
\lhead{}
\chead{Answer Key for Progress Quiz 3 Version A}
\rhead{}
\lfoot{3148-2249}
\cfoot{}
\rfoot{Spring 2021}
\begin{document}
\textbf{This key should allow you to understand why you choose the option you did (beyond just getting a question right or wrong). \href{https://xronos.clas.ufl.edu/mac1105spring2020/courseDescriptionAndMisc/Exams/LearningFromResults}{More instructions on how to use this key can be found here}.}

\textbf{If you have a suggestion to make the keys better, \href{https://forms.gle/CZkbZmPbC9XALEE88}{please fill out the short survey here}.}

\textit{Note: This key is auto-generated and may contain issues and/or errors. The keys are reviewed after each exam to ensure grading is done accurately. If there are issues (like duplicate options), they are noted in the offline gradebook. The keys are a work-in-progress to give students as many resources to improve as possible.}

\rule{\textwidth}{0.4pt}

\begin{enumerate}\litem{
Choose the \textbf{smallest} set of Complex numbers that the number below belongs to.
\[ \frac{8}{-11}+81i^2 \]

The solution is \( \text{Rational} \), which is option C.\begin{enumerate}[label=\Alph*.]
\item \( \text{Irrational} \)

These cannot be written as a fraction of Integers. Remember: $\pi$ is not an Integer!
\item \( \text{Pure Imaginary} \)

This is a Complex number $(a+bi)$ that \textbf{only} has an imaginary part like $2i$.
\item \( \text{Rational} \)

* This is the correct option!
\item \( \text{Nonreal Complex} \)

This is a Complex number $(a+bi)$ that is not Real (has $i$ as part of the number).
\item \( \text{Not a Complex Number} \)

This is not a number. The only non-Complex number we know is dividing by 0 as this is not a number!
\end{enumerate}

\textbf{General Comment:} Be sure to simplify $i^2 = -1$. This may remove the imaginary portion for your number. If you are having trouble, you may want to look at the \textit{Subgroups of the Real Numbers} section.
}
\litem{
Simplify the expression below and choose the interval the simplification is contained within.
\[ 18 - 12 \div 5 * 3 - (14 * 13) \]

The solution is \( -171.200 \), which is option C.\begin{enumerate}[label=\Alph*.]
\item \( [-169.8, -161.8] \)

 -164.800, which corresponds to an Order of Operations error: not reading left-to-right for multiplication/division.
\item \( [-42.6, -38.6] \)

 -41.600, which corresponds to not distributing a negative correctly.
\item \( [-171.2, -167.2] \)

* -171.200, which is the correct option.
\item \( [199.2, 204.2] \)

 199.200, which corresponds to not distributing addition and subtraction correctly.
\item \( \text{None of the above} \)

 You may have gotten this by making an unanticipated error. If you got a value that is not any of the others, please let the coordinator know so they can help you figure out what happened.
\end{enumerate}

\textbf{General Comment:} While you may remember (or were taught) PEMDAS is done in order, it is actually done as P/E/MD/AS. When we are at MD or AS, we read left to right.
}
\litem{
Simplify the expression below and choose the interval the simplification is contained within.
\[ 14 - 18^2 + 4 \div 7 * 16 \div 20 \]

The solution is \( -309.543 \), which is option B.\begin{enumerate}[label=\Alph*.]
\item \( [-310.18, -309.96] \)

 -309.998, which corresponds to an Order of Operations error: not reading left-to-right for multiplication/division.
\item \( [-309.57, -309.3] \)

* -309.543, this is the correct option
\item \( [338.37, 338.53] \)

 338.457, which corresponds to an Order of Operations error: multiplying by negative before squaring. For example: $(-3)^2 \neq -3^2$
\item \( [337.99, 338.44] \)

 338.002, which corresponds to two Order of Operations errors.
\item \( \text{None of the above} \)

 You may have gotten this by making an unanticipated error. If you got a value that is not any of the others, please let the coordinator know so they can help you figure out what happened.
\end{enumerate}

\textbf{General Comment:} While you may remember (or were taught) PEMDAS is done in order, it is actually done as P/E/MD/AS. When we are at MD or AS, we read left to right.
}
\litem{
Choose the \textbf{smallest} set of Real numbers that the number below belongs to.
\[ \sqrt{\frac{289}{81}} \]

The solution is \( \text{Rational} \), which is option C.\begin{enumerate}[label=\Alph*.]
\item \( \text{Whole} \)

These are the counting numbers with 0 (0, 1, 2, 3, ...)
\item \( \text{Irrational} \)

These cannot be written as a fraction of Integers.
\item \( \text{Rational} \)

* This is the correct option!
\item \( \text{Integer} \)

These are the negative and positive counting numbers (..., -3, -2, -1, 0, 1, 2, 3, ...)
\item \( \text{Not a Real number} \)

These are Nonreal Complex numbers \textbf{OR} things that are not numbers (e.g., dividing by 0).
\end{enumerate}

\textbf{General Comment:} First, you \textbf{NEED} to simplify the expression. This question simplifies to $\frac{17}{9}$. 
 
 Be sure you look at the simplified fraction and not just the decimal expansion. Numbers such as 13, 17, and 19 provide \textbf{long but repeating/terminating decimal expansions!} 
 
 The only ways to *not* be a Real number are: dividing by 0 or taking the square root of a negative number. 
 
 Irrational numbers are more than just square root of 3: adding or subtracting values from square root of 3 is also irrational.
}
\litem{
Simplify the expression below into the form $a+bi$. Then, choose the intervals that $a$ and $b$ belong to.
\[ \frac{54 - 11 i}{4 - 7 i} \]

The solution is \( 4.51  + 5.14 i \), which is option A.\begin{enumerate}[label=\Alph*.]
\item \( a \in [4, 6] \text{ and } b \in [4.5, 6] \)

* $4.51  + 5.14 i$, which is the correct option.
\item \( a \in [1, 3] \text{ and } b \in [-7, -5] \)

 $2.14  - 6.49 i$, which corresponds to forgetting to multiply the conjugate by the numerator and not computing the conjugate correctly.
\item \( a \in [291.5, 293.5] \text{ and } b \in [4.5, 6] \)

 $293.00  + 5.14 i$, which corresponds to forgetting to multiply the conjugate by the numerator and using a plus instead of a minus in the denominator.
\item \( a \in [4, 6] \text{ and } b \in [332.5, 335] \)

 $4.51  + 334.00 i$, which corresponds to forgetting to multiply the conjugate by the numerator.
\item \( a \in [13, 14] \text{ and } b \in [0.5, 2] \)

 $13.50  + 1.57 i$, which corresponds to just dividing the first term by the first term and the second by the second.
\end{enumerate}

\textbf{General Comment:} Multiply the numerator and denominator by the *conjugate* of the denominator, then simplify. For example, if we have $2+3i$, the conjugate is $2-3i$.
}
\litem{
Choose the \textbf{smallest} set of Complex numbers that the number below belongs to.
\[ \sqrt{\frac{1001}{7}}+\sqrt{119} i \]

The solution is \( \text{Nonreal Complex} \), which is option B.\begin{enumerate}[label=\Alph*.]
\item \( \text{Pure Imaginary} \)

This is a Complex number $(a+bi)$ that \textbf{only} has an imaginary part like $2i$.
\item \( \text{Nonreal Complex} \)

* This is the correct option!
\item \( \text{Irrational} \)

These cannot be written as a fraction of Integers. Remember: $\pi$ is not an Integer!
\item \( \text{Rational} \)

These are numbers that can be written as fraction of Integers (e.g., -2/3 + 5)
\item \( \text{Not a Complex Number} \)

This is not a number. The only non-Complex number we know is dividing by 0 as this is not a number!
\end{enumerate}

\textbf{General Comment:} Be sure to simplify $i^2 = -1$. This may remove the imaginary portion for your number. If you are having trouble, you may want to look at the \textit{Subgroups of the Real Numbers} section.
}
\litem{
Simplify the expression below into the form $a+bi$. Then, choose the intervals that $a$ and $b$ belong to.
\[ (-10 + 7 i)(3 - 4 i) \]

The solution is \( -2 + 61 i \), which is option C.\begin{enumerate}[label=\Alph*.]
\item \( a \in [-58, -55] \text{ and } b \in [-21, -13] \)

 $-58 - 19 i$, which corresponds to adding a minus sign in the second term.
\item \( a \in [-30, -21] \text{ and } b \in [-29, -26] \)

 $-30 - 28 i$, which corresponds to just multiplying the real terms to get the real part of the solution and the coefficients in the complex terms to get the complex part.
\item \( a \in [-3, 2] \text{ and } b \in [58, 64] \)

* $-2 + 61 i$, which is the correct option.
\item \( a \in [-58, -55] \text{ and } b \in [19, 24] \)

 $-58 + 19 i$, which corresponds to adding a minus sign in the first term.
\item \( a \in [-3, 2] \text{ and } b \in [-61, -59] \)

 $-2 - 61 i$, which corresponds to adding a minus sign in both terms.
\end{enumerate}

\textbf{General Comment:} You can treat $i$ as a variable and distribute. Just remember that $i^2=-1$, so you can continue to reduce after you distribute.
}
\litem{
Simplify the expression below into the form $a+bi$. Then, choose the intervals that $a$ and $b$ belong to.
\[ \frac{63 + 33 i}{-1 + 5 i} \]

The solution is \( 3.92  - 13.38 i \), which is option E.\begin{enumerate}[label=\Alph*.]
\item \( a \in [-64.5, -62.5] \text{ and } b \in [4.5, 7.5] \)

 $-63.00  + 6.60 i$, which corresponds to just dividing the first term by the first term and the second by the second.
\item \( a \in [3, 4.5] \text{ and } b \in [-348.5, -347.5] \)

 $3.92  - 348.00 i$, which corresponds to forgetting to multiply the conjugate by the numerator.
\item \( a \in [-10.5, -8.5] \text{ and } b \in [10, 12] \)

 $-8.77  + 10.85 i$, which corresponds to forgetting to multiply the conjugate by the numerator and not computing the conjugate correctly.
\item \( a \in [101, 102.5] \text{ and } b \in [-14, -13] \)

 $102.00  - 13.38 i$, which corresponds to forgetting to multiply the conjugate by the numerator and using a plus instead of a minus in the denominator.
\item \( a \in [3, 4.5] \text{ and } b \in [-14, -13] \)

* $3.92  - 13.38 i$, which is the correct option.
\end{enumerate}

\textbf{General Comment:} Multiply the numerator and denominator by the *conjugate* of the denominator, then simplify. For example, if we have $2+3i$, the conjugate is $2-3i$.
}
\litem{
Simplify the expression below into the form $a+bi$. Then, choose the intervals that $a$ and $b$ belong to.
\[ (-10 - 7 i)(9 - 6 i) \]

The solution is \( -132 - 3 i \), which is option C.\begin{enumerate}[label=\Alph*.]
\item \( a \in [-134, -127] \text{ and } b \in [3, 8] \)

 $-132 + 3 i$, which corresponds to adding a minus sign in both terms.
\item \( a \in [-91, -85] \text{ and } b \in [38, 48] \)

 $-90 + 42 i$, which corresponds to just multiplying the real terms to get the real part of the solution and the coefficients in the complex terms to get the complex part.
\item \( a \in [-134, -127] \text{ and } b \in [-4, -1] \)

* $-132 - 3 i$, which is the correct option.
\item \( a \in [-51, -47] \text{ and } b \in [-133, -121] \)

 $-48 - 123 i$, which corresponds to adding a minus sign in the second term.
\item \( a \in [-51, -47] \text{ and } b \in [122, 127] \)

 $-48 + 123 i$, which corresponds to adding a minus sign in the first term.
\end{enumerate}

\textbf{General Comment:} You can treat $i$ as a variable and distribute. Just remember that $i^2=-1$, so you can continue to reduce after you distribute.
}
\litem{
Choose the \textbf{smallest} set of Real numbers that the number below belongs to.
\[ \sqrt{\frac{23104}{361}} \]

The solution is \( \text{Whole} \), which is option B.\begin{enumerate}[label=\Alph*.]
\item \( \text{Irrational} \)

These cannot be written as a fraction of Integers.
\item \( \text{Whole} \)

* This is the correct option!
\item \( \text{Integer} \)

These are the negative and positive counting numbers (..., -3, -2, -1, 0, 1, 2, 3, ...)
\item \( \text{Rational} \)

These are numbers that can be written as fraction of Integers (e.g., -2/3)
\item \( \text{Not a Real number} \)

These are Nonreal Complex numbers \textbf{OR} things that are not numbers (e.g., dividing by 0).
\end{enumerate}

\textbf{General Comment:} First, you \textbf{NEED} to simplify the expression. This question simplifies to $152$. 
 
 Be sure you look at the simplified fraction and not just the decimal expansion. Numbers such as 13, 17, and 19 provide \textbf{long but repeating/terminating decimal expansions!} 
 
 The only ways to *not* be a Real number are: dividing by 0 or taking the square root of a negative number. 
 
 Irrational numbers are more than just square root of 3: adding or subtracting values from square root of 3 is also irrational.
}
\end{enumerate}

\end{document}