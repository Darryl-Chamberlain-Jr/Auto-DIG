\documentclass{extbook}[14pt]
\usepackage{multicol, enumerate, enumitem, hyperref, color, soul, setspace, parskip, fancyhdr, amssymb, amsthm, amsmath, latexsym, units, mathtools}
\everymath{\displaystyle}
\usepackage[headsep=0.5cm,headheight=0cm, left=1 in,right= 1 in,top= 1 in,bottom= 1 in]{geometry}
\usepackage{dashrule}  % Package to use the command below to create lines between items
\newcommand{\litem}[1]{\item #1

\rule{\textwidth}{0.4pt}}
\pagestyle{fancy}
\lhead{}
\chead{Answer Key for Progress Quiz 3 Version A}
\rhead{}
\lfoot{3012-8528}
\cfoot{}
\rfoot{Summer C 2021}
\begin{document}
\textbf{This key should allow you to understand why you choose the option you did (beyond just getting a question right or wrong). \href{https://xronos.clas.ufl.edu/mac1105spring2020/courseDescriptionAndMisc/Exams/LearningFromResults}{More instructions on how to use this key can be found here}.}

\textbf{If you have a suggestion to make the keys better, \href{https://forms.gle/CZkbZmPbC9XALEE88}{please fill out the short survey here}.}

\textit{Note: This key is auto-generated and may contain issues and/or errors. The keys are reviewed after each exam to ensure grading is done accurately. If there are issues (like duplicate options), they are noted in the offline gradebook. The keys are a work-in-progress to give students as many resources to improve as possible.}

\rule{\textwidth}{0.4pt}

\begin{enumerate}\litem{
Simplify the expression below into the form $a+bi$. Then, choose the intervals that $a$ and $b$ belong to.
\[ (8 + 2 i)(-9 + 7 i) \]The solution is \( -86 + 38 i \), which is option E.\begin{enumerate}[label=\Alph*.]
\item \( a \in [-59, -55] \text{ and } b \in [-78, -72] \)

 $-58 - 74 i$, which corresponds to adding a minus sign in the second term.
\item \( a \in [-73, -63] \text{ and } b \in [11, 16] \)

 $-72 + 14 i$, which corresponds to just multiplying the real terms to get the real part of the solution and the coefficients in the complex terms to get the complex part.
\item \( a \in [-87, -85] \text{ and } b \in [-44, -36] \)

 $-86 - 38 i$, which corresponds to adding a minus sign in both terms.
\item \( a \in [-59, -55] \text{ and } b \in [74, 77] \)

 $-58 + 74 i$, which corresponds to adding a minus sign in the first term.
\item \( a \in [-87, -85] \text{ and } b \in [34, 41] \)

* $-86 + 38 i$, which is the correct option.
\end{enumerate}

\textbf{General Comment:} You can treat $i$ as a variable and distribute. Just remember that $i^2=-1$, so you can continue to reduce after you distribute.
}
\litem{
Simplify the expression below and choose the interval the simplification is contained within.
\[ 3 - 2^2 + 1 \div 10 * 18 \div 11 \]The solution is \( -0.836 \), which is option D.\begin{enumerate}[label=\Alph*.]
\item \( [-1.12, -0.9] \)

 -0.999, which corresponds to an Order of Operations error: not reading left-to-right for multiplication/division.
\item \( [7.09, 7.26] \)

 7.164, which corresponds to an Order of Operations error: multiplying by negative before squaring. For example: $(-3)^2 \neq -3^2$
\item \( [6.71, 7.11] \)

 7.001, which corresponds to two Order of Operations errors.
\item \( [-0.85, -0.3] \)

* -0.836, this is the correct option
\item \( \text{None of the above} \)

 You may have gotten this by making an unanticipated error. If you got a value that is not any of the others, please let the coordinator know so they can help you figure out what happened.
\end{enumerate}

\textbf{General Comment:} While you may remember (or were taught) PEMDAS is done in order, it is actually done as P/E/MD/AS. When we are at MD or AS, we read left to right.
}
\litem{
Choose the \textbf{smallest} set of Complex numbers that the number below belongs to.
\[ \sqrt{\frac{1188}{9}}+\sqrt{45} i \]The solution is \( \text{Nonreal Complex} \), which is option B.\begin{enumerate}[label=\Alph*.]
\item \( \text{Rational} \)

These are numbers that can be written as fraction of Integers (e.g., -2/3 + 5)
\item \( \text{Nonreal Complex} \)

* This is the correct option!
\item \( \text{Pure Imaginary} \)

This is a Complex number $(a+bi)$ that \textbf{only} has an imaginary part like $2i$.
\item \( \text{Not a Complex Number} \)

This is not a number. The only non-Complex number we know is dividing by 0 as this is not a number!
\item \( \text{Irrational} \)

These cannot be written as a fraction of Integers. Remember: $\pi$ is not an Integer!
\end{enumerate}

\textbf{General Comment:} Be sure to simplify $i^2 = -1$. This may remove the imaginary portion for your number. If you are having trouble, you may want to look at the \textit{Subgroups of the Real Numbers} section.
}
\litem{
Choose the \textbf{smallest} set of Complex numbers that the number below belongs to.
\[ \sqrt{\frac{625}{0}}+\sqrt{45} i \]The solution is \( \text{Not a Complex Number} \), which is option D.\begin{enumerate}[label=\Alph*.]
\item \( \text{Irrational} \)

These cannot be written as a fraction of Integers. Remember: $\pi$ is not an Integer!
\item \( \text{Pure Imaginary} \)

This is a Complex number $(a+bi)$ that \textbf{only} has an imaginary part like $2i$.
\item \( \text{Rational} \)

These are numbers that can be written as fraction of Integers (e.g., -2/3 + 5)
\item \( \text{Not a Complex Number} \)

* This is the correct option!
\item \( \text{Nonreal Complex} \)

This is a Complex number $(a+bi)$ that is not Real (has $i$ as part of the number).
\end{enumerate}

\textbf{General Comment:} Be sure to simplify $i^2 = -1$. This may remove the imaginary portion for your number. If you are having trouble, you may want to look at the \textit{Subgroups of the Real Numbers} section.
}
\litem{
Simplify the expression below into the form $a+bi$. Then, choose the intervals that $a$ and $b$ belong to.
\[ \frac{-9 - 33 i}{-7 + 5 i} \]The solution is \( -1.38  + 3.73 i \), which is option B.\begin{enumerate}[label=\Alph*.]
\item \( a \in [-103.5, -101] \text{ and } b \in [3, 4.5] \)

 $-102.00  + 3.73 i$, which corresponds to forgetting to multiply the conjugate by the numerator and using a plus instead of a minus in the denominator.
\item \( a \in [-3, -1] \text{ and } b \in [3, 4.5] \)

* $-1.38  + 3.73 i$, which is the correct option.
\item \( a \in [-3, -1] \text{ and } b \in [275.5, 276.5] \)

 $-1.38  + 276.00 i$, which corresponds to forgetting to multiply the conjugate by the numerator.
\item \( a \in [1.5, 4] \text{ and } b \in [2, 3] \)

 $3.08  + 2.51 i$, which corresponds to forgetting to multiply the conjugate by the numerator and not computing the conjugate correctly.
\item \( a \in [0.5, 1.5] \text{ and } b \in [-8, -6.5] \)

 $1.29  - 6.60 i$, which corresponds to just dividing the first term by the first term and the second by the second.
\end{enumerate}

\textbf{General Comment:} Multiply the numerator and denominator by the *conjugate* of the denominator, then simplify. For example, if we have $2+3i$, the conjugate is $2-3i$.
}
\litem{
Simplify the expression below and choose the interval the simplification is contained within.
\[ 6 - 3^2 + 19 \div 5 * 10 \div 2 \]The solution is \( 16.000 \), which is option B.\begin{enumerate}[label=\Alph*.]
\item \( [33.31, 34.53] \)

 34.000, which corresponds to an Order of Operations error: multiplying by negative before squaring. For example: $(-3)^2 \neq -3^2$
\item \( [15.92, 16.33] \)

* 16.000, this is the correct option
\item \( [14.78, 15.26] \)

 15.190, which corresponds to two Order of Operations errors.
\item \( [-2.93, -1.92] \)

 -2.810, which corresponds to an Order of Operations error: not reading left-to-right for multiplication/division.
\item \( \text{None of the above} \)

 You may have gotten this by making an unanticipated error. If you got a value that is not any of the others, please let the coordinator know so they can help you figure out what happened.
\end{enumerate}

\textbf{General Comment:} While you may remember (or were taught) PEMDAS is done in order, it is actually done as P/E/MD/AS. When we are at MD or AS, we read left to right.
}
\litem{
Simplify the expression below into the form $a+bi$. Then, choose the intervals that $a$ and $b$ belong to.
\[ (-7 - 8 i)(3 + 10 i) \]The solution is \( 59 - 94 i \), which is option B.\begin{enumerate}[label=\Alph*.]
\item \( a \in [56, 63] \text{ and } b \in [93, 97] \)

 $59 + 94 i$, which corresponds to adding a minus sign in both terms.
\item \( a \in [56, 63] \text{ and } b \in [-96, -92] \)

* $59 - 94 i$, which is the correct option.
\item \( a \in [-103, -100] \text{ and } b \in [-46, -40] \)

 $-101 - 46 i$, which corresponds to adding a minus sign in the first term.
\item \( a \in [-24, -16] \text{ and } b \in [-85, -73] \)

 $-21 - 80 i$, which corresponds to just multiplying the real terms to get the real part of the solution and the coefficients in the complex terms to get the complex part.
\item \( a \in [-103, -100] \text{ and } b \in [46, 47] \)

 $-101 + 46 i$, which corresponds to adding a minus sign in the second term.
\end{enumerate}

\textbf{General Comment:} You can treat $i$ as a variable and distribute. Just remember that $i^2=-1$, so you can continue to reduce after you distribute.
}
\litem{
Simplify the expression below into the form $a+bi$. Then, choose the intervals that $a$ and $b$ belong to.
\[ \frac{-72 - 66 i}{3 + 4 i} \]The solution is \( -19.20  + 3.60 i \), which is option C.\begin{enumerate}[label=\Alph*.]
\item \( a \in [-25, -23.5] \text{ and } b \in [-17.5, -15.5] \)

 $-24.00  - 16.50 i$, which corresponds to just dividing the first term by the first term and the second by the second.
\item \( a \in [-20.5, -19] \text{ and } b \in [89.5, 91] \)

 $-19.20  + 90.00 i$, which corresponds to forgetting to multiply the conjugate by the numerator.
\item \( a \in [-20.5, -19] \text{ and } b \in [3, 5] \)

* $-19.20  + 3.60 i$, which is the correct option.
\item \( a \in [1.5, 2] \text{ and } b \in [-20, -19] \)

 $1.92  - 19.44 i$, which corresponds to forgetting to multiply the conjugate by the numerator and not computing the conjugate correctly.
\item \( a \in [-481, -479] \text{ and } b \in [3, 5] \)

 $-480.00  + 3.60 i$, which corresponds to forgetting to multiply the conjugate by the numerator and using a plus instead of a minus in the denominator.
\end{enumerate}

\textbf{General Comment:} Multiply the numerator and denominator by the *conjugate* of the denominator, then simplify. For example, if we have $2+3i$, the conjugate is $2-3i$.
}
\litem{
Choose the \textbf{smallest} set of Real numbers that the number below belongs to.
\[ \sqrt{\frac{23}{0}} \]The solution is \( \text{Not a Real number} \), which is option A.\begin{enumerate}[label=\Alph*.]
\item \( \text{Not a Real number} \)

* This is the correct option!
\item \( \text{Whole} \)

These are the counting numbers with 0 (0, 1, 2, 3, ...)
\item \( \text{Rational} \)

These are numbers that can be written as fraction of Integers (e.g., -2/3)
\item \( \text{Irrational} \)

These cannot be written as a fraction of Integers.
\item \( \text{Integer} \)

These are the negative and positive counting numbers (..., -3, -2, -1, 0, 1, 2, 3, ...)
\end{enumerate}

\textbf{General Comment:} First, you \textbf{NEED} to simplify the expression. This question simplifies to $\sqrt{\frac{23}{0}}$. 
 
 Be sure you look at the simplified fraction and not just the decimal expansion. Numbers such as 13, 17, and 19 provide \textbf{long but repeating/terminating decimal expansions!} 
 
 The only ways to *not* be a Real number are: dividing by 0 or taking the square root of a negative number. 
 
 Irrational numbers are more than just square root of 3: adding or subtracting values from square root of 3 is also irrational.
}
\litem{
Choose the \textbf{smallest} set of Real numbers that the number below belongs to.
\[ -\sqrt{\frac{256}{625}} \]The solution is \( \text{Rational} \), which is option B.\begin{enumerate}[label=\Alph*.]
\item \( \text{Integer} \)

These are the negative and positive counting numbers (..., -3, -2, -1, 0, 1, 2, 3, ...)
\item \( \text{Rational} \)

* This is the correct option!
\item \( \text{Not a Real number} \)

These are Nonreal Complex numbers \textbf{OR} things that are not numbers (e.g., dividing by 0).
\item \( \text{Whole} \)

These are the counting numbers with 0 (0, 1, 2, 3, ...)
\item \( \text{Irrational} \)

These cannot be written as a fraction of Integers.
\end{enumerate}

\textbf{General Comment:} First, you \textbf{NEED} to simplify the expression. This question simplifies to $-\frac{16}{25}$. 
 
 Be sure you look at the simplified fraction and not just the decimal expansion. Numbers such as 13, 17, and 19 provide \textbf{long but repeating/terminating decimal expansions!} 
 
 The only ways to *not* be a Real number are: dividing by 0 or taking the square root of a negative number. 
 
 Irrational numbers are more than just square root of 3: adding or subtracting values from square root of 3 is also irrational.
}
\end{enumerate}

\end{document}