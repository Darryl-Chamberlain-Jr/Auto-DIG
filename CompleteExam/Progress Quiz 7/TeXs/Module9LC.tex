\documentclass[14pt]{extbook}
\usepackage{multicol, enumerate, enumitem, hyperref, color, soul, setspace, parskip, fancyhdr} %General Packages
\usepackage{amssymb, amsthm, amsmath, latexsym, units, mathtools} %Math Packages
\everymath{\displaystyle} %All math in Display Style
% Packages with additional options
\usepackage[headsep=0.5cm,headheight=12pt, left=1 in,right= 1 in,top= 1 in,bottom= 1 in]{geometry}
\usepackage[usenames,dvipsnames]{xcolor}
\usepackage{dashrule}  % Package to use the command below to create lines between items
\newcommand{\litem}[1]{\item#1\hspace*{-1cm}\rule{\textwidth}{0.4pt}}
\pagestyle{fancy}
\lhead{Progress Quiz 7}
\chead{}
\rhead{Version C}
\lfoot{3510-5252}
\cfoot{}
\rfoot{Summer C 2021}
\begin{document}

\begin{enumerate}
\litem{
Add the following functions, then choose the domain of the resulting function from the list below.\[ f(x) = \sqrt{-3x-16}  \text{ and } g(x) = 3x^{3} +9 x \]\begin{enumerate}[label=\Alph*.]
\item \( \text{ The domain is all Real numbers except } x = a, \text{ where } a \in [4.2, 12.2] \)
\item \( \text{ The domain is all Real numbers less than or equal to } x = a, \text{ where } a \in [-7.33, 0.67] \)
\item \( \text{ The domain is all Real numbers greater than or equal to } x = a, \text{ where } a \in [5.33, 11.33] \)
\item \( \text{ The domain is all Real numbers except } x = a \text{ and } x = b, \text{ where } a \in [5.67, 8.67] \text{ and } b \in [4.6, 12.6] \)
\item \( \text{ The domain is all Real numbers. } \)

\end{enumerate} }
\litem{
Find the inverse of the function below. Then, evaluate the inverse at $x = 7$ and choose the interval that $f^-1(7)$ belongs to.\[ f(x) = \ln{(x-5)}-2 \]\begin{enumerate}[label=\Alph*.]
\item \( f^{-1}(7) \in [153.41, 157.41] \)
\item \( f^{-1}(7) \in [8107.08, 8112.08] \)
\item \( f^{-1}(7) \in [2.39, 7.39] \)
\item \( f^{-1}(7) \in [8096.08, 8099.08] \)
\item \( f^{-1}(7) \in [162748.79, 162755.79] \)

\end{enumerate} }
\litem{
Determine whether the function below is 1-1.\[ f(x) = 25 x^2 + 220 x + 484 \]\begin{enumerate}[label=\Alph*.]
\item \( \text{Yes, the function is 1-1.} \)
\item \( \text{No, because the range of the function is not $(-\infty, \infty)$.} \)
\item \( \text{No, because there is an $x$-value that goes to 2 different $y$-values.} \)
\item \( \text{No, because there is a $y$-value that goes to 2 different $x$-values.} \)
\item \( \text{No, because the domain of the function is not $(-\infty, \infty)$.} \)

\end{enumerate} }
\litem{
Find the inverse of the function below. Then, evaluate the inverse at $x = 10$ and choose the interval that $f^-1(10)$ belongs to.\[ f(x) = e^{x+5}-3 \]\begin{enumerate}[label=\Alph*.]
\item \( f^{-1}(10) \in [-1.1, -0.84] \)
\item \( f^{-1}(10) \in [7.32, 8.32] \)
\item \( f^{-1}(10) \in [-0.32, 0.71] \)
\item \( f^{-1}(10) \in [-2.55, -1.81] \)
\item \( f^{-1}(10) \in [-2.19, -1.3] \)

\end{enumerate} }
\litem{
Determine whether the function below is 1-1.\[ f(x) = (5 x + 17)^3 \]\begin{enumerate}[label=\Alph*.]
\item \( \text{No, because the range of the function is not $(-\infty, \infty)$.} \)
\item \( \text{Yes, the function is 1-1.} \)
\item \( \text{No, because there is a $y$-value that goes to 2 different $x$-values.} \)
\item \( \text{No, because there is an $x$-value that goes to 2 different $y$-values.} \)
\item \( \text{No, because the domain of the function is not $(-\infty, \infty)$.} \)

\end{enumerate} }
\litem{
Choose the interval below that $f$ composed with $g$ at $x=-1$ is in.\[ f(x) = -4x^{3} -2 x^{2} +4 x \text{ and } g(x) = -3x^{3} -4 x^{2} -x -3 \]\begin{enumerate}[label=\Alph*.]
\item \( (f \circ g)(-1) \in [10, 23] \)
\item \( (f \circ g)(-1) \in [83, 91] \)
\item \( (f \circ g)(-1) \in [75, 84] \)
\item \( (f \circ g)(-1) \in [3, 15] \)
\item \( \text{It is not possible to compose the two functions.} \)

\end{enumerate} }
\litem{
Choose the interval below that $f$ composed with $g$ at $x=1$ is in.\[ f(x) = 2x^{3} +2 x^{2} -3 x \text{ and } g(x) = -4x^{3} + x^{2} +2 x -3 \]\begin{enumerate}[label=\Alph*.]
\item \( (f \circ g)(1) \in [-9, -3] \)
\item \( (f \circ g)(1) \in [1, 4] \)
\item \( (f \circ g)(1) \in [-84, -83] \)
\item \( (f \circ g)(1) \in [-76, -73] \)
\item \( \text{It is not possible to compose the two functions.} \)

\end{enumerate} }
\litem{
Find the inverse of the function below (if it exists). Then, evaluate the inverse at $x = -10$ and choose the interval that $f^-1(-10)$ belongs to.\[ f(x) = \sqrt[3]{4 x - 3} \]\begin{enumerate}[label=\Alph*.]
\item \( f^{-1}(-10) \in [-250, -247] \)
\item \( f^{-1}(-10) \in [246.4, 250.6] \)
\item \( f^{-1}(-10) \in [249.9, 253] \)
\item \( f^{-1}(-10) \in [-250.9, -249.9] \)
\item \( \text{ The function is not invertible for all Real numbers. } \)

\end{enumerate} }
\litem{
Find the inverse of the function below (if it exists). Then, evaluate the inverse at $x = -15$ and choose the interval that $f^-1(-15)$ belongs to.\[ f(x) = 5 x^2 - 2 \]\begin{enumerate}[label=\Alph*.]
\item \( f^{-1}(-15) \in [5.33, 5.75] \)
\item \( f^{-1}(-15) \in [1.81, 2.05] \)
\item \( f^{-1}(-15) \in [1.48, 1.65] \)
\item \( f^{-1}(-15) \in [4.33, 4.91] \)
\item \( \text{ The function is not invertible for all Real numbers. } \)

\end{enumerate} }
\litem{
Subtract the following functions, then choose the domain of the resulting function from the list below.\[ f(x) = 9x^{2} +8 x + 6 \text{ and } g(x) = 5x^{3} +8 x^{2} +5 x + 6 \]\begin{enumerate}[label=\Alph*.]
\item \( \text{ The domain is all Real numbers less than or equal to } x = a, \text{ where } a \in [0.5, 5.5] \)
\item \( \text{ The domain is all Real numbers except } x = a, \text{ where } a \in [6.2, 7.2] \)
\item \( \text{ The domain is all Real numbers greater than or equal to } x = a, \text{ where } a \in [2.67, 9.67] \)
\item \( \text{ The domain is all Real numbers except } x = a \text{ and } x = b, \text{ where } a \in [-7.33, 1.67] \text{ and } b \in [-5.25, -2.25] \)
\item \( \text{ The domain is all Real numbers. } \)

\end{enumerate} }
\end{enumerate}

\end{document}