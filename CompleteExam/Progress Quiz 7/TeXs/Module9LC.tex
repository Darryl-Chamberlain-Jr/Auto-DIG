\documentclass[14pt]{extbook}
\usepackage{multicol, enumerate, enumitem, hyperref, color, soul, setspace, parskip, fancyhdr} %General Packages
\usepackage{amssymb, amsthm, amsmath, latexsym, units, mathtools} %Math Packages
\everymath{\displaystyle} %All math in Display Style
% Packages with additional options
\usepackage[headsep=0.5cm,headheight=12pt, left=1 in,right= 1 in,top= 1 in,bottom= 1 in]{geometry}
\usepackage[usenames,dvipsnames]{xcolor}
\usepackage{dashrule}  % Package to use the command below to create lines between items
\newcommand{\litem}[1]{\item#1\hspace*{-1cm}\rule{\textwidth}{0.4pt}}
\pagestyle{fancy}
\lhead{Progress Quiz 7}
\chead{}
\rhead{Version C}
\lfoot{4173-5738}
\cfoot{}
\rfoot{Spring 2021}
\begin{document}

\begin{enumerate}
\litem{
Find the inverse of the function below (if it exists). Then, evaluate the inverse at $x = 14$ and choose the interval that $f^{-1}(14)$ belongs to.\[ f(x) = 2 x^2 + 3 \]\begin{enumerate}[label=\Alph*.]
\item \( f^{-1}(14) \in [2.4, 3.69] \)
\item \( f^{-1}(14) \in [2.3, 2.6] \)
\item \( f^{-1}(14) \in [3.69, 4.71] \)
\item \( f^{-1}(14) \in [5.04, 5.78] \)
\item \( \text{ The function is not invertible for all Real numbers. } \)

\end{enumerate} }
\litem{
Determine whether the function below is 1-1.\[ f(x) = \sqrt{-6 x + 33} \]\begin{enumerate}[label=\Alph*.]
\item \( \text{No, because there is an $x$-value that goes to 2 different $y$-values.} \)
\item \( \text{Yes, the function is 1-1.} \)
\item \( \text{No, because the range of the function is not $(-\infty, \infty)$.} \)
\item \( \text{No, because there is a $y$-value that goes to 2 different $x$-values.} \)
\item \( \text{No, because the domain of the function is not $(-\infty, \infty)$.} \)

\end{enumerate} }
\litem{
Choose the interval below that $f$ composed with $g$ at $x=1$ is in.\[ f(x) = -2x^{3} +2 x^{2} +x -3 \text{ and } g(x) = -x^{3} +2 x^{2} -2 x -2 \]\begin{enumerate}[label=\Alph*.]
\item \( (f \circ g)(1) \in [9, 12] \)
\item \( (f \circ g)(1) \in [17, 25] \)
\item \( (f \circ g)(1) \in [71, 76] \)
\item \( (f \circ g)(1) \in [63, 68] \)
\item \( \text{It is not possible to compose the two functions.} \)

\end{enumerate} }
\litem{
Add the following functions, then choose the domain of the resulting function from the list below.\[ f(x) = \sqrt{3x-21}  \text{ and } g(x) = 3x^{3} +8 x^{2} +6 x \]\begin{enumerate}[label=\Alph*.]
\item \( \text{ The domain is all Real numbers less than or equal to } x = a, \text{ where } a \in [-0.5, 9.5] \)
\item \( \text{ The domain is all Real numbers except } x = a, \text{ where } a \in [3.67, 14.67] \)
\item \( \text{ The domain is all Real numbers greater than or equal to } x = a, \text{ where } a \in [2, 9] \)
\item \( \text{ The domain is all Real numbers except } x = a \text{ and } x = b, \text{ where } a \in [-12.67, -2.67] \text{ and } b \in [1.17, 12.17] \)
\item \( \text{ The domain is all Real numbers. } \)

\end{enumerate} }
\litem{
Find the inverse of the function below. Then, evaluate the inverse at $x = 8$ and choose the interval that $f^{-1}(8)$ belongs to.\[ f(x) = \ln{(x-4)}-5 \]\begin{enumerate}[label=\Alph*.]
\item \( f^{-1}(8) \in [442407.39, 442415.39] \)
\item \( f^{-1}(8) \in [162742.79, 162756.79] \)
\item \( f^{-1}(8) \in [442417.39, 442419.39] \)
\item \( f^{-1}(8) \in [20.09, 28.09] \)
\item \( f^{-1}(8) \in [47.6, 50.6] \)

\end{enumerate} }
\litem{
Subtract the following functions, then choose the domain of the resulting function from the list below.\[ f(x) = \frac{4}{5x-34} \text{ and } g(x) = \frac{1}{5x-21} \]\begin{enumerate}[label=\Alph*.]
\item \( \text{ The domain is all Real numbers greater than or equal to } x = a, \text{ where } a \in [-8, 2] \)
\item \( \text{ The domain is all Real numbers less than or equal to } x = a, \text{ where } a \in [-5.2, -4.2] \)
\item \( \text{ The domain is all Real numbers except } x = a, \text{ where } a \in [-9.2, 2.8] \)
\item \( \text{ The domain is all Real numbers except } x = a \text{ and } x = b, \text{ where } a \in [5.8, 7.8] \text{ and } b \in [-1.8, 14.2] \)
\item \( \text{ The domain is all Real numbers. } \)

\end{enumerate} }
\litem{
Determine whether the function below is 1-1.\[ f(x) = 16 x^2 + 136 x + 289 \]\begin{enumerate}[label=\Alph*.]
\item \( \text{Yes, the function is 1-1.} \)
\item \( \text{No, because there is a $y$-value that goes to 2 different $x$-values.} \)
\item \( \text{No, because the range of the function is not $(-\infty, \infty)$.} \)
\item \( \text{No, because the domain of the function is not $(-\infty, \infty)$.} \)
\item \( \text{No, because there is an $x$-value that goes to 2 different $y$-values.} \)

\end{enumerate} }
\litem{
Find the inverse of the function below (if it exists). Then, evaluate the inverse at $x = 13$ and choose the interval the $f^{-1}(13)$ belongs to.\[ f(x) = \sqrt[3]{3 x - 4} \]\begin{enumerate}[label=\Alph*.]
\item \( f^{-1}(13) \in [-734, -733.1] \)
\item \( f^{-1}(13) \in [-732.5, -730] \)
\item \( f^{-1}(13) \in [732.9, 735.1] \)
\item \( f^{-1}(13) \in [729.8, 732.7] \)
\item \( \text{ The function is not invertible for all Real numbers. } \)

\end{enumerate} }
\litem{
Find the inverse of the function below. Then, evaluate the inverse at $x = 7$ and choose the interval that $f^{-1}(7)$ belongs to.\[ f(x) = \ln{(x+5)}+4 \]\begin{enumerate}[label=\Alph*.]
\item \( f^{-1}(7) \in [12.5, 18.2] \)
\item \( f^{-1}(7) \in [22.8, 27] \)
\item \( f^{-1}(7) \in [59867.2, 59870.4] \)
\item \( f^{-1}(7) \in [162756.3, 162759.2] \)
\item \( f^{-1}(7) \in [10, 15] \)

\end{enumerate} }
\litem{
Choose the interval below that $f$ composed with $g$ at $x=-1$ is in.\[ f(x) = 4x^{3} +2 x^{2} +x \text{ and } g(x) = -2x^{3} -2 x^{2} -4 x -4 \]\begin{enumerate}[label=\Alph*.]
\item \( (f \circ g)(-1) \in [32, 36] \)
\item \( (f \circ g)(-1) \in [-2, 2] \)
\item \( (f \circ g)(-1) \in [44, 45] \)
\item \( (f \circ g)(-1) \in [9, 16] \)
\item \( \text{It is not possible to compose the two functions.} \)

\end{enumerate} }
\end{enumerate}

\end{document}