\documentclass{extbook}[14pt]
\usepackage{multicol, enumerate, enumitem, hyperref, color, soul, setspace, parskip, fancyhdr, amssymb, amsthm, amsmath, latexsym, units, mathtools}
\everymath{\displaystyle}
\usepackage[headsep=0.5cm,headheight=0cm, left=1 in,right= 1 in,top= 1 in,bottom= 1 in]{geometry}
\usepackage{dashrule}  % Package to use the command below to create lines between items
\newcommand{\litem}[1]{\item #1

\rule{\textwidth}{0.4pt}}
\pagestyle{fancy}
\lhead{}
\chead{Answer Key for Progress Quiz 7 Version B}
\rhead{}
\lfoot{3510-5252}
\cfoot{}
\rfoot{Summer C 2021}
\begin{document}
\textbf{This key should allow you to understand why you choose the option you did (beyond just getting a question right or wrong). \href{https://xronos.clas.ufl.edu/mac1105spring2020/courseDescriptionAndMisc/Exams/LearningFromResults}{More instructions on how to use this key can be found here}.}

\textbf{If you have a suggestion to make the keys better, \href{https://forms.gle/CZkbZmPbC9XALEE88}{please fill out the short survey here}.}

\textit{Note: This key is auto-generated and may contain issues and/or errors. The keys are reviewed after each exam to ensure grading is done accurately. If there are issues (like duplicate options), they are noted in the offline gradebook. The keys are a work-in-progress to give students as many resources to improve as possible.}

\rule{\textwidth}{0.4pt}

\begin{enumerate}\litem{
Which of the following intervals describes the Range of the function below?
\[ f(x) = \log_2{(x-7)}-1 \]The solution is \( (\infty, \infty) \), which is option E.\begin{enumerate}[label=\Alph*.]
\item \( [a, \infty), a \in [-9, -6.1] \)

$[-7, \infty)$, which corresponds to using the negative of the horizontal shift AND including the endpoint.
\item \( (-\infty, a), a \in [0.9, 2.4] \)

$(-\infty, 1)$, which corresponds to using the using the negative of vertical shift on $(0, \infty)$.
\item \( (-\infty, a), a \in [-2.7, 0.2] \)

$(-\infty, -1)$, which corresponds to using the vertical shift while the Range is $(-\infty, \infty)$.
\item \( [a, \infty), a \in [6.8, 8.3] \)

$[-1, \infty)$, which corresponds to using the flipped Domain AND including the endpoint.
\item \( (-\infty, \infty) \)

*This is the correct option.
\end{enumerate}

\textbf{General Comment:} \textbf{General Comments}: The domain of a basic logarithmic function is $(0, \infty)$ and the Range is $(-\infty, \infty)$. We can use shifts when finding the Domain, but the Range will always be all Real numbers.
}
\litem{
Solve the equation for $x$ and choose the interval that contains the solution (if it exists).
\[ \log_{5}{(4x+8)}+6 = 2 \]The solution is \( x = -2.000 \), which is option C.\begin{enumerate}[label=\Alph*.]
\item \( x \in [-258, -255] \)

$x = -258.000$, which corresponds to reversing the base and exponent when converting.
\item \( x \in [-256, -253] \)

$x = -254.000$, which corresponds to reversing the base and exponent when converting and reversing the value with $x$.
\item \( x \in [-5, 1] \)

* $x = -2.000$, which is the correct option.
\item \( x \in [3.25, 10.25] \)

$x = 4.250$, which corresponds to ignoring the vertical shift when converting to exponential form.
\item \( \text{There is no Real solution to the equation.} \)

Corresponds to believing a negative coefficient within the log equation means there is no Real solution.
\end{enumerate}

\textbf{General Comment:} \textbf{General Comments:} First, get the equation in the form $\log_b{(cx+d)} = a$. Then, convert to $b^a = cx+d$ and solve.
}
\litem{
Which of the following intervals describes the Range of the function below?
\[ f(x) = -\log_2{(x-1)}-1 \]The solution is \( (\infty, \infty) \), which is option E.\begin{enumerate}[label=\Alph*.]
\item \( (-\infty, a), a \in [0.6, 3] \)

$(-\infty, 1)$, which corresponds to using the using the negative of vertical shift on $(0, \infty)$.
\item \( [a, \infty), a \in [0.6, 3] \)

$[-1, \infty)$, which corresponds to using the flipped Domain AND including the endpoint.
\item \( (-\infty, a), a \in [-1.1, 0.8] \)

$(-\infty, -1)$, which corresponds to using the vertical shift while the Range is $(-\infty, \infty)$.
\item \( [a, \infty), a \in [-1.1, 0.8] \)

$[-1, \infty)$, which corresponds to using the negative of the horizontal shift AND including the endpoint.
\item \( (-\infty, \infty) \)

*This is the correct option.
\end{enumerate}

\textbf{General Comment:} \textbf{General Comments}: The domain of a basic logarithmic function is $(0, \infty)$ and the Range is $(-\infty, \infty)$. We can use shifts when finding the Domain, but the Range will always be all Real numbers.
}
\litem{
 Solve the equation for $x$ and choose the interval that contains $x$ (if it exists).
\[  16 = \ln{\sqrt[5]{\frac{20}{e^{3x}}}} \]The solution is \( x = -25.668, \text{ which does not fit in any of the interval options.} \), which is option E.\begin{enumerate}[label=\Alph*.]
\item \( x \in [-11.67, -5.67] \)

$x = -9.668$, which corresponds to treating any root as a square root.
\item \( x \in [23.67, 26.67] \)

$x = 25.668$, which is the negative of the correct solution.
\item \( x \in [-9.62, -3.62] \)

$x = -5.620$, which corresponds to thinking you need to take the natural log of the left side before reducing.
\item \( \text{There is no Real solution to the equation.} \)

This corresponds to believing you cannot solve the equation.
\item \( \text{None of the above.} \)

*$x = -25.668$ is the correct solution and does not fit in any of the other intervals.
\end{enumerate}

\textbf{General Comment:} \textbf{General Comments}: After using the properties of logarithmic functions to break up the right-hand side, use $\ln(e) = 1$ to reduce the question to a linear function to solve. You can put $\ln(20)$ into a calculator if you are having trouble.
}
\litem{
 Solve the equation for $x$ and choose the interval that contains $x$ (if it exists).
\[  10 = \sqrt[7]{\frac{24}{e^{5x}}} \]The solution is \( x = -2.588 \), which is option B.\begin{enumerate}[label=\Alph*.]
\item \( x \in [-1.29, 0.71] \)

$x = -0.285$, which corresponds to treating any root as a square root.
\item \( x \in [-4.59, -0.59] \)

* $x = -2.588$, which is the correct option.
\item \( x \in [-18.64, -9.64] \)

$x = -14.636$, which corresponds to thinking you don't need to take the natural log of both sides before reducing, as if the equation already had a natural log on the right side.
\item \( \text{There is no Real solution to the equation.} \)

This corresponds to believing you cannot solve the equation.
\item \( \text{None of the above.} \)

This corresponds to making an unexpected error.
\end{enumerate}

\textbf{General Comment:} \textbf{General Comments}: After using the properties of logarithmic functions to break up the right-hand side, use $\ln(e) = 1$ to reduce the question to a linear function to solve. You can put $\ln(24)$ into a calculator if you are having trouble.
}
\litem{
Which of the following intervals describes the Range of the function below?
\[ f(x) = -e^{x-8}+7 \]The solution is \( (-\infty, 7) \), which is option D.\begin{enumerate}[label=\Alph*.]
\item \( [a, \infty), a \in [-7, -4] \)

$[-7, \infty)$, which corresponds to using the negative vertical shift AND flipping the Range interval AND including the endpoint.
\item \( (-\infty, a], a \in [5, 8] \)

$(-\infty, 7]$, which corresponds to including the endpoint.
\item \( (a, \infty), a \in [-7, -4] \)

$(-7, \infty)$, which corresponds to using the negative vertical shift AND flipping the Range interval.
\item \( (-\infty, a), a \in [5, 8] \)

* $(-\infty, 7)$, which is the correct option.
\item \( (-\infty, \infty) \)

This corresponds to confusing range of an exponential function with the domain of an exponential function.
\end{enumerate}

\textbf{General Comment:} \textbf{General Comments}: Domain of a basic exponential function is $(-\infty, \infty)$ while the Range is $(0, \infty)$. We can shift these intervals [and even flip when $a<0$!] to find the new Domain/Range.
}
\litem{
Solve the equation for $x$ and choose the interval that contains the solution (if it exists).
\[ 4^{4x+4} = 125^{3x-4} \]The solution is \( x = 2.781 \), which is option A.\begin{enumerate}[label=\Alph*.]
\item \( x \in [2.4, 4.3] \)

* $x = 2.781$, which is the correct option.
\item \( x \in [-25, -24.8] \)

$x = -24.858$, which corresponds to distributing the $\ln(base)$ to the second term of the exponent only.
\item \( x \in [-9.6, -7.9] \)

$x = -8.000$, which corresponds to solving the numerators as equal while ignoring the bases are different.
\item \( x \in [0.8, 1] \)

$x = 0.895$, which corresponds to distributing the $\ln(base)$ to the first term of the exponent only.
\item \( \text{There is no Real solution to the equation.} \)

This corresponds to believing there is no solution since the bases are not powers of each other.
\end{enumerate}

\textbf{General Comment:} \textbf{General Comments:} This question was written so that the bases could not be written the same. You will need to take the log of both sides.
}
\litem{
Solve the equation for $x$ and choose the interval that contains the solution (if it exists).
\[ \log_{2}{(-3x+8)}+4 = 2 \]The solution is \( x = 2.583 \), which is option C.\begin{enumerate}[label=\Alph*.]
\item \( x \in [1.33, 2.33] \)

$x = 1.333$, which corresponds to reversing the base and exponent when converting.
\item \( x \in [-7, -2] \)

$x = -4.000$, which corresponds to reversing the base and exponent when converting and reversing the value with $x$.
\item \( x \in [1.58, 7.58] \)

* $x = 2.583$, which is the correct option.
\item \( x \in [1.33, 2.33] \)

$x = 1.333$, which corresponds to ignoring the vertical shift when converting to exponential form.
\item \( \text{There is no Real solution to the equation.} \)

Corresponds to believing a negative coefficient within the log equation means there is no Real solution.
\end{enumerate}

\textbf{General Comment:} \textbf{General Comments:} First, get the equation in the form $\log_b{(cx+d)} = a$. Then, convert to $b^a = cx+d$ and solve.
}
\litem{
Solve the equation for $x$ and choose the interval that contains the solution (if it exists).
\[ 2^{-4x-2} = \left(\frac{1}{9}\right)^{-2x+4} \]The solution is \( x = 1.033 \), which is option C.\begin{enumerate}[label=\Alph*.]
\item \( x \in [-5, -2] \)

$x = -3.000$, which corresponds to solving the numerators as equal while ignoring the bases are different.
\item \( x \in [1.7, 4.7] \)

$x = 3.701$, which corresponds to distributing the $\ln(base)$ to the second term of the exponent only.
\item \( x \in [1.03, 3.03] \)

* $x = 1.033$, which is the correct option.
\item \( x \in [-1.84, 0.16] \)

$x = -0.837$, which corresponds to distributing the $\ln(base)$ to the first term of the exponent only.
\item \( \text{There is no Real solution to the equation.} \)

This corresponds to believing there is no solution since the bases are not powers of each other.
\end{enumerate}

\textbf{General Comment:} \textbf{General Comments:} This question was written so that the bases could not be written the same. You will need to take the log of both sides.
}
\litem{
Which of the following intervals describes the Range of the function below?
\[ f(x) = -e^{x-3}+9 \]The solution is \( (-\infty, 9) \), which is option C.\begin{enumerate}[label=\Alph*.]
\item \( [a, \infty), a \in [-10, -7] \)

$[-9, \infty)$, which corresponds to using the negative vertical shift AND flipping the Range interval AND including the endpoint.
\item \( (-\infty, a], a \in [9, 12] \)

$(-\infty, 9]$, which corresponds to including the endpoint.
\item \( (-\infty, a), a \in [9, 12] \)

* $(-\infty, 9)$, which is the correct option.
\item \( (a, \infty), a \in [-10, -7] \)

$(-9, \infty)$, which corresponds to using the negative vertical shift AND flipping the Range interval.
\item \( (-\infty, \infty) \)

This corresponds to confusing range of an exponential function with the domain of an exponential function.
\end{enumerate}

\textbf{General Comment:} \textbf{General Comments}: Domain of a basic exponential function is $(-\infty, \infty)$ while the Range is $(0, \infty)$. We can shift these intervals [and even flip when $a<0$!] to find the new Domain/Range.
}
\end{enumerate}

\end{document}