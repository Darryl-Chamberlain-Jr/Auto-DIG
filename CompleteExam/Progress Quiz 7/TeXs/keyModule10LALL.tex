\documentclass{extbook}[14pt]
\usepackage{multicol, enumerate, enumitem, hyperref, color, soul, setspace, parskip, fancyhdr, amssymb, amsthm, amsmath, latexsym, units, mathtools}
\everymath{\displaystyle}
\usepackage[headsep=0.5cm,headheight=0cm, left=1 in,right= 1 in,top= 1 in,bottom= 1 in]{geometry}
\usepackage{dashrule}  % Package to use the command below to create lines between items
\newcommand{\litem}[1]{\item #1

\rule{\textwidth}{0.4pt}}
\pagestyle{fancy}
\lhead{}
\chead{Answer Key for Progress Quiz 7 Version ALL}
\rhead{}
\lfoot{3510-5252}
\cfoot{}
\rfoot{Summer C 2021}
\begin{document}
\textbf{This key should allow you to understand why you choose the option you did (beyond just getting a question right or wrong). \href{https://xronos.clas.ufl.edu/mac1105spring2020/courseDescriptionAndMisc/Exams/LearningFromResults}{More instructions on how to use this key can be found here}.}

\textbf{If you have a suggestion to make the keys better, \href{https://forms.gle/CZkbZmPbC9XALEE88}{please fill out the short survey here}.}

\textit{Note: This key is auto-generated and may contain issues and/or errors. The keys are reviewed after each exam to ensure grading is done accurately. If there are issues (like duplicate options), they are noted in the offline gradebook. The keys are a work-in-progress to give students as many resources to improve as possible.}

\rule{\textwidth}{0.4pt}

\begin{enumerate}\litem{
Perform the division below. Then, find the intervals that correspond to the quotient in the form $ax^2+bx+c$ and remainder $r$.
\[ \frac{10x^{3} -42 x^{2} + 37}{x -4} \]The solution is \( 10x^{2} -2 x -8 + \frac{5}{x -4} \), which is option B.\begin{enumerate}[label=\Alph*.]
\item \( a \in [7, 15], b \in [-89, -81], c \in [325, 333], \text{ and } r \in [-1276, -1272]. \)

 You divided by the opposite of the factor.
\item \( a \in [7, 15], b \in [-3, 2], c \in [-8, -4], \text{ and } r \in [4, 9]. \)

* This is the solution!
\item \( a \in [7, 15], b \in [-15, -10], c \in [-38, -35], \text{ and } r \in [-74, -67]. \)

 You multipled by the synthetic number and subtracted rather than adding during synthetic division.
\item \( a \in [38, 46], b \in [117, 128], c \in [472, 476], \text{ and } r \in [1924, 1933]. \)

 You multipled by the synthetic number rather than bringing the first factor down.
\item \( a \in [38, 46], b \in [-204, -197], c \in [808, 817], \text{ and } r \in [-3200, -3192]. \)

 You divided by the opposite of the factor AND multipled the first factor rather than just bringing it down.
\end{enumerate}

\textbf{General Comment:} Be sure to synthetically divide by the zero of the denominator! Also, make sure to include 0 placeholders for missing terms.
}
\litem{
What are the \textit{possible Integer} roots of the polynomial below?
\[ f(x) = 3x^{4} +6 x^{3} +6 x^{2} +3 x + 6 \]The solution is \( \pm 1,\pm 2,\pm 3,\pm 6 \), which is option B.\begin{enumerate}[label=\Alph*.]
\item \( \pm 1,\pm 3 \)

 Distractor 1: Corresponds to the plus or minus factors of a1 only.
\item \( \pm 1,\pm 2,\pm 3,\pm 6 \)

* This is the solution \textbf{since we asked for the possible Integer roots}!
\item \( \text{ All combinations of: }\frac{\pm 1,\pm 3}{\pm 1,\pm 2,\pm 3,\pm 6} \)

 Distractor 3: Corresponds to the plus or minus of the inverse quotient (an/a0) of the factors. 
\item \( \text{ All combinations of: }\frac{\pm 1,\pm 2,\pm 3,\pm 6}{\pm 1,\pm 3} \)

This would have been the solution \textbf{if asked for the possible Rational roots}!
\item \( \text{There is no formula or theorem that tells us all possible Integer roots.} \)

 Distractor 4: Corresponds to not recognizing Integers as a subset of Rationals.
\end{enumerate}

\textbf{General Comment:} We have a way to find the possible Rational roots. The possible Integer roots are the Integers in this list.
}
\litem{
Perform the division below. Then, find the intervals that correspond to the quotient in the form $ax^2+bx+c$ and remainder $r$.
\[ \frac{15x^{3} -33 x^{2} -96 x -50}{x -4} \]The solution is \( 15x^{2} +27 x + 12 + \frac{-2}{x -4} \), which is option A.\begin{enumerate}[label=\Alph*.]
\item \( a \in [14, 16], \text{   } b \in [24, 34], \text{   } c \in [12, 16], \text{   and   } r \in [-9, 0]. \)

* This is the solution!
\item \( a \in [57, 64], \text{   } b \in [207, 214], \text{   } c \in [728, 736], \text{   and   } r \in [2878, 2879]. \)

 You multiplied by the synthetic number rather than bringing the first factor down.
\item \( a \in [14, 16], \text{   } b \in [-97, -88], \text{   } c \in [275, 283], \text{   and   } r \in [-1160, -1151]. \)

 You divided by the opposite of the factor.
\item \( a \in [14, 16], \text{   } b \in [11, 15], \text{   } c \in [-65, -59], \text{   and   } r \in [-230, -226]. \)

 You multiplied by the synthetic number and subtracted rather than adding during synthetic division.
\item \( a \in [57, 64], \text{   } b \in [-277, -269], \text{   } c \in [996, 1002], \text{   and   } r \in [-4037, -4026]. \)

 You divided by the opposite of the factor AND multiplied the first factor rather than just bringing it down.
\end{enumerate}

\textbf{General Comment:} Be sure to synthetically divide by the zero of the denominator!
}
\litem{
Factor the polynomial below completely, knowing that $x + 5$ is a factor. Then, choose the intervals the zeros of the polynomial belong to, where $z_1 \leq z_2 \leq z_3 \leq z_4$. \textit{To make the problem easier, all zeros are between -5 and 5.}
\[ f(x) = 15x^{4} +29 x^{3} -233 x^{2} +3 x + 90 \]The solution is \( [-5, -0.6, 0.667, 3] \), which is option C.\begin{enumerate}[label=\Alph*.]
\item \( z_1 \in [-4, 1], \text{   }  z_2 \in [-1.52, -1.5], z_3 \in [1.55, 1.68], \text{   and   } z_4 \in [4, 8] \)

 Distractor 3: Corresponds to negatives of all zeros AND inversing rational roots.
\item \( z_1 \in [-7, -4], \text{   }  z_2 \in [-1.72, -1.66], z_3 \in [1.46, 1.55], \text{   and   } z_4 \in [1, 4] \)

 Distractor 2: Corresponds to inversing rational roots.
\item \( z_1 \in [-7, -4], \text{   }  z_2 \in [-0.62, -0.57], z_3 \in [0.64, 0.75], \text{   and   } z_4 \in [1, 4] \)

* This is the solution!
\item \( z_1 \in [-4, 1], \text{   }  z_2 \in [-0.15, -0.04], z_3 \in [2.95, 3.07], \text{   and   } z_4 \in [4, 8] \)

 Distractor 4: Corresponds to moving factors from one rational to another.
\item \( z_1 \in [-4, 1], \text{   }  z_2 \in [-0.67, -0.65], z_3 \in [0.55, 0.64], \text{   and   } z_4 \in [4, 8] \)

 Distractor 1: Corresponds to negatives of all zeros.
\end{enumerate}

\textbf{General Comment:} Remember to try the middle-most integers first as these normally are the zeros. Also, once you get it to a quadratic, you can use your other factoring techniques to finish factoring.
}
\litem{
Factor the polynomial below completely. Then, choose the intervals the zeros of the polynomial belong to, where $z_1 \leq z_2 \leq z_3$. \textit{To make the problem easier, all zeros are between -5 and 5.}
\[ f(x) = 10x^{3} +9 x^{2} -28 x -12 \]The solution is \( [-2, -0.4, 1.5] \), which is option C.\begin{enumerate}[label=\Alph*.]
\item \( z_1 \in [-3.33, -2.61], \text{   }  z_2 \in [-0.02, 0.32], \text{   and   } z_3 \in [1.78, 2.21] \)

 Distractor 4: Corresponds to moving factors from one rational to another.
\item \( z_1 \in [-2.82, -2.12], \text{   }  z_2 \in [-2.05, -1.75], \text{   and   } z_3 \in [0.04, 0.93] \)

 Distractor 2: Corresponds to inversing rational roots.
\item \( z_1 \in [-2.38, -1.94], \text{   }  z_2 \in [-0.71, -0.38], \text{   and   } z_3 \in [1.45, 1.66] \)

* This is the solution!
\item \( z_1 \in [-0.73, -0.52], \text{   }  z_2 \in [1.91, 2.19], \text{   and   } z_3 \in [2.48, 2.61] \)

 Distractor 3: Corresponds to negatives of all zeros AND inversing rational roots.
\item \( z_1 \in [-1.57, -1.33], \text{   }  z_2 \in [0.32, 0.59], \text{   and   } z_3 \in [1.78, 2.21] \)

 Distractor 1: Corresponds to negatives of all zeros.
\end{enumerate}

\textbf{General Comment:} Remember to try the middle-most integers first as these normally are the zeros. Also, once you get it to a quadratic, you can use your other factoring techniques to finish factoring.
}
\litem{
Factor the polynomial below completely, knowing that $x -5$ is a factor. Then, choose the intervals the zeros of the polynomial belong to, where $z_1 \leq z_2 \leq z_3 \leq z_4$. \textit{To make the problem easier, all zeros are between -5 and 5.}
\[ f(x) = 10x^{4} -99 x^{3} +308 x^{2} -333 x + 90 \]The solution is \( [0.4, 1.5, 3, 5] \), which is option E.\begin{enumerate}[label=\Alph*.]
\item \( z_1 \in [-5.14, -4.51], \text{   }  z_2 \in [-3.1, -2.9], z_3 \in [-2.91, -2.29], \text{   and   } z_4 \in [-0.78, -0.64] \)

 Distractor 3: Corresponds to negatives of all zeros AND inversing rational roots.
\item \( z_1 \in [0.57, 0.97], \text{   }  z_2 \in [1.9, 3.4], z_3 \in [2.25, 3.29], \text{   and   } z_4 \in [4.93, 5.07] \)

 Distractor 2: Corresponds to inversing rational roots.
\item \( z_1 \in [-5.14, -4.51], \text{   }  z_2 \in [-3.1, -2.9], z_3 \in [-1.97, -1.46], \text{   and   } z_4 \in [-0.44, -0.38] \)

 Distractor 1: Corresponds to negatives of all zeros.
\item \( z_1 \in [-5.14, -4.51], \text{   }  z_2 \in [-3.1, -2.9], z_3 \in [-2.01, -1.57], \text{   and   } z_4 \in [-0.3, -0.06] \)

 Distractor 4: Corresponds to moving factors from one rational to another.
\item \( z_1 \in [0.25, 0.54], \text{   }  z_2 \in [0.4, 2.2], z_3 \in [2.25, 3.29], \text{   and   } z_4 \in [4.93, 5.07] \)

* This is the solution!
\end{enumerate}

\textbf{General Comment:} Remember to try the middle-most integers first as these normally are the zeros. Also, once you get it to a quadratic, you can use your other factoring techniques to finish factoring.
}
\litem{
Perform the division below. Then, find the intervals that correspond to the quotient in the form $ax^2+bx+c$ and remainder $r$.
\[ \frac{9x^{3} +39 x^{2} -44}{x + 4} \]The solution is \( 9x^{2} +3 x -12 + \frac{4}{x + 4} \), which is option E.\begin{enumerate}[label=\Alph*.]
\item \( a \in [-39, -33], b \in [182, 186], c \in [-735, -729], \text{ and } r \in [2880, 2888]. \)

 You multipled by the synthetic number rather than bringing the first factor down.
\item \( a \in [-39, -33], b \in [-106, -102], c \in [-420, -411], \text{ and } r \in [-1728, -1723]. \)

 You divided by the opposite of the factor AND multipled the first factor rather than just bringing it down.
\item \( a \in [7, 16], b \in [-8, -5], c \in [28, 34], \text{ and } r \in [-196, -188]. \)

 You multipled by the synthetic number and subtracted rather than adding during synthetic division.
\item \( a \in [7, 16], b \in [71, 80], c \in [298, 307], \text{ and } r \in [1148, 1157]. \)

 You divided by the opposite of the factor.
\item \( a \in [7, 16], b \in [0, 11], c \in [-14, -9], \text{ and } r \in [2, 10]. \)

* This is the solution!
\end{enumerate}

\textbf{General Comment:} Be sure to synthetically divide by the zero of the denominator! Also, make sure to include 0 placeholders for missing terms.
}
\litem{
Factor the polynomial below completely. Then, choose the intervals the zeros of the polynomial belong to, where $z_1 \leq z_2 \leq z_3$. \textit{To make the problem easier, all zeros are between -5 and 5.}
\[ f(x) = 6x^{3} -19 x^{2} -9 x + 36 \]The solution is \( [-1.33, 1.5, 3] \), which is option A.\begin{enumerate}[label=\Alph*.]
\item \( z_1 \in [-1.96, -1.17], \text{   }  z_2 \in [1.18, 1.64], \text{   and   } z_3 \in [2, 3.4] \)

* This is the solution!
\item \( z_1 \in [-3.26, -2.9], \text{   }  z_2 \in [-0.79, -0.65], \text{   and   } z_3 \in [0.2, 0.8] \)

 Distractor 3: Corresponds to negatives of all zeros AND inversing rational roots.
\item \( z_1 \in [-1.13, -0.74], \text{   }  z_2 \in [0.49, 0.98], \text{   and   } z_3 \in [2, 3.4] \)

 Distractor 2: Corresponds to inversing rational roots.
\item \( z_1 \in [-3.26, -2.9], \text{   }  z_2 \in [-0.57, -0.38], \text{   and   } z_3 \in [3.7, 5] \)

 Distractor 4: Corresponds to moving factors from one rational to another.
\item \( z_1 \in [-3.26, -2.9], \text{   }  z_2 \in [-1.57, -1.18], \text{   and   } z_3 \in [1.2, 1.4] \)

 Distractor 1: Corresponds to negatives of all zeros.
\end{enumerate}

\textbf{General Comment:} Remember to try the middle-most integers first as these normally are the zeros. Also, once you get it to a quadratic, you can use your other factoring techniques to finish factoring.
}
\litem{
Perform the division below. Then, find the intervals that correspond to the quotient in the form $ax^2+bx+c$ and remainder $r$.
\[ \frac{20x^{3} +113 x^{2} +142 x + 42}{x + 4} \]The solution is \( 20x^{2} +33 x + 10 + \frac{2}{x + 4} \), which is option A.\begin{enumerate}[label=\Alph*.]
\item \( a \in [15, 21], \text{   } b \in [29, 34], \text{   } c \in [9, 12], \text{   and   } r \in [2, 3]. \)

* This is the solution!
\item \( a \in [15, 21], \text{   } b \in [191, 198], \text{   } c \in [907, 915], \text{   and   } r \in [3697, 3699]. \)

 You divided by the opposite of the factor.
\item \( a \in [-84, -78], \text{   } b \in [-207, -203], \text{   } c \in [-694, -681], \text{   and   } r \in [-2708, -2700]. \)

 You divided by the opposite of the factor AND multiplied the first factor rather than just bringing it down.
\item \( a \in [15, 21], \text{   } b \in [10, 14], \text{   } c \in [73, 85], \text{   and   } r \in [-347, -336]. \)

 You multiplied by the synthetic number and subtracted rather than adding during synthetic division.
\item \( a \in [-84, -78], \text{   } b \in [429, 434], \text{   } c \in [-1591, -1588], \text{   and   } r \in [6401, 6406]. \)

 You multiplied by the synthetic number rather than bringing the first factor down.
\end{enumerate}

\textbf{General Comment:} Be sure to synthetically divide by the zero of the denominator!
}
\litem{
What are the \textit{possible Integer} roots of the polynomial below?
\[ f(x) = 7x^{4} +2 x^{3} +4 x^{2} +3 x + 5 \]The solution is \( \pm 1,\pm 5 \), which is option A.\begin{enumerate}[label=\Alph*.]
\item \( \pm 1,\pm 5 \)

* This is the solution \textbf{since we asked for the possible Integer roots}!
\item \( \text{ All combinations of: }\frac{\pm 1,\pm 5}{\pm 1,\pm 7} \)

This would have been the solution \textbf{if asked for the possible Rational roots}!
\item \( \pm 1,\pm 7 \)

 Distractor 1: Corresponds to the plus or minus factors of a1 only.
\item \( \text{ All combinations of: }\frac{\pm 1,\pm 7}{\pm 1,\pm 5} \)

 Distractor 3: Corresponds to the plus or minus of the inverse quotient (an/a0) of the factors. 
\item \( \text{There is no formula or theorem that tells us all possible Integer roots.} \)

 Distractor 4: Corresponds to not recognizing Integers as a subset of Rationals.
\end{enumerate}

\textbf{General Comment:} We have a way to find the possible Rational roots. The possible Integer roots are the Integers in this list.
}
\litem{
Perform the division below. Then, find the intervals that correspond to the quotient in the form $ax^2+bx+c$ and remainder $r$.
\[ \frac{15x^{3} -65 x^{2} + 82}{x -4} \]The solution is \( 15x^{2} -5 x -20 + \frac{2}{x -4} \), which is option B.\begin{enumerate}[label=\Alph*.]
\item \( a \in [13, 16], b \in [-24, -15], c \in [-60, -55], \text{ and } r \in [-99, -97]. \)

 You multipled by the synthetic number and subtracted rather than adding during synthetic division.
\item \( a \in [13, 16], b \in [-11, -1], c \in [-25, -13], \text{ and } r \in [-5, 4]. \)

* This is the solution!
\item \( a \in [60, 61], b \in [175, 181], c \in [697, 708], \text{ and } r \in [2882, 2889]. \)

 You multipled by the synthetic number rather than bringing the first factor down.
\item \( a \in [13, 16], b \in [-125, -123], c \in [495, 504], \text{ and } r \in [-1919, -1912]. \)

 You divided by the opposite of the factor.
\item \( a \in [60, 61], b \in [-309, -304], c \in [1220, 1223], \text{ and } r \in [-4803, -4794]. \)

 You divided by the opposite of the factor AND multipled the first factor rather than just bringing it down.
\end{enumerate}

\textbf{General Comment:} Be sure to synthetically divide by the zero of the denominator! Also, make sure to include 0 placeholders for missing terms.
}
\litem{
What are the \textit{possible Rational} roots of the polynomial below?
\[ f(x) = 6x^{2} +5 x + 2 \]The solution is \( \text{ All combinations of: }\frac{\pm 1,\pm 2}{\pm 1,\pm 2,\pm 3,\pm 6} \), which is option B.\begin{enumerate}[label=\Alph*.]
\item \( \pm 1,\pm 2 \)

This would have been the solution \textbf{if asked for the possible Integer roots}!
\item \( \text{ All combinations of: }\frac{\pm 1,\pm 2}{\pm 1,\pm 2,\pm 3,\pm 6} \)

* This is the solution \textbf{since we asked for the possible Rational roots}!
\item \( \text{ All combinations of: }\frac{\pm 1,\pm 2,\pm 3,\pm 6}{\pm 1,\pm 2} \)

 Distractor 3: Corresponds to the plus or minus of the inverse quotient (an/a0) of the factors. 
\item \( \pm 1,\pm 2,\pm 3,\pm 6 \)

 Distractor 1: Corresponds to the plus or minus factors of a1 only.
\item \( \text{ There is no formula or theorem that tells us all possible Rational roots.} \)

 Distractor 4: Corresponds to not recalling the theorem for rational roots of a polynomial.
\end{enumerate}

\textbf{General Comment:} We have a way to find the possible Rational roots. The possible Integer roots are the Integers in this list.
}
\litem{
Perform the division below. Then, find the intervals that correspond to the quotient in the form $ax^2+bx+c$ and remainder $r$.
\[ \frac{10x^{3} -38 x^{2} -16 x + 34}{x -4} \]The solution is \( 10x^{2} +2 x -8 + \frac{2}{x -4} \), which is option C.\begin{enumerate}[label=\Alph*.]
\item \( a \in [37, 41], \text{   } b \in [119, 126], \text{   } c \in [468, 475], \text{   and   } r \in [1922, 1924]. \)

 You multiplied by the synthetic number rather than bringing the first factor down.
\item \( a \in [5, 14], \text{   } b \in [-78, -74], \text{   } c \in [296, 303], \text{   and   } r \in [-1152, -1147]. \)

 You divided by the opposite of the factor.
\item \( a \in [5, 14], \text{   } b \in [-3, 4], \text{   } c \in [-11, -3], \text{   and   } r \in [-1, 3]. \)

* This is the solution!
\item \( a \in [37, 41], \text{   } b \in [-201, -193], \text{   } c \in [776, 778], \text{   and   } r \in [-3074, -3063]. \)

 You divided by the opposite of the factor AND multiplied the first factor rather than just bringing it down.
\item \( a \in [5, 14], \text{   } b \in [-10, -2], \text{   } c \in [-42, -39], \text{   and   } r \in [-86, -82]. \)

 You multiplied by the synthetic number and subtracted rather than adding during synthetic division.
\end{enumerate}

\textbf{General Comment:} Be sure to synthetically divide by the zero of the denominator!
}
\litem{
Factor the polynomial below completely, knowing that $x + 4$ is a factor. Then, choose the intervals the zeros of the polynomial belong to, where $z_1 \leq z_2 \leq z_3 \leq z_4$. \textit{To make the problem easier, all zeros are between -5 and 5.}
\[ f(x) = 20x^{4} +13 x^{3} -253 x^{2} +78 x + 72 \]The solution is \( [-4, -0.4, 0.75, 3] \), which is option E.\begin{enumerate}[label=\Alph*.]
\item \( z_1 \in [-3.3, -2.6], \text{   }  z_2 \in [-1.16, -0.5], z_3 \in [0.23, 0.44], \text{   and   } z_4 \in [3.1, 4.6] \)

 Distractor 1: Corresponds to negatives of all zeros.
\item \( z_1 \in [-3.3, -2.6], \text{   }  z_2 \in [-1.59, -1.31], z_3 \in [2.3, 2.69], \text{   and   } z_4 \in [3.1, 4.6] \)

 Distractor 3: Corresponds to negatives of all zeros AND inversing rational roots.
\item \( z_1 \in [-4.7, -3.5], \text{   }  z_2 \in [-2.67, -2.31], z_3 \in [1.2, 1.91], \text{   and   } z_4 \in [1.5, 3.2] \)

 Distractor 2: Corresponds to inversing rational roots.
\item \( z_1 \in [-3.3, -2.6], \text{   }  z_2 \in [-3.23, -2.61], z_3 \in [-0.05, 0.12], \text{   and   } z_4 \in [3.1, 4.6] \)

 Distractor 4: Corresponds to moving factors from one rational to another.
\item \( z_1 \in [-4.7, -3.5], \text{   }  z_2 \in [-0.5, 0.04], z_3 \in [0.72, 0.88], \text{   and   } z_4 \in [1.5, 3.2] \)

* This is the solution!
\end{enumerate}

\textbf{General Comment:} Remember to try the middle-most integers first as these normally are the zeros. Also, once you get it to a quadratic, you can use your other factoring techniques to finish factoring.
}
\litem{
Factor the polynomial below completely. Then, choose the intervals the zeros of the polynomial belong to, where $z_1 \leq z_2 \leq z_3$. \textit{To make the problem easier, all zeros are between -5 and 5.}
\[ f(x) = 6x^{3} -1 x^{2} -39 x -36 \]The solution is \( [-1.5, -1.33, 3] \), which is option E.\begin{enumerate}[label=\Alph*.]
\item \( z_1 \in [-0.79, -0.48], \text{   }  z_2 \in [-0.68, -0.58], \text{   and   } z_3 \in [2.6, 3.4] \)

 Distractor 2: Corresponds to inversing rational roots.
\item \( z_1 \in [-3.4, -2.82], \text{   }  z_2 \in [1.28, 1.47], \text{   and   } z_3 \in [1, 1.6] \)

 Distractor 1: Corresponds to negatives of all zeros.
\item \( z_1 \in [-3.4, -2.82], \text{   }  z_2 \in [0.56, 0.82], \text{   and   } z_3 \in [-0.2, 1.1] \)

 Distractor 3: Corresponds to negatives of all zeros AND inversing rational roots.
\item \( z_1 \in [-3.4, -2.82], \text{   }  z_2 \in [0.36, 0.66], \text{   and   } z_3 \in [3.4, 5.4] \)

 Distractor 4: Corresponds to moving factors from one rational to another.
\item \( z_1 \in [-2.03, -1.3], \text{   }  z_2 \in [-1.4, -1.18], \text{   and   } z_3 \in [2.6, 3.4] \)

* This is the solution!
\end{enumerate}

\textbf{General Comment:} Remember to try the middle-most integers first as these normally are the zeros. Also, once you get it to a quadratic, you can use your other factoring techniques to finish factoring.
}
\litem{
Factor the polynomial below completely, knowing that $x -4$ is a factor. Then, choose the intervals the zeros of the polynomial belong to, where $z_1 \leq z_2 \leq z_3 \leq z_4$. \textit{To make the problem easier, all zeros are between -5 and 5.}
\[ f(x) = 8x^{4} -6 x^{3} -189 x^{2} +265 x + 300 \]The solution is \( [-5, -0.75, 2.5, 4] \), which is option A.\begin{enumerate}[label=\Alph*.]
\item \( z_1 \in [-5.9, -4.4], \text{   }  z_2 \in [-0.82, -0.46], z_3 \in [2.49, 2.51], \text{   and   } z_4 \in [2.7, 4.9] \)

* This is the solution!
\item \( z_1 \in [-4.7, -2.8], \text{   }  z_2 \in [-0.5, -0.38], z_3 \in [1.33, 1.35], \text{   and   } z_4 \in [4.7, 5.3] \)

 Distractor 3: Corresponds to negatives of all zeros AND inversing rational roots.
\item \( z_1 \in [-5.9, -4.4], \text{   }  z_2 \in [-4.11, -3.8], z_3 \in [0.35, 0.38], \text{   and   } z_4 \in [4.7, 5.3] \)

 Distractor 4: Corresponds to moving factors from one rational to another.
\item \( z_1 \in [-4.7, -2.8], \text{   }  z_2 \in [-2.96, -2.39], z_3 \in [0.74, 0.76], \text{   and   } z_4 \in [4.7, 5.3] \)

 Distractor 1: Corresponds to negatives of all zeros.
\item \( z_1 \in [-5.9, -4.4], \text{   }  z_2 \in [-1.42, -1.05], z_3 \in [0.39, 0.41], \text{   and   } z_4 \in [2.7, 4.9] \)

 Distractor 2: Corresponds to inversing rational roots.
\end{enumerate}

\textbf{General Comment:} Remember to try the middle-most integers first as these normally are the zeros. Also, once you get it to a quadratic, you can use your other factoring techniques to finish factoring.
}
\litem{
Perform the division below. Then, find the intervals that correspond to the quotient in the form $ax^2+bx+c$ and remainder $r$.
\[ \frac{10x^{3} -70 x + 65}{x + 3} \]The solution is \( 10x^{2} -30 x + 20 + \frac{5}{x + 3} \), which is option E.\begin{enumerate}[label=\Alph*.]
\item \( a \in [7, 12], b \in [30, 33], c \in [20, 26], \text{ and } r \in [124, 130]. \)

 You divided by the opposite of the factor.
\item \( a \in [-38, -25], b \in [90, 93], c \in [-344, -335], \text{ and } r \in [1078, 1091]. \)

 You multipled by the synthetic number rather than bringing the first factor down.
\item \( a \in [-38, -25], b \in [-91, -85], c \in [-344, -335], \text{ and } r \in [-958, -953]. \)

 You divided by the opposite of the factor AND multipled the first factor rather than just bringing it down.
\item \( a \in [7, 12], b \in [-40, -39], c \in [89, 91], \text{ and } r \in [-298, -294]. \)

 You multipled by the synthetic number and subtracted rather than adding during synthetic division.
\item \( a \in [7, 12], b \in [-35, -29], c \in [20, 26], \text{ and } r \in [2, 13]. \)

* This is the solution!
\end{enumerate}

\textbf{General Comment:} Be sure to synthetically divide by the zero of the denominator! Also, make sure to include 0 placeholders for missing terms.
}
\litem{
Factor the polynomial below completely. Then, choose the intervals the zeros of the polynomial belong to, where $z_1 \leq z_2 \leq z_3$. \textit{To make the problem easier, all zeros are between -5 and 5.}
\[ f(x) = 20x^{3} -33 x^{2} -20 x + 12 \]The solution is \( [-0.75, 0.4, 2] \), which is option B.\begin{enumerate}[label=\Alph*.]
\item \( z_1 \in [-2.02, -1.65], \text{   }  z_2 \in [-2.77, -1.28], \text{   and   } z_3 \in [0.09, 0.38] \)

 Distractor 4: Corresponds to moving factors from one rational to another.
\item \( z_1 \in [-1.2, -0.31], \text{   }  z_2 \in [0.22, 0.44], \text{   and   } z_3 \in [1.92, 2.22] \)

* This is the solution!
\item \( z_1 \in [-1.63, -1.11], \text{   }  z_2 \in [1.83, 2.91], \text{   and   } z_3 \in [2.28, 2.58] \)

 Distractor 2: Corresponds to inversing rational roots.
\item \( z_1 \in [-2.55, -2.31], \text{   }  z_2 \in [-2.77, -1.28], \text{   and   } z_3 \in [1.1, 1.38] \)

 Distractor 3: Corresponds to negatives of all zeros AND inversing rational roots.
\item \( z_1 \in [-2.02, -1.65], \text{   }  z_2 \in [-0.52, -0.21], \text{   and   } z_3 \in [0.69, 0.98] \)

 Distractor 1: Corresponds to negatives of all zeros.
\end{enumerate}

\textbf{General Comment:} Remember to try the middle-most integers first as these normally are the zeros. Also, once you get it to a quadratic, you can use your other factoring techniques to finish factoring.
}
\litem{
Perform the division below. Then, find the intervals that correspond to the quotient in the form $ax^2+bx+c$ and remainder $r$.
\[ \frac{15x^{3} +67 x^{2} +94 x + 35}{x + 2} \]The solution is \( 15x^{2} +37 x + 20 + \frac{-5}{x + 2} \), which is option A.\begin{enumerate}[label=\Alph*.]
\item \( a \in [13, 18], \text{   } b \in [37, 39], \text{   } c \in [16, 24], \text{   and   } r \in [-11, -3]. \)

* This is the solution!
\item \( a \in [-31, -28], \text{   } b \in [6, 13], \text{   } c \in [104, 113], \text{   and   } r \in [251, 257]. \)

 You divided by the opposite of the factor AND multiplied the first factor rather than just bringing it down.
\item \( a \in [-31, -28], \text{   } b \in [125, 129], \text{   } c \in [-161, -159], \text{   and   } r \in [354, 357]. \)

 You multiplied by the synthetic number rather than bringing the first factor down.
\item \( a \in [13, 18], \text{   } b \in [92, 101], \text{   } c \in [284, 289], \text{   and   } r \in [606, 615]. \)

 You divided by the opposite of the factor.
\item \( a \in [13, 18], \text{   } b \in [20, 23], \text{   } c \in [24, 34], \text{   and   } r \in [-50, -46]. \)

 You multiplied by the synthetic number and subtracted rather than adding during synthetic division.
\end{enumerate}

\textbf{General Comment:} Be sure to synthetically divide by the zero of the denominator!
}
\litem{
What are the \textit{possible Integer} roots of the polynomial below?
\[ f(x) = 6x^{4} +4 x^{3} +7 x^{2} +4 x + 7 \]The solution is \( \pm 1,\pm 7 \), which is option B.\begin{enumerate}[label=\Alph*.]
\item \( \text{ All combinations of: }\frac{\pm 1,\pm 2,\pm 3,\pm 6}{\pm 1,\pm 7} \)

 Distractor 3: Corresponds to the plus or minus of the inverse quotient (an/a0) of the factors. 
\item \( \pm 1,\pm 7 \)

* This is the solution \textbf{since we asked for the possible Integer roots}!
\item \( \pm 1,\pm 2,\pm 3,\pm 6 \)

 Distractor 1: Corresponds to the plus or minus factors of a1 only.
\item \( \text{ All combinations of: }\frac{\pm 1,\pm 7}{\pm 1,\pm 2,\pm 3,\pm 6} \)

This would have been the solution \textbf{if asked for the possible Rational roots}!
\item \( \text{There is no formula or theorem that tells us all possible Integer roots.} \)

 Distractor 4: Corresponds to not recognizing Integers as a subset of Rationals.
\end{enumerate}

\textbf{General Comment:} We have a way to find the possible Rational roots. The possible Integer roots are the Integers in this list.
}
\litem{
Perform the division below. Then, find the intervals that correspond to the quotient in the form $ax^2+bx+c$ and remainder $r$.
\[ \frac{9x^{3} +21 x^{2} -7}{x + 2} \]The solution is \( 9x^{2} +3 x -6 + \frac{5}{x + 2} \), which is option D.\begin{enumerate}[label=\Alph*.]
\item \( a \in [8, 17], b \in [36, 48], c \in [77, 84], \text{ and } r \in [149, 151]. \)

 You divided by the opposite of the factor.
\item \( a \in [8, 17], b \in [-8, -2], c \in [13, 21], \text{ and } r \in [-65, -60]. \)

 You multipled by the synthetic number and subtracted rather than adding during synthetic division.
\item \( a \in [-18, -14], b \in [54, 60], c \in [-114, -113], \text{ and } r \in [219, 224]. \)

 You multipled by the synthetic number rather than bringing the first factor down.
\item \( a \in [8, 17], b \in [3, 8], c \in [-11, -3], \text{ and } r \in [4, 14]. \)

* This is the solution!
\item \( a \in [-18, -14], b \in [-17, -10], c \in [-32, -25], \text{ and } r \in [-69, -66]. \)

 You divided by the opposite of the factor AND multipled the first factor rather than just bringing it down.
\end{enumerate}

\textbf{General Comment:} Be sure to synthetically divide by the zero of the denominator! Also, make sure to include 0 placeholders for missing terms.
}
\litem{
What are the \textit{possible Rational} roots of the polynomial below?
\[ f(x) = 4x^{4} +6 x^{3} +3 x^{2} +7 x + 2 \]The solution is \( \text{ All combinations of: }\frac{\pm 1,\pm 2}{\pm 1,\pm 2,\pm 4} \), which is option B.\begin{enumerate}[label=\Alph*.]
\item \( \text{ All combinations of: }\frac{\pm 1,\pm 2,\pm 4}{\pm 1,\pm 2} \)

 Distractor 3: Corresponds to the plus or minus of the inverse quotient (an/a0) of the factors. 
\item \( \text{ All combinations of: }\frac{\pm 1,\pm 2}{\pm 1,\pm 2,\pm 4} \)

* This is the solution \textbf{since we asked for the possible Rational roots}!
\item \( \pm 1,\pm 2,\pm 4 \)

 Distractor 1: Corresponds to the plus or minus factors of a1 only.
\item \( \pm 1,\pm 2 \)

This would have been the solution \textbf{if asked for the possible Integer roots}!
\item \( \text{ There is no formula or theorem that tells us all possible Rational roots.} \)

 Distractor 4: Corresponds to not recalling the theorem for rational roots of a polynomial.
\end{enumerate}

\textbf{General Comment:} We have a way to find the possible Rational roots. The possible Integer roots are the Integers in this list.
}
\litem{
Perform the division below. Then, find the intervals that correspond to the quotient in the form $ax^2+bx+c$ and remainder $r$.
\[ \frac{20x^{3} -48 x^{2} -116 x -43}{x -4} \]The solution is \( 20x^{2} +32 x + 12 + \frac{5}{x -4} \), which is option C.\begin{enumerate}[label=\Alph*.]
\item \( a \in [19, 24], \text{   } b \in [12, 13], \text{   } c \in [-85, -79], \text{   and   } r \in [-284, -280]. \)

 You multiplied by the synthetic number and subtracted rather than adding during synthetic division.
\item \( a \in [19, 24], \text{   } b \in [-131, -125], \text{   } c \in [391, 398], \text{   and   } r \in [-1632, -1622]. \)

 You divided by the opposite of the factor.
\item \( a \in [19, 24], \text{   } b \in [31, 38], \text{   } c \in [8, 19], \text{   and   } r \in [2, 9]. \)

* This is the solution!
\item \( a \in [80, 86], \text{   } b \in [-370, -364], \text{   } c \in [1356, 1360], \text{   and   } r \in [-5472, -5466]. \)

 You divided by the opposite of the factor AND multiplied the first factor rather than just bringing it down.
\item \( a \in [80, 86], \text{   } b \in [269, 275], \text{   } c \in [969, 975], \text{   and   } r \in [3842, 3846]. \)

 You multiplied by the synthetic number rather than bringing the first factor down.
\end{enumerate}

\textbf{General Comment:} Be sure to synthetically divide by the zero of the denominator!
}
\litem{
Factor the polynomial below completely, knowing that $x + 5$ is a factor. Then, choose the intervals the zeros of the polynomial belong to, where $z_1 \leq z_2 \leq z_3 \leq z_4$. \textit{To make the problem easier, all zeros are between -5 and 5.}
\[ f(x) = 25x^{4} +210 x^{3} +507 x^{2} +434 x + 120 \]The solution is \( [-5, -2, -0.8, -0.6] \), which is option D.\begin{enumerate}[label=\Alph*.]
\item \( z_1 \in [1.24, 1.46], \text{   }  z_2 \in [1.6, 1.94], z_3 \in [1.3, 2.2], \text{   and   } z_4 \in [4.71, 5.09] \)

 Distractor 3: Corresponds to negatives of all zeros AND inversing rational roots.
\item \( z_1 \in [0.1, 0.33], \text{   }  z_2 \in [1.89, 2.8], z_3 \in [2.8, 4.5], \text{   and   } z_4 \in [4.71, 5.09] \)

 Distractor 4: Corresponds to moving factors from one rational to another.
\item \( z_1 \in [-5.19, -4.79], \text{   }  z_2 \in [-2.35, -1.49], z_3 \in [-2, -1], \text{   and   } z_4 \in [-1.56, -1.17] \)

 Distractor 2: Corresponds to inversing rational roots.
\item \( z_1 \in [-5.19, -4.79], \text{   }  z_2 \in [-2.35, -1.49], z_3 \in [-1.1, 1.6], \text{   and   } z_4 \in [-0.93, 0.31] \)

* This is the solution!
\item \( z_1 \in [0.5, 0.72], \text{   }  z_2 \in [-0.1, 0.82], z_3 \in [1.3, 2.2], \text{   and   } z_4 \in [4.71, 5.09] \)

 Distractor 1: Corresponds to negatives of all zeros.
\end{enumerate}

\textbf{General Comment:} Remember to try the middle-most integers first as these normally are the zeros. Also, once you get it to a quadratic, you can use your other factoring techniques to finish factoring.
}
\litem{
Factor the polynomial below completely. Then, choose the intervals the zeros of the polynomial belong to, where $z_1 \leq z_2 \leq z_3$. \textit{To make the problem easier, all zeros are between -5 and 5.}
\[ f(x) = 6x^{3} +5 x^{2} -22 x -24 \]The solution is \( [-1.5, -1.33, 2] \), which is option A.\begin{enumerate}[label=\Alph*.]
\item \( z_1 \in [-1.67, -1.39], \text{   }  z_2 \in [-1.42, -1.18], \text{   and   } z_3 \in [1.7, 2.6] \)

* This is the solution!
\item \( z_1 \in [-2.13, -1.96], \text{   }  z_2 \in [0.46, 0.55], \text{   and   } z_3 \in [3.7, 4.4] \)

 Distractor 4: Corresponds to moving factors from one rational to another.
\item \( z_1 \in [-2.13, -1.96], \text{   }  z_2 \in [0.62, 0.81], \text{   and   } z_3 \in [-0.5, 1.2] \)

 Distractor 3: Corresponds to negatives of all zeros AND inversing rational roots.
\item \( z_1 \in [-2.13, -1.96], \text{   }  z_2 \in [1.22, 1.4], \text{   and   } z_3 \in [1, 1.9] \)

 Distractor 1: Corresponds to negatives of all zeros.
\item \( z_1 \in [-1.04, -0.67], \text{   }  z_2 \in [-0.83, -0.6], \text{   and   } z_3 \in [1.7, 2.6] \)

 Distractor 2: Corresponds to inversing rational roots.
\end{enumerate}

\textbf{General Comment:} Remember to try the middle-most integers first as these normally are the zeros. Also, once you get it to a quadratic, you can use your other factoring techniques to finish factoring.
}
\litem{
Factor the polynomial below completely, knowing that $x -4$ is a factor. Then, choose the intervals the zeros of the polynomial belong to, where $z_1 \leq z_2 \leq z_3 \leq z_4$. \textit{To make the problem easier, all zeros are between -5 and 5.}
\[ f(x) = 12x^{4} -53 x^{3} -23 x^{2} +202 x -120 \]The solution is \( [-2, 0.75, 1.667, 4] \), which is option A.\begin{enumerate}[label=\Alph*.]
\item \( z_1 \in [-3.4, -1.4], \text{   }  z_2 \in [0.68, 0.95], z_3 \in [1.54, 1.71], \text{   and   } z_4 \in [4, 6] \)

* This is the solution!
\item \( z_1 \in [-3.4, -1.4], \text{   }  z_2 \in [0.52, 0.7], z_3 \in [1.2, 1.38], \text{   and   } z_4 \in [4, 6] \)

 Distractor 2: Corresponds to inversing rational roots.
\item \( z_1 \in [-5.6, -4.6], \text{   }  z_2 \in [-4.05, -3.87], z_3 \in [-0.43, -0.2], \text{   and   } z_4 \in [0, 3] \)

 Distractor 4: Corresponds to moving factors from one rational to another.
\item \( z_1 \in [-4.7, -3.1], \text{   }  z_2 \in [-1.44, -1.16], z_3 \in [-0.71, -0.32], \text{   and   } z_4 \in [0, 3] \)

 Distractor 3: Corresponds to negatives of all zeros AND inversing rational roots.
\item \( z_1 \in [-4.7, -3.1], \text{   }  z_2 \in [-1.75, -1.65], z_3 \in [-0.85, -0.62], \text{   and   } z_4 \in [0, 3] \)

 Distractor 1: Corresponds to negatives of all zeros.
\end{enumerate}

\textbf{General Comment:} Remember to try the middle-most integers first as these normally are the zeros. Also, once you get it to a quadratic, you can use your other factoring techniques to finish factoring.
}
\litem{
Perform the division below. Then, find the intervals that correspond to the quotient in the form $ax^2+bx+c$ and remainder $r$.
\[ \frac{8x^{3} +28 x^{2} -33}{x + 3} \]The solution is \( 8x^{2} +4 x -12 + \frac{3}{x + 3} \), which is option A.\begin{enumerate}[label=\Alph*.]
\item \( a \in [5, 12], b \in [4, 6], c \in [-13, -3], \text{ and } r \in [0, 8]. \)

* This is the solution!
\item \( a \in [5, 12], b \in [52, 57], c \in [156, 158], \text{ and } r \in [435, 437]. \)

 You divided by the opposite of the factor.
\item \( a \in [5, 12], b \in [-6, 1], c \in [13, 19], \text{ and } r \in [-104, -92]. \)

 You multipled by the synthetic number and subtracted rather than adding during synthetic division.
\item \( a \in [-24, -23], b \in [97, 102], c \in [-300, -290], \text{ and } r \in [864, 875]. \)

 You multipled by the synthetic number rather than bringing the first factor down.
\item \( a \in [-24, -23], b \in [-48, -40], c \in [-135, -124], \text{ and } r \in [-432, -427]. \)

 You divided by the opposite of the factor AND multipled the first factor rather than just bringing it down.
\end{enumerate}

\textbf{General Comment:} Be sure to synthetically divide by the zero of the denominator! Also, make sure to include 0 placeholders for missing terms.
}
\litem{
Factor the polynomial below completely. Then, choose the intervals the zeros of the polynomial belong to, where $z_1 \leq z_2 \leq z_3$. \textit{To make the problem easier, all zeros are between -5 and 5.}
\[ f(x) = 10x^{3} -41 x^{2} -54 x + 45 \]The solution is \( [-1.5, 0.6, 5] \), which is option E.\begin{enumerate}[label=\Alph*.]
\item \( z_1 \in [-6, -4.8], \text{   }  z_2 \in [-0.8, -0.3], \text{   and   } z_3 \in [1, 1.6] \)

 Distractor 1: Corresponds to negatives of all zeros.
\item \( z_1 \in [-6, -4.8], \text{   }  z_2 \in [-3.3, -2.7], \text{   and   } z_3 \in [-0.7, 0.6] \)

 Distractor 4: Corresponds to moving factors from one rational to another.
\item \( z_1 \in [-6, -4.8], \text{   }  z_2 \in [-2.9, -1.5], \text{   and   } z_3 \in [0.6, 0.9] \)

 Distractor 3: Corresponds to negatives of all zeros AND inversing rational roots.
\item \( z_1 \in [-1, -0.1], \text{   }  z_2 \in [0.8, 2.1], \text{   and   } z_3 \in [4.4, 5.9] \)

 Distractor 2: Corresponds to inversing rational roots.
\item \( z_1 \in [-1.9, -1.1], \text{   }  z_2 \in [-0.1, 1.2], \text{   and   } z_3 \in [4.4, 5.9] \)

* This is the solution!
\end{enumerate}

\textbf{General Comment:} Remember to try the middle-most integers first as these normally are the zeros. Also, once you get it to a quadratic, you can use your other factoring techniques to finish factoring.
}
\litem{
Perform the division below. Then, find the intervals that correspond to the quotient in the form $ax^2+bx+c$ and remainder $r$.
\[ \frac{12x^{3} +45 x^{2} -21 x -39}{x + 4} \]The solution is \( 12x^{2} -3 x -9 + \frac{-3}{x + 4} \), which is option E.\begin{enumerate}[label=\Alph*.]
\item \( a \in [8, 13], \text{   } b \in [90, 102], \text{   } c \in [342, 359], \text{   and   } r \in [1363, 1367]. \)

 You divided by the opposite of the factor.
\item \( a \in [-49, -44], \text{   } b \in [-152, -144], \text{   } c \in [-611, -607], \text{   and   } r \in [-2475, -2471]. \)

 You divided by the opposite of the factor AND multiplied the first factor rather than just bringing it down.
\item \( a \in [-49, -44], \text{   } b \in [232, 242], \text{   } c \in [-970, -961], \text{   and   } r \in [3834, 3839]. \)

 You multiplied by the synthetic number rather than bringing the first factor down.
\item \( a \in [8, 13], \text{   } b \in [-16, -10], \text{   } c \in [53, 55], \text{   and   } r \in [-310, -305]. \)

 You multiplied by the synthetic number and subtracted rather than adding during synthetic division.
\item \( a \in [8, 13], \text{   } b \in [-3, 4], \text{   } c \in [-19, -8], \text{   and   } r \in [-5, 4]. \)

* This is the solution!
\end{enumerate}

\textbf{General Comment:} Be sure to synthetically divide by the zero of the denominator!
}
\litem{
What are the \textit{possible Rational} roots of the polynomial below?
\[ f(x) = 3x^{2} +6 x + 5 \]The solution is \( \text{ All combinations of: }\frac{\pm 1,\pm 5}{\pm 1,\pm 3} \), which is option B.\begin{enumerate}[label=\Alph*.]
\item \( \text{ All combinations of: }\frac{\pm 1,\pm 3}{\pm 1,\pm 5} \)

 Distractor 3: Corresponds to the plus or minus of the inverse quotient (an/a0) of the factors. 
\item \( \text{ All combinations of: }\frac{\pm 1,\pm 5}{\pm 1,\pm 3} \)

* This is the solution \textbf{since we asked for the possible Rational roots}!
\item \( \pm 1,\pm 5 \)

This would have been the solution \textbf{if asked for the possible Integer roots}!
\item \( \pm 1,\pm 3 \)

 Distractor 1: Corresponds to the plus or minus factors of a1 only.
\item \( \text{ There is no formula or theorem that tells us all possible Rational roots.} \)

 Distractor 4: Corresponds to not recalling the theorem for rational roots of a polynomial.
\end{enumerate}

\textbf{General Comment:} We have a way to find the possible Rational roots. The possible Integer roots are the Integers in this list.
}
\end{enumerate}

\end{document}