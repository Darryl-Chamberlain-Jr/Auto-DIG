\documentclass{extbook}[14pt]
\usepackage{multicol, enumerate, enumitem, hyperref, color, soul, setspace, parskip, fancyhdr, amssymb, amsthm, amsmath, latexsym, units, mathtools}
\everymath{\displaystyle}
\usepackage[headsep=0.5cm,headheight=0cm, left=1 in,right= 1 in,top= 1 in,bottom= 1 in]{geometry}
\usepackage{dashrule}  % Package to use the command below to create lines between items
\newcommand{\litem}[1]{\item #1

\rule{\textwidth}{0.4pt}}
\pagestyle{fancy}
\lhead{}
\chead{Answer Key for Progress Quiz 7 Version A}
\rhead{}
\lfoot{3510-5252}
\cfoot{}
\rfoot{Summer C 2021}
\begin{document}
\textbf{This key should allow you to understand why you choose the option you did (beyond just getting a question right or wrong). \href{https://xronos.clas.ufl.edu/mac1105spring2020/courseDescriptionAndMisc/Exams/LearningFromResults}{More instructions on how to use this key can be found here}.}

\textbf{If you have a suggestion to make the keys better, \href{https://forms.gle/CZkbZmPbC9XALEE88}{please fill out the short survey here}.}

\textit{Note: This key is auto-generated and may contain issues and/or errors. The keys are reviewed after each exam to ensure grading is done accurately. If there are issues (like duplicate options), they are noted in the offline gradebook. The keys are a work-in-progress to give students as many resources to improve as possible.}

\rule{\textwidth}{0.4pt}

\begin{enumerate}\litem{
Simplify the expression below and choose the interval the simplification is contained within.
\[ 19 - 18^2 + 6 \div 9 * 11 \div 3 \]The solution is \( -302.556 \), which is option C.\begin{enumerate}[label=\Alph*.]
\item \( [-306.98, -303.98] \)

 -304.980, which corresponds to an Order of Operations error: not reading left-to-right for multiplication/division.
\item \( [343.44, 350.44] \)

 345.444, which corresponds to an Order of Operations error: multiplying by negative before squaring. For example: $(-3)^2 \neq -3^2$
\item \( [-303.56, -299.56] \)

* -302.556, this is the correct option
\item \( [338.02, 345.02] \)

 343.020, which corresponds to two Order of Operations errors.
\item \( \text{None of the above} \)

 You may have gotten this by making an unanticipated error. If you got a value that is not any of the others, please let the coordinator know so they can help you figure out what happened.
\end{enumerate}

\textbf{General Comment:} While you may remember (or were taught) PEMDAS is done in order, it is actually done as P/E/MD/AS. When we are at MD or AS, we read left to right.
}
\litem{
Choose the \textbf{smallest} set of Complex numbers that the number below belongs to.
\[ \frac{0}{11 \pi}+\sqrt{9}i \]The solution is \( \text{Pure Imaginary} \), which is option E.\begin{enumerate}[label=\Alph*.]
\item \( \text{Rational} \)

These are numbers that can be written as fraction of Integers (e.g., -2/3 + 5)
\item \( \text{Nonreal Complex} \)

This is a Complex number $(a+bi)$ that is not Real (has $i$ as part of the number).
\item \( \text{Irrational} \)

These cannot be written as a fraction of Integers. Remember: $\pi$ is not an Integer!
\item \( \text{Not a Complex Number} \)

This is not a number. The only non-Complex number we know is dividing by 0 as this is not a number!
\item \( \text{Pure Imaginary} \)

* This is the correct option!
\end{enumerate}

\textbf{General Comment:} Be sure to simplify $i^2 = -1$. This may remove the imaginary portion for your number. If you are having trouble, you may want to look at the \textit{Subgroups of the Real Numbers} section.
}
\litem{
Simplify the expression below into the form $a+bi$. Then, choose the intervals that $a$ and $b$ belong to.
\[ \frac{-54 + 88 i}{2 + 4 i} \]The solution is \( 12.20  + 19.60 i \), which is option E.\begin{enumerate}[label=\Alph*.]
\item \( a \in [243, 246] \text{ and } b \in [19, 20] \)

 $244.00  + 19.60 i$, which corresponds to forgetting to multiply the conjugate by the numerator and using a plus instead of a minus in the denominator.
\item \( a \in [-24, -22.5] \text{ and } b \in [-2.5, -1.5] \)

 $-23.00  - 2.00 i$, which corresponds to forgetting to multiply the conjugate by the numerator and not computing the conjugate correctly.
\item \( a \in [-28, -26.5] \text{ and } b \in [21, 23] \)

 $-27.00  + 22.00 i$, which corresponds to just dividing the first term by the first term and the second by the second.
\item \( a \in [10.5, 12.5] \text{ and } b \in [390.5, 393] \)

 $12.20  + 392.00 i$, which corresponds to forgetting to multiply the conjugate by the numerator.
\item \( a \in [10.5, 12.5] \text{ and } b \in [19, 20] \)

* $12.20  + 19.60 i$, which is the correct option.
\end{enumerate}

\textbf{General Comment:} Multiply the numerator and denominator by the *conjugate* of the denominator, then simplify. For example, if we have $2+3i$, the conjugate is $2-3i$.
}
\litem{
Choose the \textbf{smallest} set of Real numbers that the number below belongs to.
\[ -\sqrt{\frac{4225}{25}} \]The solution is \( \text{Integer} \), which is option D.\begin{enumerate}[label=\Alph*.]
\item \( \text{Whole} \)

These are the counting numbers with 0 (0, 1, 2, 3, ...)
\item \( \text{Not a Real number} \)

These are Nonreal Complex numbers \textbf{OR} things that are not numbers (e.g., dividing by 0).
\item \( \text{Rational} \)

These are numbers that can be written as fraction of Integers (e.g., -2/3)
\item \( \text{Integer} \)

* This is the correct option!
\item \( \text{Irrational} \)

These cannot be written as a fraction of Integers.
\end{enumerate}

\textbf{General Comment:} First, you \textbf{NEED} to simplify the expression. This question simplifies to $-65$. 
 
 Be sure you look at the simplified fraction and not just the decimal expansion. Numbers such as 13, 17, and 19 provide \textbf{long but repeating/terminating decimal expansions!} 
 
 The only ways to *not* be a Real number are: dividing by 0 or taking the square root of a negative number. 
 
 Irrational numbers are more than just square root of 3: adding or subtracting values from square root of 3 is also irrational.
}
\litem{
Simplify the expression below and choose the interval the simplification is contained within.
\[ 11 - 12^2 + 5 \div 16 * 2 \div 3 \]The solution is \( -132.792 \), which is option B.\begin{enumerate}[label=\Alph*.]
\item \( [155.1, 155.41] \)

 155.208, which corresponds to an Order of Operations error: multiplying by negative before squaring. For example: $(-3)^2 \neq -3^2$
\item \( [-132.91, -132.73] \)

* -132.792, this is the correct option
\item \( [-133.09, -132.89] \)

 -132.948, which corresponds to an Order of Operations error: not reading left-to-right for multiplication/division.
\item \( [154.88, 155.12] \)

 155.052, which corresponds to two Order of Operations errors.
\item \( \text{None of the above} \)

 You may have gotten this by making an unanticipated error. If you got a value that is not any of the others, please let the coordinator know so they can help you figure out what happened.
\end{enumerate}

\textbf{General Comment:} While you may remember (or were taught) PEMDAS is done in order, it is actually done as P/E/MD/AS. When we are at MD or AS, we read left to right.
}
\litem{
Simplify the expression below into the form $a+bi$. Then, choose the intervals that $a$ and $b$ belong to.
\[ (2 - 10 i)(5 + 8 i) \]The solution is \( 90 - 34 i \), which is option C.\begin{enumerate}[label=\Alph*.]
\item \( a \in [-70, -65] \text{ and } b \in [61, 72] \)

 $-70 + 66 i$, which corresponds to adding a minus sign in the first term.
\item \( a \in [87, 93] \text{ and } b \in [30, 38] \)

 $90 + 34 i$, which corresponds to adding a minus sign in both terms.
\item \( a \in [87, 93] \text{ and } b \in [-37, -31] \)

* $90 - 34 i$, which is the correct option.
\item \( a \in [-70, -65] \text{ and } b \in [-69, -63] \)

 $-70 - 66 i$, which corresponds to adding a minus sign in the second term.
\item \( a \in [7, 14] \text{ and } b \in [-82, -77] \)

 $10 - 80 i$, which corresponds to just multiplying the real terms to get the real part of the solution and the coefficients in the complex terms to get the complex part.
\end{enumerate}

\textbf{General Comment:} You can treat $i$ as a variable and distribute. Just remember that $i^2=-1$, so you can continue to reduce after you distribute.
}
\litem{
Simplify the expression below into the form $a+bi$. Then, choose the intervals that $a$ and $b$ belong to.
\[ (-8 + 4 i)(5 + 7 i) \]The solution is \( -68 - 36 i \), which is option E.\begin{enumerate}[label=\Alph*.]
\item \( a \in [-69, -62] \text{ and } b \in [33, 39] \)

 $-68 + 36 i$, which corresponds to adding a minus sign in both terms.
\item \( a \in [-16, -4] \text{ and } b \in [-77, -75] \)

 $-12 - 76 i$, which corresponds to adding a minus sign in the first term.
\item \( a \in [-41, -38] \text{ and } b \in [25, 29] \)

 $-40 + 28 i$, which corresponds to just multiplying the real terms to get the real part of the solution and the coefficients in the complex terms to get the complex part.
\item \( a \in [-16, -4] \text{ and } b \in [74, 83] \)

 $-12 + 76 i$, which corresponds to adding a minus sign in the second term.
\item \( a \in [-69, -62] \text{ and } b \in [-38, -35] \)

* $-68 - 36 i$, which is the correct option.
\end{enumerate}

\textbf{General Comment:} You can treat $i$ as a variable and distribute. Just remember that $i^2=-1$, so you can continue to reduce after you distribute.
}
\litem{
Choose the \textbf{smallest} set of Real numbers that the number below belongs to.
\[ \sqrt{\frac{140625}{625}} \]The solution is \( \text{Whole} \), which is option D.\begin{enumerate}[label=\Alph*.]
\item \( \text{Integer} \)

These are the negative and positive counting numbers (..., -3, -2, -1, 0, 1, 2, 3, ...)
\item \( \text{Not a Real number} \)

These are Nonreal Complex numbers \textbf{OR} things that are not numbers (e.g., dividing by 0).
\item \( \text{Rational} \)

These are numbers that can be written as fraction of Integers (e.g., -2/3)
\item \( \text{Whole} \)

* This is the correct option!
\item \( \text{Irrational} \)

These cannot be written as a fraction of Integers.
\end{enumerate}

\textbf{General Comment:} First, you \textbf{NEED} to simplify the expression. This question simplifies to $375$. 
 
 Be sure you look at the simplified fraction and not just the decimal expansion. Numbers such as 13, 17, and 19 provide \textbf{long but repeating/terminating decimal expansions!} 
 
 The only ways to *not* be a Real number are: dividing by 0 or taking the square root of a negative number. 
 
 Irrational numbers are more than just square root of 3: adding or subtracting values from square root of 3 is also irrational.
}
\litem{
Choose the \textbf{smallest} set of Complex numbers that the number below belongs to.
\[ \frac{4}{2}+64i^2 \]The solution is \( \text{Rational} \), which is option E.\begin{enumerate}[label=\Alph*.]
\item \( \text{Nonreal Complex} \)

This is a Complex number $(a+bi)$ that is not Real (has $i$ as part of the number).
\item \( \text{Irrational} \)

These cannot be written as a fraction of Integers. Remember: $\pi$ is not an Integer!
\item \( \text{Pure Imaginary} \)

This is a Complex number $(a+bi)$ that \textbf{only} has an imaginary part like $2i$.
\item \( \text{Not a Complex Number} \)

This is not a number. The only non-Complex number we know is dividing by 0 as this is not a number!
\item \( \text{Rational} \)

* This is the correct option!
\end{enumerate}

\textbf{General Comment:} Be sure to simplify $i^2 = -1$. This may remove the imaginary portion for your number. If you are having trouble, you may want to look at the \textit{Subgroups of the Real Numbers} section.
}
\litem{
Simplify the expression below into the form $a+bi$. Then, choose the intervals that $a$ and $b$ belong to.
\[ \frac{72 + 44 i}{5 + 2 i} \]The solution is \( 15.45  + 2.62 i \), which is option D.\begin{enumerate}[label=\Alph*.]
\item \( a \in [14.5, 16] \text{ and } b \in [75, 77] \)

 $15.45  + 76.00 i$, which corresponds to forgetting to multiply the conjugate by the numerator.
\item \( a \in [8, 10] \text{ and } b \in [12, 14] \)

 $9.38  + 12.55 i$, which corresponds to forgetting to multiply the conjugate by the numerator and not computing the conjugate correctly.
\item \( a \in [14, 15] \text{ and } b \in [20.5, 22.5] \)

 $14.40  + 22.00 i$, which corresponds to just dividing the first term by the first term and the second by the second.
\item \( a \in [14.5, 16] \text{ and } b \in [1.5, 3] \)

* $15.45  + 2.62 i$, which is the correct option.
\item \( a \in [447.5, 449] \text{ and } b \in [1.5, 3] \)

 $448.00  + 2.62 i$, which corresponds to forgetting to multiply the conjugate by the numerator and using a plus instead of a minus in the denominator.
\end{enumerate}

\textbf{General Comment:} Multiply the numerator and denominator by the *conjugate* of the denominator, then simplify. For example, if we have $2+3i$, the conjugate is $2-3i$.
}
\end{enumerate}

\end{document}