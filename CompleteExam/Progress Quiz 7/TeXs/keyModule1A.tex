\documentclass{extbook}[14pt]
\usepackage{multicol, enumerate, enumitem, hyperref, color, soul, setspace, parskip, fancyhdr, amssymb, amsthm, amsmath, latexsym, units, mathtools}
\everymath{\displaystyle}
\usepackage[headsep=0.5cm,headheight=0cm, left=1 in,right= 1 in,top= 1 in,bottom= 1 in]{geometry}
\usepackage{dashrule}  % Package to use the command below to create lines between items
\newcommand{\litem}[1]{\item #1

\rule{\textwidth}{0.4pt}}
\pagestyle{fancy}
\lhead{}
\chead{Answer Key for Progress Quiz 7 Version A}
\rhead{}
\lfoot{6523-2736}
\cfoot{}
\rfoot{test}
\begin{document}
\textbf{This key should allow you to understand why you choose the option you did (beyond just getting a question right or wrong). \href{https://xronos.clas.ufl.edu/mac1105spring2020/courseDescriptionAndMisc/Exams/LearningFromResults}{More instructions on how to use this key can be found here}.}

\textbf{If you have a suggestion to make the keys better, \href{https://forms.gle/CZkbZmPbC9XALEE88}{please fill out the short survey here}.}

\textit{Note: This key is auto-generated and may contain issues and/or errors. The keys are reviewed after each exam to ensure grading is done accurately. If there are issues (like duplicate options), they are noted in the offline gradebook. The keys are a work-in-progress to give students as many resources to improve as possible.}

\rule{\textwidth}{0.4pt}

\begin{enumerate}\litem{
Choose the \textbf{smallest} set of Real numbers that the number below belongs to.
\[ \sqrt{\frac{49}{169}} \]The solution is \( \text{Rational} \), which is option A.\begin{enumerate}[label=\Alph*.]
\item \( \text{Rational} \)

* This is the correct option!
\item \( \text{Not a Real number} \)

These are Nonreal Complex numbers \textbf{OR} things that are not numbers (e.g., dividing by 0).
\item \( \text{Irrational} \)

These cannot be written as a fraction of Integers.
\item \( \text{Integer} \)

These are the negative and positive counting numbers (..., -3, -2, -1, 0, 1, 2, 3, ...)
\item \( \text{Whole} \)

These are the counting numbers with 0 (0, 1, 2, 3, ...)
\end{enumerate}

\textbf{General Comment:} First, you \textbf{NEED} to simplify the expression. This question simplifies to $\frac{7}{13}$. 
 
 Be sure you look at the simplified fraction and not just the decimal expansion. Numbers such as 13, 17, and 19 provide \textbf{long but repeating/terminating decimal expansions!} 
 
 The only ways to *not* be a Real number are: dividing by 0 or taking the square root of a negative number. 
 
 Irrational numbers are more than just square root of 3: adding or subtracting values from square root of 3 is also irrational.
}
\litem{
Choose the \textbf{smallest} set of Real numbers that the number below belongs to.
\[ -\sqrt{\frac{28224}{576}} \]The solution is \( \text{Integer} \), which is option B.\begin{enumerate}[label=\Alph*.]
\item \( \text{Whole} \)

These are the counting numbers with 0 (0, 1, 2, 3, ...)
\item \( \text{Integer} \)

* This is the correct option!
\item \( \text{Not a Real number} \)

These are Nonreal Complex numbers \textbf{OR} things that are not numbers (e.g., dividing by 0).
\item \( \text{Rational} \)

These are numbers that can be written as fraction of Integers (e.g., -2/3)
\item \( \text{Irrational} \)

These cannot be written as a fraction of Integers.
\end{enumerate}

\textbf{General Comment:} First, you \textbf{NEED} to simplify the expression. This question simplifies to $-168$. 
 
 Be sure you look at the simplified fraction and not just the decimal expansion. Numbers such as 13, 17, and 19 provide \textbf{long but repeating/terminating decimal expansions!} 
 
 The only ways to *not* be a Real number are: dividing by 0 or taking the square root of a negative number. 
 
 Irrational numbers are more than just square root of 3: adding or subtracting values from square root of 3 is also irrational.
}
\litem{
Simplify the expression below and choose the interval the simplification is contained within.
\[ 8 - 6 \div 15 * 9 - (3 * 18) \]The solution is \( -49.600 \), which is option B.\begin{enumerate}[label=\Alph*.]
\item \( [61.1, 65.2] \)

 61.956, which corresponds to not distributing addition and subtraction correctly.
\item \( [-52.8, -48.7] \)

* -49.600, which is the correct option.
\item \( [-46.6, -45.8] \)

 -46.044, which corresponds to an Order of Operations error: not reading left-to-right for multiplication/division.
\item \( [19.8, 26.9] \)

 25.200, which corresponds to not distributing a negative correctly.
\item \( \text{None of the above} \)

 You may have gotten this by making an unanticipated error. If you got a value that is not any of the others, please let the coordinator know so they can help you figure out what happened.
\end{enumerate}

\textbf{General Comment:} While you may remember (or were taught) PEMDAS is done in order, it is actually done as P/E/MD/AS. When we are at MD or AS, we read left to right.
}
\litem{
Choose the \textbf{smallest} set of Complex numbers that the number below belongs to.
\[ \frac{6}{-17}+64i^2 \]The solution is \( \text{Rational} \), which is option B.\begin{enumerate}[label=\Alph*.]
\item \( \text{Nonreal Complex} \)

This is a Complex number $(a+bi)$ that is not Real (has $i$ as part of the number).
\item \( \text{Rational} \)

* This is the correct option!
\item \( \text{Irrational} \)

These cannot be written as a fraction of Integers. Remember: $\pi$ is not an Integer!
\item \( \text{Pure Imaginary} \)

This is a Complex number $(a+bi)$ that \textbf{only} has an imaginary part like $2i$.
\item \( \text{Not a Complex Number} \)

This is not a number. The only non-Complex number we know is dividing by 0 as this is not a number!
\end{enumerate}

\textbf{General Comment:} Be sure to simplify $i^2 = -1$. This may remove the imaginary portion for your number. If you are having trouble, you may want to look at the \textit{Subgroups of the Real Numbers} section.
}
\litem{
Simplify the expression below into the form $a+bi$. Then, choose the intervals that $a$ and $b$ belong to.
\[ \frac{9 - 88 i}{-7 - 3 i} \]The solution is \( 3.47  + 11.09 i \), which is option D.\begin{enumerate}[label=\Alph*.]
\item \( a \in [-7, -5] \text{ and } b \in [10, 10.5] \)

 $-5.64  + 10.16 i$, which corresponds to forgetting to multiply the conjugate by the numerator and not computing the conjugate correctly.
\item \( a \in [-1.5, -0.5] \text{ and } b \in [29, 30] \)

 $-1.29  + 29.33 i$, which corresponds to just dividing the first term by the first term and the second by the second.
\item \( a \in [2.5, 5] \text{ and } b \in [641.5, 643.5] \)

 $3.47  + 643.00 i$, which corresponds to forgetting to multiply the conjugate by the numerator.
\item \( a \in [2.5, 5] \text{ and } b \in [10.5, 11.5] \)

* $3.47  + 11.09 i$, which is the correct option.
\item \( a \in [200.5, 201.5] \text{ and } b \in [10.5, 11.5] \)

 $201.00  + 11.09 i$, which corresponds to forgetting to multiply the conjugate by the numerator and using a plus instead of a minus in the denominator.
\end{enumerate}

\textbf{General Comment:} Multiply the numerator and denominator by the *conjugate* of the denominator, then simplify. For example, if we have $2+3i$, the conjugate is $2-3i$.
}
\litem{
Simplify the expression below into the form $a+bi$. Then, choose the intervals that $a$ and $b$ belong to.
\[ (-8 - 4 i)(-9 + 6 i) \]The solution is \( 96 - 12 i \), which is option D.\begin{enumerate}[label=\Alph*.]
\item \( a \in [47, 51] \text{ and } b \in [-87, -78] \)

 $48 - 84 i$, which corresponds to adding a minus sign in the first term.
\item \( a \in [94, 97] \text{ and } b \in [9, 14] \)

 $96 + 12 i$, which corresponds to adding a minus sign in both terms.
\item \( a \in [69, 77] \text{ and } b \in [-27, -22] \)

 $72 - 24 i$, which corresponds to just multiplying the real terms to get the real part of the solution and the coefficients in the complex terms to get the complex part.
\item \( a \in [94, 97] \text{ and } b \in [-16, -10] \)

* $96 - 12 i$, which is the correct option.
\item \( a \in [47, 51] \text{ and } b \in [84, 86] \)

 $48 + 84 i$, which corresponds to adding a minus sign in the second term.
\end{enumerate}

\textbf{General Comment:} You can treat $i$ as a variable and distribute. Just remember that $i^2=-1$, so you can continue to reduce after you distribute.
}
\litem{
Simplify the expression below into the form $a+bi$. Then, choose the intervals that $a$ and $b$ belong to.
\[ \frac{-9 + 33 i}{8 - 6 i} \]The solution is \( -2.70  + 2.10 i \), which is option A.\begin{enumerate}[label=\Alph*.]
\item \( a \in [-3.5, -1.5] \text{ and } b \in [1.4, 2.45] \)

* $-2.70  + 2.10 i$, which is the correct option.
\item \( a \in [1, 2] \text{ and } b \in [2.8, 3.55] \)

 $1.26  + 3.18 i$, which corresponds to forgetting to multiply the conjugate by the numerator and not computing the conjugate correctly.
\item \( a \in [-3.5, -1.5] \text{ and } b \in [209.55, 210.2] \)

 $-2.70  + 210.00 i$, which corresponds to forgetting to multiply the conjugate by the numerator.
\item \( a \in [-270.5, -269.5] \text{ and } b \in [1.4, 2.45] \)

 $-270.00  + 2.10 i$, which corresponds to forgetting to multiply the conjugate by the numerator and using a plus instead of a minus in the denominator.
\item \( a \in [-2.5, -1] \text{ and } b \in [-5.9, -4.95] \)

 $-1.12  - 5.50 i$, which corresponds to just dividing the first term by the first term and the second by the second.
\end{enumerate}

\textbf{General Comment:} Multiply the numerator and denominator by the *conjugate* of the denominator, then simplify. For example, if we have $2+3i$, the conjugate is $2-3i$.
}
\litem{
Simplify the expression below and choose the interval the simplification is contained within.
\[ 2 - 6^2 + 10 \div 15 * 19 \div 3 \]The solution is \( -29.778 \), which is option A.\begin{enumerate}[label=\Alph*.]
\item \( [-33.78, -23.78] \)

* -29.778, this is the correct option
\item \( [-33.99, -32.99] \)

 -33.988, which corresponds to an Order of Operations error: not reading left-to-right for multiplication/division.
\item \( [35.01, 41.01] \)

 38.012, which corresponds to two Order of Operations errors.
\item \( [39.22, 45.22] \)

 42.222, which corresponds to an Order of Operations error: multiplying by negative before squaring. For example: $(-3)^2 \neq -3^2$
\item \( \text{None of the above} \)

 You may have gotten this by making an unanticipated error. If you got a value that is not any of the others, please let the coordinator know so they can help you figure out what happened.
\end{enumerate}

\textbf{General Comment:} While you may remember (or were taught) PEMDAS is done in order, it is actually done as P/E/MD/AS. When we are at MD or AS, we read left to right.
}
\litem{
Choose the \textbf{smallest} set of Complex numbers that the number below belongs to.
\[ \frac{19}{-11}+64i^2 \]The solution is \( \text{Rational} \), which is option E.\begin{enumerate}[label=\Alph*.]
\item \( \text{Not a Complex Number} \)

This is not a number. The only non-Complex number we know is dividing by 0 as this is not a number!
\item \( \text{Pure Imaginary} \)

This is a Complex number $(a+bi)$ that \textbf{only} has an imaginary part like $2i$.
\item \( \text{Nonreal Complex} \)

This is a Complex number $(a+bi)$ that is not Real (has $i$ as part of the number).
\item \( \text{Irrational} \)

These cannot be written as a fraction of Integers. Remember: $\pi$ is not an Integer!
\item \( \text{Rational} \)

* This is the correct option!
\end{enumerate}

\textbf{General Comment:} Be sure to simplify $i^2 = -1$. This may remove the imaginary portion for your number. If you are having trouble, you may want to look at the \textit{Subgroups of the Real Numbers} section.
}
\litem{
Simplify the expression below into the form $a+bi$. Then, choose the intervals that $a$ and $b$ belong to.
\[ (-6 - 7 i)(-2 + 8 i) \]The solution is \( 68 - 34 i \), which is option C.\begin{enumerate}[label=\Alph*.]
\item \( a \in [12, 13] \text{ and } b \in [-57, -54] \)

 $12 - 56 i$, which corresponds to just multiplying the real terms to get the real part of the solution and the coefficients in the complex terms to get the complex part.
\item \( a \in [-44, -38] \text{ and } b \in [61, 64] \)

 $-44 + 62 i$, which corresponds to adding a minus sign in the second term.
\item \( a \in [68, 69] \text{ and } b \in [-36, -26] \)

* $68 - 34 i$, which is the correct option.
\item \( a \in [68, 69] \text{ and } b \in [33, 38] \)

 $68 + 34 i$, which corresponds to adding a minus sign in both terms.
\item \( a \in [-44, -38] \text{ and } b \in [-66, -59] \)

 $-44 - 62 i$, which corresponds to adding a minus sign in the first term.
\end{enumerate}

\textbf{General Comment:} You can treat $i$ as a variable and distribute. Just remember that $i^2=-1$, so you can continue to reduce after you distribute.
}
\end{enumerate}

\end{document}