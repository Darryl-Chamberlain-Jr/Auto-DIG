\documentclass[14pt]{extbook}
\usepackage{multicol, enumerate, enumitem, hyperref, color, soul, setspace, parskip, fancyhdr} %General Packages
\usepackage{amssymb, amsthm, amsmath, latexsym, units, mathtools} %Math Packages
\everymath{\displaystyle} %All math in Display Style
% Packages with additional options
\usepackage[headsep=0.5cm,headheight=12pt, left=1 in,right= 1 in,top= 1 in,bottom= 1 in]{geometry}
\usepackage[usenames,dvipsnames]{xcolor}
\usepackage{dashrule}  % Package to use the command below to create lines between items
\newcommand{\litem}[1]{\item#1\hspace*{-1cm}\rule{\textwidth}{0.4pt}}
\pagestyle{fancy}
\lhead{Progress Quiz 7}
\chead{}
\rhead{Version A}
\lfoot{3510-5252}
\cfoot{}
\rfoot{Summer C 2021}
\begin{document}

\begin{enumerate}
\litem{
Add the following functions, then choose the domain of the resulting function from the list below.\[ f(x) = 3x^{3} +8 x + 5 \text{ and } g(x) = \sqrt{-5x-18}  \]\begin{enumerate}[label=\Alph*.]
\item \( \text{ The domain is all Real numbers less than or equal to } x = a, \text{ where } a \in [-7.6, 1.4] \)
\item \( \text{ The domain is all Real numbers except } x = a, \text{ where } a \in [-8.6, -3.6] \)
\item \( \text{ The domain is all Real numbers greater than or equal to } x = a, \text{ where } a \in [4, 9] \)
\item \( \text{ The domain is all Real numbers except } x = a \text{ and } x = b, \text{ where } a \in [3.75, 10.75] \text{ and } b \in [1.2, 8.2] \)
\item \( \text{ The domain is all Real numbers. } \)

\end{enumerate} }
\litem{
Find the inverse of the function below. Then, evaluate the inverse at $x = 7$ and choose the interval that $f^-1(7)$ belongs to.\[ f(x) = \ln{(x-5)}-3 \]\begin{enumerate}[label=\Alph*.]
\item \( f^{-1}(7) \in [22026.47, 22032.47] \)
\item \( f^{-1}(7) \in [162748.79, 162754.79] \)
\item \( f^{-1}(7) \in [58.6, 62.6] \)
\item \( f^{-1}(7) \in [22015.47, 22023.47] \)
\item \( f^{-1}(7) \in [-2.61, 10.39] \)

\end{enumerate} }
\litem{
Determine whether the function below is 1-1.\[ f(x) = (4 x + 22)^3 \]\begin{enumerate}[label=\Alph*.]
\item \( \text{No, because the domain of the function is not $(-\infty, \infty)$.} \)
\item \( \text{Yes, the function is 1-1.} \)
\item \( \text{No, because the range of the function is not $(-\infty, \infty)$.} \)
\item \( \text{No, because there is an $x$-value that goes to 2 different $y$-values.} \)
\item \( \text{No, because there is a $y$-value that goes to 2 different $x$-values.} \)

\end{enumerate} }
\litem{
Find the inverse of the function below. Then, evaluate the inverse at $x = 4$ and choose the interval that $f^-1(4)$ belongs to.\[ f(x) = e^{x-2}-2 \]\begin{enumerate}[label=\Alph*.]
\item \( f^{-1}(4) \in [-0.26, 0.56] \)
\item \( f^{-1}(4) \in [-2.74, -0.56] \)
\item \( f^{-1}(4) \in [-0.26, 0.56] \)
\item \( f^{-1}(4) \in [-2.74, -0.56] \)
\item \( f^{-1}(4) \in [3.68, 4.18] \)

\end{enumerate} }
\litem{
Determine whether the function below is 1-1.\[ f(x) = 9 x^2 + 126 x + 441 \]\begin{enumerate}[label=\Alph*.]
\item \( \text{No, because there is a $y$-value that goes to 2 different $x$-values.} \)
\item \( \text{No, because the range of the function is not $(-\infty, \infty)$.} \)
\item \( \text{No, because there is an $x$-value that goes to 2 different $y$-values.} \)
\item \( \text{No, because the domain of the function is not $(-\infty, \infty)$.} \)
\item \( \text{Yes, the function is 1-1.} \)

\end{enumerate} }
\litem{
Choose the interval below that $f$ composed with $g$ at $x=1$ is in.\[ f(x) = 3x^{3} +2 x^{2} -4 x \text{ and } g(x) = 3x^{3} +2 x^{2} -4 x + 1 \]\begin{enumerate}[label=\Alph*.]
\item \( (f \circ g)(1) \in [-13, -3] \)
\item \( (f \circ g)(1) \in [21, 25] \)
\item \( (f \circ g)(1) \in [16, 21] \)
\item \( (f \circ g)(1) \in [1, 4] \)
\item \( \text{It is not possible to compose the two functions.} \)

\end{enumerate} }
\litem{
Choose the interval below that $f$ composed with $g$ at $x=-1$ is in.\[ f(x) = x^{3} -2 x^{2} -x + 2 \text{ and } g(x) = -x^{3} +4 x^{2} +x \]\begin{enumerate}[label=\Alph*.]
\item \( (f \circ g)(-1) \in [35, 40] \)
\item \( (f \circ g)(-1) \in [29, 31] \)
\item \( (f \circ g)(-1) \in [-4, 1] \)
\item \( (f \circ g)(-1) \in [7, 18] \)
\item \( \text{It is not possible to compose the two functions.} \)

\end{enumerate} }
\litem{
Find the inverse of the function below (if it exists). Then, evaluate the inverse at $x = -14$ and choose the interval that $f^-1(-14)$ belongs to.\[ f(x) = \sqrt[3]{5 x - 4} \]\begin{enumerate}[label=\Alph*.]
\item \( f^{-1}(-14) \in [547, 548.6] \)
\item \( f^{-1}(-14) \in [-551.3, -548.6] \)
\item \( f^{-1}(-14) \in [-548.6, -546.7] \)
\item \( f^{-1}(-14) \in [549, 552.1] \)
\item \( \text{ The function is not invertible for all Real numbers. } \)

\end{enumerate} }
\litem{
Find the inverse of the function below (if it exists). Then, evaluate the inverse at $x = -15$ and choose the interval that $f^-1(-15)$ belongs to.\[ f(x) = 3 x^2 - 4 \]\begin{enumerate}[label=\Alph*.]
\item \( f^{-1}(-15) \in [2.27, 2.72] \)
\item \( f^{-1}(-15) \in [2.73, 3.19] \)
\item \( f^{-1}(-15) \in [5.61, 6.24] \)
\item \( f^{-1}(-15) \in [1.77, 2.01] \)
\item \( \text{ The function is not invertible for all Real numbers. } \)

\end{enumerate} }
\litem{
Subtract the following functions, then choose the domain of the resulting function from the list below.\[ f(x) = \frac{3}{3x-14} \text{ and } g(x) = 2x^{2} +3 x + 7 \]\begin{enumerate}[label=\Alph*.]
\item \( \text{ The domain is all Real numbers except } x = a, \text{ where } a \in [3.67, 13.67] \)
\item \( \text{ The domain is all Real numbers greater than or equal to } x = a, \text{ where } a \in [-8.67, -1.67] \)
\item \( \text{ The domain is all Real numbers less than or equal to } x = a, \text{ where } a \in [-3, 0] \)
\item \( \text{ The domain is all Real numbers except } x = a \text{ and } x = b, \text{ where } a \in [2.67, 6.67] \text{ and } b \in [-4.67, 1.33] \)
\item \( \text{ The domain is all Real numbers. } \)

\end{enumerate} }
\end{enumerate}

\end{document}