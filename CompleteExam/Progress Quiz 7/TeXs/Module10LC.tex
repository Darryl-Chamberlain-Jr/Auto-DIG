\documentclass[14pt]{extbook}
\usepackage{multicol, enumerate, enumitem, hyperref, color, soul, setspace, parskip, fancyhdr} %General Packages
\usepackage{amssymb, amsthm, amsmath, latexsym, units, mathtools} %Math Packages
\everymath{\displaystyle} %All math in Display Style
% Packages with additional options
\usepackage[headsep=0.5cm,headheight=12pt, left=1 in,right= 1 in,top= 1 in,bottom= 1 in]{geometry}
\usepackage[usenames,dvipsnames]{xcolor}
\usepackage{dashrule}  % Package to use the command below to create lines between items
\newcommand{\litem}[1]{\item#1\hspace*{-1cm}\rule{\textwidth}{0.4pt}}
\pagestyle{fancy}
\lhead{Progress Quiz 7}
\chead{}
\rhead{Version C}
\lfoot{3510-5252}
\cfoot{}
\rfoot{Summer C 2021}
\begin{document}

\begin{enumerate}
\litem{
Perform the division below. Then, find the intervals that correspond to the quotient in the form $ax^2+bx+c$ and remainder $r$.\[ \frac{9x^{3} +21 x^{2} -7}{x + 2} \]\begin{enumerate}[label=\Alph*.]
\item \( a \in [8, 17], b \in [36, 48], c \in [77, 84], \text{ and } r \in [149, 151]. \)
\item \( a \in [8, 17], b \in [-8, -2], c \in [13, 21], \text{ and } r \in [-65, -60]. \)
\item \( a \in [-18, -14], b \in [54, 60], c \in [-114, -113], \text{ and } r \in [219, 224]. \)
\item \( a \in [8, 17], b \in [3, 8], c \in [-11, -3], \text{ and } r \in [4, 14]. \)
\item \( a \in [-18, -14], b \in [-17, -10], c \in [-32, -25], \text{ and } r \in [-69, -66]. \)

\end{enumerate} }
\litem{
What are the \textit{possible Rational} roots of the polynomial below?\[ f(x) = 4x^{4} +6 x^{3} +3 x^{2} +7 x + 2 \]\begin{enumerate}[label=\Alph*.]
\item \( \text{ All combinations of: }\frac{\pm 1,\pm 2,\pm 4}{\pm 1,\pm 2} \)
\item \( \text{ All combinations of: }\frac{\pm 1,\pm 2}{\pm 1,\pm 2,\pm 4} \)
\item \( \pm 1,\pm 2,\pm 4 \)
\item \( \pm 1,\pm 2 \)
\item \( \text{ There is no formula or theorem that tells us all possible Rational roots.} \)

\end{enumerate} }
\litem{
Perform the division below. Then, find the intervals that correspond to the quotient in the form $ax^2+bx+c$ and remainder $r$.\[ \frac{20x^{3} -48 x^{2} -116 x -43}{x -4} \]\begin{enumerate}[label=\Alph*.]
\item \( a \in [19, 24], \text{   } b \in [12, 13], \text{   } c \in [-85, -79], \text{   and   } r \in [-284, -280]. \)
\item \( a \in [19, 24], \text{   } b \in [-131, -125], \text{   } c \in [391, 398], \text{   and   } r \in [-1632, -1622]. \)
\item \( a \in [19, 24], \text{   } b \in [31, 38], \text{   } c \in [8, 19], \text{   and   } r \in [2, 9]. \)
\item \( a \in [80, 86], \text{   } b \in [-370, -364], \text{   } c \in [1356, 1360], \text{   and   } r \in [-5472, -5466]. \)
\item \( a \in [80, 86], \text{   } b \in [269, 275], \text{   } c \in [969, 975], \text{   and   } r \in [3842, 3846]. \)

\end{enumerate} }
\litem{
Factor the polynomial below completely, knowing that $x + 5$ is a factor. Then, choose the intervals the zeros of the polynomial belong to, where $z_1 \leq z_2 \leq z_3 \leq z_4$. \textit{To make the problem easier, all zeros are between -5 and 5.}\[ f(x) = 25x^{4} +210 x^{3} +507 x^{2} +434 x + 120 \]\begin{enumerate}[label=\Alph*.]
\item \( z_1 \in [1.24, 1.46], \text{   }  z_2 \in [1.6, 1.94], z_3 \in [1.3, 2.2], \text{   and   } z_4 \in [4.71, 5.09] \)
\item \( z_1 \in [0.1, 0.33], \text{   }  z_2 \in [1.89, 2.8], z_3 \in [2.8, 4.5], \text{   and   } z_4 \in [4.71, 5.09] \)
\item \( z_1 \in [-5.19, -4.79], \text{   }  z_2 \in [-2.35, -1.49], z_3 \in [-2, -1], \text{   and   } z_4 \in [-1.56, -1.17] \)
\item \( z_1 \in [-5.19, -4.79], \text{   }  z_2 \in [-2.35, -1.49], z_3 \in [-1.1, 1.6], \text{   and   } z_4 \in [-0.93, 0.31] \)
\item \( z_1 \in [0.5, 0.72], \text{   }  z_2 \in [-0.1, 0.82], z_3 \in [1.3, 2.2], \text{   and   } z_4 \in [4.71, 5.09] \)

\end{enumerate} }
\litem{
Factor the polynomial below completely. Then, choose the intervals the zeros of the polynomial belong to, where $z_1 \leq z_2 \leq z_3$. \textit{To make the problem easier, all zeros are between -5 and 5.}\[ f(x) = 6x^{3} +5 x^{2} -22 x -24 \]\begin{enumerate}[label=\Alph*.]
\item \( z_1 \in [-1.67, -1.39], \text{   }  z_2 \in [-1.42, -1.18], \text{   and   } z_3 \in [1.7, 2.6] \)
\item \( z_1 \in [-2.13, -1.96], \text{   }  z_2 \in [0.46, 0.55], \text{   and   } z_3 \in [3.7, 4.4] \)
\item \( z_1 \in [-2.13, -1.96], \text{   }  z_2 \in [0.62, 0.81], \text{   and   } z_3 \in [-0.5, 1.2] \)
\item \( z_1 \in [-2.13, -1.96], \text{   }  z_2 \in [1.22, 1.4], \text{   and   } z_3 \in [1, 1.9] \)
\item \( z_1 \in [-1.04, -0.67], \text{   }  z_2 \in [-0.83, -0.6], \text{   and   } z_3 \in [1.7, 2.6] \)

\end{enumerate} }
\litem{
Factor the polynomial below completely, knowing that $x -4$ is a factor. Then, choose the intervals the zeros of the polynomial belong to, where $z_1 \leq z_2 \leq z_3 \leq z_4$. \textit{To make the problem easier, all zeros are between -5 and 5.}\[ f(x) = 12x^{4} -53 x^{3} -23 x^{2} +202 x -120 \]\begin{enumerate}[label=\Alph*.]
\item \( z_1 \in [-3.4, -1.4], \text{   }  z_2 \in [0.68, 0.95], z_3 \in [1.54, 1.71], \text{   and   } z_4 \in [4, 6] \)
\item \( z_1 \in [-3.4, -1.4], \text{   }  z_2 \in [0.52, 0.7], z_3 \in [1.2, 1.38], \text{   and   } z_4 \in [4, 6] \)
\item \( z_1 \in [-5.6, -4.6], \text{   }  z_2 \in [-4.05, -3.87], z_3 \in [-0.43, -0.2], \text{   and   } z_4 \in [0, 3] \)
\item \( z_1 \in [-4.7, -3.1], \text{   }  z_2 \in [-1.44, -1.16], z_3 \in [-0.71, -0.32], \text{   and   } z_4 \in [0, 3] \)
\item \( z_1 \in [-4.7, -3.1], \text{   }  z_2 \in [-1.75, -1.65], z_3 \in [-0.85, -0.62], \text{   and   } z_4 \in [0, 3] \)

\end{enumerate} }
\litem{
Perform the division below. Then, find the intervals that correspond to the quotient in the form $ax^2+bx+c$ and remainder $r$.\[ \frac{8x^{3} +28 x^{2} -33}{x + 3} \]\begin{enumerate}[label=\Alph*.]
\item \( a \in [5, 12], b \in [4, 6], c \in [-13, -3], \text{ and } r \in [0, 8]. \)
\item \( a \in [5, 12], b \in [52, 57], c \in [156, 158], \text{ and } r \in [435, 437]. \)
\item \( a \in [5, 12], b \in [-6, 1], c \in [13, 19], \text{ and } r \in [-104, -92]. \)
\item \( a \in [-24, -23], b \in [97, 102], c \in [-300, -290], \text{ and } r \in [864, 875]. \)
\item \( a \in [-24, -23], b \in [-48, -40], c \in [-135, -124], \text{ and } r \in [-432, -427]. \)

\end{enumerate} }
\litem{
Factor the polynomial below completely. Then, choose the intervals the zeros of the polynomial belong to, where $z_1 \leq z_2 \leq z_3$. \textit{To make the problem easier, all zeros are between -5 and 5.}\[ f(x) = 10x^{3} -41 x^{2} -54 x + 45 \]\begin{enumerate}[label=\Alph*.]
\item \( z_1 \in [-6, -4.8], \text{   }  z_2 \in [-0.8, -0.3], \text{   and   } z_3 \in [1, 1.6] \)
\item \( z_1 \in [-6, -4.8], \text{   }  z_2 \in [-3.3, -2.7], \text{   and   } z_3 \in [-0.7, 0.6] \)
\item \( z_1 \in [-6, -4.8], \text{   }  z_2 \in [-2.9, -1.5], \text{   and   } z_3 \in [0.6, 0.9] \)
\item \( z_1 \in [-1, -0.1], \text{   }  z_2 \in [0.8, 2.1], \text{   and   } z_3 \in [4.4, 5.9] \)
\item \( z_1 \in [-1.9, -1.1], \text{   }  z_2 \in [-0.1, 1.2], \text{   and   } z_3 \in [4.4, 5.9] \)

\end{enumerate} }
\litem{
Perform the division below. Then, find the intervals that correspond to the quotient in the form $ax^2+bx+c$ and remainder $r$.\[ \frac{12x^{3} +45 x^{2} -21 x -39}{x + 4} \]\begin{enumerate}[label=\Alph*.]
\item \( a \in [8, 13], \text{   } b \in [90, 102], \text{   } c \in [342, 359], \text{   and   } r \in [1363, 1367]. \)
\item \( a \in [-49, -44], \text{   } b \in [-152, -144], \text{   } c \in [-611, -607], \text{   and   } r \in [-2475, -2471]. \)
\item \( a \in [-49, -44], \text{   } b \in [232, 242], \text{   } c \in [-970, -961], \text{   and   } r \in [3834, 3839]. \)
\item \( a \in [8, 13], \text{   } b \in [-16, -10], \text{   } c \in [53, 55], \text{   and   } r \in [-310, -305]. \)
\item \( a \in [8, 13], \text{   } b \in [-3, 4], \text{   } c \in [-19, -8], \text{   and   } r \in [-5, 4]. \)

\end{enumerate} }
\litem{
What are the \textit{possible Rational} roots of the polynomial below?\[ f(x) = 3x^{2} +6 x + 5 \]\begin{enumerate}[label=\Alph*.]
\item \( \text{ All combinations of: }\frac{\pm 1,\pm 3}{\pm 1,\pm 5} \)
\item \( \text{ All combinations of: }\frac{\pm 1,\pm 5}{\pm 1,\pm 3} \)
\item \( \pm 1,\pm 5 \)
\item \( \pm 1,\pm 3 \)
\item \( \text{ There is no formula or theorem that tells us all possible Rational roots.} \)

\end{enumerate} }
\end{enumerate}

\end{document}