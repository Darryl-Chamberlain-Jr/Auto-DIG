\documentclass{extbook}[14pt]
\usepackage{multicol, enumerate, enumitem, hyperref, color, soul, setspace, parskip, fancyhdr, amssymb, amsthm, amsmath, latexsym, units, mathtools}
\everymath{\displaystyle}
\usepackage[headsep=0.5cm,headheight=0cm, left=1 in,right= 1 in,top= 1 in,bottom= 1 in]{geometry}
\usepackage{dashrule}  % Package to use the command below to create lines between items
\newcommand{\litem}[1]{\item #1

\rule{\textwidth}{0.4pt}}
\pagestyle{fancy}
\lhead{}
\chead{Answer Key for Progress Quiz 7 Version A}
\rhead{}
\lfoot{3510-5252}
\cfoot{}
\rfoot{Summer C 2021}
\begin{document}
\textbf{This key should allow you to understand why you choose the option you did (beyond just getting a question right or wrong). \href{https://xronos.clas.ufl.edu/mac1105spring2020/courseDescriptionAndMisc/Exams/LearningFromResults}{More instructions on how to use this key can be found here}.}

\textbf{If you have a suggestion to make the keys better, \href{https://forms.gle/CZkbZmPbC9XALEE88}{please fill out the short survey here}.}

\textit{Note: This key is auto-generated and may contain issues and/or errors. The keys are reviewed after each exam to ensure grading is done accurately. If there are issues (like duplicate options), they are noted in the offline gradebook. The keys are a work-in-progress to give students as many resources to improve as possible.}

\rule{\textwidth}{0.4pt}

\begin{enumerate}\litem{
For the scenario below, use the model for the volume of a cylinder as $V = \pi r^2 h$.

\begin{center}
    \textit{ Pringles wants to add 50 \text{percent} more chips to their cylinder cans and minimize the design change of their cans. They've decided that the best way to minimize the design change is to increase the radius and height by the same percentage. What should this increase be? }
\end{center}
The solution is \( \text{About } 14 \text{ percent} \), which is option B.\begin{enumerate}[label=\Alph*.]
\item \( \text{About } 22 \text{ percent} \)

This corresponds to solving correctly but treating both radius and height as equal contributors to the volume.
\item \( \text{About } 14 \text{ percent} \)

* This is the correct option.
\item \( \text{About } 4 \text{ percent} \)

This corresponds to not solving for the increase properly.
\item \( \text{About } 25 \text{ percent} \)

This corresponds to treating both radius and height as equal contributors and not solving correctly.
\item \( \text{None of the above} \)

If you chose this, please contact the coordinator to discus how you solved the problem.
\end{enumerate}

\textbf{General Comment:} Remember that when plugging the increases of values in, you need to treat it as that percentage above 100. For example, a 5 percent increase means 105 percent.
}
\litem{
For the scenario below, use the model for the volume of a cylinder as $V = \pi r^2 h$.

\begin{center}
    \textit{ Pringles wants to add 24 \text{percent} more chips to their cylinder cans and minimize the design change of their cans. They've decided that the best way to minimize the design change is to increase the radius and height by the same percentage. What should this increase be? }
\end{center}
The solution is \( \text{About } 7 \text{ percent} \), which is option B.\begin{enumerate}[label=\Alph*.]
\item \( \text{About } 11 \text{ percent} \)

This corresponds to solving correctly but treating both radius and height as equal contributors to the volume.
\item \( \text{About } 7 \text{ percent} \)

* This is the correct option.
\item \( \text{About } 12 \text{ percent} \)

This corresponds to treating both radius and height as equal contributors and not solving correctly.
\item \( \text{About } 3 \text{ percent} \)

This corresponds to not solving for the increase properly.
\item \( \text{None of the above} \)

If you chose this, please contact the coordinator to discus how you solved the problem.
\end{enumerate}

\textbf{General Comment:} Remember that when plugging the increases of values in, you need to treat it as that percentage above 100. For example, a 5 percent increase means 105 percent.
}
\litem{
Determine the appropriate model for the graph of points below.

\begin{center}
    \includegraphics[width=0.5\textwidth]{../Figures/identifyModelGraph12CopyA.png}
\end{center}


The solution is \( \text{Linear model} \), which is option B.\begin{enumerate}[label=\Alph*.]
\item \( \text{Exponential model} \)

For this to be the correct option, we want an extremely slow change early, then a rapid change later.
\item \( \text{Linear model} \)

For this to be the correct option, we need to see a mostly straight line of points.
\item \( \text{Logarithmic model} \)

For this to be the correct option, we want a rapid change early, then an extremely slow change later.
\item \( \text{Non-linear Power model} \)

For this to be the correct option, we need to see a polynomial or rational shape.
\item \( \text{None of the above} \)

For this to be the correct option, we want to see no pattern in the points.
\end{enumerate}

\textbf{General Comment:} This question is testing if you can associate the models with their graphical representation. If you are having trouble, go back to the corresponding Core module to learn about the specific function you are having trouble recognizing.
}
\litem{
Solve the modeling problem below, if possible.

\begin{center}
    \textit{ A new virus is spreading throughout the world. There were initially 3 many cases reported, but the number of confirmed cases has doubled every 1 days. How long will it be until there are at least 1000000 confirmed cases? }
\end{center}
The solution is \( \text{About } 19 \text{ days} \), which is option A.\begin{enumerate}[label=\Alph*.]
\item \( \text{About } 19 \text{ days} \)

* This is the correct option.
\item \( \text{About } 8 \text{ days} \)

You modeled the situation correctly but did not apply the properties of log correctly.
\item \( \text{About } 7 \text{ days} \)

You modeled the situation with $e$ as the base and did not apply the properties of log correctly.
\item \( \text{About } 13 \text{ days} \)

You modeled the situation with $e$ as the base, but solved correctly otherwise.
\item \( \text{There is not enough information to solve the problem.} \)

If you chose this option, please contact the coordinator to discuss why you think this is the case.
\end{enumerate}

\textbf{General Comment:} Set up the model the same as in Module 11M. Then, plug in 1000000 and solve for $d$ in your model.
}
\litem{
Solve the modeling problem below, if possible.

\begin{center}
    \textit{ In CHM2045L, Brittany created a 29 liter 15 percent solution of chemical $\chi$ using two different solution percentages of chemical $\chi$. When she went to write her lab report, she realized she forgot to write the amount of each solution she used! If she remembers she used 15 percent and 35 percent solutions, what was the amount she used of the 35 percent solution? }
\end{center}
The solution is \( 0.00 liters \), which is option A.\begin{enumerate}[label=\Alph*.]
\item \( 0.00 liters \)

*This is the correct option.
\item \( 29.00 liters \)

This is the concentration of 15 percent solution.
\item \( 14.50 liters \)

This would be correct if Brittany used equal parts of each solution.
\item \( 1.29 liters \)

This was a random value. If this was not a guess, contact the coordinator to talk about how you got this value.
\item \( \text{There is not enough information to solve the problem.} \)

You may have chose this if you thought you needed to know how much of the second solution was used in the problem. Remember that the total minus the first solution would give you the second amount used.
\end{enumerate}

\textbf{General Comment:} Build the model exactly as you did in Module 9M. Then, solve for the volume you are looking for.
}
\litem{
Using the situation below, construct a linear model that describes the cost of the coffee beans $C(h)$ in terms of the weight of the high-quality coffee beans $h$.

\begin{center}
    \textit{ Veronica needs to prepare 220 of blended coffee beans selling for \$3.06 per pound. She has a high-quality bean that sells for \$4.31 a pound and a low-quality bean that sells for \$2.47 a pound. }
\end{center}
The solution is \( C(h) = 1.84 h + 543.40 \), which is option D.\begin{enumerate}[label=\Alph*.]
\item \( C(h) = -1.84 h + 948.20 \)

This would be correct if the question asked you to construct the cost model in terms of the weight of the low-quality bean.
\item \( C(h) = 4.31 h \)

This models the cost of the high-quality bean only, not the blended beans.
\item \( C(h) = 3.39 h \)

This assumes that exactly half of the high- and low- quality beans are mixed to create the blended coffee beans.
\item \( C(h) = 1.84 h + 543.40 \)

* This is the correct option since the questions asked you to construct the cost model in terms of the weight of the high-quality bean.
\item \( \text{None of the above.} \)

If you chose this option, please talk to the coordinator to discuss why.
\end{enumerate}

\textbf{General Comment:} This is exactly like the chemistry mixture question from the homework! If you are having trouble with this problem, be sure to review the video for building linear models.
}
\litem{
For the information below, construct a linear model that describes the total time $T$ spent on the path in terms of the distance of a particular part of the path \textit{if we know that all parts of the path are equal length}.

\begin{center}
    \textit{ A bicyclist is training for a race on a hilly path. Their bike keeps track of their speed at any time, but not the distance traveled. Their speed traveling up a hill is 7 mph, 10 mph when traveling down a hill, and 8 mph when traveling along a flat portion. }
\end{center}
The solution is \( 0.368 D \), which is option A.\begin{enumerate}[label=\Alph*.]
\item \( 0.368 D \)

* This is the correct option.
\item \( 25.000 D \)

The coefficient here is calculated as if you were trying to model the distance on the total path.
\item \( 560.000 D \)

The coefficient here is calculated by multiplying the distances together rather than adding.
\item \( \text{The model can be found with the information provided, but isn't options 1-3.} \)

Since we know all parts of the path are equal length, we can treat all distance variables as the same variable, $D$.
\item \( \text{The model cannot be found with the information provided.} \)

If you chose this option, please contact the coordinator to discuss why you think we cannot model the situation.
\end{enumerate}

\textbf{General Comment:} Be sure you pay attention to the variable we are writing the model in terms of. To create the model with a single variable, we have to know that variable is the same throughout each path!
}
\litem{
Determine the appropriate model for the graph of points below.

\begin{center}
    \includegraphics[width=0.5\textwidth]{../Figures/identifyModelGraph12A.png}
\end{center}


The solution is \( \text{Non-linear Power model} \), which is option D.\begin{enumerate}[label=\Alph*.]
\item \( \text{Exponential model} \)

For this to be the correct option, we want an extremely slow change early, then a rapid change later.
\item \( \text{Linear model} \)

For this to be the correct option, we need to see a mostly straight line of points.
\item \( \text{Logarithmic model} \)

For this to be the correct option, we want a rapid change early, then an extremely slow change later.
\item \( \text{Non-linear Power model} \)

For this to be the correct option, we need to see a polynomial or rational shape.
\item \( \text{None of the above} \)

For this to be the correct option, we want to see no pattern in the points.
\end{enumerate}

\textbf{General Comment:} This question is testing if you can associate the models with their graphical representation. If you are having trouble, go back to the corresponding Core module to learn about the specific function you are having trouble recognizing.
}
\litem{
Solve the modeling problem below, if possible.

\begin{center}
    \textit{ A new virus is spreading throughout the world. There were initially 8 many cases reported, but the number of confirmed cases has quadrupled every 5 days. How long will it be until there are at least 1000 confirmed cases? }
\end{center}
The solution is \( \text{About } 18 \text{ days} \), which is option A.\begin{enumerate}[label=\Alph*.]
\item \( \text{About } 18 \text{ days} \)

* This is the correct option.
\item \( \text{About } 12 \text{ days} \)

You modeled the situation with $e$ as the base and did not apply the properties of log correctly.
\item \( \text{About } 25 \text{ days} \)

You modeled the situation with $e$ as the base, but solved correctly otherwise.
\item \( \text{About } 10 \text{ days} \)

You modeled the situation correctly but did not apply the properties of log correctly.
\item \( \text{There is not enough information to solve the problem.} \)

If you chose this option, please contact the coordinator to discuss why you think this is the case.
\end{enumerate}

\textbf{General Comment:} Set up the model the same as in Module 11M. Then, plug in 1000 and solve for $d$ in your model.
}
\litem{
Solve the modeling problem below, if possible.

\begin{center}
    \textit{ In CHM2045L, Brittany created a 22 liter 26 percent solution of chemical $\chi$ using two different solution percentages of chemical $\chi$. When she went to write her lab report, she realized she forgot to write the amount of each solution she used! If she remembers she used 7 percent and 31 percent solutions, what was the amount she used of the 31 percent solution? }
\end{center}
The solution is \( 17.42 liters \), which is option B.\begin{enumerate}[label=\Alph*.]
\item \( 16.81 liters \)

This was a random value. If this was not a guess, contact the coordinator to talk about how you got this value.
\item \( 17.42 liters \)

*This is the correct option.
\item \( 4.58 liters \)

This is the concentration of 7 percent solution.
\item \( 11.00 liters \)

This would be correct if Brittany used equal parts of each solution.
\item \( \text{There is not enough information to solve the problem.} \)

You may have chose this if you thought you needed to know how much of the second solution was used in the problem. Remember that the total minus the first solution would give you the second amount used.
\end{enumerate}

\textbf{General Comment:} Build the model exactly as you did in Module 9M. Then, solve for the volume you are looking for.
}
\end{enumerate}

\end{document}