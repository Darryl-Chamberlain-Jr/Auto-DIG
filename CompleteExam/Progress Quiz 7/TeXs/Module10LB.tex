\documentclass[14pt]{extbook}
\usepackage{multicol, enumerate, enumitem, hyperref, color, soul, setspace, parskip, fancyhdr} %General Packages
\usepackage{amssymb, amsthm, amsmath, latexsym, units, mathtools} %Math Packages
\everymath{\displaystyle} %All math in Display Style
% Packages with additional options
\usepackage[headsep=0.5cm,headheight=12pt, left=1 in,right= 1 in,top= 1 in,bottom= 1 in]{geometry}
\usepackage[usenames,dvipsnames]{xcolor}
\usepackage{dashrule}  % Package to use the command below to create lines between items
\newcommand{\litem}[1]{\item#1\hspace*{-1cm}\rule{\textwidth}{0.4pt}}
\pagestyle{fancy}
\lhead{Progress Quiz 7}
\chead{}
\rhead{Version B}
\lfoot{3510-5252}
\cfoot{}
\rfoot{Summer C 2021}
\begin{document}

\begin{enumerate}
\litem{
Perform the division below. Then, find the intervals that correspond to the quotient in the form $ax^2+bx+c$ and remainder $r$.\[ \frac{15x^{3} -65 x^{2} + 82}{x -4} \]\begin{enumerate}[label=\Alph*.]
\item \( a \in [13, 16], b \in [-24, -15], c \in [-60, -55], \text{ and } r \in [-99, -97]. \)
\item \( a \in [13, 16], b \in [-11, -1], c \in [-25, -13], \text{ and } r \in [-5, 4]. \)
\item \( a \in [60, 61], b \in [175, 181], c \in [697, 708], \text{ and } r \in [2882, 2889]. \)
\item \( a \in [13, 16], b \in [-125, -123], c \in [495, 504], \text{ and } r \in [-1919, -1912]. \)
\item \( a \in [60, 61], b \in [-309, -304], c \in [1220, 1223], \text{ and } r \in [-4803, -4794]. \)

\end{enumerate} }
\litem{
What are the \textit{possible Rational} roots of the polynomial below?\[ f(x) = 6x^{2} +5 x + 2 \]\begin{enumerate}[label=\Alph*.]
\item \( \pm 1,\pm 2 \)
\item \( \text{ All combinations of: }\frac{\pm 1,\pm 2}{\pm 1,\pm 2,\pm 3,\pm 6} \)
\item \( \text{ All combinations of: }\frac{\pm 1,\pm 2,\pm 3,\pm 6}{\pm 1,\pm 2} \)
\item \( \pm 1,\pm 2,\pm 3,\pm 6 \)
\item \( \text{ There is no formula or theorem that tells us all possible Rational roots.} \)

\end{enumerate} }
\litem{
Perform the division below. Then, find the intervals that correspond to the quotient in the form $ax^2+bx+c$ and remainder $r$.\[ \frac{10x^{3} -38 x^{2} -16 x + 34}{x -4} \]\begin{enumerate}[label=\Alph*.]
\item \( a \in [37, 41], \text{   } b \in [119, 126], \text{   } c \in [468, 475], \text{   and   } r \in [1922, 1924]. \)
\item \( a \in [5, 14], \text{   } b \in [-78, -74], \text{   } c \in [296, 303], \text{   and   } r \in [-1152, -1147]. \)
\item \( a \in [5, 14], \text{   } b \in [-3, 4], \text{   } c \in [-11, -3], \text{   and   } r \in [-1, 3]. \)
\item \( a \in [37, 41], \text{   } b \in [-201, -193], \text{   } c \in [776, 778], \text{   and   } r \in [-3074, -3063]. \)
\item \( a \in [5, 14], \text{   } b \in [-10, -2], \text{   } c \in [-42, -39], \text{   and   } r \in [-86, -82]. \)

\end{enumerate} }
\litem{
Factor the polynomial below completely, knowing that $x + 4$ is a factor. Then, choose the intervals the zeros of the polynomial belong to, where $z_1 \leq z_2 \leq z_3 \leq z_4$. \textit{To make the problem easier, all zeros are between -5 and 5.}\[ f(x) = 20x^{4} +13 x^{3} -253 x^{2} +78 x + 72 \]\begin{enumerate}[label=\Alph*.]
\item \( z_1 \in [-3.3, -2.6], \text{   }  z_2 \in [-1.16, -0.5], z_3 \in [0.23, 0.44], \text{   and   } z_4 \in [3.1, 4.6] \)
\item \( z_1 \in [-3.3, -2.6], \text{   }  z_2 \in [-1.59, -1.31], z_3 \in [2.3, 2.69], \text{   and   } z_4 \in [3.1, 4.6] \)
\item \( z_1 \in [-4.7, -3.5], \text{   }  z_2 \in [-2.67, -2.31], z_3 \in [1.2, 1.91], \text{   and   } z_4 \in [1.5, 3.2] \)
\item \( z_1 \in [-3.3, -2.6], \text{   }  z_2 \in [-3.23, -2.61], z_3 \in [-0.05, 0.12], \text{   and   } z_4 \in [3.1, 4.6] \)
\item \( z_1 \in [-4.7, -3.5], \text{   }  z_2 \in [-0.5, 0.04], z_3 \in [0.72, 0.88], \text{   and   } z_4 \in [1.5, 3.2] \)

\end{enumerate} }
\litem{
Factor the polynomial below completely. Then, choose the intervals the zeros of the polynomial belong to, where $z_1 \leq z_2 \leq z_3$. \textit{To make the problem easier, all zeros are between -5 and 5.}\[ f(x) = 6x^{3} -1 x^{2} -39 x -36 \]\begin{enumerate}[label=\Alph*.]
\item \( z_1 \in [-0.79, -0.48], \text{   }  z_2 \in [-0.68, -0.58], \text{   and   } z_3 \in [2.6, 3.4] \)
\item \( z_1 \in [-3.4, -2.82], \text{   }  z_2 \in [1.28, 1.47], \text{   and   } z_3 \in [1, 1.6] \)
\item \( z_1 \in [-3.4, -2.82], \text{   }  z_2 \in [0.56, 0.82], \text{   and   } z_3 \in [-0.2, 1.1] \)
\item \( z_1 \in [-3.4, -2.82], \text{   }  z_2 \in [0.36, 0.66], \text{   and   } z_3 \in [3.4, 5.4] \)
\item \( z_1 \in [-2.03, -1.3], \text{   }  z_2 \in [-1.4, -1.18], \text{   and   } z_3 \in [2.6, 3.4] \)

\end{enumerate} }
\litem{
Factor the polynomial below completely, knowing that $x -4$ is a factor. Then, choose the intervals the zeros of the polynomial belong to, where $z_1 \leq z_2 \leq z_3 \leq z_4$. \textit{To make the problem easier, all zeros are between -5 and 5.}\[ f(x) = 8x^{4} -6 x^{3} -189 x^{2} +265 x + 300 \]\begin{enumerate}[label=\Alph*.]
\item \( z_1 \in [-5.9, -4.4], \text{   }  z_2 \in [-0.82, -0.46], z_3 \in [2.49, 2.51], \text{   and   } z_4 \in [2.7, 4.9] \)
\item \( z_1 \in [-4.7, -2.8], \text{   }  z_2 \in [-0.5, -0.38], z_3 \in [1.33, 1.35], \text{   and   } z_4 \in [4.7, 5.3] \)
\item \( z_1 \in [-5.9, -4.4], \text{   }  z_2 \in [-4.11, -3.8], z_3 \in [0.35, 0.38], \text{   and   } z_4 \in [4.7, 5.3] \)
\item \( z_1 \in [-4.7, -2.8], \text{   }  z_2 \in [-2.96, -2.39], z_3 \in [0.74, 0.76], \text{   and   } z_4 \in [4.7, 5.3] \)
\item \( z_1 \in [-5.9, -4.4], \text{   }  z_2 \in [-1.42, -1.05], z_3 \in [0.39, 0.41], \text{   and   } z_4 \in [2.7, 4.9] \)

\end{enumerate} }
\litem{
Perform the division below. Then, find the intervals that correspond to the quotient in the form $ax^2+bx+c$ and remainder $r$.\[ \frac{10x^{3} -70 x + 65}{x + 3} \]\begin{enumerate}[label=\Alph*.]
\item \( a \in [7, 12], b \in [30, 33], c \in [20, 26], \text{ and } r \in [124, 130]. \)
\item \( a \in [-38, -25], b \in [90, 93], c \in [-344, -335], \text{ and } r \in [1078, 1091]. \)
\item \( a \in [-38, -25], b \in [-91, -85], c \in [-344, -335], \text{ and } r \in [-958, -953]. \)
\item \( a \in [7, 12], b \in [-40, -39], c \in [89, 91], \text{ and } r \in [-298, -294]. \)
\item \( a \in [7, 12], b \in [-35, -29], c \in [20, 26], \text{ and } r \in [2, 13]. \)

\end{enumerate} }
\litem{
Factor the polynomial below completely. Then, choose the intervals the zeros of the polynomial belong to, where $z_1 \leq z_2 \leq z_3$. \textit{To make the problem easier, all zeros are between -5 and 5.}\[ f(x) = 20x^{3} -33 x^{2} -20 x + 12 \]\begin{enumerate}[label=\Alph*.]
\item \( z_1 \in [-2.02, -1.65], \text{   }  z_2 \in [-2.77, -1.28], \text{   and   } z_3 \in [0.09, 0.38] \)
\item \( z_1 \in [-1.2, -0.31], \text{   }  z_2 \in [0.22, 0.44], \text{   and   } z_3 \in [1.92, 2.22] \)
\item \( z_1 \in [-1.63, -1.11], \text{   }  z_2 \in [1.83, 2.91], \text{   and   } z_3 \in [2.28, 2.58] \)
\item \( z_1 \in [-2.55, -2.31], \text{   }  z_2 \in [-2.77, -1.28], \text{   and   } z_3 \in [1.1, 1.38] \)
\item \( z_1 \in [-2.02, -1.65], \text{   }  z_2 \in [-0.52, -0.21], \text{   and   } z_3 \in [0.69, 0.98] \)

\end{enumerate} }
\litem{
Perform the division below. Then, find the intervals that correspond to the quotient in the form $ax^2+bx+c$ and remainder $r$.\[ \frac{15x^{3} +67 x^{2} +94 x + 35}{x + 2} \]\begin{enumerate}[label=\Alph*.]
\item \( a \in [13, 18], \text{   } b \in [37, 39], \text{   } c \in [16, 24], \text{   and   } r \in [-11, -3]. \)
\item \( a \in [-31, -28], \text{   } b \in [6, 13], \text{   } c \in [104, 113], \text{   and   } r \in [251, 257]. \)
\item \( a \in [-31, -28], \text{   } b \in [125, 129], \text{   } c \in [-161, -159], \text{   and   } r \in [354, 357]. \)
\item \( a \in [13, 18], \text{   } b \in [92, 101], \text{   } c \in [284, 289], \text{   and   } r \in [606, 615]. \)
\item \( a \in [13, 18], \text{   } b \in [20, 23], \text{   } c \in [24, 34], \text{   and   } r \in [-50, -46]. \)

\end{enumerate} }
\litem{
What are the \textit{possible Integer} roots of the polynomial below?\[ f(x) = 6x^{4} +4 x^{3} +7 x^{2} +4 x + 7 \]\begin{enumerate}[label=\Alph*.]
\item \( \text{ All combinations of: }\frac{\pm 1,\pm 2,\pm 3,\pm 6}{\pm 1,\pm 7} \)
\item \( \pm 1,\pm 7 \)
\item \( \pm 1,\pm 2,\pm 3,\pm 6 \)
\item \( \text{ All combinations of: }\frac{\pm 1,\pm 7}{\pm 1,\pm 2,\pm 3,\pm 6} \)
\item \( \text{There is no formula or theorem that tells us all possible Integer roots.} \)

\end{enumerate} }
\end{enumerate}

\end{document}