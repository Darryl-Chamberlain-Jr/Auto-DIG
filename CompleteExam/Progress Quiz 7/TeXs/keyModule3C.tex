\documentclass{extbook}[14pt]
\usepackage{multicol, enumerate, enumitem, hyperref, color, soul, setspace, parskip, fancyhdr, amssymb, amsthm, amsmath, latexsym, units, mathtools}
\everymath{\displaystyle}
\usepackage[headsep=0.5cm,headheight=0cm, left=1 in,right= 1 in,top= 1 in,bottom= 1 in]{geometry}
\usepackage{dashrule}  % Package to use the command below to create lines between items
\newcommand{\litem}[1]{\item #1

\rule{\textwidth}{0.4pt}}
\pagestyle{fancy}
\lhead{}
\chead{Answer Key for Progress Quiz 7 Version C}
\rhead{}
\lfoot{3510-5252}
\cfoot{}
\rfoot{Summer C 2021}
\begin{document}
\textbf{This key should allow you to understand why you choose the option you did (beyond just getting a question right or wrong). \href{https://xronos.clas.ufl.edu/mac1105spring2020/courseDescriptionAndMisc/Exams/LearningFromResults}{More instructions on how to use this key can be found here}.}

\textbf{If you have a suggestion to make the keys better, \href{https://forms.gle/CZkbZmPbC9XALEE88}{please fill out the short survey here}.}

\textit{Note: This key is auto-generated and may contain issues and/or errors. The keys are reviewed after each exam to ensure grading is done accurately. If there are issues (like duplicate options), they are noted in the offline gradebook. The keys are a work-in-progress to give students as many resources to improve as possible.}

\rule{\textwidth}{0.4pt}

\begin{enumerate}\litem{
Solve the linear inequality below. Then, choose the constant and interval combination that describes the solution set.
\[ \frac{9}{6} - \frac{3}{7} x \leq \frac{5}{9} x - \frac{10}{4} \]The solution is \( [4.065, \infty) \), which is option A.\begin{enumerate}[label=\Alph*.]
\item \( [a, \infty), \text{ where } a \in [3, 6] \)

* $[4.065, \infty)$, which is the correct option.
\item \( (-\infty, a], \text{ where } a \in [-4.5, -2.25] \)

 $(-\infty, -4.065]$, which corresponds to switching the direction of the interval AND negating the endpoint. You likely did this if you did not flip the inequality when dividing by a negative as well as not moving values over to a side properly.
\item \( (-\infty, a], \text{ where } a \in [3, 4.5] \)

 $(-\infty, 4.065]$, which corresponds to switching the direction of the interval. You likely did this if you did not flip the inequality when dividing by a negative!
\item \( [a, \infty), \text{ where } a \in [-5.25, -3] \)

 $[-4.065, \infty)$, which corresponds to negating the endpoint of the solution.
\item \( \text{None of the above}. \)

You may have chosen this if you thought the inequality did not match the ends of the intervals.
\end{enumerate}

\textbf{General Comment:} Remember that less/greater than or equal to includes the endpoint, while less/greater do not. Also, remember that you need to flip the inequality when you multiply or divide by a negative.
}
\litem{
Solve the linear inequality below. Then, choose the constant and interval combination that describes the solution set.
\[ -3x + 6 \geq 5x + 3 \]The solution is \( (-\infty, 0.375] \), which is option C.\begin{enumerate}[label=\Alph*.]
\item \( [a, \infty), \text{ where } a \in [-0.4, -0.28] \)

 $[-0.375, \infty)$, which corresponds to switching the direction of the interval AND negating the endpoint. You likely did this if you did not flip the inequality when dividing by a negative as well as not moving values over to a side properly.
\item \( [a, \infty), \text{ where } a \in [0.07, 1.44] \)

 $[0.375, \infty)$, which corresponds to switching the direction of the interval. You likely did this if you did not flip the inequality when dividing by a negative!
\item \( (-\infty, a], \text{ where } a \in [-0.2, 0.56] \)

* $(-\infty, 0.375]$, which is the correct option.
\item \( (-\infty, a], \text{ where } a \in [-1.11, 0.15] \)

 $(-\infty, -0.375]$, which corresponds to negating the endpoint of the solution.
\item \( \text{None of the above}. \)

You may have chosen this if you thought the inequality did not match the ends of the intervals.
\end{enumerate}

\textbf{General Comment:} Remember that less/greater than or equal to includes the endpoint, while less/greater do not. Also, remember that you need to flip the inequality when you multiply or divide by a negative.
}
\litem{
Using an interval or intervals, describe all the $x$-values within or including a distance of the given values.
\[ \text{ No more than } 8 \text{ units from the number } 3. \]The solution is \( [-5, 11] \), which is option C.\begin{enumerate}[label=\Alph*.]
\item \( (-5, 11) \)

This describes the values less than 8 from 3
\item \( (-\infty, -5] \cup [11, \infty) \)

This describes the values no less than 8 from 3
\item \( [-5, 11] \)

This describes the values no more than 8 from 3
\item \( (-\infty, -5) \cup (11, \infty) \)

This describes the values more than 8 from 3
\item \( \text{None of the above} \)

You likely thought the values in the interval were not correct.
\end{enumerate}

\textbf{General Comment:} When thinking about this language, it helps to draw a number line and try points.
}
\litem{
Solve the linear inequality below. Then, choose the constant and interval combination that describes the solution set.
\[ 3 - 7 x < \frac{-33 x + 3}{6} \leq 5 - 6 x \]The solution is \( (1.67, 9.00] \), which is option A.\begin{enumerate}[label=\Alph*.]
\item \( (a, b], \text{ where } a \in [-1.5, 3] \text{ and } b \in [7.5, 11.25] \)

* $(1.67, 9.00]$, which is the correct option.
\item \( (-\infty, a] \cup (b, \infty), \text{ where } a \in [-0.75, 2.25] \text{ and } b \in [6, 12.75] \)

$(-\infty, 1.67] \cup (9.00, \infty)$, which corresponds to displaying the and-inequality as an or-inequality AND flipping the inequality.
\item \( (-\infty, a) \cup [b, \infty), \text{ where } a \in [0, 6] \text{ and } b \in [6, 11.25] \)

$(-\infty, 1.67) \cup [9.00, \infty)$, which corresponds to displaying the and-inequality as an or-inequality.
\item \( [a, b), \text{ where } a \in [0, 4.5] \text{ and } b \in [8.25, 12] \)

$[1.67, 9.00)$, which corresponds to flipping the inequality.
\item \( \text{None of the above.} \)


\end{enumerate}

\textbf{General Comment:} To solve, you will need to break up the compound inequality into two inequalities. Be sure to keep track of the inequality! It may be best to draw a number line and graph your solution.
}
\litem{
Solve the linear inequality below. Then, choose the constant and interval combination that describes the solution set.
\[ \frac{6}{3} + \frac{3}{4} x \geq \frac{9}{7} x + \frac{8}{6} \]The solution is \( (-\infty, 1.244] \), which is option A.\begin{enumerate}[label=\Alph*.]
\item \( (-\infty, a], \text{ where } a \in [0, 2.25] \)

* $(-\infty, 1.244]$, which is the correct option.
\item \( (-\infty, a], \text{ where } a \in [-4.5, 0.75] \)

 $(-\infty, -1.244]$, which corresponds to negating the endpoint of the solution.
\item \( [a, \infty), \text{ where } a \in [0, 2.25] \)

 $[1.244, \infty)$, which corresponds to switching the direction of the interval. You likely did this if you did not flip the inequality when dividing by a negative!
\item \( [a, \infty), \text{ where } a \in [-2.25, 0.75] \)

 $[-1.244, \infty)$, which corresponds to switching the direction of the interval AND negating the endpoint. You likely did this if you did not flip the inequality when dividing by a negative as well as not moving values over to a side properly.
\item \( \text{None of the above}. \)

You may have chosen this if you thought the inequality did not match the ends of the intervals.
\end{enumerate}

\textbf{General Comment:} Remember that less/greater than or equal to includes the endpoint, while less/greater do not. Also, remember that you need to flip the inequality when you multiply or divide by a negative.
}
\litem{
Solve the linear inequality below. Then, choose the constant and interval combination that describes the solution set.
\[ -5 + 8 x > 11 x \text{ or } 6 + 3 x < 4 x \]The solution is \( (-\infty, -1.667) \text{ or } (6.0, \infty) \), which is option B.\begin{enumerate}[label=\Alph*.]
\item \( (-\infty, a] \cup [b, \infty), \text{ where } a \in [-12, -5.25] \text{ and } b \in [-3.75, 3.75] \)

Corresponds to including the endpoints AND negating.
\item \( (-\infty, a) \cup (b, \infty), \text{ where } a \in [-2.25, -0.75] \text{ and } b \in [3.07, 7.27] \)

 * Correct option.
\item \( (-\infty, a] \cup [b, \infty), \text{ where } a \in [-3, 0.75] \text{ and } b \in [5.25, 9] \)

Corresponds to including the endpoints (when they should be excluded).
\item \( (-\infty, a) \cup (b, \infty), \text{ where } a \in [-6.75, -3.75] \text{ and } b \in [-0.75, 2.7] \)

Corresponds to inverting the inequality and negating the solution.
\item \( (-\infty, \infty) \)

Corresponds to the variable canceling, which does not happen in this instance.
\end{enumerate}

\textbf{General Comment:} When multiplying or dividing by a negative, flip the sign.
}
\litem{
Solve the linear inequality below. Then, choose the constant and interval combination that describes the solution set.
\[ -4 + 3 x < \frac{30 x + 4}{4} \leq 3 + 7 x \]The solution is \( (-1.11, 4.00] \), which is option B.\begin{enumerate}[label=\Alph*.]
\item \( [a, b), \text{ where } a \in [-3, -0.75] \text{ and } b \in [3, 6.75] \)

$[-1.11, 4.00)$, which corresponds to flipping the inequality.
\item \( (a, b], \text{ where } a \in [-2.92, -0.38] \text{ and } b \in [-1.5, 7.5] \)

* $(-1.11, 4.00]$, which is the correct option.
\item \( (-\infty, a) \cup [b, \infty), \text{ where } a \in [-9.75, -0.75] \text{ and } b \in [0.75, 5.25] \)

$(-\infty, -1.11) \cup [4.00, \infty)$, which corresponds to displaying the and-inequality as an or-inequality.
\item \( (-\infty, a] \cup (b, \infty), \text{ where } a \in [-1.5, 0.38] \text{ and } b \in [3, 9.75] \)

$(-\infty, -1.11] \cup (4.00, \infty)$, which corresponds to displaying the and-inequality as an or-inequality AND flipping the inequality.
\item \( \text{None of the above.} \)


\end{enumerate}

\textbf{General Comment:} To solve, you will need to break up the compound inequality into two inequalities. Be sure to keep track of the inequality! It may be best to draw a number line and graph your solution.
}
\litem{
Using an interval or intervals, describe all the $x$-values within or including a distance of the given values.
\[ \text{ No more than } 2 \text{ units from the number } 7. \]The solution is \( \text{None of the above} \), which is option E.\begin{enumerate}[label=\Alph*.]
\item \( (-\infty, -5) \cup (9, \infty) \)

This describes the values more than 7 from 2
\item \( [-5, 9] \)

This describes the values no more than 7 from 2
\item \( (-\infty, -5] \cup [9, \infty) \)

This describes the values no less than 7 from 2
\item \( (-5, 9) \)

This describes the values less than 7 from 2
\item \( \text{None of the above} \)

Options A-D described the values [more/less than] 7 units from 2, which is the reverse of what the question asked.
\end{enumerate}

\textbf{General Comment:} When thinking about this language, it helps to draw a number line and try points.
}
\litem{
Solve the linear inequality below. Then, choose the constant and interval combination that describes the solution set.
\[ 4 + 4 x > 6 x \text{ or } 9 + 6 x < 7 x \]The solution is \( (-\infty, 2.0) \text{ or } (9.0, \infty) \), which is option A.\begin{enumerate}[label=\Alph*.]
\item \( (-\infty, a) \cup (b, \infty), \text{ where } a \in [1.5, 6.75] \text{ and } b \in [8.25, 11.25] \)

 * Correct option.
\item \( (-\infty, a) \cup (b, \infty), \text{ where } a \in [-14.25, -3.75] \text{ and } b \in [-7.5, 0.75] \)

Corresponds to inverting the inequality and negating the solution.
\item \( (-\infty, a] \cup [b, \infty), \text{ where } a \in [0, 3] \text{ and } b \in [6.75, 11.25] \)

Corresponds to including the endpoints (when they should be excluded).
\item \( (-\infty, a] \cup [b, \infty), \text{ where } a \in [-11.25, -3] \text{ and } b \in [-4.5, -0.75] \)

Corresponds to including the endpoints AND negating.
\item \( (-\infty, \infty) \)

Corresponds to the variable canceling, which does not happen in this instance.
\end{enumerate}

\textbf{General Comment:} When multiplying or dividing by a negative, flip the sign.
}
\litem{
Solve the linear inequality below. Then, choose the constant and interval combination that describes the solution set.
\[ -5x -10 > 10x -9 \]The solution is \( (-\infty, -0.067) \), which is option C.\begin{enumerate}[label=\Alph*.]
\item \( (a, \infty), \text{ where } a \in [-0.02, 0.1] \)

 $(0.067, \infty)$, which corresponds to switching the direction of the interval AND negating the endpoint. You likely did this if you did not flip the inequality when dividing by a negative as well as not moving values over to a side properly.
\item \( (a, \infty), \text{ where } a \in [-0.09, -0.05] \)

 $(-0.067, \infty)$, which corresponds to switching the direction of the interval. You likely did this if you did not flip the inequality when dividing by a negative!
\item \( (-\infty, a), \text{ where } a \in [-0.08, 0.04] \)

* $(-\infty, -0.067)$, which is the correct option.
\item \( (-\infty, a), \text{ where } a \in [0, 0.47] \)

 $(-\infty, 0.067)$, which corresponds to negating the endpoint of the solution.
\item \( \text{None of the above}. \)

You may have chosen this if you thought the inequality did not match the ends of the intervals.
\end{enumerate}

\textbf{General Comment:} Remember that less/greater than or equal to includes the endpoint, while less/greater do not. Also, remember that you need to flip the inequality when you multiply or divide by a negative.
}
\end{enumerate}

\end{document}