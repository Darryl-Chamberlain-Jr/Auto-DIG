\documentclass[14pt]{extbook}
\usepackage{multicol, enumerate, enumitem, hyperref, color, soul, setspace, parskip, fancyhdr} %General Packages
\usepackage{amssymb, amsthm, amsmath, latexsym, units, mathtools} %Math Packages
\everymath{\displaystyle} %All math in Display Style
% Packages with additional options
\usepackage[headsep=0.5cm,headheight=12pt, left=1 in,right= 1 in,top= 1 in,bottom= 1 in]{geometry}
\usepackage[usenames,dvipsnames]{xcolor}
\usepackage{dashrule}  % Package to use the command below to create lines between items
\newcommand{\litem}[1]{\item#1\hspace*{-1cm}\rule{\textwidth}{0.4pt}}
\pagestyle{fancy}
\lhead{Progress Quiz 7}
\chead{}
\rhead{Version A}
\lfoot{3510-5252}
\cfoot{}
\rfoot{Summer C 2021}
\begin{document}

\begin{enumerate}
\litem{
Choose the equation of the function graphed below.
\begin{center}
    \includegraphics[width=0.5\textwidth]{../Figures/radicalGraphToEquationCopyA.png}
\end{center}
\begin{enumerate}[label=\Alph*.]
\item \( f(x) = - \sqrt[3]{x - 12} + 5 \)
\item \( f(x) = - \sqrt[3]{x + 12} + 5 \)
\item \( f(x) = \sqrt[3]{x - 12} + 5 \)
\item \( f(x) = \sqrt[3]{x + 12} + 5 \)
\item \( \text{None of the above} \)

\end{enumerate} }
\litem{
Choose the graph of the equation below.\[ f(x) = \sqrt{x + 10} + 4 \]\begin{enumerate}[label=\Alph*.]
\begin{multicols}{2}\item \includegraphics[width = 0.3\textwidth]{../Figures/radicalEquationToGraphAA.png}\item \includegraphics[width = 0.3\textwidth]{../Figures/radicalEquationToGraphBA.png}\item \includegraphics[width = 0.3\textwidth]{../Figures/radicalEquationToGraphCA.png}\item \includegraphics[width = 0.3\textwidth]{../Figures/radicalEquationToGraphDA.png}\end{multicols}\item None of the above.
\end{enumerate} }
\litem{
Solve the radical equation below. Then, choose the interval(s) that the solution(s) belongs to.\[ \sqrt{-36 x^2 - 12} - \sqrt{-42 x} = 0 \]\begin{enumerate}[label=\Alph*.]
\item \( x \in [0.2,0.59] \)
\item \( \text{All solutions lead to invalid or complex values in the equation.} \)
\item \( x_1 \in [-0.62, -0.24] \text{ and } x_2 \in [-2.67,0.33] \)
\item \( x_1 \in [0.2, 0.59] \text{ and } x_2 \in [-0.33,2.67] \)
\item \( x \in [0.66,1.01] \)

\end{enumerate} }
\litem{
Choose the equation of the function graphed below.
\begin{center}
    \includegraphics[width=0.5\textwidth]{../Figures/radicalGraphToEquationA.png}
\end{center}
\begin{enumerate}[label=\Alph*.]
\item \( f(x) = - \sqrt{x - 8} - 5 \)
\item \( f(x) = \sqrt{x - 8} - 5 \)
\item \( f(x) = - \sqrt{x + 8} - 5 \)
\item \( f(x) = \sqrt{x + 8} - 5 \)
\item \( \text{None of the above} \)

\end{enumerate} }
\litem{
Solve the radical equation below. Then, choose the interval(s) that the solution(s) belongs to.\[ \sqrt{56 x^2 + 15} - \sqrt{59 x} = 0 \]\begin{enumerate}[label=\Alph*.]
\item \( x_1 \in [-0.77, -0.42] \text{ and } x_2 \in [-1.17,0.14] \)
\item \( \text{All solutions lead to invalid or complex values in the equation.} \)
\item \( x_1 \in [0.41, 0.43] \text{ and } x_2 \in [-0.14,1.56] \)
\item \( x \in [0.58,0.79] \)
\item \( x \in [0.41,0.43] \)

\end{enumerate} }
\litem{
Solve the radical equation below. Then, choose the interval(s) that the solution(s) belongs to.\[ \sqrt{2 x + 8} - \sqrt{7 x - 5} = 0 \]\begin{enumerate}[label=\Alph*.]
\item \( x_1 \in [-4.6, -3.8] \text{ and } x_2 \in [-0.1,2.1] \)
\item \( \text{All solutions lead to invalid or complex values in the equation.} \)
\item \( x \in [0,0.9] \)
\item \( x_1 \in [-4.6, -3.8] \text{ and } x_2 \in [1.7,5] \)
\item \( x \in [1.8,2.9] \)

\end{enumerate} }
\litem{
What is the domain of the function below?\[ f(x) = \sqrt[8]{4 x - 3} \]\begin{enumerate}[label=\Alph*.]
\item \( (-\infty, a], \text{where } a \in [1.2, 1.4] \)
\item \( [a, \infty), \text{ where } a \in [0.51, 1.04] \)
\item \( [a, \infty), \text{where } a \in [0.76, 2.39] \)
\item \( (-\infty, \infty) \)
\item \( (-\infty, a], \text{where } a \in [-1.6, 1.2] \)

\end{enumerate} }
\litem{
Choose the graph of the equation below.\[ f(x) = - \sqrt{x + 14} - 7 \]\begin{enumerate}[label=\Alph*.]
\begin{multicols}{2}\item \includegraphics[width = 0.3\textwidth]{../Figures/radicalEquationToGraphCopyAA.png}\item \includegraphics[width = 0.3\textwidth]{../Figures/radicalEquationToGraphCopyBA.png}\item \includegraphics[width = 0.3\textwidth]{../Figures/radicalEquationToGraphCopyCA.png}\item \includegraphics[width = 0.3\textwidth]{../Figures/radicalEquationToGraphCopyDA.png}\end{multicols}\item None of the above.
\end{enumerate} }
\litem{
What is the domain of the function below?\[ f(x) = \sqrt[5]{-9 x - 6} \]\begin{enumerate}[label=\Alph*.]
\item \( \text{The domain is } [a, \infty), \text{   where } a \in [-0.95, -0.02] \)
\item \( \text{The domain is } (-\infty, a], \text{   where } a \in [-1.7, -0.82] \)
\item \( \text{The domain is } [a, \infty), \text{   where } a \in [-1.73, -0.78] \)
\item \( \text{The domain is } (-\infty, a], \text{   where } a \in [-1.03, -0.19] \)
\item \( (-\infty, \infty) \)

\end{enumerate} }
\litem{
Solve the radical equation below. Then, choose the interval(s) that the solution(s) belongs to.\[ \sqrt{-6 x + 5} - \sqrt{8 x + 8} = 0 \]\begin{enumerate}[label=\Alph*.]
\item \( x_1 \in [-1.15, -0.55] \text{ and } x_2 \in [-1.17,1.83] \)
\item \( x \in [-0.31,0.77] \)
\item \( \text{All solutions lead to invalid or complex values in the equation.} \)
\item \( x_1 \in [-0.31, 0.77] \text{ and } x_2 \in [-1.17,1.83] \)
\item \( x \in [0.92,1.48] \)

\end{enumerate} }
\end{enumerate}

\end{document}