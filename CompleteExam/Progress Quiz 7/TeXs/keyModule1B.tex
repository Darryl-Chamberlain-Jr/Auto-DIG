\documentclass{extbook}[14pt]
\usepackage{multicol, enumerate, enumitem, hyperref, color, soul, setspace, parskip, fancyhdr, amssymb, amsthm, amsmath, latexsym, units, mathtools}
\everymath{\displaystyle}
\usepackage[headsep=0.5cm,headheight=0cm, left=1 in,right= 1 in,top= 1 in,bottom= 1 in]{geometry}
\usepackage{dashrule}  % Package to use the command below to create lines between items
\newcommand{\litem}[1]{\item #1

\rule{\textwidth}{0.4pt}}
\pagestyle{fancy}
\lhead{}
\chead{Answer Key for Progress Quiz 7 Version B}
\rhead{}
\lfoot{6523-2736}
\cfoot{}
\rfoot{test}
\begin{document}
\textbf{This key should allow you to understand why you choose the option you did (beyond just getting a question right or wrong). \href{https://xronos.clas.ufl.edu/mac1105spring2020/courseDescriptionAndMisc/Exams/LearningFromResults}{More instructions on how to use this key can be found here}.}

\textbf{If you have a suggestion to make the keys better, \href{https://forms.gle/CZkbZmPbC9XALEE88}{please fill out the short survey here}.}

\textit{Note: This key is auto-generated and may contain issues and/or errors. The keys are reviewed after each exam to ensure grading is done accurately. If there are issues (like duplicate options), they are noted in the offline gradebook. The keys are a work-in-progress to give students as many resources to improve as possible.}

\rule{\textwidth}{0.4pt}

\begin{enumerate}\litem{
Choose the \textbf{smallest} set of Real numbers that the number below belongs to.
\[ -\sqrt{\frac{82944}{144}} \]The solution is \( \text{Integer} \), which is option B.\begin{enumerate}[label=\Alph*.]
\item \( \text{Irrational} \)

These cannot be written as a fraction of Integers.
\item \( \text{Integer} \)

* This is the correct option!
\item \( \text{Whole} \)

These are the counting numbers with 0 (0, 1, 2, 3, ...)
\item \( \text{Not a Real number} \)

These are Nonreal Complex numbers \textbf{OR} things that are not numbers (e.g., dividing by 0).
\item \( \text{Rational} \)

These are numbers that can be written as fraction of Integers (e.g., -2/3)
\end{enumerate}

\textbf{General Comment:} First, you \textbf{NEED} to simplify the expression. This question simplifies to $-288$. 
 
 Be sure you look at the simplified fraction and not just the decimal expansion. Numbers such as 13, 17, and 19 provide \textbf{long but repeating/terminating decimal expansions!} 
 
 The only ways to *not* be a Real number are: dividing by 0 or taking the square root of a negative number. 
 
 Irrational numbers are more than just square root of 3: adding or subtracting values from square root of 3 is also irrational.
}
\litem{
Choose the \textbf{smallest} set of Real numbers that the number below belongs to.
\[ -\sqrt{\frac{15}{0}} \]The solution is \( \text{Not a Real number} \), which is option B.\begin{enumerate}[label=\Alph*.]
\item \( \text{Integer} \)

These are the negative and positive counting numbers (..., -3, -2, -1, 0, 1, 2, 3, ...)
\item \( \text{Not a Real number} \)

* This is the correct option!
\item \( \text{Whole} \)

These are the counting numbers with 0 (0, 1, 2, 3, ...)
\item \( \text{Irrational} \)

These cannot be written as a fraction of Integers.
\item \( \text{Rational} \)

These are numbers that can be written as fraction of Integers (e.g., -2/3)
\end{enumerate}

\textbf{General Comment:} First, you \textbf{NEED} to simplify the expression. This question simplifies to $-\sqrt{\frac{15}{0}}$. 
 
 Be sure you look at the simplified fraction and not just the decimal expansion. Numbers such as 13, 17, and 19 provide \textbf{long but repeating/terminating decimal expansions!} 
 
 The only ways to *not* be a Real number are: dividing by 0 or taking the square root of a negative number. 
 
 Irrational numbers are more than just square root of 3: adding or subtracting values from square root of 3 is also irrational.
}
\litem{
Simplify the expression below and choose the interval the simplification is contained within.
\[ 6 - 9^2 + 18 \div 15 * 3 \div 13 \]The solution is \( -74.723 \), which is option A.\begin{enumerate}[label=\Alph*.]
\item \( [-74.84, -74.52] \)

* -74.723, this is the correct option
\item \( [-75.27, -74.92] \)

 -74.969, which corresponds to an Order of Operations error: not reading left-to-right for multiplication/division.
\item \( [86.83, 87.11] \)

 87.031, which corresponds to two Order of Operations errors.
\item \( [87.21, 87.35] \)

 87.277, which corresponds to an Order of Operations error: multiplying by negative before squaring. For example: $(-3)^2 \neq -3^2$
\item \( \text{None of the above} \)

 You may have gotten this by making an unanticipated error. If you got a value that is not any of the others, please let the coordinator know so they can help you figure out what happened.
\end{enumerate}

\textbf{General Comment:} While you may remember (or were taught) PEMDAS is done in order, it is actually done as P/E/MD/AS. When we are at MD or AS, we read left to right.
}
\litem{
Choose the \textbf{smallest} set of Complex numbers that the number below belongs to.
\[ \sqrt{\frac{441}{7}}+\sqrt{132} i \]The solution is \( \text{Nonreal Complex} \), which is option C.\begin{enumerate}[label=\Alph*.]
\item \( \text{Not a Complex Number} \)

This is not a number. The only non-Complex number we know is dividing by 0 as this is not a number!
\item \( \text{Pure Imaginary} \)

This is a Complex number $(a+bi)$ that \textbf{only} has an imaginary part like $2i$.
\item \( \text{Nonreal Complex} \)

* This is the correct option!
\item \( \text{Rational} \)

These are numbers that can be written as fraction of Integers (e.g., -2/3 + 5)
\item \( \text{Irrational} \)

These cannot be written as a fraction of Integers. Remember: $\pi$ is not an Integer!
\end{enumerate}

\textbf{General Comment:} Be sure to simplify $i^2 = -1$. This may remove the imaginary portion for your number. If you are having trouble, you may want to look at the \textit{Subgroups of the Real Numbers} section.
}
\litem{
Simplify the expression below into the form $a+bi$. Then, choose the intervals that $a$ and $b$ belong to.
\[ \frac{72 - 77 i}{6 + 5 i} \]The solution is \( 0.77  - 13.48 i \), which is option D.\begin{enumerate}[label=\Alph*.]
\item \( a \in [13.15, 13.5] \text{ and } b \in [-2, -1.5] \)

 $13.39  - 1.67 i$, which corresponds to forgetting to multiply the conjugate by the numerator and not computing the conjugate correctly.
\item \( a \in [46.45, 47.25] \text{ and } b \in [-14.5, -12.5] \)

 $47.00  - 13.48 i$, which corresponds to forgetting to multiply the conjugate by the numerator and using a plus instead of a minus in the denominator.
\item \( a \in [0.55, 1.05] \text{ and } b \in [-822.5, -821] \)

 $0.77  - 822.00 i$, which corresponds to forgetting to multiply the conjugate by the numerator.
\item \( a \in [0.55, 1.05] \text{ and } b \in [-14.5, -12.5] \)

* $0.77  - 13.48 i$, which is the correct option.
\item \( a \in [11.4, 12.45] \text{ and } b \in [-17, -14.5] \)

 $12.00  - 15.40 i$, which corresponds to just dividing the first term by the first term and the second by the second.
\end{enumerate}

\textbf{General Comment:} Multiply the numerator and denominator by the *conjugate* of the denominator, then simplify. For example, if we have $2+3i$, the conjugate is $2-3i$.
}
\litem{
Simplify the expression below into the form $a+bi$. Then, choose the intervals that $a$ and $b$ belong to.
\[ (7 + 10 i)(-4 - 2 i) \]The solution is \( -8 - 54 i \), which is option B.\begin{enumerate}[label=\Alph*.]
\item \( a \in [-12, -3] \text{ and } b \in [49, 59] \)

 $-8 + 54 i$, which corresponds to adding a minus sign in both terms.
\item \( a \in [-12, -3] \text{ and } b \in [-60, -51] \)

* $-8 - 54 i$, which is the correct option.
\item \( a \in [-48, -43] \text{ and } b \in [26, 27] \)

 $-48 + 26 i$, which corresponds to adding a minus sign in the first term.
\item \( a \in [-32, -25] \text{ and } b \in [-20, -19] \)

 $-28 - 20 i$, which corresponds to just multiplying the real terms to get the real part of the solution and the coefficients in the complex terms to get the complex part.
\item \( a \in [-48, -43] \text{ and } b \in [-31, -23] \)

 $-48 - 26 i$, which corresponds to adding a minus sign in the second term.
\end{enumerate}

\textbf{General Comment:} You can treat $i$ as a variable and distribute. Just remember that $i^2=-1$, so you can continue to reduce after you distribute.
}
\litem{
Simplify the expression below into the form $a+bi$. Then, choose the intervals that $a$ and $b$ belong to.
\[ \frac{72 - 11 i}{-4 + 5 i} \]The solution is \( -8.37  - 7.71 i \), which is option E.\begin{enumerate}[label=\Alph*.]
\item \( a \in [-343.5, -341.5] \text{ and } b \in [-8, -7.5] \)

 $-343.00  - 7.71 i$, which corresponds to forgetting to multiply the conjugate by the numerator and using a plus instead of a minus in the denominator.
\item \( a \in [-18.5, -16.5] \text{ and } b \in [-2.5, -2] \)

 $-18.00  - 2.20 i$, which corresponds to just dividing the first term by the first term and the second by the second.
\item \( a \in [-7, -5] \text{ and } b \in [8.5, 11] \)

 $-5.68  + 9.85 i$, which corresponds to forgetting to multiply the conjugate by the numerator and not computing the conjugate correctly.
\item \( a \in [-9, -7] \text{ and } b \in [-316.5, -315.5] \)

 $-8.37  - 316.00 i$, which corresponds to forgetting to multiply the conjugate by the numerator.
\item \( a \in [-9, -7] \text{ and } b \in [-8, -7.5] \)

* $-8.37  - 7.71 i$, which is the correct option.
\end{enumerate}

\textbf{General Comment:} Multiply the numerator and denominator by the *conjugate* of the denominator, then simplify. For example, if we have $2+3i$, the conjugate is $2-3i$.
}
\litem{
Simplify the expression below and choose the interval the simplification is contained within.
\[ 7 - 18 \div 4 * 11 - (12 * 15) \]The solution is \( -222.500 \), which is option D.\begin{enumerate}[label=\Alph*.]
\item \( [-177.41, -171.41] \)

 -173.409, which corresponds to an Order of Operations error: not reading left-to-right for multiplication/division.
\item \( [184.59, 193.59] \)

 186.591, which corresponds to not distributing addition and subtraction correctly.
\item \( [-821.5, -808.5] \)

 -817.500, which corresponds to not distributing a negative correctly.
\item \( [-223.5, -221.5] \)

* -222.500, which is the correct option.
\item \( \text{None of the above} \)

 You may have gotten this by making an unanticipated error. If you got a value that is not any of the others, please let the coordinator know so they can help you figure out what happened.
\end{enumerate}

\textbf{General Comment:} While you may remember (or were taught) PEMDAS is done in order, it is actually done as P/E/MD/AS. When we are at MD or AS, we read left to right.
}
\litem{
Choose the \textbf{smallest} set of Complex numbers that the number below belongs to.
\[ \frac{0}{9 \pi}+\sqrt{10}i \]The solution is \( \text{Pure Imaginary} \), which is option E.\begin{enumerate}[label=\Alph*.]
\item \( \text{Not a Complex Number} \)

This is not a number. The only non-Complex number we know is dividing by 0 as this is not a number!
\item \( \text{Irrational} \)

These cannot be written as a fraction of Integers. Remember: $\pi$ is not an Integer!
\item \( \text{Nonreal Complex} \)

This is a Complex number $(a+bi)$ that is not Real (has $i$ as part of the number).
\item \( \text{Rational} \)

These are numbers that can be written as fraction of Integers (e.g., -2/3 + 5)
\item \( \text{Pure Imaginary} \)

* This is the correct option!
\end{enumerate}

\textbf{General Comment:} Be sure to simplify $i^2 = -1$. This may remove the imaginary portion for your number. If you are having trouble, you may want to look at the \textit{Subgroups of the Real Numbers} section.
}
\litem{
Simplify the expression below into the form $a+bi$. Then, choose the intervals that $a$ and $b$ belong to.
\[ (4 - 9 i)(8 - 3 i) \]The solution is \( 5 - 84 i \), which is option C.\begin{enumerate}[label=\Alph*.]
\item \( a \in [4, 8] \text{ and } b \in [80, 85] \)

 $5 + 84 i$, which corresponds to adding a minus sign in both terms.
\item \( a \in [25, 37] \text{ and } b \in [25, 31] \)

 $32 + 27 i$, which corresponds to just multiplying the real terms to get the real part of the solution and the coefficients in the complex terms to get the complex part.
\item \( a \in [4, 8] \text{ and } b \in [-88, -81] \)

* $5 - 84 i$, which is the correct option.
\item \( a \in [56, 63] \text{ and } b \in [56, 67] \)

 $59 + 60 i$, which corresponds to adding a minus sign in the first term.
\item \( a \in [56, 63] \text{ and } b \in [-64, -53] \)

 $59 - 60 i$, which corresponds to adding a minus sign in the second term.
\end{enumerate}

\textbf{General Comment:} You can treat $i$ as a variable and distribute. Just remember that $i^2=-1$, so you can continue to reduce after you distribute.
}
\end{enumerate}

\end{document}