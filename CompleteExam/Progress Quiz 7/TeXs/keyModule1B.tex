\documentclass{extbook}[14pt]
\usepackage{multicol, enumerate, enumitem, hyperref, color, soul, setspace, parskip, fancyhdr, amssymb, amsthm, amsmath, latexsym, units, mathtools}
\everymath{\displaystyle}
\usepackage[headsep=0.5cm,headheight=0cm, left=1 in,right= 1 in,top= 1 in,bottom= 1 in]{geometry}
\usepackage{dashrule}  % Package to use the command below to create lines between items
\newcommand{\litem}[1]{\item #1

\rule{\textwidth}{0.4pt}}
\pagestyle{fancy}
\lhead{}
\chead{Answer Key for Progress Quiz 7 Version B}
\rhead{}
\lfoot{3510-5252}
\cfoot{}
\rfoot{Summer C 2021}
\begin{document}
\textbf{This key should allow you to understand why you choose the option you did (beyond just getting a question right or wrong). \href{https://xronos.clas.ufl.edu/mac1105spring2020/courseDescriptionAndMisc/Exams/LearningFromResults}{More instructions on how to use this key can be found here}.}

\textbf{If you have a suggestion to make the keys better, \href{https://forms.gle/CZkbZmPbC9XALEE88}{please fill out the short survey here}.}

\textit{Note: This key is auto-generated and may contain issues and/or errors. The keys are reviewed after each exam to ensure grading is done accurately. If there are issues (like duplicate options), they are noted in the offline gradebook. The keys are a work-in-progress to give students as many resources to improve as possible.}

\rule{\textwidth}{0.4pt}

\begin{enumerate}\litem{
Simplify the expression below and choose the interval the simplification is contained within.
\[ 4 - 2^2 + 1 \div 20 * 19 \div 5 \]The solution is \( 0.190 \), which is option A.\begin{enumerate}[label=\Alph*.]
\item \( [0.18, 0.27] \)

* 0.190, this is the correct option
\item \( [-0.03, 0.13] \)

 0.001, which corresponds to an Order of Operations error: not reading left-to-right for multiplication/division.
\item \( [7.96, 8.06] \)

 8.001, which corresponds to two Order of Operations errors.
\item \( [8.17, 8.23] \)

 8.190, which corresponds to an Order of Operations error: multiplying by negative before squaring. For example: $(-3)^2 \neq -3^2$
\item \( \text{None of the above} \)

 You may have gotten this by making an unanticipated error. If you got a value that is not any of the others, please let the coordinator know so they can help you figure out what happened.
\end{enumerate}

\textbf{General Comment:} While you may remember (or were taught) PEMDAS is done in order, it is actually done as P/E/MD/AS. When we are at MD or AS, we read left to right.
}
\litem{
Choose the \textbf{smallest} set of Complex numbers that the number below belongs to.
\[ \sqrt{\frac{-2178}{11}} i+\sqrt{165}i \]The solution is \( \text{Nonreal Complex} \), which is option A.\begin{enumerate}[label=\Alph*.]
\item \( \text{Nonreal Complex} \)

* This is the correct option!
\item \( \text{Pure Imaginary} \)

This is a Complex number $(a+bi)$ that \textbf{only} has an imaginary part like $2i$.
\item \( \text{Not a Complex Number} \)

This is not a number. The only non-Complex number we know is dividing by 0 as this is not a number!
\item \( \text{Rational} \)

These are numbers that can be written as fraction of Integers (e.g., -2/3 + 5)
\item \( \text{Irrational} \)

These cannot be written as a fraction of Integers. Remember: $\pi$ is not an Integer!
\end{enumerate}

\textbf{General Comment:} Be sure to simplify $i^2 = -1$. This may remove the imaginary portion for your number. If you are having trouble, you may want to look at the \textit{Subgroups of the Real Numbers} section.
}
\litem{
Simplify the expression below into the form $a+bi$. Then, choose the intervals that $a$ and $b$ belong to.
\[ \frac{63 + 88 i}{4 - 3 i} \]The solution is \( -0.48  + 21.64 i \), which is option B.\begin{enumerate}[label=\Alph*.]
\item \( a \in [15.5, 18.5] \text{ and } b \in [-29.5, -28.5] \)

 $15.75  - 29.33 i$, which corresponds to just dividing the first term by the first term and the second by the second.
\item \( a \in [-1, 0] \text{ and } b \in [20.5, 23] \)

* $-0.48  + 21.64 i$, which is the correct option.
\item \( a \in [-13, -10] \text{ and } b \in [20.5, 23] \)

 $-12.00  + 21.64 i$, which corresponds to forgetting to multiply the conjugate by the numerator and using a plus instead of a minus in the denominator.
\item \( a \in [19.5, 21.5] \text{ and } b \in [6, 7] \)

 $20.64  + 6.52 i$, which corresponds to forgetting to multiply the conjugate by the numerator and not computing the conjugate correctly.
\item \( a \in [-1, 0] \text{ and } b \in [540.5, 542.5] \)

 $-0.48  + 541.00 i$, which corresponds to forgetting to multiply the conjugate by the numerator.
\end{enumerate}

\textbf{General Comment:} Multiply the numerator and denominator by the *conjugate* of the denominator, then simplify. For example, if we have $2+3i$, the conjugate is $2-3i$.
}
\litem{
Choose the \textbf{smallest} set of Real numbers that the number below belongs to.
\[ \sqrt{\frac{53361}{441}} \]The solution is \( \text{Whole} \), which is option B.\begin{enumerate}[label=\Alph*.]
\item \( \text{Not a Real number} \)

These are Nonreal Complex numbers \textbf{OR} things that are not numbers (e.g., dividing by 0).
\item \( \text{Whole} \)

* This is the correct option!
\item \( \text{Integer} \)

These are the negative and positive counting numbers (..., -3, -2, -1, 0, 1, 2, 3, ...)
\item \( \text{Rational} \)

These are numbers that can be written as fraction of Integers (e.g., -2/3)
\item \( \text{Irrational} \)

These cannot be written as a fraction of Integers.
\end{enumerate}

\textbf{General Comment:} First, you \textbf{NEED} to simplify the expression. This question simplifies to $231$. 
 
 Be sure you look at the simplified fraction and not just the decimal expansion. Numbers such as 13, 17, and 19 provide \textbf{long but repeating/terminating decimal expansions!} 
 
 The only ways to *not* be a Real number are: dividing by 0 or taking the square root of a negative number. 
 
 Irrational numbers are more than just square root of 3: adding or subtracting values from square root of 3 is also irrational.
}
\litem{
Simplify the expression below and choose the interval the simplification is contained within.
\[ 9 - 19^2 + 13 \div 4 * 15 \div 1 \]The solution is \( -303.250 \), which is option A.\begin{enumerate}[label=\Alph*.]
\item \( [-305.25, -301.25] \)

* -303.250, this is the correct option
\item \( [368.22, 380.22] \)

 370.217, which corresponds to two Order of Operations errors.
\item \( [416.75, 420.75] \)

 418.750, which corresponds to an Order of Operations error: multiplying by negative before squaring. For example: $(-3)^2 \neq -3^2$
\item \( [-360.78, -350.78] \)

 -351.783, which corresponds to an Order of Operations error: not reading left-to-right for multiplication/division.
\item \( \text{None of the above} \)

 You may have gotten this by making an unanticipated error. If you got a value that is not any of the others, please let the coordinator know so they can help you figure out what happened.
\end{enumerate}

\textbf{General Comment:} While you may remember (or were taught) PEMDAS is done in order, it is actually done as P/E/MD/AS. When we are at MD or AS, we read left to right.
}
\litem{
Simplify the expression below into the form $a+bi$. Then, choose the intervals that $a$ and $b$ belong to.
\[ (2 - 10 i)(-8 - 3 i) \]The solution is \( -46 + 74 i \), which is option E.\begin{enumerate}[label=\Alph*.]
\item \( a \in [-18, -11] \text{ and } b \in [27, 31] \)

 $-16 + 30 i$, which corresponds to just multiplying the real terms to get the real part of the solution and the coefficients in the complex terms to get the complex part.
\item \( a \in [11, 15] \text{ and } b \in [82, 90] \)

 $14 + 86 i$, which corresponds to adding a minus sign in the second term.
\item \( a \in [-50, -38] \text{ and } b \in [-74, -69] \)

 $-46 - 74 i$, which corresponds to adding a minus sign in both terms.
\item \( a \in [11, 15] \text{ and } b \in [-88, -85] \)

 $14 - 86 i$, which corresponds to adding a minus sign in the first term.
\item \( a \in [-50, -38] \text{ and } b \in [73, 76] \)

* $-46 + 74 i$, which is the correct option.
\end{enumerate}

\textbf{General Comment:} You can treat $i$ as a variable and distribute. Just remember that $i^2=-1$, so you can continue to reduce after you distribute.
}
\litem{
Simplify the expression below into the form $a+bi$. Then, choose the intervals that $a$ and $b$ belong to.
\[ (-8 - 9 i)(10 - 4 i) \]The solution is \( -116 - 58 i \), which is option A.\begin{enumerate}[label=\Alph*.]
\item \( a \in [-118, -110] \text{ and } b \in [-60, -52] \)

* $-116 - 58 i$, which is the correct option.
\item \( a \in [-46, -43] \text{ and } b \in [-123, -115] \)

 $-44 - 122 i$, which corresponds to adding a minus sign in the second term.
\item \( a \in [-118, -110] \text{ and } b \in [52, 60] \)

 $-116 + 58 i$, which corresponds to adding a minus sign in both terms.
\item \( a \in [-46, -43] \text{ and } b \in [115, 125] \)

 $-44 + 122 i$, which corresponds to adding a minus sign in the first term.
\item \( a \in [-81, -74] \text{ and } b \in [34, 39] \)

 $-80 + 36 i$, which corresponds to just multiplying the real terms to get the real part of the solution and the coefficients in the complex terms to get the complex part.
\end{enumerate}

\textbf{General Comment:} You can treat $i$ as a variable and distribute. Just remember that $i^2=-1$, so you can continue to reduce after you distribute.
}
\litem{
Choose the \textbf{smallest} set of Real numbers that the number below belongs to.
\[ -\sqrt{\frac{39204}{484}} \]The solution is \( \text{Integer} \), which is option A.\begin{enumerate}[label=\Alph*.]
\item \( \text{Integer} \)

* This is the correct option!
\item \( \text{Not a Real number} \)

These are Nonreal Complex numbers \textbf{OR} things that are not numbers (e.g., dividing by 0).
\item \( \text{Whole} \)

These are the counting numbers with 0 (0, 1, 2, 3, ...)
\item \( \text{Irrational} \)

These cannot be written as a fraction of Integers.
\item \( \text{Rational} \)

These are numbers that can be written as fraction of Integers (e.g., -2/3)
\end{enumerate}

\textbf{General Comment:} First, you \textbf{NEED} to simplify the expression. This question simplifies to $-198$. 
 
 Be sure you look at the simplified fraction and not just the decimal expansion. Numbers such as 13, 17, and 19 provide \textbf{long but repeating/terminating decimal expansions!} 
 
 The only ways to *not* be a Real number are: dividing by 0 or taking the square root of a negative number. 
 
 Irrational numbers are more than just square root of 3: adding or subtracting values from square root of 3 is also irrational.
}
\litem{
Choose the \textbf{smallest} set of Complex numbers that the number below belongs to.
\[ \sqrt{\frac{-910}{5}} i+\sqrt{156}i \]The solution is \( \text{Nonreal Complex} \), which is option D.\begin{enumerate}[label=\Alph*.]
\item \( \text{Irrational} \)

These cannot be written as a fraction of Integers. Remember: $\pi$ is not an Integer!
\item \( \text{Not a Complex Number} \)

This is not a number. The only non-Complex number we know is dividing by 0 as this is not a number!
\item \( \text{Pure Imaginary} \)

This is a Complex number $(a+bi)$ that \textbf{only} has an imaginary part like $2i$.
\item \( \text{Nonreal Complex} \)

* This is the correct option!
\item \( \text{Rational} \)

These are numbers that can be written as fraction of Integers (e.g., -2/3 + 5)
\end{enumerate}

\textbf{General Comment:} Be sure to simplify $i^2 = -1$. This may remove the imaginary portion for your number. If you are having trouble, you may want to look at the \textit{Subgroups of the Real Numbers} section.
}
\litem{
Simplify the expression below into the form $a+bi$. Then, choose the intervals that $a$ and $b$ belong to.
\[ \frac{72 - 77 i}{-6 + 2 i} \]The solution is \( -14.65  + 7.95 i \), which is option A.\begin{enumerate}[label=\Alph*.]
\item \( a \in [-15, -14] \text{ and } b \in [7, 9.5] \)

* $-14.65  + 7.95 i$, which is the correct option.
\item \( a \in [-15, -14] \text{ and } b \in [316.5, 319] \)

 $-14.65  + 318.00 i$, which corresponds to forgetting to multiply the conjugate by the numerator.
\item \( a \in [-12.5, -11.5] \text{ and } b \in [-39, -38] \)

 $-12.00  - 38.50 i$, which corresponds to just dividing the first term by the first term and the second by the second.
\item \( a \in [-8, -6] \text{ and } b \in [14.5, 16.5] \)

 $-6.95  + 15.15 i$, which corresponds to forgetting to multiply the conjugate by the numerator and not computing the conjugate correctly.
\item \( a \in [-587, -585.5] \text{ and } b \in [7, 9.5] \)

 $-586.00  + 7.95 i$, which corresponds to forgetting to multiply the conjugate by the numerator and using a plus instead of a minus in the denominator.
\end{enumerate}

\textbf{General Comment:} Multiply the numerator and denominator by the *conjugate* of the denominator, then simplify. For example, if we have $2+3i$, the conjugate is $2-3i$.
}
\end{enumerate}

\end{document}