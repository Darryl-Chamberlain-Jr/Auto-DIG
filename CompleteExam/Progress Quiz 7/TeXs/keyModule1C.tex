\documentclass{extbook}[14pt]
\usepackage{multicol, enumerate, enumitem, hyperref, color, soul, setspace, parskip, fancyhdr, amssymb, amsthm, amsmath, latexsym, units, mathtools}
\everymath{\displaystyle}
\usepackage[headsep=0.5cm,headheight=0cm, left=1 in,right= 1 in,top= 1 in,bottom= 1 in]{geometry}
\usepackage{dashrule}  % Package to use the command below to create lines between items
\newcommand{\litem}[1]{\item #1

\rule{\textwidth}{0.4pt}}
\pagestyle{fancy}
\lhead{}
\chead{Answer Key for Progress Quiz 7 Version C}
\rhead{}
\lfoot{3510-5252}
\cfoot{}
\rfoot{Summer C 2021}
\begin{document}
\textbf{This key should allow you to understand why you choose the option you did (beyond just getting a question right or wrong). \href{https://xronos.clas.ufl.edu/mac1105spring2020/courseDescriptionAndMisc/Exams/LearningFromResults}{More instructions on how to use this key can be found here}.}

\textbf{If you have a suggestion to make the keys better, \href{https://forms.gle/CZkbZmPbC9XALEE88}{please fill out the short survey here}.}

\textit{Note: This key is auto-generated and may contain issues and/or errors. The keys are reviewed after each exam to ensure grading is done accurately. If there are issues (like duplicate options), they are noted in the offline gradebook. The keys are a work-in-progress to give students as many resources to improve as possible.}

\rule{\textwidth}{0.4pt}

\begin{enumerate}\litem{
Simplify the expression below and choose the interval the simplification is contained within.
\[ 9 - 1 \div 10 * 11 - (20 * 4) \]The solution is \( -72.100 \), which is option B.\begin{enumerate}[label=\Alph*.]
\item \( [85.3, 89.7] \)

 88.991, which corresponds to not distributing addition and subtraction correctly.
\item \( [-72.5, -71.4] \)

* -72.100, which is the correct option.
\item \( [-71.1, -69.5] \)

 -71.009, which corresponds to an Order of Operations error: not reading left-to-right for multiplication/division.
\item \( [-49.3, -47.3] \)

 -48.400, which corresponds to not distributing a negative correctly.
\item \( \text{None of the above} \)

 You may have gotten this by making an unanticipated error. If you got a value that is not any of the others, please let the coordinator know so they can help you figure out what happened.
\end{enumerate}

\textbf{General Comment:} While you may remember (or were taught) PEMDAS is done in order, it is actually done as P/E/MD/AS. When we are at MD or AS, we read left to right.
}
\litem{
Choose the \textbf{smallest} set of Complex numbers that the number below belongs to.
\[ \frac{8}{0}+\sqrt{80} i \]The solution is \( \text{Not a Complex Number} \), which is option B.\begin{enumerate}[label=\Alph*.]
\item \( \text{Nonreal Complex} \)

This is a Complex number $(a+bi)$ that is not Real (has $i$ as part of the number).
\item \( \text{Not a Complex Number} \)

* This is the correct option!
\item \( \text{Irrational} \)

These cannot be written as a fraction of Integers. Remember: $\pi$ is not an Integer!
\item \( \text{Rational} \)

These are numbers that can be written as fraction of Integers (e.g., -2/3 + 5)
\item \( \text{Pure Imaginary} \)

This is a Complex number $(a+bi)$ that \textbf{only} has an imaginary part like $2i$.
\end{enumerate}

\textbf{General Comment:} Be sure to simplify $i^2 = -1$. This may remove the imaginary portion for your number. If you are having trouble, you may want to look at the \textit{Subgroups of the Real Numbers} section.
}
\litem{
Simplify the expression below into the form $a+bi$. Then, choose the intervals that $a$ and $b$ belong to.
\[ \frac{-45 - 11 i}{8 + 4 i} \]The solution is \( -5.05  + 1.15 i \), which is option E.\begin{enumerate}[label=\Alph*.]
\item \( a \in [-4.5, -3.5] \text{ and } b \in [-3.5, -2.9] \)

 $-3.95  - 3.35 i$, which corresponds to forgetting to multiply the conjugate by the numerator and not computing the conjugate correctly.
\item \( a \in [-7, -5.5] \text{ and } b \in [-3.2, -2.7] \)

 $-5.62  - 2.75 i$, which corresponds to just dividing the first term by the first term and the second by the second.
\item \( a \in [-405.5, -403.5] \text{ and } b \in [0.9, 1.6] \)

 $-404.00  + 1.15 i$, which corresponds to forgetting to multiply the conjugate by the numerator and using a plus instead of a minus in the denominator.
\item \( a \in [-5.5, -4] \text{ and } b \in [91.45, 92.9] \)

 $-5.05  + 92.00 i$, which corresponds to forgetting to multiply the conjugate by the numerator.
\item \( a \in [-5.5, -4] \text{ and } b \in [0.9, 1.6] \)

* $-5.05  + 1.15 i$, which is the correct option.
\end{enumerate}

\textbf{General Comment:} Multiply the numerator and denominator by the *conjugate* of the denominator, then simplify. For example, if we have $2+3i$, the conjugate is $2-3i$.
}
\litem{
Choose the \textbf{smallest} set of Real numbers that the number below belongs to.
\[ -\sqrt{\frac{22}{0}} \]The solution is \( \text{Not a Real number} \), which is option D.\begin{enumerate}[label=\Alph*.]
\item \( \text{Integer} \)

These are the negative and positive counting numbers (..., -3, -2, -1, 0, 1, 2, 3, ...)
\item \( \text{Irrational} \)

These cannot be written as a fraction of Integers.
\item \( \text{Rational} \)

These are numbers that can be written as fraction of Integers (e.g., -2/3)
\item \( \text{Not a Real number} \)

* This is the correct option!
\item \( \text{Whole} \)

These are the counting numbers with 0 (0, 1, 2, 3, ...)
\end{enumerate}

\textbf{General Comment:} First, you \textbf{NEED} to simplify the expression. This question simplifies to $-\sqrt{\frac{22}{0}}$. 
 
 Be sure you look at the simplified fraction and not just the decimal expansion. Numbers such as 13, 17, and 19 provide \textbf{long but repeating/terminating decimal expansions!} 
 
 The only ways to *not* be a Real number are: dividing by 0 or taking the square root of a negative number. 
 
 Irrational numbers are more than just square root of 3: adding or subtracting values from square root of 3 is also irrational.
}
\litem{
Simplify the expression below and choose the interval the simplification is contained within.
\[ 1 - 3^2 + 19 \div 15 * 17 \div 16 \]The solution is \( -6.654 \), which is option C.\begin{enumerate}[label=\Alph*.]
\item \( [9.55, 10.63] \)

 10.005, which corresponds to two Order of Operations errors.
\item \( [10.15, 13.6] \)

 11.346, which corresponds to an Order of Operations error: multiplying by negative before squaring. For example: $(-3)^2 \neq -3^2$
\item \( [-7.07, -6.63] \)

* -6.654, this is the correct option
\item \( [-9.06, -7.76] \)

 -7.995, which corresponds to an Order of Operations error: not reading left-to-right for multiplication/division.
\item \( \text{None of the above} \)

 You may have gotten this by making an unanticipated error. If you got a value that is not any of the others, please let the coordinator know so they can help you figure out what happened.
\end{enumerate}

\textbf{General Comment:} While you may remember (or were taught) PEMDAS is done in order, it is actually done as P/E/MD/AS. When we are at MD or AS, we read left to right.
}
\litem{
Simplify the expression below into the form $a+bi$. Then, choose the intervals that $a$ and $b$ belong to.
\[ (6 - 5 i)(-7 - 10 i) \]The solution is \( -92 - 25 i \), which is option A.\begin{enumerate}[label=\Alph*.]
\item \( a \in [-98, -89] \text{ and } b \in [-25, -21] \)

* $-92 - 25 i$, which is the correct option.
\item \( a \in [-44, -40] \text{ and } b \in [43, 53] \)

 $-42 + 50 i$, which corresponds to just multiplying the real terms to get the real part of the solution and the coefficients in the complex terms to get the complex part.
\item \( a \in [-98, -89] \text{ and } b \in [23, 28] \)

 $-92 + 25 i$, which corresponds to adding a minus sign in both terms.
\item \( a \in [7, 11] \text{ and } b \in [-95, -92] \)

 $8 - 95 i$, which corresponds to adding a minus sign in the first term.
\item \( a \in [7, 11] \text{ and } b \in [88, 96] \)

 $8 + 95 i$, which corresponds to adding a minus sign in the second term.
\end{enumerate}

\textbf{General Comment:} You can treat $i$ as a variable and distribute. Just remember that $i^2=-1$, so you can continue to reduce after you distribute.
}
\litem{
Simplify the expression below into the form $a+bi$. Then, choose the intervals that $a$ and $b$ belong to.
\[ (5 - 8 i)(-4 + 2 i) \]The solution is \( -4 + 42 i \), which is option B.\begin{enumerate}[label=\Alph*.]
\item \( a \in [-42, -31] \text{ and } b \in [-23, -21] \)

 $-36 - 22 i$, which corresponds to adding a minus sign in the first term.
\item \( a \in [-6, -2] \text{ and } b \in [37, 48] \)

* $-4 + 42 i$, which is the correct option.
\item \( a \in [-42, -31] \text{ and } b \in [21, 25] \)

 $-36 + 22 i$, which corresponds to adding a minus sign in the second term.
\item \( a \in [-22, -19] \text{ and } b \in [-21, -14] \)

 $-20 - 16 i$, which corresponds to just multiplying the real terms to get the real part of the solution and the coefficients in the complex terms to get the complex part.
\item \( a \in [-6, -2] \text{ and } b \in [-47, -35] \)

 $-4 - 42 i$, which corresponds to adding a minus sign in both terms.
\end{enumerate}

\textbf{General Comment:} You can treat $i$ as a variable and distribute. Just remember that $i^2=-1$, so you can continue to reduce after you distribute.
}
\litem{
Choose the \textbf{smallest} set of Real numbers that the number below belongs to.
\[ -\sqrt{\frac{2574}{13}} \]The solution is \( \text{Irrational} \), which is option C.\begin{enumerate}[label=\Alph*.]
\item \( \text{Whole} \)

These are the counting numbers with 0 (0, 1, 2, 3, ...)
\item \( \text{Rational} \)

These are numbers that can be written as fraction of Integers (e.g., -2/3)
\item \( \text{Irrational} \)

* This is the correct option!
\item \( \text{Not a Real number} \)

These are Nonreal Complex numbers \textbf{OR} things that are not numbers (e.g., dividing by 0).
\item \( \text{Integer} \)

These are the negative and positive counting numbers (..., -3, -2, -1, 0, 1, 2, 3, ...)
\end{enumerate}

\textbf{General Comment:} First, you \textbf{NEED} to simplify the expression. This question simplifies to $-\sqrt{198}$. 
 
 Be sure you look at the simplified fraction and not just the decimal expansion. Numbers such as 13, 17, and 19 provide \textbf{long but repeating/terminating decimal expansions!} 
 
 The only ways to *not* be a Real number are: dividing by 0 or taking the square root of a negative number. 
 
 Irrational numbers are more than just square root of 3: adding or subtracting values from square root of 3 is also irrational.
}
\litem{
Choose the \textbf{smallest} set of Complex numbers that the number below belongs to.
\[ \sqrt{\frac{-1430}{10}}+\sqrt{154} \]The solution is \( \text{Nonreal Complex} \), which is option E.\begin{enumerate}[label=\Alph*.]
\item \( \text{Pure Imaginary} \)

This is a Complex number $(a+bi)$ that \textbf{only} has an imaginary part like $2i$.
\item \( \text{Irrational} \)

These cannot be written as a fraction of Integers. Remember: $\pi$ is not an Integer!
\item \( \text{Not a Complex Number} \)

This is not a number. The only non-Complex number we know is dividing by 0 as this is not a number!
\item \( \text{Rational} \)

These are numbers that can be written as fraction of Integers (e.g., -2/3 + 5)
\item \( \text{Nonreal Complex} \)

* This is the correct option!
\end{enumerate}

\textbf{General Comment:} Be sure to simplify $i^2 = -1$. This may remove the imaginary portion for your number. If you are having trouble, you may want to look at the \textit{Subgroups of the Real Numbers} section.
}
\litem{
Simplify the expression below into the form $a+bi$. Then, choose the intervals that $a$ and $b$ belong to.
\[ \frac{36 + 33 i}{-2 + 6 i} \]The solution is \( 3.15  - 7.05 i \), which is option E.\begin{enumerate}[label=\Alph*.]
\item \( a \in [2.5, 4] \text{ and } b \in [-282.5, -281] \)

 $3.15  - 282.00 i$, which corresponds to forgetting to multiply the conjugate by the numerator.
\item \( a \in [-19, -17.5] \text{ and } b \in [5, 6.5] \)

 $-18.00  + 5.50 i$, which corresponds to just dividing the first term by the first term and the second by the second.
\item \( a \in [-7, -6] \text{ and } b \in [3, 4.5] \)

 $-6.75  + 3.75 i$, which corresponds to forgetting to multiply the conjugate by the numerator and not computing the conjugate correctly.
\item \( a \in [125, 127] \text{ and } b \in [-8, -6.5] \)

 $126.00  - 7.05 i$, which corresponds to forgetting to multiply the conjugate by the numerator and using a plus instead of a minus in the denominator.
\item \( a \in [2.5, 4] \text{ and } b \in [-8, -6.5] \)

* $3.15  - 7.05 i$, which is the correct option.
\end{enumerate}

\textbf{General Comment:} Multiply the numerator and denominator by the *conjugate* of the denominator, then simplify. For example, if we have $2+3i$, the conjugate is $2-3i$.
}
\end{enumerate}

\end{document}