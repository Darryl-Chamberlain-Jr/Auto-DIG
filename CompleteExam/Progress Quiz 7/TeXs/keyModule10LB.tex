\documentclass{extbook}[14pt]
\usepackage{multicol, enumerate, enumitem, hyperref, color, soul, setspace, parskip, fancyhdr, amssymb, amsthm, amsmath, latexsym, units, mathtools}
\everymath{\displaystyle}
\usepackage[headsep=0.5cm,headheight=0cm, left=1 in,right= 1 in,top= 1 in,bottom= 1 in]{geometry}
\usepackage{dashrule}  % Package to use the command below to create lines between items
\newcommand{\litem}[1]{\item #1

\rule{\textwidth}{0.4pt}}
\pagestyle{fancy}
\lhead{}
\chead{Answer Key for Progress Quiz 7 Version B}
\rhead{}
\lfoot{3510-5252}
\cfoot{}
\rfoot{Summer C 2021}
\begin{document}
\textbf{This key should allow you to understand why you choose the option you did (beyond just getting a question right or wrong). \href{https://xronos.clas.ufl.edu/mac1105spring2020/courseDescriptionAndMisc/Exams/LearningFromResults}{More instructions on how to use this key can be found here}.}

\textbf{If you have a suggestion to make the keys better, \href{https://forms.gle/CZkbZmPbC9XALEE88}{please fill out the short survey here}.}

\textit{Note: This key is auto-generated and may contain issues and/or errors. The keys are reviewed after each exam to ensure grading is done accurately. If there are issues (like duplicate options), they are noted in the offline gradebook. The keys are a work-in-progress to give students as many resources to improve as possible.}

\rule{\textwidth}{0.4pt}

\begin{enumerate}\litem{
Perform the division below. Then, find the intervals that correspond to the quotient in the form $ax^2+bx+c$ and remainder $r$.
\[ \frac{15x^{3} -65 x^{2} + 82}{x -4} \]The solution is \( 15x^{2} -5 x -20 + \frac{2}{x -4} \), which is option B.\begin{enumerate}[label=\Alph*.]
\item \( a \in [13, 16], b \in [-24, -15], c \in [-60, -55], \text{ and } r \in [-99, -97]. \)

 You multipled by the synthetic number and subtracted rather than adding during synthetic division.
\item \( a \in [13, 16], b \in [-11, -1], c \in [-25, -13], \text{ and } r \in [-5, 4]. \)

* This is the solution!
\item \( a \in [60, 61], b \in [175, 181], c \in [697, 708], \text{ and } r \in [2882, 2889]. \)

 You multipled by the synthetic number rather than bringing the first factor down.
\item \( a \in [13, 16], b \in [-125, -123], c \in [495, 504], \text{ and } r \in [-1919, -1912]. \)

 You divided by the opposite of the factor.
\item \( a \in [60, 61], b \in [-309, -304], c \in [1220, 1223], \text{ and } r \in [-4803, -4794]. \)

 You divided by the opposite of the factor AND multipled the first factor rather than just bringing it down.
\end{enumerate}

\textbf{General Comment:} Be sure to synthetically divide by the zero of the denominator! Also, make sure to include 0 placeholders for missing terms.
}
\litem{
What are the \textit{possible Rational} roots of the polynomial below?
\[ f(x) = 6x^{2} +5 x + 2 \]The solution is \( \text{ All combinations of: }\frac{\pm 1,\pm 2}{\pm 1,\pm 2,\pm 3,\pm 6} \), which is option B.\begin{enumerate}[label=\Alph*.]
\item \( \pm 1,\pm 2 \)

This would have been the solution \textbf{if asked for the possible Integer roots}!
\item \( \text{ All combinations of: }\frac{\pm 1,\pm 2}{\pm 1,\pm 2,\pm 3,\pm 6} \)

* This is the solution \textbf{since we asked for the possible Rational roots}!
\item \( \text{ All combinations of: }\frac{\pm 1,\pm 2,\pm 3,\pm 6}{\pm 1,\pm 2} \)

 Distractor 3: Corresponds to the plus or minus of the inverse quotient (an/a0) of the factors. 
\item \( \pm 1,\pm 2,\pm 3,\pm 6 \)

 Distractor 1: Corresponds to the plus or minus factors of a1 only.
\item \( \text{ There is no formula or theorem that tells us all possible Rational roots.} \)

 Distractor 4: Corresponds to not recalling the theorem for rational roots of a polynomial.
\end{enumerate}

\textbf{General Comment:} We have a way to find the possible Rational roots. The possible Integer roots are the Integers in this list.
}
\litem{
Perform the division below. Then, find the intervals that correspond to the quotient in the form $ax^2+bx+c$ and remainder $r$.
\[ \frac{10x^{3} -38 x^{2} -16 x + 34}{x -4} \]The solution is \( 10x^{2} +2 x -8 + \frac{2}{x -4} \), which is option C.\begin{enumerate}[label=\Alph*.]
\item \( a \in [37, 41], \text{   } b \in [119, 126], \text{   } c \in [468, 475], \text{   and   } r \in [1922, 1924]. \)

 You multiplied by the synthetic number rather than bringing the first factor down.
\item \( a \in [5, 14], \text{   } b \in [-78, -74], \text{   } c \in [296, 303], \text{   and   } r \in [-1152, -1147]. \)

 You divided by the opposite of the factor.
\item \( a \in [5, 14], \text{   } b \in [-3, 4], \text{   } c \in [-11, -3], \text{   and   } r \in [-1, 3]. \)

* This is the solution!
\item \( a \in [37, 41], \text{   } b \in [-201, -193], \text{   } c \in [776, 778], \text{   and   } r \in [-3074, -3063]. \)

 You divided by the opposite of the factor AND multiplied the first factor rather than just bringing it down.
\item \( a \in [5, 14], \text{   } b \in [-10, -2], \text{   } c \in [-42, -39], \text{   and   } r \in [-86, -82]. \)

 You multiplied by the synthetic number and subtracted rather than adding during synthetic division.
\end{enumerate}

\textbf{General Comment:} Be sure to synthetically divide by the zero of the denominator!
}
\litem{
Factor the polynomial below completely, knowing that $x + 4$ is a factor. Then, choose the intervals the zeros of the polynomial belong to, where $z_1 \leq z_2 \leq z_3 \leq z_4$. \textit{To make the problem easier, all zeros are between -5 and 5.}
\[ f(x) = 20x^{4} +13 x^{3} -253 x^{2} +78 x + 72 \]The solution is \( [-4, -0.4, 0.75, 3] \), which is option E.\begin{enumerate}[label=\Alph*.]
\item \( z_1 \in [-3.3, -2.6], \text{   }  z_2 \in [-1.16, -0.5], z_3 \in [0.23, 0.44], \text{   and   } z_4 \in [3.1, 4.6] \)

 Distractor 1: Corresponds to negatives of all zeros.
\item \( z_1 \in [-3.3, -2.6], \text{   }  z_2 \in [-1.59, -1.31], z_3 \in [2.3, 2.69], \text{   and   } z_4 \in [3.1, 4.6] \)

 Distractor 3: Corresponds to negatives of all zeros AND inversing rational roots.
\item \( z_1 \in [-4.7, -3.5], \text{   }  z_2 \in [-2.67, -2.31], z_3 \in [1.2, 1.91], \text{   and   } z_4 \in [1.5, 3.2] \)

 Distractor 2: Corresponds to inversing rational roots.
\item \( z_1 \in [-3.3, -2.6], \text{   }  z_2 \in [-3.23, -2.61], z_3 \in [-0.05, 0.12], \text{   and   } z_4 \in [3.1, 4.6] \)

 Distractor 4: Corresponds to moving factors from one rational to another.
\item \( z_1 \in [-4.7, -3.5], \text{   }  z_2 \in [-0.5, 0.04], z_3 \in [0.72, 0.88], \text{   and   } z_4 \in [1.5, 3.2] \)

* This is the solution!
\end{enumerate}

\textbf{General Comment:} Remember to try the middle-most integers first as these normally are the zeros. Also, once you get it to a quadratic, you can use your other factoring techniques to finish factoring.
}
\litem{
Factor the polynomial below completely. Then, choose the intervals the zeros of the polynomial belong to, where $z_1 \leq z_2 \leq z_3$. \textit{To make the problem easier, all zeros are between -5 and 5.}
\[ f(x) = 6x^{3} -1 x^{2} -39 x -36 \]The solution is \( [-1.5, -1.33, 3] \), which is option E.\begin{enumerate}[label=\Alph*.]
\item \( z_1 \in [-0.79, -0.48], \text{   }  z_2 \in [-0.68, -0.58], \text{   and   } z_3 \in [2.6, 3.4] \)

 Distractor 2: Corresponds to inversing rational roots.
\item \( z_1 \in [-3.4, -2.82], \text{   }  z_2 \in [1.28, 1.47], \text{   and   } z_3 \in [1, 1.6] \)

 Distractor 1: Corresponds to negatives of all zeros.
\item \( z_1 \in [-3.4, -2.82], \text{   }  z_2 \in [0.56, 0.82], \text{   and   } z_3 \in [-0.2, 1.1] \)

 Distractor 3: Corresponds to negatives of all zeros AND inversing rational roots.
\item \( z_1 \in [-3.4, -2.82], \text{   }  z_2 \in [0.36, 0.66], \text{   and   } z_3 \in [3.4, 5.4] \)

 Distractor 4: Corresponds to moving factors from one rational to another.
\item \( z_1 \in [-2.03, -1.3], \text{   }  z_2 \in [-1.4, -1.18], \text{   and   } z_3 \in [2.6, 3.4] \)

* This is the solution!
\end{enumerate}

\textbf{General Comment:} Remember to try the middle-most integers first as these normally are the zeros. Also, once you get it to a quadratic, you can use your other factoring techniques to finish factoring.
}
\litem{
Factor the polynomial below completely, knowing that $x -4$ is a factor. Then, choose the intervals the zeros of the polynomial belong to, where $z_1 \leq z_2 \leq z_3 \leq z_4$. \textit{To make the problem easier, all zeros are between -5 and 5.}
\[ f(x) = 8x^{4} -6 x^{3} -189 x^{2} +265 x + 300 \]The solution is \( [-5, -0.75, 2.5, 4] \), which is option A.\begin{enumerate}[label=\Alph*.]
\item \( z_1 \in [-5.9, -4.4], \text{   }  z_2 \in [-0.82, -0.46], z_3 \in [2.49, 2.51], \text{   and   } z_4 \in [2.7, 4.9] \)

* This is the solution!
\item \( z_1 \in [-4.7, -2.8], \text{   }  z_2 \in [-0.5, -0.38], z_3 \in [1.33, 1.35], \text{   and   } z_4 \in [4.7, 5.3] \)

 Distractor 3: Corresponds to negatives of all zeros AND inversing rational roots.
\item \( z_1 \in [-5.9, -4.4], \text{   }  z_2 \in [-4.11, -3.8], z_3 \in [0.35, 0.38], \text{   and   } z_4 \in [4.7, 5.3] \)

 Distractor 4: Corresponds to moving factors from one rational to another.
\item \( z_1 \in [-4.7, -2.8], \text{   }  z_2 \in [-2.96, -2.39], z_3 \in [0.74, 0.76], \text{   and   } z_4 \in [4.7, 5.3] \)

 Distractor 1: Corresponds to negatives of all zeros.
\item \( z_1 \in [-5.9, -4.4], \text{   }  z_2 \in [-1.42, -1.05], z_3 \in [0.39, 0.41], \text{   and   } z_4 \in [2.7, 4.9] \)

 Distractor 2: Corresponds to inversing rational roots.
\end{enumerate}

\textbf{General Comment:} Remember to try the middle-most integers first as these normally are the zeros. Also, once you get it to a quadratic, you can use your other factoring techniques to finish factoring.
}
\litem{
Perform the division below. Then, find the intervals that correspond to the quotient in the form $ax^2+bx+c$ and remainder $r$.
\[ \frac{10x^{3} -70 x + 65}{x + 3} \]The solution is \( 10x^{2} -30 x + 20 + \frac{5}{x + 3} \), which is option E.\begin{enumerate}[label=\Alph*.]
\item \( a \in [7, 12], b \in [30, 33], c \in [20, 26], \text{ and } r \in [124, 130]. \)

 You divided by the opposite of the factor.
\item \( a \in [-38, -25], b \in [90, 93], c \in [-344, -335], \text{ and } r \in [1078, 1091]. \)

 You multipled by the synthetic number rather than bringing the first factor down.
\item \( a \in [-38, -25], b \in [-91, -85], c \in [-344, -335], \text{ and } r \in [-958, -953]. \)

 You divided by the opposite of the factor AND multipled the first factor rather than just bringing it down.
\item \( a \in [7, 12], b \in [-40, -39], c \in [89, 91], \text{ and } r \in [-298, -294]. \)

 You multipled by the synthetic number and subtracted rather than adding during synthetic division.
\item \( a \in [7, 12], b \in [-35, -29], c \in [20, 26], \text{ and } r \in [2, 13]. \)

* This is the solution!
\end{enumerate}

\textbf{General Comment:} Be sure to synthetically divide by the zero of the denominator! Also, make sure to include 0 placeholders for missing terms.
}
\litem{
Factor the polynomial below completely. Then, choose the intervals the zeros of the polynomial belong to, where $z_1 \leq z_2 \leq z_3$. \textit{To make the problem easier, all zeros are between -5 and 5.}
\[ f(x) = 20x^{3} -33 x^{2} -20 x + 12 \]The solution is \( [-0.75, 0.4, 2] \), which is option B.\begin{enumerate}[label=\Alph*.]
\item \( z_1 \in [-2.02, -1.65], \text{   }  z_2 \in [-2.77, -1.28], \text{   and   } z_3 \in [0.09, 0.38] \)

 Distractor 4: Corresponds to moving factors from one rational to another.
\item \( z_1 \in [-1.2, -0.31], \text{   }  z_2 \in [0.22, 0.44], \text{   and   } z_3 \in [1.92, 2.22] \)

* This is the solution!
\item \( z_1 \in [-1.63, -1.11], \text{   }  z_2 \in [1.83, 2.91], \text{   and   } z_3 \in [2.28, 2.58] \)

 Distractor 2: Corresponds to inversing rational roots.
\item \( z_1 \in [-2.55, -2.31], \text{   }  z_2 \in [-2.77, -1.28], \text{   and   } z_3 \in [1.1, 1.38] \)

 Distractor 3: Corresponds to negatives of all zeros AND inversing rational roots.
\item \( z_1 \in [-2.02, -1.65], \text{   }  z_2 \in [-0.52, -0.21], \text{   and   } z_3 \in [0.69, 0.98] \)

 Distractor 1: Corresponds to negatives of all zeros.
\end{enumerate}

\textbf{General Comment:} Remember to try the middle-most integers first as these normally are the zeros. Also, once you get it to a quadratic, you can use your other factoring techniques to finish factoring.
}
\litem{
Perform the division below. Then, find the intervals that correspond to the quotient in the form $ax^2+bx+c$ and remainder $r$.
\[ \frac{15x^{3} +67 x^{2} +94 x + 35}{x + 2} \]The solution is \( 15x^{2} +37 x + 20 + \frac{-5}{x + 2} \), which is option A.\begin{enumerate}[label=\Alph*.]
\item \( a \in [13, 18], \text{   } b \in [37, 39], \text{   } c \in [16, 24], \text{   and   } r \in [-11, -3]. \)

* This is the solution!
\item \( a \in [-31, -28], \text{   } b \in [6, 13], \text{   } c \in [104, 113], \text{   and   } r \in [251, 257]. \)

 You divided by the opposite of the factor AND multiplied the first factor rather than just bringing it down.
\item \( a \in [-31, -28], \text{   } b \in [125, 129], \text{   } c \in [-161, -159], \text{   and   } r \in [354, 357]. \)

 You multiplied by the synthetic number rather than bringing the first factor down.
\item \( a \in [13, 18], \text{   } b \in [92, 101], \text{   } c \in [284, 289], \text{   and   } r \in [606, 615]. \)

 You divided by the opposite of the factor.
\item \( a \in [13, 18], \text{   } b \in [20, 23], \text{   } c \in [24, 34], \text{   and   } r \in [-50, -46]. \)

 You multiplied by the synthetic number and subtracted rather than adding during synthetic division.
\end{enumerate}

\textbf{General Comment:} Be sure to synthetically divide by the zero of the denominator!
}
\litem{
What are the \textit{possible Integer} roots of the polynomial below?
\[ f(x) = 6x^{4} +4 x^{3} +7 x^{2} +4 x + 7 \]The solution is \( \pm 1,\pm 7 \), which is option B.\begin{enumerate}[label=\Alph*.]
\item \( \text{ All combinations of: }\frac{\pm 1,\pm 2,\pm 3,\pm 6}{\pm 1,\pm 7} \)

 Distractor 3: Corresponds to the plus or minus of the inverse quotient (an/a0) of the factors. 
\item \( \pm 1,\pm 7 \)

* This is the solution \textbf{since we asked for the possible Integer roots}!
\item \( \pm 1,\pm 2,\pm 3,\pm 6 \)

 Distractor 1: Corresponds to the plus or minus factors of a1 only.
\item \( \text{ All combinations of: }\frac{\pm 1,\pm 7}{\pm 1,\pm 2,\pm 3,\pm 6} \)

This would have been the solution \textbf{if asked for the possible Rational roots}!
\item \( \text{There is no formula or theorem that tells us all possible Integer roots.} \)

 Distractor 4: Corresponds to not recognizing Integers as a subset of Rationals.
\end{enumerate}

\textbf{General Comment:} We have a way to find the possible Rational roots. The possible Integer roots are the Integers in this list.
}
\end{enumerate}

\end{document}