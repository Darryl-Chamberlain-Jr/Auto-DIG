\documentclass{extbook}[14pt]
\usepackage{multicol, enumerate, enumitem, hyperref, color, soul, setspace, parskip, fancyhdr, amssymb, amsthm, amsmath, latexsym, units, mathtools}
\everymath{\displaystyle}
\usepackage[headsep=0.5cm,headheight=0cm, left=1 in,right= 1 in,top= 1 in,bottom= 1 in]{geometry}
\usepackage{dashrule}  % Package to use the command below to create lines between items
\newcommand{\litem}[1]{\item #1

\rule{\textwidth}{0.4pt}}
\pagestyle{fancy}
\lhead{}
\chead{Answer Key for Module1 Version A}
\rhead{}
\lfoot{8000-1344}
\cfoot{}
\rfoot{testing}
\begin{document}
\textbf{This key should allow you to understand why you choose the option you did (beyond just getting a question right or wrong). \href{https://xronos.clas.ufl.edu/mac1105spring2020/courseDescriptionAndMisc/Exams/LearningFromResults}{More instructions on how to use this key can be found here}.}

\textbf{If you have a suggestion to make the keys better, \href{https://forms.gle/CZkbZmPbC9XALEE88}{please fill out the short survey here}.}

\textit{Note: This key is auto-generated and may contain issues and/or errors. The keys are reviewed after each exam to ensure grading is done accurately. If there are issues (like duplicate options), they are noted in the offline gradebook. The keys are a work-in-progress to give students as many resources to improve as possible.}

\rule{\textwidth}{0.4pt}

\begin{enumerate}\litem{
Choose the \textbf{smallest} set of Real numbers that the number below belongs to.
\[ -\sqrt{\frac{49}{64}} \]The solution is \( \text{Rational} \), which is option A.\begin{enumerate}[label=\Alph*.]
\item \( \text{Rational} \)

* This is the correct option!
\item \( \text{Integer} \)

These are the negative and positive counting numbers (..., -3, -2, -1, 0, 1, 2, 3, ...)
\item \( \text{Irrational} \)

These cannot be written as a fraction of Integers.
\item \( \text{Not a Real number} \)

These are Nonreal Complex numbers \textbf{OR} things that are not numbers (e.g., dividing by 0).
\item \( \text{Whole} \)

These are the counting numbers with 0 (0, 1, 2, 3, ...)
\end{enumerate}

\textbf{General Comment:} First, you \textbf{NEED} to simplify the expression. This question simplifies to $-\frac{7}{8}$. 
 
 Be sure you look at the simplified fraction and not just the decimal expansion. Numbers such as 13, 17, and 19 provide \textbf{long but repeating/terminating decimal expansions!} 
 
 The only ways to *not* be a Real number are: dividing by 0 or taking the square root of a negative number. 
 
 Irrational numbers are more than just square root of 3: adding or subtracting values from square root of 3 is also irrational.
}
\litem{
Simplify the expression below into the form $a+bi$. Then, choose the intervals that $a$ and $b$ belong to.
\[ (-2 + 4 i)(-5 - 7 i) \]The solution is \( 38 - 6 i \), which is option C.\begin{enumerate}[label=\Alph*.]
\item \( a \in [-23, -12] \text{ and } b \in [34, 39] \)

 $-18 + 34 i$, which corresponds to adding a minus sign in the first term.
\item \( a \in [3, 17] \text{ and } b \in [-33, -21] \)

 $10 - 28 i$, which corresponds to just multiplying the real terms to get the real part of the solution and the coefficients in the complex terms to get the complex part.
\item \( a \in [30, 42] \text{ and } b \in [-7, 3] \)

* $38 - 6 i$, which is the correct option.
\item \( a \in [-23, -12] \text{ and } b \in [-37, -33] \)

 $-18 - 34 i$, which corresponds to adding a minus sign in the second term.
\item \( a \in [30, 42] \text{ and } b \in [5, 7] \)

 $38 + 6 i$, which corresponds to adding a minus sign in both terms.
\end{enumerate}

\textbf{General Comment:} You can treat $i$ as a variable and distribute. Just remember that $i^2=-1$, so you can continue to reduce after you distribute.
}
\litem{
Simplify the expression below into the form $a+bi$. Then, choose the intervals that $a$ and $b$ belong to.
\[ \frac{-9 - 55 i}{4 - 2 i} \]The solution is \( 3.70  - 11.90 i \), which is option E.\begin{enumerate}[label=\Alph*.]
\item \( a \in [-7.5, -6] \text{ and } b \in [-11.5, -9.5] \)

 $-7.30  - 10.10 i$, which corresponds to forgetting to multiply the conjugate by the numerator and not computing the conjugate correctly.
\item \( a \in [3, 4] \text{ and } b \in [-239, -237] \)

 $3.70  - 238.00 i$, which corresponds to forgetting to multiply the conjugate by the numerator.
\item \( a \in [-3, -1] \text{ and } b \in [26.5, 28] \)

 $-2.25  + 27.50 i$, which corresponds to just dividing the first term by the first term and the second by the second.
\item \( a \in [73.5, 74.5] \text{ and } b \in [-12.5, -11] \)

 $74.00  - 11.90 i$, which corresponds to forgetting to multiply the conjugate by the numerator and using a plus instead of a minus in the denominator.
\item \( a \in [3, 4] \text{ and } b \in [-12.5, -11] \)

* $3.70  - 11.90 i$, which is the correct option.
\end{enumerate}

\textbf{General Comment:} Multiply the numerator and denominator by the *conjugate* of the denominator, then simplify. For example, if we have $2+3i$, the conjugate is $2-3i$.
}
\litem{
Simplify the expression below and choose the interval the simplification is contained within.
\[ 1 - 7^2 + 13 \div 4 * 3 \div 18 \]The solution is \( -47.458 \), which is option A.\begin{enumerate}[label=\Alph*.]
\item \( [-47.66, -47.17] \)

* -47.458, this is the correct option
\item \( [-48.06, -47.92] \)

 -47.940, which corresponds to an Order of Operations error: not reading left-to-right for multiplication/division.
\item \( [50.53, 51.09] \)

 50.542, which corresponds to an Order of Operations error: multiplying by negative before squaring. For example: $(-3)^2 \neq -3^2$
\item \( [49.85, 50.46] \)

 50.060, which corresponds to two Order of Operations errors.
\item \( \text{None of the above} \)

 You may have gotten this by making an unanticipated error. If you got a value that is not any of the others, please let the coordinator know so they can help you figure out what happened.
\end{enumerate}

\textbf{General Comment:} While you may remember (or were taught) PEMDAS is done in order, it is actually done as P/E/MD/AS. When we are at MD or AS, we read left to right.
}
\litem{
Choose the \textbf{smallest} set of Complex numbers that the number below belongs to.
\[ \sqrt{\frac{0}{6}}+\sqrt{10}i \]The solution is \( \text{Pure Imaginary} \), which is option E.\begin{enumerate}[label=\Alph*.]
\item \( \text{Irrational} \)

These cannot be written as a fraction of Integers. Remember: $\pi$ is not an Integer!
\item \( \text{Rational} \)

These are numbers that can be written as fraction of Integers (e.g., -2/3 + 5)
\item \( \text{Not a Complex Number} \)

This is not a number. The only non-Complex number we know is dividing by 0 as this is not a number!
\item \( \text{Nonreal Complex} \)

This is a Complex number $(a+bi)$ that is not Real (has $i$ as part of the number).
\item \( \text{Pure Imaginary} \)

* This is the correct option!
\end{enumerate}

\textbf{General Comment:} Be sure to simplify $i^2 = -1$. This may remove the imaginary portion for your number. If you are having trouble, you may want to look at the \textit{Subgroups of the Real Numbers} section.
}
\litem{
Simplify the expression below into the form $a+bi$. Then, choose the intervals that $a$ and $b$ belong to.
\[ \frac{27 + 22 i}{-8 + 7 i} \]The solution is \( -0.55  - 3.23 i \), which is option C.\begin{enumerate}[label=\Alph*.]
\item \( a \in [-3.75, -3.31] \text{ and } b \in [2, 4] \)

 $-3.38  + 3.14 i$, which corresponds to just dividing the first term by the first term and the second by the second.
\item \( a \in [-62.08, -61.8] \text{ and } b \in [-4, -2] \)

 $-62.00  - 3.23 i$, which corresponds to forgetting to multiply the conjugate by the numerator and using a plus instead of a minus in the denominator.
\item \( a \in [-0.76, -0.45] \text{ and } b \in [-4, -2] \)

* $-0.55  - 3.23 i$, which is the correct option.
\item \( a \in [-0.76, -0.45] \text{ and } b \in [-366, -364.5] \)

 $-0.55  - 365.00 i$, which corresponds to forgetting to multiply the conjugate by the numerator.
\item \( a \in [-3.33, -3.24] \text{ and } b \in [-0.5, 0.5] \)

 $-3.27  + 0.12 i$, which corresponds to forgetting to multiply the conjugate by the numerator and not computing the conjugate correctly.
\end{enumerate}

\textbf{General Comment:} Multiply the numerator and denominator by the *conjugate* of the denominator, then simplify. For example, if we have $2+3i$, the conjugate is $2-3i$.
}
\litem{
Choose the \textbf{smallest} set of Real numbers that the number below belongs to.
\[ \sqrt{\frac{-720}{8}} \]The solution is \( \text{Not a Real number} \), which is option D.\begin{enumerate}[label=\Alph*.]
\item \( \text{Whole} \)

These are the counting numbers with 0 (0, 1, 2, 3, ...)
\item \( \text{Rational} \)

These are numbers that can be written as fraction of Integers (e.g., -2/3)
\item \( \text{Irrational} \)

These cannot be written as a fraction of Integers.
\item \( \text{Not a Real number} \)

* This is the correct option!
\item \( \text{Integer} \)

These are the negative and positive counting numbers (..., -3, -2, -1, 0, 1, 2, 3, ...)
\end{enumerate}

\textbf{General Comment:} First, you \textbf{NEED} to simplify the expression. This question simplifies to $\sqrt{90} i$. 
 
 Be sure you look at the simplified fraction and not just the decimal expansion. Numbers such as 13, 17, and 19 provide \textbf{long but repeating/terminating decimal expansions!} 
 
 The only ways to *not* be a Real number are: dividing by 0 or taking the square root of a negative number. 
 
 Irrational numbers are more than just square root of 3: adding or subtracting values from square root of 3 is also irrational.
}
\litem{
Simplify the expression below into the form $a+bi$. Then, choose the intervals that $a$ and $b$ belong to.
\[ (-9 + 2 i)(5 + 7 i) \]The solution is \( -59 - 53 i \), which is option D.\begin{enumerate}[label=\Alph*.]
\item \( a \in [-37, -27] \text{ and } b \in [-76, -72] \)

 $-31 - 73 i$, which corresponds to adding a minus sign in the first term.
\item \( a \in [-48, -44] \text{ and } b \in [14, 23] \)

 $-45 + 14 i$, which corresponds to just multiplying the real terms to get the real part of the solution and the coefficients in the complex terms to get the complex part.
\item \( a \in [-62, -58] \text{ and } b \in [51, 59] \)

 $-59 + 53 i$, which corresponds to adding a minus sign in both terms.
\item \( a \in [-62, -58] \text{ and } b \in [-55, -50] \)

* $-59 - 53 i$, which is the correct option.
\item \( a \in [-37, -27] \text{ and } b \in [73, 75] \)

 $-31 + 73 i$, which corresponds to adding a minus sign in the second term.
\end{enumerate}

\textbf{General Comment:} You can treat $i$ as a variable and distribute. Just remember that $i^2=-1$, so you can continue to reduce after you distribute.
}
\litem{
Choose the \textbf{smallest} set of Complex numbers that the number below belongs to.
\[ \sqrt{\frac{-2028}{13}}+\sqrt{0}i \]The solution is \( \text{Pure Imaginary} \), which is option B.\begin{enumerate}[label=\Alph*.]
\item \( \text{Nonreal Complex} \)

This is a Complex number $(a+bi)$ that is not Real (has $i$ as part of the number).
\item \( \text{Pure Imaginary} \)

* This is the correct option!
\item \( \text{Not a Complex Number} \)

This is not a number. The only non-Complex number we know is dividing by 0 as this is not a number!
\item \( \text{Irrational} \)

These cannot be written as a fraction of Integers. Remember: $\pi$ is not an Integer!
\item \( \text{Rational} \)

These are numbers that can be written as fraction of Integers (e.g., -2/3 + 5)
\end{enumerate}

\textbf{General Comment:} Be sure to simplify $i^2 = -1$. This may remove the imaginary portion for your number. If you are having trouble, you may want to look at the \textit{Subgroups of the Real Numbers} section.
}
\litem{
Simplify the expression below and choose the interval the simplification is contained within.
\[ 20 - 4^2 + 1 \div 16 * 11 \div 5 \]The solution is \( 4.138 \), which is option C.\begin{enumerate}[label=\Alph*.]
\item \( [35.77, 36.02] \)

 36.001, which corresponds to two Order of Operations errors.
\item \( [36.1, 36.28] \)

 36.138, which corresponds to an Order of Operations error: multiplying by negative before squaring. For example: $(-3)^2 \neq -3^2$
\item \( [4.12, 4.15] \)

* 4.138, this is the correct option
\item \( [3.99, 4.07] \)

 4.001, which corresponds to an Order of Operations error: not reading left-to-right for multiplication/division.
\item \( \text{None of the above} \)

 You may have gotten this by making an unanticipated error. If you got a value that is not any of the others, please let the coordinator know so they can help you figure out what happened.
\end{enumerate}

\textbf{General Comment:} While you may remember (or were taught) PEMDAS is done in order, it is actually done as P/E/MD/AS. When we are at MD or AS, we read left to right.
}
\end{enumerate}

\end{document}