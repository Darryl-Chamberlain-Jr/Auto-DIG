\documentclass{extbook}[14pt]
\usepackage{multicol, enumerate, enumitem, hyperref, color, soul, setspace, parskip, fancyhdr, amssymb, amsthm, amsmath, bbm, latexsym, units, mathtools}
\everymath{\displaystyle}
\usepackage[headsep=0.5cm,headheight=0cm, left=1 in,right= 1 in,top= 1 in,bottom= 1 in]{geometry}
\usepackage{dashrule}  % Package to use the command below to create lines between items
\newcommand{\litem}[1]{\item #1

\rule{\textwidth}{0.4pt}}
\pagestyle{fancy}
\lhead{}
\chead{Answer Key for Module1 Version B}
\rhead{}
\lfoot{4565-2610}
\cfoot{}
\rfoot{Fall 2020}
\begin{document}
\textbf{This key should allow you to understand why you choose the option you did (beyond just getting a question right or wrong). \href{https://xronos.clas.ufl.edu/mac1105spring2020/courseDescriptionAndMisc/Exams/LearningFromResults}{More instructions on how to use this key can be found here}.}

\textbf{If you have a suggestion to make the keys better, \href{https://forms.gle/CZkbZmPbC9XALEE88}{please fill out the short survey here}.}

\textit{Note: This key is auto-generated and may contain issues and/or errors. The keys are reviewed after each exam to ensure grading is done accurately. If there are issues (like duplicate options), they are noted in the offline gradebook. The keys are a work-in-progress to give students as many resources to improve as possible.}

\rule{\textwidth}{0.4pt}

\begin{enumerate}\litem{
Choose the \textbf{smallest} set of Complex numbers that the number below belongs to.
\[ \frac{2}{-17}+4i^2 \]
The solution is \( \text{Rational} \), which is option C.\begin{enumerate}[label=\Alph*.]
\item \( \text{Pure Imaginary} \)

This is a Complex number $(a+bi)$ that \textbf{only} has an imaginary part like $2i$.
\item \( \text{Nonreal Complex} \)

This is a Complex number $(a+bi)$ that is not Real (has $i$ as part of the number).
\item \( \text{Rational} \)

* This is the correct option!
\item \( \text{Irrational} \)

These cannot be written as a fraction of Integers. Remember: $\pi$ is not an Integer!
\item \( \text{Not a Complex Number} \)

This is not a number. The only non-Complex number we know is dividing by 0 as this is not a number!
\end{enumerate}

\textbf{General Comment:} Be sure to simplify $i^2 = -1$. This may remove the imaginary portion for your number. If you are having trouble, you may want to look at the \textit{Subgroups of the Real Numbers} section.
}
\litem{
Simplify the expression below into the form $a+bi$. Then, choose the intervals that $a$ and $b$ belong to.
\[ \frac{-18 - 44 i}{5 + 3 i} \]
The solution is \( -6.53  - 4.88 i \), which is option E.\begin{enumerate}[label=\Alph*.]
\item \( a \in [0.5, 2.5] \text{ and } b \in [-9, -8] \)

 $1.24  - 8.06 i$, which corresponds to forgetting to multiply the conjugate by the numerator and not computing the conjugate correctly.
\item \( a \in [-7, -6.5] \text{ and } b \in [-166.5, -165.5] \)

 $-6.53  - 166.00 i$, which corresponds to forgetting to multiply the conjugate by the numerator.
\item \( a \in [-4, -3] \text{ and } b \in [-15.5, -13.5] \)

 $-3.60  - 14.67 i$, which corresponds to just dividing the first term by the first term and the second by the second.
\item \( a \in [-222.5, -221] \text{ and } b \in [-5, -4.5] \)

 $-222.00  - 4.88 i$, which corresponds to forgetting to multiply the conjugate by the numerator and using a plus instead of a minus in the denominator.
\item \( a \in [-7, -6.5] \text{ and } b \in [-5, -4.5] \)

* $-6.53  - 4.88 i$, which is the correct option.
\end{enumerate}

\textbf{General Comment:} Multiply the numerator and denominator by the *conjugate* of the denominator, then simplify. For example, if we have $2+3i$, the conjugate is $2-3i$.
}
\litem{
Simplify the expression below into the form $a+bi$. Then, choose the intervals that $a$ and $b$ belong to.
\[ (8 - 7 i)(6 + 5 i) \]
The solution is \( 83 - 2 i \), which is option B.\begin{enumerate}[label=\Alph*.]
\item \( a \in [47, 52] \text{ and } b \in [-36, -33.4] \)

 $48 - 35 i$, which corresponds to just multiplying the real terms to get the real part of the solution and the coefficients in the complex terms to get the complex part.
\item \( a \in [82, 86] \text{ and } b \in [-2.2, -1.4] \)

* $83 - 2 i$, which is the correct option.
\item \( a \in [82, 86] \text{ and } b \in [0.2, 4.5] \)

 $83 + 2 i$, which corresponds to adding a minus sign in both terms.
\item \( a \in [12, 16] \text{ and } b \in [-84.2, -80.6] \)

 $13 - 82 i$, which corresponds to adding a minus sign in the second term.
\item \( a \in [12, 16] \text{ and } b \in [80.5, 84.8] \)

 $13 + 82 i$, which corresponds to adding a minus sign in the first term.
\end{enumerate}

\textbf{General Comment:} You can treat $i$ as a variable and distribute. Just remember that $i^2=-1$, so you can continue to reduce after you distribute.
}
\litem{
Choose the \textbf{smallest} set of Complex numbers that the number below belongs to.
\[ \sqrt{\frac{-2618}{0}} i+\sqrt{198}i \]
The solution is \( \text{Not a Complex Number} \), which is option B.\begin{enumerate}[label=\Alph*.]
\item \( \text{Pure Imaginary} \)

This is a Complex number $(a+bi)$ that \textbf{only} has an imaginary part like $2i$.
\item \( \text{Not a Complex Number} \)

* This is the correct option!
\item \( \text{Rational} \)

These are numbers that can be written as fraction of Integers (e.g., -2/3 + 5)
\item \( \text{Nonreal Complex} \)

This is a Complex number $(a+bi)$ that is not Real (has $i$ as part of the number).
\item \( \text{Irrational} \)

These cannot be written as a fraction of Integers. Remember: $\pi$ is not an Integer!
\end{enumerate}

\textbf{General Comment:} Be sure to simplify $i^2 = -1$. This may remove the imaginary portion for your number. If you are having trouble, you may want to look at the \textit{Subgroups of the Real Numbers} section.
}
\litem{
Simplify the expression below into the form $a+bi$. Then, choose the intervals that $a$ and $b$ belong to.
\[ (-7 + 3 i)(-2 + 8 i) \]
The solution is \( -10 - 62 i \), which is option B.\begin{enumerate}[label=\Alph*.]
\item \( a \in [11, 16] \text{ and } b \in [20, 29] \)

 $14 + 24 i$, which corresponds to just multiplying the real terms to get the real part of the solution and the coefficients in the complex terms to get the complex part.
\item \( a \in [-14, -8] \text{ and } b \in [-68, -60] \)

* $-10 - 62 i$, which is the correct option.
\item \( a \in [36, 41] \text{ and } b \in [-52, -44] \)

 $38 - 50 i$, which corresponds to adding a minus sign in the first term.
\item \( a \in [-14, -8] \text{ and } b \in [62, 65] \)

 $-10 + 62 i$, which corresponds to adding a minus sign in both terms.
\item \( a \in [36, 41] \text{ and } b \in [50, 52] \)

 $38 + 50 i$, which corresponds to adding a minus sign in the second term.
\end{enumerate}

\textbf{General Comment:} You can treat $i$ as a variable and distribute. Just remember that $i^2=-1$, so you can continue to reduce after you distribute.
}
\litem{
Simplify the expression below into the form $a+bi$. Then, choose the intervals that $a$ and $b$ belong to.
\[ \frac{72 + 11 i}{-4 - 3 i} \]
The solution is \( -12.84  + 6.88 i \), which is option A.\begin{enumerate}[label=\Alph*.]
\item \( a \in [-14, -11.5] \text{ and } b \in [6.5, 8] \)

* $-12.84  + 6.88 i$, which is the correct option.
\item \( a \in [-14, -11.5] \text{ and } b \in [171, 172.5] \)

 $-12.84  + 172.00 i$, which corresponds to forgetting to multiply the conjugate by the numerator.
\item \( a \in [-10.5, -9] \text{ and } b \in [-12, -10] \)

 $-10.20  - 10.40 i$, which corresponds to forgetting to multiply the conjugate by the numerator and not computing the conjugate correctly.
\item \( a \in [-18.5, -17] \text{ and } b \in [-4.5, -2.5] \)

 $-18.00  - 3.67 i$, which corresponds to just dividing the first term by the first term and the second by the second.
\item \( a \in [-322, -320.5] \text{ and } b \in [6.5, 8] \)

 $-321.00  + 6.88 i$, which corresponds to forgetting to multiply the conjugate by the numerator and using a plus instead of a minus in the denominator.
\end{enumerate}

\textbf{General Comment:} Multiply the numerator and denominator by the *conjugate* of the denominator, then simplify. For example, if we have $2+3i$, the conjugate is $2-3i$.
}
\litem{
Simplify the expression below and choose the interval the simplification is contained within.
\[ 4 - 8^2 + 13 \div 15 * 3 \div 11 \]
The solution is \( -59.764 \), which is option B.\begin{enumerate}[label=\Alph*.]
\item \( [67.88, 68.05] \)

 68.026, which corresponds to two Order of Operations errors.
\item \( [-59.91, -59.61] \)

* -59.764, this is the correct option
\item \( [-60.43, -59.89] \)

 -59.974, which corresponds to an Order of Operations error: not reading left-to-right for multiplication/division.
\item \( [68.2, 68.52] \)

 68.236, which corresponds to an Order of Operations error: multiplying by negative before squaring. For example: $(-3)^2 \neq -3^2$
\item \( \text{None of the above} \)

 You may have gotten this by making an unanticipated error. If you got a value that is not any of the others, please let the coordinator know so they can help you figure out what happened.
\end{enumerate}

\textbf{General Comment:} While you may remember (or were taught) PEMDAS is done in order, it is actually done as P/E/MD/AS. When we are at MD or AS, we read left to right.
}
\litem{
Choose the \textbf{smallest} set of Real numbers that the number below belongs to.
\[ \sqrt{\frac{400}{441}} \]
The solution is \( \text{Rational} \), which is option D.\begin{enumerate}[label=\Alph*.]
\item \( \text{Not a Real number} \)

These are Nonreal Complex numbers \textbf{OR} things that are not numbers (e.g., dividing by 0).
\item \( \text{Irrational} \)

These cannot be written as a fraction of Integers.
\item \( \text{Whole} \)

These are the counting numbers with 0 (0, 1, 2, 3, ...)
\item \( \text{Rational} \)

* This is the correct option!
\item \( \text{Integer} \)

These are the negative and positive counting numbers (..., -3, -2, -1, 0, 1, 2, 3, ...)
\end{enumerate}

\textbf{General Comment:} First, you \textbf{NEED} to simplify the expression. This question simplifies to $\frac{20}{21}$. 
 
 Be sure you look at the simplified fraction and not just the decimal expansion. Numbers such as 13, 17, and 19 provide \textbf{long but repeating/terminating decimal expansions!} 
 
 The only ways to *not* be a Real number are: dividing by 0 or taking the square root of a negative number. 
 
 Irrational numbers are more than just square root of 3: adding or subtracting values from square root of 3 is also irrational.
}
\litem{
Simplify the expression below and choose the interval the simplification is contained within.
\[ 7 - 10 \div 19 * 11 - (4 * 9) \]
The solution is \( -34.789 \), which is option D.\begin{enumerate}[label=\Alph*.]
\item \( [-26, -20.6] \)

 -25.105, which corresponds to not distributing a negative correctly.
\item \( [41.2, 43.1] \)

 42.952, which corresponds to not distributing addition and subtraction correctly.
\item \( [-31.2, -28.4] \)

 -29.048, which corresponds to an Order of Operations error: not reading left-to-right for multiplication/division.
\item \( [-39.6, -33.1] \)

* -34.789, which is the correct option.
\item \( \text{None of the above} \)

 You may have gotten this by making an unanticipated error. If you got a value that is not any of the others, please let the coordinator know so they can help you figure out what happened.
\end{enumerate}

\textbf{General Comment:} While you may remember (or were taught) PEMDAS is done in order, it is actually done as P/E/MD/AS. When we are at MD or AS, we read left to right.
}
\litem{
Choose the \textbf{smallest} set of Real numbers that the number below belongs to.
\[ -\sqrt{\frac{1001}{7}} \]
The solution is \( \text{Irrational} \), which is option B.\begin{enumerate}[label=\Alph*.]
\item \( \text{Whole} \)

These are the counting numbers with 0 (0, 1, 2, 3, ...)
\item \( \text{Irrational} \)

* This is the correct option!
\item \( \text{Not a Real number} \)

These are Nonreal Complex numbers \textbf{OR} things that are not numbers (e.g., dividing by 0).
\item \( \text{Rational} \)

These are numbers that can be written as fraction of Integers (e.g., -2/3)
\item \( \text{Integer} \)

These are the negative and positive counting numbers (..., -3, -2, -1, 0, 1, 2, 3, ...)
\end{enumerate}

\textbf{General Comment:} First, you \textbf{NEED} to simplify the expression. This question simplifies to $-\sqrt{143}$. 
 
 Be sure you look at the simplified fraction and not just the decimal expansion. Numbers such as 13, 17, and 19 provide \textbf{long but repeating/terminating decimal expansions!} 
 
 The only ways to *not* be a Real number are: dividing by 0 or taking the square root of a negative number. 
 
 Irrational numbers are more than just square root of 3: adding or subtracting values from square root of 3 is also irrational.
}
\end{enumerate}

\end{document}