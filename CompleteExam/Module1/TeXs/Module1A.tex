\documentclass[14pt]{extbook}
\usepackage{multicol, enumerate, enumitem, hyperref, color, soul, setspace, parskip, fancyhdr} %General Packages
\usepackage{amssymb, amsthm, amsmath, latexsym, units, mathtools} %Math Packages
\everymath{\displaystyle} %All math in Display Style
% Packages with additional options
\usepackage[headsep=0.5cm,headheight=12pt, left=1 in,right= 1 in,top= 1 in,bottom= 1 in]{geometry}
\usepackage[usenames,dvipsnames]{xcolor}
\usepackage{dashrule}  % Package to use the command below to create lines between items
\newcommand{\litem}[1]{\item#1\hspace*{-1cm}\rule{\textwidth}{0.4pt}}
\pagestyle{fancy}
\lhead{Module1}
\chead{}
\rhead{Version A}
\lfoot{4041-3414}
\cfoot{}
\rfoot{test}
\begin{document}

\begin{enumerate}
\item{
Simplify the expression below into the form $a+bi$.\[ \frac{27 + 88 i}{5 - 4 i} \]\newpage\end{enumerate} }
\item{
Simplify the expression below into the form $a+bi$.\[ \frac{-27 + 55 i}{-4 + 2 i} \]\newpage\end{enumerate} }
\item{
What is the \textbf{smallest} set of Complex numbers that the number below belongs to?\[ \sqrt{\frac{715}{11}}+6i^2 \]\newpage\end{enumerate} }
\item{
Simplify the expression below into the form $a+bi$.\[ (7 + 6 i)(5 + 4 i) \]\newpage\end{enumerate} }
\item{
What is the \textbf{smallest} set of Real numbers that the number below belongs to?\[ \sqrt{\frac{1872}{8}} \]\newpage\end{enumerate} }
\item{
What is the \textbf{smallest} set of Complex numbers that the number below belongs to?\[ \sqrt{\frac{0}{6}}+\sqrt{6}i \]\newpage\end{enumerate} }
\item{
Simplify the expression below into the form $a+bi$.\[ (-7 - 10 i)(2 + 9 i) \]\newpage\end{enumerate} }
\item{
Simplify the expression below.\[ 20 - 1 \div 19 * 13 - (2 * 3) \]\newpage\end{enumerate} }
\item{
Simplify the expression below.\[ 2 - 7^2 + 9 \div 16 * 13 \div 4 \]\newpage\end{enumerate} }
\item{
What is the \textbf{smallest} set of Real numbers that the number below belongs to?\[ -\sqrt{\frac{19}{0}} \]\newpage\end{enumerate} }
\end{enumerate}

\end{document}