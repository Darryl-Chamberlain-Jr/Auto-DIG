\documentclass{extbook}[14pt]
\usepackage{multicol, enumerate, enumitem, hyperref, color, soul, setspace, parskip, fancyhdr, amssymb, amsthm, amsmath, latexsym, units, mathtools}
\everymath{\displaystyle}
\usepackage[headsep=0.5cm,headheight=0cm, left=1 in,right= 1 in,top= 1 in,bottom= 1 in]{geometry}
\usepackage{dashrule}  % Package to use the command below to create lines between items
\newcommand{\litem}[1]{\item #1

\rule{\textwidth}{0.4pt}}
\pagestyle{fancy}
\lhead{}
\chead{Answer Key for Module1 Version ALL}
\rhead{}
\lfoot{4041-3414}
\cfoot{}
\rfoot{test}
\begin{document}
\textbf{This key should allow you to understand why you choose the option you did (beyond just getting a question right or wrong). \href{https://xronos.clas.ufl.edu/mac1105spring2020/courseDescriptionAndMisc/Exams/LearningFromResults}{More instructions on how to use this key can be found here}.}

\textbf{If you have a suggestion to make the keys better, \href{https://forms.gle/CZkbZmPbC9XALEE88}{please fill out the short survey here}.}

\textit{Note: This key is auto-generated and may contain issues and/or errors. The keys are reviewed after each exam to ensure grading is done accurately. If there are issues (like duplicate options), they are noted in the offline gradebook. The keys are a work-in-progress to give students as many resources to improve as possible.}

\rule{\textwidth}{0.4pt}

\begin{enumerate}\litem{
Simplify the expression below into the form $a+bi$.
\[ \frac{27 + 88 i}{5 - 4 i} \]The solution is \( -5.29  + 13.37 i \).\begin{enumerate}[label=\Alph*.]
\textbf{Plausible alternative answers include:} $-5.29  + 548.00 i$, which corresponds to forgetting to multiply the conjugate by the numerator.
 $5.40  - 22.00 i$, which corresponds to just dividing the first term by the first term and the second by the second.
* $-5.29  + 13.37 i$, which is the correct option.
 $11.88  + 8.10 i$, which corresponds to forgetting to multiply the conjugate by the numerator and not computing the conjugate correctly.
 $-217.00  + 13.37 i$, which corresponds to forgetting to multiply the conjugate by the numerator and using a plus instead of a minus in the denominator.
\end{enumerate}

\textbf{General Comment:} Multiply the numerator and denominator by the *conjugate* of the denominator, then simplify. For example, if we have $2+3i$, the conjugate is $2-3i$.
}
\litem{
Simplify the expression below into the form $a+bi$.
\[ \frac{-27 + 55 i}{-4 + 2 i} \]The solution is \( 10.90  - 8.30 i \).\begin{enumerate}[label=\Alph*.]
\textbf{Plausible alternative answers include:} $-0.10  - 13.70 i$, which corresponds to forgetting to multiply the conjugate by the numerator and not computing the conjugate correctly.
 $6.75  + 27.50 i$, which corresponds to just dividing the first term by the first term and the second by the second.
 $10.90  - 166.00 i$, which corresponds to forgetting to multiply the conjugate by the numerator.
* $10.90  - 8.30 i$, which is the correct option.
 $218.00  - 8.30 i$, which corresponds to forgetting to multiply the conjugate by the numerator and using a plus instead of a minus in the denominator.
\end{enumerate}

\textbf{General Comment:} Multiply the numerator and denominator by the *conjugate* of the denominator, then simplify. For example, if we have $2+3i$, the conjugate is $2-3i$.
}
\litem{
What is the \textbf{smallest} set of Complex numbers that the number below belongs to?
\[ \sqrt{\frac{715}{11}}+6i^2 \]The solution is \( \text{Irrational} \).\begin{enumerate}[label=\Alph*.]
\textbf{Plausible alternative answers include:}This is not a number. The only non-Complex number we know is dividing by 0 as this is not a number!
These are numbers that can be written as fraction of Integers (e.g., -2/3 + 5)
* This is the correct option!
This is a Complex number $(a+bi)$ that is not Real (has $i$ as part of the number).
This is a Complex number $(a+bi)$ that \textbf{only} has an imaginary part like $2i$.
\end{enumerate}

\textbf{General Comment:} Be sure to simplify $i^2 = -1$. This may remove the imaginary portion for your number. If you are having trouble, you may want to look at the \textit{Subgroups of the Real Numbers} section.
}
\litem{
Simplify the expression below into the form $a+bi$.
\[ (7 + 6 i)(5 + 4 i) \]The solution is \( 11 + 58 i \).\begin{enumerate}[label=\Alph*.]
\textbf{Plausible alternative answers include:} $11 - 58 i$, which corresponds to adding a minus sign in both terms.
 $35 + 24 i$, which corresponds to just multiplying the real terms to get the real part of the solution and the coefficients in the complex terms to get the complex part.
 $59 - 2 i$, which corresponds to adding a minus sign in the first term.
* $11 + 58 i$, which is the correct option.
 $59 + 2 i$, which corresponds to adding a minus sign in the second term.
\end{enumerate}

\textbf{General Comment:} You can treat $i$ as a variable and distribute. Just remember that $i^2=-1$, so you can continue to reduce after you distribute.
}
\litem{
What is the \textbf{smallest} set of Real numbers that the number below belongs to?
\[ \sqrt{\frac{1872}{8}} \]The solution is \( \text{Irrational} \).\begin{enumerate}[label=\Alph*.]
\textbf{Plausible alternative answers include:}These are the negative and positive counting numbers (..., -3, -2, -1, 0, 1, 2, 3, ...)
* This is the correct option!
These are Nonreal Complex numbers \textbf{OR} things that are not numbers (e.g., dividing by 0).
These are numbers that can be written as fraction of Integers (e.g., -2/3)
These are the counting numbers with 0 (0, 1, 2, 3, ...)
\end{enumerate}

\textbf{General Comment:} First, you \textbf{NEED} to simplify the expression. This question simplifies to $\sqrt{234}$. 
 
 Be sure you look at the simplified fraction and not just the decimal expansion. Numbers such as 13, 17, and 19 provide \textbf{long but repeating/terminating decimal expansions!} 
 
 The only ways to *not* be a Real number are: dividing by 0 or taking the square root of a negative number. 
 
 Irrational numbers are more than just square root of 3: adding or subtracting values from square root of 3 is also irrational.
}
\litem{
What is the \textbf{smallest} set of Complex numbers that the number below belongs to?
\[ \sqrt{\frac{0}{6}}+\sqrt{6}i \]The solution is \( \text{Pure Imaginary} \).\begin{enumerate}[label=\Alph*.]
\textbf{Plausible alternative answers include:}This is not a number. The only non-Complex number we know is dividing by 0 as this is not a number!
These are numbers that can be written as fraction of Integers (e.g., -2/3 + 5)
These cannot be written as a fraction of Integers. Remember: $\pi$ is not an Integer!
* This is the correct option!
This is a Complex number $(a+bi)$ that is not Real (has $i$ as part of the number).
\end{enumerate}

\textbf{General Comment:} Be sure to simplify $i^2 = -1$. This may remove the imaginary portion for your number. If you are having trouble, you may want to look at the \textit{Subgroups of the Real Numbers} section.
}
\litem{
Simplify the expression below into the form $a+bi$.
\[ (-7 - 10 i)(2 + 9 i) \]The solution is \( 76 - 83 i \).\begin{enumerate}[label=\Alph*.]
\textbf{Plausible alternative answers include:}* $76 - 83 i$, which is the correct option.
 $-104 - 43 i$, which corresponds to adding a minus sign in the first term.
 $76 + 83 i$, which corresponds to adding a minus sign in both terms.
 $-104 + 43 i$, which corresponds to adding a minus sign in the second term.
 $-14 - 90 i$, which corresponds to just multiplying the real terms to get the real part of the solution and the coefficients in the complex terms to get the complex part.
\end{enumerate}

\textbf{General Comment:} You can treat $i$ as a variable and distribute. Just remember that $i^2=-1$, so you can continue to reduce after you distribute.
}
\litem{
Simplify the expression below.
\[ 20 - 1 \div 19 * 13 - (2 * 3) \]The solution is \( 13.316 \).\begin{enumerate}[label=\Alph*.]
\textbf{Plausible alternative answers include:} 51.947, which corresponds to not distributing a negative correctly.
 13.996, which corresponds to an Order of Operations error: not reading left-to-right for multiplication/division.
* 13.316, which is the correct option.
 25.996, which corresponds to not distributing addition and subtraction correctly.
 You may have gotten this by making an unanticipated error. If you got a value that is not any of the others, please let the coordinator know so they can help you figure out what happened.
\end{enumerate}

\textbf{General Comment:} While you may remember (or were taught) PEMDAS is done in order, it is actually done as P/E/MD/AS. When we are at MD or AS, we read left to right.
}
\litem{
Simplify the expression below.
\[ 2 - 7^2 + 9 \div 16 * 13 \div 4 \]The solution is \( -45.172 \).\begin{enumerate}[label=\Alph*.]
\textbf{Plausible alternative answers include:} 51.011, which corresponds to two Order of Operations errors.
 52.828, which corresponds to an Order of Operations error: multiplying by negative before squaring. For example: $(-3)^2 \neq -3^2$
* -45.172, this is the correct option
 -46.989, which corresponds to an Order of Operations error: not reading left-to-right for multiplication/division.
 You may have gotten this by making an unanticipated error. If you got a value that is not any of the others, please let the coordinator know so they can help you figure out what happened.
\end{enumerate}

\textbf{General Comment:} While you may remember (or were taught) PEMDAS is done in order, it is actually done as P/E/MD/AS. When we are at MD or AS, we read left to right.
}
\litem{
What is the \textbf{smallest} set of Real numbers that the number below belongs to?
\[ -\sqrt{\frac{19}{0}} \]The solution is \( \text{Not a Real number} \).\begin{enumerate}[label=\Alph*.]
\textbf{Plausible alternative answers include:}These are the counting numbers with 0 (0, 1, 2, 3, ...)
* This is the correct option!
These are the negative and positive counting numbers (..., -3, -2, -1, 0, 1, 2, 3, ...)
These are numbers that can be written as fraction of Integers (e.g., -2/3)
These cannot be written as a fraction of Integers.
\end{enumerate}

\textbf{General Comment:} First, you \textbf{NEED} to simplify the expression. This question simplifies to $-\sqrt{\frac{19}{0}}$. 
 
 Be sure you look at the simplified fraction and not just the decimal expansion. Numbers such as 13, 17, and 19 provide \textbf{long but repeating/terminating decimal expansions!} 
 
 The only ways to *not* be a Real number are: dividing by 0 or taking the square root of a negative number. 
 
 Irrational numbers are more than just square root of 3: adding or subtracting values from square root of 3 is also irrational.
}
\litem{
Simplify the expression below into the form $a+bi$.
\[ \frac{-54 + 77 i}{-2 + 5 i} \]The solution is \( 17.00  + 4.00 i \).\begin{enumerate}[label=\Alph*.]
\textbf{Plausible alternative answers include:} $-9.55  - 14.62 i$, which corresponds to forgetting to multiply the conjugate by the numerator and not computing the conjugate correctly.
* $17.00  + 4.00 i$, which is the correct option.
 $17.00  + 116.00 i$, which corresponds to forgetting to multiply the conjugate by the numerator.
 $27.00  + 15.40 i$, which corresponds to just dividing the first term by the first term and the second by the second.
 $493.00  + 4.00 i$, which corresponds to forgetting to multiply the conjugate by the numerator and using a plus instead of a minus in the denominator.
\end{enumerate}

\textbf{General Comment:} Multiply the numerator and denominator by the *conjugate* of the denominator, then simplify. For example, if we have $2+3i$, the conjugate is $2-3i$.
}
\litem{
Simplify the expression below into the form $a+bi$.
\[ \frac{-63 + 11 i}{5 + 8 i} \]The solution is \( -2.55  + 6.28 i \).\begin{enumerate}[label=\Alph*.]
\textbf{Plausible alternative answers include:} $-2.55  + 559.00 i$, which corresponds to forgetting to multiply the conjugate by the numerator.
 $-227.00  + 6.28 i$, which corresponds to forgetting to multiply the conjugate by the numerator and using a plus instead of a minus in the denominator.
* $-2.55  + 6.28 i$, which is the correct option.
 $-12.60  + 1.38 i$, which corresponds to just dividing the first term by the first term and the second by the second.
 $-4.53  - 5.04 i$, which corresponds to forgetting to multiply the conjugate by the numerator and not computing the conjugate correctly.
\end{enumerate}

\textbf{General Comment:} Multiply the numerator and denominator by the *conjugate* of the denominator, then simplify. For example, if we have $2+3i$, the conjugate is $2-3i$.
}
\litem{
What is the \textbf{smallest} set of Complex numbers that the number below belongs to?
\[ \sqrt{\frac{-2805}{15}} i+\sqrt{182}i \]The solution is \( \text{Nonreal Complex} \).\begin{enumerate}[label=\Alph*.]
\textbf{Plausible alternative answers include:}These are numbers that can be written as fraction of Integers (e.g., -2/3 + 5)
This is a Complex number $(a+bi)$ that \textbf{only} has an imaginary part like $2i$.
These cannot be written as a fraction of Integers. Remember: $\pi$ is not an Integer!
* This is the correct option!
This is not a number. The only non-Complex number we know is dividing by 0 as this is not a number!
\end{enumerate}

\textbf{General Comment:} Be sure to simplify $i^2 = -1$. This may remove the imaginary portion for your number. If you are having trouble, you may want to look at the \textit{Subgroups of the Real Numbers} section.
}
\litem{
Simplify the expression below into the form $a+bi$.
\[ (2 - 5 i)(-4 + 7 i) \]The solution is \( 27 + 34 i \).\begin{enumerate}[label=\Alph*.]
\textbf{Plausible alternative answers include:} $-43 + 6 i$, which corresponds to adding a minus sign in the second term.
 $-43 - 6 i$, which corresponds to adding a minus sign in the first term.
* $27 + 34 i$, which is the correct option.
 $-8 - 35 i$, which corresponds to just multiplying the real terms to get the real part of the solution and the coefficients in the complex terms to get the complex part.
 $27 - 34 i$, which corresponds to adding a minus sign in both terms.
\end{enumerate}

\textbf{General Comment:} You can treat $i$ as a variable and distribute. Just remember that $i^2=-1$, so you can continue to reduce after you distribute.
}
\litem{
What is the \textbf{smallest} set of Real numbers that the number below belongs to?
\[ -\sqrt{\frac{130321}{361}} \]The solution is \( \text{Integer} \).\begin{enumerate}[label=\Alph*.]
\textbf{Plausible alternative answers include:}These cannot be written as a fraction of Integers.
These are Nonreal Complex numbers \textbf{OR} things that are not numbers (e.g., dividing by 0).
These are the counting numbers with 0 (0, 1, 2, 3, ...)
These are numbers that can be written as fraction of Integers (e.g., -2/3)
* This is the correct option!
\end{enumerate}

\textbf{General Comment:} First, you \textbf{NEED} to simplify the expression. This question simplifies to $-361$. 
 
 Be sure you look at the simplified fraction and not just the decimal expansion. Numbers such as 13, 17, and 19 provide \textbf{long but repeating/terminating decimal expansions!} 
 
 The only ways to *not* be a Real number are: dividing by 0 or taking the square root of a negative number. 
 
 Irrational numbers are more than just square root of 3: adding or subtracting values from square root of 3 is also irrational.
}
\litem{
What is the \textbf{smallest} set of Complex numbers that the number below belongs to?
\[ \sqrt{\frac{0}{289}}+\sqrt{3}i \]The solution is \( \text{Pure Imaginary} \).\begin{enumerate}[label=\Alph*.]
\textbf{Plausible alternative answers include:}This is not a number. The only non-Complex number we know is dividing by 0 as this is not a number!
These are numbers that can be written as fraction of Integers (e.g., -2/3 + 5)
These cannot be written as a fraction of Integers. Remember: $\pi$ is not an Integer!
* This is the correct option!
This is a Complex number $(a+bi)$ that is not Real (has $i$ as part of the number).
\end{enumerate}

\textbf{General Comment:} Be sure to simplify $i^2 = -1$. This may remove the imaginary portion for your number. If you are having trouble, you may want to look at the \textit{Subgroups of the Real Numbers} section.
}
\litem{
Simplify the expression below into the form $a+bi$.
\[ (4 - 6 i)(5 + 2 i) \]The solution is \( 32 - 22 i \).\begin{enumerate}[label=\Alph*.]
\textbf{Plausible alternative answers include:} $32 + 22 i$, which corresponds to adding a minus sign in both terms.
 $8 - 38 i$, which corresponds to adding a minus sign in the second term.
 $8 + 38 i$, which corresponds to adding a minus sign in the first term.
* $32 - 22 i$, which is the correct option.
 $20 - 12 i$, which corresponds to just multiplying the real terms to get the real part of the solution and the coefficients in the complex terms to get the complex part.
\end{enumerate}

\textbf{General Comment:} You can treat $i$ as a variable and distribute. Just remember that $i^2=-1$, so you can continue to reduce after you distribute.
}
\litem{
Simplify the expression below.
\[ 11 - 5^2 + 7 \div 1 * 19 \div 4 \]The solution is \( 19.250 \).\begin{enumerate}[label=\Alph*.]
\textbf{Plausible alternative answers include:} -13.908, which corresponds to an Order of Operations error: not reading left-to-right for multiplication/division.
 36.092, which corresponds to two Order of Operations errors.
* 19.250, this is the correct option
 69.250, which corresponds to an Order of Operations error: multiplying by negative before squaring. For example: $(-3)^2 \neq -3^2$
 You may have gotten this by making an unanticipated error. If you got a value that is not any of the others, please let the coordinator know so they can help you figure out what happened.
\end{enumerate}

\textbf{General Comment:} While you may remember (or were taught) PEMDAS is done in order, it is actually done as P/E/MD/AS. When we are at MD or AS, we read left to right.
}
\litem{
Simplify the expression below.
\[ 11 - 5 \div 17 * 19 - (1 * 4) \]The solution is \( 1.412 \).\begin{enumerate}[label=\Alph*.]
\textbf{Plausible alternative answers include:} 6.985, which corresponds to an Order of Operations error: not reading left-to-right for multiplication/division.
* 1.412, which is the correct option.
 14.985, which corresponds to not distributing addition and subtraction correctly.
 17.647, which corresponds to not distributing a negative correctly.
 You may have gotten this by making an unanticipated error. If you got a value that is not any of the others, please let the coordinator know so they can help you figure out what happened.
\end{enumerate}

\textbf{General Comment:} While you may remember (or were taught) PEMDAS is done in order, it is actually done as P/E/MD/AS. When we are at MD or AS, we read left to right.
}
\litem{
What is the \textbf{smallest} set of Real numbers that the number below belongs to?
\[ \sqrt{\frac{-2366}{13}} \]The solution is \( \text{Not a Real number} \).\begin{enumerate}[label=\Alph*.]
\textbf{Plausible alternative answers include:}These are the negative and positive counting numbers (..., -3, -2, -1, 0, 1, 2, 3, ...)
These cannot be written as a fraction of Integers.
These are the counting numbers with 0 (0, 1, 2, 3, ...)
These are numbers that can be written as fraction of Integers (e.g., -2/3)
* This is the correct option!
\end{enumerate}

\textbf{General Comment:} First, you \textbf{NEED} to simplify the expression. This question simplifies to $\sqrt{182} i$. 
 
 Be sure you look at the simplified fraction and not just the decimal expansion. Numbers such as 13, 17, and 19 provide \textbf{long but repeating/terminating decimal expansions!} 
 
 The only ways to *not* be a Real number are: dividing by 0 or taking the square root of a negative number. 
 
 Irrational numbers are more than just square root of 3: adding or subtracting values from square root of 3 is also irrational.
}
\litem{
Simplify the expression below into the form $a+bi$.
\[ \frac{54 - 11 i}{-8 + 3 i} \]The solution is \( -6.37  - 1.01 i \).\begin{enumerate}[label=\Alph*.]
\textbf{Plausible alternative answers include:} $-5.47  + 3.42 i$, which corresponds to forgetting to multiply the conjugate by the numerator and not computing the conjugate correctly.
 $-465.00  - 1.01 i$, which corresponds to forgetting to multiply the conjugate by the numerator and using a plus instead of a minus in the denominator.
 $-6.37  - 74.00 i$, which corresponds to forgetting to multiply the conjugate by the numerator.
 $-6.75  - 3.67 i$, which corresponds to just dividing the first term by the first term and the second by the second.
* $-6.37  - 1.01 i$, which is the correct option.
\end{enumerate}

\textbf{General Comment:} Multiply the numerator and denominator by the *conjugate* of the denominator, then simplify. For example, if we have $2+3i$, the conjugate is $2-3i$.
}
\litem{
Simplify the expression below into the form $a+bi$.
\[ \frac{-63 - 55 i}{-3 + 6 i} \]The solution is \( -3.13  + 12.07 i \).\begin{enumerate}[label=\Alph*.]
\textbf{Plausible alternative answers include:}* $-3.13  + 12.07 i$, which is the correct option.
 $-141.00  + 12.07 i$, which corresponds to forgetting to multiply the conjugate by the numerator and using a plus instead of a minus in the denominator.
 $11.53  - 4.73 i$, which corresponds to forgetting to multiply the conjugate by the numerator and not computing the conjugate correctly.
 $-3.13  + 543.00 i$, which corresponds to forgetting to multiply the conjugate by the numerator.
 $21.00  - 9.17 i$, which corresponds to just dividing the first term by the first term and the second by the second.
\end{enumerate}

\textbf{General Comment:} Multiply the numerator and denominator by the *conjugate* of the denominator, then simplify. For example, if we have $2+3i$, the conjugate is $2-3i$.
}
\litem{
What is the \textbf{smallest} set of Complex numbers that the number below belongs to?
\[ \sqrt{\frac{361}{225}} + 25i^2 \]The solution is \( \text{Rational} \).\begin{enumerate}[label=\Alph*.]
\textbf{Plausible alternative answers include:}* This is the correct option!
This is not a number. The only non-Complex number we know is dividing by 0 as this is not a number!
These cannot be written as a fraction of Integers. Remember: $\pi$ is not an Integer!
This is a Complex number $(a+bi)$ that \textbf{only} has an imaginary part like $2i$.
This is a Complex number $(a+bi)$ that is not Real (has $i$ as part of the number).
\end{enumerate}

\textbf{General Comment:} Be sure to simplify $i^2 = -1$. This may remove the imaginary portion for your number. If you are having trouble, you may want to look at the \textit{Subgroups of the Real Numbers} section.
}
\litem{
Simplify the expression below into the form $a+bi$.
\[ (5 - 9 i)(2 + 8 i) \]The solution is \( 82 + 22 i \).\begin{enumerate}[label=\Alph*.]
\textbf{Plausible alternative answers include:} $82 - 22 i$, which corresponds to adding a minus sign in both terms.
 $-62 - 58 i$, which corresponds to adding a minus sign in the second term.
 $10 - 72 i$, which corresponds to just multiplying the real terms to get the real part of the solution and the coefficients in the complex terms to get the complex part.
 $-62 + 58 i$, which corresponds to adding a minus sign in the first term.
* $82 + 22 i$, which is the correct option.
\end{enumerate}

\textbf{General Comment:} You can treat $i$ as a variable and distribute. Just remember that $i^2=-1$, so you can continue to reduce after you distribute.
}
\litem{
What is the \textbf{smallest} set of Real numbers that the number below belongs to?
\[ -\sqrt{\frac{1320}{10}} \]The solution is \( \text{Irrational} \).\begin{enumerate}[label=\Alph*.]
\textbf{Plausible alternative answers include:}These are the counting numbers with 0 (0, 1, 2, 3, ...)
* This is the correct option!
These are numbers that can be written as fraction of Integers (e.g., -2/3)
These are Nonreal Complex numbers \textbf{OR} things that are not numbers (e.g., dividing by 0).
These are the negative and positive counting numbers (..., -3, -2, -1, 0, 1, 2, 3, ...)
\end{enumerate}

\textbf{General Comment:} First, you \textbf{NEED} to simplify the expression. This question simplifies to $-\sqrt{132}$. 
 
 Be sure you look at the simplified fraction and not just the decimal expansion. Numbers such as 13, 17, and 19 provide \textbf{long but repeating/terminating decimal expansions!} 
 
 The only ways to *not* be a Real number are: dividing by 0 or taking the square root of a negative number. 
 
 Irrational numbers are more than just square root of 3: adding or subtracting values from square root of 3 is also irrational.
}
\litem{
What is the \textbf{smallest} set of Complex numbers that the number below belongs to?
\[ \frac{-19}{-12}+\sqrt{-36}i \]The solution is \( \text{Rational} \).\begin{enumerate}[label=\Alph*.]
\textbf{Plausible alternative answers include:}This is a Complex number $(a+bi)$ that is not Real (has $i$ as part of the number).
This is a Complex number $(a+bi)$ that \textbf{only} has an imaginary part like $2i$.
* This is the correct option!
This is not a number. The only non-Complex number we know is dividing by 0 as this is not a number!
These cannot be written as a fraction of Integers. Remember: $\pi$ is not an Integer!
\end{enumerate}

\textbf{General Comment:} Be sure to simplify $i^2 = -1$. This may remove the imaginary portion for your number. If you are having trouble, you may want to look at the \textit{Subgroups of the Real Numbers} section.
}
\litem{
Simplify the expression below into the form $a+bi$.
\[ (7 + 2 i)(9 - 5 i) \]The solution is \( 73 - 17 i \).\begin{enumerate}[label=\Alph*.]
\textbf{Plausible alternative answers include:} $63 - 10 i$, which corresponds to just multiplying the real terms to get the real part of the solution and the coefficients in the complex terms to get the complex part.
 $73 + 17 i$, which corresponds to adding a minus sign in both terms.
 $53 - 53 i$, which corresponds to adding a minus sign in the first term.
 $53 + 53 i$, which corresponds to adding a minus sign in the second term.
* $73 - 17 i$, which is the correct option.
\end{enumerate}

\textbf{General Comment:} You can treat $i$ as a variable and distribute. Just remember that $i^2=-1$, so you can continue to reduce after you distribute.
}
\litem{
Simplify the expression below.
\[ 14 - 10 \div 8 * 5 - (3 * 15) \]The solution is \( -37.250 \).\begin{enumerate}[label=\Alph*.]
\textbf{Plausible alternative answers include:} -31.250, which corresponds to an Order of Operations error: not reading left-to-right for multiplication/division.
 58.750, which corresponds to not distributing addition and subtraction correctly.
 71.250, which corresponds to not distributing a negative correctly.
* -37.250, which is the correct option.
 You may have gotten this by making an unanticipated error. If you got a value that is not any of the others, please let the coordinator know so they can help you figure out what happened.
\end{enumerate}

\textbf{General Comment:} While you may remember (or were taught) PEMDAS is done in order, it is actually done as P/E/MD/AS. When we are at MD or AS, we read left to right.
}
\litem{
Simplify the expression below.
\[ 11 - 19^2 + 2 \div 9 * 8 \div 1 \]The solution is \( -348.222 \).\begin{enumerate}[label=\Alph*.]
\textbf{Plausible alternative answers include:} 373.778, which corresponds to an Order of Operations error: multiplying by negative before squaring. For example: $(-3)^2 \neq -3^2$
 -349.972, which corresponds to an Order of Operations error: not reading left-to-right for multiplication/division.
 372.028, which corresponds to two Order of Operations errors.
* -348.222, this is the correct option
 You may have gotten this by making an unanticipated error. If you got a value that is not any of the others, please let the coordinator know so they can help you figure out what happened.
\end{enumerate}

\textbf{General Comment:} While you may remember (or were taught) PEMDAS is done in order, it is actually done as P/E/MD/AS. When we are at MD or AS, we read left to right.
}
\litem{
What is the \textbf{smallest} set of Real numbers that the number below belongs to?
\[ \sqrt{\frac{10816}{169}} \]The solution is \( \text{Whole} \).\begin{enumerate}[label=\Alph*.]
\textbf{Plausible alternative answers include:}* This is the correct option!
These are the negative and positive counting numbers (..., -3, -2, -1, 0, 1, 2, 3, ...)
These cannot be written as a fraction of Integers.
These are numbers that can be written as fraction of Integers (e.g., -2/3)
These are Nonreal Complex numbers \textbf{OR} things that are not numbers (e.g., dividing by 0).
\end{enumerate}

\textbf{General Comment:} First, you \textbf{NEED} to simplify the expression. This question simplifies to $104$. 
 
 Be sure you look at the simplified fraction and not just the decimal expansion. Numbers such as 13, 17, and 19 provide \textbf{long but repeating/terminating decimal expansions!} 
 
 The only ways to *not* be a Real number are: dividing by 0 or taking the square root of a negative number. 
 
 Irrational numbers are more than just square root of 3: adding or subtracting values from square root of 3 is also irrational.
}
\end{enumerate}

\end{document}