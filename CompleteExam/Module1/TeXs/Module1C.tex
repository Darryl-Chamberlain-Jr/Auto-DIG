\documentclass[14pt]{extbook}
\usepackage{multicol, enumerate, enumitem, hyperref, color, soul, setspace, parskip, fancyhdr} %General Packages
\usepackage{amssymb, amsthm, amsmath, latexsym, units, mathtools} %Math Packages
\everymath{\displaystyle} %All math in Display Style
% Packages with additional options
\usepackage[headsep=0.5cm,headheight=12pt, left=1 in,right= 1 in,top= 1 in,bottom= 1 in]{geometry}
\usepackage[usenames,dvipsnames]{xcolor}
\usepackage{dashrule}  % Package to use the command below to create lines between items
\newcommand{\litem}[1]{\item#1\hspace*{-1cm}\rule{\textwidth}{0.4pt}}
\pagestyle{fancy}
\lhead{Module1}
\chead{}
\rhead{Version C}
\lfoot{4041-3414}
\cfoot{}
\rfoot{test}
\begin{document}

\begin{enumerate}
\item{
Simplify the expression below into the form $a+bi$.\[ \frac{54 - 11 i}{-8 + 3 i} \]\newpage\end{enumerate} }
\item{
Simplify the expression below into the form $a+bi$.\[ \frac{-63 - 55 i}{-3 + 6 i} \]\newpage\end{enumerate} }
\item{
What is the \textbf{smallest} set of Complex numbers that the number below belongs to?\[ \sqrt{\frac{361}{225}} + 25i^2 \]\newpage\end{enumerate} }
\item{
Simplify the expression below into the form $a+bi$.\[ (5 - 9 i)(2 + 8 i) \]\newpage\end{enumerate} }
\item{
What is the \textbf{smallest} set of Real numbers that the number below belongs to?\[ -\sqrt{\frac{1320}{10}} \]\newpage\end{enumerate} }
\item{
What is the \textbf{smallest} set of Complex numbers that the number below belongs to?\[ \frac{-19}{-12}+\sqrt{-36}i \]\newpage\end{enumerate} }
\item{
Simplify the expression below into the form $a+bi$.\[ (7 + 2 i)(9 - 5 i) \]\newpage\end{enumerate} }
\item{
Simplify the expression below.\[ 14 - 10 \div 8 * 5 - (3 * 15) \]\newpage\end{enumerate} }
\item{
Simplify the expression below.\[ 11 - 19^2 + 2 \div 9 * 8 \div 1 \]\newpage\end{enumerate} }
\item{
What is the \textbf{smallest} set of Real numbers that the number below belongs to?\[ \sqrt{\frac{10816}{169}} \]\newpage\end{enumerate} }
\end{enumerate}

\end{document}