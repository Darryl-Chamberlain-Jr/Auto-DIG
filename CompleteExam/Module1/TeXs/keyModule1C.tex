\documentclass{extbook}[14pt]
\usepackage{multicol, enumerate, enumitem, hyperref, color, soul, setspace, parskip, fancyhdr, amssymb, amsthm, amsmath, bbm, latexsym, units, mathtools}
\everymath{\displaystyle}
\usepackage[headsep=0.5cm,headheight=0cm, left=1 in,right= 1 in,top= 1 in,bottom= 1 in]{geometry}
\usepackage{dashrule}  % Package to use the command below to create lines between items
\newcommand{\litem}[1]{\item #1

\rule{\textwidth}{0.4pt}}
\pagestyle{fancy}
\lhead{}
\chead{Answer Key for Module1 Version C}
\rhead{}
\lfoot{4565-2610}
\cfoot{}
\rfoot{Fall 2020}
\begin{document}
\textbf{This key should allow you to understand why you choose the option you did (beyond just getting a question right or wrong). \href{https://xronos.clas.ufl.edu/mac1105spring2020/courseDescriptionAndMisc/Exams/LearningFromResults}{More instructions on how to use this key can be found here}.}

\textbf{If you have a suggestion to make the keys better, \href{https://forms.gle/CZkbZmPbC9XALEE88}{please fill out the short survey here}.}

\textit{Note: This key is auto-generated and may contain issues and/or errors. The keys are reviewed after each exam to ensure grading is done accurately. If there are issues (like duplicate options), they are noted in the offline gradebook. The keys are a work-in-progress to give students as many resources to improve as possible.}

\rule{\textwidth}{0.4pt}

\begin{enumerate}\litem{
Choose the \textbf{smallest} set of Complex numbers that the number below belongs to.
\[ \sqrt{\frac{-2730}{0}}+\sqrt{221} \]
The solution is \( \text{Not a Complex Number} \), which is option B.\begin{enumerate}[label=\Alph*.]
\item \( \text{Nonreal Complex} \)

This is a Complex number $(a+bi)$ that is not Real (has $i$ as part of the number).
\item \( \text{Not a Complex Number} \)

* This is the correct option!
\item \( \text{Rational} \)

These are numbers that can be written as fraction of Integers (e.g., -2/3 + 5)
\item \( \text{Irrational} \)

These cannot be written as a fraction of Integers. Remember: $\pi$ is not an Integer!
\item \( \text{Pure Imaginary} \)

This is a Complex number $(a+bi)$ that \textbf{only} has an imaginary part like $2i$.
\end{enumerate}

\textbf{General Comment:} Be sure to simplify $i^2 = -1$. This may remove the imaginary portion for your number. If you are having trouble, you may want to look at the \textit{Subgroups of the Real Numbers} section.
}
\litem{
Simplify the expression below into the form $a+bi$. Then, choose the intervals that $a$ and $b$ belong to.
\[ \frac{-9 + 66 i}{2 + 8 i} \]
The solution is \( 7.50  + 3.00 i \), which is option C.\begin{enumerate}[label=\Alph*.]
\item \( a \in [509.5, 511.5] \text{ and } b \in [2.5, 3.5] \)

 $510.00  + 3.00 i$, which corresponds to forgetting to multiply the conjugate by the numerator and using a plus instead of a minus in the denominator.
\item \( a \in [7, 9] \text{ and } b \in [203.5, 205] \)

 $7.50  + 204.00 i$, which corresponds to forgetting to multiply the conjugate by the numerator.
\item \( a \in [7, 9] \text{ and } b \in [2.5, 3.5] \)

* $7.50  + 3.00 i$, which is the correct option.
\item \( a \in [-7, -4] \text{ and } b \in [8, 9] \)

 $-4.50  + 8.25 i$, which corresponds to just dividing the first term by the first term and the second by the second.
\item \( a \in [-9, -7.5] \text{ and } b \in [-1, 1.5] \)

 $-8.03  + 0.88 i$, which corresponds to forgetting to multiply the conjugate by the numerator and not computing the conjugate correctly.
\end{enumerate}

\textbf{General Comment:} Multiply the numerator and denominator by the *conjugate* of the denominator, then simplify. For example, if we have $2+3i$, the conjugate is $2-3i$.
}
\litem{
Simplify the expression below into the form $a+bi$. Then, choose the intervals that $a$ and $b$ belong to.
\[ (-6 - 3 i)(-4 + 8 i) \]
The solution is \( 48 - 36 i \), which is option E.\begin{enumerate}[label=\Alph*.]
\item \( a \in [0, 3] \text{ and } b \in [-61, -59] \)

 $0 - 60 i$, which corresponds to adding a minus sign in the first term.
\item \( a \in [22, 30] \text{ and } b \in [-27, -19] \)

 $24 - 24 i$, which corresponds to just multiplying the real terms to get the real part of the solution and the coefficients in the complex terms to get the complex part.
\item \( a \in [42, 50] \text{ and } b \in [36, 41] \)

 $48 + 36 i$, which corresponds to adding a minus sign in both terms.
\item \( a \in [0, 3] \text{ and } b \in [55, 62] \)

 $0 + 60 i$, which corresponds to adding a minus sign in the second term.
\item \( a \in [42, 50] \text{ and } b \in [-39, -31] \)

* $48 - 36 i$, which is the correct option.
\end{enumerate}

\textbf{General Comment:} You can treat $i$ as a variable and distribute. Just remember that $i^2=-1$, so you can continue to reduce after you distribute.
}
\litem{
Choose the \textbf{smallest} set of Complex numbers that the number below belongs to.
\[ \sqrt{\frac{-3570}{0}} i+\sqrt{130}i \]
The solution is \( \text{Not a Complex Number} \), which is option D.\begin{enumerate}[label=\Alph*.]
\item \( \text{Rational} \)

These are numbers that can be written as fraction of Integers (e.g., -2/3 + 5)
\item \( \text{Nonreal Complex} \)

This is a Complex number $(a+bi)$ that is not Real (has $i$ as part of the number).
\item \( \text{Irrational} \)

These cannot be written as a fraction of Integers. Remember: $\pi$ is not an Integer!
\item \( \text{Not a Complex Number} \)

* This is the correct option!
\item \( \text{Pure Imaginary} \)

This is a Complex number $(a+bi)$ that \textbf{only} has an imaginary part like $2i$.
\end{enumerate}

\textbf{General Comment:} Be sure to simplify $i^2 = -1$. This may remove the imaginary portion for your number. If you are having trouble, you may want to look at the \textit{Subgroups of the Real Numbers} section.
}
\litem{
Simplify the expression below into the form $a+bi$. Then, choose the intervals that $a$ and $b$ belong to.
\[ (8 + 5 i)(-4 - 9 i) \]
The solution is \( 13 - 92 i \), which is option D.\begin{enumerate}[label=\Alph*.]
\item \( a \in [-79, -73] \text{ and } b \in [-52, -49] \)

 $-77 - 52 i$, which corresponds to adding a minus sign in the first term.
\item \( a \in [-79, -73] \text{ and } b \in [50, 55] \)

 $-77 + 52 i$, which corresponds to adding a minus sign in the second term.
\item \( a \in [-35, -29] \text{ and } b \in [-45, -41] \)

 $-32 - 45 i$, which corresponds to just multiplying the real terms to get the real part of the solution and the coefficients in the complex terms to get the complex part.
\item \( a \in [5, 14] \text{ and } b \in [-95, -89] \)

* $13 - 92 i$, which is the correct option.
\item \( a \in [5, 14] \text{ and } b \in [90, 96] \)

 $13 + 92 i$, which corresponds to adding a minus sign in both terms.
\end{enumerate}

\textbf{General Comment:} You can treat $i$ as a variable and distribute. Just remember that $i^2=-1$, so you can continue to reduce after you distribute.
}
\litem{
Simplify the expression below into the form $a+bi$. Then, choose the intervals that $a$ and $b$ belong to.
\[ \frac{9 - 66 i}{-3 - 4 i} \]
The solution is \( 9.48  + 9.36 i \), which is option B.\begin{enumerate}[label=\Alph*.]
\item \( a \in [236, 238.5] \text{ and } b \in [8.5, 10.5] \)

 $237.00  + 9.36 i$, which corresponds to forgetting to multiply the conjugate by the numerator and using a plus instead of a minus in the denominator.
\item \( a \in [8.5, 10] \text{ and } b \in [8.5, 10.5] \)

* $9.48  + 9.36 i$, which is the correct option.
\item \( a \in [-3.5, -1.5] \text{ and } b \in [16, 17.5] \)

 $-3.00  + 16.50 i$, which corresponds to just dividing the first term by the first term and the second by the second.
\item \( a \in [8.5, 10] \text{ and } b \in [233, 235.5] \)

 $9.48  + 234.00 i$, which corresponds to forgetting to multiply the conjugate by the numerator.
\item \( a \in [-13, -11] \text{ and } b \in [5, 6.5] \)

 $-11.64  + 6.48 i$, which corresponds to forgetting to multiply the conjugate by the numerator and not computing the conjugate correctly.
\end{enumerate}

\textbf{General Comment:} Multiply the numerator and denominator by the *conjugate* of the denominator, then simplify. For example, if we have $2+3i$, the conjugate is $2-3i$.
}
\litem{
Simplify the expression below and choose the interval the simplification is contained within.
\[ 11 - 18 \div 19 * 2 - (6 * 17) \]
The solution is \( -92.895 \), which is option D.\begin{enumerate}[label=\Alph*.]
\item \( [49.1, 55.9] \)

 52.789, which corresponds to not distributing a negative correctly.
\item \( [112.2, 113.7] \)

 112.526, which corresponds to not distributing addition and subtraction correctly.
\item \( [-92, -86.8] \)

 -91.474, which corresponds to an Order of Operations error: not reading left-to-right for multiplication/division.
\item \( [-94.5, -92.1] \)

* -92.895, which is the correct option.
\item \( \text{None of the above} \)

 You may have gotten this by making an unanticipated error. If you got a value that is not any of the others, please let the coordinator know so they can help you figure out what happened.
\end{enumerate}

\textbf{General Comment:} While you may remember (or were taught) PEMDAS is done in order, it is actually done as P/E/MD/AS. When we are at MD or AS, we read left to right.
}
\litem{
Choose the \textbf{smallest} set of Real numbers that the number below belongs to.
\[ \sqrt{\frac{1664}{8}} \]
The solution is \( \text{Irrational} \), which is option C.\begin{enumerate}[label=\Alph*.]
\item \( \text{Integer} \)

These are the negative and positive counting numbers (..., -3, -2, -1, 0, 1, 2, 3, ...)
\item \( \text{Whole} \)

These are the counting numbers with 0 (0, 1, 2, 3, ...)
\item \( \text{Irrational} \)

* This is the correct option!
\item \( \text{Rational} \)

These are numbers that can be written as fraction of Integers (e.g., -2/3)
\item \( \text{Not a Real number} \)

These are Nonreal Complex numbers \textbf{OR} things that are not numbers (e.g., dividing by 0).
\end{enumerate}

\textbf{General Comment:} First, you \textbf{NEED} to simplify the expression. This question simplifies to $\sqrt{208}$. 
 
 Be sure you look at the simplified fraction and not just the decimal expansion. Numbers such as 13, 17, and 19 provide \textbf{long but repeating/terminating decimal expansions!} 
 
 The only ways to *not* be a Real number are: dividing by 0 or taking the square root of a negative number. 
 
 Irrational numbers are more than just square root of 3: adding or subtracting values from square root of 3 is also irrational.
}
\litem{
Simplify the expression below and choose the interval the simplification is contained within.
\[ 11 - 15 \div 3 * 12 - (6 * 5) \]
The solution is \( -79.000 \), which is option B.\begin{enumerate}[label=\Alph*.]
\item \( [-279, -274] \)

 -275.000, which corresponds to not distributing a negative correctly.
\item \( [-81, -77] \)

* -79.000, which is the correct option.
\item \( [40.58, 45.58] \)

 40.583, which corresponds to not distributing addition and subtraction correctly.
\item \( [-19.42, -18.42] \)

 -19.417, which corresponds to an Order of Operations error: not reading left-to-right for multiplication/division.
\item \( \text{None of the above} \)

 You may have gotten this by making an unanticipated error. If you got a value that is not any of the others, please let the coordinator know so they can help you figure out what happened.
\end{enumerate}

\textbf{General Comment:} While you may remember (or were taught) PEMDAS is done in order, it is actually done as P/E/MD/AS. When we are at MD or AS, we read left to right.
}
\litem{
Choose the \textbf{smallest} set of Real numbers that the number below belongs to.
\[ -\sqrt{\frac{400}{529}} \]
The solution is \( \text{Rational} \), which is option D.\begin{enumerate}[label=\Alph*.]
\item \( \text{Irrational} \)

These cannot be written as a fraction of Integers.
\item \( \text{Not a Real number} \)

These are Nonreal Complex numbers \textbf{OR} things that are not numbers (e.g., dividing by 0).
\item \( \text{Integer} \)

These are the negative and positive counting numbers (..., -3, -2, -1, 0, 1, 2, 3, ...)
\item \( \text{Rational} \)

* This is the correct option!
\item \( \text{Whole} \)

These are the counting numbers with 0 (0, 1, 2, 3, ...)
\end{enumerate}

\textbf{General Comment:} First, you \textbf{NEED} to simplify the expression. This question simplifies to $-\frac{20}{23}$. 
 
 Be sure you look at the simplified fraction and not just the decimal expansion. Numbers such as 13, 17, and 19 provide \textbf{long but repeating/terminating decimal expansions!} 
 
 The only ways to *not* be a Real number are: dividing by 0 or taking the square root of a negative number. 
 
 Irrational numbers are more than just square root of 3: adding or subtracting values from square root of 3 is also irrational.
}
\end{enumerate}

\end{document}