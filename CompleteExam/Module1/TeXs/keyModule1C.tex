\documentclass{extbook}[14pt]
\usepackage{multicol, enumerate, enumitem, hyperref, color, soul, setspace, parskip, fancyhdr, amssymb, amsthm, amsmath, latexsym, units, mathtools}
\everymath{\displaystyle}
\usepackage[headsep=0.5cm,headheight=0cm, left=1 in,right= 1 in,top= 1 in,bottom= 1 in]{geometry}
\usepackage{dashrule}  % Package to use the command below to create lines between items
\newcommand{\litem}[1]{\item #1

\rule{\textwidth}{0.4pt}}
\pagestyle{fancy}
\lhead{}
\chead{Answer Key for Module1 Version C}
\rhead{}
\lfoot{4877-7341}
\cfoot{}
\rfoot{test}
\begin{document}
\textbf{This key should allow you to understand why you choose the option you did (beyond just getting a question right or wrong). \href{https://xronos.clas.ufl.edu/mac1105spring2020/courseDescriptionAndMisc/Exams/LearningFromResults}{More instructions on how to use this key can be found here}.}

\textbf{If you have a suggestion to make the keys better, \href{https://forms.gle/CZkbZmPbC9XALEE88}{please fill out the short survey here}.}

\textit{Note: This key is auto-generated and may contain issues and/or errors. The keys are reviewed after each exam to ensure grading is done accurately. If there are issues (like duplicate options), they are noted in the offline gradebook. The keys are a work-in-progress to give students as many resources to improve as possible.}

\rule{\textwidth}{0.4pt}

\begin{enumerate}\litem{
Choose the \textbf{smallest} set of Complex numbers that the number below belongs to.
\[ -\sqrt{\frac{1764}{14}}+3i^2 \]The solution is \( \text{Irrational} \), which is option C.\begin{enumerate}[label=\Alph*.]
\item \( \text{Not a Complex Number} \)

This is not a number. The only non-Complex number we know is dividing by 0 as this is not a number!
\item \( \text{Rational} \)

These are numbers that can be written as fraction of Integers (e.g., -2/3 + 5)
\item \( \text{Irrational} \)

* This is the correct option!
\item \( \text{Pure Imaginary} \)

This is a Complex number $(a+bi)$ that \textbf{only} has an imaginary part like $2i$.
\item \( \text{Nonreal Complex} \)

This is a Complex number $(a+bi)$ that is not Real (has $i$ as part of the number).
\end{enumerate}

\textbf{General Comment:} Be sure to simplify $i^2 = -1$. This may remove the imaginary portion for your number. If you are having trouble, you may want to look at the \textit{Subgroups of the Real Numbers} section.
}
\litem{
Simplify the expression below and choose the interval the simplification is contained within.
\[ 19 - 3^2 + 11 \div 1 * 6 \div 13 \]The solution is \( 15.077 \), which is option A.\begin{enumerate}[label=\Alph*.]
\item \( [13.2, 17.4] \)

* 15.077, this is the correct option
\item \( [27.3, 28.8] \)

 28.141, which corresponds to two Order of Operations errors.
\item \( [30.9, 36] \)

 33.077, which corresponds to an Order of Operations error: multiplying by negative before squaring. For example: $(-3)^2 \neq -3^2$
\item \( [9.3, 11] \)

 10.141, which corresponds to an Order of Operations error: not reading left-to-right for multiplication/division.
\item \( \text{None of the above} \)

 You may have gotten this by making an unanticipated error. If you got a value that is not any of the others, please let the coordinator know so they can help you figure out what happened.
\end{enumerate}

\textbf{General Comment:} While you may remember (or were taught) PEMDAS is done in order, it is actually done as P/E/MD/AS. When we are at MD or AS, we read left to right.
}
\litem{
Choose the \textbf{smallest} set of Complex numbers that the number below belongs to.
\[ \frac{\sqrt{60}}{11}+\sqrt{-5}i \]The solution is \( \text{Irrational} \), which is option E.\begin{enumerate}[label=\Alph*.]
\item \( \text{Nonreal Complex} \)

This is a Complex number $(a+bi)$ that is not Real (has $i$ as part of the number).
\item \( \text{Rational} \)

These are numbers that can be written as fraction of Integers (e.g., -2/3 + 5)
\item \( \text{Pure Imaginary} \)

This is a Complex number $(a+bi)$ that \textbf{only} has an imaginary part like $2i$.
\item \( \text{Not a Complex Number} \)

This is not a number. The only non-Complex number we know is dividing by 0 as this is not a number!
\item \( \text{Irrational} \)

* This is the correct option!
\end{enumerate}

\textbf{General Comment:} Be sure to simplify $i^2 = -1$. This may remove the imaginary portion for your number. If you are having trouble, you may want to look at the \textit{Subgroups of the Real Numbers} section.
}
\litem{
Simplify the expression below into the form $a+bi$. Then, choose the intervals that $a$ and $b$ belong to.
\[ \frac{-9 - 44 i}{-5 + 7 i} \]The solution is \( -3.55  + 3.82 i \), which is option A.\begin{enumerate}[label=\Alph*.]
\item \( a \in [-4.5, -2.5] \text{ and } b \in [3.5, 4] \)

* $-3.55  + 3.82 i$, which is the correct option.
\item \( a \in [-4.5, -2.5] \text{ and } b \in [282, 284.5] \)

 $-3.55  + 283.00 i$, which corresponds to forgetting to multiply the conjugate by the numerator.
\item \( a \in [4, 5.5] \text{ and } b \in [2, 3] \)

 $4.77  + 2.12 i$, which corresponds to forgetting to multiply the conjugate by the numerator and not computing the conjugate correctly.
\item \( a \in [1.5, 2.5] \text{ and } b \in [-8, -6] \)

 $1.80  - 6.29 i$, which corresponds to just dividing the first term by the first term and the second by the second.
\item \( a \in [-264, -262.5] \text{ and } b \in [3.5, 4] \)

 $-263.00  + 3.82 i$, which corresponds to forgetting to multiply the conjugate by the numerator and using a plus instead of a minus in the denominator.
\end{enumerate}

\textbf{General Comment:} Multiply the numerator and denominator by the *conjugate* of the denominator, then simplify. For example, if we have $2+3i$, the conjugate is $2-3i$.
}
\litem{
Simplify the expression below into the form $a+bi$. Then, choose the intervals that $a$ and $b$ belong to.
\[ \frac{-63 - 55 i}{-2 - 8 i} \]The solution is \( 8.32  - 5.79 i \), which is option E.\begin{enumerate}[label=\Alph*.]
\item \( a \in [564.5, 567.5] \text{ and } b \in [-7, -3.5] \)

 $566.00  - 5.79 i$, which corresponds to forgetting to multiply the conjugate by the numerator and using a plus instead of a minus in the denominator.
\item \( a \in [30, 32] \text{ and } b \in [6, 7.5] \)

 $31.50  + 6.88 i$, which corresponds to just dividing the first term by the first term and the second by the second.
\item \( a \in [-5.5, -4] \text{ and } b \in [8, 9.5] \)

 $-4.62  + 9.03 i$, which corresponds to forgetting to multiply the conjugate by the numerator and not computing the conjugate correctly.
\item \( a \in [7.5, 9] \text{ and } b \in [-396, -393] \)

 $8.32  - 394.00 i$, which corresponds to forgetting to multiply the conjugate by the numerator.
\item \( a \in [7.5, 9] \text{ and } b \in [-7, -3.5] \)

* $8.32  - 5.79 i$, which is the correct option.
\end{enumerate}

\textbf{General Comment:} Multiply the numerator and denominator by the *conjugate* of the denominator, then simplify. For example, if we have $2+3i$, the conjugate is $2-3i$.
}
\litem{
Simplify the expression below and choose the interval the simplification is contained within.
\[ 11 - 19^2 + 3 \div 7 * 13 \div 12 \]The solution is \( -349.536 \), which is option A.\begin{enumerate}[label=\Alph*.]
\item \( [-349.54, -349.31] \)

* -349.536, this is the correct option
\item \( [372.21, 372.49] \)

 372.464, which corresponds to an Order of Operations error: multiplying by negative before squaring. For example: $(-3)^2 \neq -3^2$
\item \( [371.92, 372.46] \)

 372.003, which corresponds to two Order of Operations errors.
\item \( [-350.27, -349.92] \)

 -349.997, which corresponds to an Order of Operations error: not reading left-to-right for multiplication/division.
\item \( \text{None of the above} \)

 You may have gotten this by making an unanticipated error. If you got a value that is not any of the others, please let the coordinator know so they can help you figure out what happened.
\end{enumerate}

\textbf{General Comment:} While you may remember (or were taught) PEMDAS is done in order, it is actually done as P/E/MD/AS. When we are at MD or AS, we read left to right.
}
\litem{
Simplify the expression below into the form $a+bi$. Then, choose the intervals that $a$ and $b$ belong to.
\[ (8 + 5 i)(-7 - 10 i) \]The solution is \( -6 - 115 i \), which is option E.\begin{enumerate}[label=\Alph*.]
\item \( a \in [-57, -55] \text{ and } b \in [-51.6, -49.6] \)

 $-56 - 50 i$, which corresponds to just multiplying the real terms to get the real part of the solution and the coefficients in the complex terms to get the complex part.
\item \( a \in [-106, -104] \text{ and } b \in [44, 47.6] \)

 $-106 + 45 i$, which corresponds to adding a minus sign in the second term.
\item \( a \in [-106, -104] \text{ and } b \in [-47.6, -42.1] \)

 $-106 - 45 i$, which corresponds to adding a minus sign in the first term.
\item \( a \in [-8, -1] \text{ and } b \in [114.8, 116.5] \)

 $-6 + 115 i$, which corresponds to adding a minus sign in both terms.
\item \( a \in [-8, -1] \text{ and } b \in [-116.5, -114.7] \)

* $-6 - 115 i$, which is the correct option.
\end{enumerate}

\textbf{General Comment:} You can treat $i$ as a variable and distribute. Just remember that $i^2=-1$, so you can continue to reduce after you distribute.
}
\litem{
Choose the \textbf{smallest} set of Real numbers that the number below belongs to.
\[ -\sqrt{\frac{455}{7}} \]The solution is \( \text{Irrational} \), which is option D.\begin{enumerate}[label=\Alph*.]
\item \( \text{Integer} \)

These are the negative and positive counting numbers (..., -3, -2, -1, 0, 1, 2, 3, ...)
\item \( \text{Rational} \)

These are numbers that can be written as fraction of Integers (e.g., -2/3)
\item \( \text{Whole} \)

These are the counting numbers with 0 (0, 1, 2, 3, ...)
\item \( \text{Irrational} \)

* This is the correct option!
\item \( \text{Not a Real number} \)

These are Nonreal Complex numbers \textbf{OR} things that are not numbers (e.g., dividing by 0).
\end{enumerate}

\textbf{General Comment:} First, you \textbf{NEED} to simplify the expression. This question simplifies to $-\sqrt{65}$. 
 
 Be sure you look at the simplified fraction and not just the decimal expansion. Numbers such as 13, 17, and 19 provide \textbf{long but repeating/terminating decimal expansions!} 
 
 The only ways to *not* be a Real number are: dividing by 0 or taking the square root of a negative number. 
 
 Irrational numbers are more than just square root of 3: adding or subtracting values from square root of 3 is also irrational.
}
\litem{
Simplify the expression below into the form $a+bi$. Then, choose the intervals that $a$ and $b$ belong to.
\[ (6 + 9 i)(10 - 8 i) \]The solution is \( 132 + 42 i \), which is option B.\begin{enumerate}[label=\Alph*.]
\item \( a \in [-16, -10] \text{ and } b \in [-140, -136] \)

 $-12 - 138 i$, which corresponds to adding a minus sign in the first term.
\item \( a \in [129, 135] \text{ and } b \in [36, 43] \)

* $132 + 42 i$, which is the correct option.
\item \( a \in [-16, -10] \text{ and } b \in [137, 145] \)

 $-12 + 138 i$, which corresponds to adding a minus sign in the second term.
\item \( a \in [57, 65] \text{ and } b \in [-72, -69] \)

 $60 - 72 i$, which corresponds to just multiplying the real terms to get the real part of the solution and the coefficients in the complex terms to get the complex part.
\item \( a \in [129, 135] \text{ and } b \in [-45, -40] \)

 $132 - 42 i$, which corresponds to adding a minus sign in both terms.
\end{enumerate}

\textbf{General Comment:} You can treat $i$ as a variable and distribute. Just remember that $i^2=-1$, so you can continue to reduce after you distribute.
}
\litem{
Choose the \textbf{smallest} set of Real numbers that the number below belongs to.
\[ -\sqrt{\frac{144}{169}} \]The solution is \( \text{Rational} \), which is option A.\begin{enumerate}[label=\Alph*.]
\item \( \text{Rational} \)

* This is the correct option!
\item \( \text{Whole} \)

These are the counting numbers with 0 (0, 1, 2, 3, ...)
\item \( \text{Integer} \)

These are the negative and positive counting numbers (..., -3, -2, -1, 0, 1, 2, 3, ...)
\item \( \text{Not a Real number} \)

These are Nonreal Complex numbers \textbf{OR} things that are not numbers (e.g., dividing by 0).
\item \( \text{Irrational} \)

These cannot be written as a fraction of Integers.
\end{enumerate}

\textbf{General Comment:} First, you \textbf{NEED} to simplify the expression. This question simplifies to $-\frac{12}{13}$. 
 
 Be sure you look at the simplified fraction and not just the decimal expansion. Numbers such as 13, 17, and 19 provide \textbf{long but repeating/terminating decimal expansions!} 
 
 The only ways to *not* be a Real number are: dividing by 0 or taking the square root of a negative number. 
 
 Irrational numbers are more than just square root of 3: adding or subtracting values from square root of 3 is also irrational.
}
\end{enumerate}

\end{document}