\documentclass{extbook}[14pt]
\usepackage{multicol, enumerate, enumitem, hyperref, color, soul, setspace, parskip, fancyhdr, amssymb, amsthm, amsmath, latexsym, units, mathtools}
\everymath{\displaystyle}
\usepackage[headsep=0.5cm,headheight=0cm, left=1 in,right= 1 in,top= 1 in,bottom= 1 in]{geometry}
\usepackage{dashrule}  % Package to use the command below to create lines between items
\newcommand{\litem}[1]{\item #1

\rule{\textwidth}{0.4pt}}
\pagestyle{fancy}
\lhead{}
\chead{Answer Key for Progress Quiz 6 Version A}
\rhead{}
\lfoot{4563-7456}
\cfoot{}
\rfoot{Summer C 2021}
\begin{document}
\textbf{This key should allow you to understand why you choose the option you did (beyond just getting a question right or wrong). \href{https://xronos.clas.ufl.edu/mac1105spring2020/courseDescriptionAndMisc/Exams/LearningFromResults}{More instructions on how to use this key can be found here}.}

\textbf{If you have a suggestion to make the keys better, \href{https://forms.gle/CZkbZmPbC9XALEE88}{please fill out the short survey here}.}

\textit{Note: This key is auto-generated and may contain issues and/or errors. The keys are reviewed after each exam to ensure grading is done accurately. If there are issues (like duplicate options), they are noted in the offline gradebook. The keys are a work-in-progress to give students as many resources to improve as possible.}

\rule{\textwidth}{0.4pt}

\begin{enumerate}\litem{
Choose the \textbf{smallest} set of Real numbers that the number below belongs to.
\[ \sqrt{\frac{13225}{25}} \]The solution is \( \text{Whole} \), which is option E.\begin{enumerate}[label=\Alph*.]
\item \( \text{Irrational} \)

These cannot be written as a fraction of Integers.
\item \( \text{Not a Real number} \)

These are Nonreal Complex numbers \textbf{OR} things that are not numbers (e.g., dividing by 0).
\item \( \text{Integer} \)

These are the negative and positive counting numbers (..., -3, -2, -1, 0, 1, 2, 3, ...)
\item \( \text{Rational} \)

These are numbers that can be written as fraction of Integers (e.g., -2/3)
\item \( \text{Whole} \)

* This is the correct option!
\end{enumerate}

\textbf{General Comment:} First, you \textbf{NEED} to simplify the expression. This question simplifies to $115$. 
 
 Be sure you look at the simplified fraction and not just the decimal expansion. Numbers such as 13, 17, and 19 provide \textbf{long but repeating/terminating decimal expansions!} 
 
 The only ways to *not* be a Real number are: dividing by 0 or taking the square root of a negative number. 
 
 Irrational numbers are more than just square root of 3: adding or subtracting values from square root of 3 is also irrational.
}
\litem{
Simplify the expression below into the form $a+bi$. Then, choose the intervals that $a$ and $b$ belong to.
\[ (7 - 5 i)(3 - 4 i) \]The solution is \( 1 - 43 i \), which is option B.\begin{enumerate}[label=\Alph*.]
\item \( a \in [1, 5] \text{ and } b \in [41, 51] \)

 $1 + 43 i$, which corresponds to adding a minus sign in both terms.
\item \( a \in [1, 5] \text{ and } b \in [-43, -38] \)

* $1 - 43 i$, which is the correct option.
\item \( a \in [40, 44] \text{ and } b \in [12, 17] \)

 $41 + 13 i$, which corresponds to adding a minus sign in the second term.
\item \( a \in [40, 44] \text{ and } b \in [-15, -7] \)

 $41 - 13 i$, which corresponds to adding a minus sign in the first term.
\item \( a \in [21, 24] \text{ and } b \in [20, 23] \)

 $21 + 20 i$, which corresponds to just multiplying the real terms to get the real part of the solution and the coefficients in the complex terms to get the complex part.
\end{enumerate}

\textbf{General Comment:} You can treat $i$ as a variable and distribute. Just remember that $i^2=-1$, so you can continue to reduce after you distribute.
}
\litem{
Simplify the expression below and choose the interval the simplification is contained within.
\[ 12 - 16^2 + 7 \div 6 * 5 \div 2 \]The solution is \( -241.083 \), which is option D.\begin{enumerate}[label=\Alph*.]
\item \( [270, 274.4] \)

 270.917, which corresponds to an Order of Operations error: multiplying by negative before squaring. For example: $(-3)^2 \neq -3^2$
\item \( [267.4, 269.4] \)

 268.117, which corresponds to two Order of Operations errors.
\item \( [-245.1, -241.9] \)

 -243.883, which corresponds to an Order of Operations error: not reading left-to-right for multiplication/division.
\item \( [-242.5, -240.2] \)

* -241.083, this is the correct option
\item \( \text{None of the above} \)

 You may have gotten this by making an unanticipated error. If you got a value that is not any of the others, please let the coordinator know so they can help you figure out what happened.
\end{enumerate}

\textbf{General Comment:} While you may remember (or were taught) PEMDAS is done in order, it is actually done as P/E/MD/AS. When we are at MD or AS, we read left to right.
}
\litem{
Choose the \textbf{smallest} set of Complex numbers that the number below belongs to.
\[ \sqrt{\frac{225}{196}} + 4i^2 \]The solution is \( \text{Rational} \), which is option C.\begin{enumerate}[label=\Alph*.]
\item \( \text{Nonreal Complex} \)

This is a Complex number $(a+bi)$ that is not Real (has $i$ as part of the number).
\item \( \text{Not a Complex Number} \)

This is not a number. The only non-Complex number we know is dividing by 0 as this is not a number!
\item \( \text{Rational} \)

* This is the correct option!
\item \( \text{Pure Imaginary} \)

This is a Complex number $(a+bi)$ that \textbf{only} has an imaginary part like $2i$.
\item \( \text{Irrational} \)

These cannot be written as a fraction of Integers. Remember: $\pi$ is not an Integer!
\end{enumerate}

\textbf{General Comment:} Be sure to simplify $i^2 = -1$. This may remove the imaginary portion for your number. If you are having trouble, you may want to look at the \textit{Subgroups of the Real Numbers} section.
}
\litem{
Simplify the expression below into the form $a+bi$. Then, choose the intervals that $a$ and $b$ belong to.
\[ (-5 - 2 i)(3 + 4 i) \]The solution is \( -7 - 26 i \), which is option E.\begin{enumerate}[label=\Alph*.]
\item \( a \in [-18, -11] \text{ and } b \in [-11, -5] \)

 $-15 - 8 i$, which corresponds to just multiplying the real terms to get the real part of the solution and the coefficients in the complex terms to get the complex part.
\item \( a \in [-32, -21] \text{ and } b \in [13, 19] \)

 $-23 + 14 i$, which corresponds to adding a minus sign in the second term.
\item \( a \in [-9, -5] \text{ and } b \in [24, 31] \)

 $-7 + 26 i$, which corresponds to adding a minus sign in both terms.
\item \( a \in [-32, -21] \text{ and } b \in [-20, -11] \)

 $-23 - 14 i$, which corresponds to adding a minus sign in the first term.
\item \( a \in [-9, -5] \text{ and } b \in [-29, -19] \)

* $-7 - 26 i$, which is the correct option.
\end{enumerate}

\textbf{General Comment:} You can treat $i$ as a variable and distribute. Just remember that $i^2=-1$, so you can continue to reduce after you distribute.
}
\litem{
Simplify the expression below and choose the interval the simplification is contained within.
\[ 4 - 6 \div 9 * 5 - (20 * 3) \]The solution is \( -59.333 \), which is option C.\begin{enumerate}[label=\Alph*.]
\item \( [-58.3, -57.8] \)

 -58.000, which corresponds to not distributing a negative correctly.
\item \( [-56.4, -53.1] \)

 -56.133, which corresponds to an Order of Operations error: not reading left-to-right for multiplication/division.
\item \( [-61.6, -58.2] \)

* -59.333, which is the correct option.
\item \( [61.6, 64.1] \)

 63.867, which corresponds to not distributing addition and subtraction correctly.
\item \( \text{None of the above} \)

 You may have gotten this by making an unanticipated error. If you got a value that is not any of the others, please let the coordinator know so they can help you figure out what happened.
\end{enumerate}

\textbf{General Comment:} While you may remember (or were taught) PEMDAS is done in order, it is actually done as P/E/MD/AS. When we are at MD or AS, we read left to right.
}
\litem{
Simplify the expression below into the form $a+bi$. Then, choose the intervals that $a$ and $b$ belong to.
\[ \frac{54 + 44 i}{-2 - 3 i} \]The solution is \( -18.46  + 5.69 i \), which is option C.\begin{enumerate}[label=\Alph*.]
\item \( a \in [-27.5, -25.5] \text{ and } b \in [-15, -13.5] \)

 $-27.00  - 14.67 i$, which corresponds to just dividing the first term by the first term and the second by the second.
\item \( a \in [0.5, 3] \text{ and } b \in [-19.5, -18.5] \)

 $1.85  - 19.23 i$, which corresponds to forgetting to multiply the conjugate by the numerator and not computing the conjugate correctly.
\item \( a \in [-20.5, -17.5] \text{ and } b \in [4.5, 7.5] \)

* $-18.46  + 5.69 i$, which is the correct option.
\item \( a \in [-240.5, -239] \text{ and } b \in [4.5, 7.5] \)

 $-240.00  + 5.69 i$, which corresponds to forgetting to multiply the conjugate by the numerator and using a plus instead of a minus in the denominator.
\item \( a \in [-20.5, -17.5] \text{ and } b \in [72.5, 74.5] \)

 $-18.46  + 74.00 i$, which corresponds to forgetting to multiply the conjugate by the numerator.
\end{enumerate}

\textbf{General Comment:} Multiply the numerator and denominator by the *conjugate* of the denominator, then simplify. For example, if we have $2+3i$, the conjugate is $2-3i$.
}
\litem{
Choose the \textbf{smallest} set of Complex numbers that the number below belongs to.
\[ \sqrt{\frac{400}{49}}+\sqrt{156} i \]The solution is \( \text{Nonreal Complex} \), which is option D.\begin{enumerate}[label=\Alph*.]
\item \( \text{Not a Complex Number} \)

This is not a number. The only non-Complex number we know is dividing by 0 as this is not a number!
\item \( \text{Rational} \)

These are numbers that can be written as fraction of Integers (e.g., -2/3 + 5)
\item \( \text{Pure Imaginary} \)

This is a Complex number $(a+bi)$ that \textbf{only} has an imaginary part like $2i$.
\item \( \text{Nonreal Complex} \)

* This is the correct option!
\item \( \text{Irrational} \)

These cannot be written as a fraction of Integers. Remember: $\pi$ is not an Integer!
\end{enumerate}

\textbf{General Comment:} Be sure to simplify $i^2 = -1$. This may remove the imaginary portion for your number. If you are having trouble, you may want to look at the \textit{Subgroups of the Real Numbers} section.
}
\litem{
Simplify the expression below into the form $a+bi$. Then, choose the intervals that $a$ and $b$ belong to.
\[ \frac{-36 + 77 i}{-1 + 8 i} \]The solution is \( 10.03  + 3.25 i \), which is option E.\begin{enumerate}[label=\Alph*.]
\item \( a \in [650.5, 653] \text{ and } b \in [2, 5] \)

 $652.00  + 3.25 i$, which corresponds to forgetting to multiply the conjugate by the numerator and using a plus instead of a minus in the denominator.
\item \( a \in [9.5, 12] \text{ and } b \in [210, 211.5] \)

 $10.03  + 211.00 i$, which corresponds to forgetting to multiply the conjugate by the numerator.
\item \( a \in [-10, -7.5] \text{ and } b \in [-7, -5.5] \)

 $-8.92  - 5.62 i$, which corresponds to forgetting to multiply the conjugate by the numerator and not computing the conjugate correctly.
\item \( a \in [34.5, 37.5] \text{ and } b \in [9, 10] \)

 $36.00  + 9.62 i$, which corresponds to just dividing the first term by the first term and the second by the second.
\item \( a \in [9.5, 12] \text{ and } b \in [2, 5] \)

* $10.03  + 3.25 i$, which is the correct option.
\end{enumerate}

\textbf{General Comment:} Multiply the numerator and denominator by the *conjugate* of the denominator, then simplify. For example, if we have $2+3i$, the conjugate is $2-3i$.
}
\litem{
Choose the \textbf{smallest} set of Real numbers that the number below belongs to.
\[ -\sqrt{\frac{10}{0}} \]The solution is \( \text{Not a Real number} \), which is option B.\begin{enumerate}[label=\Alph*.]
\item \( \text{Rational} \)

These are numbers that can be written as fraction of Integers (e.g., -2/3)
\item \( \text{Not a Real number} \)

* This is the correct option!
\item \( \text{Irrational} \)

These cannot be written as a fraction of Integers.
\item \( \text{Integer} \)

These are the negative and positive counting numbers (..., -3, -2, -1, 0, 1, 2, 3, ...)
\item \( \text{Whole} \)

These are the counting numbers with 0 (0, 1, 2, 3, ...)
\end{enumerate}

\textbf{General Comment:} First, you \textbf{NEED} to simplify the expression. This question simplifies to $-\sqrt{\frac{10}{0}}$. 
 
 Be sure you look at the simplified fraction and not just the decimal expansion. Numbers such as 13, 17, and 19 provide \textbf{long but repeating/terminating decimal expansions!} 
 
 The only ways to *not* be a Real number are: dividing by 0 or taking the square root of a negative number. 
 
 Irrational numbers are more than just square root of 3: adding or subtracting values from square root of 3 is also irrational.
}
\end{enumerate}

\end{document}