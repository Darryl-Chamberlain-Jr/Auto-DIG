\documentclass[14pt]{extbook}
\usepackage{multicol, enumerate, enumitem, hyperref, color, soul, setspace, parskip, fancyhdr} %General Packages
\usepackage{amssymb, amsthm, amsmath, latexsym, units, mathtools} %Math Packages
\everymath{\displaystyle} %All math in Display Style
% Packages with additional options
\usepackage[headsep=0.5cm,headheight=12pt, left=1 in,right= 1 in,top= 1 in,bottom= 1 in]{geometry}
\usepackage[usenames,dvipsnames]{xcolor}
\usepackage{dashrule}  % Package to use the command below to create lines between items
\newcommand{\litem}[1]{\item#1\hspace*{-1cm}\rule{\textwidth}{0.4pt}}
\pagestyle{fancy}
\lhead{Progress Quiz 6}
\chead{}
\rhead{Version ALL}
\lfoot{4563-7456}
\cfoot{}
\rfoot{Summer C 2021}
\begin{document}

\begin{enumerate}
\litem{
For the graph below, find the value(s) $a$ that makes the statement true: $ \displaystyle \lim_{x \rightarrow a} f(x) = 3$.
\begin{center}
    \includegraphics[width=0.5\textwidth]{../Figures/evaluateLimitGraphicallyA.png}
\end{center}
\begin{enumerate}[label=\Alph*.]
\item \( -\infty \)
\item \( -2 \)
\item \( 1 \)
\item \( \text{Multiple } a \text{ make the statement true}. \)
\item \( \text{No } a \text{ make the statement true}. \)

\end{enumerate} }
\litem{
Evaluate the limit below, if possible.\[ \lim_{x \rightarrow 7} \frac{\sqrt{9x - 47} - 4}{5x - 35} \]\begin{enumerate}[label=\Alph*.]
\item \( 0.125 \)
\item \( 0.600 \)
\item \( 0.025 \)
\item \( \infty \)
\item \( \text{None of the above} \)

\end{enumerate} }
\litem{
Evaluate the one-sided limit of the function $f(x)$ below, if possible.\[ \lim_{x \rightarrow -3^-} \frac{3}{(x+3)^7}+4 \]\begin{enumerate}[label=\Alph*.]
\item \( f(-3) \)
\item \( \infty \)
\item \( -\infty \)
\item \( \text{The limit does not exist} \)
\item \( \text{None of the above} \)

\end{enumerate} }
\litem{
To estimate the one-sided limit of the function below as $x$ approaches 5 from the left, which of the following sets of numbers should you use?\[ \frac{\frac{5}{x} - 1}{x - 5} \]\begin{enumerate}[label=\Alph*.]
\item \( \{ 4.9000, 4.9900, 5.0100, 5.1000 \} \)
\item \( \{ 5.0000, 4.9000, 4.9900, 4.9990 \} \)
\item \( \{ 4.9000, 4.9900, 4.9990, 4.9999 \} \)
\item \( \{ 5.0000, 5.1000, 5.0100, 5.0010 \} \)
\item \( \{ 5.1000, 5.0100, 5.0010, 5.0001 \} \)

\end{enumerate} }
\litem{
Based on the information below, which of the following statements is always true?
\begin{center}
    \textit{ As $x$ approaches $4$, $f(x)$ approaches $3.047$. }
\end{center}
\begin{enumerate}[label=\Alph*.]
\item \( f(3) \text{ is close to or exactly } 4 \)
\item \( f(4) = 3 \)
\item \( f(4) \text{ is close to or exactly } 3 \)
\item \( f(3) = 4 \)
\item \( \text{None of the above are always true.} \)

\end{enumerate} }
\litem{
Based on the information below, which of the following statements is always true?
\begin{center}
    \textit{ $f(x)$ approaches $13.392$ as $x$ approaches $\infty$. }
\end{center}
\begin{enumerate}[label=\Alph*.]
\item \( f(x) \text{ is close to or exactly } \infty \text{ when } x \text{ is large enough}. \)
\item \( f(x) \text{ is close to or exactly } 13.392 \text{ when } x \text{ is large enough}. \)
\item \( x \text{ is undefined when } f(x) \text{ is large enough}. \)
\item \( f(x) \text{ is undefined when } x \text{ is large enough}. \)
\item \( \text{None of the above are always true.} \)

\end{enumerate} }
\litem{
To estimate the one-sided limit of the function below as $x$ approaches 6 from the right, which of the following sets of numbers should you use?\[ \frac{\frac{6}{x} - 1}{x - 6} \]\begin{enumerate}[label=\Alph*.]
\item \( \{ 6.0000, 6.1000, 6.0100, 6.0010 \} \)
\item \( \{ 6.0000, 5.9000, 5.9900, 5.9990 \} \)
\item \( \{ 6.1000, 6.0100, 6.0010, 6.0001 \} \)
\item \( \{ 5.9000, 5.9900, 6.0100, 6.1000 \} \)
\item \( \{ 5.9000, 5.9900, 5.9990, 5.9999 \} \)

\end{enumerate} }
\litem{
Evaluate the one-sided limit of the function $f(x)$ below, if possible.\[ \lim_{x \rightarrow -1^-} \frac{2}{(x+1)^3}+7 \]\begin{enumerate}[label=\Alph*.]
\item \( f(-1) \)
\item \( \infty \)
\item \( -\infty \)
\item \( \text{The limit does not exist} \)
\item \( \text{None of the above} \)

\end{enumerate} }
\litem{
Evaluate the limit below, if possible.\[ \lim_{x \rightarrow 5} \frac{\sqrt{5x - 9} - 4}{7x - 35} \]\begin{enumerate}[label=\Alph*.]
\item \( 0.089 \)
\item \( 0.125 \)
\item \( 0.319 \)
\item \( \infty \)
\item \( \text{None of the above} \)

\end{enumerate} }
\litem{
For the graph below, find the value(s) $a$ that makes the statement true: $ \displaystyle \lim_{x \rightarrow a} f(x)$ does not exist.
\begin{center}
    \includegraphics[width=0.5\textwidth]{../Figures/evaluateLimitGraphicallyCopyA.png}
\end{center}
\begin{enumerate}[label=\Alph*.]
\item \( 3 \)
\item \( 1 \)
\item \( -2 \)
\item \( \text{Multiple } a \text{ make the statement true}. \)
\item \( \text{No } a \text{ make the statement true}. \)

\end{enumerate} }
\litem{
For the graph below, find the value(s) $a$ that makes the statement true: $ \displaystyle \lim_{x \rightarrow a} f(x)$ does not exist.
\begin{center}
    \includegraphics[width=0.5\textwidth]{../Figures/evaluateLimitGraphicallyB.png}
\end{center}
\begin{enumerate}[label=\Alph*.]
\item \( 1 \)
\item \( -2 \)
\item \( 3 \)
\item \( \text{Multiple } a \text{ make the statement true}. \)
\item \( \text{No } a \text{ make the statement true}. \)

\end{enumerate} }
\litem{
Evaluate the limit below, if possible.\[ \lim_{x \rightarrow 7} \frac{\sqrt{7x - 33} - 4}{2x - 14} \]\begin{enumerate}[label=\Alph*.]
\item \( 0.062 \)
\item \( 1.323 \)
\item \( \infty \)
\item \( 0.125 \)
\item \( \text{None of the above} \)

\end{enumerate} }
\litem{
Evaluate the one-sided limit of the function $f(x)$ below, if possible.\[ \lim_{x \rightarrow -9^+} \frac{-2}{(x+9)^6}+3 \]\begin{enumerate}[label=\Alph*.]
\item \( \infty \)
\item \( f(-9) \)
\item \( -\infty \)
\item \( \text{The limit does not exist} \)
\item \( \text{None of the above} \)

\end{enumerate} }
\litem{
To estimate the one-sided limit of the function below as $x$ approaches 4 from the left, which of the following sets of numbers should you use?\[ \frac{\frac{4}{x} - 1}{x - 4} \]\begin{enumerate}[label=\Alph*.]
\item \( \{ 4.0000, 4.1000, 4.0100, 4.0010 \} \)
\item \( \{ 4.0000, 3.9000, 3.9900, 3.9990 \} \)
\item \( \{ 3.9000, 3.9900, 3.9990, 3.9999 \} \)
\item \( \{ 4.1000, 4.0100, 4.0010, 4.0001 \} \)
\item \( \{ 3.9000, 3.9900, 4.0100, 4.1000 \} \)

\end{enumerate} }
\litem{
Based on the information below, which of the following statements is always true?
\begin{center}
    \textit{ As $x$ approaches $\infty$, $f(x)$ approaches $12.374$. }
\end{center}
\begin{enumerate}[label=\Alph*.]
\item \( x \text{ is undefined when } f(x) \text{ is large enough}. \)
\item \( f(x) \text{ is close to or exactly } 12.374 \text{ when } x \text{ is large enough}. \)
\item \( f(x) \text{ is close to or exactly } \infty \text{ when } x \text{ is large enough}. \)
\item \( f(x) \text{ is undefined when } x \text{ is large enough}. \)
\item \( \text{None of the above are always true.} \)

\end{enumerate} }
\litem{
Based on the information below, which of the following statements is always true?
\begin{center}
    \textit{ $f(x)$ approaches $18.962$ as $x$ approaches $\infty$. }
\end{center}
\begin{enumerate}[label=\Alph*.]
\item \( f(x) \text{ is undefined when } x \text{ is large enough}. \)
\item \( f(x) \text{ is close to or exactly } \infty \text{ when } x \text{ is large enough}. \)
\item \( f(x) \text{ is close to or exactly } 18.962 \text{ when } x \text{ is large enough}. \)
\item \( x \text{ is undefined when } f(x) \text{ is large enough}. \)
\item \( \text{None of the above are always true.} \)

\end{enumerate} }
\litem{
To estimate the one-sided limit of the function below as $x$ approaches 1 from the left, which of the following sets of numbers should you use?\[ \frac{\frac{1}{x} - 1}{x - 1} \]\begin{enumerate}[label=\Alph*.]
\item \( \{ 1.0000, 0.9000, 0.9900, 0.9990 \} \)
\item \( \{ 1.0000, 1.1000, 1.0100, 1.0010 \} \)
\item \( \{ 0.9000, 0.9900, 0.9990, 0.9999 \} \)
\item \( \{ 1.1000, 1.0100, 1.0010, 1.0001 \} \)
\item \( \{ 0.9000, 0.9900, 1.0100, 1.1000 \} \)

\end{enumerate} }
\litem{
Evaluate the one-sided limit of the function $f(x)$ below, if possible.\[ \lim_{x \rightarrow 5^+} \frac{-3}{(x+5)^7}+2 \]\begin{enumerate}[label=\Alph*.]
\item \( \infty \)
\item \( -\infty \)
\item \( f(5) \)
\item \( \text{The limit does not exist} \)
\item \( \text{None of the above} \)

\end{enumerate} }
\litem{
Evaluate the limit below, if possible.\[ \lim_{x \rightarrow 5} \frac{\sqrt{9x - 29} - 4}{6x - 30} \]\begin{enumerate}[label=\Alph*.]
\item \( \infty \)
\item \( 0.125 \)
\item \( 0.021 \)
\item \( 0.500 \)
\item \( \text{None of the above} \)

\end{enumerate} }
\litem{
For the graph below, find the value(s) $a$ that makes the statement true: $ \displaystyle \lim_{x \rightarrow a} f(x) = 3$.
\begin{center}
    \includegraphics[width=0.5\textwidth]{../Figures/evaluateLimitGraphicallyCopyB.png}
\end{center}
\begin{enumerate}[label=\Alph*.]
\item \( -2 \)
\item \( -\infty \)
\item \( 1 \)
\item \( \text{Multiple } a \text{ make the statement true}. \)
\item \( \text{No } a \text{ make the statement true}. \)

\end{enumerate} }
\litem{
For the graph below, find the value(s) $a$ that makes the statement true: $ \displaystyle \lim_{x \rightarrow a} f(x) = 3$.
\begin{center}
    \includegraphics[width=0.5\textwidth]{../Figures/evaluateLimitGraphicallyC.png}
\end{center}
\begin{enumerate}[label=\Alph*.]
\item \( -2 \)
\item \( 1 \)
\item \( -\infty \)
\item \( \text{Multiple } a \text{ make the statement true}. \)
\item \( \text{No } a \text{ make the statement true}. \)

\end{enumerate} }
\litem{
Evaluate the limit below, if possible.\[ \lim_{x \rightarrow 9} \frac{\sqrt{7x - 14} - 7}{2x - 18} \]\begin{enumerate}[label=\Alph*.]
\item \( 1.323 \)
\item \( \infty \)
\item \( 0.036 \)
\item \( 0.071 \)
\item \( \text{None of the above} \)

\end{enumerate} }
\litem{
Evaluate the one-sided limit of the function $f(x)$ below, if possible.\[ \lim_{x \rightarrow 1^+} \frac{1}{(x-1)^8}+4 \]\begin{enumerate}[label=\Alph*.]
\item \( -\infty \)
\item \( f(1) \)
\item \( \infty \)
\item \( \text{The limit does not exist} \)
\item \( \text{None of the above} \)

\end{enumerate} }
\litem{
To estimate the one-sided limit of the function below as $x$ approaches 8 from the left, which of the following sets of numbers should you use?\[ \frac{\frac{8}{x} - 1}{x - 8} \]\begin{enumerate}[label=\Alph*.]
\item \( \{ 7.9000, 7.9900, 8.0100, 8.1000 \} \)
\item \( \{ 8.1000, 8.0100, 8.0010, 8.0001 \} \)
\item \( \{ 7.9000, 7.9900, 7.9990, 7.9999 \} \)
\item \( \{ 8.0000, 7.9000, 7.9900, 7.9990 \} \)
\item \( \{ 8.0000, 8.1000, 8.0100, 8.0010 \} \)

\end{enumerate} }
\litem{
Based on the information below, which of the following statements is always true?
\begin{center}
    \textit{ As $x$ approaches $9$, $f(x)$ approaches $7.206$. }
\end{center}
\begin{enumerate}[label=\Alph*.]
\item \( f(x) = 9 \text{ when } x \text{ is close to } 7.206 \)
\item \( f(x) = 7.206 \text{ when } x \text{ is close to } 9 \)
\item \( f(x) \text{ is close to or exactly } 9 \text{ when } x \text{ is close to } 7.206 \)
\item \( f(x) \text{ is close to or exactly } 7.206 \text{ when } x \text{ is close to } 9 \)
\item \( \text{None of the above are always true.} \)

\end{enumerate} }
\litem{
Based on the information below, which of the following statements is always true?
\begin{center}
    \textit{ As $x$ approaches $9$, $f(x)$ approaches $8.194$. }
\end{center}
\begin{enumerate}[label=\Alph*.]
\item \( f(8) \text{ is close to or exactly } 9 \)
\item \( f(9) \text{ is close to or exactly } 8 \)
\item \( f(9) = 8 \)
\item \( f(8) = 9 \)
\item \( \text{None of the above are always true.} \)

\end{enumerate} }
\litem{
To estimate the one-sided limit of the function below as $x$ approaches 3 from the left, which of the following sets of numbers should you use?\[ \frac{\frac{3}{x} - 1}{x - 3} \]\begin{enumerate}[label=\Alph*.]
\item \( \{ 2.9000, 2.9900, 3.0100, 3.1000 \} \)
\item \( \{ 3.0000, 2.9000, 2.9900, 2.9990 \} \)
\item \( \{ 3.1000, 3.0100, 3.0010, 3.0001 \} \)
\item \( \{ 3.0000, 3.1000, 3.0100, 3.0010 \} \)
\item \( \{ 2.9000, 2.9900, 2.9990, 2.9999 \} \)

\end{enumerate} }
\litem{
Evaluate the one-sided limit of the function $f(x)$ below, if possible.\[ \lim_{x \rightarrow -1^-} \frac{8}{(x+1)^5}+1 \]\begin{enumerate}[label=\Alph*.]
\item \( \infty \)
\item \( f(-1) \)
\item \( -\infty \)
\item \( \text{The limit does not exist} \)
\item \( \text{None of the above} \)

\end{enumerate} }
\litem{
Evaluate the limit below, if possible.\[ \lim_{x \rightarrow 8} \frac{\sqrt{3x - 8} - 4}{6x - 48} \]\begin{enumerate}[label=\Alph*.]
\item \( \infty \)
\item \( 0.021 \)
\item \( 0.289 \)
\item \( 0.125 \)
\item \( \text{None of the above} \)

\end{enumerate} }
\litem{
For the graph below, find the value(s) $a$ that makes the statement true: $ \displaystyle \lim_{x \rightarrow a} f(x) = 0$.
\begin{center}
    \includegraphics[width=0.5\textwidth]{../Figures/evaluateLimitGraphicallyCopyC.png}
\end{center}
\begin{enumerate}[label=\Alph*.]
\item \( 3 \)
\item \( -4 \)
\item \( 0 \)
\item \( \text{Multiple } a \text{ make the statement true}. \)
\item \( \text{No } a \text{ make the statement true}. \)

\end{enumerate} }
\end{enumerate}

\end{document}