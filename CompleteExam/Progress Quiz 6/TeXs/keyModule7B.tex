\documentclass{extbook}[14pt]
\usepackage{multicol, enumerate, enumitem, hyperref, color, soul, setspace, parskip, fancyhdr, amssymb, amsthm, amsmath, latexsym, units, mathtools}
\everymath{\displaystyle}
\usepackage[headsep=0.5cm,headheight=0cm, left=1 in,right= 1 in,top= 1 in,bottom= 1 in]{geometry}
\usepackage{dashrule}  % Package to use the command below to create lines between items
\newcommand{\litem}[1]{\item #1

\rule{\textwidth}{0.4pt}}
\pagestyle{fancy}
\lhead{}
\chead{Answer Key for Progress Quiz 6 Version B}
\rhead{}
\lfoot{4563-7456}
\cfoot{}
\rfoot{Summer C 2021}
\begin{document}
\textbf{This key should allow you to understand why you choose the option you did (beyond just getting a question right or wrong). \href{https://xronos.clas.ufl.edu/mac1105spring2020/courseDescriptionAndMisc/Exams/LearningFromResults}{More instructions on how to use this key can be found here}.}

\textbf{If you have a suggestion to make the keys better, \href{https://forms.gle/CZkbZmPbC9XALEE88}{please fill out the short survey here}.}

\textit{Note: This key is auto-generated and may contain issues and/or errors. The keys are reviewed after each exam to ensure grading is done accurately. If there are issues (like duplicate options), they are noted in the offline gradebook. The keys are a work-in-progress to give students as many resources to improve as possible.}

\rule{\textwidth}{0.4pt}

\begin{enumerate}\litem{
Choose the graph of the equation below.
\[ f(x) = \frac{1}{(x - 1)^2} + 1 \]The solution is the graph below, which is option E.
    \begin{center}
        \includegraphics[width=0.3\textwidth]{../Figures/rationalEquationToGraphEB.png}
    \end{center}\begin{enumerate}[label=\Alph*.]
\begin{multicols}{2}
\item \includegraphics[width = 0.3\textwidth]{../Figures/rationalEquationToGraphAB.png}
\item \includegraphics[width = 0.3\textwidth]{../Figures/rationalEquationToGraphBB.png}
\item \includegraphics[width = 0.3\textwidth]{../Figures/rationalEquationToGraphCB.png}
\item \includegraphics[width = 0.3\textwidth]{../Figures/rationalEquationToGraphDB.png}
\end{multicols}\item None of the above.\end{enumerate}
\textbf{General Comment:} Remember that the general form of a basic rational equation is $ f(x) = \frac{a}{(x-h)^n} + k$, where $a$ is the leading coefficient (and in this case, we assume is either $1$ or $-1$), $n$ is the degree (in this case, either $1$ or $2$), and $(h, k)$ is the intersection of the asymptotes.
}
\litem{
Choose the equation of the function graphed below.

\begin{center}
    \includegraphics[width=0.5\textwidth]{../Figures/rationalGraphToEquationB.png}
\end{center}


The solution is \( \text{None of the above as it should be } f(x) = \frac{-1}{x + 3} - 3 \), which is option E.\begin{enumerate}[label=\Alph*.]
\item \( f(x) = \frac{1}{(x - 3)^2} - 4 \)

Corresponds to thinking the graph was a shifted version of $\frac{1}{x^2}$, using the general form $f(x) = \frac{a}{x+h}+k$, the opposite leading coefficient, AND not noticing the $y$-value was wrong.
\item \( f(x) = \frac{1}{x - 3} - 4 \)

Corresponds to using the general form $f(x) = \frac{a}{x+h}+k$, the opposite leading coefficient AND not noticing the $y$-value was wrong.
\item \( f(x) = \frac{-1}{x + 3} - 4 \)

The $y$-value of the equation does not match the graph.
\item \( f(x) = \frac{-1}{(x + 3)^2} - 4 \)

Corresponds to thinking the graph was a shifted version of $\frac{1}{x^2}$ not noticing the $y$-value was wrong.
\item \( \text{None of the above} \)

None of the equation options were the correct equation.
\end{enumerate}

\textbf{General Comment:} Remember that the general form of a basic rational equation is $ f(x) = \frac{a}{(x-h)^n} + k$, where $a$ is the leading coefficient (and in this case, we assume is either $1$ or $-1$), $n$ is the degree (in this case, either $1$ or $2$), and $(h, k)$ is the intersection of the asymptotes.
}
\litem{
Solve the rational equation below. Then, choose the interval(s) that the solution(s) belongs to.
\[ \frac{5x}{5x -7} + \frac{-3x^{2}}{-20x^{2} +43 x -21} = \frac{-4}{-4x + 3} \]The solution is \( \text{All solutions are invalid or lead to complex values in the equation.} \), which is option B.\begin{enumerate}[label=\Alph*.]
\item \( x_1 \in [1.73, 2.02] \text{ and } x_2 \in [0.03,0.4] \)

$x = 1.796 \text{ and } x = 0.263$, which corresponds to making the discriminant from the Quadratic Formula positive to avoid complex solutions.
\item \( \text{All solutions lead to invalid or complex values in the equation.} \)

* The equation leads to solving $-17x^{2} +35 x -28=0$, which leads to complex solutions. This is the correct option.
\item \( x_1 \in [1.08, 1.47] \text{ and } x_2 \in [0.57,0.79] \)

$x = 1.400 \text{ and } x = 0.750$, which corresponds to solving $5x -7 = 0$ and $-4x + 3 = 0$ and treating them as solutions to the equation.
\item \( x \in [1.08,1.47] \)

$x = 1.400$, which corresponds to solving $5x -7 = 0$ and treating it as a solution to the equation.
\item \( x \in [-0.56,0.93] \)

$x = 0.750$, which corresponds to solving $-4x + 3 = 0$ and treating it as a solution to the equation.
\end{enumerate}

\textbf{General Comment:} Distractors are different based on the number of solutions. Remember that after solving, we need to make sure our solution does not make the original equation divide by zero!
}
\litem{
Solve the rational equation below. Then, choose the interval(s) that the solution(s) belongs to.
\[ \frac{72}{18x + 18} + 1 = \frac{72}{18x + 18} \]The solution is \( \text{all solutions are invalid or lead to complex values in the equation.} \), which is option A.\begin{enumerate}[label=\Alph*.]
\item \( \text{All solutions lead to invalid or complex values in the equation.} \)

*$x = -1.000$ leads to dividing by 0 in the original equation and thus is not a valid solution, which is the correct option.
\item \( x_1 \in [-1, 0] \text{ and } x_2 \in [-1,0] \)

$x = -1.000 \text{ and } x = -1.000$, which corresponds to getting the correct solution and believing there should be a second solution to the equation.
\item \( x \in [0,3] \)

$x = 1.000$, which corresponds to not distributing the factor $18x + 18$ correctly when trying to eliminate the fraction.
\item \( x_1 \in [-1, 0] \text{ and } x_2 \in [1,2] \)

$x = -1.000 \text{ and } x = 1.000$, which corresponds to getting the correct solution and believing there should be a second solution to the equation.
\item \( x \in [-1.0,0.0] \)

$x = -1.000$, which corresponds to not checking if this value leads to dividing by 0 in the original equation and thus is not a valid solution.
\end{enumerate}

\textbf{General Comment:} Distractors are different based on the number of solutions. Remember that after solving, we need to make sure our solution does not make the original equation divide by zero!
}
\litem{
Choose the equation of the function graphed below.

\begin{center}
    \includegraphics[width=0.5\textwidth]{../Figures/rationalGraphToEquationCopyB.png}
\end{center}


The solution is \( f(x) = \frac{1}{x + 3} + 3 \), which is option C.\begin{enumerate}[label=\Alph*.]
\item \( f(x) = \frac{1}{(x + 3)^2} + 3 \)

Corresponds to thinking the graph was a shifted version of $\frac{1}{x^2}$.
\item \( f(x) = \frac{-1}{(x - 3)^2} + 3 \)

Corresponds to thinking the graph was a shifted version of $\frac{1}{x^2}$, using the general form $f(x) = \frac{a}{x+h}+k$, and the opposite leading coefficient.
\item \( f(x) = \frac{1}{x + 3} + 3 \)

This is the correct option.
\item \( f(x) = \frac{-1}{x - 3} + 3 \)

Corresponds to using the general form $f(x) = \frac{a}{x+h}+k$ and the opposite leading coefficient.
\item \( \text{None of the above} \)

This corresponds to believing the vertex of the graph was not correct.
\end{enumerate}

\textbf{General Comment:} Remember that the general form of a basic rational equation is $ f(x) = \frac{a}{(x-h)^n} + k$, where $a$ is the leading coefficient (and in this case, we assume is either $1$ or $-1$), $n$ is the degree (in this case, either $1$ or $2$), and $(h, k)$ is the intersection of the asymptotes.
}
\litem{
Solve the rational equation below. Then, choose the interval(s) that the solution(s) belongs to.
\[ \frac{-8}{-6x + 2} + -9 = \frac{-9}{48x -16} \]The solution is \( x = 0.502 \), which is option A.\begin{enumerate}[label=\Alph*.]
\item \( x \in [0.5,2.5] \)

* $x = 0.502$, which is the correct option.
\item \( x_1 \in [-0.29, 0.02] \text{ and } x_2 \in [-1.5,3.5] \)

$x = -0.164 \text{ and } x = 0.502$, which corresponds to getting the correct solution and believing there should be a second solution to the equation.
\item \( \text{All solutions lead to invalid or complex values in the equation.} \)

This corresponds to thinking $x = 0.502$ leads to dividing by zero in the original equation, which it does not.
\item \( x \in [-0.29,0.02] \)

$x = -0.164$, which corresponds to not distributing the factor $-6x + 2$ correctly when trying to eliminate the fraction.
\item \( x_1 \in [0.09, 0.41] \text{ and } x_2 \in [-1.5,3.5] \)

$x = 0.315 \text{ and } x = 0.502$, which corresponds to getting the correct solution and believing there should be a second solution to the equation.
\end{enumerate}

\textbf{General Comment:} Distractors are different based on the number of solutions. Remember that after solving, we need to make sure our solution does not make the original equation divide by zero!
}
\litem{
Choose the graph of the equation below.
\[ f(x) = \frac{1}{x + 2} - 2 \]The solution is the graph below, which is option E.
    \begin{center}
        \includegraphics[width=0.3\textwidth]{../Figures/rationalEquationToGraphCopyEB.png}
    \end{center}\begin{enumerate}[label=\Alph*.]
\begin{multicols}{2}
\item \includegraphics[width = 0.3\textwidth]{../Figures/rationalEquationToGraphCopyAB.png}
\item \includegraphics[width = 0.3\textwidth]{../Figures/rationalEquationToGraphCopyBB.png}
\item \includegraphics[width = 0.3\textwidth]{../Figures/rationalEquationToGraphCopyCB.png}
\item \includegraphics[width = 0.3\textwidth]{../Figures/rationalEquationToGraphCopyDB.png}
\end{multicols}\item None of the above.\end{enumerate}
\textbf{General Comment:} Remember that the general form of a basic rational equation is $ f(x) = \frac{a}{(x-h)^n} + k$, where $a$ is the leading coefficient (and in this case, we assume is either $1$ or $-1$), $n$ is the degree (in this case, either $1$ or $2$), and $(h, k)$ is the intersection of the asymptotes.
}
\litem{
Determine the domain of the function below.
\[ f(x) = \frac{4}{20x^{2} -54 x + 36} \]The solution is \( \text{All Real numbers except } x = 1.200 \text{ and } x = 1.500. \), which is option D.\begin{enumerate}[label=\Alph*.]
\item \( \text{All Real numbers.} \)

This corresponds to thinking the denominator has complex roots or that rational functions have a domain of all Real numbers.
\item \( \text{All Real numbers except } x = a \text{ and } x = b, \text{ where } a \in [23.82, 24.02] \text{ and } b \in [29.77, 30.1] \)

All Real numbers except $x = 24.000$ and $x = 30.000$, which corresponds to not factoring the denominator correctly.
\item \( \text{All Real numbers except } x = a, \text{ where } a \in [23.82, 24.02] \)

All Real numbers except $x = 24.000$, which corresponds to removing a distractor value from the denominator.
\item \( \text{All Real numbers except } x = a \text{ and } x = b, \text{ where } a \in [0.78, 1.31] \text{ and } b \in [1.33, 1.73] \)

All Real numbers except $x = 1.200$ and $x = 1.500$, which is the correct option.
\item \( \text{All Real numbers except } x = a, \text{ where } a \in [0.78, 1.31] \)

All Real numbers except $x = 1.200$, which corresponds to removing only 1 value from the denominator.
\end{enumerate}

\textbf{General Comment:} Recall that dividing by zero is not a real number. Therefore the domain is all real numbers \textbf{except} those that make the denominator 0.
}
\litem{
Determine the domain of the function below.
\[ f(x) = \frac{3}{12x^{2} -42 x + 36} \]The solution is \( \text{All Real numbers except } x = 1.500 \text{ and } x = 2.000. \), which is option C.\begin{enumerate}[label=\Alph*.]
\item \( \text{All Real numbers except } x = a, \text{ where } a \in [17.7, 20.1] \)

All Real numbers except $x = 18.000$, which corresponds to removing a distractor value from the denominator.
\item \( \text{All Real numbers.} \)

This corresponds to thinking the denominator has complex roots or that rational functions have a domain of all Real numbers.
\item \( \text{All Real numbers except } x = a \text{ and } x = b, \text{ where } a \in [0, 1.6] \text{ and } b \in [1.8, 2.3] \)

All Real numbers except $x = 1.500$ and $x = 2.000$, which is the correct option.
\item \( \text{All Real numbers except } x = a, \text{ where } a \in [0, 1.6] \)

All Real numbers except $x = 1.500$, which corresponds to removing only 1 value from the denominator.
\item \( \text{All Real numbers except } x = a \text{ and } x = b, \text{ where } a \in [17.7, 20.1] \text{ and } b \in [22.7, 24.9] \)

All Real numbers except $x = 18.000$ and $x = 24.000$, which corresponds to not factoring the denominator correctly.
\end{enumerate}

\textbf{General Comment:} Recall that dividing by zero is not a real number. Therefore the domain is all real numbers \textbf{except} those that make the denominator 0.
}
\litem{
Solve the rational equation below. Then, choose the interval(s) that the solution(s) belongs to.
\[ \frac{-4x}{3x + 3} + \frac{-2x^{2}}{-21x^{2} -12 x + 9} = \frac{7}{-7x + 3} \]The solution is \( \text{There are two solutions: } x = -0.466 \text{ and } x = 1.735 \), which is option C.\begin{enumerate}[label=\Alph*.]
\item \( x_1 \in [-0.53, -0.32] \text{ and } x_2 \in [-1.3,-0.4] \)


\item \( \text{All solutions lead to invalid or complex values in the equation.} \)


\item \( x_1 \in [-0.53, -0.32] \text{ and } x_2 \in [0.1,1.9] \)

* $x = -0.466 \text{ and } x = 1.735$, which is the correct option.
\item \( x \in [0.36,0.88] \)


\item \( x \in [1.51,2.18] \)


\end{enumerate}

\textbf{General Comment:} Distractors are different based on the number of solutions. Remember that after solving, we need to make sure our solution does not make the original equation divide by zero!
}
\end{enumerate}

\end{document}