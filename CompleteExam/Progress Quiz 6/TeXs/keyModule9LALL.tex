\documentclass{extbook}[14pt]
\usepackage{multicol, enumerate, enumitem, hyperref, color, soul, setspace, parskip, fancyhdr, amssymb, amsthm, amsmath, latexsym, units, mathtools}
\everymath{\displaystyle}
\usepackage[headsep=0.5cm,headheight=0cm, left=1 in,right= 1 in,top= 1 in,bottom= 1 in]{geometry}
\usepackage{dashrule}  % Package to use the command below to create lines between items
\newcommand{\litem}[1]{\item #1

\rule{\textwidth}{0.4pt}}
\pagestyle{fancy}
\lhead{}
\chead{Answer Key for Progress Quiz 6 Version ALL}
\rhead{}
\lfoot{4563-7456}
\cfoot{}
\rfoot{Summer C 2021}
\begin{document}
\textbf{This key should allow you to understand why you choose the option you did (beyond just getting a question right or wrong). \href{https://xronos.clas.ufl.edu/mac1105spring2020/courseDescriptionAndMisc/Exams/LearningFromResults}{More instructions on how to use this key can be found here}.}

\textbf{If you have a suggestion to make the keys better, \href{https://forms.gle/CZkbZmPbC9XALEE88}{please fill out the short survey here}.}

\textit{Note: This key is auto-generated and may contain issues and/or errors. The keys are reviewed after each exam to ensure grading is done accurately. If there are issues (like duplicate options), they are noted in the offline gradebook. The keys are a work-in-progress to give students as many resources to improve as possible.}

\rule{\textwidth}{0.4pt}

\begin{enumerate}\litem{
Choose the interval below that $f$ composed with $g$ at $x=1$ is in.
\[ f(x) = -2x^{3} +2 x^{2} -3 x \text{ and } g(x) = 2x^{3} +2 x^{2} -2 x \]The solution is \( -14.0 \), which is option B.\begin{enumerate}[label=\Alph*.]
\item \( (f \circ g)(1) \in [-36, -34] \)

 Distractor 3: Corresponds to being slightly off from the solution.
\item \( (f \circ g)(1) \in [-19, -13] \)

* This is the correct solution
\item \( (f \circ g)(1) \in [-10, 3] \)

 Distractor 2: Corresponds to being slightly off from the solution.
\item \( (f \circ g)(1) \in [-31, -29] \)

 Distractor 1: Corresponds to reversing the composition.
\item \( \text{It is not possible to compose the two functions.} \)


\end{enumerate}

\textbf{General Comment:} $f$ composed with $g$ at $x$ means $f(g(x))$. The order matters!
}
\litem{
Choose the interval below that $f$ composed with $g$ at $x=-1$ is in.
\[ f(x) = x^{3} -4 x^{2} -4 x -2 \text{ and } g(x) = -4x^{3} -1 x^{2} +2 x + 2 \]The solution is \( -23.0 \), which is option A.\begin{enumerate}[label=\Alph*.]
\item \( (f \circ g)(-1) \in [-24, -18] \)

* This is the correct solution
\item \( (f \circ g)(-1) \in [89, 99] \)

 Distractor 1: Corresponds to reversing the composition.
\item \( (f \circ g)(-1) \in [97, 107] \)

 Distractor 3: Corresponds to being slightly off from the solution.
\item \( (f \circ g)(-1) \in [-31, -26] \)

 Distractor 2: Corresponds to being slightly off from the solution.
\item \( \text{It is not possible to compose the two functions.} \)


\end{enumerate}

\textbf{General Comment:} $f$ composed with $g$ at $x$ means $f(g(x))$. The order matters!
}
\litem{
Determine whether the function below is 1-1.
\[ f(x) = -16 x^2 - 24 x + 247 \]The solution is \( \text{no} \), which is option B.\begin{enumerate}[label=\Alph*.]
\item \( \text{No, because the domain of the function is not $(-\infty, \infty)$.} \)

Corresponds to believing 1-1 means the domain is all Real numbers.
\item \( \text{No, because there is a $y$-value that goes to 2 different $x$-values.} \)

* This is the solution.
\item \( \text{No, because there is an $x$-value that goes to 2 different $y$-values.} \)

Corresponds to the Vertical Line test, which checks if an expression is a function.
\item \( \text{Yes, the function is 1-1.} \)

Corresponds to believing the function passes the Horizontal Line test.
\item \( \text{No, because the range of the function is not $(-\infty, \infty)$.} \)

Corresponds to believing 1-1 means the range is all Real numbers.
\end{enumerate}

\textbf{General Comment:} There are only two valid options: The function is 1-1 OR No because there is a $y$-value that goes to 2 different $x$-values.
}
\litem{
Find the inverse of the function below. Then, evaluate the inverse at $x = 10$ and choose the interval that $f^-1(10)$ belongs to.
\[ f(x) = e^{x+2}-5 \]The solution is \( f^{-1}(10) = 0.708 \), which is option E.\begin{enumerate}[label=\Alph*.]
\item \( f^{-1}(10) \in [-3.55, -3.34] \)

 This solution corresponds to distractor 2.
\item \( f^{-1}(10) \in [-3.16, -2.67] \)

 This solution corresponds to distractor 3.
\item \( f^{-1}(10) \in [4.47, 4.86] \)

 This solution corresponds to distractor 1.
\item \( f^{-1}(10) \in [-2.54, -2.41] \)

 This solution corresponds to distractor 4.
\item \( f^{-1}(10) \in [0.55, 0.96] \)

 This is the solution.
\end{enumerate}

\textbf{General Comment:} Natural log and exponential functions always have an inverse. Once you switch the $x$ and $y$, use the conversion $ e^y = x \leftrightarrow y=\ln(x)$.
}
\litem{
Find the inverse of the function below (if it exists). Then, evaluate the inverse at $x = 10$ and choose the interval that $f^-1(10)$ belongs to.
\[ f(x) = 2 x^2 - 4 \]The solution is \( \text{ The function is not invertible for all Real numbers. } \), which is option E.\begin{enumerate}[label=\Alph*.]
\item \( f^{-1}(10) \in [1.77, 2.8] \)

 Distractor 1: This corresponds to trying to find the inverse even though the function is not 1-1. 
\item \( f^{-1}(10) \in [2.88, 4.02] \)

 Distractor 3: This corresponds to finding the (nonexistent) inverse and dividing by a negative.
\item \( f^{-1}(10) \in [6.38, 7.96] \)

 Distractor 4: This corresponds to both distractors 2 and 3.
\item \( f^{-1}(10) \in [0.91, 2.03] \)

 Distractor 2: This corresponds to finding the (nonexistent) inverse and not subtracting by the vertical shift.
\item \( \text{ The function is not invertible for all Real numbers. } \)

* This is the correct option.
\end{enumerate}

\textbf{General Comment:} Be sure you check that the function is 1-1 before trying to find the inverse!
}
\litem{
Find the inverse of the function below. Then, evaluate the inverse at $x = 7$ and choose the interval that $f^-1(7)$ belongs to.
\[ f(x) = \ln{(x+5)}+3 \]The solution is \( f^{-1}(7) = 49.598 \), which is option D.\begin{enumerate}[label=\Alph*.]
\item \( f^{-1}(7) \in [162755.79, 162763.79] \)

 This solution corresponds to distractor 4.
\item \( f^{-1}(7) \in [58.6, 61.6] \)

 This solution corresponds to distractor 3.
\item \( f^{-1}(7) \in [22020.47, 22024.47] \)

 This solution corresponds to distractor 1.
\item \( f^{-1}(7) \in [47.6, 54.6] \)

 This is the solution.
\item \( f^{-1}(7) \in [7.39, 11.39] \)

 This solution corresponds to distractor 2.
\end{enumerate}

\textbf{General Comment:} Natural log and exponential functions always have an inverse. Once you switch the $x$ and $y$, use the conversion $ e^y = x \leftrightarrow y=\ln(x)$.
}
\litem{
Find the inverse of the function below (if it exists). Then, evaluate the inverse at $x = 14$ and choose the interval that $f^-1(14)$ belongs to.
\[ f(x) = \sqrt[3]{3 x - 4} \]The solution is \( 916.0 \), which is option C.\begin{enumerate}[label=\Alph*.]
\item \( f^{-1}(14) \in [-916.4, -914.3] \)

 This solution corresponds to distractor 2.
\item \( f^{-1}(14) \in [-913.6, -911.7] \)

 This solution corresponds to distractor 3.
\item \( f^{-1}(14) \in [914.9, 919.4] \)

* This is the correct solution.
\item \( f^{-1}(14) \in [911.6, 915.6] \)

 Distractor 1: This corresponds to 
\item \( \text{ The function is not invertible for all Real numbers. } \)

 This solution corresponds to distractor 4.
\end{enumerate}

\textbf{General Comment:} Be sure you check that the function is 1-1 before trying to find the inverse!
}
\litem{
Add the following functions, then choose the domain of the resulting function from the list below.
\[ f(x) = \frac{1}{4x+25} \text{ and } g(x) = \frac{4}{6x-29} \]The solution is \( \text{ The domain is all Real numbers except } x = -6.25 \text{ and } x = 4.83 \), which is option D.\begin{enumerate}[label=\Alph*.]
\item \( \text{ The domain is all Real numbers except } x = a, \text{ where } a \in [5.67, 14.67] \)


\item \( \text{ The domain is all Real numbers less than or equal to } x = a, \text{ where } a \in [0.33, 12.33] \)


\item \( \text{ The domain is all Real numbers greater than or equal to } x = a, \text{ where } a \in [-8.5, -4.5] \)


\item \( \text{ The domain is all Real numbers except } x = a \text{ and } x = b, \text{ where } a \in [-15.25, -2.25] \text{ and } b \in [2.83, 9.83] \)


\item \( \text{ The domain is all Real numbers. } \)


\end{enumerate}

\textbf{General Comment:} The new domain is the intersection of the previous domains.
}
\litem{
Determine whether the function below is 1-1.
\[ f(x) = (4 x - 18)^3 \]The solution is \( \text{yes} \), which is option B.\begin{enumerate}[label=\Alph*.]
\item \( \text{No, because there is a $y$-value that goes to 2 different $x$-values.} \)

Corresponds to the Horizontal Line test, which this function passes.
\item \( \text{Yes, the function is 1-1.} \)

* This is the solution.
\item \( \text{No, because there is an $x$-value that goes to 2 different $y$-values.} \)

Corresponds to the Vertical Line test, which checks if an expression is a function.
\item \( \text{No, because the range of the function is not $(-\infty, \infty)$.} \)

Corresponds to believing 1-1 means the range is all Real numbers.
\item \( \text{No, because the domain of the function is not $(-\infty, \infty)$.} \)

Corresponds to believing 1-1 means the domain is all Real numbers.
\end{enumerate}

\textbf{General Comment:} There are only two valid options: The function is 1-1 OR No because there is a $y$-value that goes to 2 different $x$-values.
}
\litem{
Add the following functions, then choose the domain of the resulting function from the list below.
\[ f(x) = 6x + 4 \text{ and } g(x) = \frac{1}{4x-21} \]The solution is \( \text{ The domain is all Real numbers except } x = 5.25 \), which is option C.\begin{enumerate}[label=\Alph*.]
\item \( \text{ The domain is all Real numbers less than or equal to } x = a, \text{ where } a \in [-1.5, 4.5] \)


\item \( \text{ The domain is all Real numbers greater than or equal to } x = a, \text{ where } a \in [-6.67, -0.67] \)


\item \( \text{ The domain is all Real numbers except } x = a, \text{ where } a \in [4.25, 8.25] \)


\item \( \text{ The domain is all Real numbers except } x = a \text{ and } x = b, \text{ where } a \in [2.83, 7.83] \text{ and } b \in [-7.33, 1.67] \)


\item \( \text{ The domain is all Real numbers. } \)


\end{enumerate}

\textbf{General Comment:} The new domain is the intersection of the previous domains.
}
\litem{
Choose the interval below that $f$ composed with $g$ at $x=-1$ is in.
\[ f(x) = -x^{3} +3 x^{2} +4 x \text{ and } g(x) = -4x^{3} -4 x^{2} +4 x + 3 \]The solution is \( 0.0 \), which is option B.\begin{enumerate}[label=\Alph*.]
\item \( (f \circ g)(-1) \in [-7, -4] \)

 Distractor 3: Corresponds to being slightly off from the solution.
\item \( (f \circ g)(-1) \in [-3, 1] \)

* This is the correct solution
\item \( (f \circ g)(-1) \in [3, 10] \)

 Distractor 1: Corresponds to reversing the composition.
\item \( (f \circ g)(-1) \in [-12, -8] \)

 Distractor 2: Corresponds to being slightly off from the solution.
\item \( \text{It is not possible to compose the two functions.} \)


\end{enumerate}

\textbf{General Comment:} $f$ composed with $g$ at $x$ means $f(g(x))$. The order matters!
}
\litem{
Choose the interval below that $f$ composed with $g$ at $x=1$ is in.
\[ f(x) = -2x^{3} +2 x^{2} -2 x \text{ and } g(x) = x^{3} -2 x^{2} +x \]The solution is \( 0.0 \), which is option A.\begin{enumerate}[label=\Alph*.]
\item \( (f \circ g)(1) \in [-0.7, 1.9] \)

* This is the correct solution
\item \( (f \circ g)(1) \in [-16.8, -11.2] \)

 Distractor 3: Corresponds to being slightly off from the solution.
\item \( (f \circ g)(1) \in [-5.9, -3.1] \)

 Distractor 2: Corresponds to being slightly off from the solution.
\item \( (f \circ g)(1) \in [-19.7, -15.7] \)

 Distractor 1: Corresponds to reversing the composition.
\item \( \text{It is not possible to compose the two functions.} \)


\end{enumerate}

\textbf{General Comment:} $f$ composed with $g$ at $x$ means $f(g(x))$. The order matters!
}
\litem{
Determine whether the function below is 1-1.
\[ f(x) = 15 x^2 - 189 x + 594 \]The solution is \( \text{no} \), which is option D.\begin{enumerate}[label=\Alph*.]
\item \( \text{No, because there is an $x$-value that goes to 2 different $y$-values.} \)

Corresponds to the Vertical Line test, which checks if an expression is a function.
\item \( \text{Yes, the function is 1-1.} \)

Corresponds to believing the function passes the Horizontal Line test.
\item \( \text{No, because the range of the function is not $(-\infty, \infty)$.} \)

Corresponds to believing 1-1 means the range is all Real numbers.
\item \( \text{No, because there is a $y$-value that goes to 2 different $x$-values.} \)

* This is the solution.
\item \( \text{No, because the domain of the function is not $(-\infty, \infty)$.} \)

Corresponds to believing 1-1 means the domain is all Real numbers.
\end{enumerate}

\textbf{General Comment:} There are only two valid options: The function is 1-1 OR No because there is a $y$-value that goes to 2 different $x$-values.
}
\litem{
Find the inverse of the function below. Then, evaluate the inverse at $x = 8$ and choose the interval that $f^-1(8)$ belongs to.
\[ f(x) = \ln{(x-2)}-5 \]The solution is \( f^{-1}(8) = 442415.392 \), which is option A.\begin{enumerate}[label=\Alph*.]
\item \( f^{-1}(8) \in [442414.39, 442417.39] \)

 This is the solution.
\item \( f^{-1}(8) \in [22016.47, 22027.47] \)

 This solution corresponds to distractor 2.
\item \( f^{-1}(8) \in [15.09, 25.09] \)

 This solution corresponds to distractor 1.
\item \( f^{-1}(8) \in [396.43, 399.43] \)

 This solution corresponds to distractor 4.
\item \( f^{-1}(8) \in [442405.39, 442412.39] \)

 This solution corresponds to distractor 3.
\end{enumerate}

\textbf{General Comment:} Natural log and exponential functions always have an inverse. Once you switch the $x$ and $y$, use the conversion $ e^y = x \leftrightarrow y=\ln(x)$.
}
\litem{
Find the inverse of the function below (if it exists). Then, evaluate the inverse at $x = 15$ and choose the interval that $f^-1(15)$ belongs to.
\[ f(x) = \sqrt[3]{4 x - 3} \]The solution is \( 844.5 \), which is option A.\begin{enumerate}[label=\Alph*.]
\item \( f^{-1}(15) \in [843.5, 844.8] \)

* This is the correct solution.
\item \( f^{-1}(15) \in [-847.1, -843.8] \)

 This solution corresponds to distractor 2.
\item \( f^{-1}(15) \in [841.1, 843.1] \)

 Distractor 1: This corresponds to 
\item \( f^{-1}(15) \in [-843.1, -839.4] \)

 This solution corresponds to distractor 3.
\item \( \text{ The function is not invertible for all Real numbers. } \)

 This solution corresponds to distractor 4.
\end{enumerate}

\textbf{General Comment:} Be sure you check that the function is 1-1 before trying to find the inverse!
}
\litem{
Find the inverse of the function below. Then, evaluate the inverse at $x = 9$ and choose the interval that $f^-1(9)$ belongs to.
\[ f(x) = e^{x+4}-3 \]The solution is \( f^{-1}(9) = -1.515 \), which is option B.\begin{enumerate}[label=\Alph*.]
\item \( f^{-1}(9) \in [-1.44, -1.23] \)

 This solution corresponds to distractor 3.
\item \( f^{-1}(9) \in [-1.58, -1.46] \)

 This is the solution.
\item \( f^{-1}(9) \in [-0.58, -0.38] \)

 This solution corresponds to distractor 4.
\item \( f^{-1}(9) \in [-1.36, -1.19] \)

 This solution corresponds to distractor 2.
\item \( f^{-1}(9) \in [6.47, 6.65] \)

 This solution corresponds to distractor 1.
\end{enumerate}

\textbf{General Comment:} Natural log and exponential functions always have an inverse. Once you switch the $x$ and $y$, use the conversion $ e^y = x \leftrightarrow y=\ln(x)$.
}
\litem{
Find the inverse of the function below (if it exists). Then, evaluate the inverse at $x = -11$ and choose the interval that $f^-1(-11)$ belongs to.
\[ f(x) = \sqrt[3]{2 x + 4} \]The solution is \( -667.5 \), which is option C.\begin{enumerate}[label=\Alph*.]
\item \( f^{-1}(-11) \in [-663.5, -660.5] \)

 Distractor 1: This corresponds to 
\item \( f^{-1}(-11) \in [662.5, 664.5] \)

 This solution corresponds to distractor 3.
\item \( f^{-1}(-11) \in [-674.5, -665.5] \)

* This is the correct solution.
\item \( f^{-1}(-11) \in [664.5, 668.5] \)

 This solution corresponds to distractor 2.
\item \( \text{ The function is not invertible for all Real numbers. } \)

 This solution corresponds to distractor 4.
\end{enumerate}

\textbf{General Comment:} Be sure you check that the function is 1-1 before trying to find the inverse!
}
\litem{
Multiply the following functions, then choose the domain of the resulting function from the list below.
\[ f(x) = 8x^{2} + 8 \text{ and } g(x) = \sqrt{3x+15}  \]The solution is \( \text{ The domain is all Real numbers greater than or equal to} x = -5.0. \), which is option A.\begin{enumerate}[label=\Alph*.]
\item \( \text{ The domain is all Real numbers greater than or equal to } x = a, \text{ where } a \in [-6, -1] \)


\item \( \text{ The domain is all Real numbers except } x = a, \text{ where } a \in [0.83, 5.83] \)


\item \( \text{ The domain is all Real numbers less than or equal to } x = a, \text{ where } a \in [-0.6, 8.4] \)


\item \( \text{ The domain is all Real numbers except } x = a \text{ and } x = b, \text{ where } a \in [-9.67, -1.67] \text{ and } b \in [-3.75, 1.25] \)


\item \( \text{ The domain is all Real numbers. } \)


\end{enumerate}

\textbf{General Comment:} The new domain is the intersection of the previous domains.
}
\litem{
Determine whether the function below is 1-1.
\[ f(x) = (5 x - 18)^3 \]The solution is \( \text{yes} \), which is option A.\begin{enumerate}[label=\Alph*.]
\item \( \text{Yes, the function is 1-1.} \)

* This is the solution.
\item \( \text{No, because the range of the function is not $(-\infty, \infty)$.} \)

Corresponds to believing 1-1 means the range is all Real numbers.
\item \( \text{No, because there is a $y$-value that goes to 2 different $x$-values.} \)

Corresponds to the Horizontal Line test, which this function passes.
\item \( \text{No, because the domain of the function is not $(-\infty, \infty)$.} \)

Corresponds to believing 1-1 means the domain is all Real numbers.
\item \( \text{No, because there is an $x$-value that goes to 2 different $y$-values.} \)

Corresponds to the Vertical Line test, which checks if an expression is a function.
\end{enumerate}

\textbf{General Comment:} There are only two valid options: The function is 1-1 OR No because there is a $y$-value that goes to 2 different $x$-values.
}
\litem{
Add the following functions, then choose the domain of the resulting function from the list below.
\[ f(x) = 9x^{3} +8 x^{2} +6 x \text{ and } g(x) = \sqrt{-3x+10}  \]The solution is \( \text{ The domain is all Real numbers less than or equal to} x = 3.33. \), which is option C.\begin{enumerate}[label=\Alph*.]
\item \( \text{ The domain is all Real numbers greater than or equal to } x = a, \text{ where } a \in [4.5, 10.5] \)


\item \( \text{ The domain is all Real numbers except } x = a, \text{ where } a \in [-8.25, 0.75] \)


\item \( \text{ The domain is all Real numbers less than or equal to } x = a, \text{ where } a \in [3.33, 4.33] \)


\item \( \text{ The domain is all Real numbers except } x = a \text{ and } x = b, \text{ where } a \in [3.75, 5.75] \text{ and } b \in [-6.2, -3.2] \)


\item \( \text{ The domain is all Real numbers. } \)


\end{enumerate}

\textbf{General Comment:} The new domain is the intersection of the previous domains.
}
\litem{
Choose the interval below that $f$ composed with $g$ at $x=-2$ is in.
\[ f(x) = -2x^{3} -4 x^{2} +x -1 \text{ and } g(x) = -2x^{3} -3 x^{2} +x \]The solution is \( -31.0 \), which is option B.\begin{enumerate}[label=\Alph*.]
\item \( (f \circ g)(-2) \in [31, 36] \)

 Distractor 3: Corresponds to being slightly off from the solution.
\item \( (f \circ g)(-2) \in [-34, -26] \)

* This is the correct solution
\item \( (f \circ g)(-2) \in [24, 26] \)

 Distractor 1: Corresponds to reversing the composition.
\item \( (f \circ g)(-2) \in [-38, -35] \)

 Distractor 2: Corresponds to being slightly off from the solution.
\item \( \text{It is not possible to compose the two functions.} \)


\end{enumerate}

\textbf{General Comment:} $f$ composed with $g$ at $x$ means $f(g(x))$. The order matters!
}
\litem{
Choose the interval below that $f$ composed with $g$ at $x=-1$ is in.
\[ f(x) = 4x^{3} +4 x^{2} -2 x \text{ and } g(x) = 2x^{3} -2 x^{2} -3 x -1 \]The solution is \( -12.0 \), which is option B.\begin{enumerate}[label=\Alph*.]
\item \( (f \circ g)(-1) \in [-2, 7] \)

 Distractor 1: Corresponds to reversing the composition.
\item \( (f \circ g)(-1) \in [-17, -9] \)

* This is the correct solution
\item \( (f \circ g)(-1) \in [5, 16] \)

 Distractor 3: Corresponds to being slightly off from the solution.
\item \( (f \circ g)(-1) \in [-8, -1] \)

 Distractor 2: Corresponds to being slightly off from the solution.
\item \( \text{It is not possible to compose the two functions.} \)


\end{enumerate}

\textbf{General Comment:} $f$ composed with $g$ at $x$ means $f(g(x))$. The order matters!
}
\litem{
Determine whether the function below is 1-1.
\[ f(x) = (5 x - 26)^3 \]The solution is \( \text{yes} \), which is option A.\begin{enumerate}[label=\Alph*.]
\item \( \text{Yes, the function is 1-1.} \)

* This is the solution.
\item \( \text{No, because the range of the function is not $(-\infty, \infty)$.} \)

Corresponds to believing 1-1 means the range is all Real numbers.
\item \( \text{No, because there is a $y$-value that goes to 2 different $x$-values.} \)

Corresponds to the Horizontal Line test, which this function passes.
\item \( \text{No, because there is an $x$-value that goes to 2 different $y$-values.} \)

Corresponds to the Vertical Line test, which checks if an expression is a function.
\item \( \text{No, because the domain of the function is not $(-\infty, \infty)$.} \)

Corresponds to believing 1-1 means the domain is all Real numbers.
\end{enumerate}

\textbf{General Comment:} There are only two valid options: The function is 1-1 OR No because there is a $y$-value that goes to 2 different $x$-values.
}
\litem{
Find the inverse of the function below. Then, evaluate the inverse at $x = 10$ and choose the interval that $f^-1(10)$ belongs to.
\[ f(x) = e^{x-3}-5 \]The solution is \( f^{-1}(10) = 5.708 \), which is option A.\begin{enumerate}[label=\Alph*.]
\item \( f^{-1}(10) \in [5.61, 5.73] \)

 This is the solution.
\item \( f^{-1}(10) \in [-3.24, -2.83] \)

 This solution corresponds to distractor 4.
\item \( f^{-1}(10) \in [-0.54, -0.06] \)

 This solution corresponds to distractor 1.
\item \( f^{-1}(10) \in [-3.66, -3.28] \)

 This solution corresponds to distractor 2.
\item \( f^{-1}(10) \in [-2.64, -2.19] \)

 This solution corresponds to distractor 3.
\end{enumerate}

\textbf{General Comment:} Natural log and exponential functions always have an inverse. Once you switch the $x$ and $y$, use the conversion $ e^y = x \leftrightarrow y=\ln(x)$.
}
\litem{
Find the inverse of the function below (if it exists). Then, evaluate the inverse at $x = -14$ and choose the interval that $f^-1(-14)$ belongs to.
\[ f(x) = \sqrt[3]{2 x - 5} \]The solution is \( -1369.5 \), which is option A.\begin{enumerate}[label=\Alph*.]
\item \( f^{-1}(-14) \in [-1373.5, -1362.5] \)

* This is the correct solution.
\item \( f^{-1}(-14) \in [1371.5, 1375.5] \)

 This solution corresponds to distractor 3.
\item \( f^{-1}(-14) \in [1369.5, 1370.5] \)

 This solution corresponds to distractor 2.
\item \( f^{-1}(-14) \in [-1374.5, -1371.5] \)

 Distractor 1: This corresponds to 
\item \( \text{ The function is not invertible for all Real numbers. } \)

 This solution corresponds to distractor 4.
\end{enumerate}

\textbf{General Comment:} Be sure you check that the function is 1-1 before trying to find the inverse!
}
\litem{
Find the inverse of the function below. Then, evaluate the inverse at $x = 5$ and choose the interval that $f^-1(5)$ belongs to.
\[ f(x) = e^{x-2}-2 \]The solution is \( f^{-1}(5) = 3.946 \), which is option A.\begin{enumerate}[label=\Alph*.]
\item \( f^{-1}(5) \in [3.7, 4.98] \)

 This is the solution.
\item \( f^{-1}(5) \in [-0.07, 1.7] \)

 This solution corresponds to distractor 1.
\item \( f^{-1}(5) \in [-0.93, -0.88] \)

 This solution corresponds to distractor 2.
\item \( f^{-1}(5) \in [-0.93, -0.88] \)

 This solution corresponds to distractor 4.
\item \( f^{-1}(5) \in [-0.07, 1.7] \)

 This solution corresponds to distractor 3.
\end{enumerate}

\textbf{General Comment:} Natural log and exponential functions always have an inverse. Once you switch the $x$ and $y$, use the conversion $ e^y = x \leftrightarrow y=\ln(x)$.
}
\litem{
Find the inverse of the function below (if it exists). Then, evaluate the inverse at $x = 10$ and choose the interval that $f^-1(10)$ belongs to.
\[ f(x) = 4 x^2 - 5 \]The solution is \( \text{ The function is not invertible for all Real numbers. } \), which is option E.\begin{enumerate}[label=\Alph*.]
\item \( f^{-1}(10) \in [0.71, 1.71] \)

 Distractor 2: This corresponds to finding the (nonexistent) inverse and not subtracting by the vertical shift.
\item \( f^{-1}(10) \in [3.05, 4.01] \)

 Distractor 3: This corresponds to finding the (nonexistent) inverse and dividing by a negative.
\item \( f^{-1}(10) \in [4.69, 5.8] \)

 Distractor 4: This corresponds to both distractors 2 and 3.
\item \( f^{-1}(10) \in [1.81, 2.61] \)

 Distractor 1: This corresponds to trying to find the inverse even though the function is not 1-1. 
\item \( \text{ The function is not invertible for all Real numbers. } \)

* This is the correct option.
\end{enumerate}

\textbf{General Comment:} Be sure you check that the function is 1-1 before trying to find the inverse!
}
\litem{
Multiply the following functions, then choose the domain of the resulting function from the list below.
\[ f(x) = 9x + 7 \text{ and } g(x) = \sqrt{-3x-9}  \]The solution is \( \text{ The domain is all Real numbers less than or equal to} x = -3.0. \), which is option B.\begin{enumerate}[label=\Alph*.]
\item \( \text{ The domain is all Real numbers greater than or equal to } x = a, \text{ where } a \in [-6.33, -0.33] \)


\item \( \text{ The domain is all Real numbers less than or equal to } x = a, \text{ where } a \in [-6, 0] \)


\item \( \text{ The domain is all Real numbers except } x = a, \text{ where } a \in [2.33, 8.33] \)


\item \( \text{ The domain is all Real numbers except } x = a \text{ and } x = b, \text{ where } a \in [-7.83, -1.83] \text{ and } b \in [1.2, 7.2] \)


\item \( \text{ The domain is all Real numbers. } \)


\end{enumerate}

\textbf{General Comment:} The new domain is the intersection of the previous domains.
}
\litem{
Determine whether the function below is 1-1.
\[ f(x) = -12 x^2 - 167 x - 575 \]The solution is \( \text{no} \), which is option D.\begin{enumerate}[label=\Alph*.]
\item \( \text{No, because the domain of the function is not $(-\infty, \infty)$.} \)

Corresponds to believing 1-1 means the domain is all Real numbers.
\item \( \text{No, because the range of the function is not $(-\infty, \infty)$.} \)

Corresponds to believing 1-1 means the range is all Real numbers.
\item \( \text{Yes, the function is 1-1.} \)

Corresponds to believing the function passes the Horizontal Line test.
\item \( \text{No, because there is a $y$-value that goes to 2 different $x$-values.} \)

* This is the solution.
\item \( \text{No, because there is an $x$-value that goes to 2 different $y$-values.} \)

Corresponds to the Vertical Line test, which checks if an expression is a function.
\end{enumerate}

\textbf{General Comment:} There are only two valid options: The function is 1-1 OR No because there is a $y$-value that goes to 2 different $x$-values.
}
\litem{
Add the following functions, then choose the domain of the resulting function from the list below.
\[ f(x) = 3x^{4} +6 x^{3} +4 x^{2} +5 x \text{ and } g(x) = \sqrt{-6x-18}  \]The solution is \( \text{ The domain is all Real numbers less than or equal to} x = -3.0. \), which is option C.\begin{enumerate}[label=\Alph*.]
\item \( \text{ The domain is all Real numbers greater than or equal to } x = a, \text{ where } a \in [-8.83, -0.83] \)


\item \( \text{ The domain is all Real numbers except } x = a, \text{ where } a \in [5.33, 6.33] \)


\item \( \text{ The domain is all Real numbers less than or equal to } x = a, \text{ where } a \in [-7, 1] \)


\item \( \text{ The domain is all Real numbers except } x = a \text{ and } x = b, \text{ where } a \in [-1.17, 4.83] \text{ and } b \in [5.25, 11.25] \)


\item \( \text{ The domain is all Real numbers. } \)


\end{enumerate}

\textbf{General Comment:} The new domain is the intersection of the previous domains.
}
\end{enumerate}

\end{document}