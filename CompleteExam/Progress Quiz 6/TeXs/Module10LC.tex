\documentclass[14pt]{extbook}
\usepackage{multicol, enumerate, enumitem, hyperref, color, soul, setspace, parskip, fancyhdr} %General Packages
\usepackage{amssymb, amsthm, amsmath, latexsym, units, mathtools} %Math Packages
\everymath{\displaystyle} %All math in Display Style
% Packages with additional options
\usepackage[headsep=0.5cm,headheight=12pt, left=1 in,right= 1 in,top= 1 in,bottom= 1 in]{geometry}
\usepackage[usenames,dvipsnames]{xcolor}
\usepackage{dashrule}  % Package to use the command below to create lines between items
\newcommand{\litem}[1]{\item#1\hspace*{-1cm}\rule{\textwidth}{0.4pt}}
\pagestyle{fancy}
\lhead{Progress Quiz 6}
\chead{}
\rhead{Version C}
\lfoot{4563-7456}
\cfoot{}
\rfoot{Summer C 2021}
\begin{document}

\begin{enumerate}
\litem{
Factor the polynomial below completely. Then, choose the intervals the zeros of the polynomial belong to, where $z_1 \leq z_2 \leq z_3$. \textit{To make the problem easier, all zeros are between -5 and 5.}\[ f(x) = 8x^{3} -6 x^{2} -45 x -27 \]\begin{enumerate}[label=\Alph*.]
\item \( z_1 \in [-3.3, -1.7], \text{   }  z_2 \in [0.68, 0.83], \text{   and   } z_3 \in [1.44, 1.51] \)
\item \( z_1 \in [-3.3, -1.7], \text{   }  z_2 \in [0.64, 0.69], \text{   and   } z_3 \in [1.17, 1.48] \)
\item \( z_1 \in [-2.4, -1.4], \text{   }  z_2 \in [-0.77, -0.75], \text{   and   } z_3 \in [2.78, 3.13] \)
\item \( z_1 \in [-3.3, -1.7], \text{   }  z_2 \in [0.3, 0.41], \text{   and   } z_3 \in [2.78, 3.13] \)
\item \( z_1 \in [-1.4, -1.1], \text{   }  z_2 \in [-0.7, -0.63], \text{   and   } z_3 \in [2.78, 3.13] \)

\end{enumerate} }
\litem{
Perform the division below. Then, find the intervals that correspond to the quotient in the form $ax^2+bx+c$ and remainder $r$.\[ \frac{8x^{3} -8 x^{2} -40 x -29}{x -3} \]\begin{enumerate}[label=\Alph*.]
\item \( a \in [6, 12], \text{   } b \in [14, 19], \text{   } c \in [6, 9], \text{   and   } r \in [-5, 2]. \)
\item \( a \in [6, 12], \text{   } b \in [3, 10], \text{   } c \in [-27, -22], \text{   and   } r \in [-80, -72]. \)
\item \( a \in [6, 12], \text{   } b \in [-32, -31], \text{   } c \in [54, 57], \text{   and   } r \in [-197, -193]. \)
\item \( a \in [24, 32], \text{   } b \in [63, 69], \text{   } c \in [152, 155], \text{   and   } r \in [427, 428]. \)
\item \( a \in [24, 32], \text{   } b \in [-83, -75], \text{   } c \in [199, 207], \text{   and   } r \in [-634, -628]. \)

\end{enumerate} }
\litem{
Perform the division below. Then, find the intervals that correspond to the quotient in the form $ax^2+bx+c$ and remainder $r$.\[ \frac{6x^{3} +26 x^{2} -28}{x + 4} \]\begin{enumerate}[label=\Alph*.]
\item \( a \in [1, 9], b \in [48, 55], c \in [200, 202], \text{ and } r \in [771, 774]. \)
\item \( a \in [1, 9], b \in [-4, 0], c \in [19, 28], \text{ and } r \in [-130, -124]. \)
\item \( a \in [1, 9], b \in [1, 5], c \in [-12, -4], \text{ and } r \in [-1, 10]. \)
\item \( a \in [-24, -22], b \in [-73, -66], c \in [-283, -279], \text{ and } r \in [-1154, -1140]. \)
\item \( a \in [-24, -22], b \in [121, 123], c \in [-491, -483], \text{ and } r \in [1921, 1929]. \)

\end{enumerate} }
\litem{
Factor the polynomial below completely. Then, choose the intervals the zeros of the polynomial belong to, where $z_1 \leq z_2 \leq z_3$. \textit{To make the problem easier, all zeros are between -5 and 5.}\[ f(x) = 16x^{3} -40 x^{2} +x + 30 \]\begin{enumerate}[label=\Alph*.]
\item \( z_1 \in [-2.25, -1.94], \text{   }  z_2 \in [-1.04, -0.34], \text{   and   } z_3 \in [0.79, 1.94] \)
\item \( z_1 \in [-1.34, -0.77], \text{   }  z_2 \in [0.55, 0.88], \text{   and   } z_3 \in [1.9, 2.19] \)
\item \( z_1 \in [-2.25, -1.94], \text{   }  z_2 \in [-1.46, -1.14], \text{   and   } z_3 \in [0.51, 0.89] \)
\item \( z_1 \in [-1.28, -0.5], \text{   }  z_2 \in [1.07, 1.34], \text{   and   } z_3 \in [1.9, 2.19] \)
\item \( z_1 \in [-5.28, -4.63], \text{   }  z_2 \in [-2.05, -1.9], \text{   and   } z_3 \in [-0.1, 0.58] \)

\end{enumerate} }
\litem{
Factor the polynomial below completely, knowing that $x -4$ is a factor. Then, choose the intervals the zeros of the polynomial belong to, where $z_1 \leq z_2 \leq z_3 \leq z_4$. \textit{To make the problem easier, all zeros are between -5 and 5.}\[ f(x) = 15x^{4} -59 x^{3} -50 x^{2} +208 x -96 \]\begin{enumerate}[label=\Alph*.]
\item \( z_1 \in [-4, -3], \text{   }  z_2 \in [-1.68, -1.66], z_3 \in [-0.81, -0.63], \text{   and   } z_4 \in [1.7, 2.7] \)
\item \( z_1 \in [-2, 1], \text{   }  z_2 \in [0.62, 0.9], z_3 \in [1.57, 1.69], \text{   and   } z_4 \in [3.9, 5.2] \)
\item \( z_1 \in [-4, -3], \text{   }  z_2 \in [-1.44, -1.27], z_3 \in [-0.67, -0.6], \text{   and   } z_4 \in [1.7, 2.7] \)
\item \( z_1 \in [-4, -3], \text{   }  z_2 \in [-3.02, -2.95], z_3 \in [-0.44, -0.14], \text{   and   } z_4 \in [1.7, 2.7] \)
\item \( z_1 \in [-2, 1], \text{   }  z_2 \in [0.55, 0.65], z_3 \in [1.28, 1.46], \text{   and   } z_4 \in [3.9, 5.2] \)

\end{enumerate} }
\litem{
Perform the division below. Then, find the intervals that correspond to the quotient in the form $ax^2+bx+c$ and remainder $r$.\[ \frac{12x^{3} -4 x^{2} -40 x + 37}{x + 2} \]\begin{enumerate}[label=\Alph*.]
\item \( a \in [-32, -22], \text{   } b \in [-55, -47], \text{   } c \in [-151, -142], \text{   and   } r \in [-251, -245]. \)
\item \( a \in [-32, -22], \text{   } b \in [36, 49], \text{   } c \in [-129, -124], \text{   and   } r \in [292, 296]. \)
\item \( a \in [12, 13], \text{   } b \in [-32, -24], \text{   } c \in [11, 19], \text{   and   } r \in [4, 9]. \)
\item \( a \in [12, 13], \text{   } b \in [-43, -38], \text{   } c \in [79, 82], \text{   and   } r \in [-205, -196]. \)
\item \( a \in [12, 13], \text{   } b \in [18, 26], \text{   } c \in [0, 1], \text{   and   } r \in [29, 47]. \)

\end{enumerate} }
\litem{
Perform the division below. Then, find the intervals that correspond to the quotient in the form $ax^2+bx+c$ and remainder $r$.\[ \frac{16x^{3} -48 x -28}{x -2} \]\begin{enumerate}[label=\Alph*.]
\item \( a \in [32, 33], b \in [-64, -59], c \in [80, 83], \text{ and } r \in [-195, -187]. \)
\item \( a \in [12, 23], b \in [-35, -27], c \in [9, 21], \text{ and } r \in [-60, -53]. \)
\item \( a \in [12, 23], b \in [26, 33], c \in [9, 21], \text{ and } r \in [3, 5]. \)
\item \( a \in [32, 33], b \in [61, 67], c \in [80, 83], \text{ and } r \in [130, 141]. \)
\item \( a \in [12, 23], b \in [15, 17], c \in [-39, -28], \text{ and } r \in [-60, -53]. \)

\end{enumerate} }
\litem{
What are the \textit{possible Integer} roots of the polynomial below?\[ f(x) = 2x^{2} +3 x + 7 \]\begin{enumerate}[label=\Alph*.]
\item \( \text{ All combinations of: }\frac{\pm 1,\pm 2}{\pm 1,\pm 7} \)
\item \( \text{ All combinations of: }\frac{\pm 1,\pm 7}{\pm 1,\pm 2} \)
\item \( \pm 1,\pm 7 \)
\item \( \pm 1,\pm 2 \)
\item \( \text{There is no formula or theorem that tells us all possible Integer roots.} \)

\end{enumerate} }
\litem{
Factor the polynomial below completely, knowing that $x -4$ is a factor. Then, choose the intervals the zeros of the polynomial belong to, where $z_1 \leq z_2 \leq z_3 \leq z_4$. \textit{To make the problem easier, all zeros are between -5 and 5.}\[ f(x) = 25x^{4} -80 x^{3} -132 x^{2} +224 x -64 \]\begin{enumerate}[label=\Alph*.]
\item \( z_1 \in [-2, 2], \text{   }  z_2 \in [-0.17, 0.41], z_3 \in [0.76, 0.9], \text{   and   } z_4 \in [3, 6] \)
\item \( z_1 \in [-5, -3], \text{   }  z_2 \in [-1.71, -0.25], z_3 \in [-0.63, -0.31], \text{   and   } z_4 \in [1, 3] \)
\item \( z_1 \in [-5, -3], \text{   }  z_2 \in [-2.13, -1.06], z_3 \in [-0.19, 0.06], \text{   and   } z_4 \in [1, 3] \)
\item \( z_1 \in [-2, 2], \text{   }  z_2 \in [1.21, 2.56], z_3 \in [2.4, 2.52], \text{   and   } z_4 \in [3, 6] \)
\item \( z_1 \in [-5, -3], \text{   }  z_2 \in [-2.8, -2.06], z_3 \in [-1.34, -1.19], \text{   and   } z_4 \in [1, 3] \)

\end{enumerate} }
\litem{
What are the \textit{possible Integer} roots of the polynomial below?\[ f(x) = 6x^{3} +4 x^{2} +4 x + 7 \]\begin{enumerate}[label=\Alph*.]
\item \( \pm 1,\pm 2,\pm 3,\pm 6 \)
\item \( \text{ All combinations of: }\frac{\pm 1,\pm 2,\pm 3,\pm 6}{\pm 1,\pm 7} \)
\item \( \pm 1,\pm 7 \)
\item \( \text{ All combinations of: }\frac{\pm 1,\pm 7}{\pm 1,\pm 2,\pm 3,\pm 6} \)
\item \( \text{There is no formula or theorem that tells us all possible Integer roots.} \)

\end{enumerate} }
\end{enumerate}

\end{document}