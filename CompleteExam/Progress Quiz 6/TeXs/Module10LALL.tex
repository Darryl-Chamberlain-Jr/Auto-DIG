\documentclass[14pt]{extbook}
\usepackage{multicol, enumerate, enumitem, hyperref, color, soul, setspace, parskip, fancyhdr} %General Packages
\usepackage{amssymb, amsthm, amsmath, latexsym, units, mathtools} %Math Packages
\everymath{\displaystyle} %All math in Display Style
% Packages with additional options
\usepackage[headsep=0.5cm,headheight=12pt, left=1 in,right= 1 in,top= 1 in,bottom= 1 in]{geometry}
\usepackage[usenames,dvipsnames]{xcolor}
\usepackage{dashrule}  % Package to use the command below to create lines between items
\newcommand{\litem}[1]{\item#1\hspace*{-1cm}\rule{\textwidth}{0.4pt}}
\pagestyle{fancy}
\lhead{Progress Quiz 6}
\chead{}
\rhead{Version ALL}
\lfoot{4563-7456}
\cfoot{}
\rfoot{Summer C 2021}
\begin{document}

\begin{enumerate}
\litem{
Factor the polynomial below completely. Then, choose the intervals the zeros of the polynomial belong to, where $z_1 \leq z_2 \leq z_3$. \textit{To make the problem easier, all zeros are between -5 and 5.}\[ f(x) = 25x^{3} +75 x^{2} -16 x -48 \]\begin{enumerate}[label=\Alph*.]
\item \( z_1 \in [-3.16, -2.71], \text{   }  z_2 \in [-1.32, -1.21], \text{   and   } z_3 \in [1.09, 1.63] \)
\item \( z_1 \in [-1.28, -1.18], \text{   }  z_2 \in [1.09, 1.35], \text{   and   } z_3 \in [2.81, 3.32] \)
\item \( z_1 \in [-3.16, -2.71], \text{   }  z_2 \in [-0.86, -0.49], \text{   and   } z_3 \in [0.52, 0.91] \)
\item \( z_1 \in [-4.08, -3.92], \text{   }  z_2 \in [0.08, 0.21], \text{   and   } z_3 \in [2.81, 3.32] \)
\item \( z_1 \in [-0.9, -0.7], \text{   }  z_2 \in [0.52, 0.96], \text{   and   } z_3 \in [2.81, 3.32] \)

\end{enumerate} }
\litem{
Perform the division below. Then, find the intervals that correspond to the quotient in the form $ax^2+bx+c$ and remainder $r$.\[ \frac{20x^{3} -106 x^{2} +112 x -30}{x -4} \]\begin{enumerate}[label=\Alph*.]
\item \( a \in [79, 82], \text{   } b \in [-426, -424], \text{   } c \in [1811, 1818], \text{   and   } r \in [-7295, -7290]. \)
\item \( a \in [79, 82], \text{   } b \in [212, 216], \text{   } c \in [965, 973], \text{   and   } r \in [3836, 3844]. \)
\item \( a \in [17, 26], \text{   } b \in [-47, -44], \text{   } c \in [-27, -22], \text{   and   } r \in [-109, -104]. \)
\item \( a \in [17, 26], \text{   } b \in [-28, -23], \text{   } c \in [4, 11], \text{   and   } r \in [-1, 5]. \)
\item \( a \in [17, 26], \text{   } b \in [-192, -184], \text{   } c \in [855, 861], \text{   and   } r \in [-3457, -3450]. \)

\end{enumerate} }
\litem{
Perform the division below. Then, find the intervals that correspond to the quotient in the form $ax^2+bx+c$ and remainder $r$.\[ \frac{6x^{3} +28 x^{2} -68}{x + 4} \]\begin{enumerate}[label=\Alph*.]
\item \( a \in [-27, -23], b \in [123, 125], c \in [-498, -495], \text{ and } r \in [1913, 1919]. \)
\item \( a \in [3, 9], b \in [4, 9], c \in [-19, -11], \text{ and } r \in [-5, -3]. \)
\item \( a \in [-27, -23], b \in [-68, -63], c \in [-277, -267], \text{ and } r \in [-1157, -1153]. \)
\item \( a \in [3, 9], b \in [51, 53], c \in [208, 211], \text{ and } r \in [762, 767]. \)
\item \( a \in [3, 9], b \in [-6, 1], c \in [4, 15], \text{ and } r \in [-125, -117]. \)

\end{enumerate} }
\litem{
Factor the polynomial below completely. Then, choose the intervals the zeros of the polynomial belong to, where $z_1 \leq z_2 \leq z_3$. \textit{To make the problem easier, all zeros are between -5 and 5.}\[ f(x) = 10x^{3} -21 x^{2} -135 x -50 \]\begin{enumerate}[label=\Alph*.]
\item \( z_1 \in [-4.5, -1.5], \text{   }  z_2 \in [-0.52, -0.38], \text{   and   } z_3 \in [5, 7] \)
\item \( z_1 \in [-4.5, -1.5], \text{   }  z_2 \in [-0.52, -0.38], \text{   and   } z_3 \in [5, 7] \)
\item \( z_1 \in [-6, -4], \text{   }  z_2 \in [0.36, 0.46], \text{   and   } z_3 \in [1.5, 4.5] \)
\item \( z_1 \in [-6, -4], \text{   }  z_2 \in [0.01, 0.37], \text{   and   } z_3 \in [5, 7] \)
\item \( z_1 \in [-6, -4], \text{   }  z_2 \in [0.36, 0.46], \text{   and   } z_3 \in [1.5, 4.5] \)

\end{enumerate} }
\litem{
Factor the polynomial below completely, knowing that $x -3$ is a factor. Then, choose the intervals the zeros of the polynomial belong to, where $z_1 \leq z_2 \leq z_3 \leq z_4$. \textit{To make the problem easier, all zeros are between -5 and 5.}\[ f(x) = 9x^{4} +9 x^{3} -163 x^{2} +115 x + 150 \]\begin{enumerate}[label=\Alph*.]
\item \( z_1 \in [-5.2, -4.7], \text{   }  z_2 \in [-1.62, -1.48], z_3 \in [0.52, 0.63], \text{   and   } z_4 \in [2.4, 3.2] \)
\item \( z_1 \in [-5.2, -4.7], \text{   }  z_2 \in [-3.05, -2.99], z_3 \in [0.12, 0.28], \text{   and   } z_4 \in [4, 5.3] \)
\item \( z_1 \in [-3.7, -2], \text{   }  z_2 \in [-0.65, -0.6], z_3 \in [1.47, 1.5], \text{   and   } z_4 \in [4, 5.3] \)
\item \( z_1 \in [-5.2, -4.7], \text{   }  z_2 \in [-0.74, -0.64], z_3 \in [1.64, 1.74], \text{   and   } z_4 \in [2.4, 3.2] \)
\item \( z_1 \in [-3.7, -2], \text{   }  z_2 \in [-1.71, -1.63], z_3 \in [0.66, 0.71], \text{   and   } z_4 \in [4, 5.3] \)

\end{enumerate} }
\litem{
Perform the division below. Then, find the intervals that correspond to the quotient in the form $ax^2+bx+c$ and remainder $r$.\[ \frac{10x^{3} +26 x^{2} -68 x -53}{x + 4} \]\begin{enumerate}[label=\Alph*.]
\item \( a \in [-42, -33], \text{   } b \in [-134, -130], \text{   } c \in [-607, -603], \text{   and   } r \in [-2473, -2463]. \)
\item \( a \in [-42, -33], \text{   } b \in [185, 188], \text{   } c \in [-818, -809], \text{   and   } r \in [3195, 3197]. \)
\item \( a \in [9, 11], \text{   } b \in [-30, -21], \text{   } c \in [52, 57], \text{   and   } r \in [-321, -310]. \)
\item \( a \in [9, 11], \text{   } b \in [-16, -13], \text{   } c \in [-14, -10], \text{   and   } r \in [-8, -1]. \)
\item \( a \in [9, 11], \text{   } b \in [66, 70], \text{   } c \in [196, 202], \text{   and   } r \in [722, 739]. \)

\end{enumerate} }
\litem{
Perform the division below. Then, find the intervals that correspond to the quotient in the form $ax^2+bx+c$ and remainder $r$.\[ \frac{6x^{3} +28 x^{2} -62}{x + 4} \]\begin{enumerate}[label=\Alph*.]
\item \( a \in [4, 10], b \in [51, 53], c \in [207, 212], \text{ and } r \in [762, 773]. \)
\item \( a \in [4, 10], b \in [-6, 1], c \in [8, 13], \text{ and } r \in [-115, -105]. \)
\item \( a \in [-25, -21], b \in [120, 125], c \in [-503, -493], \text{ and } r \in [1917, 1926]. \)
\item \( a \in [4, 10], b \in [3, 8], c \in [-17, -14], \text{ and } r \in [2, 3]. \)
\item \( a \in [-25, -21], b \in [-72, -65], c \in [-277, -268], \text{ and } r \in [-1150, -1148]. \)

\end{enumerate} }
\litem{
What are the \textit{possible Integer} roots of the polynomial below?\[ f(x) = 6x^{2} +2 x + 3 \]\begin{enumerate}[label=\Alph*.]
\item \( \pm 1,\pm 3 \)
\item \( \pm 1,\pm 2,\pm 3,\pm 6 \)
\item \( \text{ All combinations of: }\frac{\pm 1,\pm 3}{\pm 1,\pm 2,\pm 3,\pm 6} \)
\item \( \text{ All combinations of: }\frac{\pm 1,\pm 2,\pm 3,\pm 6}{\pm 1,\pm 3} \)
\item \( \text{There is no formula or theorem that tells us all possible Integer roots.} \)

\end{enumerate} }
\litem{
Factor the polynomial below completely, knowing that $x -3$ is a factor. Then, choose the intervals the zeros of the polynomial belong to, where $z_1 \leq z_2 \leq z_3 \leq z_4$. \textit{To make the problem easier, all zeros are between -5 and 5.}\[ f(x) = 8x^{4} -90 x^{3} +343 x^{2} -510 x + 225 \]\begin{enumerate}[label=\Alph*.]
\item \( z_1 \in [-5.86, -4.88], \text{   }  z_2 \in [-3.65, -2.93], z_3 \in [-3.38, -2.77], \text{   and   } z_4 \in [-0.71, -0.43] \)
\item \( z_1 \in [-5.86, -4.88], \text{   }  z_2 \in [-3.65, -2.93], z_3 \in [-2.14, -0.63], \text{   and   } z_4 \in [-0.49, -0.24] \)
\item \( z_1 \in [-5.86, -4.88], \text{   }  z_2 \in [-3.65, -2.93], z_3 \in [-2.84, -2.19], \text{   and   } z_4 \in [-0.83, -0.74] \)
\item \( z_1 \in [0.6, 0.85], \text{   }  z_2 \in [2.02, 3.75], z_3 \in [2.3, 3.15], \text{   and   } z_4 \in [4.99, 5.07] \)
\item \( z_1 \in [0.14, 0.74], \text{   }  z_2 \in [1.06, 1.46], z_3 \in [2.3, 3.15], \text{   and   } z_4 \in [4.99, 5.07] \)

\end{enumerate} }
\litem{
What are the \textit{possible Rational} roots of the polynomial below?\[ f(x) = 7x^{3} +5 x^{2} +2 x + 5 \]\begin{enumerate}[label=\Alph*.]
\item \( \text{ All combinations of: }\frac{\pm 1,\pm 7}{\pm 1,\pm 5} \)
\item \( \pm 1,\pm 7 \)
\item \( \pm 1,\pm 5 \)
\item \( \text{ All combinations of: }\frac{\pm 1,\pm 5}{\pm 1,\pm 7} \)
\item \( \text{ There is no formula or theorem that tells us all possible Rational roots.} \)

\end{enumerate} }
\litem{
Factor the polynomial below completely. Then, choose the intervals the zeros of the polynomial belong to, where $z_1 \leq z_2 \leq z_3$. \textit{To make the problem easier, all zeros are between -5 and 5.}\[ f(x) = 10x^{3} -41 x^{2} +27 x + 18 \]\begin{enumerate}[label=\Alph*.]
\item \( z_1 \in [-3.1, -2.9], \text{   }  z_2 \in [-1.7, -0.7], \text{   and   } z_3 \in [0.23, 0.73] \)
\item \( z_1 \in [-2.9, -1.5], \text{   }  z_2 \in [0.2, 0.8], \text{   and   } z_3 \in [2.77, 3.08] \)
\item \( z_1 \in [-3.1, -2.9], \text{   }  z_2 \in [-1.1, 0], \text{   and   } z_3 \in [2.35, 2.69] \)
\item \( z_1 \in [-3.1, -2.9], \text{   }  z_2 \in [-3.2, -2.7], \text{   and   } z_3 \in [0, 0.29] \)
\item \( z_1 \in [-1.9, 0], \text{   }  z_2 \in [0.8, 2], \text{   and   } z_3 \in [2.77, 3.08] \)

\end{enumerate} }
\litem{
Perform the division below. Then, find the intervals that correspond to the quotient in the form $ax^2+bx+c$ and remainder $r$.\[ \frac{10x^{3} +11 x^{2} -106 x + 44}{x + 4} \]\begin{enumerate}[label=\Alph*.]
\item \( a \in [9, 11], \text{   } b \in [-46, -31], \text{   } c \in [84, 95], \text{   and   } r \in [-401, -396]. \)
\item \( a \in [-44, -39], \text{   } b \in [-152, -148], \text{   } c \in [-703, -700], \text{   and   } r \in [-2765, -2760]. \)
\item \( a \in [9, 11], \text{   } b \in [46, 56], \text{   } c \in [98, 100], \text{   and   } r \in [425, 441]. \)
\item \( a \in [-44, -39], \text{   } b \in [168, 175], \text{   } c \in [-796, -787], \text{   and   } r \in [3196, 3209]. \)
\item \( a \in [9, 11], \text{   } b \in [-29, -25], \text{   } c \in [8, 15], \text{   and   } r \in [3, 7]. \)

\end{enumerate} }
\litem{
Perform the division below. Then, find the intervals that correspond to the quotient in the form $ax^2+bx+c$ and remainder $r$.\[ \frac{9x^{3} -28 x -19}{x -2} \]\begin{enumerate}[label=\Alph*.]
\item \( a \in [14, 26], b \in [-36, -34], c \in [40, 45], \text{ and } r \in [-108, -105]. \)
\item \( a \in [6, 12], b \in [-22, -17], c \in [0, 12], \text{ and } r \in [-35, -34]. \)
\item \( a \in [6, 12], b \in [6, 15], c \in [-23, -18], \text{ and } r \in [-42, -37]. \)
\item \( a \in [14, 26], b \in [36, 38], c \in [40, 45], \text{ and } r \in [65, 74]. \)
\item \( a \in [6, 12], b \in [16, 20], c \in [0, 12], \text{ and } r \in [-8, 1]. \)

\end{enumerate} }
\litem{
Factor the polynomial below completely. Then, choose the intervals the zeros of the polynomial belong to, where $z_1 \leq z_2 \leq z_3$. \textit{To make the problem easier, all zeros are between -5 and 5.}\[ f(x) = 12x^{3} +35 x^{2} -9 x -18 \]\begin{enumerate}[label=\Alph*.]
\item \( z_1 \in [-1.5, -1.26], \text{   }  z_2 \in [0.92, 1.85], \text{   and   } z_3 \in [2.5, 3.1] \)
\item \( z_1 \in [-0.8, -0.64], \text{   }  z_2 \in [0.48, 0.77], \text{   and   } z_3 \in [2.5, 3.1] \)
\item \( z_1 \in [-3.17, -2.34], \text{   }  z_2 \in [-1.51, -1.46], \text{   and   } z_3 \in [1.1, 2.4] \)
\item \( z_1 \in [-3.17, -2.34], \text{   }  z_2 \in [-0.74, -0.58], \text{   and   } z_3 \in [0, 1.1] \)
\item \( z_1 \in [-0.34, -0.12], \text{   }  z_2 \in [1.58, 2.09], \text{   and   } z_3 \in [2.5, 3.1] \)

\end{enumerate} }
\litem{
Factor the polynomial below completely, knowing that $x -4$ is a factor. Then, choose the intervals the zeros of the polynomial belong to, where $z_1 \leq z_2 \leq z_3 \leq z_4$. \textit{To make the problem easier, all zeros are between -5 and 5.}\[ f(x) = 15x^{4} -14 x^{3} -248 x^{2} +224 x + 128 \]\begin{enumerate}[label=\Alph*.]
\item \( z_1 \in [-5, -2], \text{   }  z_2 \in [-4.5, -3.89], z_3 \in [-0.08, 0.26], \text{   and   } z_4 \in [2, 8] \)
\item \( z_1 \in [-5, -2], \text{   }  z_2 \in [-3.04, -2.45], z_3 \in [0.73, 0.85], \text{   and   } z_4 \in [2, 8] \)
\item \( z_1 \in [-5, -2], \text{   }  z_2 \in [-0.52, -0.21], z_3 \in [1.24, 1.46], \text{   and   } z_4 \in [2, 8] \)
\item \( z_1 \in [-5, -2], \text{   }  z_2 \in [-1.72, -1.22], z_3 \in [0.18, 0.71], \text{   and   } z_4 \in [2, 8] \)
\item \( z_1 \in [-5, -2], \text{   }  z_2 \in [-0.94, -0.69], z_3 \in [2.44, 2.56], \text{   and   } z_4 \in [2, 8] \)

\end{enumerate} }
\litem{
Perform the division below. Then, find the intervals that correspond to the quotient in the form $ax^2+bx+c$ and remainder $r$.\[ \frac{9x^{3} +27 x^{2} -25 x -77}{x + 3} \]\begin{enumerate}[label=\Alph*.]
\item \( a \in [9, 10], \text{   } b \in [-2, 2], \text{   } c \in [-27, -24], \text{   and   } r \in [-9, 0]. \)
\item \( a \in [-32, -24], \text{   } b \in [107, 111], \text{   } c \in [-350, -348], \text{   and   } r \in [968, 976]. \)
\item \( a \in [9, 10], \text{   } b \in [53, 60], \text{   } c \in [133, 143], \text{   and   } r \in [334, 340]. \)
\item \( a \in [9, 10], \text{   } b \in [-10, -7], \text{   } c \in [11, 16], \text{   and   } r \in [-121, -118]. \)
\item \( a \in [-32, -24], \text{   } b \in [-55, -51], \text{   } c \in [-188, -186], \text{   and   } r \in [-642, -633]. \)

\end{enumerate} }
\litem{
Perform the division below. Then, find the intervals that correspond to the quotient in the form $ax^2+bx+c$ and remainder $r$.\[ \frac{10x^{3} +30 x^{2} -44}{x + 2} \]\begin{enumerate}[label=\Alph*.]
\item \( a \in [-21, -16], b \in [69, 73], c \in [-140, -137], \text{ and } r \in [232, 243]. \)
\item \( a \in [-21, -16], b \in [-12, -7], c \in [-21, -15], \text{ and } r \in [-88, -81]. \)
\item \( a \in [6, 11], b \in [46, 51], c \in [97, 101], \text{ and } r \in [153, 159]. \)
\item \( a \in [6, 11], b \in [9, 17], c \in [-21, -15], \text{ and } r \in [-4, -3]. \)
\item \( a \in [6, 11], b \in [-5, 6], c \in [-2, 3], \text{ and } r \in [-50, -42]. \)

\end{enumerate} }
\litem{
What are the \textit{possible Rational} roots of the polynomial below?\[ f(x) = 3x^{3} +2 x^{2} +4 x + 7 \]\begin{enumerate}[label=\Alph*.]
\item \( \text{ All combinations of: }\frac{\pm 1,\pm 3}{\pm 1,\pm 7} \)
\item \( \pm 1,\pm 3 \)
\item \( \text{ All combinations of: }\frac{\pm 1,\pm 7}{\pm 1,\pm 3} \)
\item \( \pm 1,\pm 7 \)
\item \( \text{ There is no formula or theorem that tells us all possible Rational roots.} \)

\end{enumerate} }
\litem{
Factor the polynomial below completely, knowing that $x + 3$ is a factor. Then, choose the intervals the zeros of the polynomial belong to, where $z_1 \leq z_2 \leq z_3 \leq z_4$. \textit{To make the problem easier, all zeros are between -5 and 5.}\[ f(x) = 10x^{4} +51 x^{3} -28 x^{2} -333 x -180 \]\begin{enumerate}[label=\Alph*.]
\item \( z_1 \in [-0.74, -0.5], \text{   }  z_2 \in [2.94, 3.06], z_3 \in [1.8, 3.3], \text{   and   } z_4 \in [4, 5] \)
\item \( z_1 \in [-4.18, -3.74], \text{   }  z_2 \in [-3.13, -2.89], z_3 \in [-0.8, -0.4], \text{   and   } z_4 \in [2.5, 3.5] \)
\item \( z_1 \in [-0.47, -0.25], \text{   }  z_2 \in [1.43, 2.14], z_3 \in [1.8, 3.3], \text{   and   } z_4 \in [4, 5] \)
\item \( z_1 \in [-2.55, -2.47], \text{   }  z_2 \in [-0.59, 1.11], z_3 \in [1.8, 3.3], \text{   and   } z_4 \in [4, 5] \)
\item \( z_1 \in [-4.18, -3.74], \text{   }  z_2 \in [-3.13, -2.89], z_3 \in [-2.8, -1.2], \text{   and   } z_4 \in [0.4, 1.4] \)

\end{enumerate} }
\litem{
What are the \textit{possible Rational} roots of the polynomial below?\[ f(x) = 4x^{3} +4 x^{2} +6 x + 6 \]\begin{enumerate}[label=\Alph*.]
\item \( \pm 1,\pm 2,\pm 3,\pm 6 \)
\item \( \text{ All combinations of: }\frac{\pm 1,\pm 2,\pm 4}{\pm 1,\pm 2,\pm 3,\pm 6} \)
\item \( \pm 1,\pm 2,\pm 4 \)
\item \( \text{ All combinations of: }\frac{\pm 1,\pm 2,\pm 3,\pm 6}{\pm 1,\pm 2,\pm 4} \)
\item \( \text{ There is no formula or theorem that tells us all possible Rational roots.} \)

\end{enumerate} }
\litem{
Factor the polynomial below completely. Then, choose the intervals the zeros of the polynomial belong to, where $z_1 \leq z_2 \leq z_3$. \textit{To make the problem easier, all zeros are between -5 and 5.}\[ f(x) = 8x^{3} -6 x^{2} -45 x -27 \]\begin{enumerate}[label=\Alph*.]
\item \( z_1 \in [-3.3, -1.7], \text{   }  z_2 \in [0.68, 0.83], \text{   and   } z_3 \in [1.44, 1.51] \)
\item \( z_1 \in [-3.3, -1.7], \text{   }  z_2 \in [0.64, 0.69], \text{   and   } z_3 \in [1.17, 1.48] \)
\item \( z_1 \in [-2.4, -1.4], \text{   }  z_2 \in [-0.77, -0.75], \text{   and   } z_3 \in [2.78, 3.13] \)
\item \( z_1 \in [-3.3, -1.7], \text{   }  z_2 \in [0.3, 0.41], \text{   and   } z_3 \in [2.78, 3.13] \)
\item \( z_1 \in [-1.4, -1.1], \text{   }  z_2 \in [-0.7, -0.63], \text{   and   } z_3 \in [2.78, 3.13] \)

\end{enumerate} }
\litem{
Perform the division below. Then, find the intervals that correspond to the quotient in the form $ax^2+bx+c$ and remainder $r$.\[ \frac{8x^{3} -8 x^{2} -40 x -29}{x -3} \]\begin{enumerate}[label=\Alph*.]
\item \( a \in [6, 12], \text{   } b \in [14, 19], \text{   } c \in [6, 9], \text{   and   } r \in [-5, 2]. \)
\item \( a \in [6, 12], \text{   } b \in [3, 10], \text{   } c \in [-27, -22], \text{   and   } r \in [-80, -72]. \)
\item \( a \in [6, 12], \text{   } b \in [-32, -31], \text{   } c \in [54, 57], \text{   and   } r \in [-197, -193]. \)
\item \( a \in [24, 32], \text{   } b \in [63, 69], \text{   } c \in [152, 155], \text{   and   } r \in [427, 428]. \)
\item \( a \in [24, 32], \text{   } b \in [-83, -75], \text{   } c \in [199, 207], \text{   and   } r \in [-634, -628]. \)

\end{enumerate} }
\litem{
Perform the division below. Then, find the intervals that correspond to the quotient in the form $ax^2+bx+c$ and remainder $r$.\[ \frac{6x^{3} +26 x^{2} -28}{x + 4} \]\begin{enumerate}[label=\Alph*.]
\item \( a \in [1, 9], b \in [48, 55], c \in [200, 202], \text{ and } r \in [771, 774]. \)
\item \( a \in [1, 9], b \in [-4, 0], c \in [19, 28], \text{ and } r \in [-130, -124]. \)
\item \( a \in [1, 9], b \in [1, 5], c \in [-12, -4], \text{ and } r \in [-1, 10]. \)
\item \( a \in [-24, -22], b \in [-73, -66], c \in [-283, -279], \text{ and } r \in [-1154, -1140]. \)
\item \( a \in [-24, -22], b \in [121, 123], c \in [-491, -483], \text{ and } r \in [1921, 1929]. \)

\end{enumerate} }
\litem{
Factor the polynomial below completely. Then, choose the intervals the zeros of the polynomial belong to, where $z_1 \leq z_2 \leq z_3$. \textit{To make the problem easier, all zeros are between -5 and 5.}\[ f(x) = 16x^{3} -40 x^{2} +x + 30 \]\begin{enumerate}[label=\Alph*.]
\item \( z_1 \in [-2.25, -1.94], \text{   }  z_2 \in [-1.04, -0.34], \text{   and   } z_3 \in [0.79, 1.94] \)
\item \( z_1 \in [-1.34, -0.77], \text{   }  z_2 \in [0.55, 0.88], \text{   and   } z_3 \in [1.9, 2.19] \)
\item \( z_1 \in [-2.25, -1.94], \text{   }  z_2 \in [-1.46, -1.14], \text{   and   } z_3 \in [0.51, 0.89] \)
\item \( z_1 \in [-1.28, -0.5], \text{   }  z_2 \in [1.07, 1.34], \text{   and   } z_3 \in [1.9, 2.19] \)
\item \( z_1 \in [-5.28, -4.63], \text{   }  z_2 \in [-2.05, -1.9], \text{   and   } z_3 \in [-0.1, 0.58] \)

\end{enumerate} }
\litem{
Factor the polynomial below completely, knowing that $x -4$ is a factor. Then, choose the intervals the zeros of the polynomial belong to, where $z_1 \leq z_2 \leq z_3 \leq z_4$. \textit{To make the problem easier, all zeros are between -5 and 5.}\[ f(x) = 15x^{4} -59 x^{3} -50 x^{2} +208 x -96 \]\begin{enumerate}[label=\Alph*.]
\item \( z_1 \in [-4, -3], \text{   }  z_2 \in [-1.68, -1.66], z_3 \in [-0.81, -0.63], \text{   and   } z_4 \in [1.7, 2.7] \)
\item \( z_1 \in [-2, 1], \text{   }  z_2 \in [0.62, 0.9], z_3 \in [1.57, 1.69], \text{   and   } z_4 \in [3.9, 5.2] \)
\item \( z_1 \in [-4, -3], \text{   }  z_2 \in [-1.44, -1.27], z_3 \in [-0.67, -0.6], \text{   and   } z_4 \in [1.7, 2.7] \)
\item \( z_1 \in [-4, -3], \text{   }  z_2 \in [-3.02, -2.95], z_3 \in [-0.44, -0.14], \text{   and   } z_4 \in [1.7, 2.7] \)
\item \( z_1 \in [-2, 1], \text{   }  z_2 \in [0.55, 0.65], z_3 \in [1.28, 1.46], \text{   and   } z_4 \in [3.9, 5.2] \)

\end{enumerate} }
\litem{
Perform the division below. Then, find the intervals that correspond to the quotient in the form $ax^2+bx+c$ and remainder $r$.\[ \frac{12x^{3} -4 x^{2} -40 x + 37}{x + 2} \]\begin{enumerate}[label=\Alph*.]
\item \( a \in [-32, -22], \text{   } b \in [-55, -47], \text{   } c \in [-151, -142], \text{   and   } r \in [-251, -245]. \)
\item \( a \in [-32, -22], \text{   } b \in [36, 49], \text{   } c \in [-129, -124], \text{   and   } r \in [292, 296]. \)
\item \( a \in [12, 13], \text{   } b \in [-32, -24], \text{   } c \in [11, 19], \text{   and   } r \in [4, 9]. \)
\item \( a \in [12, 13], \text{   } b \in [-43, -38], \text{   } c \in [79, 82], \text{   and   } r \in [-205, -196]. \)
\item \( a \in [12, 13], \text{   } b \in [18, 26], \text{   } c \in [0, 1], \text{   and   } r \in [29, 47]. \)

\end{enumerate} }
\litem{
Perform the division below. Then, find the intervals that correspond to the quotient in the form $ax^2+bx+c$ and remainder $r$.\[ \frac{16x^{3} -48 x -28}{x -2} \]\begin{enumerate}[label=\Alph*.]
\item \( a \in [32, 33], b \in [-64, -59], c \in [80, 83], \text{ and } r \in [-195, -187]. \)
\item \( a \in [12, 23], b \in [-35, -27], c \in [9, 21], \text{ and } r \in [-60, -53]. \)
\item \( a \in [12, 23], b \in [26, 33], c \in [9, 21], \text{ and } r \in [3, 5]. \)
\item \( a \in [32, 33], b \in [61, 67], c \in [80, 83], \text{ and } r \in [130, 141]. \)
\item \( a \in [12, 23], b \in [15, 17], c \in [-39, -28], \text{ and } r \in [-60, -53]. \)

\end{enumerate} }
\litem{
What are the \textit{possible Integer} roots of the polynomial below?\[ f(x) = 2x^{2} +3 x + 7 \]\begin{enumerate}[label=\Alph*.]
\item \( \text{ All combinations of: }\frac{\pm 1,\pm 2}{\pm 1,\pm 7} \)
\item \( \text{ All combinations of: }\frac{\pm 1,\pm 7}{\pm 1,\pm 2} \)
\item \( \pm 1,\pm 7 \)
\item \( \pm 1,\pm 2 \)
\item \( \text{There is no formula or theorem that tells us all possible Integer roots.} \)

\end{enumerate} }
\litem{
Factor the polynomial below completely, knowing that $x -4$ is a factor. Then, choose the intervals the zeros of the polynomial belong to, where $z_1 \leq z_2 \leq z_3 \leq z_4$. \textit{To make the problem easier, all zeros are between -5 and 5.}\[ f(x) = 25x^{4} -80 x^{3} -132 x^{2} +224 x -64 \]\begin{enumerate}[label=\Alph*.]
\item \( z_1 \in [-2, 2], \text{   }  z_2 \in [-0.17, 0.41], z_3 \in [0.76, 0.9], \text{   and   } z_4 \in [3, 6] \)
\item \( z_1 \in [-5, -3], \text{   }  z_2 \in [-1.71, -0.25], z_3 \in [-0.63, -0.31], \text{   and   } z_4 \in [1, 3] \)
\item \( z_1 \in [-5, -3], \text{   }  z_2 \in [-2.13, -1.06], z_3 \in [-0.19, 0.06], \text{   and   } z_4 \in [1, 3] \)
\item \( z_1 \in [-2, 2], \text{   }  z_2 \in [1.21, 2.56], z_3 \in [2.4, 2.52], \text{   and   } z_4 \in [3, 6] \)
\item \( z_1 \in [-5, -3], \text{   }  z_2 \in [-2.8, -2.06], z_3 \in [-1.34, -1.19], \text{   and   } z_4 \in [1, 3] \)

\end{enumerate} }
\litem{
What are the \textit{possible Integer} roots of the polynomial below?\[ f(x) = 6x^{3} +4 x^{2} +4 x + 7 \]\begin{enumerate}[label=\Alph*.]
\item \( \pm 1,\pm 2,\pm 3,\pm 6 \)
\item \( \text{ All combinations of: }\frac{\pm 1,\pm 2,\pm 3,\pm 6}{\pm 1,\pm 7} \)
\item \( \pm 1,\pm 7 \)
\item \( \text{ All combinations of: }\frac{\pm 1,\pm 7}{\pm 1,\pm 2,\pm 3,\pm 6} \)
\item \( \text{There is no formula or theorem that tells us all possible Integer roots.} \)

\end{enumerate} }
\end{enumerate}

\end{document}