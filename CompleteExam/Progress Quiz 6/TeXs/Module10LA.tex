\documentclass[14pt]{extbook}
\usepackage{multicol, enumerate, enumitem, hyperref, color, soul, setspace, parskip, fancyhdr} %General Packages
\usepackage{amssymb, amsthm, amsmath, latexsym, units, mathtools} %Math Packages
\everymath{\displaystyle} %All math in Display Style
% Packages with additional options
\usepackage[headsep=0.5cm,headheight=12pt, left=1 in,right= 1 in,top= 1 in,bottom= 1 in]{geometry}
\usepackage[usenames,dvipsnames]{xcolor}
\usepackage{dashrule}  % Package to use the command below to create lines between items
\newcommand{\litem}[1]{\item#1\hspace*{-1cm}\rule{\textwidth}{0.4pt}}
\pagestyle{fancy}
\lhead{Progress Quiz 6}
\chead{}
\rhead{Version A}
\lfoot{4563-7456}
\cfoot{}
\rfoot{Summer C 2021}
\begin{document}

\begin{enumerate}
\litem{
Factor the polynomial below completely. Then, choose the intervals the zeros of the polynomial belong to, where $z_1 \leq z_2 \leq z_3$. \textit{To make the problem easier, all zeros are between -5 and 5.}\[ f(x) = 25x^{3} +75 x^{2} -16 x -48 \]\begin{enumerate}[label=\Alph*.]
\item \( z_1 \in [-3.16, -2.71], \text{   }  z_2 \in [-1.32, -1.21], \text{   and   } z_3 \in [1.09, 1.63] \)
\item \( z_1 \in [-1.28, -1.18], \text{   }  z_2 \in [1.09, 1.35], \text{   and   } z_3 \in [2.81, 3.32] \)
\item \( z_1 \in [-3.16, -2.71], \text{   }  z_2 \in [-0.86, -0.49], \text{   and   } z_3 \in [0.52, 0.91] \)
\item \( z_1 \in [-4.08, -3.92], \text{   }  z_2 \in [0.08, 0.21], \text{   and   } z_3 \in [2.81, 3.32] \)
\item \( z_1 \in [-0.9, -0.7], \text{   }  z_2 \in [0.52, 0.96], \text{   and   } z_3 \in [2.81, 3.32] \)

\end{enumerate} }
\litem{
Perform the division below. Then, find the intervals that correspond to the quotient in the form $ax^2+bx+c$ and remainder $r$.\[ \frac{20x^{3} -106 x^{2} +112 x -30}{x -4} \]\begin{enumerate}[label=\Alph*.]
\item \( a \in [79, 82], \text{   } b \in [-426, -424], \text{   } c \in [1811, 1818], \text{   and   } r \in [-7295, -7290]. \)
\item \( a \in [79, 82], \text{   } b \in [212, 216], \text{   } c \in [965, 973], \text{   and   } r \in [3836, 3844]. \)
\item \( a \in [17, 26], \text{   } b \in [-47, -44], \text{   } c \in [-27, -22], \text{   and   } r \in [-109, -104]. \)
\item \( a \in [17, 26], \text{   } b \in [-28, -23], \text{   } c \in [4, 11], \text{   and   } r \in [-1, 5]. \)
\item \( a \in [17, 26], \text{   } b \in [-192, -184], \text{   } c \in [855, 861], \text{   and   } r \in [-3457, -3450]. \)

\end{enumerate} }
\litem{
Perform the division below. Then, find the intervals that correspond to the quotient in the form $ax^2+bx+c$ and remainder $r$.\[ \frac{6x^{3} +28 x^{2} -68}{x + 4} \]\begin{enumerate}[label=\Alph*.]
\item \( a \in [-27, -23], b \in [123, 125], c \in [-498, -495], \text{ and } r \in [1913, 1919]. \)
\item \( a \in [3, 9], b \in [4, 9], c \in [-19, -11], \text{ and } r \in [-5, -3]. \)
\item \( a \in [-27, -23], b \in [-68, -63], c \in [-277, -267], \text{ and } r \in [-1157, -1153]. \)
\item \( a \in [3, 9], b \in [51, 53], c \in [208, 211], \text{ and } r \in [762, 767]. \)
\item \( a \in [3, 9], b \in [-6, 1], c \in [4, 15], \text{ and } r \in [-125, -117]. \)

\end{enumerate} }
\litem{
Factor the polynomial below completely. Then, choose the intervals the zeros of the polynomial belong to, where $z_1 \leq z_2 \leq z_3$. \textit{To make the problem easier, all zeros are between -5 and 5.}\[ f(x) = 10x^{3} -21 x^{2} -135 x -50 \]\begin{enumerate}[label=\Alph*.]
\item \( z_1 \in [-4.5, -1.5], \text{   }  z_2 \in [-0.52, -0.38], \text{   and   } z_3 \in [5, 7] \)
\item \( z_1 \in [-4.5, -1.5], \text{   }  z_2 \in [-0.52, -0.38], \text{   and   } z_3 \in [5, 7] \)
\item \( z_1 \in [-6, -4], \text{   }  z_2 \in [0.36, 0.46], \text{   and   } z_3 \in [1.5, 4.5] \)
\item \( z_1 \in [-6, -4], \text{   }  z_2 \in [0.01, 0.37], \text{   and   } z_3 \in [5, 7] \)
\item \( z_1 \in [-6, -4], \text{   }  z_2 \in [0.36, 0.46], \text{   and   } z_3 \in [1.5, 4.5] \)

\end{enumerate} }
\litem{
Factor the polynomial below completely, knowing that $x -3$ is a factor. Then, choose the intervals the zeros of the polynomial belong to, where $z_1 \leq z_2 \leq z_3 \leq z_4$. \textit{To make the problem easier, all zeros are between -5 and 5.}\[ f(x) = 9x^{4} +9 x^{3} -163 x^{2} +115 x + 150 \]\begin{enumerate}[label=\Alph*.]
\item \( z_1 \in [-5.2, -4.7], \text{   }  z_2 \in [-1.62, -1.48], z_3 \in [0.52, 0.63], \text{   and   } z_4 \in [2.4, 3.2] \)
\item \( z_1 \in [-5.2, -4.7], \text{   }  z_2 \in [-3.05, -2.99], z_3 \in [0.12, 0.28], \text{   and   } z_4 \in [4, 5.3] \)
\item \( z_1 \in [-3.7, -2], \text{   }  z_2 \in [-0.65, -0.6], z_3 \in [1.47, 1.5], \text{   and   } z_4 \in [4, 5.3] \)
\item \( z_1 \in [-5.2, -4.7], \text{   }  z_2 \in [-0.74, -0.64], z_3 \in [1.64, 1.74], \text{   and   } z_4 \in [2.4, 3.2] \)
\item \( z_1 \in [-3.7, -2], \text{   }  z_2 \in [-1.71, -1.63], z_3 \in [0.66, 0.71], \text{   and   } z_4 \in [4, 5.3] \)

\end{enumerate} }
\litem{
Perform the division below. Then, find the intervals that correspond to the quotient in the form $ax^2+bx+c$ and remainder $r$.\[ \frac{10x^{3} +26 x^{2} -68 x -53}{x + 4} \]\begin{enumerate}[label=\Alph*.]
\item \( a \in [-42, -33], \text{   } b \in [-134, -130], \text{   } c \in [-607, -603], \text{   and   } r \in [-2473, -2463]. \)
\item \( a \in [-42, -33], \text{   } b \in [185, 188], \text{   } c \in [-818, -809], \text{   and   } r \in [3195, 3197]. \)
\item \( a \in [9, 11], \text{   } b \in [-30, -21], \text{   } c \in [52, 57], \text{   and   } r \in [-321, -310]. \)
\item \( a \in [9, 11], \text{   } b \in [-16, -13], \text{   } c \in [-14, -10], \text{   and   } r \in [-8, -1]. \)
\item \( a \in [9, 11], \text{   } b \in [66, 70], \text{   } c \in [196, 202], \text{   and   } r \in [722, 739]. \)

\end{enumerate} }
\litem{
Perform the division below. Then, find the intervals that correspond to the quotient in the form $ax^2+bx+c$ and remainder $r$.\[ \frac{6x^{3} +28 x^{2} -62}{x + 4} \]\begin{enumerate}[label=\Alph*.]
\item \( a \in [4, 10], b \in [51, 53], c \in [207, 212], \text{ and } r \in [762, 773]. \)
\item \( a \in [4, 10], b \in [-6, 1], c \in [8, 13], \text{ and } r \in [-115, -105]. \)
\item \( a \in [-25, -21], b \in [120, 125], c \in [-503, -493], \text{ and } r \in [1917, 1926]. \)
\item \( a \in [4, 10], b \in [3, 8], c \in [-17, -14], \text{ and } r \in [2, 3]. \)
\item \( a \in [-25, -21], b \in [-72, -65], c \in [-277, -268], \text{ and } r \in [-1150, -1148]. \)

\end{enumerate} }
\litem{
What are the \textit{possible Integer} roots of the polynomial below?\[ f(x) = 6x^{2} +2 x + 3 \]\begin{enumerate}[label=\Alph*.]
\item \( \pm 1,\pm 3 \)
\item \( \pm 1,\pm 2,\pm 3,\pm 6 \)
\item \( \text{ All combinations of: }\frac{\pm 1,\pm 3}{\pm 1,\pm 2,\pm 3,\pm 6} \)
\item \( \text{ All combinations of: }\frac{\pm 1,\pm 2,\pm 3,\pm 6}{\pm 1,\pm 3} \)
\item \( \text{There is no formula or theorem that tells us all possible Integer roots.} \)

\end{enumerate} }
\litem{
Factor the polynomial below completely, knowing that $x -3$ is a factor. Then, choose the intervals the zeros of the polynomial belong to, where $z_1 \leq z_2 \leq z_3 \leq z_4$. \textit{To make the problem easier, all zeros are between -5 and 5.}\[ f(x) = 8x^{4} -90 x^{3} +343 x^{2} -510 x + 225 \]\begin{enumerate}[label=\Alph*.]
\item \( z_1 \in [-5.86, -4.88], \text{   }  z_2 \in [-3.65, -2.93], z_3 \in [-3.38, -2.77], \text{   and   } z_4 \in [-0.71, -0.43] \)
\item \( z_1 \in [-5.86, -4.88], \text{   }  z_2 \in [-3.65, -2.93], z_3 \in [-2.14, -0.63], \text{   and   } z_4 \in [-0.49, -0.24] \)
\item \( z_1 \in [-5.86, -4.88], \text{   }  z_2 \in [-3.65, -2.93], z_3 \in [-2.84, -2.19], \text{   and   } z_4 \in [-0.83, -0.74] \)
\item \( z_1 \in [0.6, 0.85], \text{   }  z_2 \in [2.02, 3.75], z_3 \in [2.3, 3.15], \text{   and   } z_4 \in [4.99, 5.07] \)
\item \( z_1 \in [0.14, 0.74], \text{   }  z_2 \in [1.06, 1.46], z_3 \in [2.3, 3.15], \text{   and   } z_4 \in [4.99, 5.07] \)

\end{enumerate} }
\litem{
What are the \textit{possible Rational} roots of the polynomial below?\[ f(x) = 7x^{3} +5 x^{2} +2 x + 5 \]\begin{enumerate}[label=\Alph*.]
\item \( \text{ All combinations of: }\frac{\pm 1,\pm 7}{\pm 1,\pm 5} \)
\item \( \pm 1,\pm 7 \)
\item \( \pm 1,\pm 5 \)
\item \( \text{ All combinations of: }\frac{\pm 1,\pm 5}{\pm 1,\pm 7} \)
\item \( \text{ There is no formula or theorem that tells us all possible Rational roots.} \)

\end{enumerate} }
\end{enumerate}

\end{document}