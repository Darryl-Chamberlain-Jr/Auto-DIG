\documentclass{extbook}[14pt]
\usepackage{multicol, enumerate, enumitem, hyperref, color, soul, setspace, parskip, fancyhdr, amssymb, amsthm, amsmath, latexsym, units, mathtools}
\everymath{\displaystyle}
\usepackage[headsep=0.5cm,headheight=0cm, left=1 in,right= 1 in,top= 1 in,bottom= 1 in]{geometry}
\usepackage{dashrule}  % Package to use the command below to create lines between items
\newcommand{\litem}[1]{\item #1

\rule{\textwidth}{0.4pt}}
\pagestyle{fancy}
\lhead{}
\chead{Answer Key for Progress Quiz 6 Version C}
\rhead{}
\lfoot{4563-7456}
\cfoot{}
\rfoot{Summer C 2021}
\begin{document}
\textbf{This key should allow you to understand why you choose the option you did (beyond just getting a question right or wrong). \href{https://xronos.clas.ufl.edu/mac1105spring2020/courseDescriptionAndMisc/Exams/LearningFromResults}{More instructions on how to use this key can be found here}.}

\textbf{If you have a suggestion to make the keys better, \href{https://forms.gle/CZkbZmPbC9XALEE88}{please fill out the short survey here}.}

\textit{Note: This key is auto-generated and may contain issues and/or errors. The keys are reviewed after each exam to ensure grading is done accurately. If there are issues (like duplicate options), they are noted in the offline gradebook. The keys are a work-in-progress to give students as many resources to improve as possible.}

\rule{\textwidth}{0.4pt}

\begin{enumerate}\litem{
Choose the \textbf{smallest} set of Real numbers that the number below belongs to.
\[ \sqrt{\frac{52900}{100}} \]The solution is \( \text{Whole} \), which is option A.\begin{enumerate}[label=\Alph*.]
\item \( \text{Whole} \)

* This is the correct option!
\item \( \text{Integer} \)

These are the negative and positive counting numbers (..., -3, -2, -1, 0, 1, 2, 3, ...)
\item \( \text{Rational} \)

These are numbers that can be written as fraction of Integers (e.g., -2/3)
\item \( \text{Irrational} \)

These cannot be written as a fraction of Integers.
\item \( \text{Not a Real number} \)

These are Nonreal Complex numbers \textbf{OR} things that are not numbers (e.g., dividing by 0).
\end{enumerate}

\textbf{General Comment:} First, you \textbf{NEED} to simplify the expression. This question simplifies to $230$. 
 
 Be sure you look at the simplified fraction and not just the decimal expansion. Numbers such as 13, 17, and 19 provide \textbf{long but repeating/terminating decimal expansions!} 
 
 The only ways to *not* be a Real number are: dividing by 0 or taking the square root of a negative number. 
 
 Irrational numbers are more than just square root of 3: adding or subtracting values from square root of 3 is also irrational.
}
\litem{
Simplify the expression below into the form $a+bi$. Then, choose the intervals that $a$ and $b$ belong to.
\[ (-4 - 5 i)(7 + 9 i) \]The solution is \( 17 - 71 i \), which is option A.\begin{enumerate}[label=\Alph*.]
\item \( a \in [17, 26] \text{ and } b \in [-74, -65] \)

* $17 - 71 i$, which is the correct option.
\item \( a \in [-31, -20] \text{ and } b \in [-48, -42] \)

 $-28 - 45 i$, which corresponds to just multiplying the real terms to get the real part of the solution and the coefficients in the complex terms to get the complex part.
\item \( a \in [-77, -70] \text{ and } b \in [0, 2] \)

 $-73 + i$, which corresponds to adding a minus sign in the second term.
\item \( a \in [-77, -70] \text{ and } b \in [-5, 0] \)

 $-73 - i$, which corresponds to adding a minus sign in the first term.
\item \( a \in [17, 26] \text{ and } b \in [71, 78] \)

 $17 + 71 i$, which corresponds to adding a minus sign in both terms.
\end{enumerate}

\textbf{General Comment:} You can treat $i$ as a variable and distribute. Just remember that $i^2=-1$, so you can continue to reduce after you distribute.
}
\litem{
Simplify the expression below and choose the interval the simplification is contained within.
\[ 13 - 1 \div 4 * 2 - (7 * 20) \]The solution is \( -127.500 \), which is option C.\begin{enumerate}[label=\Alph*.]
\item \( [109.29, 110.17] \)

 110.000, which corresponds to not distributing a negative correctly.
\item \( [-127.31, -126.91] \)

 -127.125, which corresponds to an Order of Operations error: not reading left-to-right for multiplication/division.
\item \( [-127.51, -127.25] \)

* -127.500, which is the correct option.
\item \( [152.53, 153.05] \)

 152.875, which corresponds to not distributing addition and subtraction correctly.
\item \( \text{None of the above} \)

 You may have gotten this by making an unanticipated error. If you got a value that is not any of the others, please let the coordinator know so they can help you figure out what happened.
\end{enumerate}

\textbf{General Comment:} While you may remember (or were taught) PEMDAS is done in order, it is actually done as P/E/MD/AS. When we are at MD or AS, we read left to right.
}
\litem{
Choose the \textbf{smallest} set of Complex numbers that the number below belongs to.
\[ \frac{2}{8}+\sqrt{-4}i \]The solution is \( \text{Rational} \), which is option A.\begin{enumerate}[label=\Alph*.]
\item \( \text{Rational} \)

* This is the correct option!
\item \( \text{Not a Complex Number} \)

This is not a number. The only non-Complex number we know is dividing by 0 as this is not a number!
\item \( \text{Pure Imaginary} \)

This is a Complex number $(a+bi)$ that \textbf{only} has an imaginary part like $2i$.
\item \( \text{Nonreal Complex} \)

This is a Complex number $(a+bi)$ that is not Real (has $i$ as part of the number).
\item \( \text{Irrational} \)

These cannot be written as a fraction of Integers. Remember: $\pi$ is not an Integer!
\end{enumerate}

\textbf{General Comment:} Be sure to simplify $i^2 = -1$. This may remove the imaginary portion for your number. If you are having trouble, you may want to look at the \textit{Subgroups of the Real Numbers} section.
}
\litem{
Simplify the expression below into the form $a+bi$. Then, choose the intervals that $a$ and $b$ belong to.
\[ (-10 - 6 i)(5 - 8 i) \]The solution is \( -98 + 50 i \), which is option D.\begin{enumerate}[label=\Alph*.]
\item \( a \in [-3, 0] \text{ and } b \in [108.8, 113] \)

 $-2 + 110 i$, which corresponds to adding a minus sign in the first term.
\item \( a \in [-99, -97] \text{ and } b \in [-52.5, -49.9] \)

 $-98 - 50 i$, which corresponds to adding a minus sign in both terms.
\item \( a \in [-3, 0] \text{ and } b \in [-112.2, -107.1] \)

 $-2 - 110 i$, which corresponds to adding a minus sign in the second term.
\item \( a \in [-99, -97] \text{ and } b \in [48.4, 50.8] \)

* $-98 + 50 i$, which is the correct option.
\item \( a \in [-52, -43] \text{ and } b \in [46.2, 48.4] \)

 $-50 + 48 i$, which corresponds to just multiplying the real terms to get the real part of the solution and the coefficients in the complex terms to get the complex part.
\end{enumerate}

\textbf{General Comment:} You can treat $i$ as a variable and distribute. Just remember that $i^2=-1$, so you can continue to reduce after you distribute.
}
\litem{
Simplify the expression below and choose the interval the simplification is contained within.
\[ 5 - 18^2 + 14 \div 1 * 19 \div 4 \]The solution is \( -252.500 \), which is option C.\begin{enumerate}[label=\Alph*.]
\item \( [327.18, 332.18] \)

 329.184, which corresponds to two Order of Operations errors.
\item \( [-325.82, -316.82] \)

 -318.816, which corresponds to an Order of Operations error: not reading left-to-right for multiplication/division.
\item \( [-255.5, -250.5] \)

* -252.500, this is the correct option
\item \( [395.5, 401.5] \)

 395.500, which corresponds to an Order of Operations error: multiplying by negative before squaring. For example: $(-3)^2 \neq -3^2$
\item \( \text{None of the above} \)

 You may have gotten this by making an unanticipated error. If you got a value that is not any of the others, please let the coordinator know so they can help you figure out what happened.
\end{enumerate}

\textbf{General Comment:} While you may remember (or were taught) PEMDAS is done in order, it is actually done as P/E/MD/AS. When we are at MD or AS, we read left to right.
}
\litem{
Simplify the expression below into the form $a+bi$. Then, choose the intervals that $a$ and $b$ belong to.
\[ \frac{-36 - 88 i}{2 - 6 i} \]The solution is \( 11.40  - 9.80 i \), which is option B.\begin{enumerate}[label=\Alph*.]
\item \( a \in [-15.5, -14] \text{ and } b \in [0, 1.5] \)

 $-15.00 + i$, which corresponds to forgetting to multiply the conjugate by the numerator and not computing the conjugate correctly.
\item \( a \in [9.5, 11.5] \text{ and } b \in [-10.5, -9] \)

* $11.40  - 9.80 i$, which is the correct option.
\item \( a \in [-19, -17] \text{ and } b \in [13.5, 15] \)

 $-18.00  + 14.67 i$, which corresponds to just dividing the first term by the first term and the second by the second.
\item \( a \in [9.5, 11.5] \text{ and } b \in [-393, -391] \)

 $11.40  - 392.00 i$, which corresponds to forgetting to multiply the conjugate by the numerator.
\item \( a \in [455, 456.5] \text{ and } b \in [-10.5, -9] \)

 $456.00  - 9.80 i$, which corresponds to forgetting to multiply the conjugate by the numerator and using a plus instead of a minus in the denominator.
\end{enumerate}

\textbf{General Comment:} Multiply the numerator and denominator by the *conjugate* of the denominator, then simplify. For example, if we have $2+3i$, the conjugate is $2-3i$.
}
\litem{
Choose the \textbf{smallest} set of Complex numbers that the number below belongs to.
\[ \frac{\sqrt{110}}{14}+5i^2 \]The solution is \( \text{Irrational} \), which is option D.\begin{enumerate}[label=\Alph*.]
\item \( \text{Pure Imaginary} \)

This is a Complex number $(a+bi)$ that \textbf{only} has an imaginary part like $2i$.
\item \( \text{Not a Complex Number} \)

This is not a number. The only non-Complex number we know is dividing by 0 as this is not a number!
\item \( \text{Nonreal Complex} \)

This is a Complex number $(a+bi)$ that is not Real (has $i$ as part of the number).
\item \( \text{Irrational} \)

* This is the correct option!
\item \( \text{Rational} \)

These are numbers that can be written as fraction of Integers (e.g., -2/3 + 5)
\end{enumerate}

\textbf{General Comment:} Be sure to simplify $i^2 = -1$. This may remove the imaginary portion for your number. If you are having trouble, you may want to look at the \textit{Subgroups of the Real Numbers} section.
}
\litem{
Simplify the expression below into the form $a+bi$. Then, choose the intervals that $a$ and $b$ belong to.
\[ \frac{36 - 55 i}{-1 - 8 i} \]The solution is \( 6.22  + 5.28 i \), which is option E.\begin{enumerate}[label=\Alph*.]
\item \( a \in [-37, -35] \text{ and } b \in [6, 8] \)

 $-36.00  + 6.88 i$, which corresponds to just dividing the first term by the first term and the second by the second.
\item \( a \in [6, 7] \text{ and } b \in [342.5, 343.5] \)

 $6.22  + 343.00 i$, which corresponds to forgetting to multiply the conjugate by the numerator.
\item \( a \in [-8, -6] \text{ and } b \in [-4.5, -2] \)

 $-7.32  - 3.58 i$, which corresponds to forgetting to multiply the conjugate by the numerator and not computing the conjugate correctly.
\item \( a \in [403.5, 404.5] \text{ and } b \in [4.5, 6] \)

 $404.00  + 5.28 i$, which corresponds to forgetting to multiply the conjugate by the numerator and using a plus instead of a minus in the denominator.
\item \( a \in [6, 7] \text{ and } b \in [4.5, 6] \)

* $6.22  + 5.28 i$, which is the correct option.
\end{enumerate}

\textbf{General Comment:} Multiply the numerator and denominator by the *conjugate* of the denominator, then simplify. For example, if we have $2+3i$, the conjugate is $2-3i$.
}
\litem{
Choose the \textbf{smallest} set of Real numbers that the number below belongs to.
\[ \sqrt{\frac{1540}{10}} \]The solution is \( \text{Irrational} \), which is option B.\begin{enumerate}[label=\Alph*.]
\item \( \text{Integer} \)

These are the negative and positive counting numbers (..., -3, -2, -1, 0, 1, 2, 3, ...)
\item \( \text{Irrational} \)

* This is the correct option!
\item \( \text{Whole} \)

These are the counting numbers with 0 (0, 1, 2, 3, ...)
\item \( \text{Rational} \)

These are numbers that can be written as fraction of Integers (e.g., -2/3)
\item \( \text{Not a Real number} \)

These are Nonreal Complex numbers \textbf{OR} things that are not numbers (e.g., dividing by 0).
\end{enumerate}

\textbf{General Comment:} First, you \textbf{NEED} to simplify the expression. This question simplifies to $\sqrt{154}$. 
 
 Be sure you look at the simplified fraction and not just the decimal expansion. Numbers such as 13, 17, and 19 provide \textbf{long but repeating/terminating decimal expansions!} 
 
 The only ways to *not* be a Real number are: dividing by 0 or taking the square root of a negative number. 
 
 Irrational numbers are more than just square root of 3: adding or subtracting values from square root of 3 is also irrational.
}
\end{enumerate}

\end{document}