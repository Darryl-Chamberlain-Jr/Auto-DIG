\documentclass{extbook}[14pt]
\usepackage{multicol, enumerate, enumitem, hyperref, color, soul, setspace, parskip, fancyhdr, amssymb, amsthm, amsmath, latexsym, units, mathtools}
\everymath{\displaystyle}
\usepackage[headsep=0.5cm,headheight=0cm, left=1 in,right= 1 in,top= 1 in,bottom= 1 in]{geometry}
\usepackage{dashrule}  % Package to use the command below to create lines between items
\newcommand{\litem}[1]{\item #1

\rule{\textwidth}{0.4pt}}
\pagestyle{fancy}
\lhead{}
\chead{Answer Key for Progress Quiz 6 Version C}
\rhead{}
\lfoot{4563-7456}
\cfoot{}
\rfoot{Summer C 2021}
\begin{document}
\textbf{This key should allow you to understand why you choose the option you did (beyond just getting a question right or wrong). \href{https://xronos.clas.ufl.edu/mac1105spring2020/courseDescriptionAndMisc/Exams/LearningFromResults}{More instructions on how to use this key can be found here}.}

\textbf{If you have a suggestion to make the keys better, \href{https://forms.gle/CZkbZmPbC9XALEE88}{please fill out the short survey here}.}

\textit{Note: This key is auto-generated and may contain issues and/or errors. The keys are reviewed after each exam to ensure grading is done accurately. If there are issues (like duplicate options), they are noted in the offline gradebook. The keys are a work-in-progress to give students as many resources to improve as possible.}

\rule{\textwidth}{0.4pt}

\begin{enumerate}\litem{
Choose the graph of the equation below.
\[ f(x) = \frac{-1}{(x + 1)^2} - 1 \]The solution is the graph below, which is option D.
    \begin{center}
        \includegraphics[width=0.3\textwidth]{../Figures/rationalEquationToGraphDC.png}
    \end{center}\begin{enumerate}[label=\Alph*.]
\begin{multicols}{2}
\item \includegraphics[width = 0.3\textwidth]{../Figures/rationalEquationToGraphAC.png}
\item \includegraphics[width = 0.3\textwidth]{../Figures/rationalEquationToGraphBC.png}
\item \includegraphics[width = 0.3\textwidth]{../Figures/rationalEquationToGraphCC.png}
\item \includegraphics[width = 0.3\textwidth]{../Figures/rationalEquationToGraphDC.png}
\end{multicols}\item None of the above.\end{enumerate}
\textbf{General Comment:} Remember that the general form of a basic rational equation is $ f(x) = \frac{a}{(x-h)^n} + k$, where $a$ is the leading coefficient (and in this case, we assume is either $1$ or $-1$), $n$ is the degree (in this case, either $1$ or $2$), and $(h, k)$ is the intersection of the asymptotes.
}
\litem{
Choose the equation of the function graphed below.

\begin{center}
    \includegraphics[width=0.5\textwidth]{../Figures/rationalGraphToEquationC.png}
\end{center}


The solution is \( f(x) = \frac{1}{(x + 1)^2} + 2 \), which is option A.\begin{enumerate}[label=\Alph*.]
\item \( f(x) = \frac{1}{(x + 1)^2} + 2 \)

This is the correct option.
\item \( f(x) = \frac{1}{x + 1} + 2 \)

Corresponds to thinking the graph was a shifted version of $\frac{1}{x}$.
\item \( f(x) = \frac{-1}{x - 1} + 2 \)

Corresponds to thinking the graph was a shifted version of $\frac{1}{x}$, using the general form $f(x) = \frac{a}{(x+h)^2}+k$, and the opposite leading coefficient.
\item \( f(x) = \frac{-1}{(x - 1)^2} + 2 \)

Corresponds to using the general form $f(x) = \frac{a}{(x+h)^2}+k$ and the opposite leading coefficient.
\item \( \text{None of the above} \)

This corresponds to believing the vertex of the graph was not correct.
\end{enumerate}

\textbf{General Comment:} Remember that the general form of a basic rational equation is $ f(x) = \frac{a}{(x-h)^n} + k$, where $a$ is the leading coefficient (and in this case, we assume is either $1$ or $-1$), $n$ is the degree (in this case, either $1$ or $2$), and $(h, k)$ is the intersection of the asymptotes.
}
\litem{
Solve the rational equation below. Then, choose the interval(s) that the solution(s) belongs to.
\[ \frac{7x}{-3x + 2} + \frac{-4x^{2}}{-15x^{2} -2 x + 8} = \frac{3}{5x + 4} \]The solution is \( \text{There are two solutions: } x = 0.145 \text{ and } x = -1.338 \), which is option D.\begin{enumerate}[label=\Alph*.]
\item \( x_1 \in [-0.23, 0.38] \text{ and } x_2 \in [-0.4,2.07] \)


\item \( x \in [-1.01,-0.56] \)


\item \( x \in [-1.6,-1.11] \)


\item \( x_1 \in [-0.23, 0.38] \text{ and } x_2 \in [-1.8,-1.25] \)

* $x = 0.145 \text{ and } x = -1.338$, which is the correct option.
\item \( \text{All solutions lead to invalid or complex values in the equation.} \)


\end{enumerate}

\textbf{General Comment:} Distractors are different based on the number of solutions. Remember that after solving, we need to make sure our solution does not make the original equation divide by zero!
}
\litem{
Solve the rational equation below. Then, choose the interval(s) that the solution(s) belongs to.
\[ \frac{-63}{63x -21} + 1 = \frac{-63}{63x -21} \]The solution is \( \text{all solutions are invalid or lead to complex values in the equation.} \), which is option D.\begin{enumerate}[label=\Alph*.]
\item \( x_1 \in [-1.1, 0] \text{ and } x_2 \in [0.33,2.33] \)

$x = -0.333 \text{ and } x = 0.333$, which corresponds to getting the correct solution and believing there should be a second solution to the equation.
\item \( x \in [-1.1,0] \)

$x = -0.333$, which corresponds to not distributing the factor $63x -21$ correctly when trying to eliminate the fraction.
\item \( x \in [0.33,2.33] \)

$x = 0.333$, which corresponds to not checking if this value leads to dividing by 0 in the original equation and thus is not a valid solution.
\item \( \text{All solutions lead to invalid or complex values in the equation.} \)

*$x = 0.333$ leads to dividing by 0 in the original equation and thus is not a valid solution, which is the correct option.
\item \( x_1 \in [-0.1, 0.6] \text{ and } x_2 \in [0.33,2.33] \)

$x = 0.333 \text{ and } x = 0.333$, which corresponds to getting the correct solution and believing there should be a second solution to the equation.
\end{enumerate}

\textbf{General Comment:} Distractors are different based on the number of solutions. Remember that after solving, we need to make sure our solution does not make the original equation divide by zero!
}
\litem{
Choose the equation of the function graphed below.

\begin{center}
    \includegraphics[width=0.5\textwidth]{../Figures/rationalGraphToEquationCopyC.png}
\end{center}


The solution is \( \text{None of the above as it should be } f(x) = \frac{-1}{(x + 2)^2} - 3 \), which is option E.\begin{enumerate}[label=\Alph*.]
\item \( f(x) = \frac{1}{x + 2} - 3 \)

Corresponds to thinking the graph was a shifted version of $\frac{1}{x}$, using the general form $f(x) = \frac{a}{(x-h)^2}+k$, and the opposite leading coefficient.
\item \( f(x) = \frac{-1}{(x - 2)^2} - 3 \)

The $x$-value of the equation does not match the graph.
\item \( f(x) = \frac{1}{(x + 2)^2} - 3 \)

Corresponds to using the general form $f(x) = \frac{a}{(x-h)^2}+k$ and the opposite leading coefficient.
\item \( f(x) = \frac{-1}{x - 2} - 3 \)

Corresponds to thinking the graph was a shifted version of $\frac{1}{x}$.
\item \( \text{None of the above} \)

None of the equation options were the correct equation.
\end{enumerate}

\textbf{General Comment:} Remember that the general form of a basic rational equation is $ f(x) = \frac{a}{(x-h)^n} + k$, where $a$ is the leading coefficient (and in this case, we assume is either $1$ or $-1$), $n$ is the degree (in this case, either $1$ or $2$), and $(h, k)$ is the intersection of the asymptotes.
}
\litem{
Solve the rational equation below. Then, choose the interval(s) that the solution(s) belongs to.
\[ \frac{-5}{-6x + 2} + 8 = \frac{-6}{-12x + 4} \]The solution is \( x = 0.292 \), which is option B.\begin{enumerate}[label=\Alph*.]
\item \( x_1 \in [0.03, 0.67] \text{ and } x_2 \in [0.34,0.41] \)

$x = 0.292 \text{ and } x = 0.354$, which corresponds to getting the correct solution and believing there should be a second solution to the equation.
\item \( x \in [0.29,1.29] \)

* $x = 0.292$, which is the correct option.
\item \( x_1 \in [-0.4, -0.35] \text{ and } x_2 \in [0.13,0.33] \)

$x = -0.375 \text{ and } x = 0.292$, which corresponds to getting the correct solution and believing there should be a second solution to the equation.
\item \( \text{All solutions lead to invalid or complex values in the equation.} \)

This corresponds to thinking $x = 0.292$ leads to dividing by zero in the original equation, which it does not.
\item \( x \in [-0.4,-0.35] \)

$x = -0.375$, which corresponds to not distributing the factor $-6x + 2$ correctly when trying to eliminate the fraction.
\end{enumerate}

\textbf{General Comment:} Distractors are different based on the number of solutions. Remember that after solving, we need to make sure our solution does not make the original equation divide by zero!
}
\litem{
Choose the graph of the equation below.
\[ f(x) = \frac{-1}{x - 1} - 1 \]The solution is the graph below, which is option E.
    \begin{center}
        \includegraphics[width=0.3\textwidth]{../Figures/rationalEquationToGraphCopyEC.png}
    \end{center}\begin{enumerate}[label=\Alph*.]
\begin{multicols}{2}
\item \includegraphics[width = 0.3\textwidth]{../Figures/rationalEquationToGraphCopyAC.png}
\item \includegraphics[width = 0.3\textwidth]{../Figures/rationalEquationToGraphCopyBC.png}
\item \includegraphics[width = 0.3\textwidth]{../Figures/rationalEquationToGraphCopyCC.png}
\item \includegraphics[width = 0.3\textwidth]{../Figures/rationalEquationToGraphCopyDC.png}
\end{multicols}\item None of the above.\end{enumerate}
\textbf{General Comment:} Remember that the general form of a basic rational equation is $ f(x) = \frac{a}{(x-h)^n} + k$, where $a$ is the leading coefficient (and in this case, we assume is either $1$ or $-1$), $n$ is the degree (in this case, either $1$ or $2$), and $(h, k)$ is the intersection of the asymptotes.
}
\litem{
Determine the domain of the function below.
\[ f(x) = \frac{6}{12x^{2} -33 x + 18} \]The solution is \( \text{All Real numbers except } x = 0.750 \text{ and } x = 2.000. \), which is option E.\begin{enumerate}[label=\Alph*.]
\item \( \text{All Real numbers except } x = a, \text{ where } a \in [0.75, 1.75] \)

All Real numbers except $x = 0.750$, which corresponds to removing only 1 value from the denominator.
\item \( \text{All Real numbers.} \)

This corresponds to thinking the denominator has complex roots or that rational functions have a domain of all Real numbers.
\item \( \text{All Real numbers except } x = a, \text{ where } a \in [11, 14] \)

All Real numbers except $x = 12.000$, which corresponds to removing a distractor value from the denominator.
\item \( \text{All Real numbers except } x = a \text{ and } x = b, \text{ where } a \in [11, 14] \text{ and } b \in [18, 20] \)

All Real numbers except $x = 12.000$ and $x = 18.000$, which corresponds to not factoring the denominator correctly.
\item \( \text{All Real numbers except } x = a \text{ and } x = b, \text{ where } a \in [0.75, 1.75] \text{ and } b \in [2, 4] \)

All Real numbers except $x = 0.750$ and $x = 2.000$, which is the correct option.
\end{enumerate}

\textbf{General Comment:} Recall that dividing by zero is not a real number. Therefore the domain is all real numbers \textbf{except} those that make the denominator 0.
}
\litem{
Determine the domain of the function below.
\[ f(x) = \frac{6}{18x^{2} -6 x -12} \]The solution is \( \text{All Real numbers except } x = -0.667 \text{ and } x = 1.000. \), which is option E.\begin{enumerate}[label=\Alph*.]
\item \( \text{All Real numbers.} \)

This corresponds to thinking the denominator has complex roots or that rational functions have a domain of all Real numbers.
\item \( \text{All Real numbers except } x = a, \text{ where } a \in [-1.67, 0.33] \)

All Real numbers except $x = -0.667$, which corresponds to removing only 1 value from the denominator.
\item \( \text{All Real numbers except } x = a \text{ and } x = b, \text{ where } a \in [-25, -21] \text{ and } b \in [5, 11] \)

All Real numbers except $x = -24.000$ and $x = 9.000$, which corresponds to not factoring the denominator correctly.
\item \( \text{All Real numbers except } x = a, \text{ where } a \in [-25, -21] \)

All Real numbers except $x = -24.000$, which corresponds to removing a distractor value from the denominator.
\item \( \text{All Real numbers except } x = a \text{ and } x = b, \text{ where } a \in [-1.67, 0.33] \text{ and } b \in [0, 3] \)

All Real numbers except $x = -0.667$ and $x = 1.000$, which is the correct option.
\end{enumerate}

\textbf{General Comment:} Recall that dividing by zero is not a real number. Therefore the domain is all real numbers \textbf{except} those that make the denominator 0.
}
\litem{
Solve the rational equation below. Then, choose the interval(s) that the solution(s) belongs to.
\[ \frac{6x}{7x + 3} + \frac{-3x^{2}}{35x^{2} -27 x -18} = \frac{6}{5x -6} \]The solution is \( \text{There are two solutions: } x = -0.215 \text{ and } x = 3.104 \), which is option B.\begin{enumerate}[label=\Alph*.]
\item \( x \in [1.92,3.52] \)


\item \( x_1 \in [-0.41, 0.06] \text{ and } x_2 \in [1.1,6.1] \)

* $x = -0.215 \text{ and } x = 3.104$, which is the correct option.
\item \( x_1 \in [-0.41, 0.06] \text{ and } x_2 \in [-6.43,0.57] \)


\item \( \text{All solutions lead to invalid or complex values in the equation.} \)


\item \( x \in [0.93,1.32] \)


\end{enumerate}

\textbf{General Comment:} Distractors are different based on the number of solutions. Remember that after solving, we need to make sure our solution does not make the original equation divide by zero!
}
\end{enumerate}

\end{document}