\documentclass{extbook}[14pt]
\usepackage{multicol, enumerate, enumitem, hyperref, color, soul, setspace, parskip, fancyhdr, amssymb, amsthm, amsmath, latexsym, units, mathtools}
\everymath{\displaystyle}
\usepackage[headsep=0.5cm,headheight=0cm, left=1 in,right= 1 in,top= 1 in,bottom= 1 in]{geometry}
\usepackage{dashrule}  % Package to use the command below to create lines between items
\newcommand{\litem}[1]{\item #1

\rule{\textwidth}{0.4pt}}
\pagestyle{fancy}
\lhead{}
\chead{Answer Key for Progress Quiz 6 Version C}
\rhead{}
\lfoot{1430-1829}
\cfoot{}
\rfoot{test}
\begin{document}
\textbf{This key should allow you to understand why you choose the option you did (beyond just getting a question right or wrong). \href{https://xronos.clas.ufl.edu/mac1105spring2020/courseDescriptionAndMisc/Exams/LearningFromResults}{More instructions on how to use this key can be found here}.}

\textbf{If you have a suggestion to make the keys better, \href{https://forms.gle/CZkbZmPbC9XALEE88}{please fill out the short survey here}.}

\textit{Note: This key is auto-generated and may contain issues and/or errors. The keys are reviewed after each exam to ensure grading is done accurately. If there are issues (like duplicate options), they are noted in the offline gradebook. The keys are a work-in-progress to give students as many resources to improve as possible.}

\rule{\textwidth}{0.4pt}

\begin{enumerate}\litem{
Determine the domain of the function below.
\[ f(x) = \frac{6}{12x^{2} -32 x + 20} \]The solution is \( \text{All Real numbers except } x = 1.000 \text{ and } x = 1.667. \), which is option D.\begin{enumerate}[label=\Alph*.]
\item \( \text{All Real numbers.} \)

This corresponds to thinking the denominator has complex roots or that rational functions have a domain of all Real numbers.
\item \( \text{All Real numbers except } x = a \text{ and } x = b, \text{ where } a \in [14.14, 15.28] \text{ and } b \in [15.67, 16.82] \)

All Real numbers except $x = 15.000$ and $x = 16.000$, which corresponds to not factoring the denominator correctly.
\item \( \text{All Real numbers except } x = a, \text{ where } a \in [14.14, 15.28] \)

All Real numbers except $x = 15.000$, which corresponds to removing a distractor value from the denominator.
\item \( \text{All Real numbers except } x = a \text{ and } x = b, \text{ where } a \in [0.63, 1.58] \text{ and } b \in [1.16, 2.36] \)

All Real numbers except $x = 1.000$ and $x = 1.667$, which is the correct option.
\item \( \text{All Real numbers except } x = a, \text{ where } a \in [0.63, 1.58] \)

All Real numbers except $x = 1.000$, which corresponds to removing only 1 value from the denominator.
\end{enumerate}

\textbf{General Comment:} Recall that dividing by zero is not a real number. Therefore the domain is all real numbers \textbf{except} those that make the denominator 0.
}
\litem{
Solve the rational equation below. Then, choose the interval(s) that the solution(s) belongs to.
\[ \frac{48}{32x -72} + 1 = \frac{48}{32x -72} \]The solution is \( \text{all solutions are invalid or lead to complex values in the equation.} \), which is option D.\begin{enumerate}[label=\Alph*.]
\item \( x_1 \in [0.25, 4.25] \text{ and } x_2 \in [0.25,3.25] \)

$x = 2.250 \text{ and } x = 2.250$, which corresponds to getting the correct solution and believing there should be a second solution to the equation.
\item \( x \in [2.25,3.25] \)

$x = 2.250$, which corresponds to not checking if this value leads to dividing by 0 in the original equation and thus is not a valid solution.
\item \( x_1 \in [-2.25, -0.25] \text{ and } x_2 \in [0.25,3.25] \)

$x = -2.250 \text{ and } x = 2.250$, which corresponds to getting the correct solution and believing there should be a second solution to the equation.
\item \( \text{All solutions lead to invalid or complex values in the equation.} \)

*$x = 2.250$ leads to dividing by 0 in the original equation and thus is not a valid solution, which is the correct option.
\item \( x \in [-2.25,-0.25] \)

$x = -2.250$, which corresponds to not distributing the factor $32x -72$ correctly when trying to eliminate the fraction.
\end{enumerate}

\textbf{General Comment:} Distractors are different based on the number of solutions. Remember that after solving, we need to make sure our solution does not make the original equation divide by zero!
}
\litem{
Solve the rational equation below. Then, choose the interval(s) that the solution(s) belongs to.
\[ \frac{7x + 0}{4x + 5} + \frac{-7x^{2} +0 x + 0}{12x^{2} -x -20} = \frac{3}{3x -4} \]The solution is \( \text{There are two solutions: } x = -0.336 \text{ and } x = 3.193 \), which is option C.\begin{enumerate}[label=\Alph*.]
\item \( x \in [2.97,4.51] \)


\item \( x_1 \in [-0.36, -0.23] \text{ and } x_2 \in [-3.1,2.8] \)


\item \( x_1 \in [-0.36, -0.23] \text{ and } x_2 \in [2.2,4] \)

* $x = -0.336 \text{ and } x = 3.193$, which is the correct option.
\item \( \text{All solutions lead to invalid or complex values in the equation.} \)


\item \( x \in [0.66,2.27] \)


\end{enumerate}

\textbf{General Comment:} Distractors are different based on the number of solutions. Remember that after solving, we need to make sure our solution does not make the original equation divide by zero!
}
\litem{
Determine the domain of the function below.
\[ f(x) = \frac{3}{12x^{2} -6 x -18} \]The solution is \( \text{All Real numbers except } x = -1.000 \text{ and } x = 1.500. \), which is option A.\begin{enumerate}[label=\Alph*.]
\item \( \text{All Real numbers except } x = a \text{ and } x = b, \text{ where } a \in [-2, 0] \text{ and } b \in [-0.5, 2.5] \)

All Real numbers except $x = -1.000$ and $x = 1.500$, which is the correct option.
\item \( \text{All Real numbers except } x = a \text{ and } x = b, \text{ where } a \in [-11, -6] \text{ and } b \in [21, 27] \)

All Real numbers except $x = -9.000$ and $x = 24.000$, which corresponds to not factoring the denominator correctly.
\item \( \text{All Real numbers except } x = a, \text{ where } a \in [-11, -6] \)

All Real numbers except $x = -9.000$, which corresponds to removing a distractor value from the denominator.
\item \( \text{All Real numbers except } x = a, \text{ where } a \in [-2, 0] \)

All Real numbers except $x = -1.000$, which corresponds to removing only 1 value from the denominator.
\item \( \text{All Real numbers.} \)

This corresponds to thinking the denominator has complex roots or that rational functions have a domain of all Real numbers.
\end{enumerate}

\textbf{General Comment:} Recall that dividing by zero is not a real number. Therefore the domain is all real numbers \textbf{except} those that make the denominator 0.
}
\litem{
Choose the equation of the function graphed below.

\begin{center}
    \includegraphics[width=0.5\textwidth]{../Figures/rationalGraphToEquationCopyC.png}
\end{center}


The solution is \( f(x) = \frac{-1}{x + 1} + 1 \), which is option B.\begin{enumerate}[label=\Alph*.]
\item \( f(x) = \frac{1}{(x - 1)^2} + 1 \)

Corresponds to thinking the graph was a shifted version of $\frac{1}{x^2}$, using the general form $f(x) = \frac{a}{x+h}+k$, and the opposite leading coefficient.
\item \( f(x) = \frac{-1}{x + 1} + 1 \)

This is the correct option.
\item \( f(x) = \frac{-1}{(x + 1)^2} + 1 \)

Corresponds to thinking the graph was a shifted version of $\frac{1}{x^2}$.
\item \( f(x) = \frac{1}{x - 1} + 1 \)

Corresponds to using the general form $f(x) = \frac{a}{x+h}+k$ and the opposite leading coefficient.
\item \( \text{None of the above} \)

This corresponds to believing the vertex of the graph was not correct.
\end{enumerate}

\textbf{General Comment:} Remember that the general form of a basic rational equation is $ f(x) = \frac{a}{(x-h)^n} + k$, where $a$ is the leading coefficient (and in this case, we assume is either $1$ or $-1$), $n$ is the degree (in this case, either $1$ or $2$), and $(h, k)$ is the intersection of the asymptotes.
}
\litem{
Choose the graph of the equation below.
\[ f(x) = \frac{-1}{(x - 2)^2} + 3 \]The solution is the graph below, which is option B.
\begin{center}
    \includegraphics[width=0.3\textwidth]{../Figures/rationalEquationToGraphBC.png}
\end{center}\begin{enumerate}[label=\Alph*.]
\begin{multicols}{2}
\item \includegraphics[width = 0.3\textwidth]{../Figures/rationalEquationToGraphAC.png}
\item \includegraphics[width = 0.3\textwidth]{../Figures/rationalEquationToGraphBC.png}
\item \includegraphics[width = 0.3\textwidth]{../Figures/rationalEquationToGraphCC.png}
\item \includegraphics[width = 0.3\textwidth]{../Figures/rationalEquationToGraphDC.png}
\end{multicols}\item None of the above.\end{enumerate}
\textbf{General Comment:} Remember that the general form of a basic rational equation is $ f(x) = \frac{a}{(x-h)^n} + k$, where $a$ is the leading coefficient (and in this case, we assume is either $1$ or $-1$), $n$ is the degree (in this case, either $1$ or $2$), and $(h, k)$ is the intersection of the asymptotes.
}
\litem{
Solve the rational equation below. Then, choose the interval(s) that the solution(s) belongs to.
\[ \frac{65}{52x + 39} + 1 = \frac{65}{52x + 39} \]The solution is \( \text{all solutions are invalid or lead to complex values in the equation.} \), which is option A.\begin{enumerate}[label=\Alph*.]
\item \( \text{All solutions lead to invalid or complex values in the equation.} \)

*$x = -0.750$ leads to dividing by 0 in the original equation and thus is not a valid solution, which is the correct option.
\item \( x \in [-0.75,0.25] \)

$x = -0.750$, which corresponds to not checking if this value leads to dividing by 0 in the original equation and thus is not a valid solution.
\item \( x_1 \in [-0.75, 0.25] \text{ and } x_2 \in [-1.2,-0.1] \)

$x = -0.750 \text{ and } x = -0.750$, which corresponds to getting the correct solution and believing there should be a second solution to the equation.
\item \( x \in [-0.25,1.75] \)

$x = 0.750$, which corresponds to not distributing the factor $52x + 39$ correctly when trying to eliminate the fraction.
\item \( x_1 \in [-0.75, 0.25] \text{ and } x_2 \in [0.2,1.7] \)

$x = -0.750 \text{ and } x = 0.750$, which corresponds to getting the correct solution and believing there should be a second solution to the equation.
\end{enumerate}

\textbf{General Comment:} Distractors are different based on the number of solutions. Remember that after solving, we need to make sure our solution does not make the original equation divide by zero!
}
\litem{
Choose the graph of the equation below.
\[ f(x) = \frac{1}{x - 1} - 3 \]The solution is the graph below, which is option E.
\begin{center}
    \includegraphics[width=0.3\textwidth]{../Figures/rationalEquationToGraphCopyEC.png}
\end{center}\begin{enumerate}[label=\Alph*.]
\begin{multicols}{2}
\item \includegraphics[width = 0.3\textwidth]{../Figures/rationalEquationToGraphCopyAC.png}
\item \includegraphics[width = 0.3\textwidth]{../Figures/rationalEquationToGraphCopyBC.png}
\item \includegraphics[width = 0.3\textwidth]{../Figures/rationalEquationToGraphCopyCC.png}
\item \includegraphics[width = 0.3\textwidth]{../Figures/rationalEquationToGraphCopyDC.png}
\end{multicols}\item None of the above.\end{enumerate}
\textbf{General Comment:} Remember that the general form of a basic rational equation is $ f(x) = \frac{a}{(x-h)^n} + k$, where $a$ is the leading coefficient (and in this case, we assume is either $1$ or $-1$), $n$ is the degree (in this case, either $1$ or $2$), and $(h, k)$ is the intersection of the asymptotes.
}
\litem{
Choose the equation of the function graphed below.

\begin{center}
    \includegraphics[width=0.5\textwidth]{../Figures/rationalGraphToEquationC.png}
\end{center}


The solution is \( \text{None of the above as it should be } f(x) = \frac{-1}{x - 3} + 2 \), which is option E.\begin{enumerate}[label=\Alph*.]
\item \( f(x) = \frac{-1}{(x + 3)^2} + 2 \)

Corresponds to thinking the graph was a shifted version of $\frac{1}{x^2}$.
\item \( f(x) = \frac{1}{(x - 3)^2} + 2 \)

Corresponds to thinking the graph was a shifted version of $\frac{1}{x^2}$, using the general form $f(x) = \frac{a}{x-h}+k$, and the opposite leading coefficient.
\item \( f(x) = \frac{-1}{x + 3} + 2 \)

The $x$-value of the equation does not match the graph.
\item \( f(x) = \frac{1}{x - 3} + 2 \)

Corresponds to using the general form $f(x) = \frac{a}{x-h}+k$ and the opposite leading coefficient.
\item \( \text{None of the above} \)

None of the equation options were the correct equation.
\end{enumerate}

\textbf{General Comment:} Remember that the general form of a basic rational equation is $ f(x) = \frac{a}{(x-h)^n} + k$, where $a$ is the leading coefficient (and in this case, we assume is either $1$ or $-1$), $n$ is the degree (in this case, either $1$ or $2$), and $(h, k)$ is the intersection of the asymptotes.
}
\litem{
Solve the rational equation below. Then, choose the interval(s) that the solution(s) belongs to.
\[ \frac{7x + 0}{-7x -5} + \frac{-7x^{2} +0 x + 0}{-14x^{2} -45 x -25} = \frac{6}{2x + 5} \]The solution is \( \text{There are two solutions: } x = -10.596 \text{ and } x = -0.404 \), which is option B.\begin{enumerate}[label=\Alph*.]
\item \( \text{All solutions lead to invalid or complex values in the equation.} \)


\item \( x_1 \in [-11.4, -10] \text{ and } x_2 \in [-0.58,-0.22] \)

* $x = -10.596 \text{ and } x = -0.404$, which is the correct option.
\item \( x_1 \in [-11.4, -10] \text{ and } x_2 \in [-0.98,-0.71] \)


\item \( x \in [-5.5,-1.1] \)


\item \( x \in [-0.9,0.2] \)


\end{enumerate}

\textbf{General Comment:} Distractors are different based on the number of solutions. Remember that after solving, we need to make sure our solution does not make the original equation divide by zero!
}
\end{enumerate}

\end{document}