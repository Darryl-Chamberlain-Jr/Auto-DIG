\documentclass{extbook}[14pt]
\usepackage{multicol, enumerate, enumitem, hyperref, color, soul, setspace, parskip, fancyhdr, amssymb, amsthm, amsmath, latexsym, units, mathtools}
\everymath{\displaystyle}
\usepackage[headsep=0.5cm,headheight=0cm, left=1 in,right= 1 in,top= 1 in,bottom= 1 in]{geometry}
\usepackage{dashrule}  % Package to use the command below to create lines between items
\newcommand{\litem}[1]{\item #1

\rule{\textwidth}{0.4pt}}
\pagestyle{fancy}
\lhead{}
\chead{Answer Key for Progress Quiz 6 Version C}
\rhead{}
\lfoot{4563-7456}
\cfoot{}
\rfoot{Summer C 2021}
\begin{document}
\textbf{This key should allow you to understand why you choose the option you did (beyond just getting a question right or wrong). \href{https://xronos.clas.ufl.edu/mac1105spring2020/courseDescriptionAndMisc/Exams/LearningFromResults}{More instructions on how to use this key can be found here}.}

\textbf{If you have a suggestion to make the keys better, \href{https://forms.gle/CZkbZmPbC9XALEE88}{please fill out the short survey here}.}

\textit{Note: This key is auto-generated and may contain issues and/or errors. The keys are reviewed after each exam to ensure grading is done accurately. If there are issues (like duplicate options), they are noted in the offline gradebook. The keys are a work-in-progress to give students as many resources to improve as possible.}

\rule{\textwidth}{0.4pt}

\begin{enumerate}\litem{
Factor the polynomial below completely. Then, choose the intervals the zeros of the polynomial belong to, where $z_1 \leq z_2 \leq z_3$. \textit{To make the problem easier, all zeros are between -5 and 5.}
\[ f(x) = 8x^{3} -6 x^{2} -45 x -27 \]The solution is \( [-1.5, -0.75, 3] \), which is option C.\begin{enumerate}[label=\Alph*.]
\item \( z_1 \in [-3.3, -1.7], \text{   }  z_2 \in [0.68, 0.83], \text{   and   } z_3 \in [1.44, 1.51] \)

 Distractor 1: Corresponds to negatives of all zeros.
\item \( z_1 \in [-3.3, -1.7], \text{   }  z_2 \in [0.64, 0.69], \text{   and   } z_3 \in [1.17, 1.48] \)

 Distractor 3: Corresponds to negatives of all zeros AND inversing rational roots.
\item \( z_1 \in [-2.4, -1.4], \text{   }  z_2 \in [-0.77, -0.75], \text{   and   } z_3 \in [2.78, 3.13] \)

* This is the solution!
\item \( z_1 \in [-3.3, -1.7], \text{   }  z_2 \in [0.3, 0.41], \text{   and   } z_3 \in [2.78, 3.13] \)

 Distractor 4: Corresponds to moving factors from one rational to another.
\item \( z_1 \in [-1.4, -1.1], \text{   }  z_2 \in [-0.7, -0.63], \text{   and   } z_3 \in [2.78, 3.13] \)

 Distractor 2: Corresponds to inversing rational roots.
\end{enumerate}

\textbf{General Comment:} Remember to try the middle-most integers first as these normally are the zeros. Also, once you get it to a quadratic, you can use your other factoring techniques to finish factoring.
}
\litem{
Perform the division below. Then, find the intervals that correspond to the quotient in the form $ax^2+bx+c$ and remainder $r$.
\[ \frac{8x^{3} -8 x^{2} -40 x -29}{x -3} \]The solution is \( 8x^{2} +16 x + 8 + \frac{-5}{x -3} \), which is option A.\begin{enumerate}[label=\Alph*.]
\item \( a \in [6, 12], \text{   } b \in [14, 19], \text{   } c \in [6, 9], \text{   and   } r \in [-5, 2]. \)

* This is the solution!
\item \( a \in [6, 12], \text{   } b \in [3, 10], \text{   } c \in [-27, -22], \text{   and   } r \in [-80, -72]. \)

 You multiplied by the synthetic number and subtracted rather than adding during synthetic division.
\item \( a \in [6, 12], \text{   } b \in [-32, -31], \text{   } c \in [54, 57], \text{   and   } r \in [-197, -193]. \)

 You divided by the opposite of the factor.
\item \( a \in [24, 32], \text{   } b \in [63, 69], \text{   } c \in [152, 155], \text{   and   } r \in [427, 428]. \)

 You multiplied by the synthetic number rather than bringing the first factor down.
\item \( a \in [24, 32], \text{   } b \in [-83, -75], \text{   } c \in [199, 207], \text{   and   } r \in [-634, -628]. \)

 You divided by the opposite of the factor AND multiplied the first factor rather than just bringing it down.
\end{enumerate}

\textbf{General Comment:} Be sure to synthetically divide by the zero of the denominator!
}
\litem{
Perform the division below. Then, find the intervals that correspond to the quotient in the form $ax^2+bx+c$ and remainder $r$.
\[ \frac{6x^{3} +26 x^{2} -28}{x + 4} \]The solution is \( 6x^{2} +2 x -8 + \frac{4}{x + 4} \), which is option C.\begin{enumerate}[label=\Alph*.]
\item \( a \in [1, 9], b \in [48, 55], c \in [200, 202], \text{ and } r \in [771, 774]. \)

 You divided by the opposite of the factor.
\item \( a \in [1, 9], b \in [-4, 0], c \in [19, 28], \text{ and } r \in [-130, -124]. \)

 You multipled by the synthetic number and subtracted rather than adding during synthetic division.
\item \( a \in [1, 9], b \in [1, 5], c \in [-12, -4], \text{ and } r \in [-1, 10]. \)

* This is the solution!
\item \( a \in [-24, -22], b \in [-73, -66], c \in [-283, -279], \text{ and } r \in [-1154, -1140]. \)

 You divided by the opposite of the factor AND multipled the first factor rather than just bringing it down.
\item \( a \in [-24, -22], b \in [121, 123], c \in [-491, -483], \text{ and } r \in [1921, 1929]. \)

 You multipled by the synthetic number rather than bringing the first factor down.
\end{enumerate}

\textbf{General Comment:} Be sure to synthetically divide by the zero of the denominator! Also, make sure to include 0 placeholders for missing terms.
}
\litem{
Factor the polynomial below completely. Then, choose the intervals the zeros of the polynomial belong to, where $z_1 \leq z_2 \leq z_3$. \textit{To make the problem easier, all zeros are between -5 and 5.}
\[ f(x) = 16x^{3} -40 x^{2} +x + 30 \]The solution is \( [-0.75, 1.25, 2] \), which is option D.\begin{enumerate}[label=\Alph*.]
\item \( z_1 \in [-2.25, -1.94], \text{   }  z_2 \in [-1.04, -0.34], \text{   and   } z_3 \in [0.79, 1.94] \)

 Distractor 3: Corresponds to negatives of all zeros AND inversing rational roots.
\item \( z_1 \in [-1.34, -0.77], \text{   }  z_2 \in [0.55, 0.88], \text{   and   } z_3 \in [1.9, 2.19] \)

 Distractor 2: Corresponds to inversing rational roots.
\item \( z_1 \in [-2.25, -1.94], \text{   }  z_2 \in [-1.46, -1.14], \text{   and   } z_3 \in [0.51, 0.89] \)

 Distractor 1: Corresponds to negatives of all zeros.
\item \( z_1 \in [-1.28, -0.5], \text{   }  z_2 \in [1.07, 1.34], \text{   and   } z_3 \in [1.9, 2.19] \)

* This is the solution!
\item \( z_1 \in [-5.28, -4.63], \text{   }  z_2 \in [-2.05, -1.9], \text{   and   } z_3 \in [-0.1, 0.58] \)

 Distractor 4: Corresponds to moving factors from one rational to another.
\end{enumerate}

\textbf{General Comment:} Remember to try the middle-most integers first as these normally are the zeros. Also, once you get it to a quadratic, you can use your other factoring techniques to finish factoring.
}
\litem{
Factor the polynomial below completely, knowing that $x -4$ is a factor. Then, choose the intervals the zeros of the polynomial belong to, where $z_1 \leq z_2 \leq z_3 \leq z_4$. \textit{To make the problem easier, all zeros are between -5 and 5.}
\[ f(x) = 15x^{4} -59 x^{3} -50 x^{2} +208 x -96 \]The solution is \( [-2, 0.6, 1.333, 4] \), which is option E.\begin{enumerate}[label=\Alph*.]
\item \( z_1 \in [-4, -3], \text{   }  z_2 \in [-1.68, -1.66], z_3 \in [-0.81, -0.63], \text{   and   } z_4 \in [1.7, 2.7] \)

 Distractor 3: Corresponds to negatives of all zeros AND inversing rational roots.
\item \( z_1 \in [-2, 1], \text{   }  z_2 \in [0.62, 0.9], z_3 \in [1.57, 1.69], \text{   and   } z_4 \in [3.9, 5.2] \)

 Distractor 2: Corresponds to inversing rational roots.
\item \( z_1 \in [-4, -3], \text{   }  z_2 \in [-1.44, -1.27], z_3 \in [-0.67, -0.6], \text{   and   } z_4 \in [1.7, 2.7] \)

 Distractor 1: Corresponds to negatives of all zeros.
\item \( z_1 \in [-4, -3], \text{   }  z_2 \in [-3.02, -2.95], z_3 \in [-0.44, -0.14], \text{   and   } z_4 \in [1.7, 2.7] \)

 Distractor 4: Corresponds to moving factors from one rational to another.
\item \( z_1 \in [-2, 1], \text{   }  z_2 \in [0.55, 0.65], z_3 \in [1.28, 1.46], \text{   and   } z_4 \in [3.9, 5.2] \)

* This is the solution!
\end{enumerate}

\textbf{General Comment:} Remember to try the middle-most integers first as these normally are the zeros. Also, once you get it to a quadratic, you can use your other factoring techniques to finish factoring.
}
\litem{
Perform the division below. Then, find the intervals that correspond to the quotient in the form $ax^2+bx+c$ and remainder $r$.
\[ \frac{12x^{3} -4 x^{2} -40 x + 37}{x + 2} \]The solution is \( 12x^{2} -28 x + 16 + \frac{5}{x + 2} \), which is option C.\begin{enumerate}[label=\Alph*.]
\item \( a \in [-32, -22], \text{   } b \in [-55, -47], \text{   } c \in [-151, -142], \text{   and   } r \in [-251, -245]. \)

 You divided by the opposite of the factor AND multiplied the first factor rather than just bringing it down.
\item \( a \in [-32, -22], \text{   } b \in [36, 49], \text{   } c \in [-129, -124], \text{   and   } r \in [292, 296]. \)

 You multiplied by the synthetic number rather than bringing the first factor down.
\item \( a \in [12, 13], \text{   } b \in [-32, -24], \text{   } c \in [11, 19], \text{   and   } r \in [4, 9]. \)

* This is the solution!
\item \( a \in [12, 13], \text{   } b \in [-43, -38], \text{   } c \in [79, 82], \text{   and   } r \in [-205, -196]. \)

 You multiplied by the synthetic number and subtracted rather than adding during synthetic division.
\item \( a \in [12, 13], \text{   } b \in [18, 26], \text{   } c \in [0, 1], \text{   and   } r \in [29, 47]. \)

 You divided by the opposite of the factor.
\end{enumerate}

\textbf{General Comment:} Be sure to synthetically divide by the zero of the denominator!
}
\litem{
Perform the division below. Then, find the intervals that correspond to the quotient in the form $ax^2+bx+c$ and remainder $r$.
\[ \frac{16x^{3} -48 x -28}{x -2} \]The solution is \( 16x^{2} +32 x + 16 + \frac{4}{x -2} \), which is option C.\begin{enumerate}[label=\Alph*.]
\item \( a \in [32, 33], b \in [-64, -59], c \in [80, 83], \text{ and } r \in [-195, -187]. \)

 You divided by the opposite of the factor AND multipled the first factor rather than just bringing it down.
\item \( a \in [12, 23], b \in [-35, -27], c \in [9, 21], \text{ and } r \in [-60, -53]. \)

 You divided by the opposite of the factor.
\item \( a \in [12, 23], b \in [26, 33], c \in [9, 21], \text{ and } r \in [3, 5]. \)

* This is the solution!
\item \( a \in [32, 33], b \in [61, 67], c \in [80, 83], \text{ and } r \in [130, 141]. \)

 You multipled by the synthetic number rather than bringing the first factor down.
\item \( a \in [12, 23], b \in [15, 17], c \in [-39, -28], \text{ and } r \in [-60, -53]. \)

 You multipled by the synthetic number and subtracted rather than adding during synthetic division.
\end{enumerate}

\textbf{General Comment:} Be sure to synthetically divide by the zero of the denominator! Also, make sure to include 0 placeholders for missing terms.
}
\litem{
What are the \textit{possible Integer} roots of the polynomial below?
\[ f(x) = 2x^{2} +3 x + 7 \]The solution is \( \pm 1,\pm 7 \), which is option C.\begin{enumerate}[label=\Alph*.]
\item \( \text{ All combinations of: }\frac{\pm 1,\pm 2}{\pm 1,\pm 7} \)

 Distractor 3: Corresponds to the plus or minus of the inverse quotient (an/a0) of the factors. 
\item \( \text{ All combinations of: }\frac{\pm 1,\pm 7}{\pm 1,\pm 2} \)

This would have been the solution \textbf{if asked for the possible Rational roots}!
\item \( \pm 1,\pm 7 \)

* This is the solution \textbf{since we asked for the possible Integer roots}!
\item \( \pm 1,\pm 2 \)

 Distractor 1: Corresponds to the plus or minus factors of a1 only.
\item \( \text{There is no formula or theorem that tells us all possible Integer roots.} \)

 Distractor 4: Corresponds to not recognizing Integers as a subset of Rationals.
\end{enumerate}

\textbf{General Comment:} We have a way to find the possible Rational roots. The possible Integer roots are the Integers in this list.
}
\litem{
Factor the polynomial below completely, knowing that $x -4$ is a factor. Then, choose the intervals the zeros of the polynomial belong to, where $z_1 \leq z_2 \leq z_3 \leq z_4$. \textit{To make the problem easier, all zeros are between -5 and 5.}
\[ f(x) = 25x^{4} -80 x^{3} -132 x^{2} +224 x -64 \]The solution is \( [-2, 0.4, 0.8, 4] \), which is option A.\begin{enumerate}[label=\Alph*.]
\item \( z_1 \in [-2, 2], \text{   }  z_2 \in [-0.17, 0.41], z_3 \in [0.76, 0.9], \text{   and   } z_4 \in [3, 6] \)

* This is the solution!
\item \( z_1 \in [-5, -3], \text{   }  z_2 \in [-1.71, -0.25], z_3 \in [-0.63, -0.31], \text{   and   } z_4 \in [1, 3] \)

 Distractor 1: Corresponds to negatives of all zeros.
\item \( z_1 \in [-5, -3], \text{   }  z_2 \in [-2.13, -1.06], z_3 \in [-0.19, 0.06], \text{   and   } z_4 \in [1, 3] \)

 Distractor 4: Corresponds to moving factors from one rational to another.
\item \( z_1 \in [-2, 2], \text{   }  z_2 \in [1.21, 2.56], z_3 \in [2.4, 2.52], \text{   and   } z_4 \in [3, 6] \)

 Distractor 2: Corresponds to inversing rational roots.
\item \( z_1 \in [-5, -3], \text{   }  z_2 \in [-2.8, -2.06], z_3 \in [-1.34, -1.19], \text{   and   } z_4 \in [1, 3] \)

 Distractor 3: Corresponds to negatives of all zeros AND inversing rational roots.
\end{enumerate}

\textbf{General Comment:} Remember to try the middle-most integers first as these normally are the zeros. Also, once you get it to a quadratic, you can use your other factoring techniques to finish factoring.
}
\litem{
What are the \textit{possible Integer} roots of the polynomial below?
\[ f(x) = 6x^{3} +4 x^{2} +4 x + 7 \]The solution is \( \pm 1,\pm 7 \), which is option C.\begin{enumerate}[label=\Alph*.]
\item \( \pm 1,\pm 2,\pm 3,\pm 6 \)

 Distractor 1: Corresponds to the plus or minus factors of a1 only.
\item \( \text{ All combinations of: }\frac{\pm 1,\pm 2,\pm 3,\pm 6}{\pm 1,\pm 7} \)

 Distractor 3: Corresponds to the plus or minus of the inverse quotient (an/a0) of the factors. 
\item \( \pm 1,\pm 7 \)

* This is the solution \textbf{since we asked for the possible Integer roots}!
\item \( \text{ All combinations of: }\frac{\pm 1,\pm 7}{\pm 1,\pm 2,\pm 3,\pm 6} \)

This would have been the solution \textbf{if asked for the possible Rational roots}!
\item \( \text{There is no formula or theorem that tells us all possible Integer roots.} \)

 Distractor 4: Corresponds to not recognizing Integers as a subset of Rationals.
\end{enumerate}

\textbf{General Comment:} We have a way to find the possible Rational roots. The possible Integer roots are the Integers in this list.
}
\end{enumerate}

\end{document}