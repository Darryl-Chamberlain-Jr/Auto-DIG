\documentclass[14pt]{extbook}
\usepackage{multicol, enumerate, enumitem, hyperref, color, soul, setspace, parskip, fancyhdr} %General Packages
\usepackage{amssymb, amsthm, amsmath, latexsym, units, mathtools} %Math Packages
\everymath{\displaystyle} %All math in Display Style
% Packages with additional options
\usepackage[headsep=0.5cm,headheight=12pt, left=1 in,right= 1 in,top= 1 in,bottom= 1 in]{geometry}
\usepackage[usenames,dvipsnames]{xcolor}
\usepackage{dashrule}  % Package to use the command below to create lines between items
\newcommand{\litem}[1]{\item#1\hspace*{-1cm}\rule{\textwidth}{0.4pt}}
\pagestyle{fancy}
\lhead{Progress Quiz 6}
\chead{}
\rhead{Version A}
\lfoot{4563-7456}
\cfoot{}
\rfoot{Summer C 2021}
\begin{document}

\begin{enumerate}
\litem{
Choose the interval below that $f$ composed with $g$ at $x=1$ is in.\[ f(x) = -2x^{3} +2 x^{2} -3 x \text{ and } g(x) = 2x^{3} +2 x^{2} -2 x \]\begin{enumerate}[label=\Alph*.]
\item \( (f \circ g)(1) \in [-36, -34] \)
\item \( (f \circ g)(1) \in [-19, -13] \)
\item \( (f \circ g)(1) \in [-10, 3] \)
\item \( (f \circ g)(1) \in [-31, -29] \)
\item \( \text{It is not possible to compose the two functions.} \)

\end{enumerate} }
\litem{
Choose the interval below that $f$ composed with $g$ at $x=-1$ is in.\[ f(x) = x^{3} -4 x^{2} -4 x -2 \text{ and } g(x) = -4x^{3} -1 x^{2} +2 x + 2 \]\begin{enumerate}[label=\Alph*.]
\item \( (f \circ g)(-1) \in [-24, -18] \)
\item \( (f \circ g)(-1) \in [89, 99] \)
\item \( (f \circ g)(-1) \in [97, 107] \)
\item \( (f \circ g)(-1) \in [-31, -26] \)
\item \( \text{It is not possible to compose the two functions.} \)

\end{enumerate} }
\litem{
Determine whether the function below is 1-1.\[ f(x) = -16 x^2 - 24 x + 247 \]\begin{enumerate}[label=\Alph*.]
\item \( \text{No, because the domain of the function is not $(-\infty, \infty)$.} \)
\item \( \text{No, because there is a $y$-value that goes to 2 different $x$-values.} \)
\item \( \text{No, because there is an $x$-value that goes to 2 different $y$-values.} \)
\item \( \text{Yes, the function is 1-1.} \)
\item \( \text{No, because the range of the function is not $(-\infty, \infty)$.} \)

\end{enumerate} }
\litem{
Find the inverse of the function below. Then, evaluate the inverse at $x = 10$ and choose the interval that $f^-1(10)$ belongs to.\[ f(x) = e^{x+2}-5 \]\begin{enumerate}[label=\Alph*.]
\item \( f^{-1}(10) \in [-3.55, -3.34] \)
\item \( f^{-1}(10) \in [-3.16, -2.67] \)
\item \( f^{-1}(10) \in [4.47, 4.86] \)
\item \( f^{-1}(10) \in [-2.54, -2.41] \)
\item \( f^{-1}(10) \in [0.55, 0.96] \)

\end{enumerate} }
\litem{
Find the inverse of the function below (if it exists). Then, evaluate the inverse at $x = 10$ and choose the interval that $f^-1(10)$ belongs to.\[ f(x) = 2 x^2 - 4 \]\begin{enumerate}[label=\Alph*.]
\item \( f^{-1}(10) \in [1.77, 2.8] \)
\item \( f^{-1}(10) \in [2.88, 4.02] \)
\item \( f^{-1}(10) \in [6.38, 7.96] \)
\item \( f^{-1}(10) \in [0.91, 2.03] \)
\item \( \text{ The function is not invertible for all Real numbers. } \)

\end{enumerate} }
\litem{
Find the inverse of the function below. Then, evaluate the inverse at $x = 7$ and choose the interval that $f^-1(7)$ belongs to.\[ f(x) = \ln{(x+5)}+3 \]\begin{enumerate}[label=\Alph*.]
\item \( f^{-1}(7) \in [162755.79, 162763.79] \)
\item \( f^{-1}(7) \in [58.6, 61.6] \)
\item \( f^{-1}(7) \in [22020.47, 22024.47] \)
\item \( f^{-1}(7) \in [47.6, 54.6] \)
\item \( f^{-1}(7) \in [7.39, 11.39] \)

\end{enumerate} }
\litem{
Find the inverse of the function below (if it exists). Then, evaluate the inverse at $x = 14$ and choose the interval that $f^-1(14)$ belongs to.\[ f(x) = \sqrt[3]{3 x - 4} \]\begin{enumerate}[label=\Alph*.]
\item \( f^{-1}(14) \in [-916.4, -914.3] \)
\item \( f^{-1}(14) \in [-913.6, -911.7] \)
\item \( f^{-1}(14) \in [914.9, 919.4] \)
\item \( f^{-1}(14) \in [911.6, 915.6] \)
\item \( \text{ The function is not invertible for all Real numbers. } \)

\end{enumerate} }
\litem{
Add the following functions, then choose the domain of the resulting function from the list below.\[ f(x) = \frac{1}{4x+25} \text{ and } g(x) = \frac{4}{6x-29} \]\begin{enumerate}[label=\Alph*.]
\item \( \text{ The domain is all Real numbers except } x = a, \text{ where } a \in [5.67, 14.67] \)
\item \( \text{ The domain is all Real numbers less than or equal to } x = a, \text{ where } a \in [0.33, 12.33] \)
\item \( \text{ The domain is all Real numbers greater than or equal to } x = a, \text{ where } a \in [-8.5, -4.5] \)
\item \( \text{ The domain is all Real numbers except } x = a \text{ and } x = b, \text{ where } a \in [-15.25, -2.25] \text{ and } b \in [2.83, 9.83] \)
\item \( \text{ The domain is all Real numbers. } \)

\end{enumerate} }
\litem{
Determine whether the function below is 1-1.\[ f(x) = (4 x - 18)^3 \]\begin{enumerate}[label=\Alph*.]
\item \( \text{No, because there is a $y$-value that goes to 2 different $x$-values.} \)
\item \( \text{Yes, the function is 1-1.} \)
\item \( \text{No, because there is an $x$-value that goes to 2 different $y$-values.} \)
\item \( \text{No, because the range of the function is not $(-\infty, \infty)$.} \)
\item \( \text{No, because the domain of the function is not $(-\infty, \infty)$.} \)

\end{enumerate} }
\litem{
Add the following functions, then choose the domain of the resulting function from the list below.\[ f(x) = 6x + 4 \text{ and } g(x) = \frac{1}{4x-21} \]\begin{enumerate}[label=\Alph*.]
\item \( \text{ The domain is all Real numbers less than or equal to } x = a, \text{ where } a \in [-1.5, 4.5] \)
\item \( \text{ The domain is all Real numbers greater than or equal to } x = a, \text{ where } a \in [-6.67, -0.67] \)
\item \( \text{ The domain is all Real numbers except } x = a, \text{ where } a \in [4.25, 8.25] \)
\item \( \text{ The domain is all Real numbers except } x = a \text{ and } x = b, \text{ where } a \in [2.83, 7.83] \text{ and } b \in [-7.33, 1.67] \)
\item \( \text{ The domain is all Real numbers. } \)

\end{enumerate} }
\end{enumerate}

\end{document}