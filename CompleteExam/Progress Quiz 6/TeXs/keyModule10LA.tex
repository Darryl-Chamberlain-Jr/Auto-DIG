\documentclass{extbook}[14pt]
\usepackage{multicol, enumerate, enumitem, hyperref, color, soul, setspace, parskip, fancyhdr, amssymb, amsthm, amsmath, latexsym, units, mathtools}
\everymath{\displaystyle}
\usepackage[headsep=0.5cm,headheight=0cm, left=1 in,right= 1 in,top= 1 in,bottom= 1 in]{geometry}
\usepackage{dashrule}  % Package to use the command below to create lines between items
\newcommand{\litem}[1]{\item #1

\rule{\textwidth}{0.4pt}}
\pagestyle{fancy}
\lhead{}
\chead{Answer Key for Progress Quiz 6 Version A}
\rhead{}
\lfoot{4563-7456}
\cfoot{}
\rfoot{Summer C 2021}
\begin{document}
\textbf{This key should allow you to understand why you choose the option you did (beyond just getting a question right or wrong). \href{https://xronos.clas.ufl.edu/mac1105spring2020/courseDescriptionAndMisc/Exams/LearningFromResults}{More instructions on how to use this key can be found here}.}

\textbf{If you have a suggestion to make the keys better, \href{https://forms.gle/CZkbZmPbC9XALEE88}{please fill out the short survey here}.}

\textit{Note: This key is auto-generated and may contain issues and/or errors. The keys are reviewed after each exam to ensure grading is done accurately. If there are issues (like duplicate options), they are noted in the offline gradebook. The keys are a work-in-progress to give students as many resources to improve as possible.}

\rule{\textwidth}{0.4pt}

\begin{enumerate}\litem{
Factor the polynomial below completely. Then, choose the intervals the zeros of the polynomial belong to, where $z_1 \leq z_2 \leq z_3$. \textit{To make the problem easier, all zeros are between -5 and 5.}
\[ f(x) = 25x^{3} +75 x^{2} -16 x -48 \]The solution is \( [-3, -0.8, 0.8] \), which is option C.\begin{enumerate}[label=\Alph*.]
\item \( z_1 \in [-3.16, -2.71], \text{   }  z_2 \in [-1.32, -1.21], \text{   and   } z_3 \in [1.09, 1.63] \)

 Distractor 2: Corresponds to inversing rational roots.
\item \( z_1 \in [-1.28, -1.18], \text{   }  z_2 \in [1.09, 1.35], \text{   and   } z_3 \in [2.81, 3.32] \)

 Distractor 3: Corresponds to negatives of all zeros AND inversing rational roots.
\item \( z_1 \in [-3.16, -2.71], \text{   }  z_2 \in [-0.86, -0.49], \text{   and   } z_3 \in [0.52, 0.91] \)

* This is the solution!
\item \( z_1 \in [-4.08, -3.92], \text{   }  z_2 \in [0.08, 0.21], \text{   and   } z_3 \in [2.81, 3.32] \)

 Distractor 4: Corresponds to moving factors from one rational to another.
\item \( z_1 \in [-0.9, -0.7], \text{   }  z_2 \in [0.52, 0.96], \text{   and   } z_3 \in [2.81, 3.32] \)

 Distractor 1: Corresponds to negatives of all zeros.
\end{enumerate}

\textbf{General Comment:} Remember to try the middle-most integers first as these normally are the zeros. Also, once you get it to a quadratic, you can use your other factoring techniques to finish factoring.
}
\litem{
Perform the division below. Then, find the intervals that correspond to the quotient in the form $ax^2+bx+c$ and remainder $r$.
\[ \frac{20x^{3} -106 x^{2} +112 x -30}{x -4} \]The solution is \( 20x^{2} -26 x + 8 + \frac{2}{x -4} \), which is option D.\begin{enumerate}[label=\Alph*.]
\item \( a \in [79, 82], \text{   } b \in [-426, -424], \text{   } c \in [1811, 1818], \text{   and   } r \in [-7295, -7290]. \)

 You divided by the opposite of the factor AND multiplied the first factor rather than just bringing it down.
\item \( a \in [79, 82], \text{   } b \in [212, 216], \text{   } c \in [965, 973], \text{   and   } r \in [3836, 3844]. \)

 You multiplied by the synthetic number rather than bringing the first factor down.
\item \( a \in [17, 26], \text{   } b \in [-47, -44], \text{   } c \in [-27, -22], \text{   and   } r \in [-109, -104]. \)

 You multiplied by the synthetic number and subtracted rather than adding during synthetic division.
\item \( a \in [17, 26], \text{   } b \in [-28, -23], \text{   } c \in [4, 11], \text{   and   } r \in [-1, 5]. \)

* This is the solution!
\item \( a \in [17, 26], \text{   } b \in [-192, -184], \text{   } c \in [855, 861], \text{   and   } r \in [-3457, -3450]. \)

 You divided by the opposite of the factor.
\end{enumerate}

\textbf{General Comment:} Be sure to synthetically divide by the zero of the denominator!
}
\litem{
Perform the division below. Then, find the intervals that correspond to the quotient in the form $ax^2+bx+c$ and remainder $r$.
\[ \frac{6x^{3} +28 x^{2} -68}{x + 4} \]The solution is \( 6x^{2} +4 x -16 + \frac{-4}{x + 4} \), which is option B.\begin{enumerate}[label=\Alph*.]
\item \( a \in [-27, -23], b \in [123, 125], c \in [-498, -495], \text{ and } r \in [1913, 1919]. \)

 You multipled by the synthetic number rather than bringing the first factor down.
\item \( a \in [3, 9], b \in [4, 9], c \in [-19, -11], \text{ and } r \in [-5, -3]. \)

* This is the solution!
\item \( a \in [-27, -23], b \in [-68, -63], c \in [-277, -267], \text{ and } r \in [-1157, -1153]. \)

 You divided by the opposite of the factor AND multipled the first factor rather than just bringing it down.
\item \( a \in [3, 9], b \in [51, 53], c \in [208, 211], \text{ and } r \in [762, 767]. \)

 You divided by the opposite of the factor.
\item \( a \in [3, 9], b \in [-6, 1], c \in [4, 15], \text{ and } r \in [-125, -117]. \)

 You multipled by the synthetic number and subtracted rather than adding during synthetic division.
\end{enumerate}

\textbf{General Comment:} Be sure to synthetically divide by the zero of the denominator! Also, make sure to include 0 placeholders for missing terms.
}
\litem{
Factor the polynomial below completely. Then, choose the intervals the zeros of the polynomial belong to, where $z_1 \leq z_2 \leq z_3$. \textit{To make the problem easier, all zeros are between -5 and 5.}
\[ f(x) = 10x^{3} -21 x^{2} -135 x -50 \]The solution is \( [-2.5, -0.4, 5] \), which is option B.\begin{enumerate}[label=\Alph*.]
\item \( z_1 \in [-4.5, -1.5], \text{   }  z_2 \in [-0.52, -0.38], \text{   and   } z_3 \in [5, 7] \)

 Distractor 2: Corresponds to inversing rational roots.
\item \( z_1 \in [-4.5, -1.5], \text{   }  z_2 \in [-0.52, -0.38], \text{   and   } z_3 \in [5, 7] \)

* This is the solution!
\item \( z_1 \in [-6, -4], \text{   }  z_2 \in [0.36, 0.46], \text{   and   } z_3 \in [1.5, 4.5] \)

 Distractor 1: Corresponds to negatives of all zeros.
\item \( z_1 \in [-6, -4], \text{   }  z_2 \in [0.01, 0.37], \text{   and   } z_3 \in [5, 7] \)

 Distractor 4: Corresponds to moving factors from one rational to another.
\item \( z_1 \in [-6, -4], \text{   }  z_2 \in [0.36, 0.46], \text{   and   } z_3 \in [1.5, 4.5] \)

 Distractor 3: Corresponds to negatives of all zeros AND inversing rational roots.
\end{enumerate}

\textbf{General Comment:} Remember to try the middle-most integers first as these normally are the zeros. Also, once you get it to a quadratic, you can use your other factoring techniques to finish factoring.
}
\litem{
Factor the polynomial below completely, knowing that $x -3$ is a factor. Then, choose the intervals the zeros of the polynomial belong to, where $z_1 \leq z_2 \leq z_3 \leq z_4$. \textit{To make the problem easier, all zeros are between -5 and 5.}
\[ f(x) = 9x^{4} +9 x^{3} -163 x^{2} +115 x + 150 \]The solution is \( [-5, -0.667, 1.667, 3] \), which is option D.\begin{enumerate}[label=\Alph*.]
\item \( z_1 \in [-5.2, -4.7], \text{   }  z_2 \in [-1.62, -1.48], z_3 \in [0.52, 0.63], \text{   and   } z_4 \in [2.4, 3.2] \)

 Distractor 2: Corresponds to inversing rational roots.
\item \( z_1 \in [-5.2, -4.7], \text{   }  z_2 \in [-3.05, -2.99], z_3 \in [0.12, 0.28], \text{   and   } z_4 \in [4, 5.3] \)

 Distractor 4: Corresponds to moving factors from one rational to another.
\item \( z_1 \in [-3.7, -2], \text{   }  z_2 \in [-0.65, -0.6], z_3 \in [1.47, 1.5], \text{   and   } z_4 \in [4, 5.3] \)

 Distractor 3: Corresponds to negatives of all zeros AND inversing rational roots.
\item \( z_1 \in [-5.2, -4.7], \text{   }  z_2 \in [-0.74, -0.64], z_3 \in [1.64, 1.74], \text{   and   } z_4 \in [2.4, 3.2] \)

* This is the solution!
\item \( z_1 \in [-3.7, -2], \text{   }  z_2 \in [-1.71, -1.63], z_3 \in [0.66, 0.71], \text{   and   } z_4 \in [4, 5.3] \)

 Distractor 1: Corresponds to negatives of all zeros.
\end{enumerate}

\textbf{General Comment:} Remember to try the middle-most integers first as these normally are the zeros. Also, once you get it to a quadratic, you can use your other factoring techniques to finish factoring.
}
\litem{
Perform the division below. Then, find the intervals that correspond to the quotient in the form $ax^2+bx+c$ and remainder $r$.
\[ \frac{10x^{3} +26 x^{2} -68 x -53}{x + 4} \]The solution is \( 10x^{2} -14 x -12 + \frac{-5}{x + 4} \), which is option D.\begin{enumerate}[label=\Alph*.]
\item \( a \in [-42, -33], \text{   } b \in [-134, -130], \text{   } c \in [-607, -603], \text{   and   } r \in [-2473, -2463]. \)

 You divided by the opposite of the factor AND multiplied the first factor rather than just bringing it down.
\item \( a \in [-42, -33], \text{   } b \in [185, 188], \text{   } c \in [-818, -809], \text{   and   } r \in [3195, 3197]. \)

 You multiplied by the synthetic number rather than bringing the first factor down.
\item \( a \in [9, 11], \text{   } b \in [-30, -21], \text{   } c \in [52, 57], \text{   and   } r \in [-321, -310]. \)

 You multiplied by the synthetic number and subtracted rather than adding during synthetic division.
\item \( a \in [9, 11], \text{   } b \in [-16, -13], \text{   } c \in [-14, -10], \text{   and   } r \in [-8, -1]. \)

* This is the solution!
\item \( a \in [9, 11], \text{   } b \in [66, 70], \text{   } c \in [196, 202], \text{   and   } r \in [722, 739]. \)

 You divided by the opposite of the factor.
\end{enumerate}

\textbf{General Comment:} Be sure to synthetically divide by the zero of the denominator!
}
\litem{
Perform the division below. Then, find the intervals that correspond to the quotient in the form $ax^2+bx+c$ and remainder $r$.
\[ \frac{6x^{3} +28 x^{2} -62}{x + 4} \]The solution is \( 6x^{2} +4 x -16 + \frac{2}{x + 4} \), which is option D.\begin{enumerate}[label=\Alph*.]
\item \( a \in [4, 10], b \in [51, 53], c \in [207, 212], \text{ and } r \in [762, 773]. \)

 You divided by the opposite of the factor.
\item \( a \in [4, 10], b \in [-6, 1], c \in [8, 13], \text{ and } r \in [-115, -105]. \)

 You multipled by the synthetic number and subtracted rather than adding during synthetic division.
\item \( a \in [-25, -21], b \in [120, 125], c \in [-503, -493], \text{ and } r \in [1917, 1926]. \)

 You multipled by the synthetic number rather than bringing the first factor down.
\item \( a \in [4, 10], b \in [3, 8], c \in [-17, -14], \text{ and } r \in [2, 3]. \)

* This is the solution!
\item \( a \in [-25, -21], b \in [-72, -65], c \in [-277, -268], \text{ and } r \in [-1150, -1148]. \)

 You divided by the opposite of the factor AND multipled the first factor rather than just bringing it down.
\end{enumerate}

\textbf{General Comment:} Be sure to synthetically divide by the zero of the denominator! Also, make sure to include 0 placeholders for missing terms.
}
\litem{
What are the \textit{possible Integer} roots of the polynomial below?
\[ f(x) = 6x^{2} +2 x + 3 \]The solution is \( \pm 1,\pm 3 \), which is option A.\begin{enumerate}[label=\Alph*.]
\item \( \pm 1,\pm 3 \)

* This is the solution \textbf{since we asked for the possible Integer roots}!
\item \( \pm 1,\pm 2,\pm 3,\pm 6 \)

 Distractor 1: Corresponds to the plus or minus factors of a1 only.
\item \( \text{ All combinations of: }\frac{\pm 1,\pm 3}{\pm 1,\pm 2,\pm 3,\pm 6} \)

This would have been the solution \textbf{if asked for the possible Rational roots}!
\item \( \text{ All combinations of: }\frac{\pm 1,\pm 2,\pm 3,\pm 6}{\pm 1,\pm 3} \)

 Distractor 3: Corresponds to the plus or minus of the inverse quotient (an/a0) of the factors. 
\item \( \text{There is no formula or theorem that tells us all possible Integer roots.} \)

 Distractor 4: Corresponds to not recognizing Integers as a subset of Rationals.
\end{enumerate}

\textbf{General Comment:} We have a way to find the possible Rational roots. The possible Integer roots are the Integers in this list.
}
\litem{
Factor the polynomial below completely, knowing that $x -3$ is a factor. Then, choose the intervals the zeros of the polynomial belong to, where $z_1 \leq z_2 \leq z_3 \leq z_4$. \textit{To make the problem easier, all zeros are between -5 and 5.}
\[ f(x) = 8x^{4} -90 x^{3} +343 x^{2} -510 x + 225 \]The solution is \( [0.75, 2.5, 3, 5] \), which is option D.\begin{enumerate}[label=\Alph*.]
\item \( z_1 \in [-5.86, -4.88], \text{   }  z_2 \in [-3.65, -2.93], z_3 \in [-3.38, -2.77], \text{   and   } z_4 \in [-0.71, -0.43] \)

 Distractor 4: Corresponds to moving factors from one rational to another.
\item \( z_1 \in [-5.86, -4.88], \text{   }  z_2 \in [-3.65, -2.93], z_3 \in [-2.14, -0.63], \text{   and   } z_4 \in [-0.49, -0.24] \)

 Distractor 3: Corresponds to negatives of all zeros AND inversing rational roots.
\item \( z_1 \in [-5.86, -4.88], \text{   }  z_2 \in [-3.65, -2.93], z_3 \in [-2.84, -2.19], \text{   and   } z_4 \in [-0.83, -0.74] \)

 Distractor 1: Corresponds to negatives of all zeros.
\item \( z_1 \in [0.6, 0.85], \text{   }  z_2 \in [2.02, 3.75], z_3 \in [2.3, 3.15], \text{   and   } z_4 \in [4.99, 5.07] \)

* This is the solution!
\item \( z_1 \in [0.14, 0.74], \text{   }  z_2 \in [1.06, 1.46], z_3 \in [2.3, 3.15], \text{   and   } z_4 \in [4.99, 5.07] \)

 Distractor 2: Corresponds to inversing rational roots.
\end{enumerate}

\textbf{General Comment:} Remember to try the middle-most integers first as these normally are the zeros. Also, once you get it to a quadratic, you can use your other factoring techniques to finish factoring.
}
\litem{
What are the \textit{possible Rational} roots of the polynomial below?
\[ f(x) = 7x^{3} +5 x^{2} +2 x + 5 \]The solution is \( \text{ All combinations of: }\frac{\pm 1,\pm 5}{\pm 1,\pm 7} \), which is option D.\begin{enumerate}[label=\Alph*.]
\item \( \text{ All combinations of: }\frac{\pm 1,\pm 7}{\pm 1,\pm 5} \)

 Distractor 3: Corresponds to the plus or minus of the inverse quotient (an/a0) of the factors. 
\item \( \pm 1,\pm 7 \)

 Distractor 1: Corresponds to the plus or minus factors of a1 only.
\item \( \pm 1,\pm 5 \)

This would have been the solution \textbf{if asked for the possible Integer roots}!
\item \( \text{ All combinations of: }\frac{\pm 1,\pm 5}{\pm 1,\pm 7} \)

* This is the solution \textbf{since we asked for the possible Rational roots}!
\item \( \text{ There is no formula or theorem that tells us all possible Rational roots.} \)

 Distractor 4: Corresponds to not recalling the theorem for rational roots of a polynomial.
\end{enumerate}

\textbf{General Comment:} We have a way to find the possible Rational roots. The possible Integer roots are the Integers in this list.
}
\end{enumerate}

\end{document}