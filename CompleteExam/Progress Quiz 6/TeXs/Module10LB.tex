\documentclass[14pt]{extbook}
\usepackage{multicol, enumerate, enumitem, hyperref, color, soul, setspace, parskip, fancyhdr} %General Packages
\usepackage{amssymb, amsthm, amsmath, latexsym, units, mathtools} %Math Packages
\everymath{\displaystyle} %All math in Display Style
% Packages with additional options
\usepackage[headsep=0.5cm,headheight=12pt, left=1 in,right= 1 in,top= 1 in,bottom= 1 in]{geometry}
\usepackage[usenames,dvipsnames]{xcolor}
\usepackage{dashrule}  % Package to use the command below to create lines between items
\newcommand{\litem}[1]{\item#1\hspace*{-1cm}\rule{\textwidth}{0.4pt}}
\pagestyle{fancy}
\lhead{Progress Quiz 6}
\chead{}
\rhead{Version B}
\lfoot{4563-7456}
\cfoot{}
\rfoot{Summer C 2021}
\begin{document}

\begin{enumerate}
\litem{
Factor the polynomial below completely. Then, choose the intervals the zeros of the polynomial belong to, where $z_1 \leq z_2 \leq z_3$. \textit{To make the problem easier, all zeros are between -5 and 5.}\[ f(x) = 10x^{3} -41 x^{2} +27 x + 18 \]\begin{enumerate}[label=\Alph*.]
\item \( z_1 \in [-3.1, -2.9], \text{   }  z_2 \in [-1.7, -0.7], \text{   and   } z_3 \in [0.23, 0.73] \)
\item \( z_1 \in [-2.9, -1.5], \text{   }  z_2 \in [0.2, 0.8], \text{   and   } z_3 \in [2.77, 3.08] \)
\item \( z_1 \in [-3.1, -2.9], \text{   }  z_2 \in [-1.1, 0], \text{   and   } z_3 \in [2.35, 2.69] \)
\item \( z_1 \in [-3.1, -2.9], \text{   }  z_2 \in [-3.2, -2.7], \text{   and   } z_3 \in [0, 0.29] \)
\item \( z_1 \in [-1.9, 0], \text{   }  z_2 \in [0.8, 2], \text{   and   } z_3 \in [2.77, 3.08] \)

\end{enumerate} }
\litem{
Perform the division below. Then, find the intervals that correspond to the quotient in the form $ax^2+bx+c$ and remainder $r$.\[ \frac{10x^{3} +11 x^{2} -106 x + 44}{x + 4} \]\begin{enumerate}[label=\Alph*.]
\item \( a \in [9, 11], \text{   } b \in [-46, -31], \text{   } c \in [84, 95], \text{   and   } r \in [-401, -396]. \)
\item \( a \in [-44, -39], \text{   } b \in [-152, -148], \text{   } c \in [-703, -700], \text{   and   } r \in [-2765, -2760]. \)
\item \( a \in [9, 11], \text{   } b \in [46, 56], \text{   } c \in [98, 100], \text{   and   } r \in [425, 441]. \)
\item \( a \in [-44, -39], \text{   } b \in [168, 175], \text{   } c \in [-796, -787], \text{   and   } r \in [3196, 3209]. \)
\item \( a \in [9, 11], \text{   } b \in [-29, -25], \text{   } c \in [8, 15], \text{   and   } r \in [3, 7]. \)

\end{enumerate} }
\litem{
Perform the division below. Then, find the intervals that correspond to the quotient in the form $ax^2+bx+c$ and remainder $r$.\[ \frac{9x^{3} -28 x -19}{x -2} \]\begin{enumerate}[label=\Alph*.]
\item \( a \in [14, 26], b \in [-36, -34], c \in [40, 45], \text{ and } r \in [-108, -105]. \)
\item \( a \in [6, 12], b \in [-22, -17], c \in [0, 12], \text{ and } r \in [-35, -34]. \)
\item \( a \in [6, 12], b \in [6, 15], c \in [-23, -18], \text{ and } r \in [-42, -37]. \)
\item \( a \in [14, 26], b \in [36, 38], c \in [40, 45], \text{ and } r \in [65, 74]. \)
\item \( a \in [6, 12], b \in [16, 20], c \in [0, 12], \text{ and } r \in [-8, 1]. \)

\end{enumerate} }
\litem{
Factor the polynomial below completely. Then, choose the intervals the zeros of the polynomial belong to, where $z_1 \leq z_2 \leq z_3$. \textit{To make the problem easier, all zeros are between -5 and 5.}\[ f(x) = 12x^{3} +35 x^{2} -9 x -18 \]\begin{enumerate}[label=\Alph*.]
\item \( z_1 \in [-1.5, -1.26], \text{   }  z_2 \in [0.92, 1.85], \text{   and   } z_3 \in [2.5, 3.1] \)
\item \( z_1 \in [-0.8, -0.64], \text{   }  z_2 \in [0.48, 0.77], \text{   and   } z_3 \in [2.5, 3.1] \)
\item \( z_1 \in [-3.17, -2.34], \text{   }  z_2 \in [-1.51, -1.46], \text{   and   } z_3 \in [1.1, 2.4] \)
\item \( z_1 \in [-3.17, -2.34], \text{   }  z_2 \in [-0.74, -0.58], \text{   and   } z_3 \in [0, 1.1] \)
\item \( z_1 \in [-0.34, -0.12], \text{   }  z_2 \in [1.58, 2.09], \text{   and   } z_3 \in [2.5, 3.1] \)

\end{enumerate} }
\litem{
Factor the polynomial below completely, knowing that $x -4$ is a factor. Then, choose the intervals the zeros of the polynomial belong to, where $z_1 \leq z_2 \leq z_3 \leq z_4$. \textit{To make the problem easier, all zeros are between -5 and 5.}\[ f(x) = 15x^{4} -14 x^{3} -248 x^{2} +224 x + 128 \]\begin{enumerate}[label=\Alph*.]
\item \( z_1 \in [-5, -2], \text{   }  z_2 \in [-4.5, -3.89], z_3 \in [-0.08, 0.26], \text{   and   } z_4 \in [2, 8] \)
\item \( z_1 \in [-5, -2], \text{   }  z_2 \in [-3.04, -2.45], z_3 \in [0.73, 0.85], \text{   and   } z_4 \in [2, 8] \)
\item \( z_1 \in [-5, -2], \text{   }  z_2 \in [-0.52, -0.21], z_3 \in [1.24, 1.46], \text{   and   } z_4 \in [2, 8] \)
\item \( z_1 \in [-5, -2], \text{   }  z_2 \in [-1.72, -1.22], z_3 \in [0.18, 0.71], \text{   and   } z_4 \in [2, 8] \)
\item \( z_1 \in [-5, -2], \text{   }  z_2 \in [-0.94, -0.69], z_3 \in [2.44, 2.56], \text{   and   } z_4 \in [2, 8] \)

\end{enumerate} }
\litem{
Perform the division below. Then, find the intervals that correspond to the quotient in the form $ax^2+bx+c$ and remainder $r$.\[ \frac{9x^{3} +27 x^{2} -25 x -77}{x + 3} \]\begin{enumerate}[label=\Alph*.]
\item \( a \in [9, 10], \text{   } b \in [-2, 2], \text{   } c \in [-27, -24], \text{   and   } r \in [-9, 0]. \)
\item \( a \in [-32, -24], \text{   } b \in [107, 111], \text{   } c \in [-350, -348], \text{   and   } r \in [968, 976]. \)
\item \( a \in [9, 10], \text{   } b \in [53, 60], \text{   } c \in [133, 143], \text{   and   } r \in [334, 340]. \)
\item \( a \in [9, 10], \text{   } b \in [-10, -7], \text{   } c \in [11, 16], \text{   and   } r \in [-121, -118]. \)
\item \( a \in [-32, -24], \text{   } b \in [-55, -51], \text{   } c \in [-188, -186], \text{   and   } r \in [-642, -633]. \)

\end{enumerate} }
\litem{
Perform the division below. Then, find the intervals that correspond to the quotient in the form $ax^2+bx+c$ and remainder $r$.\[ \frac{10x^{3} +30 x^{2} -44}{x + 2} \]\begin{enumerate}[label=\Alph*.]
\item \( a \in [-21, -16], b \in [69, 73], c \in [-140, -137], \text{ and } r \in [232, 243]. \)
\item \( a \in [-21, -16], b \in [-12, -7], c \in [-21, -15], \text{ and } r \in [-88, -81]. \)
\item \( a \in [6, 11], b \in [46, 51], c \in [97, 101], \text{ and } r \in [153, 159]. \)
\item \( a \in [6, 11], b \in [9, 17], c \in [-21, -15], \text{ and } r \in [-4, -3]. \)
\item \( a \in [6, 11], b \in [-5, 6], c \in [-2, 3], \text{ and } r \in [-50, -42]. \)

\end{enumerate} }
\litem{
What are the \textit{possible Rational} roots of the polynomial below?\[ f(x) = 3x^{3} +2 x^{2} +4 x + 7 \]\begin{enumerate}[label=\Alph*.]
\item \( \text{ All combinations of: }\frac{\pm 1,\pm 3}{\pm 1,\pm 7} \)
\item \( \pm 1,\pm 3 \)
\item \( \text{ All combinations of: }\frac{\pm 1,\pm 7}{\pm 1,\pm 3} \)
\item \( \pm 1,\pm 7 \)
\item \( \text{ There is no formula or theorem that tells us all possible Rational roots.} \)

\end{enumerate} }
\litem{
Factor the polynomial below completely, knowing that $x + 3$ is a factor. Then, choose the intervals the zeros of the polynomial belong to, where $z_1 \leq z_2 \leq z_3 \leq z_4$. \textit{To make the problem easier, all zeros are between -5 and 5.}\[ f(x) = 10x^{4} +51 x^{3} -28 x^{2} -333 x -180 \]\begin{enumerate}[label=\Alph*.]
\item \( z_1 \in [-0.74, -0.5], \text{   }  z_2 \in [2.94, 3.06], z_3 \in [1.8, 3.3], \text{   and   } z_4 \in [4, 5] \)
\item \( z_1 \in [-4.18, -3.74], \text{   }  z_2 \in [-3.13, -2.89], z_3 \in [-0.8, -0.4], \text{   and   } z_4 \in [2.5, 3.5] \)
\item \( z_1 \in [-0.47, -0.25], \text{   }  z_2 \in [1.43, 2.14], z_3 \in [1.8, 3.3], \text{   and   } z_4 \in [4, 5] \)
\item \( z_1 \in [-2.55, -2.47], \text{   }  z_2 \in [-0.59, 1.11], z_3 \in [1.8, 3.3], \text{   and   } z_4 \in [4, 5] \)
\item \( z_1 \in [-4.18, -3.74], \text{   }  z_2 \in [-3.13, -2.89], z_3 \in [-2.8, -1.2], \text{   and   } z_4 \in [0.4, 1.4] \)

\end{enumerate} }
\litem{
What are the \textit{possible Rational} roots of the polynomial below?\[ f(x) = 4x^{3} +4 x^{2} +6 x + 6 \]\begin{enumerate}[label=\Alph*.]
\item \( \pm 1,\pm 2,\pm 3,\pm 6 \)
\item \( \text{ All combinations of: }\frac{\pm 1,\pm 2,\pm 4}{\pm 1,\pm 2,\pm 3,\pm 6} \)
\item \( \pm 1,\pm 2,\pm 4 \)
\item \( \text{ All combinations of: }\frac{\pm 1,\pm 2,\pm 3,\pm 6}{\pm 1,\pm 2,\pm 4} \)
\item \( \text{ There is no formula or theorem that tells us all possible Rational roots.} \)

\end{enumerate} }
\end{enumerate}

\end{document}