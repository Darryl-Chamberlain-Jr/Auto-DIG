\documentclass{extbook}[14pt]
\usepackage{multicol, enumerate, enumitem, hyperref, color, soul, setspace, parskip, fancyhdr, amssymb, amsthm, amsmath, latexsym, units, mathtools}
\everymath{\displaystyle}
\usepackage[headsep=0.5cm,headheight=0cm, left=1 in,right= 1 in,top= 1 in,bottom= 1 in]{geometry}
\usepackage{dashrule}  % Package to use the command below to create lines between items
\newcommand{\litem}[1]{\item #1

\rule{\textwidth}{0.4pt}}
\pagestyle{fancy}
\lhead{}
\chead{Answer Key for Progress Quiz 6 Version B}
\rhead{}
\lfoot{4563-7456}
\cfoot{}
\rfoot{Summer C 2021}
\begin{document}
\textbf{This key should allow you to understand why you choose the option you did (beyond just getting a question right or wrong). \href{https://xronos.clas.ufl.edu/mac1105spring2020/courseDescriptionAndMisc/Exams/LearningFromResults}{More instructions on how to use this key can be found here}.}

\textbf{If you have a suggestion to make the keys better, \href{https://forms.gle/CZkbZmPbC9XALEE88}{please fill out the short survey here}.}

\textit{Note: This key is auto-generated and may contain issues and/or errors. The keys are reviewed after each exam to ensure grading is done accurately. If there are issues (like duplicate options), they are noted in the offline gradebook. The keys are a work-in-progress to give students as many resources to improve as possible.}

\rule{\textwidth}{0.4pt}

\begin{enumerate}\litem{
Which of the following intervals describes the Domain of the function below?
\[ f(x) = -\log_2{(x+3)}+6 \]The solution is \( (-3, \infty) \), which is option D.\begin{enumerate}[label=\Alph*.]
\item \( [a, \infty), a \in [4.8, 8.1] \)

$[6, \infty)$, which corresponds to using the vertical shift when shifting the Domain AND including the endpoint.
\item \( (-\infty, a), a \in [1.5, 5.8] \)

$(-\infty, 3)$, which corresponds to flipping the Domain. Remember: the general for is $a*\log(x-h)+k$, \textbf{where $a$ does not affect the domain}.
\item \( (-\infty, a], a \in [-7.7, -3.1] \)

$(-\infty, -6]$, which corresponds to using the negative vertical shift AND including the endpoint AND flipping the domain.
\item \( (a, \infty), a \in [-3.1, -2.7] \)

* $(-3, \infty)$, which is the correct option.
\item \( (-\infty, \infty) \)

This corresponds to thinking of the range of the log function (or the domain of the exponential function).
\end{enumerate}

\textbf{General Comment:} \textbf{General Comments}: The domain of a basic logarithmic function is $(0, \infty)$ and the Range is $(-\infty, \infty)$. We can use shifts when finding the Domain, but the Range will always be all Real numbers.
}
\litem{
Which of the following intervals describes the Domain of the function below?
\[ f(x) = e^{x+2}+7 \]The solution is \( (-\infty, \infty) \), which is option E.\begin{enumerate}[label=\Alph*.]
\item \( [a, \infty), a \in [-13, -2] \)

$[-7, \infty)$, which corresponds to using the negative vertical shift AND flipping the Range interval AND including the endpoint.
\item \( (-\infty, a), a \in [7, 8] \)

$(-\infty, 7)$, which corresponds to using the correct vertical shift *if we wanted the Range*.
\item \( (a, \infty), a \in [-13, -2] \)

$(-7, \infty)$, which corresponds to using the negative vertical shift AND flipping the Range interval.
\item \( (-\infty, a], a \in [7, 8] \)

$(-\infty, 7]$, which corresponds to using the correct vertical shift *if we wanted the Range* AND including the endpoint.
\item \( (-\infty, \infty) \)

* This is the correct option.
\end{enumerate}

\textbf{General Comment:} \textbf{General Comments}: Domain of a basic exponential function is $(-\infty, \infty)$ while the Range is $(0, \infty)$. We can shift these intervals [and even flip when $a<0$!] to find the new Domain/Range.
}
\litem{
Which of the following intervals describes the Range of the function below?
\[ f(x) = \log_2{(x+3)}-6 \]The solution is \( (\infty, \infty) \), which is option E.\begin{enumerate}[label=\Alph*.]
\item \( (-\infty, a), a \in [4.4, 8.6] \)

$(-\infty, 6)$, which corresponds to using the using the negative of vertical shift on $(0, \infty)$.
\item \( [a, \infty), a \in [0.6, 3.9] \)

$[3, \infty)$, which corresponds to using the negative of the horizontal shift AND including the endpoint.
\item \( [a, \infty), a \in [-3.9, 0.8] \)

$[-6, \infty)$, which corresponds to using the flipped Domain AND including the endpoint.
\item \( (-\infty, a), a \in [-8.8, -5.8] \)

$(-\infty, -6)$, which corresponds to using the vertical shift while the Range is $(-\infty, \infty)$.
\item \( (-\infty, \infty) \)

*This is the correct option.
\end{enumerate}

\textbf{General Comment:} \textbf{General Comments}: The domain of a basic logarithmic function is $(0, \infty)$ and the Range is $(-\infty, \infty)$. We can use shifts when finding the Domain, but the Range will always be all Real numbers.
}
\litem{
 Solve the equation for $x$ and choose the interval that contains $x$ (if it exists).
\[  24 = \sqrt[3]{\frac{18}{e^{4x}}} \]The solution is \( x = -1.661, \text{ which does not fit in any of the interval options.} \), which is option E.\begin{enumerate}[label=\Alph*.]
\item \( x \in [0.8, 2.9] \)

$x = 1.661$, which is the negative of the correct solution.
\item \( x \in [-1.1, 0.5] \)

$x = -0.866$, which corresponds to treating any root as a square root.
\item \( x \in [-19.8, -17.9] \)

$x = -18.723$, which corresponds to thinking you don't need to take the natural log of both sides before reducing, as if the right side already has a natural log.
\item \( \text{There is no Real solution to the equation.} \)

This corresponds to believing you cannot solve the equation.
\item \( \text{None of the above.} \)

* $x = -1.661$ is the correct solution and does not fit in any of the other intervals.
\end{enumerate}

\textbf{General Comment:} \textbf{General Comments}: After using the properties of logarithmic functions to break up the right-hand side, use $\ln(e) = 1$ to reduce the question to a linear function to solve. You can put $\ln(18)$ into a calculator if you are having trouble.
}
\litem{
Solve the equation for $x$ and choose the interval that contains the solution (if it exists).
\[ 5^{5x-3} = 64^{2x+4} \]The solution is \( x = -79.326 \), which is option B.\begin{enumerate}[label=\Alph*.]
\item \( x \in [2.33, 3.33] \)

$x = 2.333$, which corresponds to solving the numerators as equal while ignoring the bases are different.
\item \( x \in [-79.33, -76.33] \)

* $x = -79.326$, which is the correct option.
\item \( x \in [5.15, 8.15] \)

$x = 7.155$, which corresponds to distributing the $\ln(base)$ to the second term of the exponent only.
\item \( x \in [-26.87, -24.87] \)

$x = -25.871$, which corresponds to distributing the $\ln(base)$ to the first term of the exponent only.
\item \( \text{There is no Real solution to the equation.} \)

This corresponds to believing there is no solution since the bases are not powers of each other.
\end{enumerate}

\textbf{General Comment:} \textbf{General Comments:} This question was written so that the bases could not be written the same. You will need to take the log of both sides.
}
\litem{
Solve the equation for $x$ and choose the interval that contains the solution (if it exists).
\[ 2^{3x-3} = \left(\frac{1}{125}\right)^{4x+4} \]The solution is \( x = -0.806 \), which is option D.\begin{enumerate}[label=\Alph*.]
\item \( x \in [-8.8, -6.3] \)

$x = -7.000$, which corresponds to solving the numerators as equal while ignoring the bases are different.
\item \( x \in [-0.7, 2.1] \)

$x = 0.327$, which corresponds to distributing the $\ln(base)$ to the first term of the exponent only.
\item \( x \in [16.5, 18.1] \)

$x = 17.234$, which corresponds to distributing the $\ln(base)$ to the second term of the exponent only.
\item \( x \in [-1.3, -0.3] \)

* $x = -0.806$, which is the correct option.
\item \( \text{There is no Real solution to the equation.} \)

This corresponds to believing there is no solution since the bases are not powers of each other.
\end{enumerate}

\textbf{General Comment:} \textbf{General Comments:} This question was written so that the bases could not be written the same. You will need to take the log of both sides.
}
\litem{
Which of the following intervals describes the Domain of the function below?
\[ f(x) = -e^{x-7}-5 \]The solution is \( (-\infty, \infty) \), which is option E.\begin{enumerate}[label=\Alph*.]
\item \( (-\infty, a), a \in [-9, -4] \)

$(-\infty, -5)$, which corresponds to using the correct vertical shift *if we wanted the Range*.
\item \( [a, \infty), a \in [4, 9] \)

$[5, \infty)$, which corresponds to using the negative vertical shift AND flipping the Range interval AND including the endpoint.
\item \( (-\infty, a], a \in [-9, -4] \)

$(-\infty, -5]$, which corresponds to using the correct vertical shift *if we wanted the Range* AND including the endpoint.
\item \( (a, \infty), a \in [4, 9] \)

$(5, \infty)$, which corresponds to using the negative vertical shift AND flipping the Range interval.
\item \( (-\infty, \infty) \)

* This is the correct option.
\end{enumerate}

\textbf{General Comment:} \textbf{General Comments}: Domain of a basic exponential function is $(-\infty, \infty)$ while the Range is $(0, \infty)$. We can shift these intervals [and even flip when $a<0$!] to find the new Domain/Range.
}
\litem{
Solve the equation for $x$ and choose the interval that contains the solution (if it exists).
\[ \log_{5}{(2x+6)}+5 = 2 \]The solution is \( x = -2.996 \), which is option D.\begin{enumerate}[label=\Alph*.]
\item \( x \in [-119.5, -117.5] \)

$x = -118.500$, which corresponds to reversing the base and exponent when converting and reversing the value with $x$.
\item \( x \in [-125.5, -120.5] \)

$x = -124.500$, which corresponds to reversing the base and exponent when converting.
\item \( x \in [7.5, 17.5] \)

$x = 9.500$, which corresponds to ignoring the vertical shift when converting to exponential form.
\item \( x \in [-3, -1] \)

* $x = -2.996$, which is the correct option.
\item \( \text{There is no Real solution to the equation.} \)

Corresponds to believing a negative coefficient within the log equation means there is no Real solution.
\end{enumerate}

\textbf{General Comment:} \textbf{General Comments:} First, get the equation in the form $\log_b{(cx+d)} = a$. Then, convert to $b^a = cx+d$ and solve.
}
\litem{
Solve the equation for $x$ and choose the interval that contains the solution (if it exists).
\[ \log_{4}{(4x+6)}+5 = 2 \]The solution is \( x = -1.496 \), which is option A.\begin{enumerate}[label=\Alph*.]
\item \( x \in [-1.8, 0.4] \)

* $x = -1.496$, which is the correct option.
\item \( x \in [16.2, 21.4] \)

$x = 18.750$, which corresponds to reversing the base and exponent when converting.
\item \( x \in [1.6, 2.6] \)

$x = 2.500$, which corresponds to ignoring the vertical shift when converting to exponential form.
\item \( x \in [21, 23.7] \)

$x = 21.750$, which corresponds to reversing the base and exponent when converting and reversing the value with $x$.
\item \( \text{There is no Real solution to the equation.} \)

Corresponds to believing a negative coefficient within the log equation means there is no Real solution.
\end{enumerate}

\textbf{General Comment:} \textbf{General Comments:} First, get the equation in the form $\log_b{(cx+d)} = a$. Then, convert to $b^a = cx+d$ and solve.
}
\litem{
 Solve the equation for $x$ and choose the interval that contains $x$ (if it exists).
\[  15 = \ln{\sqrt[3]{\frac{14}{e^{6x}}}} \]The solution is \( x = -7.06 \), which is option C.\begin{enumerate}[label=\Alph*.]
\item \( x \in [-6.56, -2.56] \)

$x = -4.560$, which corresponds to treating any root as a square root.
\item \( x \in [-3.79, -0.79] \)

$x = -1.794$, which corresponds to thinking you need to take the natural log of on the left before reducing.
\item \( x \in [-8.06, -6.06] \)

* $x = -7.060$, which is the correct option.
\item \( \text{There is no Real solution to the equation.} \)

This corresponds to believing you cannot solve the equation.
\item \( \text{None of the above.} \)

This corresponds to making an unexpected error.
\end{enumerate}

\textbf{General Comment:} \textbf{General Comments}: After using the properties of logarithmic functions to break up the right-hand side, use $\ln(e) = 1$ to reduce the question to a linear function to solve. You can put $\ln(14)$ into a calculator if you are having trouble.
}
\end{enumerate}

\end{document}