\documentclass{extbook}[14pt]
\usepackage{multicol, enumerate, enumitem, hyperref, color, soul, setspace, parskip, fancyhdr, amssymb, amsthm, amsmath, latexsym, units, mathtools}
\everymath{\displaystyle}
\usepackage[headsep=0.5cm,headheight=0cm, left=1 in,right= 1 in,top= 1 in,bottom= 1 in]{geometry}
\usepackage{dashrule}  % Package to use the command below to create lines between items
\newcommand{\litem}[1]{\item #1

\rule{\textwidth}{0.4pt}}
\pagestyle{fancy}
\lhead{}
\chead{Answer Key for Progress Quiz 6 Version B}
\rhead{}
\lfoot{4563-7456}
\cfoot{}
\rfoot{Summer C 2021}
\begin{document}
\textbf{This key should allow you to understand why you choose the option you did (beyond just getting a question right or wrong). \href{https://xronos.clas.ufl.edu/mac1105spring2020/courseDescriptionAndMisc/Exams/LearningFromResults}{More instructions on how to use this key can be found here}.}

\textbf{If you have a suggestion to make the keys better, \href{https://forms.gle/CZkbZmPbC9XALEE88}{please fill out the short survey here}.}

\textit{Note: This key is auto-generated and may contain issues and/or errors. The keys are reviewed after each exam to ensure grading is done accurately. If there are issues (like duplicate options), they are noted in the offline gradebook. The keys are a work-in-progress to give students as many resources to improve as possible.}

\rule{\textwidth}{0.4pt}

\begin{enumerate}\litem{
Choose the \textbf{smallest} set of Real numbers that the number below belongs to.
\[ \sqrt{\frac{193600}{484}} \]The solution is \( \text{Whole} \), which is option A.\begin{enumerate}[label=\Alph*.]
\item \( \text{Whole} \)

* This is the correct option!
\item \( \text{Integer} \)

These are the negative and positive counting numbers (..., -3, -2, -1, 0, 1, 2, 3, ...)
\item \( \text{Not a Real number} \)

These are Nonreal Complex numbers \textbf{OR} things that are not numbers (e.g., dividing by 0).
\item \( \text{Irrational} \)

These cannot be written as a fraction of Integers.
\item \( \text{Rational} \)

These are numbers that can be written as fraction of Integers (e.g., -2/3)
\end{enumerate}

\textbf{General Comment:} First, you \textbf{NEED} to simplify the expression. This question simplifies to $440$. 
 
 Be sure you look at the simplified fraction and not just the decimal expansion. Numbers such as 13, 17, and 19 provide \textbf{long but repeating/terminating decimal expansions!} 
 
 The only ways to *not* be a Real number are: dividing by 0 or taking the square root of a negative number. 
 
 Irrational numbers are more than just square root of 3: adding or subtracting values from square root of 3 is also irrational.
}
\litem{
Simplify the expression below into the form $a+bi$. Then, choose the intervals that $a$ and $b$ belong to.
\[ (-4 + 5 i)(-3 - 10 i) \]The solution is \( 62 + 25 i \), which is option D.\begin{enumerate}[label=\Alph*.]
\item \( a \in [-42, -35] \text{ and } b \in [55, 58] \)

 $-38 + 55 i$, which corresponds to adding a minus sign in the first term.
\item \( a \in [-42, -35] \text{ and } b \in [-57, -53] \)

 $-38 - 55 i$, which corresponds to adding a minus sign in the second term.
\item \( a \in [12, 14] \text{ and } b \in [-52, -47] \)

 $12 - 50 i$, which corresponds to just multiplying the real terms to get the real part of the solution and the coefficients in the complex terms to get the complex part.
\item \( a \in [58, 63] \text{ and } b \in [24, 30] \)

* $62 + 25 i$, which is the correct option.
\item \( a \in [58, 63] \text{ and } b \in [-31, -19] \)

 $62 - 25 i$, which corresponds to adding a minus sign in both terms.
\end{enumerate}

\textbf{General Comment:} You can treat $i$ as a variable and distribute. Just remember that $i^2=-1$, so you can continue to reduce after you distribute.
}
\litem{
Simplify the expression below and choose the interval the simplification is contained within.
\[ 6 - 1^2 + 3 \div 20 * 5 \div 10 \]The solution is \( 5.075 \), which is option D.\begin{enumerate}[label=\Alph*.]
\item \( [7, 7.04] \)

 7.003, which corresponds to two Order of Operations errors.
\item \( [7.07, 7.09] \)

 7.075, which corresponds to an Order of Operations error: multiplying by negative before squaring. For example: $(-3)^2 \neq -3^2$
\item \( [4.97, 5.02] \)

 5.003, which corresponds to an Order of Operations error: not reading left-to-right for multiplication/division.
\item \( [5.07, 5.09] \)

* 5.075, this is the correct option
\item \( \text{None of the above} \)

 You may have gotten this by making an unanticipated error. If you got a value that is not any of the others, please let the coordinator know so they can help you figure out what happened.
\end{enumerate}

\textbf{General Comment:} While you may remember (or were taught) PEMDAS is done in order, it is actually done as P/E/MD/AS. When we are at MD or AS, we read left to right.
}
\litem{
Choose the \textbf{smallest} set of Complex numbers that the number below belongs to.
\[ \sqrt{\frac{-780}{6}} i+\sqrt{143}i \]The solution is \( \text{Nonreal Complex} \), which is option E.\begin{enumerate}[label=\Alph*.]
\item \( \text{Pure Imaginary} \)

This is a Complex number $(a+bi)$ that \textbf{only} has an imaginary part like $2i$.
\item \( \text{Not a Complex Number} \)

This is not a number. The only non-Complex number we know is dividing by 0 as this is not a number!
\item \( \text{Irrational} \)

These cannot be written as a fraction of Integers. Remember: $\pi$ is not an Integer!
\item \( \text{Rational} \)

These are numbers that can be written as fraction of Integers (e.g., -2/3 + 5)
\item \( \text{Nonreal Complex} \)

* This is the correct option!
\end{enumerate}

\textbf{General Comment:} Be sure to simplify $i^2 = -1$. This may remove the imaginary portion for your number. If you are having trouble, you may want to look at the \textit{Subgroups of the Real Numbers} section.
}
\litem{
Simplify the expression below into the form $a+bi$. Then, choose the intervals that $a$ and $b$ belong to.
\[ (-4 + 9 i)(6 - 3 i) \]The solution is \( 3 + 66 i \), which is option A.\begin{enumerate}[label=\Alph*.]
\item \( a \in [-2, 4] \text{ and } b \in [65, 68] \)

* $3 + 66 i$, which is the correct option.
\item \( a \in [-26, -23] \text{ and } b \in [-33, -25] \)

 $-24 - 27 i$, which corresponds to just multiplying the real terms to get the real part of the solution and the coefficients in the complex terms to get the complex part.
\item \( a \in [-52, -50] \text{ and } b \in [-43, -41] \)

 $-51 - 42 i$, which corresponds to adding a minus sign in the first term.
\item \( a \in [-2, 4] \text{ and } b \in [-71, -61] \)

 $3 - 66 i$, which corresponds to adding a minus sign in both terms.
\item \( a \in [-52, -50] \text{ and } b \in [42, 49] \)

 $-51 + 42 i$, which corresponds to adding a minus sign in the second term.
\end{enumerate}

\textbf{General Comment:} You can treat $i$ as a variable and distribute. Just remember that $i^2=-1$, so you can continue to reduce after you distribute.
}
\litem{
Simplify the expression below and choose the interval the simplification is contained within.
\[ 16 - 8 \div 6 * 9 - (18 * 20) \]The solution is \( -356.000 \), which is option C.\begin{enumerate}[label=\Alph*.]
\item \( [-344.15, -341.15] \)

 -344.148, which corresponds to an Order of Operations error: not reading left-to-right for multiplication/division.
\item \( [370.85, 377.85] \)

 375.852, which corresponds to not distributing addition and subtraction correctly.
\item \( [-357, -353] \)

* -356.000, which is the correct option.
\item \( [-288, -276] \)

 -280.000, which corresponds to not distributing a negative correctly.
\item \( \text{None of the above} \)

 You may have gotten this by making an unanticipated error. If you got a value that is not any of the others, please let the coordinator know so they can help you figure out what happened.
\end{enumerate}

\textbf{General Comment:} While you may remember (or were taught) PEMDAS is done in order, it is actually done as P/E/MD/AS. When we are at MD or AS, we read left to right.
}
\litem{
Simplify the expression below into the form $a+bi$. Then, choose the intervals that $a$ and $b$ belong to.
\[ \frac{9 - 77 i}{6 - 4 i} \]The solution is \( 6.96  - 8.19 i \), which is option B.\begin{enumerate}[label=\Alph*.]
\item \( a \in [-6.5, -4.5] \text{ and } b \in [-10, -9] \)

 $-4.88  - 9.58 i$, which corresponds to forgetting to multiply the conjugate by the numerator and not computing the conjugate correctly.
\item \( a \in [6.5, 7.5] \text{ and } b \in [-8.5, -7] \)

* $6.96  - 8.19 i$, which is the correct option.
\item \( a \in [6.5, 7.5] \text{ and } b \in [-426.5, -425] \)

 $6.96  - 426.00 i$, which corresponds to forgetting to multiply the conjugate by the numerator.
\item \( a \in [0, 2.5] \text{ and } b \in [18, 20] \)

 $1.50  + 19.25 i$, which corresponds to just dividing the first term by the first term and the second by the second.
\item \( a \in [361.5, 363] \text{ and } b \in [-8.5, -7] \)

 $362.00  - 8.19 i$, which corresponds to forgetting to multiply the conjugate by the numerator and using a plus instead of a minus in the denominator.
\end{enumerate}

\textbf{General Comment:} Multiply the numerator and denominator by the *conjugate* of the denominator, then simplify. For example, if we have $2+3i$, the conjugate is $2-3i$.
}
\litem{
Choose the \textbf{smallest} set of Complex numbers that the number below belongs to.
\[ \sqrt{\frac{-990}{9}} i+\sqrt{165}i \]The solution is \( \text{Nonreal Complex} \), which is option D.\begin{enumerate}[label=\Alph*.]
\item \( \text{Pure Imaginary} \)

This is a Complex number $(a+bi)$ that \textbf{only} has an imaginary part like $2i$.
\item \( \text{Irrational} \)

These cannot be written as a fraction of Integers. Remember: $\pi$ is not an Integer!
\item \( \text{Not a Complex Number} \)

This is not a number. The only non-Complex number we know is dividing by 0 as this is not a number!
\item \( \text{Nonreal Complex} \)

* This is the correct option!
\item \( \text{Rational} \)

These are numbers that can be written as fraction of Integers (e.g., -2/3 + 5)
\end{enumerate}

\textbf{General Comment:} Be sure to simplify $i^2 = -1$. This may remove the imaginary portion for your number. If you are having trouble, you may want to look at the \textit{Subgroups of the Real Numbers} section.
}
\litem{
Simplify the expression below into the form $a+bi$. Then, choose the intervals that $a$ and $b$ belong to.
\[ \frac{54 + 55 i}{-1 + 8 i} \]The solution is \( 5.94  - 7.49 i \), which is option A.\begin{enumerate}[label=\Alph*.]
\item \( a \in [5, 6.5] \text{ and } b \in [-8.5, -7] \)

* $5.94  - 7.49 i$, which is the correct option.
\item \( a \in [-54.5, -53] \text{ and } b \in [6, 7.5] \)

 $-54.00  + 6.88 i$, which corresponds to just dividing the first term by the first term and the second by the second.
\item \( a \in [-8.5, -7] \text{ and } b \in [5, 6.5] \)

 $-7.60  + 5.80 i$, which corresponds to forgetting to multiply the conjugate by the numerator and not computing the conjugate correctly.
\item \( a \in [385, 386.5] \text{ and } b \in [-8.5, -7] \)

 $386.00  - 7.49 i$, which corresponds to forgetting to multiply the conjugate by the numerator and using a plus instead of a minus in the denominator.
\item \( a \in [5, 6.5] \text{ and } b \in [-487.5, -486] \)

 $5.94  - 487.00 i$, which corresponds to forgetting to multiply the conjugate by the numerator.
\end{enumerate}

\textbf{General Comment:} Multiply the numerator and denominator by the *conjugate* of the denominator, then simplify. For example, if we have $2+3i$, the conjugate is $2-3i$.
}
\litem{
Choose the \textbf{smallest} set of Real numbers that the number below belongs to.
\[ -\sqrt{\frac{42849}{529}} \]The solution is \( \text{Integer} \), which is option D.\begin{enumerate}[label=\Alph*.]
\item \( \text{Whole} \)

These are the counting numbers with 0 (0, 1, 2, 3, ...)
\item \( \text{Rational} \)

These are numbers that can be written as fraction of Integers (e.g., -2/3)
\item \( \text{Irrational} \)

These cannot be written as a fraction of Integers.
\item \( \text{Integer} \)

* This is the correct option!
\item \( \text{Not a Real number} \)

These are Nonreal Complex numbers \textbf{OR} things that are not numbers (e.g., dividing by 0).
\end{enumerate}

\textbf{General Comment:} First, you \textbf{NEED} to simplify the expression. This question simplifies to $-207$. 
 
 Be sure you look at the simplified fraction and not just the decimal expansion. Numbers such as 13, 17, and 19 provide \textbf{long but repeating/terminating decimal expansions!} 
 
 The only ways to *not* be a Real number are: dividing by 0 or taking the square root of a negative number. 
 
 Irrational numbers are more than just square root of 3: adding or subtracting values from square root of 3 is also irrational.
}
\end{enumerate}

\end{document}