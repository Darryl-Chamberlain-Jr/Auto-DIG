\documentclass{extbook}[14pt]
\usepackage{multicol, enumerate, enumitem, hyperref, color, soul, setspace, parskip, fancyhdr, amssymb, amsthm, amsmath, latexsym, units, mathtools}
\everymath{\displaystyle}
\usepackage[headsep=0.5cm,headheight=0cm, left=1 in,right= 1 in,top= 1 in,bottom= 1 in]{geometry}
\usepackage{dashrule}  % Package to use the command below to create lines between items
\newcommand{\litem}[1]{\item #1

\rule{\textwidth}{0.4pt}}
\pagestyle{fancy}
\lhead{}
\chead{Answer Key for Progress Quiz 6 Version B}
\rhead{}
\lfoot{4563-7456}
\cfoot{}
\rfoot{Summer C 2021}
\begin{document}
\textbf{This key should allow you to understand why you choose the option you did (beyond just getting a question right or wrong). \href{https://xronos.clas.ufl.edu/mac1105spring2020/courseDescriptionAndMisc/Exams/LearningFromResults}{More instructions on how to use this key can be found here}.}

\textbf{If you have a suggestion to make the keys better, \href{https://forms.gle/CZkbZmPbC9XALEE88}{please fill out the short survey here}.}

\textit{Note: This key is auto-generated and may contain issues and/or errors. The keys are reviewed after each exam to ensure grading is done accurately. If there are issues (like duplicate options), they are noted in the offline gradebook. The keys are a work-in-progress to give students as many resources to improve as possible.}

\rule{\textwidth}{0.4pt}

\begin{enumerate}\litem{
Choose the interval below that $f$ composed with $g$ at $x=-1$ is in.
\[ f(x) = -x^{3} +3 x^{2} +4 x \text{ and } g(x) = -4x^{3} -4 x^{2} +4 x + 3 \]The solution is \( 0.0 \), which is option B.\begin{enumerate}[label=\Alph*.]
\item \( (f \circ g)(-1) \in [-7, -4] \)

 Distractor 3: Corresponds to being slightly off from the solution.
\item \( (f \circ g)(-1) \in [-3, 1] \)

* This is the correct solution
\item \( (f \circ g)(-1) \in [3, 10] \)

 Distractor 1: Corresponds to reversing the composition.
\item \( (f \circ g)(-1) \in [-12, -8] \)

 Distractor 2: Corresponds to being slightly off from the solution.
\item \( \text{It is not possible to compose the two functions.} \)


\end{enumerate}

\textbf{General Comment:} $f$ composed with $g$ at $x$ means $f(g(x))$. The order matters!
}
\litem{
Choose the interval below that $f$ composed with $g$ at $x=1$ is in.
\[ f(x) = -2x^{3} +2 x^{2} -2 x \text{ and } g(x) = x^{3} -2 x^{2} +x \]The solution is \( 0.0 \), which is option A.\begin{enumerate}[label=\Alph*.]
\item \( (f \circ g)(1) \in [-0.7, 1.9] \)

* This is the correct solution
\item \( (f \circ g)(1) \in [-16.8, -11.2] \)

 Distractor 3: Corresponds to being slightly off from the solution.
\item \( (f \circ g)(1) \in [-5.9, -3.1] \)

 Distractor 2: Corresponds to being slightly off from the solution.
\item \( (f \circ g)(1) \in [-19.7, -15.7] \)

 Distractor 1: Corresponds to reversing the composition.
\item \( \text{It is not possible to compose the two functions.} \)


\end{enumerate}

\textbf{General Comment:} $f$ composed with $g$ at $x$ means $f(g(x))$. The order matters!
}
\litem{
Determine whether the function below is 1-1.
\[ f(x) = 15 x^2 - 189 x + 594 \]The solution is \( \text{no} \), which is option D.\begin{enumerate}[label=\Alph*.]
\item \( \text{No, because there is an $x$-value that goes to 2 different $y$-values.} \)

Corresponds to the Vertical Line test, which checks if an expression is a function.
\item \( \text{Yes, the function is 1-1.} \)

Corresponds to believing the function passes the Horizontal Line test.
\item \( \text{No, because the range of the function is not $(-\infty, \infty)$.} \)

Corresponds to believing 1-1 means the range is all Real numbers.
\item \( \text{No, because there is a $y$-value that goes to 2 different $x$-values.} \)

* This is the solution.
\item \( \text{No, because the domain of the function is not $(-\infty, \infty)$.} \)

Corresponds to believing 1-1 means the domain is all Real numbers.
\end{enumerate}

\textbf{General Comment:} There are only two valid options: The function is 1-1 OR No because there is a $y$-value that goes to 2 different $x$-values.
}
\litem{
Find the inverse of the function below. Then, evaluate the inverse at $x = 8$ and choose the interval that $f^-1(8)$ belongs to.
\[ f(x) = \ln{(x-2)}-5 \]The solution is \( f^{-1}(8) = 442415.392 \), which is option A.\begin{enumerate}[label=\Alph*.]
\item \( f^{-1}(8) \in [442414.39, 442417.39] \)

 This is the solution.
\item \( f^{-1}(8) \in [22016.47, 22027.47] \)

 This solution corresponds to distractor 2.
\item \( f^{-1}(8) \in [15.09, 25.09] \)

 This solution corresponds to distractor 1.
\item \( f^{-1}(8) \in [396.43, 399.43] \)

 This solution corresponds to distractor 4.
\item \( f^{-1}(8) \in [442405.39, 442412.39] \)

 This solution corresponds to distractor 3.
\end{enumerate}

\textbf{General Comment:} Natural log and exponential functions always have an inverse. Once you switch the $x$ and $y$, use the conversion $ e^y = x \leftrightarrow y=\ln(x)$.
}
\litem{
Find the inverse of the function below (if it exists). Then, evaluate the inverse at $x = 15$ and choose the interval that $f^-1(15)$ belongs to.
\[ f(x) = \sqrt[3]{4 x - 3} \]The solution is \( 844.5 \), which is option A.\begin{enumerate}[label=\Alph*.]
\item \( f^{-1}(15) \in [843.5, 844.8] \)

* This is the correct solution.
\item \( f^{-1}(15) \in [-847.1, -843.8] \)

 This solution corresponds to distractor 2.
\item \( f^{-1}(15) \in [841.1, 843.1] \)

 Distractor 1: This corresponds to 
\item \( f^{-1}(15) \in [-843.1, -839.4] \)

 This solution corresponds to distractor 3.
\item \( \text{ The function is not invertible for all Real numbers. } \)

 This solution corresponds to distractor 4.
\end{enumerate}

\textbf{General Comment:} Be sure you check that the function is 1-1 before trying to find the inverse!
}
\litem{
Find the inverse of the function below. Then, evaluate the inverse at $x = 9$ and choose the interval that $f^-1(9)$ belongs to.
\[ f(x) = e^{x+4}-3 \]The solution is \( f^{-1}(9) = -1.515 \), which is option B.\begin{enumerate}[label=\Alph*.]
\item \( f^{-1}(9) \in [-1.44, -1.23] \)

 This solution corresponds to distractor 3.
\item \( f^{-1}(9) \in [-1.58, -1.46] \)

 This is the solution.
\item \( f^{-1}(9) \in [-0.58, -0.38] \)

 This solution corresponds to distractor 4.
\item \( f^{-1}(9) \in [-1.36, -1.19] \)

 This solution corresponds to distractor 2.
\item \( f^{-1}(9) \in [6.47, 6.65] \)

 This solution corresponds to distractor 1.
\end{enumerate}

\textbf{General Comment:} Natural log and exponential functions always have an inverse. Once you switch the $x$ and $y$, use the conversion $ e^y = x \leftrightarrow y=\ln(x)$.
}
\litem{
Find the inverse of the function below (if it exists). Then, evaluate the inverse at $x = -11$ and choose the interval that $f^-1(-11)$ belongs to.
\[ f(x) = \sqrt[3]{2 x + 4} \]The solution is \( -667.5 \), which is option C.\begin{enumerate}[label=\Alph*.]
\item \( f^{-1}(-11) \in [-663.5, -660.5] \)

 Distractor 1: This corresponds to 
\item \( f^{-1}(-11) \in [662.5, 664.5] \)

 This solution corresponds to distractor 3.
\item \( f^{-1}(-11) \in [-674.5, -665.5] \)

* This is the correct solution.
\item \( f^{-1}(-11) \in [664.5, 668.5] \)

 This solution corresponds to distractor 2.
\item \( \text{ The function is not invertible for all Real numbers. } \)

 This solution corresponds to distractor 4.
\end{enumerate}

\textbf{General Comment:} Be sure you check that the function is 1-1 before trying to find the inverse!
}
\litem{
Multiply the following functions, then choose the domain of the resulting function from the list below.
\[ f(x) = 8x^{2} + 8 \text{ and } g(x) = \sqrt{3x+15}  \]The solution is \( \text{ The domain is all Real numbers greater than or equal to} x = -5.0. \), which is option A.\begin{enumerate}[label=\Alph*.]
\item \( \text{ The domain is all Real numbers greater than or equal to } x = a, \text{ where } a \in [-6, -1] \)


\item \( \text{ The domain is all Real numbers except } x = a, \text{ where } a \in [0.83, 5.83] \)


\item \( \text{ The domain is all Real numbers less than or equal to } x = a, \text{ where } a \in [-0.6, 8.4] \)


\item \( \text{ The domain is all Real numbers except } x = a \text{ and } x = b, \text{ where } a \in [-9.67, -1.67] \text{ and } b \in [-3.75, 1.25] \)


\item \( \text{ The domain is all Real numbers. } \)


\end{enumerate}

\textbf{General Comment:} The new domain is the intersection of the previous domains.
}
\litem{
Determine whether the function below is 1-1.
\[ f(x) = (5 x - 18)^3 \]The solution is \( \text{yes} \), which is option A.\begin{enumerate}[label=\Alph*.]
\item \( \text{Yes, the function is 1-1.} \)

* This is the solution.
\item \( \text{No, because the range of the function is not $(-\infty, \infty)$.} \)

Corresponds to believing 1-1 means the range is all Real numbers.
\item \( \text{No, because there is a $y$-value that goes to 2 different $x$-values.} \)

Corresponds to the Horizontal Line test, which this function passes.
\item \( \text{No, because the domain of the function is not $(-\infty, \infty)$.} \)

Corresponds to believing 1-1 means the domain is all Real numbers.
\item \( \text{No, because there is an $x$-value that goes to 2 different $y$-values.} \)

Corresponds to the Vertical Line test, which checks if an expression is a function.
\end{enumerate}

\textbf{General Comment:} There are only two valid options: The function is 1-1 OR No because there is a $y$-value that goes to 2 different $x$-values.
}
\litem{
Add the following functions, then choose the domain of the resulting function from the list below.
\[ f(x) = 9x^{3} +8 x^{2} +6 x \text{ and } g(x) = \sqrt{-3x+10}  \]The solution is \( \text{ The domain is all Real numbers less than or equal to} x = 3.33. \), which is option C.\begin{enumerate}[label=\Alph*.]
\item \( \text{ The domain is all Real numbers greater than or equal to } x = a, \text{ where } a \in [4.5, 10.5] \)


\item \( \text{ The domain is all Real numbers except } x = a, \text{ where } a \in [-8.25, 0.75] \)


\item \( \text{ The domain is all Real numbers less than or equal to } x = a, \text{ where } a \in [3.33, 4.33] \)


\item \( \text{ The domain is all Real numbers except } x = a \text{ and } x = b, \text{ where } a \in [3.75, 5.75] \text{ and } b \in [-6.2, -3.2] \)


\item \( \text{ The domain is all Real numbers. } \)


\end{enumerate}

\textbf{General Comment:} The new domain is the intersection of the previous domains.
}
\end{enumerate}

\end{document}