\documentclass[14pt]{extbook}
\usepackage{multicol, enumerate, enumitem, hyperref, color, soul, setspace, parskip, fancyhdr} %General Packages
\usepackage{amssymb, amsthm, amsmath, latexsym, units, mathtools} %Math Packages
\everymath{\displaystyle} %All math in Display Style
% Packages with additional options
\usepackage[headsep=0.5cm,headheight=12pt, left=1 in,right= 1 in,top= 1 in,bottom= 1 in]{geometry}
\usepackage[usenames,dvipsnames]{xcolor}
\usepackage{dashrule}  % Package to use the command below to create lines between items
\newcommand{\litem}[1]{\item#1\hspace*{-1cm}\rule{\textwidth}{0.4pt}}
\pagestyle{fancy}
\lhead{Progress Quiz 6}
\chead{}
\rhead{Version B}
\lfoot{4563-7456}
\cfoot{}
\rfoot{Summer C 2021}
\begin{document}

\begin{enumerate}
\litem{
Choose the interval below that $f$ composed with $g$ at $x=-1$ is in.\[ f(x) = -x^{3} +3 x^{2} +4 x \text{ and } g(x) = -4x^{3} -4 x^{2} +4 x + 3 \]\begin{enumerate}[label=\Alph*.]
\item \( (f \circ g)(-1) \in [-7, -4] \)
\item \( (f \circ g)(-1) \in [-3, 1] \)
\item \( (f \circ g)(-1) \in [3, 10] \)
\item \( (f \circ g)(-1) \in [-12, -8] \)
\item \( \text{It is not possible to compose the two functions.} \)

\end{enumerate} }
\litem{
Choose the interval below that $f$ composed with $g$ at $x=1$ is in.\[ f(x) = -2x^{3} +2 x^{2} -2 x \text{ and } g(x) = x^{3} -2 x^{2} +x \]\begin{enumerate}[label=\Alph*.]
\item \( (f \circ g)(1) \in [-0.7, 1.9] \)
\item \( (f \circ g)(1) \in [-16.8, -11.2] \)
\item \( (f \circ g)(1) \in [-5.9, -3.1] \)
\item \( (f \circ g)(1) \in [-19.7, -15.7] \)
\item \( \text{It is not possible to compose the two functions.} \)

\end{enumerate} }
\litem{
Determine whether the function below is 1-1.\[ f(x) = 15 x^2 - 189 x + 594 \]\begin{enumerate}[label=\Alph*.]
\item \( \text{No, because there is an $x$-value that goes to 2 different $y$-values.} \)
\item \( \text{Yes, the function is 1-1.} \)
\item \( \text{No, because the range of the function is not $(-\infty, \infty)$.} \)
\item \( \text{No, because there is a $y$-value that goes to 2 different $x$-values.} \)
\item \( \text{No, because the domain of the function is not $(-\infty, \infty)$.} \)

\end{enumerate} }
\litem{
Find the inverse of the function below. Then, evaluate the inverse at $x = 8$ and choose the interval that $f^-1(8)$ belongs to.\[ f(x) = \ln{(x-2)}-5 \]\begin{enumerate}[label=\Alph*.]
\item \( f^{-1}(8) \in [442414.39, 442417.39] \)
\item \( f^{-1}(8) \in [22016.47, 22027.47] \)
\item \( f^{-1}(8) \in [15.09, 25.09] \)
\item \( f^{-1}(8) \in [396.43, 399.43] \)
\item \( f^{-1}(8) \in [442405.39, 442412.39] \)

\end{enumerate} }
\litem{
Find the inverse of the function below (if it exists). Then, evaluate the inverse at $x = 15$ and choose the interval that $f^-1(15)$ belongs to.\[ f(x) = \sqrt[3]{4 x - 3} \]\begin{enumerate}[label=\Alph*.]
\item \( f^{-1}(15) \in [843.5, 844.8] \)
\item \( f^{-1}(15) \in [-847.1, -843.8] \)
\item \( f^{-1}(15) \in [841.1, 843.1] \)
\item \( f^{-1}(15) \in [-843.1, -839.4] \)
\item \( \text{ The function is not invertible for all Real numbers. } \)

\end{enumerate} }
\litem{
Find the inverse of the function below. Then, evaluate the inverse at $x = 9$ and choose the interval that $f^-1(9)$ belongs to.\[ f(x) = e^{x+4}-3 \]\begin{enumerate}[label=\Alph*.]
\item \( f^{-1}(9) \in [-1.44, -1.23] \)
\item \( f^{-1}(9) \in [-1.58, -1.46] \)
\item \( f^{-1}(9) \in [-0.58, -0.38] \)
\item \( f^{-1}(9) \in [-1.36, -1.19] \)
\item \( f^{-1}(9) \in [6.47, 6.65] \)

\end{enumerate} }
\litem{
Find the inverse of the function below (if it exists). Then, evaluate the inverse at $x = -11$ and choose the interval that $f^-1(-11)$ belongs to.\[ f(x) = \sqrt[3]{2 x + 4} \]\begin{enumerate}[label=\Alph*.]
\item \( f^{-1}(-11) \in [-663.5, -660.5] \)
\item \( f^{-1}(-11) \in [662.5, 664.5] \)
\item \( f^{-1}(-11) \in [-674.5, -665.5] \)
\item \( f^{-1}(-11) \in [664.5, 668.5] \)
\item \( \text{ The function is not invertible for all Real numbers. } \)

\end{enumerate} }
\litem{
Multiply the following functions, then choose the domain of the resulting function from the list below.\[ f(x) = 8x^{2} + 8 \text{ and } g(x) = \sqrt{3x+15}  \]\begin{enumerate}[label=\Alph*.]
\item \( \text{ The domain is all Real numbers greater than or equal to } x = a, \text{ where } a \in [-6, -1] \)
\item \( \text{ The domain is all Real numbers except } x = a, \text{ where } a \in [0.83, 5.83] \)
\item \( \text{ The domain is all Real numbers less than or equal to } x = a, \text{ where } a \in [-0.6, 8.4] \)
\item \( \text{ The domain is all Real numbers except } x = a \text{ and } x = b, \text{ where } a \in [-9.67, -1.67] \text{ and } b \in [-3.75, 1.25] \)
\item \( \text{ The domain is all Real numbers. } \)

\end{enumerate} }
\litem{
Determine whether the function below is 1-1.\[ f(x) = (5 x - 18)^3 \]\begin{enumerate}[label=\Alph*.]
\item \( \text{Yes, the function is 1-1.} \)
\item \( \text{No, because the range of the function is not $(-\infty, \infty)$.} \)
\item \( \text{No, because there is a $y$-value that goes to 2 different $x$-values.} \)
\item \( \text{No, because the domain of the function is not $(-\infty, \infty)$.} \)
\item \( \text{No, because there is an $x$-value that goes to 2 different $y$-values.} \)

\end{enumerate} }
\litem{
Add the following functions, then choose the domain of the resulting function from the list below.\[ f(x) = 9x^{3} +8 x^{2} +6 x \text{ and } g(x) = \sqrt{-3x+10}  \]\begin{enumerate}[label=\Alph*.]
\item \( \text{ The domain is all Real numbers greater than or equal to } x = a, \text{ where } a \in [4.5, 10.5] \)
\item \( \text{ The domain is all Real numbers except } x = a, \text{ where } a \in [-8.25, 0.75] \)
\item \( \text{ The domain is all Real numbers less than or equal to } x = a, \text{ where } a \in [3.33, 4.33] \)
\item \( \text{ The domain is all Real numbers except } x = a \text{ and } x = b, \text{ where } a \in [3.75, 5.75] \text{ and } b \in [-6.2, -3.2] \)
\item \( \text{ The domain is all Real numbers. } \)

\end{enumerate} }
\end{enumerate}

\end{document}