\documentclass{extbook}[14pt]
\usepackage{multicol, enumerate, enumitem, hyperref, color, soul, setspace, parskip, fancyhdr, amssymb, amsthm, amsmath, latexsym, units, mathtools}
\everymath{\displaystyle}
\usepackage[headsep=0.5cm,headheight=0cm, left=1 in,right= 1 in,top= 1 in,bottom= 1 in]{geometry}
\usepackage{dashrule}  % Package to use the command below to create lines between items
\newcommand{\litem}[1]{\item #1

\rule{\textwidth}{0.4pt}}
\pagestyle{fancy}
\lhead{}
\chead{Answer Key for Progress Quiz 6 Version B}
\rhead{}
\lfoot{4563-7456}
\cfoot{}
\rfoot{Summer C 2021}
\begin{document}
\textbf{This key should allow you to understand why you choose the option you did (beyond just getting a question right or wrong). \href{https://xronos.clas.ufl.edu/mac1105spring2020/courseDescriptionAndMisc/Exams/LearningFromResults}{More instructions on how to use this key can be found here}.}

\textbf{If you have a suggestion to make the keys better, \href{https://forms.gle/CZkbZmPbC9XALEE88}{please fill out the short survey here}.}

\textit{Note: This key is auto-generated and may contain issues and/or errors. The keys are reviewed after each exam to ensure grading is done accurately. If there are issues (like duplicate options), they are noted in the offline gradebook. The keys are a work-in-progress to give students as many resources to improve as possible.}

\rule{\textwidth}{0.4pt}

\begin{enumerate}\litem{
Factor the polynomial below completely. Then, choose the intervals the zeros of the polynomial belong to, where $z_1 \leq z_2 \leq z_3$. \textit{To make the problem easier, all zeros are between -5 and 5.}
\[ f(x) = 10x^{3} -41 x^{2} +27 x + 18 \]The solution is \( [-0.4, 1.5, 3] \), which is option E.\begin{enumerate}[label=\Alph*.]
\item \( z_1 \in [-3.1, -2.9], \text{   }  z_2 \in [-1.7, -0.7], \text{   and   } z_3 \in [0.23, 0.73] \)

 Distractor 1: Corresponds to negatives of all zeros.
\item \( z_1 \in [-2.9, -1.5], \text{   }  z_2 \in [0.2, 0.8], \text{   and   } z_3 \in [2.77, 3.08] \)

 Distractor 2: Corresponds to inversing rational roots.
\item \( z_1 \in [-3.1, -2.9], \text{   }  z_2 \in [-1.1, 0], \text{   and   } z_3 \in [2.35, 2.69] \)

 Distractor 3: Corresponds to negatives of all zeros AND inversing rational roots.
\item \( z_1 \in [-3.1, -2.9], \text{   }  z_2 \in [-3.2, -2.7], \text{   and   } z_3 \in [0, 0.29] \)

 Distractor 4: Corresponds to moving factors from one rational to another.
\item \( z_1 \in [-1.9, 0], \text{   }  z_2 \in [0.8, 2], \text{   and   } z_3 \in [2.77, 3.08] \)

* This is the solution!
\end{enumerate}

\textbf{General Comment:} Remember to try the middle-most integers first as these normally are the zeros. Also, once you get it to a quadratic, you can use your other factoring techniques to finish factoring.
}
\litem{
Perform the division below. Then, find the intervals that correspond to the quotient in the form $ax^2+bx+c$ and remainder $r$.
\[ \frac{10x^{3} +11 x^{2} -106 x + 44}{x + 4} \]The solution is \( 10x^{2} -29 x + 10 + \frac{4}{x + 4} \), which is option E.\begin{enumerate}[label=\Alph*.]
\item \( a \in [9, 11], \text{   } b \in [-46, -31], \text{   } c \in [84, 95], \text{   and   } r \in [-401, -396]. \)

 You multiplied by the synthetic number and subtracted rather than adding during synthetic division.
\item \( a \in [-44, -39], \text{   } b \in [-152, -148], \text{   } c \in [-703, -700], \text{   and   } r \in [-2765, -2760]. \)

 You divided by the opposite of the factor AND multiplied the first factor rather than just bringing it down.
\item \( a \in [9, 11], \text{   } b \in [46, 56], \text{   } c \in [98, 100], \text{   and   } r \in [425, 441]. \)

 You divided by the opposite of the factor.
\item \( a \in [-44, -39], \text{   } b \in [168, 175], \text{   } c \in [-796, -787], \text{   and   } r \in [3196, 3209]. \)

 You multiplied by the synthetic number rather than bringing the first factor down.
\item \( a \in [9, 11], \text{   } b \in [-29, -25], \text{   } c \in [8, 15], \text{   and   } r \in [3, 7]. \)

* This is the solution!
\end{enumerate}

\textbf{General Comment:} Be sure to synthetically divide by the zero of the denominator!
}
\litem{
Perform the division below. Then, find the intervals that correspond to the quotient in the form $ax^2+bx+c$ and remainder $r$.
\[ \frac{9x^{3} -28 x -19}{x -2} \]The solution is \( 9x^{2} +18 x + 8 + \frac{-3}{x -2} \), which is option E.\begin{enumerate}[label=\Alph*.]
\item \( a \in [14, 26], b \in [-36, -34], c \in [40, 45], \text{ and } r \in [-108, -105]. \)

 You divided by the opposite of the factor AND multipled the first factor rather than just bringing it down.
\item \( a \in [6, 12], b \in [-22, -17], c \in [0, 12], \text{ and } r \in [-35, -34]. \)

 You divided by the opposite of the factor.
\item \( a \in [6, 12], b \in [6, 15], c \in [-23, -18], \text{ and } r \in [-42, -37]. \)

 You multipled by the synthetic number and subtracted rather than adding during synthetic division.
\item \( a \in [14, 26], b \in [36, 38], c \in [40, 45], \text{ and } r \in [65, 74]. \)

 You multipled by the synthetic number rather than bringing the first factor down.
\item \( a \in [6, 12], b \in [16, 20], c \in [0, 12], \text{ and } r \in [-8, 1]. \)

* This is the solution!
\end{enumerate}

\textbf{General Comment:} Be sure to synthetically divide by the zero of the denominator! Also, make sure to include 0 placeholders for missing terms.
}
\litem{
Factor the polynomial below completely. Then, choose the intervals the zeros of the polynomial belong to, where $z_1 \leq z_2 \leq z_3$. \textit{To make the problem easier, all zeros are between -5 and 5.}
\[ f(x) = 12x^{3} +35 x^{2} -9 x -18 \]The solution is \( [-3, -0.67, 0.75] \), which is option D.\begin{enumerate}[label=\Alph*.]
\item \( z_1 \in [-1.5, -1.26], \text{   }  z_2 \in [0.92, 1.85], \text{   and   } z_3 \in [2.5, 3.1] \)

 Distractor 3: Corresponds to negatives of all zeros AND inversing rational roots.
\item \( z_1 \in [-0.8, -0.64], \text{   }  z_2 \in [0.48, 0.77], \text{   and   } z_3 \in [2.5, 3.1] \)

 Distractor 1: Corresponds to negatives of all zeros.
\item \( z_1 \in [-3.17, -2.34], \text{   }  z_2 \in [-1.51, -1.46], \text{   and   } z_3 \in [1.1, 2.4] \)

 Distractor 2: Corresponds to inversing rational roots.
\item \( z_1 \in [-3.17, -2.34], \text{   }  z_2 \in [-0.74, -0.58], \text{   and   } z_3 \in [0, 1.1] \)

* This is the solution!
\item \( z_1 \in [-0.34, -0.12], \text{   }  z_2 \in [1.58, 2.09], \text{   and   } z_3 \in [2.5, 3.1] \)

 Distractor 4: Corresponds to moving factors from one rational to another.
\end{enumerate}

\textbf{General Comment:} Remember to try the middle-most integers first as these normally are the zeros. Also, once you get it to a quadratic, you can use your other factoring techniques to finish factoring.
}
\litem{
Factor the polynomial below completely, knowing that $x -4$ is a factor. Then, choose the intervals the zeros of the polynomial belong to, where $z_1 \leq z_2 \leq z_3 \leq z_4$. \textit{To make the problem easier, all zeros are between -5 and 5.}
\[ f(x) = 15x^{4} -14 x^{3} -248 x^{2} +224 x + 128 \]The solution is \( [-4, -0.4, 1.333, 4] \), which is option C.\begin{enumerate}[label=\Alph*.]
\item \( z_1 \in [-5, -2], \text{   }  z_2 \in [-4.5, -3.89], z_3 \in [-0.08, 0.26], \text{   and   } z_4 \in [2, 8] \)

 Distractor 4: Corresponds to moving factors from one rational to another.
\item \( z_1 \in [-5, -2], \text{   }  z_2 \in [-3.04, -2.45], z_3 \in [0.73, 0.85], \text{   and   } z_4 \in [2, 8] \)

 Distractor 2: Corresponds to inversing rational roots.
\item \( z_1 \in [-5, -2], \text{   }  z_2 \in [-0.52, -0.21], z_3 \in [1.24, 1.46], \text{   and   } z_4 \in [2, 8] \)

* This is the solution!
\item \( z_1 \in [-5, -2], \text{   }  z_2 \in [-1.72, -1.22], z_3 \in [0.18, 0.71], \text{   and   } z_4 \in [2, 8] \)

 Distractor 1: Corresponds to negatives of all zeros.
\item \( z_1 \in [-5, -2], \text{   }  z_2 \in [-0.94, -0.69], z_3 \in [2.44, 2.56], \text{   and   } z_4 \in [2, 8] \)

 Distractor 3: Corresponds to negatives of all zeros AND inversing rational roots.
\end{enumerate}

\textbf{General Comment:} Remember to try the middle-most integers first as these normally are the zeros. Also, once you get it to a quadratic, you can use your other factoring techniques to finish factoring.
}
\litem{
Perform the division below. Then, find the intervals that correspond to the quotient in the form $ax^2+bx+c$ and remainder $r$.
\[ \frac{9x^{3} +27 x^{2} -25 x -77}{x + 3} \]The solution is \( 9x^{2} -25 + \frac{-2}{x + 3} \), which is option A.\begin{enumerate}[label=\Alph*.]
\item \( a \in [9, 10], \text{   } b \in [-2, 2], \text{   } c \in [-27, -24], \text{   and   } r \in [-9, 0]. \)

* This is the solution!
\item \( a \in [-32, -24], \text{   } b \in [107, 111], \text{   } c \in [-350, -348], \text{   and   } r \in [968, 976]. \)

 You multiplied by the synthetic number rather than bringing the first factor down.
\item \( a \in [9, 10], \text{   } b \in [53, 60], \text{   } c \in [133, 143], \text{   and   } r \in [334, 340]. \)

 You divided by the opposite of the factor.
\item \( a \in [9, 10], \text{   } b \in [-10, -7], \text{   } c \in [11, 16], \text{   and   } r \in [-121, -118]. \)

 You multiplied by the synthetic number and subtracted rather than adding during synthetic division.
\item \( a \in [-32, -24], \text{   } b \in [-55, -51], \text{   } c \in [-188, -186], \text{   and   } r \in [-642, -633]. \)

 You divided by the opposite of the factor AND multiplied the first factor rather than just bringing it down.
\end{enumerate}

\textbf{General Comment:} Be sure to synthetically divide by the zero of the denominator!
}
\litem{
Perform the division below. Then, find the intervals that correspond to the quotient in the form $ax^2+bx+c$ and remainder $r$.
\[ \frac{10x^{3} +30 x^{2} -44}{x + 2} \]The solution is \( 10x^{2} +10 x -20 + \frac{-4}{x + 2} \), which is option D.\begin{enumerate}[label=\Alph*.]
\item \( a \in [-21, -16], b \in [69, 73], c \in [-140, -137], \text{ and } r \in [232, 243]. \)

 You multipled by the synthetic number rather than bringing the first factor down.
\item \( a \in [-21, -16], b \in [-12, -7], c \in [-21, -15], \text{ and } r \in [-88, -81]. \)

 You divided by the opposite of the factor AND multipled the first factor rather than just bringing it down.
\item \( a \in [6, 11], b \in [46, 51], c \in [97, 101], \text{ and } r \in [153, 159]. \)

 You divided by the opposite of the factor.
\item \( a \in [6, 11], b \in [9, 17], c \in [-21, -15], \text{ and } r \in [-4, -3]. \)

* This is the solution!
\item \( a \in [6, 11], b \in [-5, 6], c \in [-2, 3], \text{ and } r \in [-50, -42]. \)

 You multipled by the synthetic number and subtracted rather than adding during synthetic division.
\end{enumerate}

\textbf{General Comment:} Be sure to synthetically divide by the zero of the denominator! Also, make sure to include 0 placeholders for missing terms.
}
\litem{
What are the \textit{possible Rational} roots of the polynomial below?
\[ f(x) = 3x^{3} +2 x^{2} +4 x + 7 \]The solution is \( \text{ All combinations of: }\frac{\pm 1,\pm 7}{\pm 1,\pm 3} \), which is option C.\begin{enumerate}[label=\Alph*.]
\item \( \text{ All combinations of: }\frac{\pm 1,\pm 3}{\pm 1,\pm 7} \)

 Distractor 3: Corresponds to the plus or minus of the inverse quotient (an/a0) of the factors. 
\item \( \pm 1,\pm 3 \)

 Distractor 1: Corresponds to the plus or minus factors of a1 only.
\item \( \text{ All combinations of: }\frac{\pm 1,\pm 7}{\pm 1,\pm 3} \)

* This is the solution \textbf{since we asked for the possible Rational roots}!
\item \( \pm 1,\pm 7 \)

This would have been the solution \textbf{if asked for the possible Integer roots}!
\item \( \text{ There is no formula or theorem that tells us all possible Rational roots.} \)

 Distractor 4: Corresponds to not recalling the theorem for rational roots of a polynomial.
\end{enumerate}

\textbf{General Comment:} We have a way to find the possible Rational roots. The possible Integer roots are the Integers in this list.
}
\litem{
Factor the polynomial below completely, knowing that $x + 3$ is a factor. Then, choose the intervals the zeros of the polynomial belong to, where $z_1 \leq z_2 \leq z_3 \leq z_4$. \textit{To make the problem easier, all zeros are between -5 and 5.}
\[ f(x) = 10x^{4} +51 x^{3} -28 x^{2} -333 x -180 \]The solution is \( [-4, -3, -0.6, 2.5] \), which is option B.\begin{enumerate}[label=\Alph*.]
\item \( z_1 \in [-0.74, -0.5], \text{   }  z_2 \in [2.94, 3.06], z_3 \in [1.8, 3.3], \text{   and   } z_4 \in [4, 5] \)

 Distractor 4: Corresponds to moving factors from one rational to another.
\item \( z_1 \in [-4.18, -3.74], \text{   }  z_2 \in [-3.13, -2.89], z_3 \in [-0.8, -0.4], \text{   and   } z_4 \in [2.5, 3.5] \)

* This is the solution!
\item \( z_1 \in [-0.47, -0.25], \text{   }  z_2 \in [1.43, 2.14], z_3 \in [1.8, 3.3], \text{   and   } z_4 \in [4, 5] \)

 Distractor 3: Corresponds to negatives of all zeros AND inversing rational roots.
\item \( z_1 \in [-2.55, -2.47], \text{   }  z_2 \in [-0.59, 1.11], z_3 \in [1.8, 3.3], \text{   and   } z_4 \in [4, 5] \)

 Distractor 1: Corresponds to negatives of all zeros.
\item \( z_1 \in [-4.18, -3.74], \text{   }  z_2 \in [-3.13, -2.89], z_3 \in [-2.8, -1.2], \text{   and   } z_4 \in [0.4, 1.4] \)

 Distractor 2: Corresponds to inversing rational roots.
\end{enumerate}

\textbf{General Comment:} Remember to try the middle-most integers first as these normally are the zeros. Also, once you get it to a quadratic, you can use your other factoring techniques to finish factoring.
}
\litem{
What are the \textit{possible Rational} roots of the polynomial below?
\[ f(x) = 4x^{3} +4 x^{2} +6 x + 6 \]The solution is \( \text{ All combinations of: }\frac{\pm 1,\pm 2,\pm 3,\pm 6}{\pm 1,\pm 2,\pm 4} \), which is option D.\begin{enumerate}[label=\Alph*.]
\item \( \pm 1,\pm 2,\pm 3,\pm 6 \)

This would have been the solution \textbf{if asked for the possible Integer roots}!
\item \( \text{ All combinations of: }\frac{\pm 1,\pm 2,\pm 4}{\pm 1,\pm 2,\pm 3,\pm 6} \)

 Distractor 3: Corresponds to the plus or minus of the inverse quotient (an/a0) of the factors. 
\item \( \pm 1,\pm 2,\pm 4 \)

 Distractor 1: Corresponds to the plus or minus factors of a1 only.
\item \( \text{ All combinations of: }\frac{\pm 1,\pm 2,\pm 3,\pm 6}{\pm 1,\pm 2,\pm 4} \)

* This is the solution \textbf{since we asked for the possible Rational roots}!
\item \( \text{ There is no formula or theorem that tells us all possible Rational roots.} \)

 Distractor 4: Corresponds to not recalling the theorem for rational roots of a polynomial.
\end{enumerate}

\textbf{General Comment:} We have a way to find the possible Rational roots. The possible Integer roots are the Integers in this list.
}
\end{enumerate}

\end{document}