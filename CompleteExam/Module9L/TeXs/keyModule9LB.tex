\documentclass{extbook}[14pt]
\usepackage{multicol, enumerate, enumitem, hyperref, color, soul, setspace, parskip, fancyhdr, amssymb, amsthm, amsmath, bbm, latexsym, units, mathtools}
\everymath{\displaystyle}
\usepackage[headsep=0.5cm,headheight=0cm, left=1 in,right= 1 in,top= 1 in,bottom= 1 in]{geometry}
\usepackage{dashrule}  % Package to use the command below to create lines between items
\newcommand{\litem}[1]{\item #1

\rule{\textwidth}{0.4pt}}
\pagestyle{fancy}
\lhead{}
\chead{Answer Key for Module9L Version B}
\rhead{}
\lfoot{3286-3678}
\cfoot{}
\rfoot{testing}
\begin{document}
\textbf{This key should allow you to understand why you choose the option you did (beyond just getting a question right or wrong). \href{https://xronos.clas.ufl.edu/mac1105spring2020/courseDescriptionAndMisc/Exams/LearningFromResults}{More instructions on how to use this key can be found here}.}

\textbf{If you have a suggestion to make the keys better, \href{https://forms.gle/CZkbZmPbC9XALEE88}{please fill out the short survey here}.}

\textit{Note: This key is auto-generated and may contain issues and/or errors. The keys are reviewed after each exam to ensure grading is done accurately. If there are issues (like duplicate options), they are noted in the offline gradebook. The keys are a work-in-progress to give students as many resources to improve as possible.}

\rule{\textwidth}{0.4pt}

\begin{enumerate}\litem{
Determine whether the function below is 1-1.
\[ f(x) = 9 x^2 + 15 x - 456 \]The solution is \( \text{no} \), which is option A.\begin{enumerate}[label=\Alph*.]
\item \( \text{No, because there is a $y$-value that goes to 2 different $x$-values.} \)

* This is the solution.
\item \( \text{No, because there is an $x$-value that goes to 2 different $y$-values.} \)

Corresponds to the Vertical Line test, which checks if an expression is a function.
\item \( \text{Yes, the function is 1-1.} \)

Corresponds to believing the function passes the Horizontal Line test.
\item \( \text{No, because the domain of the function is not $(-\infty, \infty)$.} \)

Corresponds to believing 1-1 means the domain is all Real numbers.
\item \( \text{No, because the range of the function is not $(-\infty, \infty)$.} \)

Corresponds to believing 1-1 means the range is all Real numbers.
\end{enumerate}

\textbf{General Comment:} There are only two valid options: The function is 1-1 OR No because there is a $y$-value that goes to 2 different $x$-values.
}
\litem{
Choose the interval below that $f$ composed with $g$ at $x=-1$ is in.
\[ f(x) = x^{3} -4 x^{2} +x \text{ and } g(x) = x^{3} +4 x^{2} -x \]The solution is \( 4.0 \), which is option A.\begin{enumerate}[label=\Alph*.]
\item \( (f \circ g)(-1) \in [2, 8] \)

* This is the correct solution
\item \( (f \circ g)(-1) \in [-67, -61] \)

 Distractor 1: Corresponds to reversing the composition.
\item \( (f \circ g)(-1) \in [-3, 1] \)

 Distractor 2: Corresponds to being slightly off from the solution.
\item \( (f \circ g)(-1) \in [-83, -73] \)

 Distractor 3: Corresponds to being slightly off from the solution.
\item \( \text{It is not possible to compose the two functions.} \)


\end{enumerate}

\textbf{General Comment:} $f$ composed with $g$ at $x$ means $f(g(x))$. The order matters!
}
\litem{
Subtract the following functions, then choose the domain of the resulting function from the list below.
\[ f(x) = x^{4} +5 x^{3} +5 x^{2} + 2 \text{ and } g(x) = \frac{5}{4x+15} \]The solution is \( \text{ The domain is all Real numbers except } x = -3.75 \), which is option A.\begin{enumerate}[label=\Alph*.]
\item \( \text{ The domain is all Real numbers except } x = a, \text{ where } a \in [-3.75, 0.25] \)


\item \( \text{ The domain is all Real numbers greater than or equal to } x = a, \text{ where } a \in [-9.25, -2.25] \)


\item \( \text{ The domain is all Real numbers less than or equal to } x = a, \text{ where } a \in [-2, 0] \)


\item \( \text{ The domain is all Real numbers except } x = a \text{ and } x = b, \text{ where } a \in [-8.33, 1.67] \text{ and } b \in [6.2, 7.2] \)


\item \( \text{ The domain is all Real numbers. } \)


\end{enumerate}

\textbf{General Comment:} The new domain is the intersection of the previous domains.
}
\litem{
Find the inverse of the function below. Then, evaluate the inverse at $x = 9$ and choose the interval that $f^{-1}(9)$ belongs to.
\[ f(x) = e^{x-4}-3 \]The solution is \( f^{-1}(9) = 6.485 \), which is option E.\begin{enumerate}[label=\Alph*.]
\item \( f^{-1}(9) \in [-1.36, -1.19] \)

 This solution corresponds to distractor 2.
\item \( f^{-1}(9) \in [-0.61, -0.43] \)

 This solution corresponds to distractor 3.
\item \( f^{-1}(9) \in [-1.43, -1.22] \)

 This solution corresponds to distractor 4.
\item \( f^{-1}(9) \in [-1.76, -1.5] \)

 This solution corresponds to distractor 1.
\item \( f^{-1}(9) \in [6.36, 6.51] \)

 This is the solution.
\end{enumerate}

\textbf{General Comment:} Natural log and exponential functions always have an inverse. Once you switch the $x$ and $y$, use the conversion $ e^y = x \leftrightarrow y=\ln(x)$.
}
\litem{
Find the inverse of the function below (if it exists). Then, evaluate the inverse at $x = -11$ and choose the interval the $f^{-1}(-11)$ belongs to.
\[ f(x) = \sqrt[3]{2 x - 3} \]The solution is \( -664.0 \), which is option C.\begin{enumerate}[label=\Alph*.]
\item \( f^{-1}(-11) \in [663.1, 665.4] \)

 This solution corresponds to distractor 2.
\item \( f^{-1}(-11) \in [-668.4, -665.4] \)

 Distractor 1: This corresponds to 
\item \( f^{-1}(-11) \in [-665.8, -660.6] \)

* This is the correct solution.
\item \( f^{-1}(-11) \in [666.8, 669.9] \)

 This solution corresponds to distractor 3.
\item \( \text{ The function is not invertible for all Real numbers. } \)

 This solution corresponds to distractor 4.
\end{enumerate}

\textbf{General Comment:} Be sure you check that the function is 1-1 before trying to find the inverse!
}
\litem{
Add the following functions, then choose the domain of the resulting function from the list below.
\[ f(x) = \frac{5}{3x-16} \text{ and } g(x) = \frac{3}{4x+21} \]The solution is \( \text{ The domain is all Real numbers except } x = 5.333333333333333 \text{ and } x = -5.25 \), which is option D.\begin{enumerate}[label=\Alph*.]
\item \( \text{ The domain is all Real numbers except } x = a, \text{ where } a \in [4.33, 12.33] \)


\item \( \text{ The domain is all Real numbers less than or equal to } x = a, \text{ where } a \in [-5.67, 2.33] \)


\item \( \text{ The domain is all Real numbers greater than or equal to } x = a, \text{ where } a \in [3.75, 7.75] \)


\item \( \text{ The domain is all Real numbers except } x = a \text{ and } x = b, \text{ where } a \in [4.33, 7.33] \text{ and } b \in [-9.25, -3.25] \)


\item \( \text{ The domain is all Real numbers. } \)


\end{enumerate}

\textbf{General Comment:} The new domain is the intersection of the previous domains.
}
\litem{
Choose the interval below that $f$ composed with $g$ at $x=-1$ is in.
\[ f(x) = -2x^{3} +2 x^{2} +4 x + 3 \text{ and } g(x) = 2x^{3} + x^{2} -4 x + 1 \]The solution is \( -77.0 \), which is option A.\begin{enumerate}[label=\Alph*.]
\item \( (f \circ g)(-1) \in [-79, -74] \)

* This is the correct solution
\item \( (f \circ g)(-1) \in [-68, -67] \)

 Distractor 2: Corresponds to being slightly off from the solution.
\item \( (f \circ g)(-1) \in [51, 55] \)

 Distractor 1: Corresponds to reversing the composition.
\item \( (f \circ g)(-1) \in [39, 48] \)

 Distractor 3: Corresponds to being slightly off from the solution.
\item \( \text{It is not possible to compose the two functions.} \)


\end{enumerate}

\textbf{General Comment:} $f$ composed with $g$ at $x$ means $f(g(x))$. The order matters!
}
\litem{
Find the inverse of the function below (if it exists). Then, evaluate the inverse at $x = 10$ and choose the interval the $f^{-1}(10)$ belongs to.
\[ f(x) = \sqrt[3]{3 x + 2} \]The solution is \( 332.6666666666667 \), which is option B.\begin{enumerate}[label=\Alph*.]
\item \( f^{-1}(10) \in [-334.41, -333.58] \)

 This solution corresponds to distractor 3.
\item \( f^{-1}(10) \in [332.26, 333.46] \)

* This is the correct solution.
\item \( f^{-1}(10) \in [333.98, 334.06] \)

 Distractor 1: This corresponds to 
\item \( f^{-1}(10) \in [-333.03, -332.01] \)

 This solution corresponds to distractor 2.
\item \( \text{ The function is not invertible for all Real numbers. } \)

 This solution corresponds to distractor 4.
\end{enumerate}

\textbf{General Comment:} Be sure you check that the function is 1-1 before trying to find the inverse!
}
\litem{
Find the inverse of the function below. Then, evaluate the inverse at $x = 9$ and choose the interval that $f^{-1}(9)$ belongs to.
\[ f(x) = e^{x-2}+5 \]The solution is \( f^{-1}(9) = 3.386 \), which is option A.\begin{enumerate}[label=\Alph*.]
\item \( f^{-1}(9) \in [3.15, 3.4] \)

 This is the solution.
\item \( f^{-1}(9) \in [7.34, 7.43] \)

 This solution corresponds to distractor 3.
\item \( f^{-1}(9) \in [7.59, 7.66] \)

 This solution corresponds to distractor 2.
\item \( f^{-1}(9) \in [6.77, 7.03] \)

 This solution corresponds to distractor 4.
\item \( f^{-1}(9) \in [-0.78, -0.44] \)

 This solution corresponds to distractor 1.
\end{enumerate}

\textbf{General Comment:} Natural log and exponential functions always have an inverse. Once you switch the $x$ and $y$, use the conversion $ e^y = x \leftrightarrow y=\ln(x)$.
}
\litem{
Determine whether the function below is 1-1.
\[ f(x) = 36 x^2 - 348 x + 841 \]The solution is \( \text{no} \), which is option A.\begin{enumerate}[label=\Alph*.]
\item \( \text{No, because there is a $y$-value that goes to 2 different $x$-values.} \)

* This is the solution.
\item \( \text{No, because the domain of the function is not $(-\infty, \infty)$.} \)

Corresponds to believing 1-1 means the domain is all Real numbers.
\item \( \text{No, because there is an $x$-value that goes to 2 different $y$-values.} \)

Corresponds to the Vertical Line test, which checks if an expression is a function.
\item \( \text{Yes, the function is 1-1.} \)

Corresponds to believing the function passes the Horizontal Line test.
\item \( \text{No, because the range of the function is not $(-\infty, \infty)$.} \)

Corresponds to believing 1-1 means the range is all Real numbers.
\end{enumerate}

\textbf{General Comment:} There are only two valid options: The function is 1-1 OR No because there is a $y$-value that goes to 2 different $x$-values.
}
\end{enumerate}

\end{document}