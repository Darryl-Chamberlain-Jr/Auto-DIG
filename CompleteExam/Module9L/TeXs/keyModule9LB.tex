\documentclass{extbook}[14pt]
\usepackage{multicol, enumerate, enumitem, hyperref, color, soul, setspace, parskip, fancyhdr, amssymb, amsthm, amsmath, bbm, latexsym, units, mathtools}
\everymath{\displaystyle}
\usepackage[headsep=0.5cm,headheight=0cm, left=1 in,right= 1 in,top= 1 in,bottom= 1 in]{geometry}
\usepackage{dashrule}  % Package to use the command below to create lines between items
\newcommand{\litem}[1]{\item #1

\rule{\textwidth}{0.4pt}}
\pagestyle{fancy}
\lhead{}
\chead{Answer Key for Module9L Version B}
\rhead{}
\lfoot{2346-2618}
\cfoot{}
\rfoot{Fall 2020}
\begin{document}
\textbf{This key should allow you to understand why you choose the option you did (beyond just getting a question right or wrong). \href{https://xronos.clas.ufl.edu/mac1105spring2020/courseDescriptionAndMisc/Exams/LearningFromResults}{More instructions on how to use this key can be found here}.}

\textbf{If you have a suggestion to make the keys better, \href{https://forms.gle/CZkbZmPbC9XALEE88}{please fill out the short survey here}.}

\textit{Note: This key is auto-generated and may contain issues and/or errors. The keys are reviewed after each exam to ensure grading is done accurately. If there are issues (like duplicate options), they are noted in the offline gradebook. The keys are a work-in-progress to give students as many resources to improve as possible.}

\rule{\textwidth}{0.4pt}

\begin{enumerate}\litem{
Add the following functions, then choose the domain of the resulting function from the list below.
\[ f(x) = 7x^{2} +3 x + 1 \text{ and } g(x) = 6x + 4 \]
The solution is \( (-\infty, \infty) \), which is option E.\begin{enumerate}[label=\Alph*.]
\item \( \text{ The domain is all Real numbers except } x = a, \text{ where } a \in [2.6, 10.6] \)


\item \( \text{ The domain is all Real numbers less than or equal to } x = a, \text{ where } a \in [-8.6, 0.4] \)


\item \( \text{ The domain is all Real numbers greater than or equal to } x = a, \text{ where } a \in [-7, -4] \)


\item \( \text{ The domain is all Real numbers except } x = a \text{ and } x = b, \text{ where } a \in [-9.83, -4.83] \text{ and } b \in [-6.17, 1.83] \)


\item \( \text{ The domain is all Real numbers. } \)


\end{enumerate}

\textbf{General Comment:} The new domain is the intersection of the previous domains.
}
\litem{
Find the inverse of the function below. Then, evaluate the inverse at $x = 8$ and choose the interval that $f^{-1}(8)$ belongs to.
\[ f(x) = e^{x+2}-5 \]
The solution is \( f^{-1}(8) = 0.565 \), which is option C.\begin{enumerate}[label=\Alph*.]
\item \( f^{-1}(8) \in [3.96, 4.92] \)

 This solution corresponds to distractor 1.
\item \( f^{-1}(8) \in [-4.14, -3.73] \)

 This solution corresponds to distractor 2.
\item \( f^{-1}(8) \in [0.09, 0.79] \)

 This is the solution.
\item \( f^{-1}(8) \in [-3.63, -2.86] \)

 This solution corresponds to distractor 3.
\item \( f^{-1}(8) \in [-3.11, -2.22] \)

 This solution corresponds to distractor 4.
\end{enumerate}

\textbf{General Comment:} Natural log and exponential functions always have an inverse. Once you switch the $x$ and $y$, use the conversion $ e^y = x \leftrightarrow y=\ln(x)$.
}
\litem{
Choose the interval below that $f$ composed with $g$ at $x=1$ is in.
\[ f(x) = -2x^{3} -1 x^{2} +4 x -4 \text{ and } g(x) = -x^{3} +3 x^{2} -x -3 \]
The solution is \( 0.0 \), which is option B.\begin{enumerate}[label=\Alph*.]
\item \( (f \circ g)(1) \in [53, 61] \)

 Distractor 1: Corresponds to reversing the composition.
\item \( (f \circ g)(1) \in [0, 12] \)

* This is the correct solution
\item \( (f \circ g)(1) \in [40, 46] \)

 Distractor 3: Corresponds to being slightly off from the solution.
\item \( (f \circ g)(1) \in [-11, -7] \)

 Distractor 2: Corresponds to being slightly off from the solution.
\item \( \text{It is not possible to compose the two functions.} \)


\end{enumerate}

\textbf{General Comment:} $f$ composed with $g$ at $x$ means $f(g(x))$. The order matters!
}
\litem{
Multiply the following functions, then choose the domain of the resulting function from the list below.
\[ f(x) = \sqrt{6x-22}  \text{ and } g(x) = 3x^{4} +3 x^{3} +5 x^{2} +3 x + 3 \]
The solution is \( \text{ The domain is all Real numbers greater than or equal to} x = 3.6666666666666665. \), which is option A.\begin{enumerate}[label=\Alph*.]
\item \( \text{ The domain is all Real numbers greater than or equal to } x = a, \text{ where } a \in [-2.33, 6.67] \)


\item \( \text{ The domain is all Real numbers except } x = a, \text{ where } a \in [3.2, 12.2] \)


\item \( \text{ The domain is all Real numbers less than or equal to } x = a, \text{ where } a \in [4.83, 11.83] \)


\item \( \text{ The domain is all Real numbers except } x = a \text{ and } x = b, \text{ where } a \in [-0.17, 8.83] \text{ and } b \in [1.6, 6.6] \)


\item \( \text{ The domain is all Real numbers. } \)


\end{enumerate}

\textbf{General Comment:} The new domain is the intersection of the previous domains.
}
\litem{
Find the inverse of the function below. Then, evaluate the inverse at $x = 7$ and choose the interval that $f^{-1}(7)$ belongs to.
\[ f(x) = e^{x+3}-2 \]
The solution is \( f^{-1}(7) = -0.803 \), which is option E.\begin{enumerate}[label=\Alph*.]
\item \( f^{-1}(7) \in [4.72, 5.49] \)

 This solution corresponds to distractor 1.
\item \( f^{-1}(7) \in [0.18, 0.6] \)

 This solution corresponds to distractor 4.
\item \( f^{-1}(7) \in [-0.62, -0.49] \)

 This solution corresponds to distractor 3.
\item \( f^{-1}(7) \in [-0.47, -0.2] \)

 This solution corresponds to distractor 2.
\item \( f^{-1}(7) \in [-1.09, -0.71] \)

 This is the solution.
\end{enumerate}

\textbf{General Comment:} Natural log and exponential functions always have an inverse. Once you switch the $x$ and $y$, use the conversion $ e^y = x \leftrightarrow y=\ln(x)$.
}
\litem{
Choose the interval below that $f$ composed with $g$ at $x=1$ is in.
\[ f(x) = 2x^{3} -4 x^{2} +x \text{ and } g(x) = 4x^{3} -2 x^{2} +x \]
The solution is \( 21.0 \), which is option A.\begin{enumerate}[label=\Alph*.]
\item \( (f \circ g)(1) \in [20.5, 25.1] \)

* This is the correct solution
\item \( (f \circ g)(1) \in [24.8, 28.5] \)

 Distractor 2: Corresponds to being slightly off from the solution.
\item \( (f \circ g)(1) \in [-14.9, -11.7] \)

 Distractor 3: Corresponds to being slightly off from the solution.
\item \( (f \circ g)(1) \in [-9.6, -6.6] \)

 Distractor 1: Corresponds to reversing the composition.
\item \( \text{It is not possible to compose the two functions.} \)


\end{enumerate}

\textbf{General Comment:} $f$ composed with $g$ at $x$ means $f(g(x))$. The order matters!
}
\litem{
Find the inverse of the function below (if it exists). Then, evaluate the inverse at $x = -15$ and choose the interval the $f^{-1}(-15)$ belongs to.
\[ f(x) = \sqrt[3]{5 x + 4} \]
The solution is \( -675.8 \), which is option C.\begin{enumerate}[label=\Alph*.]
\item \( f^{-1}(-15) \in [675.15, 675.9] \)

 This solution corresponds to distractor 2.
\item \( f^{-1}(-15) \in [674.07, 674.56] \)

 This solution corresponds to distractor 3.
\item \( f^{-1}(-15) \in [-676.04, -675.8] \)

* This is the correct solution.
\item \( f^{-1}(-15) \in [-674.37, -673.71] \)

 Distractor 1: This corresponds to 
\item \( \text{ The function is not invertible for all Real numbers. } \)

 This solution corresponds to distractor 4.
\end{enumerate}

\textbf{General Comment:} Be sure you check that the function is 1-1 before trying to find the inverse!
}
\litem{
Determine whether the function below is 1-1.
\[ f(x) = (6 x - 29)^3 \]
The solution is \( \text{yes} \), which is option D.\begin{enumerate}[label=\Alph*.]
\item \( \text{No, because the domain of the function is not $(-\infty, \infty)$.} \)

Corresponds to believing 1-1 means the domain is all Real numbers.
\item \( \text{No, because there is an $x$-value that goes to 2 different $y$-values.} \)

Corresponds to the Vertical Line test, which checks if an expression is a function.
\item \( \text{No, because the range of the function is not $(-\infty, \infty)$.} \)

Corresponds to believing 1-1 means the range is all Real numbers.
\item \( \text{Yes, the function is 1-1.} \)

* This is the solution.
\item \( \text{No, because there is a $y$-value that goes to 2 different $x$-values.} \)

Corresponds to the Horizontal Line test, which this function passes.
\end{enumerate}

\textbf{General Comment:} There are only two valid options: The function is 1-1 OR No because there is a $y$-value that goes to 2 different $x$-values.
}
\litem{
Find the inverse of the function below (if it exists). Then, evaluate the inverse at $x = 10$ and choose the interval that $f^{-1}(10)$ belongs to.
\[ f(x) = 2 x^2 + 3 \]
The solution is \( \text{ The function is not invertible for all Real numbers. } \), which is option E.\begin{enumerate}[label=\Alph*.]
\item \( f^{-1}(10) \in [2.77, 3.36] \)

 Distractor 3: This corresponds to finding the (nonexistent) inverse and dividing by a negative.
\item \( f^{-1}(10) \in [1.73, 2.01] \)

 Distractor 1: This corresponds to trying to find the inverse even though the function is not 1-1. 
\item \( f^{-1}(10) \in [2.51, 2.65] \)

 Distractor 2: This corresponds to finding the (nonexistent) inverse and not subtracting by the vertical shift.
\item \( f^{-1}(10) \in [4.62, 4.91] \)

 Distractor 4: This corresponds to both distractors 2 and 3.
\item \( \text{ The function is not invertible for all Real numbers. } \)

* This is the correct option.
\end{enumerate}

\textbf{General Comment:} Be sure you check that the function is 1-1 before trying to find the inverse!
}
\litem{
Determine whether the function below is 1-1.
\[ f(x) = 20 x^2 + 14 x - 528 \]
The solution is \( \text{no} \), which is option D.\begin{enumerate}[label=\Alph*.]
\item \( \text{No, because there is an $x$-value that goes to 2 different $y$-values.} \)

Corresponds to the Vertical Line test, which checks if an expression is a function.
\item \( \text{No, because the range of the function is not $(-\infty, \infty)$.} \)

Corresponds to believing 1-1 means the range is all Real numbers.
\item \( \text{Yes, the function is 1-1.} \)

Corresponds to believing the function passes the Horizontal Line test.
\item \( \text{No, because there is a $y$-value that goes to 2 different $x$-values.} \)

* This is the solution.
\item \( \text{No, because the domain of the function is not $(-\infty, \infty)$.} \)

Corresponds to believing 1-1 means the domain is all Real numbers.
\end{enumerate}

\textbf{General Comment:} There are only two valid options: The function is 1-1 OR No because there is a $y$-value that goes to 2 different $x$-values.
}
\end{enumerate}

\end{document}