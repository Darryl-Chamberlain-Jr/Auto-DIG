\documentclass[14pt]{extbook}
\usepackage{multicol, enumerate, enumitem, hyperref, color, soul, setspace, parskip, fancyhdr} %General Packages
\usepackage{amssymb, amsthm, amsmath, bbm, latexsym, units, mathtools} %Math Packages
\everymath{\displaystyle} %All math in Display Style
% Packages with additional options
\usepackage[headsep=0.5cm,headheight=12pt, left=1 in,right= 1 in,top= 1 in,bottom= 1 in]{geometry}
\usepackage[usenames,dvipsnames]{xcolor}
\usepackage{dashrule}  % Package to use the command below to create lines between items
\newcommand{\litem}[1]{\item#1\hspace*{-1cm}\rule{\textwidth}{0.4pt}}
\pagestyle{fancy}
\lhead{Module9L}
\chead{}
\rhead{Version A}
\lfoot{3286-3678}
\cfoot{}
\rfoot{testing}
\begin{document}

\begin{enumerate}
\litem{
Determine whether the function below is 1-1.\[ f(x) = (6 x - 42)^3 \]\begin{enumerate}[label=\Alph*.]
\item \( \text{No, because there is an $x$-value that goes to 2 different $y$-values.} \)
\item \( \text{No, because the domain of the function is not $(-\infty, \infty)$.} \)
\item \( \text{No, because there is a $y$-value that goes to 2 different $x$-values.} \)
\item \( \text{No, because the range of the function is not $(-\infty, \infty)$.} \)
\item \( \text{Yes, the function is 1-1.} \)

\end{enumerate} }
\litem{
Choose the interval below that $f$ composed with $g$ at $x=-1$ is in.\[ f(x) = -x^{3} -3 x^{2} +2 x + 2 \text{ and } g(x) = x^{3} -4 x^{2} -4 x + 1 \]\begin{enumerate}[label=\Alph*.]
\item \( (f \circ g)(-1) \in [0.5, 3.6] \)
\item \( (f \circ g)(-1) \in [-23.9, -17.1] \)
\item \( (f \circ g)(-1) \in [-4.7, -2.3] \)
\item \( (f \circ g)(-1) \in [-15.4, -13.6] \)
\item \( \text{It is not possible to compose the two functions.} \)

\end{enumerate} }
\litem{
Add the following functions, then choose the domain of the resulting function from the list below.\[ f(x) = \sqrt{-3x+14}  \text{ and } g(x) = 5x^{4} +6 x^{2} +6 x + 7 \]\begin{enumerate}[label=\Alph*.]
\item \( \text{ The domain is all Real numbers greater than or equal to } x = a, \text{ where } a \in [-6.8, -1.8] \)
\item \( \text{ The domain is all Real numbers less than or equal to } x = a, \text{ where } a \in [1.67, 13.67] \)
\item \( \text{ The domain is all Real numbers except } x = a, \text{ where } a \in [0.75, 8.75] \)
\item \( \text{ The domain is all Real numbers except } x = a \text{ and } x = b, \text{ where } a \in [-12.67, -1.67] \text{ and } b \in [-9.25, -3.25] \)
\item \( \text{ The domain is all Real numbers. } \)

\end{enumerate} }
\litem{
Find the inverse of the function below. Then, evaluate the inverse at $x = 8$ and choose the interval that $f^{-1}(8)$ belongs to.\[ f(x) = e^{x+5}+3 \]\begin{enumerate}[label=\Alph*.]
\item \( f^{-1}(8) \in [6.59, 6.73] \)
\item \( f^{-1}(8) \in [5.22, 5.5] \)
\item \( f^{-1}(8) \in [5.46, 5.64] \)
\item \( f^{-1}(8) \in [3.88, 4.13] \)
\item \( f^{-1}(8) \in [-3.47, -3.25] \)

\end{enumerate} }
\litem{
Find the inverse of the function below (if it exists). Then, evaluate the inverse at $x = -15$ and choose the interval that $f^{-1}(-15)$ belongs to.\[ f(x) = 2 x^2 + 4 \]\begin{enumerate}[label=\Alph*.]
\item \( f^{-1}(-15) \in [2.52, 3.2] \)
\item \( f^{-1}(-15) \in [5.44, 6.22] \)
\item \( f^{-1}(-15) \in [3.96, 4.25] \)
\item \( f^{-1}(-15) \in [2.07, 3.05] \)
\item \( \text{ The function is not invertible for all Real numbers. } \)

\end{enumerate} }
\litem{
Subtract the following functions, then choose the domain of the resulting function from the list below.\[ f(x) = \frac{5}{4x+15} \text{ and } g(x) = \frac{2}{3x-16} \]\begin{enumerate}[label=\Alph*.]
\item \( \text{ The domain is all Real numbers except } x = a, \text{ where } a \in [-0.8, 7.2] \)
\item \( \text{ The domain is all Real numbers less than or equal to } x = a, \text{ where } a \in [-5.33, 4.67] \)
\item \( \text{ The domain is all Real numbers greater than or equal to } x = a, \text{ where } a \in [-6.2, -4.2] \)
\item \( \text{ The domain is all Real numbers except } x = a \text{ and } x = b, \text{ where } a \in [-6.75, 5.25] \text{ and } b \in [1.33, 8.33] \)
\item \( \text{ The domain is all Real numbers. } \)

\end{enumerate} }
\litem{
Choose the interval below that $f$ composed with $g$ at $x=1$ is in.\[ f(x) = 4x^{3} + x^{2} -3 x + 2 \text{ and } g(x) = -2x^{3} +3 x^{2} -x \]\begin{enumerate}[label=\Alph*.]
\item \( (f \circ g)(1) \in [-79, -78] \)
\item \( (f \circ g)(1) \in [8, 16] \)
\item \( (f \circ g)(1) \in [-86, -83] \)
\item \( (f \circ g)(1) \in [-1, 6] \)
\item \( \text{It is not possible to compose the two functions.} \)

\end{enumerate} }
\litem{
Find the inverse of the function below (if it exists). Then, evaluate the inverse at $x = -10$ and choose the interval the $f^{-1}(-10)$ belongs to.\[ f(x) = \sqrt[3]{5 x - 3} \]\begin{enumerate}[label=\Alph*.]
\item \( f^{-1}(-10) \in [199.05, 200.26] \)
\item \( f^{-1}(-10) \in [-199.96, -198.03] \)
\item \( f^{-1}(-10) \in [-201.02, -199.92] \)
\item \( f^{-1}(-10) \in [200.39, 201.01] \)
\item \( \text{ The function is not invertible for all Real numbers. } \)

\end{enumerate} }
\litem{
Find the inverse of the function below. Then, evaluate the inverse at $x = 7$ and choose the interval that $f^{-1}(7)$ belongs to.\[ f(x) = \ln{(x+2)}+4 \]\begin{enumerate}[label=\Alph*.]
\item \( f^{-1}(7) \in [150.6, 155.8] \)
\item \( f^{-1}(7) \in [59870.8, 59876.5] \)
\item \( f^{-1}(7) \in [17.6, 20] \)
\item \( f^{-1}(7) \in [8106.5, 8108] \)
\item \( f^{-1}(7) \in [20.6, 23.4] \)

\end{enumerate} }
\litem{
Determine whether the function below is 1-1.\[ f(x) = \sqrt{3 x + 17} \]\begin{enumerate}[label=\Alph*.]
\item \( \text{No, because there is an $x$-value that goes to 2 different $y$-values.} \)
\item \( \text{No, because the range of the function is not $(-\infty, \infty)$.} \)
\item \( \text{No, because there is a $y$-value that goes to 2 different $x$-values.} \)
\item \( \text{No, because the domain of the function is not $(-\infty, \infty)$.} \)
\item \( \text{Yes, the function is 1-1.} \)

\end{enumerate} }
\end{enumerate}

\end{document}