\documentclass[14pt]{extbook}
\usepackage{multicol, enumerate, enumitem, hyperref, color, soul, setspace, parskip, fancyhdr} %General Packages
\usepackage{amssymb, amsthm, amsmath, bbm, latexsym, units, mathtools} %Math Packages
\everymath{\displaystyle} %All math in Display Style
% Packages with additional options
\usepackage[headsep=0.5cm,headheight=12pt, left=1 in,right= 1 in,top= 1 in,bottom= 1 in]{geometry}
\usepackage[usenames,dvipsnames]{xcolor}
\usepackage{dashrule}  % Package to use the command below to create lines between items
\newcommand{\litem}[1]{\item#1\hspace*{-1cm}\rule{\textwidth}{0.4pt}}
\pagestyle{fancy}
\lhead{Module9L}
\chead{}
\rhead{Version A}
\lfoot{2346-2618}
\cfoot{}
\rfoot{Fall 2020}
\begin{document}

\begin{enumerate}
\litem{
Find the inverse of the function below. Then, evaluate the inverse at $x = 10$ and choose the interval that $f^{-1}(10)$ belongs to.\[ f(x) = e^{x-5}-3 \]\begin{enumerate}[label=\Alph*.]
\item \( f^{-1}(10) \in [-2.06, -1.26] \)
\item \( f^{-1}(10) \in [-0.43, 0.04] \)
\item \( f^{-1}(10) \in [-2.62, -2.38] \)
\item \( f^{-1}(10) \in [-1.11, -0.53] \)
\item \( f^{-1}(10) \in [7.54, 8.2] \)

\end{enumerate} }
\litem{
Subtract the following functions, then choose the domain of the resulting function from the list below.\[ f(x) = 5x^{4} +4 x^{3} +4 x^{2} +3 x + 2 \text{ and } g(x) = \frac{4}{5x-33} \]\begin{enumerate}[label=\Alph*.]
\item \( \text{ The domain is all Real numbers except } x = a, \text{ where } a \in [4.6, 10.6] \)
\item \( \text{ The domain is all Real numbers greater than or equal to } x = a, \text{ where } a \in [4.5, 5.5] \)
\item \( \text{ The domain is all Real numbers less than or equal to } x = a, \text{ where } a \in [-6, -4] \)
\item \( \text{ The domain is all Real numbers except } x = a \text{ and } x = b, \text{ where } a \in [5.25, 6.25] \text{ and } b \in [4.6, 9.6] \)
\item \( \text{ The domain is all Real numbers. } \)

\end{enumerate} }
\litem{
Choose the interval below that $f$ composed with $g$ at $x=1$ is in.\[ f(x) = -x^{3} -3 x^{2} +3 x + 2 \text{ and } g(x) = -x^{3} +3 x^{2} +x \]\begin{enumerate}[label=\Alph*.]
\item \( (f \circ g)(1) \in [9, 13] \)
\item \( (f \circ g)(1) \in [-55, -49] \)
\item \( (f \circ g)(1) \in [-43, -40] \)
\item \( (f \circ g)(1) \in [1, 7] \)
\item \( \text{It is not possible to compose the two functions.} \)

\end{enumerate} }
\litem{
Multiply the following functions, then choose the domain of the resulting function from the list below.\[ f(x) = \frac{1}{4x-25} \text{ and } g(x) = \frac{5}{5x+26} \]\begin{enumerate}[label=\Alph*.]
\item \( \text{ The domain is all Real numbers less than or equal to } x = a, \text{ where } a \in [-4.33, 1.67] \)
\item \( \text{ The domain is all Real numbers except } x = a, \text{ where } a \in [-7.6, 0.4] \)
\item \( \text{ The domain is all Real numbers greater than or equal to } x = a, \text{ where } a \in [-15, -5] \)
\item \( \text{ The domain is all Real numbers except } x = a \text{ and } x = b, \text{ where } a \in [5.25, 10.25] \text{ and } b \in [-5.2, -2.2] \)
\item \( \text{ The domain is all Real numbers. } \)

\end{enumerate} }
\litem{
Choose the interval below that $f$ composed with $g$ at $x=1$ is in.\[ f(x) = x^{3} -2 x^{2} +x \text{ and } g(x) = -x^{3} -4 x^{2} +4 x \]\begin{enumerate}[label=\Alph*.]
\item \( (f \circ g)(1) \in [-8.6, -4.2] \)
\item \( (f \circ g)(1) \in [-4.6, -3.6] \)
\item \( (f \circ g)(1) \in [-3.7, 0.7] \)
\item \( (f \circ g)(1) \in [-15.4, -12.3] \)
\item \( \text{It is not possible to compose the two functions.} \)

\end{enumerate} }
\litem{
Find the inverse of the function below (if it exists). Then, evaluate the inverse at $x = -15$ and choose the interval that $f^{-1}(-15)$ belongs to.\[ f(x) = 5 x^2 + 2 \]\begin{enumerate}[label=\Alph*.]
\item \( f^{-1}(-15) \in [4.73, 5.5] \)
\item \( f^{-1}(-15) \in [2.5, 3.56] \)
\item \( f^{-1}(-15) \in [1.7, 2.39] \)
\item \( f^{-1}(-15) \in [1.43, 1.8] \)
\item \( \text{ The function is not invertible for all Real numbers. } \)

\end{enumerate} }
\litem{
Find the inverse of the function below. Then, evaluate the inverse at $x = 8$ and choose the interval that $f^{-1}(8)$ belongs to.\[ f(x) = e^{x+2}+5 \]\begin{enumerate}[label=\Alph*.]
\item \( f^{-1}(8) \in [2.94, 3.21] \)
\item \( f^{-1}(8) \in [7.13, 7.36] \)
\item \( f^{-1}(8) \in [6.4, 6.84] \)
\item \( f^{-1}(8) \in [7.52, 7.9] \)
\item \( f^{-1}(8) \in [-1.07, -0.64] \)

\end{enumerate} }
\litem{
Find the inverse of the function below (if it exists). Then, evaluate the inverse at $x = 10$ and choose the interval that $f^{-1}(10)$ belongs to.\[ f(x) = 5 x^2 + 4 \]\begin{enumerate}[label=\Alph*.]
\item \( f^{-1}(10) \in [0.41, 1.38] \)
\item \( f^{-1}(10) \in [6.96, 7.59] \)
\item \( f^{-1}(10) \in [1.26, 2.04] \)
\item \( f^{-1}(10) \in [3.23, 4.28] \)
\item \( \text{ The function is not invertible for all Real numbers. } \)

\end{enumerate} }
\litem{
Determine whether the function below is 1-1.\[ f(x) = 16 x^2 + 176 x + 484 \]\begin{enumerate}[label=\Alph*.]
\item \( \text{Yes, the function is 1-1.} \)
\item \( \text{No, because there is a $y$-value that goes to 2 different $x$-values.} \)
\item \( \text{No, because there is an $x$-value that goes to 2 different $y$-values.} \)
\item \( \text{No, because the range of the function is not $(-\infty, \infty)$.} \)
\item \( \text{No, because the domain of the function is not $(-\infty, \infty)$.} \)

\end{enumerate} }
\litem{
Determine whether the function below is 1-1.\[ f(x) = (3 x - 20)^3 \]\begin{enumerate}[label=\Alph*.]
\item \( \text{No, because the range of the function is not $(-\infty, \infty)$.} \)
\item \( \text{No, because there is an $x$-value that goes to 2 different $y$-values.} \)
\item \( \text{No, because there is a $y$-value that goes to 2 different $x$-values.} \)
\item \( \text{Yes, the function is 1-1.} \)
\item \( \text{No, because the domain of the function is not $(-\infty, \infty)$.} \)

\end{enumerate} }
\end{enumerate}

\end{document}