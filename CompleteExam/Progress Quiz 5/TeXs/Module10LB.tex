\documentclass[14pt]{extbook}
\usepackage{multicol, enumerate, enumitem, hyperref, color, soul, setspace, parskip, fancyhdr} %General Packages
\usepackage{amssymb, amsthm, amsmath, latexsym, units, mathtools} %Math Packages
\everymath{\displaystyle} %All math in Display Style
% Packages with additional options
\usepackage[headsep=0.5cm,headheight=12pt, left=1 in,right= 1 in,top= 1 in,bottom= 1 in]{geometry}
\usepackage[usenames,dvipsnames]{xcolor}
\usepackage{dashrule}  % Package to use the command below to create lines between items
\newcommand{\litem}[1]{\item#1\hspace*{-1cm}\rule{\textwidth}{0.4pt}}
\pagestyle{fancy}
\lhead{Progress Quiz 5}
\chead{}
\rhead{Version B}
\lfoot{8497-6012}
\cfoot{}
\rfoot{Summer C 2021}
\begin{document}

\begin{enumerate}
\litem{
Perform the division below. Then, find the intervals that correspond to the quotient in the form $ax^2+bx+c$ and remainder $r$.\[ \frac{20x^{3} +105 x^{2} -128}{x + 5} \]\begin{enumerate}[label=\Alph*.]
\item \( a \in [19, 27], b \in [-15, -11], c \in [89, 92], \text{ and } r \in [-670, -662]. \)
\item \( a \in [19, 27], b \in [2, 11], c \in [-30, -24], \text{ and } r \in [-7, -2]. \)
\item \( a \in [-105, -94], b \in [-397, -394], c \in [-1976, -1973], \text{ and } r \in [-10008, -9998]. \)
\item \( a \in [-105, -94], b \in [602, 607], c \in [-3027, -3024], \text{ and } r \in [14989, 15000]. \)
\item \( a \in [19, 27], b \in [203, 206], c \in [1023, 1026], \text{ and } r \in [4997, 5002]. \)

\end{enumerate} }
\litem{
Factor the polynomial below completely. Then, choose the intervals the zeros of the polynomial belong to, where $z_1 \leq z_2 \leq z_3$. \textit{To make the problem easier, all zeros are between -5 and 5.}\[ f(x) = 10x^{3} -39 x^{2} -61 x + 30 \]\begin{enumerate}[label=\Alph*.]
\item \( z_1 \in [-2.4, -0.9], \text{   }  z_2 \in [0.36, 0.97], \text{   and   } z_3 \in [4.87, 5.67] \)
\item \( z_1 \in [-5.1, -4.1], \text{   }  z_2 \in [-0.78, -0.09], \text{   and   } z_3 \in [0.97, 1.69] \)
\item \( z_1 \in [-1.4, 0.1], \text{   }  z_2 \in [2.08, 3.12], \text{   and   } z_3 \in [4.87, 5.67] \)
\item \( z_1 \in [-5.1, -4.1], \text{   }  z_2 \in [-3.19, -2.32], \text{   and   } z_3 \in [0.45, 0.75] \)
\item \( z_1 \in [-5.1, -4.1], \text{   }  z_2 \in [-2.36, -1.87], \text{   and   } z_3 \in [0.14, 0.36] \)

\end{enumerate} }
\litem{
Factor the polynomial below completely. Then, choose the intervals the zeros of the polynomial belong to, where $z_1 \leq z_2 \leq z_3$. \textit{To make the problem easier, all zeros are between -5 and 5.}\[ f(x) = 15x^{3} -1 x^{2} -52 x + 20 \]\begin{enumerate}[label=\Alph*.]
\item \( z_1 \in [-1.85, -1.24], \text{   }  z_2 \in [-0.43, -0.35], \text{   and   } z_3 \in [1.81, 2.03] \)
\item \( z_1 \in [-2.78, -2.35], \text{   }  z_2 \in [-0.6, -0.46], \text{   and   } z_3 \in [1.81, 2.03] \)
\item \( z_1 \in [-5.02, -4.61], \text{   }  z_2 \in [-0.17, -0.02], \text{   and   } z_3 \in [1.81, 2.03] \)
\item \( z_1 \in [-2.33, -1.98], \text{   }  z_2 \in [0.59, 0.64], \text{   and   } z_3 \in [2.15, 2.76] \)
\item \( z_1 \in [-2.33, -1.98], \text{   }  z_2 \in [0.38, 0.48], \text{   and   } z_3 \in [1.36, 1.85] \)

\end{enumerate} }
\litem{
Perform the division below. Then, find the intervals that correspond to the quotient in the form $ax^2+bx+c$ and remainder $r$.\[ \frac{8x^{3} -62 x + 33}{x + 3} \]\begin{enumerate}[label=\Alph*.]
\item \( a \in [4, 9], b \in [-39, -31], c \in [62, 69], \text{ and } r \in [-232, -225]. \)
\item \( a \in [-27, -21], b \in [-72, -67], c \in [-280, -277], \text{ and } r \in [-804, -800]. \)
\item \( a \in [4, 9], b \in [20, 26], c \in [7, 15], \text{ and } r \in [58, 66]. \)
\item \( a \in [-27, -21], b \in [71, 77], c \in [-280, -277], \text{ and } r \in [867, 868]. \)
\item \( a \in [4, 9], b \in [-28, -21], c \in [7, 15], \text{ and } r \in [2, 5]. \)

\end{enumerate} }
\litem{
Perform the division below. Then, find the intervals that correspond to the quotient in the form $ax^2+bx+c$ and remainder $r$.\[ \frac{4x^{3} -22 x^{2} +4 x + 26}{x -5} \]\begin{enumerate}[label=\Alph*.]
\item \( a \in [2, 5], \text{   } b \in [-2, 2], \text{   } c \in [-6, -5], \text{   and   } r \in [-7, -1]. \)
\item \( a \in [20, 23], \text{   } b \in [75, 79], \text{   } c \in [394, 399], \text{   and   } r \in [1991, 1997]. \)
\item \( a \in [2, 5], \text{   } b \in [-8, -5], \text{   } c \in [-24, -18], \text{   and   } r \in [-58, -52]. \)
\item \( a \in [2, 5], \text{   } b \in [-43, -39], \text{   } c \in [213, 221], \text{   and   } r \in [-1044, -1043]. \)
\item \( a \in [20, 23], \text{   } b \in [-125, -115], \text{   } c \in [610, 618], \text{   and   } r \in [-3050, -3036]. \)

\end{enumerate} }
\litem{
Factor the polynomial below completely, knowing that $x + 3$ is a factor. Then, choose the intervals the zeros of the polynomial belong to, where $z_1 \leq z_2 \leq z_3 \leq z_4$. \textit{To make the problem easier, all zeros are between -5 and 5.}\[ f(x) = 4x^{4} +4 x^{3} -51 x^{2} -36 x + 135 \]\begin{enumerate}[label=\Alph*.]
\item \( z_1 \in [-5, 1], \text{   }  z_2 \in [-2.54, -2.45], z_3 \in [1.24, 1.53], \text{   and   } z_4 \in [3, 4] \)
\item \( z_1 \in [-5, 1], \text{   }  z_2 \in [-0.8, -0.68], z_3 \in [2.74, 3.16], \text{   and   } z_4 \in [5, 7] \)
\item \( z_1 \in [-5, 1], \text{   }  z_2 \in [-0.72, -0.56], z_3 \in [0.32, 0.59], \text{   and   } z_4 \in [3, 4] \)
\item \( z_1 \in [-5, 1], \text{   }  z_2 \in [-0.5, -0.35], z_3 \in [0.42, 0.9], \text{   and   } z_4 \in [3, 4] \)
\item \( z_1 \in [-5, 1], \text{   }  z_2 \in [-1.5, -1.46], z_3 \in [2.24, 2.69], \text{   and   } z_4 \in [3, 4] \)

\end{enumerate} }
\litem{
Factor the polynomial below completely, knowing that $x + 4$ is a factor. Then, choose the intervals the zeros of the polynomial belong to, where $z_1 \leq z_2 \leq z_3 \leq z_4$. \textit{To make the problem easier, all zeros are between -5 and 5.}\[ f(x) = 12x^{4} +101 x^{3} +165 x^{2} -248 x -240 \]\begin{enumerate}[label=\Alph*.]
\item \( z_1 \in [-0.46, 0.02], \text{   }  z_2 \in [2.74, 3.09], z_3 \in [3.87, 4.03], \text{   and   } z_4 \in [3.99, 5.65] \)
\item \( z_1 \in [-5.22, -4.73], \text{   }  z_2 \in [-4.54, -3.29], z_3 \in [-2.25, -0.9], \text{   and   } z_4 \in [-0.17, 1] \)
\item \( z_1 \in [-1.56, -0.95], \text{   }  z_2 \in [0.63, 0.84], z_3 \in [3.87, 4.03], \text{   and   } z_4 \in [3.99, 5.65] \)
\item \( z_1 \in [-0.96, -0.61], \text{   }  z_2 \in [1.26, 1.46], z_3 \in [3.87, 4.03], \text{   and   } z_4 \in [3.99, 5.65] \)
\item \( z_1 \in [-5.22, -4.73], \text{   }  z_2 \in [-4.54, -3.29], z_3 \in [-1, -0.5], \text{   and   } z_4 \in [0.79, 1.62] \)

\end{enumerate} }
\litem{
Perform the division below. Then, find the intervals that correspond to the quotient in the form $ax^2+bx+c$ and remainder $r$.\[ \frac{25x^{3} -85 x^{2} +15 x + 40}{x -3} \]\begin{enumerate}[label=\Alph*.]
\item \( a \in [73, 76], \text{   } b \in [-314, -306], \text{   } c \in [945, 951], \text{   and   } r \in [-2795, -2791]. \)
\item \( a \in [25, 26], \text{   } b \in [-163, -157], \text{   } c \in [492, 496], \text{   and   } r \in [-1445, -1441]. \)
\item \( a \in [73, 76], \text{   } b \in [136, 145], \text{   } c \in [432, 438], \text{   and   } r \in [1340, 1346]. \)
\item \( a \in [25, 26], \text{   } b \in [-42, -31], \text{   } c \in [-60, -51], \text{   and   } r \in [-71, -65]. \)
\item \( a \in [25, 26], \text{   } b \in [-19, -9], \text{   } c \in [-17, -12], \text{   and   } r \in [-5, -1]. \)

\end{enumerate} }
\litem{
What are the \textit{possible Integer} roots of the polynomial below?\[ f(x) = 3x^{2} +5 x + 4 \]\begin{enumerate}[label=\Alph*.]
\item \( \pm 1,\pm 2,\pm 4 \)
\item \( \text{ All combinations of: }\frac{\pm 1,\pm 3}{\pm 1,\pm 2,\pm 4} \)
\item \( \text{ All combinations of: }\frac{\pm 1,\pm 2,\pm 4}{\pm 1,\pm 3} \)
\item \( \pm 1,\pm 3 \)
\item \( \text{There is no formula or theorem that tells us all possible Integer roots.} \)

\end{enumerate} }
\litem{
What are the \textit{possible Integer} roots of the polynomial below?\[ f(x) = 5x^{2} +5 x + 2 \]\begin{enumerate}[label=\Alph*.]
\item \( \text{ All combinations of: }\frac{\pm 1,\pm 5}{\pm 1,\pm 2} \)
\item \( \pm 1,\pm 2 \)
\item \( \pm 1,\pm 5 \)
\item \( \text{ All combinations of: }\frac{\pm 1,\pm 2}{\pm 1,\pm 5} \)
\item \( \text{There is no formula or theorem that tells us all possible Integer roots.} \)

\end{enumerate} }
\end{enumerate}

\end{document}