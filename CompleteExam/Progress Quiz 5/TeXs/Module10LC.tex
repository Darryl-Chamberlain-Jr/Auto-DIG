\documentclass[14pt]{extbook}
\usepackage{multicol, enumerate, enumitem, hyperref, color, soul, setspace, parskip, fancyhdr} %General Packages
\usepackage{amssymb, amsthm, amsmath, latexsym, units, mathtools} %Math Packages
\everymath{\displaystyle} %All math in Display Style
% Packages with additional options
\usepackage[headsep=0.5cm,headheight=12pt, left=1 in,right= 1 in,top= 1 in,bottom= 1 in]{geometry}
\usepackage[usenames,dvipsnames]{xcolor}
\usepackage{dashrule}  % Package to use the command below to create lines between items
\newcommand{\litem}[1]{\item#1\hspace*{-1cm}\rule{\textwidth}{0.4pt}}
\pagestyle{fancy}
\lhead{Progress Quiz 5}
\chead{}
\rhead{Version C}
\lfoot{8497-6012}
\cfoot{}
\rfoot{Summer C 2021}
\begin{document}

\begin{enumerate}
\litem{
Perform the division below. Then, find the intervals that correspond to the quotient in the form $ax^2+bx+c$ and remainder $r$.\[ \frac{16x^{3} -49 x + 32}{x + 2} \]\begin{enumerate}[label=\Alph*.]
\item \( a \in [16, 18], b \in [31, 38], c \in [12, 17], \text{ and } r \in [59, 67]. \)
\item \( a \in [-34, -25], b \in [-69, -63], c \in [-182, -175], \text{ and } r \in [-324, -318]. \)
\item \( a \in [-34, -25], b \in [57, 67], c \in [-182, -175], \text{ and } r \in [385, 392]. \)
\item \( a \in [16, 18], b \in [-49, -47], c \in [91, 99], \text{ and } r \in [-253, -246]. \)
\item \( a \in [16, 18], b \in [-40, -28], c \in [12, 17], \text{ and } r \in [-1, 5]. \)

\end{enumerate} }
\litem{
Factor the polynomial below completely. Then, choose the intervals the zeros of the polynomial belong to, where $z_1 \leq z_2 \leq z_3$. \textit{To make the problem easier, all zeros are between -5 and 5.}\[ f(x) = 6x^{3} -35 x^{2} +66 x -40 \]\begin{enumerate}[label=\Alph*.]
\item \( z_1 \in [-2.69, -2.1], \text{   }  z_2 \in [-3, -1.8], \text{   and   } z_3 \in [-1.45, -1.14] \)
\item \( z_1 \in [-5.19, -4.42], \text{   }  z_2 \in [-3, -1.8], \text{   and   } z_3 \in [-0.8, -0.42] \)
\item \( z_1 \in [1.1, 1.67], \text{   }  z_2 \in [1, 2.5], \text{   and   } z_3 \in [2.32, 2.71] \)
\item \( z_1 \in [-2.18, -1.47], \text{   }  z_2 \in [-1.1, -0.6], \text{   and   } z_3 \in [-0.62, -0.23] \)
\item \( z_1 \in [0.05, 0.53], \text{   }  z_2 \in [0.4, 1.5], \text{   and   } z_3 \in [1.95, 2.11] \)

\end{enumerate} }
\litem{
Factor the polynomial below completely. Then, choose the intervals the zeros of the polynomial belong to, where $z_1 \leq z_2 \leq z_3$. \textit{To make the problem easier, all zeros are between -5 and 5.}\[ f(x) = 25x^{3} -45 x^{2} -82 x -24 \]\begin{enumerate}[label=\Alph*.]
\item \( z_1 \in [-3.11, -2.79], \text{   }  z_2 \in [0.24, 0.6], \text{   and   } z_3 \in [0.38, 0.88] \)
\item \( z_1 \in [-3.11, -2.79], \text{   }  z_2 \in [1.07, 1.31], \text{   and   } z_3 \in [2.14, 2.54] \)
\item \( z_1 \in [-1.16, -0.39], \text{   }  z_2 \in [-0.67, -0.28], \text{   and   } z_3 \in [2.54, 3.27] \)
\item \( z_1 \in [-2.65, -2.14], \text{   }  z_2 \in [-1.42, -1.09], \text{   and   } z_3 \in [2.54, 3.27] \)
\item \( z_1 \in [-3.11, -2.79], \text{   }  z_2 \in [0.09, 0.19], \text{   and   } z_3 \in [1.46, 2.4] \)

\end{enumerate} }
\litem{
Perform the division below. Then, find the intervals that correspond to the quotient in the form $ax^2+bx+c$ and remainder $r$.\[ \frac{10x^{3} -35 x^{2} + 42}{x -3} \]\begin{enumerate}[label=\Alph*.]
\item \( a \in [28, 31], b \in [54, 58], c \in [160, 169], \text{ and } r \in [535, 539]. \)
\item \( a \in [28, 31], b \in [-126, -122], c \in [369, 376], \text{ and } r \in [-1084, -1081]. \)
\item \( a \in [5, 15], b \in [-6, -2], c \in [-20, -6], \text{ and } r \in [-5, 1]. \)
\item \( a \in [5, 15], b \in [-17, -7], c \in [-34, -25], \text{ and } r \in [-20, -12]. \)
\item \( a \in [5, 15], b \in [-65, -61], c \in [193, 197], \text{ and } r \in [-545, -541]. \)

\end{enumerate} }
\litem{
Perform the division below. Then, find the intervals that correspond to the quotient in the form $ax^2+bx+c$ and remainder $r$.\[ \frac{12x^{3} -34 x^{2} -10 x + 7}{x -3} \]\begin{enumerate}[label=\Alph*.]
\item \( a \in [31, 39], \text{   } b \in [71, 77], \text{   } c \in [211, 218], \text{   and   } r \in [643, 650]. \)
\item \( a \in [10, 17], \text{   } b \in [-11, -8], \text{   } c \in [-30, -25], \text{   and   } r \in [-58, -51]. \)
\item \( a \in [10, 17], \text{   } b \in [-2, 3], \text{   } c \in [-5, -2], \text{   and   } r \in [-6, 0]. \)
\item \( a \in [10, 17], \text{   } b \in [-75, -64], \text{   } c \in [194, 203], \text{   and   } r \in [-596, -584]. \)
\item \( a \in [31, 39], \text{   } b \in [-148, -138], \text{   } c \in [415, 418], \text{   and   } r \in [-1243, -1237]. \)

\end{enumerate} }
\litem{
Factor the polynomial below completely, knowing that $x + 3$ is a factor. Then, choose the intervals the zeros of the polynomial belong to, where $z_1 \leq z_2 \leq z_3 \leq z_4$. \textit{To make the problem easier, all zeros are between -5 and 5.}\[ f(x) = 8x^{4} +26 x^{3} -37 x^{2} -159 x -90 \]\begin{enumerate}[label=\Alph*.]
\item \( z_1 \in [-1.23, -0.19], \text{   }  z_2 \in [0.77, 1.43], z_3 \in [1.38, 2.29], \text{   and   } z_4 \in [2.6, 4.3] \)
\item \( z_1 \in [-5.3, -4.32], \text{   }  z_2 \in [0.15, 0.47], z_3 \in [1.38, 2.29], \text{   and   } z_4 \in [2.6, 4.3] \)
\item \( z_1 \in [-3.63, -2.76], \text{   }  z_2 \in [-2.18, -1.85], z_3 \in [-0.83, -0.16], \text{   and   } z_4 \in [0.6, 2.9] \)
\item \( z_1 \in [-3.63, -2.76], \text{   }  z_2 \in [-2.18, -1.85], z_3 \in [-1.36, -0.79], \text{   and   } z_4 \in [-0.4, 1.4] \)
\item \( z_1 \in [-2.68, -2.27], \text{   }  z_2 \in [0.61, 0.85], z_3 \in [1.38, 2.29], \text{   and   } z_4 \in [2.6, 4.3] \)

\end{enumerate} }
\litem{
Factor the polynomial below completely, knowing that $x + 2$ is a factor. Then, choose the intervals the zeros of the polynomial belong to, where $z_1 \leq z_2 \leq z_3 \leq z_4$. \textit{To make the problem easier, all zeros are between -5 and 5.}\[ f(x) = 12x^{4} -29 x^{3} -33 x^{2} +116 x -60 \]\begin{enumerate}[label=\Alph*.]
\item \( z_1 \in [-2.5, -1.9], \text{   }  z_2 \in [-1.75, -1.65], z_3 \in [-0.91, -0.74], \text{   and   } z_4 \in [1, 5] \)
\item \( z_1 \in [-2.5, -1.9], \text{   }  z_2 \in [0.67, 0.79], z_3 \in [1.58, 1.81], \text{   and   } z_4 \in [1, 5] \)
\item \( z_1 \in [-2.5, -1.9], \text{   }  z_2 \in [0.58, 0.66], z_3 \in [1.23, 1.34], \text{   and   } z_4 \in [1, 5] \)
\item \( z_1 \in [-2.5, -1.9], \text{   }  z_2 \in [-1.42, -1.31], z_3 \in [-0.72, -0.45], \text{   and   } z_4 \in [1, 5] \)
\item \( z_1 \in [-3.2, -2.7], \text{   }  z_2 \in [-2.01, -1.99], z_3 \in [-0.58, -0.21], \text{   and   } z_4 \in [1, 5] \)

\end{enumerate} }
\litem{
Perform the division below. Then, find the intervals that correspond to the quotient in the form $ax^2+bx+c$ and remainder $r$.\[ \frac{20x^{3} -45 x^{2} -15 x + 45}{x -2} \]\begin{enumerate}[label=\Alph*.]
\item \( a \in [18, 23], \text{   } b \in [-8, -2], \text{   } c \in [-30, -22], \text{   and   } r \in [-5, -2]. \)
\item \( a \in [40, 42], \text{   } b \in [-130, -123], \text{   } c \in [233, 239], \text{   and   } r \in [-425, -423]. \)
\item \( a \in [18, 23], \text{   } b \in [-87, -83], \text{   } c \in [152, 156], \text{   and   } r \in [-269, -264]. \)
\item \( a \in [18, 23], \text{   } b \in [-27, -22], \text{   } c \in [-40, -39], \text{   and   } r \in [5, 10]. \)
\item \( a \in [40, 42], \text{   } b \in [31, 36], \text{   } c \in [52, 57], \text{   and   } r \in [155, 161]. \)

\end{enumerate} }
\litem{
What are the \textit{possible Rational} roots of the polynomial below?\[ f(x) = 6x^{3} +2 x^{2} +2 x + 2 \]\begin{enumerate}[label=\Alph*.]
\item \( \pm 1,\pm 2 \)
\item \( \pm 1,\pm 2,\pm 3,\pm 6 \)
\item \( \text{ All combinations of: }\frac{\pm 1,\pm 2}{\pm 1,\pm 2,\pm 3,\pm 6} \)
\item \( \text{ All combinations of: }\frac{\pm 1,\pm 2,\pm 3,\pm 6}{\pm 1,\pm 2} \)
\item \( \text{ There is no formula or theorem that tells us all possible Rational roots.} \)

\end{enumerate} }
\litem{
What are the \textit{possible Integer} roots of the polynomial below?\[ f(x) = 3x^{4} +2 x^{3} +6 x^{2} +7 x + 7 \]\begin{enumerate}[label=\Alph*.]
\item \( \text{ All combinations of: }\frac{\pm 1,\pm 3}{\pm 1,\pm 7} \)
\item \( \text{ All combinations of: }\frac{\pm 1,\pm 7}{\pm 1,\pm 3} \)
\item \( \pm 1,\pm 7 \)
\item \( \pm 1,\pm 3 \)
\item \( \text{There is no formula or theorem that tells us all possible Integer roots.} \)

\end{enumerate} }
\end{enumerate}

\end{document}