\documentclass{extbook}[14pt]
\usepackage{multicol, enumerate, enumitem, hyperref, color, soul, setspace, parskip, fancyhdr, amssymb, amsthm, amsmath, latexsym, units, mathtools}
\everymath{\displaystyle}
\usepackage[headsep=0.5cm,headheight=0cm, left=1 in,right= 1 in,top= 1 in,bottom= 1 in]{geometry}
\usepackage{dashrule}  % Package to use the command below to create lines between items
\newcommand{\litem}[1]{\item #1

\rule{\textwidth}{0.4pt}}
\pagestyle{fancy}
\lhead{}
\chead{Answer Key for Progress Quiz 5 Version A}
\rhead{}
\lfoot{8497-6012}
\cfoot{}
\rfoot{Summer C 2021}
\begin{document}
\textbf{This key should allow you to understand why you choose the option you did (beyond just getting a question right or wrong). \href{https://xronos.clas.ufl.edu/mac1105spring2020/courseDescriptionAndMisc/Exams/LearningFromResults}{More instructions on how to use this key can be found here}.}

\textbf{If you have a suggestion to make the keys better, \href{https://forms.gle/CZkbZmPbC9XALEE88}{please fill out the short survey here}.}

\textit{Note: This key is auto-generated and may contain issues and/or errors. The keys are reviewed after each exam to ensure grading is done accurately. If there are issues (like duplicate options), they are noted in the offline gradebook. The keys are a work-in-progress to give students as many resources to improve as possible.}

\rule{\textwidth}{0.4pt}

\begin{enumerate}\litem{
Simplify the expression below into the form $a+bi$. Then, choose the intervals that $a$ and $b$ belong to.
\[ (-9 - 10 i)(-3 - 2 i) \]The solution is \( 7 + 48 i \), which is option B.\begin{enumerate}[label=\Alph*.]
\item \( a \in [44, 52] \text{ and } b \in [-13, -7] \)

 $47 - 12 i$, which corresponds to adding a minus sign in the first term.
\item \( a \in [5, 13] \text{ and } b \in [45, 52] \)

* $7 + 48 i$, which is the correct option.
\item \( a \in [44, 52] \text{ and } b \in [9, 16] \)

 $47 + 12 i$, which corresponds to adding a minus sign in the second term.
\item \( a \in [5, 13] \text{ and } b \in [-48, -41] \)

 $7 - 48 i$, which corresponds to adding a minus sign in both terms.
\item \( a \in [27, 35] \text{ and } b \in [14, 25] \)

 $27 + 20 i$, which corresponds to just multiplying the real terms to get the real part of the solution and the coefficients in the complex terms to get the complex part.
\end{enumerate}

\textbf{General Comment:} You can treat $i$ as a variable and distribute. Just remember that $i^2=-1$, so you can continue to reduce after you distribute.
}
\litem{
Choose the \textbf{smallest} set of Real numbers that the number below belongs to.
\[ -\sqrt{\frac{12}{0}} \]The solution is \( \text{Not a Real number} \), which is option C.\begin{enumerate}[label=\Alph*.]
\item \( \text{Whole} \)

These are the counting numbers with 0 (0, 1, 2, 3, ...)
\item \( \text{Irrational} \)

These cannot be written as a fraction of Integers.
\item \( \text{Not a Real number} \)

* This is the correct option!
\item \( \text{Integer} \)

These are the negative and positive counting numbers (..., -3, -2, -1, 0, 1, 2, 3, ...)
\item \( \text{Rational} \)

These are numbers that can be written as fraction of Integers (e.g., -2/3)
\end{enumerate}

\textbf{General Comment:} First, you \textbf{NEED} to simplify the expression. This question simplifies to $-\sqrt{\frac{12}{0}}$. 
 
 Be sure you look at the simplified fraction and not just the decimal expansion. Numbers such as 13, 17, and 19 provide \textbf{long but repeating/terminating decimal expansions!} 
 
 The only ways to *not* be a Real number are: dividing by 0 or taking the square root of a negative number. 
 
 Irrational numbers are more than just square root of 3: adding or subtracting values from square root of 3 is also irrational.
}
\litem{
Simplify the expression below into the form $a+bi$. Then, choose the intervals that $a$ and $b$ belong to.
\[ (-7 - 4 i)(6 - 3 i) \]The solution is \( -54 - 3 i \), which is option E.\begin{enumerate}[label=\Alph*.]
\item \( a \in [-44, -41] \text{ and } b \in [7, 16] \)

 $-42 + 12 i$, which corresponds to just multiplying the real terms to get the real part of the solution and the coefficients in the complex terms to get the complex part.
\item \( a \in [-32, -28] \text{ and } b \in [42, 49] \)

 $-30 + 45 i$, which corresponds to adding a minus sign in the first term.
\item \( a \in [-57, -52] \text{ and } b \in [3, 5] \)

 $-54 + 3 i$, which corresponds to adding a minus sign in both terms.
\item \( a \in [-32, -28] \text{ and } b \in [-47, -40] \)

 $-30 - 45 i$, which corresponds to adding a minus sign in the second term.
\item \( a \in [-57, -52] \text{ and } b \in [-7, -2] \)

* $-54 - 3 i$, which is the correct option.
\end{enumerate}

\textbf{General Comment:} You can treat $i$ as a variable and distribute. Just remember that $i^2=-1$, so you can continue to reduce after you distribute.
}
\litem{
Simplify the expression below into the form $a+bi$. Then, choose the intervals that $a$ and $b$ belong to.
\[ \frac{36 + 33 i}{6 - 8 i} \]The solution is \( -0.48  + 4.86 i \), which is option D.\begin{enumerate}[label=\Alph*.]
\item \( a \in [4.4, 5.3] \text{ and } b \in [-1.5, 0.5] \)

 $4.80  - 0.90 i$, which corresponds to forgetting to multiply the conjugate by the numerator and not computing the conjugate correctly.
\item \( a \in [-48.35, -47.25] \text{ and } b \in [4, 6.5] \)

 $-48.00  + 4.86 i$, which corresponds to forgetting to multiply the conjugate by the numerator and using a plus instead of a minus in the denominator.
\item \( a \in [-0.9, 0.45] \text{ and } b \in [485.5, 486.5] \)

 $-0.48  + 486.00 i$, which corresponds to forgetting to multiply the conjugate by the numerator.
\item \( a \in [-0.9, 0.45] \text{ and } b \in [4, 6.5] \)

* $-0.48  + 4.86 i$, which is the correct option.
\item \( a \in [5.95, 6.45] \text{ and } b \in [-5, -3] \)

 $6.00  - 4.12 i$, which corresponds to just dividing the first term by the first term and the second by the second.
\end{enumerate}

\textbf{General Comment:} Multiply the numerator and denominator by the *conjugate* of the denominator, then simplify. For example, if we have $2+3i$, the conjugate is $2-3i$.
}
\litem{
Simplify the expression below into the form $a+bi$. Then, choose the intervals that $a$ and $b$ belong to.
\[ \frac{-9 + 22 i}{-3 + 4 i} \]The solution is \( 4.60  - 1.20 i \), which is option A.\begin{enumerate}[label=\Alph*.]
\item \( a \in [3.5, 5] \text{ and } b \in [-2, -1] \)

* $4.60  - 1.20 i$, which is the correct option.
\item \( a \in [3.5, 5] \text{ and } b \in [-30.5, -29] \)

 $4.60  - 30.00 i$, which corresponds to forgetting to multiply the conjugate by the numerator.
\item \( a \in [114.5, 115.5] \text{ and } b \in [-2, -1] \)

 $115.00  - 1.20 i$, which corresponds to forgetting to multiply the conjugate by the numerator and using a plus instead of a minus in the denominator.
\item \( a \in [2.5, 3.5] \text{ and } b \in [3.5, 6.5] \)

 $3.00  + 5.50 i$, which corresponds to just dividing the first term by the first term and the second by the second.
\item \( a \in [-3, -2] \text{ and } b \in [-5.5, -3.5] \)

 $-2.44  - 4.08 i$, which corresponds to forgetting to multiply the conjugate by the numerator and not computing the conjugate correctly.
\end{enumerate}

\textbf{General Comment:} Multiply the numerator and denominator by the *conjugate* of the denominator, then simplify. For example, if we have $2+3i$, the conjugate is $2-3i$.
}
\litem{
Choose the \textbf{smallest} set of Complex numbers that the number below belongs to.
\[ \sqrt{\frac{0}{49}}+\sqrt{4}i \]The solution is \( \text{Pure Imaginary} \), which is option A.\begin{enumerate}[label=\Alph*.]
\item \( \text{Pure Imaginary} \)

* This is the correct option!
\item \( \text{Nonreal Complex} \)

This is a Complex number $(a+bi)$ that is not Real (has $i$ as part of the number).
\item \( \text{Irrational} \)

These cannot be written as a fraction of Integers. Remember: $\pi$ is not an Integer!
\item \( \text{Rational} \)

These are numbers that can be written as fraction of Integers (e.g., -2/3 + 5)
\item \( \text{Not a Complex Number} \)

This is not a number. The only non-Complex number we know is dividing by 0 as this is not a number!
\end{enumerate}

\textbf{General Comment:} Be sure to simplify $i^2 = -1$. This may remove the imaginary portion for your number. If you are having trouble, you may want to look at the \textit{Subgroups of the Real Numbers} section.
}
\litem{
Simplify the expression below and choose the interval the simplification is contained within.
\[ 5 - 16^2 + 19 \div 4 * 11 \div 20 \]The solution is \( -248.387 \), which is option C.\begin{enumerate}[label=\Alph*.]
\item \( [262.9, 263.9] \)

 263.613, which corresponds to an Order of Operations error: multiplying by negative before squaring. For example: $(-3)^2 \neq -3^2$
\item \( [-255.4, -249.1] \)

 -250.978, which corresponds to an Order of Operations error: not reading left-to-right for multiplication/division.
\item \( [-249.7, -247.8] \)

* -248.387, this is the correct option
\item \( [259.8, 261.5] \)

 261.022, which corresponds to two Order of Operations errors.
\item \( \text{None of the above} \)

 You may have gotten this by making an unanticipated error. If you got a value that is not any of the others, please let the coordinator know so they can help you figure out what happened.
\end{enumerate}

\textbf{General Comment:} While you may remember (or were taught) PEMDAS is done in order, it is actually done as P/E/MD/AS. When we are at MD or AS, we read left to right.
}
\litem{
Choose the \textbf{smallest} set of Real numbers that the number below belongs to.
\[ -\sqrt{\frac{1430}{10}} \]The solution is \( \text{Irrational} \), which is option C.\begin{enumerate}[label=\Alph*.]
\item \( \text{Whole} \)

These are the counting numbers with 0 (0, 1, 2, 3, ...)
\item \( \text{Rational} \)

These are numbers that can be written as fraction of Integers (e.g., -2/3)
\item \( \text{Irrational} \)

* This is the correct option!
\item \( \text{Integer} \)

These are the negative and positive counting numbers (..., -3, -2, -1, 0, 1, 2, 3, ...)
\item \( \text{Not a Real number} \)

These are Nonreal Complex numbers \textbf{OR} things that are not numbers (e.g., dividing by 0).
\end{enumerate}

\textbf{General Comment:} First, you \textbf{NEED} to simplify the expression. This question simplifies to $-\sqrt{143}$. 
 
 Be sure you look at the simplified fraction and not just the decimal expansion. Numbers such as 13, 17, and 19 provide \textbf{long but repeating/terminating decimal expansions!} 
 
 The only ways to *not* be a Real number are: dividing by 0 or taking the square root of a negative number. 
 
 Irrational numbers are more than just square root of 3: adding or subtracting values from square root of 3 is also irrational.
}
\litem{
Choose the \textbf{smallest} set of Complex numbers that the number below belongs to.
\[ \frac{16}{16}+81i^2 \]The solution is \( \text{Rational} \), which is option A.\begin{enumerate}[label=\Alph*.]
\item \( \text{Rational} \)

* This is the correct option!
\item \( \text{Not a Complex Number} \)

This is not a number. The only non-Complex number we know is dividing by 0 as this is not a number!
\item \( \text{Pure Imaginary} \)

This is a Complex number $(a+bi)$ that \textbf{only} has an imaginary part like $2i$.
\item \( \text{Nonreal Complex} \)

This is a Complex number $(a+bi)$ that is not Real (has $i$ as part of the number).
\item \( \text{Irrational} \)

These cannot be written as a fraction of Integers. Remember: $\pi$ is not an Integer!
\end{enumerate}

\textbf{General Comment:} Be sure to simplify $i^2 = -1$. This may remove the imaginary portion for your number. If you are having trouble, you may want to look at the \textit{Subgroups of the Real Numbers} section.
}
\litem{
Simplify the expression below and choose the interval the simplification is contained within.
\[ 2 - 10 \div 7 * 16 - (14 * 4) \]The solution is \( -76.857 \), which is option B.\begin{enumerate}[label=\Alph*.]
\item \( [-57.09, -50.09] \)

 -54.089, which corresponds to an Order of Operations error: not reading left-to-right for multiplication/division.
\item \( [-78.86, -75.86] \)

* -76.857, which is the correct option.
\item \( [49.91, 63.91] \)

 57.911, which corresponds to not distributing addition and subtraction correctly.
\item \( [-139.43, -138.43] \)

 -139.429, which corresponds to not distributing a negative correctly.
\item \( \text{None of the above} \)

 You may have gotten this by making an unanticipated error. If you got a value that is not any of the others, please let the coordinator know so they can help you figure out what happened.
\end{enumerate}

\textbf{General Comment:} While you may remember (or were taught) PEMDAS is done in order, it is actually done as P/E/MD/AS. When we are at MD or AS, we read left to right.
}
\end{enumerate}

\end{document}