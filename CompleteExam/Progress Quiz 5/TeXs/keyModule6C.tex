\documentclass{extbook}[14pt]
\usepackage{multicol, enumerate, enumitem, hyperref, color, soul, setspace, parskip, fancyhdr, amssymb, amsthm, amsmath, latexsym, units, mathtools}
\everymath{\displaystyle}
\usepackage[headsep=0.5cm,headheight=0cm, left=1 in,right= 1 in,top= 1 in,bottom= 1 in]{geometry}
\usepackage{dashrule}  % Package to use the command below to create lines between items
\newcommand{\litem}[1]{\item #1

\rule{\textwidth}{0.4pt}}
\pagestyle{fancy}
\lhead{}
\chead{Answer Key for Progress Quiz 5 Version C}
\rhead{}
\lfoot{8497-6012}
\cfoot{}
\rfoot{Summer C 2021}
\begin{document}
\textbf{This key should allow you to understand why you choose the option you did (beyond just getting a question right or wrong). \href{https://xronos.clas.ufl.edu/mac1105spring2020/courseDescriptionAndMisc/Exams/LearningFromResults}{More instructions on how to use this key can be found here}.}

\textbf{If you have a suggestion to make the keys better, \href{https://forms.gle/CZkbZmPbC9XALEE88}{please fill out the short survey here}.}

\textit{Note: This key is auto-generated and may contain issues and/or errors. The keys are reviewed after each exam to ensure grading is done accurately. If there are issues (like duplicate options), they are noted in the offline gradebook. The keys are a work-in-progress to give students as many resources to improve as possible.}

\rule{\textwidth}{0.4pt}

\begin{enumerate}\litem{
Construct the lowest-degree polynomial given the zeros below. Then, choose the intervals that contain the coefficients of the polynomial in the form $x^3+bx^2+cx+d$.
\[ 4 - 2 i \text{ and } -4 \]The solution is \( x^{3} -4 x^{2} -12 x + 80 \), which is option B.\begin{enumerate}[label=\Alph*.]
\item \( b \in [2.8, 5.1], c \in [-12, -11], \text{ and } d \in [-87, -78] \)

$x^{3} +4 x^{2} -12 x -80$, which corresponds to multiplying out $(x-(4 - 2 i))(x-(4 + 2 i))(x -4)$.
\item \( b \in [-7.8, -3.5], c \in [-12, -11], \text{ and } d \in [78, 84] \)

* $x^{3} -4 x^{2} -12 x + 80$, which is the correct option.
\item \( b \in [-0.6, 1.1], c \in [3, 7], \text{ and } d \in [3, 9] \)

$x^{3} + x^{2} +6 x + 8$, which corresponds to multiplying out $(x + 2)(x + 4)$.
\item \( b \in [-0.6, 1.1], c \in [0, 5], \text{ and } d \in [-20, -13] \)

$x^{3} + x^{2} -16$, which corresponds to multiplying out $(x -4)(x + 4)$.
\item \( \text{None of the above.} \)

This corresponds to making an unanticipated error or not understanding how to use nonreal complex numbers to create the lowest-degree polynomial. If you chose this and are not sure what you did wrong, please contact the coordinator for help.
\end{enumerate}

\textbf{General Comment:} Remember that the conjugate of $a+bi$ is $a-bi$. Since these zeros always come in pairs, we need to multiply out $(x-(4 - 2 i))(x-(4 + 2 i))(x-(-4))$.
}
\litem{
Construct the lowest-degree polynomial given the zeros below. Then, choose the intervals that contain the coefficients of the polynomial in the form $ax^3+bx^2+cx+d$.
\[ -6, \frac{1}{3}, \text{ and } \frac{-3}{2} \]The solution is \( 6x^{3} +43 x^{2} +39 x -18 \), which is option B.\begin{enumerate}[label=\Alph*.]
\item \( a \in [6, 12], b \in [-25.3, -24.5], c \in [-64, -61], \text{ and } d \in [-24, -15] \)

$6x^{3} -25 x^{2} -63 x -18$, which corresponds to multiplying out $(x -6)(3x + 1)(2x + 3)$.
\item \( a \in [6, 12], b \in [40.1, 45.7], c \in [33, 40], \text{ and } d \in [-24, -15] \)

* $6x^{3} +43 x^{2} +39 x -18$, which is the correct option.
\item \( a \in [6, 12], b \in [-30.7, -26], c \in [-53, -38], \text{ and } d \in [11, 26] \)

$6x^{3} -29 x^{2} -45 x + 18$, which corresponds to multiplying out $(x -6)(3x -1)(2x + 3)$.
\item \( a \in [6, 12], b \in [-44.1, -41], c \in [33, 40], \text{ and } d \in [11, 26] \)

$6x^{3} -43 x^{2} +39 x + 18$, which corresponds to multiplying out $(x -6)(3x + 1)(2x -3)$.
\item \( a \in [6, 12], b \in [40.1, 45.7], c \in [33, 40], \text{ and } d \in [11, 26] \)

$6x^{3} +43 x^{2} +39 x + 18$, which corresponds to multiplying everything correctly except the constant term.
\end{enumerate}

\textbf{General Comment:} To construct the lowest-degree polynomial, you want to multiply out $(x + 6)(3x -1)(2x + 3)$
}
\litem{
Describe the end behavior of the polynomial below.
\[ f(x) = -6(x + 7)^{5}(x - 7)^{10}(x - 8)^{3}(x + 8)^{3} \]The solution is the graph below, which is option A.
    \begin{center}
        \includegraphics[width=0.3\textwidth]{../Figures/polyEndBehaviorAC.png}
    \end{center}\begin{enumerate}[label=\Alph*.]
\begin{multicols}{2}
\item \includegraphics[width = 0.3\textwidth]{../Figures/polyEndBehaviorAC.png}
\item \includegraphics[width = 0.3\textwidth]{../Figures/polyEndBehaviorBC.png}
\item \includegraphics[width = 0.3\textwidth]{../Figures/polyEndBehaviorCC.png}
\item \includegraphics[width = 0.3\textwidth]{../Figures/polyEndBehaviorDC.png}
\end{multicols}\item None of the above.\end{enumerate}
\textbf{General Comment:} Remember that end behavior is determined by the leading coefficient AND whether the \textbf{sum} of the multiplicities is positive or negative.
}
\litem{
Which of the following equations \textit{could} be of the graph presented below?

\begin{center}
    \includegraphics[width=0.5\textwidth]{../Figures/polyGraphToFunctionCopyC.png}
\end{center}


The solution is \( -13(x + 3)^{9} (x - 1)^{7} (x + 4)^{5} \), which is option E.\begin{enumerate}[label=\Alph*.]
\item \( -10(x + 3)^{10} (x - 1)^{11} (x + 4)^{9} \)

The factor $-3$ should have been an odd power.
\item \( 9(x + 3)^{10} (x - 1)^{5} (x + 4)^{5} \)

The factor $(x + 3)$ should have an odd power and the leading coefficient should be the opposite sign.
\item \( 16(x + 3)^{7} (x - 1)^{7} (x + 4)^{5} \)

This corresponds to the leading coefficient being the opposite value than it should be.
\item \( -17(x + 3)^{10} (x - 1)^{6} (x + 4)^{11} \)

The factors $-3$ and $1$ have have been odd power.
\item \( -13(x + 3)^{9} (x - 1)^{7} (x + 4)^{5} \)

* This is the correct option.
\end{enumerate}

\textbf{General Comment:} General Comments: Draw the x-axis to determine which zeros are touching (and so have even multiplicity) or cross (and have odd multiplicity).
}
\litem{
Construct the lowest-degree polynomial given the zeros below. Then, choose the intervals that contain the coefficients of the polynomial in the form $ax^3+bx^2+cx+d$.
\[ \frac{1}{4}, \frac{7}{4}, \text{ and } \frac{-2}{3} \]The solution is \( 48x^{3} -64 x^{2} -43 x + 14 \), which is option C.\begin{enumerate}[label=\Alph*.]
\item \( a \in [45, 50], b \in [123, 130], c \in [83, 89], \text{ and } d \in [12, 19] \)

$48x^{3} +128 x^{2} +85 x + 14$, which corresponds to multiplying out $(4x + 1)(4x + 7)(3x + 2)$.
\item \( a \in [45, 50], b \in [-41, -33], c \in [-70, -66], \text{ and } d \in [-20, -13] \)

$48x^{3} -40 x^{2} -69 x -14$, which corresponds to multiplying out $(4x + 1)(4x -7)(3x + 2)$.
\item \( a \in [45, 50], b \in [-66, -60], c \in [-43, -33], \text{ and } d \in [12, 19] \)

* $48x^{3} -64 x^{2} -43 x + 14$, which is the correct option.
\item \( a \in [45, 50], b \in [-66, -60], c \in [-43, -33], \text{ and } d \in [-20, -13] \)

$48x^{3} -64 x^{2} -43 x -14$, which corresponds to multiplying everything correctly except the constant term.
\item \( a \in [45, 50], b \in [64, 70], c \in [-43, -33], \text{ and } d \in [-20, -13] \)

$48x^{3} +64 x^{2} -43 x -14$, which corresponds to multiplying out $(4x + 1)(4x + 7)(3x -2)$.
\end{enumerate}

\textbf{General Comment:} To construct the lowest-degree polynomial, you want to multiply out $(4x -1)(4x -7)(3x + 2)$
}
\litem{
Which of the following equations \textit{could} be of the graph presented below?

\begin{center}
    \includegraphics[width=0.5\textwidth]{../Figures/polyGraphToFunctionC.png}
\end{center}


The solution is \( -12x^{5} (x + 2)^{5} (x - 1)^{9} \), which is option C.\begin{enumerate}[label=\Alph*.]
\item \( 11x^{9} (x + 2)^{6} (x - 1)^{5} \)

The factor $(x + 2)$ should have an odd power and the leading coefficient should be the opposite sign.
\item \( 11x^{11} (x + 2)^{5} (x - 1)^{5} \)

This corresponds to the leading coefficient being the opposite value than it should be.
\item \( -12x^{5} (x + 2)^{5} (x - 1)^{9} \)

* This is the correct option.
\item \( -6x^{7} (x + 2)^{10} (x - 1)^{7} \)

The factor $-2$ should have been an odd power.
\item \( -9x^{6} (x + 2)^{4} (x - 1)^{7} \)

The factors $-2$ and $0$ have have been odd power.
\end{enumerate}

\textbf{General Comment:} General Comments: Draw the x-axis to determine which zeros are touching (and so have even multiplicity) or cross (and have odd multiplicity).
}
\litem{
Describe the zero behavior of the zero $x = -8$ of the polynomial below.
\[ f(x) = 3(x + 2)^{5}(x - 2)^{2}(x + 8)^{7}(x - 8)^{2} \]The solution is the graph below, which is option A.
    \begin{center}
        \includegraphics[width=0.3\textwidth]{../Figures/polyZeroBehaviorAC.png}
    \end{center}\begin{enumerate}[label=\Alph*.]
\begin{multicols}{2}
\item \includegraphics[width = 0.3\textwidth]{../Figures/polyZeroBehaviorAC.png}
\item \includegraphics[width = 0.3\textwidth]{../Figures/polyZeroBehaviorBC.png}
\item \includegraphics[width = 0.3\textwidth]{../Figures/polyZeroBehaviorCC.png}
\item \includegraphics[width = 0.3\textwidth]{../Figures/polyZeroBehaviorDC.png}
\end{multicols}\item None of the above.\end{enumerate}
\textbf{General Comment:} You will need to sketch the entire graph, then zoom in on the zero the question asks about.
}
\litem{
Describe the zero behavior of the zero $x = -5$ of the polynomial below.
\[ f(x) = 7(x - 5)^{2}(x + 5)^{5}(x + 9)^{8}(x - 9)^{11} \]The solution is the graph below, which is option A.
    \begin{center}
        \includegraphics[width=0.3\textwidth]{../Figures/polyZeroBehaviorCopyAC.png}
    \end{center}\begin{enumerate}[label=\Alph*.]
\begin{multicols}{2}
\item \includegraphics[width = 0.3\textwidth]{../Figures/polyZeroBehaviorCopyAC.png}
\item \includegraphics[width = 0.3\textwidth]{../Figures/polyZeroBehaviorCopyBC.png}
\item \includegraphics[width = 0.3\textwidth]{../Figures/polyZeroBehaviorCopyCC.png}
\item \includegraphics[width = 0.3\textwidth]{../Figures/polyZeroBehaviorCopyDC.png}
\end{multicols}\item None of the above.\end{enumerate}
\textbf{General Comment:} You will need to sketch the entire graph, then zoom in on the zero the question asks about.
}
\litem{
Construct the lowest-degree polynomial given the zeros below. Then, choose the intervals that contain the coefficients of the polynomial in the form $x^3+bx^2+cx+d$.
\[ -4 + 3 i \text{ and } 3 \]The solution is \( x^{3} +5 x^{2} +x -75 \), which is option B.\begin{enumerate}[label=\Alph*.]
\item \( b \in [-0.5, 2], c \in [-15, -4], \text{ and } d \in [7, 12] \)

$x^{3} + x^{2} -6 x + 9$, which corresponds to multiplying out $(x -3)(x -3)$.
\item \( b \in [3.6, 8.1], c \in [0, 5], \text{ and } d \in [-77, -74] \)

* $x^{3} +5 x^{2} +x -75$, which is the correct option.
\item \( b \in [-5.2, 0.5], c \in [0, 5], \text{ and } d \in [70, 77] \)

$x^{3} -5 x^{2} +x + 75$, which corresponds to multiplying out $(x-(-4 + 3 i))(x-(-4 - 3 i))(x + 3)$.
\item \( b \in [-0.5, 2], c \in [0, 5], \text{ and } d \in [-14, -5] \)

$x^{3} + x^{2} +x -12$, which corresponds to multiplying out $(x + 4)(x -3)$.
\item \( \text{None of the above.} \)

This corresponds to making an unanticipated error or not understanding how to use nonreal complex numbers to create the lowest-degree polynomial. If you chose this and are not sure what you did wrong, please contact the coordinator for help.
\end{enumerate}

\textbf{General Comment:} Remember that the conjugate of $a+bi$ is $a-bi$. Since these zeros always come in pairs, we need to multiply out $(x-(-4 + 3 i))(x-(-4 - 3 i))(x-(3))$.
}
\litem{
Describe the end behavior of the polynomial below.
\[ f(x) = 5(x - 6)^{2}(x + 6)^{3}(x + 3)^{5}(x - 3)^{7} \]The solution is the graph below, which is option D.
    \begin{center}
        \includegraphics[width=0.3\textwidth]{../Figures/polyEndBehaviorCopyDC.png}
    \end{center}\begin{enumerate}[label=\Alph*.]
\begin{multicols}{2}
\item \includegraphics[width = 0.3\textwidth]{../Figures/polyEndBehaviorCopyAC.png}
\item \includegraphics[width = 0.3\textwidth]{../Figures/polyEndBehaviorCopyBC.png}
\item \includegraphics[width = 0.3\textwidth]{../Figures/polyEndBehaviorCopyCC.png}
\item \includegraphics[width = 0.3\textwidth]{../Figures/polyEndBehaviorCopyDC.png}
\end{multicols}\item None of the above.\end{enumerate}
\textbf{General Comment:} Remember that end behavior is determined by the leading coefficient AND whether the \textbf{sum} of the multiplicities is positive or negative.
}
\end{enumerate}

\end{document}