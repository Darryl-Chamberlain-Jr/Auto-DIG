\documentclass[14pt]{extbook}
\usepackage{multicol, enumerate, enumitem, hyperref, color, soul, setspace, parskip, fancyhdr} %General Packages
\usepackage{amssymb, amsthm, amsmath, latexsym, units, mathtools} %Math Packages
\everymath{\displaystyle} %All math in Display Style
% Packages with additional options
\usepackage[headsep=0.5cm,headheight=12pt, left=1 in,right= 1 in,top= 1 in,bottom= 1 in]{geometry}
\usepackage[usenames,dvipsnames]{xcolor}
\usepackage{dashrule}  % Package to use the command below to create lines between items
\newcommand{\litem}[1]{\item#1\hspace*{-1cm}\rule{\textwidth}{0.4pt}}
\pagestyle{fancy}
\lhead{Progress Quiz 5}
\chead{}
\rhead{Version ALL}
\lfoot{8497-6012}
\cfoot{}
\rfoot{Summer C 2021}
\begin{document}

\begin{enumerate}
\litem{
Find the inverse of the function below. Then, evaluate the inverse at $x = 7$ and choose the interval that $f^-1(7)$ belongs to.\[ f(x) = e^{x+3}+3 \]\begin{enumerate}[label=\Alph*.]
\item \( f^{-1}(7) \in [4.3, 4.7] \)
\item \( f^{-1}(7) \in [4.4, 5.9] \)
\item \( f^{-1}(7) \in [4.4, 5.9] \)
\item \( f^{-1}(7) \in [4.3, 4.7] \)
\item \( f^{-1}(7) \in [-2.8, -1.3] \)

\end{enumerate} }
\litem{
Subtract the following functions, then choose the domain of the resulting function from the list below.\[ f(x) = 3x^{2} +5 x + 8 \text{ and } g(x) = \frac{2}{4x+27} \]\begin{enumerate}[label=\Alph*.]
\item \( \text{ The domain is all Real numbers greater than or equal to } x = a, \text{ where } a \in [-9.5, -2.5] \)
\item \( \text{ The domain is all Real numbers except } x = a, \text{ where } a \in [-13.75, -4.75] \)
\item \( \text{ The domain is all Real numbers less than or equal to } x = a, \text{ where } a \in [-2, 5] \)
\item \( \text{ The domain is all Real numbers except } x = a \text{ and } x = b, \text{ where } a \in [3.33, 9.33] \text{ and } b \in [-3.2, -2.2] \)
\item \( \text{ The domain is all Real numbers. } \)

\end{enumerate} }
\litem{
Find the inverse of the function below (if it exists). Then, evaluate the inverse at $x = -10$ and choose the interval that $f^-1(-10)$ belongs to.\[ f(x) = \sqrt[3]{2 x + 5} \]\begin{enumerate}[label=\Alph*.]
\item \( f^{-1}(-10) \in [-499.5, -496.5] \)
\item \( f^{-1}(-10) \in [501.1, 503.1] \)
\item \( f^{-1}(-10) \in [495.1, 500.2] \)
\item \( f^{-1}(-10) \in [-503.9, -500.4] \)
\item \( \text{ The function is not invertible for all Real numbers. } \)

\end{enumerate} }
\litem{
Determine whether the function below is 1-1.\[ f(x) = (5 x - 35)^3 \]\begin{enumerate}[label=\Alph*.]
\item \( \text{Yes, the function is 1-1.} \)
\item \( \text{No, because the range of the function is not $(-\infty, \infty)$.} \)
\item \( \text{No, because there is a $y$-value that goes to 2 different $x$-values.} \)
\item \( \text{No, because there is an $x$-value that goes to 2 different $y$-values.} \)
\item \( \text{No, because the domain of the function is not $(-\infty, \infty)$.} \)

\end{enumerate} }
\litem{
Add the following functions, then choose the domain of the resulting function from the list below.\[ f(x) = \frac{3}{6x+37} \text{ and } g(x) = x + 7 \]\begin{enumerate}[label=\Alph*.]
\item \( \text{ The domain is all Real numbers greater than or equal to } x = a, \text{ where } a \in [-6.5, -2.5] \)
\item \( \text{ The domain is all Real numbers except } x = a, \text{ where } a \in [-6.17, -2.17] \)
\item \( \text{ The domain is all Real numbers less than or equal to } x = a, \text{ where } a \in [0, 3] \)
\item \( \text{ The domain is all Real numbers except } x = a \text{ and } x = b, \text{ where } a \in [-3.8, 1.2] \text{ and } b \in [4.33, 8.33] \)
\item \( \text{ The domain is all Real numbers. } \)

\end{enumerate} }
\litem{
Find the inverse of the function below. Then, evaluate the inverse at $x = 10$ and choose the interval that $f^-1(10)$ belongs to.\[ f(x) = \ln{(x+4)}-5 \]\begin{enumerate}[label=\Alph*.]
\item \( f^{-1}(10) \in [142.41, 149.41] \)
\item \( f^{-1}(10) \in [396.43, 399.43] \)
\item \( f^{-1}(10) \in [3269012.37, 3269019.37] \)
\item \( f^{-1}(10) \in [1202597.28, 1202600.28] \)
\item \( f^{-1}(10) \in [3269019.37, 3269022.37] \)

\end{enumerate} }
\litem{
Choose the interval below that $f$ composed with $g$ at $x=-1$ is in.\[ f(x) = -2x^{3} + x^{2} +2 x \text{ and } g(x) = -x^{3} -2 x^{2} -3 x -4 \]\begin{enumerate}[label=\Alph*.]
\item \( (f \circ g)(-1) \in [13, 17] \)
\item \( (f \circ g)(-1) \in [-16, -6] \)
\item \( (f \circ g)(-1) \in [-8, -3] \)
\item \( (f \circ g)(-1) \in [5, 12] \)
\item \( \text{It is not possible to compose the two functions.} \)

\end{enumerate} }
\litem{
Find the inverse of the function below (if it exists). Then, evaluate the inverse at $x = -15$ and choose the interval that $f^-1(-15)$ belongs to.\[ f(x) = 3 x^2 - 4 \]\begin{enumerate}[label=\Alph*.]
\item \( f^{-1}(-15) \in [1.82, 1.95] \)
\item \( f^{-1}(-15) \in [6.9, 7.22] \)
\item \( f^{-1}(-15) \in [3.76, 4.2] \)
\item \( f^{-1}(-15) \in [2.22, 2.55] \)
\item \( \text{ The function is not invertible for all Real numbers. } \)

\end{enumerate} }
\litem{
Choose the interval below that $f$ composed with $g$ at $x=-1$ is in.\[ f(x) = 2x^{3} -2 x^{2} +2 x + 3 \text{ and } g(x) = x^{3} +2 x^{2} +3 x \]\begin{enumerate}[label=\Alph*.]
\item \( (f \circ g)(-1) \in [-18.39, -17.32] \)
\item \( (f \circ g)(-1) \in [-13.09, -12.75] \)
\item \( (f \circ g)(-1) \in [-25.46, -22.42] \)
\item \( (f \circ g)(-1) \in [-21.79, -18.43] \)
\item \( \text{It is not possible to compose the two functions.} \)

\end{enumerate} }
\litem{
Determine whether the function below is 1-1.\[ f(x) = (3 x + 21)^3 \]\begin{enumerate}[label=\Alph*.]
\item \( \text{No, because the range of the function is not $(-\infty, \infty)$.} \)
\item \( \text{No, because there is a $y$-value that goes to 2 different $x$-values.} \)
\item \( \text{No, because there is an $x$-value that goes to 2 different $y$-values.} \)
\item \( \text{Yes, the function is 1-1.} \)
\item \( \text{No, because the domain of the function is not $(-\infty, \infty)$.} \)

\end{enumerate} }
\litem{
Find the inverse of the function below. Then, evaluate the inverse at $x = 7$ and choose the interval that $f^-1(7)$ belongs to.\[ f(x) = \ln{(x+2)}-5 \]\begin{enumerate}[label=\Alph*.]
\item \( f^{-1}(7) \in [162750.79, 162754.79] \)
\item \( f^{-1}(7) \in [-0.61, 7.39] \)
\item \( f^{-1}(7) \in [143.41, 144.41] \)
\item \( f^{-1}(7) \in [162753.79, 162764.79] \)
\item \( f^{-1}(7) \in [8098.08, 8102.08] \)

\end{enumerate} }
\litem{
Multiply the following functions, then choose the domain of the resulting function from the list below.\[ f(x) = \frac{2}{4x+21} \text{ and } g(x) = \frac{2}{6x-23} \]\begin{enumerate}[label=\Alph*.]
\item \( \text{ The domain is all Real numbers except } x = a, \text{ where } a \in [0.4, 11.4] \)
\item \( \text{ The domain is all Real numbers greater than or equal to } x = a, \text{ where } a \in [1, 9] \)
\item \( \text{ The domain is all Real numbers less than or equal to } x = a, \text{ where } a \in [-6.67, -2.67] \)
\item \( \text{ The domain is all Real numbers except } x = a \text{ and } x = b, \text{ where } a \in [-7.25, -4.25] \text{ and } b \in [0.83, 7.83] \)
\item \( \text{ The domain is all Real numbers. } \)

\end{enumerate} }
\litem{
Find the inverse of the function below (if it exists). Then, evaluate the inverse at $x = 11$ and choose the interval that $f^-1(11)$ belongs to.\[ f(x) = \sqrt[3]{2 x + 3} \]\begin{enumerate}[label=\Alph*.]
\item \( f^{-1}(11) \in [663.3, 664.8] \)
\item \( f^{-1}(11) \in [664.9, 667.9] \)
\item \( f^{-1}(11) \in [-664.5, -661.8] \)
\item \( f^{-1}(11) \in [-669.6, -664.4] \)
\item \( \text{ The function is not invertible for all Real numbers. } \)

\end{enumerate} }
\litem{
Determine whether the function below is 1-1.\[ f(x) = -9 x^2 + 15 x + 234 \]\begin{enumerate}[label=\Alph*.]
\item \( \text{No, because the domain of the function is not $(-\infty, \infty)$.} \)
\item \( \text{No, because there is an $x$-value that goes to 2 different $y$-values.} \)
\item \( \text{No, because the range of the function is not $(-\infty, \infty)$.} \)
\item \( \text{Yes, the function is 1-1.} \)
\item \( \text{No, because there is a $y$-value that goes to 2 different $x$-values.} \)

\end{enumerate} }
\litem{
Add the following functions, then choose the domain of the resulting function from the list below.\[ f(x) = x + 6 \text{ and } g(x) = \frac{1}{4x-13} \]\begin{enumerate}[label=\Alph*.]
\item \( \text{ The domain is all Real numbers except } x = a, \text{ where } a \in [2.25, 6.25] \)
\item \( \text{ The domain is all Real numbers less than or equal to } x = a, \text{ where } a \in [-6.4, -2.4] \)
\item \( \text{ The domain is all Real numbers greater than or equal to } x = a, \text{ where } a \in [-6.75, -2.75] \)
\item \( \text{ The domain is all Real numbers except } x = a \text{ and } x = b, \text{ where } a \in [-12.33, 2.67] \text{ and } b \in [-8.67, -3.67] \)
\item \( \text{ The domain is all Real numbers. } \)

\end{enumerate} }
\litem{
Find the inverse of the function below. Then, evaluate the inverse at $x = 7$ and choose the interval that $f^-1(7)$ belongs to.\[ f(x) = e^{x-4}+5 \]\begin{enumerate}[label=\Alph*.]
\item \( f^{-1}(7) \in [6.02, 6.21] \)
\item \( f^{-1}(7) \in [4.66, 4.73] \)
\item \( f^{-1}(7) \in [7.35, 7.45] \)
\item \( f^{-1}(7) \in [-3.34, -3.28] \)
\item \( f^{-1}(7) \in [7.41, 7.5] \)

\end{enumerate} }
\litem{
Choose the interval below that $f$ composed with $g$ at $x=1$ is in.\[ f(x) = -2x^{3} -2 x^{2} +2 x \text{ and } g(x) = -2x^{3} -3 x^{2} +3 x + 1 \]\begin{enumerate}[label=\Alph*.]
\item \( (f \circ g)(1) \in [-1.78, -0.74] \)
\item \( (f \circ g)(1) \in [2.64, 3.93] \)
\item \( (f \circ g)(1) \in [-6.26, -5.68] \)
\item \( (f \circ g)(1) \in [-2.2, -1.72] \)
\item \( \text{It is not possible to compose the two functions.} \)

\end{enumerate} }
\litem{
Find the inverse of the function below (if it exists). Then, evaluate the inverse at $x = 12$ and choose the interval that $f^-1(12)$ belongs to.\[ f(x) = \sqrt[3]{3 x + 4} \]\begin{enumerate}[label=\Alph*.]
\item \( f^{-1}(12) \in [574, 576.8] \)
\item \( f^{-1}(12) \in [577.3, 578.8] \)
\item \( f^{-1}(12) \in [-580.7, -574.7] \)
\item \( f^{-1}(12) \in [-575.6, -573.9] \)
\item \( \text{ The function is not invertible for all Real numbers. } \)

\end{enumerate} }
\litem{
Choose the interval below that $f$ composed with $g$ at $x=1$ is in.\[ f(x) = -3x^{3} -2 x^{2} +3 x + 4 \text{ and } g(x) = x^{3} -2 x^{2} +3 x \]\begin{enumerate}[label=\Alph*.]
\item \( (f \circ g)(1) \in [-30, -24] \)
\item \( (f \circ g)(1) \in [6, 11] \)
\item \( (f \circ g)(1) \in [-26, -20] \)
\item \( (f \circ g)(1) \in [-6, 1] \)
\item \( \text{It is not possible to compose the two functions.} \)

\end{enumerate} }
\litem{
Determine whether the function below is 1-1.\[ f(x) = (4 x + 13)^3 \]\begin{enumerate}[label=\Alph*.]
\item \( \text{No, because the domain of the function is not $(-\infty, \infty)$.} \)
\item \( \text{No, because there is an $x$-value that goes to 2 different $y$-values.} \)
\item \( \text{Yes, the function is 1-1.} \)
\item \( \text{No, because there is a $y$-value that goes to 2 different $x$-values.} \)
\item \( \text{No, because the range of the function is not $(-\infty, \infty)$.} \)

\end{enumerate} }
\litem{
Find the inverse of the function below. Then, evaluate the inverse at $x = 8$ and choose the interval that $f^-1(8)$ belongs to.\[ f(x) = \ln{(x+5)}+2 \]\begin{enumerate}[label=\Alph*.]
\item \( f^{-1}(8) \in [22020.47, 22024.47] \)
\item \( f^{-1}(8) \in [21.09, 27.09] \)
\item \( f^{-1}(8) \in [442414.39, 442420.39] \)
\item \( f^{-1}(8) \in [405.43, 413.43] \)
\item \( f^{-1}(8) \in [388.43, 399.43] \)

\end{enumerate} }
\litem{
Add the following functions, then choose the domain of the resulting function from the list below.\[ f(x) = \sqrt{6x-28}  \text{ and } g(x) = x + 6 \]\begin{enumerate}[label=\Alph*.]
\item \( \text{ The domain is all Real numbers except } x = a, \text{ where } a \in [1.17, 5.17] \)
\item \( \text{ The domain is all Real numbers greater than or equal to } x = a, \text{ where } a \in [0.67, 5.67] \)
\item \( \text{ The domain is all Real numbers less than or equal to } x = a, \text{ where } a \in [-5.5, -0.5] \)
\item \( \text{ The domain is all Real numbers except } x = a \text{ and } x = b, \text{ where } a \in [-1.67, 4.33] \text{ and } b \in [-4.2, -3.2] \)
\item \( \text{ The domain is all Real numbers. } \)

\end{enumerate} }
\litem{
Find the inverse of the function below (if it exists). Then, evaluate the inverse at $x = -15$ and choose the interval that $f^-1(-15)$ belongs to.\[ f(x) = \sqrt[3]{3 x + 4} \]\begin{enumerate}[label=\Alph*.]
\item \( f^{-1}(-15) \in [1125.5, 1129.3] \)
\item \( f^{-1}(-15) \in [1122.7, 1126] \)
\item \( f^{-1}(-15) \in [-1125.4, -1122.4] \)
\item \( f^{-1}(-15) \in [-1129, -1126.3] \)
\item \( \text{ The function is not invertible for all Real numbers. } \)

\end{enumerate} }
\litem{
Determine whether the function below is 1-1.\[ f(x) = 9 x^2 - 39 x - 230 \]\begin{enumerate}[label=\Alph*.]
\item \( \text{No, because the domain of the function is not $(-\infty, \infty)$.} \)
\item \( \text{No, because there is a $y$-value that goes to 2 different $x$-values.} \)
\item \( \text{Yes, the function is 1-1.} \)
\item \( \text{No, because the range of the function is not $(-\infty, \infty)$.} \)
\item \( \text{No, because there is an $x$-value that goes to 2 different $y$-values.} \)

\end{enumerate} }
\litem{
Multiply the following functions, then choose the domain of the resulting function from the list below.\[ f(x) = \frac{3}{3x-16} \text{ and } g(x) = \frac{2}{3x+16} \]\begin{enumerate}[label=\Alph*.]
\item \( \text{ The domain is all Real numbers less than or equal to } x = a, \text{ where } a \in [-6.6, 5.4] \)
\item \( \text{ The domain is all Real numbers except } x = a, \text{ where } a \in [-8.25, -4.25] \)
\item \( \text{ The domain is all Real numbers greater than or equal to } x = a, \text{ where } a \in [5, 13] \)
\item \( \text{ The domain is all Real numbers except } x = a \text{ and } x = b, \text{ where } a \in [0.33, 6.33] \text{ and } b \in [-11.33, -1.33] \)
\item \( \text{ The domain is all Real numbers. } \)

\end{enumerate} }
\litem{
Find the inverse of the function below. Then, evaluate the inverse at $x = 4$ and choose the interval that $f^-1(4)$ belongs to.\[ f(x) = e^{x+2}+2 \]\begin{enumerate}[label=\Alph*.]
\item \( f^{-1}(4) \in [-0.7, 2.7] \)
\item \( f^{-1}(4) \in [-3.5, -0.7] \)
\item \( f^{-1}(4) \in [-0.7, 2.7] \)
\item \( f^{-1}(4) \in [2.8, 5.5] \)
\item \( f^{-1}(4) \in [2.8, 5.5] \)

\end{enumerate} }
\litem{
Choose the interval below that $f$ composed with $g$ at $x=-1$ is in.\[ f(x) = -4x^{3} -2 x^{2} +4 x -1 \text{ and } g(x) = -2x^{3} -2 x^{2} -x \]\begin{enumerate}[label=\Alph*.]
\item \( (f \circ g)(-1) \in [42, 47] \)
\item \( (f \circ g)(-1) \in [38, 40] \)
\item \( (f \circ g)(-1) \in [4, 6] \)
\item \( (f \circ g)(-1) \in [-12, 0] \)
\item \( \text{It is not possible to compose the two functions.} \)

\end{enumerate} }
\litem{
Find the inverse of the function below (if it exists). Then, evaluate the inverse at $x = 13$ and choose the interval that $f^-1(13)$ belongs to.\[ f(x) = \sqrt[3]{5 x + 3} \]\begin{enumerate}[label=\Alph*.]
\item \( f^{-1}(13) \in [438.67, 438.81] \)
\item \( f^{-1}(13) \in [439.45, 441.34] \)
\item \( f^{-1}(13) \in [-439.21, -438.65] \)
\item \( f^{-1}(13) \in [-440.29, -439.8] \)
\item \( \text{ The function is not invertible for all Real numbers. } \)

\end{enumerate} }
\litem{
Choose the interval below that $f$ composed with $g$ at $x=-1$ is in.\[ f(x) = -2x^{3} +3 x^{2} +4 x \text{ and } g(x) = 3x^{3} -1 x^{2} -2 x \]\begin{enumerate}[label=\Alph*.]
\item \( (f \circ g)(-1) \in [24, 37] \)
\item \( (f \circ g)(-1) \in [20, 21] \)
\item \( (f \circ g)(-1) \in [1, 14] \)
\item \( (f \circ g)(-1) \in [-5, 3] \)
\item \( \text{It is not possible to compose the two functions.} \)

\end{enumerate} }
\litem{
Determine whether the function below is 1-1.\[ f(x) = 15 x^2 - 56 x - 396 \]\begin{enumerate}[label=\Alph*.]
\item \( \text{No, because there is a $y$-value that goes to 2 different $x$-values.} \)
\item \( \text{No, because the range of the function is not $(-\infty, \infty)$.} \)
\item \( \text{No, because there is an $x$-value that goes to 2 different $y$-values.} \)
\item \( \text{No, because the domain of the function is not $(-\infty, \infty)$.} \)
\item \( \text{Yes, the function is 1-1.} \)

\end{enumerate} }
\end{enumerate}

\end{document}