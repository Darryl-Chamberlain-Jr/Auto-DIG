\documentclass[14pt]{extbook}
\usepackage{multicol, enumerate, enumitem, hyperref, color, soul, setspace, parskip, fancyhdr} %General Packages
\usepackage{amssymb, amsthm, amsmath, latexsym, units, mathtools} %Math Packages
\everymath{\displaystyle} %All math in Display Style
% Packages with additional options
\usepackage[headsep=0.5cm,headheight=12pt, left=1 in,right= 1 in,top= 1 in,bottom= 1 in]{geometry}
\usepackage[usenames,dvipsnames]{xcolor}
\usepackage{dashrule}  % Package to use the command below to create lines between items
\newcommand{\litem}[1]{\item#1\hspace*{-1cm}\rule{\textwidth}{0.4pt}}
\pagestyle{fancy}
\lhead{Progress Quiz 5}
\chead{}
\rhead{Version A}
\lfoot{8497-6012}
\cfoot{}
\rfoot{Summer C 2021}
\begin{document}

\begin{enumerate}
\litem{
Simplify the expression below into the form $a+bi$. Then, choose the intervals that $a$ and $b$ belong to.\[ (-9 - 10 i)(-3 - 2 i) \]\begin{enumerate}[label=\Alph*.]
\item \( a \in [44, 52] \text{ and } b \in [-13, -7] \)
\item \( a \in [5, 13] \text{ and } b \in [45, 52] \)
\item \( a \in [44, 52] \text{ and } b \in [9, 16] \)
\item \( a \in [5, 13] \text{ and } b \in [-48, -41] \)
\item \( a \in [27, 35] \text{ and } b \in [14, 25] \)

\end{enumerate} }
\litem{
Choose the \textbf{smallest} set of Real numbers that the number below belongs to.\[ -\sqrt{\frac{12}{0}} \]\begin{enumerate}[label=\Alph*.]
\item \( \text{Whole} \)
\item \( \text{Irrational} \)
\item \( \text{Not a Real number} \)
\item \( \text{Integer} \)
\item \( \text{Rational} \)

\end{enumerate} }
\litem{
Simplify the expression below into the form $a+bi$. Then, choose the intervals that $a$ and $b$ belong to.\[ (-7 - 4 i)(6 - 3 i) \]\begin{enumerate}[label=\Alph*.]
\item \( a \in [-44, -41] \text{ and } b \in [7, 16] \)
\item \( a \in [-32, -28] \text{ and } b \in [42, 49] \)
\item \( a \in [-57, -52] \text{ and } b \in [3, 5] \)
\item \( a \in [-32, -28] \text{ and } b \in [-47, -40] \)
\item \( a \in [-57, -52] \text{ and } b \in [-7, -2] \)

\end{enumerate} }
\litem{
Simplify the expression below into the form $a+bi$. Then, choose the intervals that $a$ and $b$ belong to.\[ \frac{36 + 33 i}{6 - 8 i} \]\begin{enumerate}[label=\Alph*.]
\item \( a \in [4.4, 5.3] \text{ and } b \in [-1.5, 0.5] \)
\item \( a \in [-48.35, -47.25] \text{ and } b \in [4, 6.5] \)
\item \( a \in [-0.9, 0.45] \text{ and } b \in [485.5, 486.5] \)
\item \( a \in [-0.9, 0.45] \text{ and } b \in [4, 6.5] \)
\item \( a \in [5.95, 6.45] \text{ and } b \in [-5, -3] \)

\end{enumerate} }
\litem{
Simplify the expression below into the form $a+bi$. Then, choose the intervals that $a$ and $b$ belong to.\[ \frac{-9 + 22 i}{-3 + 4 i} \]\begin{enumerate}[label=\Alph*.]
\item \( a \in [3.5, 5] \text{ and } b \in [-2, -1] \)
\item \( a \in [3.5, 5] \text{ and } b \in [-30.5, -29] \)
\item \( a \in [114.5, 115.5] \text{ and } b \in [-2, -1] \)
\item \( a \in [2.5, 3.5] \text{ and } b \in [3.5, 6.5] \)
\item \( a \in [-3, -2] \text{ and } b \in [-5.5, -3.5] \)

\end{enumerate} }
\litem{
Choose the \textbf{smallest} set of Complex numbers that the number below belongs to.\[ \sqrt{\frac{0}{49}}+\sqrt{4}i \]\begin{enumerate}[label=\Alph*.]
\item \( \text{Pure Imaginary} \)
\item \( \text{Nonreal Complex} \)
\item \( \text{Irrational} \)
\item \( \text{Rational} \)
\item \( \text{Not a Complex Number} \)

\end{enumerate} }
\litem{
Simplify the expression below and choose the interval the simplification is contained within.\[ 5 - 16^2 + 19 \div 4 * 11 \div 20 \]\begin{enumerate}[label=\Alph*.]
\item \( [262.9, 263.9] \)
\item \( [-255.4, -249.1] \)
\item \( [-249.7, -247.8] \)
\item \( [259.8, 261.5] \)
\item \( \text{None of the above} \)

\end{enumerate} }
\litem{
Choose the \textbf{smallest} set of Real numbers that the number below belongs to.\[ -\sqrt{\frac{1430}{10}} \]\begin{enumerate}[label=\Alph*.]
\item \( \text{Whole} \)
\item \( \text{Rational} \)
\item \( \text{Irrational} \)
\item \( \text{Integer} \)
\item \( \text{Not a Real number} \)

\end{enumerate} }
\litem{
Choose the \textbf{smallest} set of Complex numbers that the number below belongs to.\[ \frac{16}{16}+81i^2 \]\begin{enumerate}[label=\Alph*.]
\item \( \text{Rational} \)
\item \( \text{Not a Complex Number} \)
\item \( \text{Pure Imaginary} \)
\item \( \text{Nonreal Complex} \)
\item \( \text{Irrational} \)

\end{enumerate} }
\litem{
Simplify the expression below and choose the interval the simplification is contained within.\[ 2 - 10 \div 7 * 16 - (14 * 4) \]\begin{enumerate}[label=\Alph*.]
\item \( [-57.09, -50.09] \)
\item \( [-78.86, -75.86] \)
\item \( [49.91, 63.91] \)
\item \( [-139.43, -138.43] \)
\item \( \text{None of the above} \)

\end{enumerate} }
\end{enumerate}

\end{document}