\documentclass{extbook}[14pt]
\usepackage{multicol, enumerate, enumitem, hyperref, color, soul, setspace, parskip, fancyhdr, amssymb, amsthm, amsmath, latexsym, units, mathtools}
\everymath{\displaystyle}
\usepackage[headsep=0.5cm,headheight=0cm, left=1 in,right= 1 in,top= 1 in,bottom= 1 in]{geometry}
\usepackage{dashrule}  % Package to use the command below to create lines between items
\newcommand{\litem}[1]{\item #1

\rule{\textwidth}{0.4pt}}
\pagestyle{fancy}
\lhead{}
\chead{Answer Key for Progress Quiz 5 Version B}
\rhead{}
\lfoot{8497-6012}
\cfoot{}
\rfoot{Summer C 2021}
\begin{document}
\textbf{This key should allow you to understand why you choose the option you did (beyond just getting a question right or wrong). \href{https://xronos.clas.ufl.edu/mac1105spring2020/courseDescriptionAndMisc/Exams/LearningFromResults}{More instructions on how to use this key can be found here}.}

\textbf{If you have a suggestion to make the keys better, \href{https://forms.gle/CZkbZmPbC9XALEE88}{please fill out the short survey here}.}

\textit{Note: This key is auto-generated and may contain issues and/or errors. The keys are reviewed after each exam to ensure grading is done accurately. If there are issues (like duplicate options), they are noted in the offline gradebook. The keys are a work-in-progress to give students as many resources to improve as possible.}

\rule{\textwidth}{0.4pt}

\begin{enumerate}\litem{
Solve the equation for $x$ and choose the interval that contains the solution (if it exists).
\[ 2^{3x-2} = \left(\frac{1}{25}\right)^{2x-5} \]The solution is \( x = 2.052 \), which is option C.\begin{enumerate}[label=\Alph*.]
\item \( x \in [14.48, 20.48] \)

$x = 17.481$, which corresponds to distributing the $\ln(base)$ to the second term of the exponent only.
\item \( x \in [-4, -2] \)

$x = -3.000$, which corresponds to solving the numerators as equal while ignoring the bases are different.
\item \( x \in [1.05, 3.05] \)

* $x = 2.052$, which is the correct option.
\item \( x \in [-1.35, 0.65] \)

$x = -0.352$, which corresponds to distributing the $\ln(base)$ to the first term of the exponent only.
\item \( \text{There is no Real solution to the equation.} \)

This corresponds to believing there is no solution since the bases are not powers of each other.
\end{enumerate}

\textbf{General Comment:} \textbf{General Comments:} This question was written so that the bases could not be written the same. You will need to take the log of both sides.
}
\litem{
Solve the equation for $x$ and choose the interval that contains the solution (if it exists).
\[ \log_{4}{(2x+6)}+5 = 3 \]The solution is \( x = -2.969 \), which is option D.\begin{enumerate}[label=\Alph*.]
\item \( x \in [27, 30] \)

$x = 29.000$, which corresponds to ignoring the vertical shift when converting to exponential form.
\item \( x \in [5, 10] \)

$x = 5.000$, which corresponds to reversing the base and exponent when converting.
\item \( x \in [10, 21] \)

$x = 11.000$, which corresponds to reversing the base and exponent when converting and reversing the value with $x$.
\item \( x \in [-7.97, -0.97] \)

* $x = -2.969$, which is the correct option.
\item \( \text{There is no Real solution to the equation.} \)

Corresponds to believing a negative coefficient within the log equation means there is no Real solution.
\end{enumerate}

\textbf{General Comment:} \textbf{General Comments:} First, get the equation in the form $\log_b{(cx+d)} = a$. Then, convert to $b^a = cx+d$ and solve.
}
\litem{
Which of the following intervals describes the Domain of the function below?
\[ f(x) = \log_2{(x+2)}+5 \]The solution is \( (-2, \infty) \), which is option A.\begin{enumerate}[label=\Alph*.]
\item \( (a, \infty), a \in [-2.6, -1.5] \)

* $(-2, \infty)$, which is the correct option.
\item \( (-\infty, a), a \in [1, 4.8] \)

$(-\infty, 2)$, which corresponds to flipping the Domain. Remember: the general for is $a*\log(x-h)+k$, \textbf{where $a$ does not affect the domain}.
\item \( (-\infty, a], a \in [-5.2, -2.1] \)

$(-\infty, -5]$, which corresponds to using the negative vertical shift AND including the endpoint AND flipping the domain.
\item \( [a, \infty), a \in [3.6, 6] \)

$[5, \infty)$, which corresponds to using the vertical shift when shifting the Domain AND including the endpoint.
\item \( (-\infty, \infty) \)

This corresponds to thinking of the range of the log function (or the domain of the exponential function).
\end{enumerate}

\textbf{General Comment:} \textbf{General Comments}: The domain of a basic logarithmic function is $(0, \infty)$ and the Range is $(-\infty, \infty)$. We can use shifts when finding the Domain, but the Range will always be all Real numbers.
}
\litem{
Solve the equation for $x$ and choose the interval that contains the solution (if it exists).
\[ 5^{-3x-5} = 9^{-5x+5} \]The solution is \( x = 3.091 \), which is option D.\begin{enumerate}[label=\Alph*.]
\item \( x \in [1.3, 2.9] \)

$x = 1.624$, which corresponds to distributing the $\ln(base)$ to the first term of the exponent only.
\item \( x \in [7.3, 9.7] \)

$x = 9.517$, which corresponds to distributing the $\ln(base)$ to the second term of the exponent only.
\item \( x \in [3.7, 5.7] \)

$x = 5.000$, which corresponds to solving the numerators as equal while ignoring the bases are different.
\item \( x \in [3, 4] \)

* $x = 3.091$, which is the correct option.
\item \( \text{There is no Real solution to the equation.} \)

This corresponds to believing there is no solution since the bases are not powers of each other.
\end{enumerate}

\textbf{General Comment:} \textbf{General Comments:} This question was written so that the bases could not be written the same. You will need to take the log of both sides.
}
\litem{
Which of the following intervals describes the Range of the function below?
\[ f(x) = -\log_2{(x+4)}-9 \]The solution is \( (\infty, \infty) \), which is option E.\begin{enumerate}[label=\Alph*.]
\item \( [a, \infty), a \in [2, 5] \)

$[4, \infty)$, which corresponds to using the negative of the horizontal shift AND including the endpoint.
\item \( (-\infty, a), a \in [-14, -8] \)

$(-\infty, -9)$, which corresponds to using the vertical shift while the Range is $(-\infty, \infty)$.
\item \( [a, \infty), a \in [-6, 1] \)

$[-9, \infty)$, which corresponds to using the flipped Domain AND including the endpoint.
\item \( (-\infty, a), a \in [8, 15] \)

$(-\infty, 9)$, which corresponds to using the using the negative of vertical shift on $(0, \infty)$.
\item \( (-\infty, \infty) \)

*This is the correct option.
\end{enumerate}

\textbf{General Comment:} \textbf{General Comments}: The domain of a basic logarithmic function is $(0, \infty)$ and the Range is $(-\infty, \infty)$. We can use shifts when finding the Domain, but the Range will always be all Real numbers.
}
\litem{
Which of the following intervals describes the Domain of the function below?
\[ f(x) = e^{x+3}-4 \]The solution is \( (-\infty, \infty) \), which is option E.\begin{enumerate}[label=\Alph*.]
\item \( (-\infty, a), a \in [-9, 2] \)

$(-\infty, -4)$, which corresponds to using the correct vertical shift *if we wanted the Range*.
\item \( [a, \infty), a \in [1, 7] \)

$[4, \infty)$, which corresponds to using the negative vertical shift AND flipping the Range interval AND including the endpoint.
\item \( (a, \infty), a \in [1, 7] \)

$(4, \infty)$, which corresponds to using the negative vertical shift AND flipping the Range interval.
\item \( (-\infty, a], a \in [-9, 2] \)

$(-\infty, -4]$, which corresponds to using the correct vertical shift *if we wanted the Range* AND including the endpoint.
\item \( (-\infty, \infty) \)

* This is the correct option.
\end{enumerate}

\textbf{General Comment:} \textbf{General Comments}: Domain of a basic exponential function is $(-\infty, \infty)$ while the Range is $(0, \infty)$. We can shift these intervals [and even flip when $a<0$!] to find the new Domain/Range.
}
\litem{
Solve the equation for $x$ and choose the interval that contains the solution (if it exists).
\[ \log_{4}{(-2x+6)}+6 = 3 \]The solution is \( x = 2.992 \), which is option A.\begin{enumerate}[label=\Alph*.]
\item \( x \in [0.99, 3.99] \)

* $x = 2.992$, which is the correct option.
\item \( x \in [-45.5, -38.5] \)

$x = -43.500$, which corresponds to reversing the base and exponent when converting and reversing the value with $x$.
\item \( x \in [-29, -27] \)

$x = -29.000$, which corresponds to ignoring the vertical shift when converting to exponential form.
\item \( x \in [-38.5, -33.5] \)

$x = -37.500$, which corresponds to reversing the base and exponent when converting.
\item \( \text{There is no Real solution to the equation.} \)

Corresponds to believing a negative coefficient within the log equation means there is no Real solution.
\end{enumerate}

\textbf{General Comment:} \textbf{General Comments:} First, get the equation in the form $\log_b{(cx+d)} = a$. Then, convert to $b^a = cx+d$ and solve.
}
\litem{
Which of the following intervals describes the Range of the function below?
\[ f(x) = -e^{x+2}+9 \]The solution is \( (-\infty, 9) \), which is option B.\begin{enumerate}[label=\Alph*.]
\item \( [a, \infty), a \in [-11, -6] \)

$[-9, \infty)$, which corresponds to using the negative vertical shift AND flipping the Range interval AND including the endpoint.
\item \( (-\infty, a), a \in [7, 14] \)

* $(-\infty, 9)$, which is the correct option.
\item \( (a, \infty), a \in [-11, -6] \)

$(-9, \infty)$, which corresponds to using the negative vertical shift AND flipping the Range interval.
\item \( (-\infty, a], a \in [7, 14] \)

$(-\infty, 9]$, which corresponds to including the endpoint.
\item \( (-\infty, \infty) \)

This corresponds to confusing range of an exponential function with the domain of an exponential function.
\end{enumerate}

\textbf{General Comment:} \textbf{General Comments}: Domain of a basic exponential function is $(-\infty, \infty)$ while the Range is $(0, \infty)$. We can shift these intervals [and even flip when $a<0$!] to find the new Domain/Range.
}
\litem{
 Solve the equation for $x$ and choose the interval that contains $x$ (if it exists).
\[  23 = \ln{\sqrt[3]{\frac{28}{e^{8x}}}} \]The solution is \( x = -8.208 \), which is option C.\begin{enumerate}[label=\Alph*.]
\item \( x \in [-2.59, -0.59] \)

$x = -1.592$, which corresponds to thinking you need to take the natural log of on the left before reducing.
\item \( x \in [-7.33, -4.33] \)

$x = -5.333$, which corresponds to treating any root as a square root.
\item \( x \in [-8.21, -6.21] \)

* $x = -8.208$, which is the correct option.
\item \( \text{There is no Real solution to the equation.} \)

This corresponds to believing you cannot solve the equation.
\item \( \text{None of the above.} \)

This corresponds to making an unexpected error.
\end{enumerate}

\textbf{General Comment:} \textbf{General Comments}: After using the properties of logarithmic functions to break up the right-hand side, use $\ln(e) = 1$ to reduce the question to a linear function to solve. You can put $\ln(28)$ into a calculator if you are having trouble.
}
\litem{
 Solve the equation for $x$ and choose the interval that contains $x$ (if it exists).
\[  5 = \sqrt[3]{\frac{24}{e^{7x}}} \]The solution is \( x = -0.236, \text{ which does not fit in any of the interval options.} \), which is option E.\begin{enumerate}[label=\Alph*.]
\item \( x \in [-2.71, -2.53] \)

$x = -2.597$, which corresponds to thinking you don't need to take the natural log of both sides before reducing, as if the right side already has a natural log.
\item \( x \in [-0.11, 0.1] \)

$x = -0.006$, which corresponds to treating any root as a square root.
\item \( x \in [0.11, 0.24] \)

$x = 0.236$, which is the negative of the correct solution.
\item \( \text{There is no Real solution to the equation.} \)

This corresponds to believing you cannot solve the equation.
\item \( \text{None of the above.} \)

* $x = -0.236$ is the correct solution and does not fit in any of the other intervals.
\end{enumerate}

\textbf{General Comment:} \textbf{General Comments}: After using the properties of logarithmic functions to break up the right-hand side, use $\ln(e) = 1$ to reduce the question to a linear function to solve. You can put $\ln(24)$ into a calculator if you are having trouble.
}
\end{enumerate}

\end{document}