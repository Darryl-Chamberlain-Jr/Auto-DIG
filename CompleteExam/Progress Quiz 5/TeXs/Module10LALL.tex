\documentclass[14pt]{extbook}
\usepackage{multicol, enumerate, enumitem, hyperref, color, soul, setspace, parskip, fancyhdr} %General Packages
\usepackage{amssymb, amsthm, amsmath, latexsym, units, mathtools} %Math Packages
\everymath{\displaystyle} %All math in Display Style
% Packages with additional options
\usepackage[headsep=0.5cm,headheight=12pt, left=1 in,right= 1 in,top= 1 in,bottom= 1 in]{geometry}
\usepackage[usenames,dvipsnames]{xcolor}
\usepackage{dashrule}  % Package to use the command below to create lines between items
\newcommand{\litem}[1]{\item#1\hspace*{-1cm}\rule{\textwidth}{0.4pt}}
\pagestyle{fancy}
\lhead{Progress Quiz 5}
\chead{}
\rhead{Version ALL}
\lfoot{8497-6012}
\cfoot{}
\rfoot{Summer C 2021}
\begin{document}

\begin{enumerate}
\litem{
Perform the division below. Then, find the intervals that correspond to the quotient in the form $ax^2+bx+c$ and remainder $r$.\[ \frac{12x^{3} +39 x^{2} -30}{x + 3} \]\begin{enumerate}[label=\Alph*.]
\item \( a \in [-38, -33], b \in [147, 149], c \in [-441, -436], \text{ and } r \in [1291, 1296]. \)
\item \( a \in [11, 13], b \in [75, 77], c \in [220, 232], \text{ and } r \in [644, 646]. \)
\item \( a \in [11, 13], b \in [-3, 5], c \in [-12, -2], \text{ and } r \in [-7, 2]. \)
\item \( a \in [-38, -33], b \in [-70, -65], c \in [-207, -199], \text{ and } r \in [-654, -650]. \)
\item \( a \in [11, 13], b \in [-14, -8], c \in [29, 38], \text{ and } r \in [-181, -171]. \)

\end{enumerate} }
\litem{
Factor the polynomial below completely. Then, choose the intervals the zeros of the polynomial belong to, where $z_1 \leq z_2 \leq z_3$. \textit{To make the problem easier, all zeros are between -5 and 5.}\[ f(x) = 15x^{3} -44 x^{2} -79 x + 60 \]\begin{enumerate}[label=\Alph*.]
\item \( z_1 \in [-5, -2], \text{   }  z_2 \in [-0.9, -0.2], \text{   and   } z_3 \in [0.92, 1.85] \)
\item \( z_1 \in [-5, -2], \text{   }  z_2 \in [-2.2, -0.9], \text{   and   } z_3 \in [0.42, 0.84] \)
\item \( z_1 \in [-0.6, 0.4], \text{   }  z_2 \in [1.1, 2.8], \text{   and   } z_3 \in [3.7, 4.42] \)
\item \( z_1 \in [-1.67, -0.67], \text{   }  z_2 \in [-0.3, 0.8], \text{   and   } z_3 \in [3.7, 4.42] \)
\item \( z_1 \in [-5, -2], \text{   }  z_2 \in [-3.3, -2.1], \text{   and   } z_3 \in [-0.02, 0.34] \)

\end{enumerate} }
\litem{
Factor the polynomial below completely. Then, choose the intervals the zeros of the polynomial belong to, where $z_1 \leq z_2 \leq z_3$. \textit{To make the problem easier, all zeros are between -5 and 5.}\[ f(x) = 10x^{3} +3 x^{2} -79 x -60 \]\begin{enumerate}[label=\Alph*.]
\item \( z_1 \in [-3.1, -2.7], \text{   }  z_2 \in [0, 0.71], \text{   and   } z_3 \in [4.9, 5.12] \)
\item \( z_1 \in [-3.1, -2.7], \text{   }  z_2 \in [0, 0.71], \text{   and   } z_3 \in [1.12, 1.66] \)
\item \( z_1 \in [-2.8, -1.6], \text{   }  z_2 \in [-0.89, -0.5], \text{   and   } z_3 \in [2.53, 3.23] \)
\item \( z_1 \in [-3.1, -2.7], \text{   }  z_2 \in [0.74, 1.16], \text{   and   } z_3 \in [2.35, 2.78] \)
\item \( z_1 \in [-1.5, -0.9], \text{   }  z_2 \in [-0.58, -0.22], \text{   and   } z_3 \in [2.53, 3.23] \)

\end{enumerate} }
\litem{
Perform the division below. Then, find the intervals that correspond to the quotient in the form $ax^2+bx+c$ and remainder $r$.\[ \frac{4x^{3} -75 x -129}{x -5} \]\begin{enumerate}[label=\Alph*.]
\item \( a \in [2, 7], b \in [12, 18], c \in [-14, -6], \text{ and } r \in [-181, -171]. \)
\item \( a \in [2, 7], b \in [-26, -14], c \in [21, 26], \text{ and } r \in [-257, -246]. \)
\item \( a \in [17, 22], b \in [-103, -96], c \in [424, 427], \text{ and } r \in [-2255, -2252]. \)
\item \( a \in [17, 22], b \in [96, 105], c \in [424, 427], \text{ and } r \in [1992, 1999]. \)
\item \( a \in [2, 7], b \in [17, 23], c \in [21, 26], \text{ and } r \in [-7, -3]. \)

\end{enumerate} }
\litem{
Perform the division below. Then, find the intervals that correspond to the quotient in the form $ax^2+bx+c$ and remainder $r$.\[ \frac{20x^{3} +55 x^{2} -30 x -43}{x + 3} \]\begin{enumerate}[label=\Alph*.]
\item \( a \in [-62, -56], \text{   } b \in [-130, -124], \text{   } c \in [-406, -400], \text{   and   } r \in [-1263, -1256]. \)
\item \( a \in [19, 26], \text{   } b \in [111, 121], \text{   } c \in [310, 319], \text{   and   } r \in [898, 908]. \)
\item \( a \in [-62, -56], \text{   } b \in [231, 237], \text{   } c \in [-735, -733], \text{   and   } r \in [2160, 2164]. \)
\item \( a \in [19, 26], \text{   } b \in [-26, -22], \text{   } c \in [66, 73], \text{   and   } r \in [-327, -319]. \)
\item \( a \in [19, 26], \text{   } b \in [-6, -1], \text{   } c \in [-18, -14], \text{   and   } r \in [1, 8]. \)

\end{enumerate} }
\litem{
Factor the polynomial below completely, knowing that $x -5$ is a factor. Then, choose the intervals the zeros of the polynomial belong to, where $z_1 \leq z_2 \leq z_3 \leq z_4$. \textit{To make the problem easier, all zeros are between -5 and 5.}\[ f(x) = 10x^{4} -113 x^{3} +434 x^{2} -655 x + 300 \]\begin{enumerate}[label=\Alph*.]
\item \( z_1 \in [-1.5, 0.7], \text{   }  z_2 \in [0.88, 2.15], z_3 \in [2.83, 3.07], \text{   and   } z_4 \in [4.76, 5.22] \)
\item \( z_1 \in [-6.1, -4.5], \text{   }  z_2 \in [-3.06, -1.3], z_3 \in [-1.58, -0.95], \text{   and   } z_4 \in [-0.43, -0.37] \)
\item \( z_1 \in [0.5, 0.9], \text{   }  z_2 \in [2.3, 2.76], z_3 \in [2.83, 3.07], \text{   and   } z_4 \in [4.76, 5.22] \)
\item \( z_1 \in [-6.1, -4.5], \text{   }  z_2 \in [-3.06, -1.3], z_3 \in [-2.56, -2.44], \text{   and   } z_4 \in [-0.91, -0.66] \)
\item \( z_1 \in [-6.1, -4.5], \text{   }  z_2 \in [-4.78, -3.6], z_3 \in [-3.23, -2.63], \text{   and   } z_4 \in [-0.61, -0.48] \)

\end{enumerate} }
\litem{
Factor the polynomial below completely, knowing that $x -2$ is a factor. Then, choose the intervals the zeros of the polynomial belong to, where $z_1 \leq z_2 \leq z_3 \leq z_4$. \textit{To make the problem easier, all zeros are between -5 and 5.}\[ f(x) = 12x^{4} -83 x^{3} +197 x^{2} -188 x + 60 \]\begin{enumerate}[label=\Alph*.]
\item \( z_1 \in [-3.21, -2.92], \text{   }  z_2 \in [-2.11, -1.9], z_3 \in [-1.87, -1.4], \text{   and   } z_4 \in [-0.97, -0.76] \)
\item \( z_1 \in [-3.21, -2.92], \text{   }  z_2 \in [-2.11, -1.9], z_3 \in [-2.06, -1.61], \text{   and   } z_4 \in [-0.48, -0.26] \)
\item \( z_1 \in [0.79, 1.04], \text{   }  z_2 \in [1.45, 1.69], z_3 \in [1.79, 2.39], \text{   and   } z_4 \in [2.98, 3.13] \)
\item \( z_1 \in [-3.21, -2.92], \text{   }  z_2 \in [-2.11, -1.9], z_3 \in [-1.42, -1.16], \text{   and   } z_4 \in [-0.73, -0.63] \)
\item \( z_1 \in [0.42, 0.78], \text{   }  z_2 \in [0.65, 1.49], z_3 \in [1.79, 2.39], \text{   and   } z_4 \in [2.98, 3.13] \)

\end{enumerate} }
\litem{
Perform the division below. Then, find the intervals that correspond to the quotient in the form $ax^2+bx+c$ and remainder $r$.\[ \frac{10x^{3} -46 x^{2} +40 x + 22}{x -3} \]\begin{enumerate}[label=\Alph*.]
\item \( a \in [29, 35], \text{   } b \in [41, 48], \text{   } c \in [169, 174], \text{   and   } r \in [534, 540]. \)
\item \( a \in [10, 11], \text{   } b \in [-18, -9], \text{   } c \in [-8, -7], \text{   and   } r \in [-5, 2]. \)
\item \( a \in [10, 11], \text{   } b \in [-76, -75], \text{   } c \in [265, 271], \text{   and   } r \in [-787, -778]. \)
\item \( a \in [29, 35], \text{   } b \in [-137, -132], \text{   } c \in [448, 452], \text{   and   } r \in [-1326, -1315]. \)
\item \( a \in [10, 11], \text{   } b \in [-30, -22], \text{   } c \in [-12, -9], \text{   and   } r \in [-5, 2]. \)

\end{enumerate} }
\litem{
What are the \textit{possible Integer} roots of the polynomial below?\[ f(x) = 6x^{2} +7 x + 7 \]\begin{enumerate}[label=\Alph*.]
\item \( \pm 1,\pm 2,\pm 3,\pm 6 \)
\item \( \pm 1,\pm 7 \)
\item \( \text{ All combinations of: }\frac{\pm 1,\pm 2,\pm 3,\pm 6}{\pm 1,\pm 7} \)
\item \( \text{ All combinations of: }\frac{\pm 1,\pm 7}{\pm 1,\pm 2,\pm 3,\pm 6} \)
\item \( \text{There is no formula or theorem that tells us all possible Integer roots.} \)

\end{enumerate} }
\litem{
What are the \textit{possible Integer} roots of the polynomial below?\[ f(x) = 6x^{2} +6 x + 4 \]\begin{enumerate}[label=\Alph*.]
\item \( \pm 1,\pm 2,\pm 4 \)
\item \( \pm 1,\pm 2,\pm 3,\pm 6 \)
\item \( \text{ All combinations of: }\frac{\pm 1,\pm 2,\pm 3,\pm 6}{\pm 1,\pm 2,\pm 4} \)
\item \( \text{ All combinations of: }\frac{\pm 1,\pm 2,\pm 4}{\pm 1,\pm 2,\pm 3,\pm 6} \)
\item \( \text{There is no formula or theorem that tells us all possible Integer roots.} \)

\end{enumerate} }
\litem{
Perform the division below. Then, find the intervals that correspond to the quotient in the form $ax^2+bx+c$ and remainder $r$.\[ \frac{20x^{3} +105 x^{2} -128}{x + 5} \]\begin{enumerate}[label=\Alph*.]
\item \( a \in [19, 27], b \in [-15, -11], c \in [89, 92], \text{ and } r \in [-670, -662]. \)
\item \( a \in [19, 27], b \in [2, 11], c \in [-30, -24], \text{ and } r \in [-7, -2]. \)
\item \( a \in [-105, -94], b \in [-397, -394], c \in [-1976, -1973], \text{ and } r \in [-10008, -9998]. \)
\item \( a \in [-105, -94], b \in [602, 607], c \in [-3027, -3024], \text{ and } r \in [14989, 15000]. \)
\item \( a \in [19, 27], b \in [203, 206], c \in [1023, 1026], \text{ and } r \in [4997, 5002]. \)

\end{enumerate} }
\litem{
Factor the polynomial below completely. Then, choose the intervals the zeros of the polynomial belong to, where $z_1 \leq z_2 \leq z_3$. \textit{To make the problem easier, all zeros are between -5 and 5.}\[ f(x) = 10x^{3} -39 x^{2} -61 x + 30 \]\begin{enumerate}[label=\Alph*.]
\item \( z_1 \in [-2.4, -0.9], \text{   }  z_2 \in [0.36, 0.97], \text{   and   } z_3 \in [4.87, 5.67] \)
\item \( z_1 \in [-5.1, -4.1], \text{   }  z_2 \in [-0.78, -0.09], \text{   and   } z_3 \in [0.97, 1.69] \)
\item \( z_1 \in [-1.4, 0.1], \text{   }  z_2 \in [2.08, 3.12], \text{   and   } z_3 \in [4.87, 5.67] \)
\item \( z_1 \in [-5.1, -4.1], \text{   }  z_2 \in [-3.19, -2.32], \text{   and   } z_3 \in [0.45, 0.75] \)
\item \( z_1 \in [-5.1, -4.1], \text{   }  z_2 \in [-2.36, -1.87], \text{   and   } z_3 \in [0.14, 0.36] \)

\end{enumerate} }
\litem{
Factor the polynomial below completely. Then, choose the intervals the zeros of the polynomial belong to, where $z_1 \leq z_2 \leq z_3$. \textit{To make the problem easier, all zeros are between -5 and 5.}\[ f(x) = 15x^{3} -1 x^{2} -52 x + 20 \]\begin{enumerate}[label=\Alph*.]
\item \( z_1 \in [-1.85, -1.24], \text{   }  z_2 \in [-0.43, -0.35], \text{   and   } z_3 \in [1.81, 2.03] \)
\item \( z_1 \in [-2.78, -2.35], \text{   }  z_2 \in [-0.6, -0.46], \text{   and   } z_3 \in [1.81, 2.03] \)
\item \( z_1 \in [-5.02, -4.61], \text{   }  z_2 \in [-0.17, -0.02], \text{   and   } z_3 \in [1.81, 2.03] \)
\item \( z_1 \in [-2.33, -1.98], \text{   }  z_2 \in [0.59, 0.64], \text{   and   } z_3 \in [2.15, 2.76] \)
\item \( z_1 \in [-2.33, -1.98], \text{   }  z_2 \in [0.38, 0.48], \text{   and   } z_3 \in [1.36, 1.85] \)

\end{enumerate} }
\litem{
Perform the division below. Then, find the intervals that correspond to the quotient in the form $ax^2+bx+c$ and remainder $r$.\[ \frac{8x^{3} -62 x + 33}{x + 3} \]\begin{enumerate}[label=\Alph*.]
\item \( a \in [4, 9], b \in [-39, -31], c \in [62, 69], \text{ and } r \in [-232, -225]. \)
\item \( a \in [-27, -21], b \in [-72, -67], c \in [-280, -277], \text{ and } r \in [-804, -800]. \)
\item \( a \in [4, 9], b \in [20, 26], c \in [7, 15], \text{ and } r \in [58, 66]. \)
\item \( a \in [-27, -21], b \in [71, 77], c \in [-280, -277], \text{ and } r \in [867, 868]. \)
\item \( a \in [4, 9], b \in [-28, -21], c \in [7, 15], \text{ and } r \in [2, 5]. \)

\end{enumerate} }
\litem{
Perform the division below. Then, find the intervals that correspond to the quotient in the form $ax^2+bx+c$ and remainder $r$.\[ \frac{4x^{3} -22 x^{2} +4 x + 26}{x -5} \]\begin{enumerate}[label=\Alph*.]
\item \( a \in [2, 5], \text{   } b \in [-2, 2], \text{   } c \in [-6, -5], \text{   and   } r \in [-7, -1]. \)
\item \( a \in [20, 23], \text{   } b \in [75, 79], \text{   } c \in [394, 399], \text{   and   } r \in [1991, 1997]. \)
\item \( a \in [2, 5], \text{   } b \in [-8, -5], \text{   } c \in [-24, -18], \text{   and   } r \in [-58, -52]. \)
\item \( a \in [2, 5], \text{   } b \in [-43, -39], \text{   } c \in [213, 221], \text{   and   } r \in [-1044, -1043]. \)
\item \( a \in [20, 23], \text{   } b \in [-125, -115], \text{   } c \in [610, 618], \text{   and   } r \in [-3050, -3036]. \)

\end{enumerate} }
\litem{
Factor the polynomial below completely, knowing that $x + 3$ is a factor. Then, choose the intervals the zeros of the polynomial belong to, where $z_1 \leq z_2 \leq z_3 \leq z_4$. \textit{To make the problem easier, all zeros are between -5 and 5.}\[ f(x) = 4x^{4} +4 x^{3} -51 x^{2} -36 x + 135 \]\begin{enumerate}[label=\Alph*.]
\item \( z_1 \in [-5, 1], \text{   }  z_2 \in [-2.54, -2.45], z_3 \in [1.24, 1.53], \text{   and   } z_4 \in [3, 4] \)
\item \( z_1 \in [-5, 1], \text{   }  z_2 \in [-0.8, -0.68], z_3 \in [2.74, 3.16], \text{   and   } z_4 \in [5, 7] \)
\item \( z_1 \in [-5, 1], \text{   }  z_2 \in [-0.72, -0.56], z_3 \in [0.32, 0.59], \text{   and   } z_4 \in [3, 4] \)
\item \( z_1 \in [-5, 1], \text{   }  z_2 \in [-0.5, -0.35], z_3 \in [0.42, 0.9], \text{   and   } z_4 \in [3, 4] \)
\item \( z_1 \in [-5, 1], \text{   }  z_2 \in [-1.5, -1.46], z_3 \in [2.24, 2.69], \text{   and   } z_4 \in [3, 4] \)

\end{enumerate} }
\litem{
Factor the polynomial below completely, knowing that $x + 4$ is a factor. Then, choose the intervals the zeros of the polynomial belong to, where $z_1 \leq z_2 \leq z_3 \leq z_4$. \textit{To make the problem easier, all zeros are between -5 and 5.}\[ f(x) = 12x^{4} +101 x^{3} +165 x^{2} -248 x -240 \]\begin{enumerate}[label=\Alph*.]
\item \( z_1 \in [-0.46, 0.02], \text{   }  z_2 \in [2.74, 3.09], z_3 \in [3.87, 4.03], \text{   and   } z_4 \in [3.99, 5.65] \)
\item \( z_1 \in [-5.22, -4.73], \text{   }  z_2 \in [-4.54, -3.29], z_3 \in [-2.25, -0.9], \text{   and   } z_4 \in [-0.17, 1] \)
\item \( z_1 \in [-1.56, -0.95], \text{   }  z_2 \in [0.63, 0.84], z_3 \in [3.87, 4.03], \text{   and   } z_4 \in [3.99, 5.65] \)
\item \( z_1 \in [-0.96, -0.61], \text{   }  z_2 \in [1.26, 1.46], z_3 \in [3.87, 4.03], \text{   and   } z_4 \in [3.99, 5.65] \)
\item \( z_1 \in [-5.22, -4.73], \text{   }  z_2 \in [-4.54, -3.29], z_3 \in [-1, -0.5], \text{   and   } z_4 \in [0.79, 1.62] \)

\end{enumerate} }
\litem{
Perform the division below. Then, find the intervals that correspond to the quotient in the form $ax^2+bx+c$ and remainder $r$.\[ \frac{25x^{3} -85 x^{2} +15 x + 40}{x -3} \]\begin{enumerate}[label=\Alph*.]
\item \( a \in [73, 76], \text{   } b \in [-314, -306], \text{   } c \in [945, 951], \text{   and   } r \in [-2795, -2791]. \)
\item \( a \in [25, 26], \text{   } b \in [-163, -157], \text{   } c \in [492, 496], \text{   and   } r \in [-1445, -1441]. \)
\item \( a \in [73, 76], \text{   } b \in [136, 145], \text{   } c \in [432, 438], \text{   and   } r \in [1340, 1346]. \)
\item \( a \in [25, 26], \text{   } b \in [-42, -31], \text{   } c \in [-60, -51], \text{   and   } r \in [-71, -65]. \)
\item \( a \in [25, 26], \text{   } b \in [-19, -9], \text{   } c \in [-17, -12], \text{   and   } r \in [-5, -1]. \)

\end{enumerate} }
\litem{
What are the \textit{possible Integer} roots of the polynomial below?\[ f(x) = 3x^{2} +5 x + 4 \]\begin{enumerate}[label=\Alph*.]
\item \( \pm 1,\pm 2,\pm 4 \)
\item \( \text{ All combinations of: }\frac{\pm 1,\pm 3}{\pm 1,\pm 2,\pm 4} \)
\item \( \text{ All combinations of: }\frac{\pm 1,\pm 2,\pm 4}{\pm 1,\pm 3} \)
\item \( \pm 1,\pm 3 \)
\item \( \text{There is no formula or theorem that tells us all possible Integer roots.} \)

\end{enumerate} }
\litem{
What are the \textit{possible Integer} roots of the polynomial below?\[ f(x) = 5x^{2} +5 x + 2 \]\begin{enumerate}[label=\Alph*.]
\item \( \text{ All combinations of: }\frac{\pm 1,\pm 5}{\pm 1,\pm 2} \)
\item \( \pm 1,\pm 2 \)
\item \( \pm 1,\pm 5 \)
\item \( \text{ All combinations of: }\frac{\pm 1,\pm 2}{\pm 1,\pm 5} \)
\item \( \text{There is no formula or theorem that tells us all possible Integer roots.} \)

\end{enumerate} }
\litem{
Perform the division below. Then, find the intervals that correspond to the quotient in the form $ax^2+bx+c$ and remainder $r$.\[ \frac{16x^{3} -49 x + 32}{x + 2} \]\begin{enumerate}[label=\Alph*.]
\item \( a \in [16, 18], b \in [31, 38], c \in [12, 17], \text{ and } r \in [59, 67]. \)
\item \( a \in [-34, -25], b \in [-69, -63], c \in [-182, -175], \text{ and } r \in [-324, -318]. \)
\item \( a \in [-34, -25], b \in [57, 67], c \in [-182, -175], \text{ and } r \in [385, 392]. \)
\item \( a \in [16, 18], b \in [-49, -47], c \in [91, 99], \text{ and } r \in [-253, -246]. \)
\item \( a \in [16, 18], b \in [-40, -28], c \in [12, 17], \text{ and } r \in [-1, 5]. \)

\end{enumerate} }
\litem{
Factor the polynomial below completely. Then, choose the intervals the zeros of the polynomial belong to, where $z_1 \leq z_2 \leq z_3$. \textit{To make the problem easier, all zeros are between -5 and 5.}\[ f(x) = 6x^{3} -35 x^{2} +66 x -40 \]\begin{enumerate}[label=\Alph*.]
\item \( z_1 \in [-2.69, -2.1], \text{   }  z_2 \in [-3, -1.8], \text{   and   } z_3 \in [-1.45, -1.14] \)
\item \( z_1 \in [-5.19, -4.42], \text{   }  z_2 \in [-3, -1.8], \text{   and   } z_3 \in [-0.8, -0.42] \)
\item \( z_1 \in [1.1, 1.67], \text{   }  z_2 \in [1, 2.5], \text{   and   } z_3 \in [2.32, 2.71] \)
\item \( z_1 \in [-2.18, -1.47], \text{   }  z_2 \in [-1.1, -0.6], \text{   and   } z_3 \in [-0.62, -0.23] \)
\item \( z_1 \in [0.05, 0.53], \text{   }  z_2 \in [0.4, 1.5], \text{   and   } z_3 \in [1.95, 2.11] \)

\end{enumerate} }
\litem{
Factor the polynomial below completely. Then, choose the intervals the zeros of the polynomial belong to, where $z_1 \leq z_2 \leq z_3$. \textit{To make the problem easier, all zeros are between -5 and 5.}\[ f(x) = 25x^{3} -45 x^{2} -82 x -24 \]\begin{enumerate}[label=\Alph*.]
\item \( z_1 \in [-3.11, -2.79], \text{   }  z_2 \in [0.24, 0.6], \text{   and   } z_3 \in [0.38, 0.88] \)
\item \( z_1 \in [-3.11, -2.79], \text{   }  z_2 \in [1.07, 1.31], \text{   and   } z_3 \in [2.14, 2.54] \)
\item \( z_1 \in [-1.16, -0.39], \text{   }  z_2 \in [-0.67, -0.28], \text{   and   } z_3 \in [2.54, 3.27] \)
\item \( z_1 \in [-2.65, -2.14], \text{   }  z_2 \in [-1.42, -1.09], \text{   and   } z_3 \in [2.54, 3.27] \)
\item \( z_1 \in [-3.11, -2.79], \text{   }  z_2 \in [0.09, 0.19], \text{   and   } z_3 \in [1.46, 2.4] \)

\end{enumerate} }
\litem{
Perform the division below. Then, find the intervals that correspond to the quotient in the form $ax^2+bx+c$ and remainder $r$.\[ \frac{10x^{3} -35 x^{2} + 42}{x -3} \]\begin{enumerate}[label=\Alph*.]
\item \( a \in [28, 31], b \in [54, 58], c \in [160, 169], \text{ and } r \in [535, 539]. \)
\item \( a \in [28, 31], b \in [-126, -122], c \in [369, 376], \text{ and } r \in [-1084, -1081]. \)
\item \( a \in [5, 15], b \in [-6, -2], c \in [-20, -6], \text{ and } r \in [-5, 1]. \)
\item \( a \in [5, 15], b \in [-17, -7], c \in [-34, -25], \text{ and } r \in [-20, -12]. \)
\item \( a \in [5, 15], b \in [-65, -61], c \in [193, 197], \text{ and } r \in [-545, -541]. \)

\end{enumerate} }
\litem{
Perform the division below. Then, find the intervals that correspond to the quotient in the form $ax^2+bx+c$ and remainder $r$.\[ \frac{12x^{3} -34 x^{2} -10 x + 7}{x -3} \]\begin{enumerate}[label=\Alph*.]
\item \( a \in [31, 39], \text{   } b \in [71, 77], \text{   } c \in [211, 218], \text{   and   } r \in [643, 650]. \)
\item \( a \in [10, 17], \text{   } b \in [-11, -8], \text{   } c \in [-30, -25], \text{   and   } r \in [-58, -51]. \)
\item \( a \in [10, 17], \text{   } b \in [-2, 3], \text{   } c \in [-5, -2], \text{   and   } r \in [-6, 0]. \)
\item \( a \in [10, 17], \text{   } b \in [-75, -64], \text{   } c \in [194, 203], \text{   and   } r \in [-596, -584]. \)
\item \( a \in [31, 39], \text{   } b \in [-148, -138], \text{   } c \in [415, 418], \text{   and   } r \in [-1243, -1237]. \)

\end{enumerate} }
\litem{
Factor the polynomial below completely, knowing that $x + 3$ is a factor. Then, choose the intervals the zeros of the polynomial belong to, where $z_1 \leq z_2 \leq z_3 \leq z_4$. \textit{To make the problem easier, all zeros are between -5 and 5.}\[ f(x) = 8x^{4} +26 x^{3} -37 x^{2} -159 x -90 \]\begin{enumerate}[label=\Alph*.]
\item \( z_1 \in [-1.23, -0.19], \text{   }  z_2 \in [0.77, 1.43], z_3 \in [1.38, 2.29], \text{   and   } z_4 \in [2.6, 4.3] \)
\item \( z_1 \in [-5.3, -4.32], \text{   }  z_2 \in [0.15, 0.47], z_3 \in [1.38, 2.29], \text{   and   } z_4 \in [2.6, 4.3] \)
\item \( z_1 \in [-3.63, -2.76], \text{   }  z_2 \in [-2.18, -1.85], z_3 \in [-0.83, -0.16], \text{   and   } z_4 \in [0.6, 2.9] \)
\item \( z_1 \in [-3.63, -2.76], \text{   }  z_2 \in [-2.18, -1.85], z_3 \in [-1.36, -0.79], \text{   and   } z_4 \in [-0.4, 1.4] \)
\item \( z_1 \in [-2.68, -2.27], \text{   }  z_2 \in [0.61, 0.85], z_3 \in [1.38, 2.29], \text{   and   } z_4 \in [2.6, 4.3] \)

\end{enumerate} }
\litem{
Factor the polynomial below completely, knowing that $x + 2$ is a factor. Then, choose the intervals the zeros of the polynomial belong to, where $z_1 \leq z_2 \leq z_3 \leq z_4$. \textit{To make the problem easier, all zeros are between -5 and 5.}\[ f(x) = 12x^{4} -29 x^{3} -33 x^{2} +116 x -60 \]\begin{enumerate}[label=\Alph*.]
\item \( z_1 \in [-2.5, -1.9], \text{   }  z_2 \in [-1.75, -1.65], z_3 \in [-0.91, -0.74], \text{   and   } z_4 \in [1, 5] \)
\item \( z_1 \in [-2.5, -1.9], \text{   }  z_2 \in [0.67, 0.79], z_3 \in [1.58, 1.81], \text{   and   } z_4 \in [1, 5] \)
\item \( z_1 \in [-2.5, -1.9], \text{   }  z_2 \in [0.58, 0.66], z_3 \in [1.23, 1.34], \text{   and   } z_4 \in [1, 5] \)
\item \( z_1 \in [-2.5, -1.9], \text{   }  z_2 \in [-1.42, -1.31], z_3 \in [-0.72, -0.45], \text{   and   } z_4 \in [1, 5] \)
\item \( z_1 \in [-3.2, -2.7], \text{   }  z_2 \in [-2.01, -1.99], z_3 \in [-0.58, -0.21], \text{   and   } z_4 \in [1, 5] \)

\end{enumerate} }
\litem{
Perform the division below. Then, find the intervals that correspond to the quotient in the form $ax^2+bx+c$ and remainder $r$.\[ \frac{20x^{3} -45 x^{2} -15 x + 45}{x -2} \]\begin{enumerate}[label=\Alph*.]
\item \( a \in [18, 23], \text{   } b \in [-8, -2], \text{   } c \in [-30, -22], \text{   and   } r \in [-5, -2]. \)
\item \( a \in [40, 42], \text{   } b \in [-130, -123], \text{   } c \in [233, 239], \text{   and   } r \in [-425, -423]. \)
\item \( a \in [18, 23], \text{   } b \in [-87, -83], \text{   } c \in [152, 156], \text{   and   } r \in [-269, -264]. \)
\item \( a \in [18, 23], \text{   } b \in [-27, -22], \text{   } c \in [-40, -39], \text{   and   } r \in [5, 10]. \)
\item \( a \in [40, 42], \text{   } b \in [31, 36], \text{   } c \in [52, 57], \text{   and   } r \in [155, 161]. \)

\end{enumerate} }
\litem{
What are the \textit{possible Rational} roots of the polynomial below?\[ f(x) = 6x^{3} +2 x^{2} +2 x + 2 \]\begin{enumerate}[label=\Alph*.]
\item \( \pm 1,\pm 2 \)
\item \( \pm 1,\pm 2,\pm 3,\pm 6 \)
\item \( \text{ All combinations of: }\frac{\pm 1,\pm 2}{\pm 1,\pm 2,\pm 3,\pm 6} \)
\item \( \text{ All combinations of: }\frac{\pm 1,\pm 2,\pm 3,\pm 6}{\pm 1,\pm 2} \)
\item \( \text{ There is no formula or theorem that tells us all possible Rational roots.} \)

\end{enumerate} }
\litem{
What are the \textit{possible Integer} roots of the polynomial below?\[ f(x) = 3x^{4} +2 x^{3} +6 x^{2} +7 x + 7 \]\begin{enumerate}[label=\Alph*.]
\item \( \text{ All combinations of: }\frac{\pm 1,\pm 3}{\pm 1,\pm 7} \)
\item \( \text{ All combinations of: }\frac{\pm 1,\pm 7}{\pm 1,\pm 3} \)
\item \( \pm 1,\pm 7 \)
\item \( \pm 1,\pm 3 \)
\item \( \text{There is no formula or theorem that tells us all possible Integer roots.} \)

\end{enumerate} }
\end{enumerate}

\end{document}