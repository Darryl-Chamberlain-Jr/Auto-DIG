\documentclass{extbook}[14pt]
\usepackage{multicol, enumerate, enumitem, hyperref, color, soul, setspace, parskip, fancyhdr, amssymb, amsthm, amsmath, bbm, latexsym, units, mathtools}
\everymath{\displaystyle}
\usepackage[headsep=0.5cm,headheight=0cm, left=1 in,right= 1 in,top= 1 in,bottom= 1 in]{geometry}
\usepackage{dashrule}  % Package to use the command below to create lines between items
\newcommand{\litem}[1]{\item #1

\rule{\textwidth}{0.4pt}}
\pagestyle{fancy}
\lhead{}
\chead{Answer Key for Progress Quiz 5 Version B}
\rhead{}
\lfoot{9912-2038}
\cfoot{}
\rfoot{Spring 2021}
\begin{document}
\textbf{This key should allow you to understand why you choose the option you did (beyond just getting a question right or wrong). \href{https://xronos.clas.ufl.edu/mac1105spring2020/courseDescriptionAndMisc/Exams/LearningFromResults}{More instructions on how to use this key can be found here}.}

\textbf{If you have a suggestion to make the keys better, \href{https://forms.gle/CZkbZmPbC9XALEE88}{please fill out the short survey here}.}

\textit{Note: This key is auto-generated and may contain issues and/or errors. The keys are reviewed after each exam to ensure grading is done accurately. If there are issues (like duplicate options), they are noted in the offline gradebook. The keys are a work-in-progress to give students as many resources to improve as possible.}

\rule{\textwidth}{0.4pt}

\begin{enumerate}\litem{
What are the \textit{possible Integer} roots of the polynomial below?
\[ f(x) = 6x^{3} +4 x^{2} +6 x + 7 \]The solution is \( \pm 1,\pm 7 \), which is option D.\begin{enumerate}[label=\Alph*.]
\item \( \text{ All combinations of: }\frac{\pm 1,\pm 7}{\pm 1,\pm 2,\pm 3,\pm 6} \)

This would have been the solution \textbf{if asked for the possible Rational roots}!
\item \( \text{ All combinations of: }\frac{\pm 1,\pm 2,\pm 3,\pm 6}{\pm 1,\pm 7} \)

 Distractor 3: Corresponds to the plus or minus of the inverse quotient (an/a0) of the factors. 
\item \( \pm 1,\pm 2,\pm 3,\pm 6 \)

 Distractor 1: Corresponds to the plus or minus factors of a1 only.
\item \( \pm 1,\pm 7 \)

* This is the solution \textbf{since we asked for the possible Integer roots}!
\item \( \text{There is no formula or theorem that tells us all possible Integer roots.} \)

 Distractor 4: Corresponds to not recognizing Integers as a subset of Rationals.
\end{enumerate}

\textbf{General Comment:} We have a way to find the possible Rational roots. The possible Integer roots are the Integers in this list.
}
\litem{
Factor the polynomial below completely. Then, choose the intervals the zeros of the polynomial belong to, where $z_1 \leq z_2 \leq z_3$. \textit{To make the problem easier, all zeros are between -5 and 5.}
\[ f(x) = 15x^{3} +91 x^{2} +84 x + 20 \]The solution is \( [-5, -0.6666666666666666, -0.4] \), which is option C.\begin{enumerate}[label=\Alph*.]
\item \( z_1 \in [0.07, 0.25], \text{   }  z_2 \in [1.86, 2.35], \text{   and   } z_3 \in [2, 7] \)

 Distractor 4: Corresponds to moving factors from one rational to another.
\item \( z_1 \in [-5.17, -4.99], \text{   }  z_2 \in [-2.52, -2.48], \text{   and   } z_3 \in [-1.5, -0.5] \)

 Distractor 2: Corresponds to inversing rational roots.
\item \( z_1 \in [-5.17, -4.99], \text{   }  z_2 \in [-1.06, -0.1], \text{   and   } z_3 \in [-1.4, 1.6] \)

* This is the solution!
\item \( z_1 \in [1.27, 1.51], \text{   }  z_2 \in [2.02, 3.25], \text{   and   } z_3 \in [2, 7] \)

 Distractor 3: Corresponds to negatives of all zeros AND inversing rational roots.
\item \( z_1 \in [0.32, 0.61], \text{   }  z_2 \in [0.61, 1.51], \text{   and   } z_3 \in [2, 7] \)

 Distractor 1: Corresponds to negatives of all zeros.
\end{enumerate}

\textbf{General Comment:} Remember to try the middle-most integers first as these normally are the zeros. Also, once you get it to a quadratic, you can use your other factoring techniques to finish factoring.
}
\litem{
Perform the division below. Then, find the intervals that correspond to the quotient in the form $ax^2+bx+c$ and remainder $r$.
\[ \frac{10x^{3} +30 x^{2} -35}{x + 2} \]The solution is \( 10x^{2} +10 x -20 + \frac{5}{x + 2} \), which is option E.\begin{enumerate}[label=\Alph*.]
\item \( a \in [-22, -18], b \in [68, 72], c \in [-144, -136], \text{ and } r \in [241, 251]. \)

 You multipled by the synthetic number rather than bringing the first factor down.
\item \( a \in [-22, -18], b \in [-10, -6], c \in [-28, -18], \text{ and } r \in [-81, -72]. \)

 You divided by the opposite of the factor AND multipled the first factor rather than just bringing it down.
\item \( a \in [2, 11], b \in [48, 55], c \in [98, 107], \text{ and } r \in [158, 170]. \)

 You divided by the opposite of the factor.
\item \( a \in [2, 11], b \in [-1, 6], c \in [-2, 3], \text{ and } r \in [-37, -31]. \)

 You multipled by the synthetic number and subtracted rather than adding during synthetic division.
\item \( a \in [2, 11], b \in [4, 11], c \in [-28, -18], \text{ and } r \in [-1, 9]. \)

* This is the solution!
\end{enumerate}

\textbf{General Comment:} Be sure to synthetically divide by the zero of the denominator! Also, make sure to include 0 placeholders for missing terms.
}
\litem{
Factor the polynomial below completely. Then, choose the intervals the zeros of the polynomial belong to, where $z_1 \leq z_2 \leq z_3$. \textit{To make the problem easier, all zeros are between -5 and 5.}
\[ f(x) = 12x^{3} +35 x^{2} +7 x -30 \]The solution is \( [-2, -1.6666666666666667, 0.75] \), which is option A.\begin{enumerate}[label=\Alph*.]
\item \( z_1 \in [-2.03, -1.79], \text{   }  z_2 \in [-2.18, -1.66], \text{   and   } z_3 \in [0.5, 1.3] \)

* This is the solution!
\item \( z_1 \in [-0.9, -0.7], \text{   }  z_2 \in [1.65, 1.76], \text{   and   } z_3 \in [1.8, 3.1] \)

 Distractor 1: Corresponds to negatives of all zeros.
\item \( z_1 \in [-0.57, 0.04], \text{   }  z_2 \in [1.99, 2.02], \text{   and   } z_3 \in [4, 6] \)

 Distractor 4: Corresponds to moving factors from one rational to another.
\item \( z_1 \in [-2.03, -1.79], \text{   }  z_2 \in [-0.82, -0.31], \text{   and   } z_3 \in [1.3, 1.5] \)

 Distractor 2: Corresponds to inversing rational roots.
\item \( z_1 \in [-1.75, -1.19], \text{   }  z_2 \in [0.57, 0.61], \text{   and   } z_3 \in [1.8, 3.1] \)

 Distractor 3: Corresponds to negatives of all zeros AND inversing rational roots.
\end{enumerate}

\textbf{General Comment:} Remember to try the middle-most integers first as these normally are the zeros. Also, once you get it to a quadratic, you can use your other factoring techniques to finish factoring.
}
\litem{
Perform the division below. Then, find the intervals that correspond to the quotient in the form $ax^2+bx+c$ and remainder $r$.
\[ \frac{6x^{3} +26 x^{2} -30}{x + 4} \]The solution is \( 6x^{2} +2 x -8 + \frac{2}{x + 4} \), which is option A.\begin{enumerate}[label=\Alph*.]
\item \( a \in [4, 8], b \in [1, 8], c \in [-15, -6], \text{ and } r \in [-3, 6]. \)

* This is the solution!
\item \( a \in [4, 8], b \in [45, 55], c \in [198, 205], \text{ and } r \in [763, 774]. \)

 You divided by the opposite of the factor.
\item \( a \in [4, 8], b \in [-6, 0], c \in [18, 25], \text{ and } r \in [-139, -127]. \)

 You multipled by the synthetic number and subtracted rather than adding during synthetic division.
\item \( a \in [-27, -23], b \in [-71, -64], c \in [-284, -277], \text{ and } r \in [-1154, -1149]. \)

 You divided by the opposite of the factor AND multipled the first factor rather than just bringing it down.
\item \( a \in [-27, -23], b \in [118, 129], c \in [-490, -487], \text{ and } r \in [1922, 1924]. \)

 You multipled by the synthetic number rather than bringing the first factor down.
\end{enumerate}

\textbf{General Comment:} Be sure to synthetically divide by the zero of the denominator! Also, make sure to include 0 placeholders for missing terms.
}
\litem{
Perform the division below. Then, find the intervals that correspond to the quotient in the form $ax^2+bx+c$ and remainder $r$.
\[ \frac{8x^{3} -18 x^{2} -6 x + 15}{x -2} \]The solution is \( 8x^{2} -2 x -10 + \frac{-5}{x -2} \), which is option B.\begin{enumerate}[label=\Alph*.]
\item \( a \in [14, 17], \text{   } b \in [13, 21], \text{   } c \in [21, 26], \text{   and   } r \in [57, 66]. \)

 You multiplied by the synthetic number rather than bringing the first factor down.
\item \( a \in [7, 10], \text{   } b \in [-4, 2], \text{   } c \in [-15, -5], \text{   and   } r \in [-6, -2]. \)

* This is the solution!
\item \( a \in [7, 10], \text{   } b \in [-38, -32], \text{   } c \in [56, 66], \text{   and   } r \in [-110, -108]. \)

 You divided by the opposite of the factor.
\item \( a \in [7, 10], \text{   } b \in [-14, -6], \text{   } c \in [-21, -14], \text{   and   } r \in [-3, 6]. \)

 You multiplied by the synthetic number and subtracted rather than adding during synthetic division.
\item \( a \in [14, 17], \text{   } b \in [-50, -46], \text{   } c \in [89, 95], \text{   and   } r \in [-179, -163]. \)

 You divided by the opposite of the factor AND multiplied the first factor rather than just bringing it down.
\end{enumerate}

\textbf{General Comment:} Be sure to synthetically divide by the zero of the denominator!
}
\litem{
Perform the division below. Then, find the intervals that correspond to the quotient in the form $ax^2+bx+c$ and remainder $r$.
\[ \frac{8x^{3} -22 x^{2} -21 x + 49}{x -3} \]The solution is \( 8x^{2} +2 x -15 + \frac{4}{x -3} \), which is option A.\begin{enumerate}[label=\Alph*.]
\item \( a \in [4, 14], \text{   } b \in [-1, 3], \text{   } c \in [-21, -13], \text{   and   } r \in [3, 8]. \)

* This is the solution!
\item \( a \in [4, 14], \text{   } b \in [-6, 0], \text{   } c \in [-33, -31], \text{   and   } r \in [-17, -13]. \)

 You multiplied by the synthetic number and subtracted rather than adding during synthetic division.
\item \( a \in [21, 27], \text{   } b \in [45, 52], \text{   } c \in [127, 131], \text{   and   } r \in [436, 443]. \)

 You multiplied by the synthetic number rather than bringing the first factor down.
\item \( a \in [4, 14], \text{   } b \in [-49, -45], \text{   } c \in [117, 124], \text{   and   } r \in [-302, -299]. \)

 You divided by the opposite of the factor.
\item \( a \in [21, 27], \text{   } b \in [-98, -90], \text{   } c \in [261, 269], \text{   and   } r \in [-736, -730]. \)

 You divided by the opposite of the factor AND multiplied the first factor rather than just bringing it down.
\end{enumerate}

\textbf{General Comment:} Be sure to synthetically divide by the zero of the denominator!
}
\litem{
Factor the polynomial below completely, knowing that $x+3$ is a factor. Then, choose the intervals the zeros of the polynomial belong to, where $z_1 \leq z_2 \leq z_3 \leq z_4$. \textit{To make the problem easier, all zeros are between -5 and 5.}
\[ f(x) = 8x^{4} -22 x^{3} -53 x^{2} +205 x -150 \]The solution is \( [-3, 1.25, 2, 2.5] \), which is option E.\begin{enumerate}[label=\Alph*.]
\item \( z_1 \in [-5.07, -4.81], \text{   }  z_2 \in [-2.23, -1.27], z_3 \in [-0.67, -0.55], \text{   and   } z_4 \in [2.63, 3.42] \)

 Distractor 4: Corresponds to moving factors from one rational to another.
\item \( z_1 \in [-2.97, -2.29], \text{   }  z_2 \in [-2.23, -1.27], z_3 \in [-1.44, -1.1], \text{   and   } z_4 \in [2.63, 3.42] \)

 Distractor 1: Corresponds to negatives of all zeros.
\item \( z_1 \in [-3.31, -2.85], \text{   }  z_2 \in [0.01, 0.55], z_3 \in [0.76, 0.85], \text{   and   } z_4 \in [1.89, 2.33] \)

 Distractor 2: Corresponds to inversing rational roots.
\item \( z_1 \in [-2.26, -0.67], \text{   }  z_2 \in [-0.89, -0.77], z_3 \in [-0.43, -0.2], \text{   and   } z_4 \in [2.63, 3.42] \)

 Distractor 3: Corresponds to negatives of all zeros AND inversing rational roots.
\item \( z_1 \in [-3.31, -2.85], \text{   }  z_2 \in [1.13, 1.73], z_3 \in [1.98, 2.07], \text{   and   } z_4 \in [2.05, 2.5] \)

* This is the solution!
\end{enumerate}

\textbf{General Comment:} Remember to try the middle-most integers first as these normally are the zeros. Also, once you get it to a quadratic, you can use your other factoring techniques to finish factoring.
}
\litem{
Factor the polynomial below completely, knowing that $x-4$ is a factor. Then, choose the intervals the zeros of the polynomial belong to, where $z_1 \leq z_2 \leq z_3 \leq z_4$. \textit{To make the problem easier, all zeros are between -5 and 5.}
\[ f(x) = 15x^{4} -11 x^{3} -318 x^{2} +528 x -160 \]The solution is \( [-5, 0.4, 1.3333333333333333, 4] \), which is option C.\begin{enumerate}[label=\Alph*.]
\item \( z_1 \in [-4.68, -3.48], \text{   }  z_2 \in [-2.64, -2.37], z_3 \in [-0.83, -0.51], \text{   and   } z_4 \in [4.59, 5.75] \)

 Distractor 3: Corresponds to negatives of all zeros AND inversing rational roots.
\item \( z_1 \in [-4.68, -3.48], \text{   }  z_2 \in [-4.04, -3.53], z_3 \in [-0.25, -0.12], \text{   and   } z_4 \in [4.59, 5.75] \)

 Distractor 4: Corresponds to moving factors from one rational to another.
\item \( z_1 \in [-5.34, -4.95], \text{   }  z_2 \in [0.32, 0.69], z_3 \in [1.29, 1.64], \text{   and   } z_4 \in [3.28, 4.56] \)

* This is the solution!
\item \( z_1 \in [-4.68, -3.48], \text{   }  z_2 \in [-1.54, -0.8], z_3 \in [-0.59, -0.26], \text{   and   } z_4 \in [4.59, 5.75] \)

 Distractor 1: Corresponds to negatives of all zeros.
\item \( z_1 \in [-5.34, -4.95], \text{   }  z_2 \in [0.72, 1.26], z_3 \in [2.04, 2.75], \text{   and   } z_4 \in [3.28, 4.56] \)

 Distractor 2: Corresponds to inversing rational roots.
\end{enumerate}

\textbf{General Comment:} Remember to try the middle-most integers first as these normally are the zeros. Also, once you get it to a quadratic, you can use your other factoring techniques to finish factoring.
}
\litem{
What are the \textit{possible Integer} roots of the polynomial below?
\[ f(x) = 5x^{4} +2 x^{3} +7 x^{2} +7 x + 2 \]The solution is \( \pm 1,\pm 2 \), which is option C.\begin{enumerate}[label=\Alph*.]
\item \( \text{ All combinations of: }\frac{\pm 1,\pm 2}{\pm 1,\pm 5} \)

This would have been the solution \textbf{if asked for the possible Rational roots}!
\item \( \pm 1,\pm 5 \)

 Distractor 1: Corresponds to the plus or minus factors of a1 only.
\item \( \pm 1,\pm 2 \)

* This is the solution \textbf{since we asked for the possible Integer roots}!
\item \( \text{ All combinations of: }\frac{\pm 1,\pm 5}{\pm 1,\pm 2} \)

 Distractor 3: Corresponds to the plus or minus of the inverse quotient (an/a0) of the factors. 
\item \( \text{There is no formula or theorem that tells us all possible Integer roots.} \)

 Distractor 4: Corresponds to not recognizing Integers as a subset of Rationals.
\end{enumerate}

\textbf{General Comment:} We have a way to find the possible Rational roots. The possible Integer roots are the Integers in this list.
}
\end{enumerate}

\end{document}