\documentclass{extbook}[14pt]
\usepackage{multicol, enumerate, enumitem, hyperref, color, soul, setspace, parskip, fancyhdr, amssymb, amsthm, amsmath, latexsym, units, mathtools}
\everymath{\displaystyle}
\usepackage[headsep=0.5cm,headheight=0cm, left=1 in,right= 1 in,top= 1 in,bottom= 1 in]{geometry}
\usepackage{dashrule}  % Package to use the command below to create lines between items
\newcommand{\litem}[1]{\item #1

\rule{\textwidth}{0.4pt}}
\pagestyle{fancy}
\lhead{}
\chead{Answer Key for Progress Quiz 5 Version B}
\rhead{}
\lfoot{8497-6012}
\cfoot{}
\rfoot{Summer C 2021}
\begin{document}
\textbf{This key should allow you to understand why you choose the option you did (beyond just getting a question right or wrong). \href{https://xronos.clas.ufl.edu/mac1105spring2020/courseDescriptionAndMisc/Exams/LearningFromResults}{More instructions on how to use this key can be found here}.}

\textbf{If you have a suggestion to make the keys better, \href{https://forms.gle/CZkbZmPbC9XALEE88}{please fill out the short survey here}.}

\textit{Note: This key is auto-generated and may contain issues and/or errors. The keys are reviewed after each exam to ensure grading is done accurately. If there are issues (like duplicate options), they are noted in the offline gradebook. The keys are a work-in-progress to give students as many resources to improve as possible.}

\rule{\textwidth}{0.4pt}

\begin{enumerate}\litem{
Perform the division below. Then, find the intervals that correspond to the quotient in the form $ax^2+bx+c$ and remainder $r$.
\[ \frac{20x^{3} +105 x^{2} -128}{x + 5} \]The solution is \( 20x^{2} +5 x -25 + \frac{-3}{x + 5} \), which is option B.\begin{enumerate}[label=\Alph*.]
\item \( a \in [19, 27], b \in [-15, -11], c \in [89, 92], \text{ and } r \in [-670, -662]. \)

 You multipled by the synthetic number and subtracted rather than adding during synthetic division.
\item \( a \in [19, 27], b \in [2, 11], c \in [-30, -24], \text{ and } r \in [-7, -2]. \)

* This is the solution!
\item \( a \in [-105, -94], b \in [-397, -394], c \in [-1976, -1973], \text{ and } r \in [-10008, -9998]. \)

 You divided by the opposite of the factor AND multipled the first factor rather than just bringing it down.
\item \( a \in [-105, -94], b \in [602, 607], c \in [-3027, -3024], \text{ and } r \in [14989, 15000]. \)

 You multipled by the synthetic number rather than bringing the first factor down.
\item \( a \in [19, 27], b \in [203, 206], c \in [1023, 1026], \text{ and } r \in [4997, 5002]. \)

 You divided by the opposite of the factor.
\end{enumerate}

\textbf{General Comment:} Be sure to synthetically divide by the zero of the denominator! Also, make sure to include 0 placeholders for missing terms.
}
\litem{
Factor the polynomial below completely. Then, choose the intervals the zeros of the polynomial belong to, where $z_1 \leq z_2 \leq z_3$. \textit{To make the problem easier, all zeros are between -5 and 5.}
\[ f(x) = 10x^{3} -39 x^{2} -61 x + 30 \]The solution is \( [-1.5, 0.4, 5] \), which is option A.\begin{enumerate}[label=\Alph*.]
\item \( z_1 \in [-2.4, -0.9], \text{   }  z_2 \in [0.36, 0.97], \text{   and   } z_3 \in [4.87, 5.67] \)

* This is the solution!
\item \( z_1 \in [-5.1, -4.1], \text{   }  z_2 \in [-0.78, -0.09], \text{   and   } z_3 \in [0.97, 1.69] \)

 Distractor 1: Corresponds to negatives of all zeros.
\item \( z_1 \in [-1.4, 0.1], \text{   }  z_2 \in [2.08, 3.12], \text{   and   } z_3 \in [4.87, 5.67] \)

 Distractor 2: Corresponds to inversing rational roots.
\item \( z_1 \in [-5.1, -4.1], \text{   }  z_2 \in [-3.19, -2.32], \text{   and   } z_3 \in [0.45, 0.75] \)

 Distractor 3: Corresponds to negatives of all zeros AND inversing rational roots.
\item \( z_1 \in [-5.1, -4.1], \text{   }  z_2 \in [-2.36, -1.87], \text{   and   } z_3 \in [0.14, 0.36] \)

 Distractor 4: Corresponds to moving factors from one rational to another.
\end{enumerate}

\textbf{General Comment:} Remember to try the middle-most integers first as these normally are the zeros. Also, once you get it to a quadratic, you can use your other factoring techniques to finish factoring.
}
\litem{
Factor the polynomial below completely. Then, choose the intervals the zeros of the polynomial belong to, where $z_1 \leq z_2 \leq z_3$. \textit{To make the problem easier, all zeros are between -5 and 5.}
\[ f(x) = 15x^{3} -1 x^{2} -52 x + 20 \]The solution is \( [-2, 0.4, 1.67] \), which is option E.\begin{enumerate}[label=\Alph*.]
\item \( z_1 \in [-1.85, -1.24], \text{   }  z_2 \in [-0.43, -0.35], \text{   and   } z_3 \in [1.81, 2.03] \)

 Distractor 1: Corresponds to negatives of all zeros.
\item \( z_1 \in [-2.78, -2.35], \text{   }  z_2 \in [-0.6, -0.46], \text{   and   } z_3 \in [1.81, 2.03] \)

 Distractor 3: Corresponds to negatives of all zeros AND inversing rational roots.
\item \( z_1 \in [-5.02, -4.61], \text{   }  z_2 \in [-0.17, -0.02], \text{   and   } z_3 \in [1.81, 2.03] \)

 Distractor 4: Corresponds to moving factors from one rational to another.
\item \( z_1 \in [-2.33, -1.98], \text{   }  z_2 \in [0.59, 0.64], \text{   and   } z_3 \in [2.15, 2.76] \)

 Distractor 2: Corresponds to inversing rational roots.
\item \( z_1 \in [-2.33, -1.98], \text{   }  z_2 \in [0.38, 0.48], \text{   and   } z_3 \in [1.36, 1.85] \)

* This is the solution!
\end{enumerate}

\textbf{General Comment:} Remember to try the middle-most integers first as these normally are the zeros. Also, once you get it to a quadratic, you can use your other factoring techniques to finish factoring.
}
\litem{
Perform the division below. Then, find the intervals that correspond to the quotient in the form $ax^2+bx+c$ and remainder $r$.
\[ \frac{8x^{3} -62 x + 33}{x + 3} \]The solution is \( 8x^{2} -24 x + 10 + \frac{3}{x + 3} \), which is option E.\begin{enumerate}[label=\Alph*.]
\item \( a \in [4, 9], b \in [-39, -31], c \in [62, 69], \text{ and } r \in [-232, -225]. \)

 You multipled by the synthetic number and subtracted rather than adding during synthetic division.
\item \( a \in [-27, -21], b \in [-72, -67], c \in [-280, -277], \text{ and } r \in [-804, -800]. \)

 You divided by the opposite of the factor AND multipled the first factor rather than just bringing it down.
\item \( a \in [4, 9], b \in [20, 26], c \in [7, 15], \text{ and } r \in [58, 66]. \)

 You divided by the opposite of the factor.
\item \( a \in [-27, -21], b \in [71, 77], c \in [-280, -277], \text{ and } r \in [867, 868]. \)

 You multipled by the synthetic number rather than bringing the first factor down.
\item \( a \in [4, 9], b \in [-28, -21], c \in [7, 15], \text{ and } r \in [2, 5]. \)

* This is the solution!
\end{enumerate}

\textbf{General Comment:} Be sure to synthetically divide by the zero of the denominator! Also, make sure to include 0 placeholders for missing terms.
}
\litem{
Perform the division below. Then, find the intervals that correspond to the quotient in the form $ax^2+bx+c$ and remainder $r$.
\[ \frac{4x^{3} -22 x^{2} +4 x + 26}{x -5} \]The solution is \( 4x^{2} -2 x -6 + \frac{-4}{x -5} \), which is option A.\begin{enumerate}[label=\Alph*.]
\item \( a \in [2, 5], \text{   } b \in [-2, 2], \text{   } c \in [-6, -5], \text{   and   } r \in [-7, -1]. \)

* This is the solution!
\item \( a \in [20, 23], \text{   } b \in [75, 79], \text{   } c \in [394, 399], \text{   and   } r \in [1991, 1997]. \)

 You multiplied by the synthetic number rather than bringing the first factor down.
\item \( a \in [2, 5], \text{   } b \in [-8, -5], \text{   } c \in [-24, -18], \text{   and   } r \in [-58, -52]. \)

 You multiplied by the synthetic number and subtracted rather than adding during synthetic division.
\item \( a \in [2, 5], \text{   } b \in [-43, -39], \text{   } c \in [213, 221], \text{   and   } r \in [-1044, -1043]. \)

 You divided by the opposite of the factor.
\item \( a \in [20, 23], \text{   } b \in [-125, -115], \text{   } c \in [610, 618], \text{   and   } r \in [-3050, -3036]. \)

 You divided by the opposite of the factor AND multiplied the first factor rather than just bringing it down.
\end{enumerate}

\textbf{General Comment:} Be sure to synthetically divide by the zero of the denominator!
}
\litem{
Factor the polynomial below completely, knowing that $x + 3$ is a factor. Then, choose the intervals the zeros of the polynomial belong to, where $z_1 \leq z_2 \leq z_3 \leq z_4$. \textit{To make the problem easier, all zeros are between -5 and 5.}
\[ f(x) = 4x^{4} +4 x^{3} -51 x^{2} -36 x + 135 \]The solution is \( [-3, -2.5, 1.5, 3] \), which is option A.\begin{enumerate}[label=\Alph*.]
\item \( z_1 \in [-5, 1], \text{   }  z_2 \in [-2.54, -2.45], z_3 \in [1.24, 1.53], \text{   and   } z_4 \in [3, 4] \)

* This is the solution!
\item \( z_1 \in [-5, 1], \text{   }  z_2 \in [-0.8, -0.68], z_3 \in [2.74, 3.16], \text{   and   } z_4 \in [5, 7] \)

 Distractor 4: Corresponds to moving factors from one rational to another.
\item \( z_1 \in [-5, 1], \text{   }  z_2 \in [-0.72, -0.56], z_3 \in [0.32, 0.59], \text{   and   } z_4 \in [3, 4] \)

 Distractor 3: Corresponds to negatives of all zeros AND inversing rational roots.
\item \( z_1 \in [-5, 1], \text{   }  z_2 \in [-0.5, -0.35], z_3 \in [0.42, 0.9], \text{   and   } z_4 \in [3, 4] \)

 Distractor 2: Corresponds to inversing rational roots.
\item \( z_1 \in [-5, 1], \text{   }  z_2 \in [-1.5, -1.46], z_3 \in [2.24, 2.69], \text{   and   } z_4 \in [3, 4] \)

 Distractor 1: Corresponds to negatives of all zeros.
\end{enumerate}

\textbf{General Comment:} Remember to try the middle-most integers first as these normally are the zeros. Also, once you get it to a quadratic, you can use your other factoring techniques to finish factoring.
}
\litem{
Factor the polynomial below completely, knowing that $x + 4$ is a factor. Then, choose the intervals the zeros of the polynomial belong to, where $z_1 \leq z_2 \leq z_3 \leq z_4$. \textit{To make the problem easier, all zeros are between -5 and 5.}
\[ f(x) = 12x^{4} +101 x^{3} +165 x^{2} -248 x -240 \]The solution is \( [-5, -4, -0.75, 1.333] \), which is option E.\begin{enumerate}[label=\Alph*.]
\item \( z_1 \in [-0.46, 0.02], \text{   }  z_2 \in [2.74, 3.09], z_3 \in [3.87, 4.03], \text{   and   } z_4 \in [3.99, 5.65] \)

 Distractor 4: Corresponds to moving factors from one rational to another.
\item \( z_1 \in [-5.22, -4.73], \text{   }  z_2 \in [-4.54, -3.29], z_3 \in [-2.25, -0.9], \text{   and   } z_4 \in [-0.17, 1] \)

 Distractor 2: Corresponds to inversing rational roots.
\item \( z_1 \in [-1.56, -0.95], \text{   }  z_2 \in [0.63, 0.84], z_3 \in [3.87, 4.03], \text{   and   } z_4 \in [3.99, 5.65] \)

 Distractor 1: Corresponds to negatives of all zeros.
\item \( z_1 \in [-0.96, -0.61], \text{   }  z_2 \in [1.26, 1.46], z_3 \in [3.87, 4.03], \text{   and   } z_4 \in [3.99, 5.65] \)

 Distractor 3: Corresponds to negatives of all zeros AND inversing rational roots.
\item \( z_1 \in [-5.22, -4.73], \text{   }  z_2 \in [-4.54, -3.29], z_3 \in [-1, -0.5], \text{   and   } z_4 \in [0.79, 1.62] \)

* This is the solution!
\end{enumerate}

\textbf{General Comment:} Remember to try the middle-most integers first as these normally are the zeros. Also, once you get it to a quadratic, you can use your other factoring techniques to finish factoring.
}
\litem{
Perform the division below. Then, find the intervals that correspond to the quotient in the form $ax^2+bx+c$ and remainder $r$.
\[ \frac{25x^{3} -85 x^{2} +15 x + 40}{x -3} \]The solution is \( 25x^{2} -10 x -15 + \frac{-5}{x -3} \), which is option E.\begin{enumerate}[label=\Alph*.]
\item \( a \in [73, 76], \text{   } b \in [-314, -306], \text{   } c \in [945, 951], \text{   and   } r \in [-2795, -2791]. \)

 You divided by the opposite of the factor AND multiplied the first factor rather than just bringing it down.
\item \( a \in [25, 26], \text{   } b \in [-163, -157], \text{   } c \in [492, 496], \text{   and   } r \in [-1445, -1441]. \)

 You divided by the opposite of the factor.
\item \( a \in [73, 76], \text{   } b \in [136, 145], \text{   } c \in [432, 438], \text{   and   } r \in [1340, 1346]. \)

 You multiplied by the synthetic number rather than bringing the first factor down.
\item \( a \in [25, 26], \text{   } b \in [-42, -31], \text{   } c \in [-60, -51], \text{   and   } r \in [-71, -65]. \)

 You multiplied by the synthetic number and subtracted rather than adding during synthetic division.
\item \( a \in [25, 26], \text{   } b \in [-19, -9], \text{   } c \in [-17, -12], \text{   and   } r \in [-5, -1]. \)

* This is the solution!
\end{enumerate}

\textbf{General Comment:} Be sure to synthetically divide by the zero of the denominator!
}
\litem{
What are the \textit{possible Integer} roots of the polynomial below?
\[ f(x) = 3x^{2} +5 x + 4 \]The solution is \( \pm 1,\pm 2,\pm 4 \), which is option A.\begin{enumerate}[label=\Alph*.]
\item \( \pm 1,\pm 2,\pm 4 \)

* This is the solution \textbf{since we asked for the possible Integer roots}!
\item \( \text{ All combinations of: }\frac{\pm 1,\pm 3}{\pm 1,\pm 2,\pm 4} \)

 Distractor 3: Corresponds to the plus or minus of the inverse quotient (an/a0) of the factors. 
\item \( \text{ All combinations of: }\frac{\pm 1,\pm 2,\pm 4}{\pm 1,\pm 3} \)

This would have been the solution \textbf{if asked for the possible Rational roots}!
\item \( \pm 1,\pm 3 \)

 Distractor 1: Corresponds to the plus or minus factors of a1 only.
\item \( \text{There is no formula or theorem that tells us all possible Integer roots.} \)

 Distractor 4: Corresponds to not recognizing Integers as a subset of Rationals.
\end{enumerate}

\textbf{General Comment:} We have a way to find the possible Rational roots. The possible Integer roots are the Integers in this list.
}
\litem{
What are the \textit{possible Integer} roots of the polynomial below?
\[ f(x) = 5x^{2} +5 x + 2 \]The solution is \( \pm 1,\pm 2 \), which is option B.\begin{enumerate}[label=\Alph*.]
\item \( \text{ All combinations of: }\frac{\pm 1,\pm 5}{\pm 1,\pm 2} \)

 Distractor 3: Corresponds to the plus or minus of the inverse quotient (an/a0) of the factors. 
\item \( \pm 1,\pm 2 \)

* This is the solution \textbf{since we asked for the possible Integer roots}!
\item \( \pm 1,\pm 5 \)

 Distractor 1: Corresponds to the plus or minus factors of a1 only.
\item \( \text{ All combinations of: }\frac{\pm 1,\pm 2}{\pm 1,\pm 5} \)

This would have been the solution \textbf{if asked for the possible Rational roots}!
\item \( \text{There is no formula or theorem that tells us all possible Integer roots.} \)

 Distractor 4: Corresponds to not recognizing Integers as a subset of Rationals.
\end{enumerate}

\textbf{General Comment:} We have a way to find the possible Rational roots. The possible Integer roots are the Integers in this list.
}
\end{enumerate}

\end{document}