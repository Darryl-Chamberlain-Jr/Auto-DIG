\documentclass[14pt]{extbook}
\usepackage{multicol, enumerate, enumitem, hyperref, color, soul, setspace, parskip, fancyhdr} %General Packages
\usepackage{amssymb, amsthm, amsmath, bbm, latexsym, units, mathtools} %Math Packages
\everymath{\displaystyle} %All math in Display Style
% Packages with additional options
\usepackage[headsep=0.5cm,headheight=12pt, left=1 in,right= 1 in,top= 1 in,bottom= 1 in]{geometry}
\usepackage[usenames,dvipsnames]{xcolor}
\usepackage{dashrule}  % Package to use the command below to create lines between items
\newcommand{\litem}[1]{\item#1\hspace*{-1cm}\rule{\textwidth}{0.4pt}}
\pagestyle{fancy}
\lhead{Progress Quiz 5}
\chead{}
\rhead{Version A}
\lfoot{9912-2038}
\cfoot{}
\rfoot{Spring 2021}
\begin{document}

\begin{enumerate}
\litem{
Find the inverse of the function below (if it exists). Then, evaluate the inverse at $x = -14$ and choose the interval the $f^{-1}(-14)$ belongs to.\[ f(x) = \sqrt[3]{5 x - 4} \]\begin{enumerate}[label=\Alph*.]
\item \( f^{-1}(-14) \in [549.35, 550.03] \)
\item \( f^{-1}(-14) \in [547.73, 548.1] \)
\item \( f^{-1}(-14) \in [-548.94, -547.75] \)
\item \( f^{-1}(-14) \in [-549.82, -549.15] \)
\item \( \text{ The function is not invertible for all Real numbers. } \)

\end{enumerate} }
\litem{
Choose the interval below that $f$ composed with $g$ at $x=-1$ is in.\[ f(x) = -2x^{3} + x^{2} +x -2 \text{ and } g(x) = 2x^{3} -2 x^{2} -x \]\begin{enumerate}[label=\Alph*.]
\item \( (f \circ g)(-1) \in [62, 67] \)
\item \( (f \circ g)(-1) \in [-2, 2] \)
\item \( (f \circ g)(-1) \in [6, 18] \)
\item \( (f \circ g)(-1) \in [56, 59] \)
\item \( \text{It is not possible to compose the two functions.} \)

\end{enumerate} }
\litem{
Find the inverse of the function below. Then, evaluate the inverse at $x = 8$ and choose the interval that $f^{-1}(8)$ belongs to.\[ f(x) = e^{x+2}+3 \]\begin{enumerate}[label=\Alph*.]
\item \( f^{-1}(8) \in [4.74, 4.85] \)
\item \( f^{-1}(8) \in [5.36, 5.45] \)
\item \( f^{-1}(8) \in [-0.49, -0.34] \)
\item \( f^{-1}(8) \in [5.3, 5.38] \)
\item \( f^{-1}(8) \in [3.59, 3.66] \)

\end{enumerate} }
\litem{
Determine whether the function below is 1-1.\[ f(x) = 25 x^2 - 130 x + 169 \]\begin{enumerate}[label=\Alph*.]
\item \( \text{Yes, the function is 1-1.} \)
\item \( \text{No, because there is a $y$-value that goes to 2 different $x$-values.} \)
\item \( \text{No, because there is an $x$-value that goes to 2 different $y$-values.} \)
\item \( \text{No, because the range of the function is not $(-\infty, \infty)$.} \)
\item \( \text{No, because the domain of the function is not $(-\infty, \infty)$.} \)

\end{enumerate} }
\litem{
Determine whether the function below is 1-1.\[ f(x) = 25 x^2 + 110 x + 121 \]\begin{enumerate}[label=\Alph*.]
\item \( \text{No, because the domain of the function is not $(-\infty, \infty)$.} \)
\item \( \text{No, because there is a $y$-value that goes to 2 different $x$-values.} \)
\item \( \text{Yes, the function is 1-1.} \)
\item \( \text{No, because the range of the function is not $(-\infty, \infty)$.} \)
\item \( \text{No, because there is an $x$-value that goes to 2 different $y$-values.} \)

\end{enumerate} }
\litem{
Choose the interval below that $f$ composed with $g$ at $x=1$ is in.\[ f(x) = -2x^{3} +3 x^{2} +3 x -2 \text{ and } g(x) = 2x^{3} -3 x^{2} -2 x \]\begin{enumerate}[label=\Alph*.]
\item \( (f \circ g)(1) \in [68, 73] \)
\item \( (f \circ g)(1) \in [62, 66] \)
\item \( (f \circ g)(1) \in [5, 12] \)
\item \( (f \circ g)(1) \in [-1, 2] \)
\item \( \text{It is not possible to compose the two functions.} \)

\end{enumerate} }
\litem{
Find the inverse of the function below (if it exists). Then, evaluate the inverse at $x = 15$ and choose the interval that $f^{-1}(15)$ belongs to.\[ f(x) = 4 x^2 + 2 \]\begin{enumerate}[label=\Alph*.]
\item \( f^{-1}(15) \in [3.75, 3.82] \)
\item \( f^{-1}(15) \in [1.72, 1.95] \)
\item \( f^{-1}(15) \in [1.89, 2.4] \)
\item \( f^{-1}(15) \in [2.7, 2.87] \)
\item \( \text{ The function is not invertible for all Real numbers. } \)

\end{enumerate} }
\litem{
Find the inverse of the function below. Then, evaluate the inverse at $x = 8$ and choose the interval that $f^{-1}(8)$ belongs to.\[ f(x) = e^{x+4}+5 \]\begin{enumerate}[label=\Alph*.]
\item \( f^{-1}(8) \in [-2.97, -2.87] \)
\item \( f^{-1}(8) \in [6.35, 6.44] \)
\item \( f^{-1}(8) \in [7.52, 7.57] \)
\item \( f^{-1}(8) \in [7.46, 7.51] \)
\item \( f^{-1}(8) \in [5.03, 5.16] \)

\end{enumerate} }
\litem{
Add the following functions, then choose the domain of the resulting function from the list below.\[ f(x) = \sqrt{5x-16}  \text{ and } g(x) = 5x^{3} +4 x^{2} +5 x + 1 \]\begin{enumerate}[label=\Alph*.]
\item \( \text{ The domain is all Real numbers except } x = a, \text{ where } a \in [-8.75, -0.75] \)
\item \( \text{ The domain is all Real numbers less than or equal to } x = a, \text{ where } a \in [-11, -2] \)
\item \( \text{ The domain is all Real numbers greater than or equal to } x = a, \text{ where } a \in [0.2, 5.2] \)
\item \( \text{ The domain is all Real numbers except } x = a \text{ and } x = b, \text{ where } a \in [-9.67, -1.67] \text{ and } b \in [-4.67, -2.67] \)
\item \( \text{ The domain is all Real numbers. } \)

\end{enumerate} }
\litem{
Multiply the following functions, then choose the domain of the resulting function from the list below.\[ f(x) = \frac{1}{5x-21} \text{ and } g(x) = \frac{1}{5x-21} \]\begin{enumerate}[label=\Alph*.]
\item \( \text{ The domain is all Real numbers less than or equal to } x = a, \text{ where } a \in [-4.25, -1.25] \)
\item \( \text{ The domain is all Real numbers greater than or equal to } x = a, \text{ where } a \in [-10.25, -2.25] \)
\item \( \text{ The domain is all Real numbers except } x = a, \text{ where } a \in [-9.6, -4.6] \)
\item \( \text{ The domain is all Real numbers except } x = a \text{ and } x = b, \text{ where } a \in [3.2, 6.2] \text{ and } b \in [-0.8, 7.2] \)
\item \( \text{ The domain is all Real numbers. } \)

\end{enumerate} }
\end{enumerate}

\end{document}