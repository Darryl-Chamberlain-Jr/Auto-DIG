\documentclass[14pt]{extbook}
\usepackage{multicol, enumerate, enumitem, hyperref, color, soul, setspace, parskip, fancyhdr} %General Packages
\usepackage{amssymb, amsthm, amsmath, latexsym, units, mathtools} %Math Packages
\everymath{\displaystyle} %All math in Display Style
% Packages with additional options
\usepackage[headsep=0.5cm,headheight=12pt, left=1 in,right= 1 in,top= 1 in,bottom= 1 in]{geometry}
\usepackage[usenames,dvipsnames]{xcolor}
\usepackage{dashrule}  % Package to use the command below to create lines between items
\newcommand{\litem}[1]{\item#1\hspace*{-1cm}\rule{\textwidth}{0.4pt}}
\pagestyle{fancy}
\lhead{Progress Quiz 5}
\chead{}
\rhead{Version A}
\lfoot{8497-6012}
\cfoot{}
\rfoot{Summer C 2021}
\begin{document}

\begin{enumerate}
\litem{
Find the inverse of the function below. Then, evaluate the inverse at $x = 7$ and choose the interval that $f^-1(7)$ belongs to.\[ f(x) = e^{x+3}+3 \]\begin{enumerate}[label=\Alph*.]
\item \( f^{-1}(7) \in [4.3, 4.7] \)
\item \( f^{-1}(7) \in [4.4, 5.9] \)
\item \( f^{-1}(7) \in [4.4, 5.9] \)
\item \( f^{-1}(7) \in [4.3, 4.7] \)
\item \( f^{-1}(7) \in [-2.8, -1.3] \)

\end{enumerate} }
\litem{
Subtract the following functions, then choose the domain of the resulting function from the list below.\[ f(x) = 3x^{2} +5 x + 8 \text{ and } g(x) = \frac{2}{4x+27} \]\begin{enumerate}[label=\Alph*.]
\item \( \text{ The domain is all Real numbers greater than or equal to } x = a, \text{ where } a \in [-9.5, -2.5] \)
\item \( \text{ The domain is all Real numbers except } x = a, \text{ where } a \in [-13.75, -4.75] \)
\item \( \text{ The domain is all Real numbers less than or equal to } x = a, \text{ where } a \in [-2, 5] \)
\item \( \text{ The domain is all Real numbers except } x = a \text{ and } x = b, \text{ where } a \in [3.33, 9.33] \text{ and } b \in [-3.2, -2.2] \)
\item \( \text{ The domain is all Real numbers. } \)

\end{enumerate} }
\litem{
Find the inverse of the function below (if it exists). Then, evaluate the inverse at $x = -10$ and choose the interval that $f^-1(-10)$ belongs to.\[ f(x) = \sqrt[3]{2 x + 5} \]\begin{enumerate}[label=\Alph*.]
\item \( f^{-1}(-10) \in [-499.5, -496.5] \)
\item \( f^{-1}(-10) \in [501.1, 503.1] \)
\item \( f^{-1}(-10) \in [495.1, 500.2] \)
\item \( f^{-1}(-10) \in [-503.9, -500.4] \)
\item \( \text{ The function is not invertible for all Real numbers. } \)

\end{enumerate} }
\litem{
Determine whether the function below is 1-1.\[ f(x) = (5 x - 35)^3 \]\begin{enumerate}[label=\Alph*.]
\item \( \text{Yes, the function is 1-1.} \)
\item \( \text{No, because the range of the function is not $(-\infty, \infty)$.} \)
\item \( \text{No, because there is a $y$-value that goes to 2 different $x$-values.} \)
\item \( \text{No, because there is an $x$-value that goes to 2 different $y$-values.} \)
\item \( \text{No, because the domain of the function is not $(-\infty, \infty)$.} \)

\end{enumerate} }
\litem{
Add the following functions, then choose the domain of the resulting function from the list below.\[ f(x) = \frac{3}{6x+37} \text{ and } g(x) = x + 7 \]\begin{enumerate}[label=\Alph*.]
\item \( \text{ The domain is all Real numbers greater than or equal to } x = a, \text{ where } a \in [-6.5, -2.5] \)
\item \( \text{ The domain is all Real numbers except } x = a, \text{ where } a \in [-6.17, -2.17] \)
\item \( \text{ The domain is all Real numbers less than or equal to } x = a, \text{ where } a \in [0, 3] \)
\item \( \text{ The domain is all Real numbers except } x = a \text{ and } x = b, \text{ where } a \in [-3.8, 1.2] \text{ and } b \in [4.33, 8.33] \)
\item \( \text{ The domain is all Real numbers. } \)

\end{enumerate} }
\litem{
Find the inverse of the function below. Then, evaluate the inverse at $x = 10$ and choose the interval that $f^-1(10)$ belongs to.\[ f(x) = \ln{(x+4)}-5 \]\begin{enumerate}[label=\Alph*.]
\item \( f^{-1}(10) \in [142.41, 149.41] \)
\item \( f^{-1}(10) \in [396.43, 399.43] \)
\item \( f^{-1}(10) \in [3269012.37, 3269019.37] \)
\item \( f^{-1}(10) \in [1202597.28, 1202600.28] \)
\item \( f^{-1}(10) \in [3269019.37, 3269022.37] \)

\end{enumerate} }
\litem{
Choose the interval below that $f$ composed with $g$ at $x=-1$ is in.\[ f(x) = -2x^{3} + x^{2} +2 x \text{ and } g(x) = -x^{3} -2 x^{2} -3 x -4 \]\begin{enumerate}[label=\Alph*.]
\item \( (f \circ g)(-1) \in [13, 17] \)
\item \( (f \circ g)(-1) \in [-16, -6] \)
\item \( (f \circ g)(-1) \in [-8, -3] \)
\item \( (f \circ g)(-1) \in [5, 12] \)
\item \( \text{It is not possible to compose the two functions.} \)

\end{enumerate} }
\litem{
Find the inverse of the function below (if it exists). Then, evaluate the inverse at $x = -15$ and choose the interval that $f^-1(-15)$ belongs to.\[ f(x) = 3 x^2 - 4 \]\begin{enumerate}[label=\Alph*.]
\item \( f^{-1}(-15) \in [1.82, 1.95] \)
\item \( f^{-1}(-15) \in [6.9, 7.22] \)
\item \( f^{-1}(-15) \in [3.76, 4.2] \)
\item \( f^{-1}(-15) \in [2.22, 2.55] \)
\item \( \text{ The function is not invertible for all Real numbers. } \)

\end{enumerate} }
\litem{
Choose the interval below that $f$ composed with $g$ at $x=-1$ is in.\[ f(x) = 2x^{3} -2 x^{2} +2 x + 3 \text{ and } g(x) = x^{3} +2 x^{2} +3 x \]\begin{enumerate}[label=\Alph*.]
\item \( (f \circ g)(-1) \in [-18.39, -17.32] \)
\item \( (f \circ g)(-1) \in [-13.09, -12.75] \)
\item \( (f \circ g)(-1) \in [-25.46, -22.42] \)
\item \( (f \circ g)(-1) \in [-21.79, -18.43] \)
\item \( \text{It is not possible to compose the two functions.} \)

\end{enumerate} }
\litem{
Determine whether the function below is 1-1.\[ f(x) = (3 x + 21)^3 \]\begin{enumerate}[label=\Alph*.]
\item \( \text{No, because the range of the function is not $(-\infty, \infty)$.} \)
\item \( \text{No, because there is a $y$-value that goes to 2 different $x$-values.} \)
\item \( \text{No, because there is an $x$-value that goes to 2 different $y$-values.} \)
\item \( \text{Yes, the function is 1-1.} \)
\item \( \text{No, because the domain of the function is not $(-\infty, \infty)$.} \)

\end{enumerate} }
\end{enumerate}

\end{document}