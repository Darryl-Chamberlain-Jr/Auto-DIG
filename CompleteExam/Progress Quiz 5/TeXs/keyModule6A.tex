\documentclass{extbook}[14pt]
\usepackage{multicol, enumerate, enumitem, hyperref, color, soul, setspace, parskip, fancyhdr, amssymb, amsthm, amsmath, latexsym, units, mathtools}
\everymath{\displaystyle}
\usepackage[headsep=0.5cm,headheight=0cm, left=1 in,right= 1 in,top= 1 in,bottom= 1 in]{geometry}
\usepackage{dashrule}  % Package to use the command below to create lines between items
\newcommand{\litem}[1]{\item #1

\rule{\textwidth}{0.4pt}}
\pagestyle{fancy}
\lhead{}
\chead{Answer Key for Progress Quiz 5 Version A}
\rhead{}
\lfoot{8497-6012}
\cfoot{}
\rfoot{Summer C 2021}
\begin{document}
\textbf{This key should allow you to understand why you choose the option you did (beyond just getting a question right or wrong). \href{https://xronos.clas.ufl.edu/mac1105spring2020/courseDescriptionAndMisc/Exams/LearningFromResults}{More instructions on how to use this key can be found here}.}

\textbf{If you have a suggestion to make the keys better, \href{https://forms.gle/CZkbZmPbC9XALEE88}{please fill out the short survey here}.}

\textit{Note: This key is auto-generated and may contain issues and/or errors. The keys are reviewed after each exam to ensure grading is done accurately. If there are issues (like duplicate options), they are noted in the offline gradebook. The keys are a work-in-progress to give students as many resources to improve as possible.}

\rule{\textwidth}{0.4pt}

\begin{enumerate}\litem{
Construct the lowest-degree polynomial given the zeros below. Then, choose the intervals that contain the coefficients of the polynomial in the form $x^3+bx^2+cx+d$.
\[ -5 + 4 i \text{ and } 4 \]The solution is \( x^{3} +6 x^{2} +x -164 \), which is option A.\begin{enumerate}[label=\Alph*.]
\item \( b \in [5, 14], c \in [-1, 5], \text{ and } d \in [-165, -162] \)

* $x^{3} +6 x^{2} +x -164$, which is the correct option.
\item \( b \in [-2, 4], c \in [-1, 5], \text{ and } d \in [-22, -18] \)

$x^{3} + x^{2} +x -20$, which corresponds to multiplying out $(x + 5)(x -4)$.
\item \( b \in [-2, 4], c \in [-10, -7], \text{ and } d \in [11, 20] \)

$x^{3} + x^{2} -8 x + 16$, which corresponds to multiplying out $(x -4)(x -4)$.
\item \( b \in [-12, -3], c \in [-1, 5], \text{ and } d \in [164, 169] \)

$x^{3} -6 x^{2} +x + 164$, which corresponds to multiplying out $(x-(-5 + 4 i))(x-(-5 - 4 i))(x + 4)$.
\item \( \text{None of the above.} \)

This corresponds to making an unanticipated error or not understanding how to use nonreal complex numbers to create the lowest-degree polynomial. If you chose this and are not sure what you did wrong, please contact the coordinator for help.
\end{enumerate}

\textbf{General Comment:} Remember that the conjugate of $a+bi$ is $a-bi$. Since these zeros always come in pairs, we need to multiply out $(x-(-5 + 4 i))(x-(-5 - 4 i))(x-(4))$.
}
\litem{
Construct the lowest-degree polynomial given the zeros below. Then, choose the intervals that contain the coefficients of the polynomial in the form $ax^3+bx^2+cx+d$.
\[ \frac{7}{4}, 5, \text{ and } \frac{7}{3} \]The solution is \( 12x^{3} -109 x^{2} +294 x -245 \), which is option D.\begin{enumerate}[label=\Alph*.]
\item \( a \in [12, 14], b \in [-112, -107], c \in [293, 297], \text{ and } d \in [245, 252] \)

$12x^{3} -109 x^{2} +294 x + 245$, which corresponds to multiplying everything correctly except the constant term.
\item \( a \in [12, 14], b \in [-76, -65], c \in [-15, -8], \text{ and } d \in [245, 252] \)

$12x^{3} -67 x^{2} -14 x + 245$, which corresponds to multiplying out $(4x + 7)(x -5)(3x -7)$.
\item \( a \in [12, 14], b \in [108, 115], c \in [293, 297], \text{ and } d \in [245, 252] \)

$12x^{3} +109 x^{2} +294 x + 245$, which corresponds to multiplying out $(4x + 7)(x + 5)(3x + 7)$.
\item \( a \in [12, 14], b \in [-112, -107], c \in [293, 297], \text{ and } d \in [-246, -240] \)

* $12x^{3} -109 x^{2} +294 x -245$, which is the correct option.
\item \( a \in [12, 14], b \in [49, 58], c \in [-88, -83], \text{ and } d \in [-246, -240] \)

$12x^{3} +53 x^{2} -84 x -245$, which corresponds to multiplying out $(4x + 7)(x + 5)(3x -7)$.
\end{enumerate}

\textbf{General Comment:} To construct the lowest-degree polynomial, you want to multiply out $(4x -7)(x -5)(3x -7)$
}
\litem{
Describe the end behavior of the polynomial below.
\[ f(x) = -5(x + 3)^{4}(x - 3)^{5}(x + 2)^{3}(x - 2)^{5} \]The solution is the graph below, which is option A.
    \begin{center}
        \includegraphics[width=0.3\textwidth]{../Figures/polyEndBehaviorAA.png}
    \end{center}\begin{enumerate}[label=\Alph*.]
\begin{multicols}{2}
\item \includegraphics[width = 0.3\textwidth]{../Figures/polyEndBehaviorAA.png}
\item \includegraphics[width = 0.3\textwidth]{../Figures/polyEndBehaviorBA.png}
\item \includegraphics[width = 0.3\textwidth]{../Figures/polyEndBehaviorCA.png}
\item \includegraphics[width = 0.3\textwidth]{../Figures/polyEndBehaviorDA.png}
\end{multicols}\item None of the above.\end{enumerate}
\textbf{General Comment:} Remember that end behavior is determined by the leading coefficient AND whether the \textbf{sum} of the multiplicities is positive or negative.
}
\litem{
Which of the following equations \textit{could} be of the graph presented below?

\begin{center}
    \includegraphics[width=0.5\textwidth]{../Figures/polyGraphToFunctionCopyA.png}
\end{center}


The solution is \( -5(x + 3)^{6} (x - 3)^{5} (x - 2)^{9} \), which is option B.\begin{enumerate}[label=\Alph*.]
\item \( 18(x + 3)^{10} (x - 3)^{9} (x - 2)^{7} \)

This corresponds to the leading coefficient being the opposite value than it should be.
\item \( -5(x + 3)^{6} (x - 3)^{5} (x - 2)^{9} \)

* This is the correct option.
\item \( 4(x + 3)^{4} (x - 3)^{9} (x - 2)^{4} \)

The factor $(x - 2)$ should have an odd power and the leading coefficient should be the opposite sign.
\item \( -16(x + 3)^{5} (x - 3)^{8} (x - 2)^{7} \)

The factor $-3$ should have an even power and the factor $3$ should have an odd power.
\item \( -13(x + 3)^{10} (x - 3)^{8} (x - 2)^{11} \)

The factor $(x - 3)$ should have an odd power.
\end{enumerate}

\textbf{General Comment:} General Comments: Draw the x-axis to determine which zeros are touching (and so have even multiplicity) or cross (and have odd multiplicity).
}
\litem{
Construct the lowest-degree polynomial given the zeros below. Then, choose the intervals that contain the coefficients of the polynomial in the form $ax^3+bx^2+cx+d$.
\[ \frac{-1}{2}, \frac{5}{4}, \text{ and } \frac{7}{5} \]The solution is \( 40x^{3} -86 x^{2} +17 x + 35 \), which is option A.\begin{enumerate}[label=\Alph*.]
\item \( a \in [35, 45], b \in [-86, -81], c \in [16, 18], \text{ and } d \in [32, 42] \)

* $40x^{3} -86 x^{2} +17 x + 35$, which is the correct option.
\item \( a \in [35, 45], b \in [-128, -124], c \in [121, 126], \text{ and } d \in [-38, -33] \)

$40x^{3} -126 x^{2} +123 x -35$, which corresponds to multiplying out $(2x -1)(4x -5)(5x -7)$.
\item \( a \in [35, 45], b \in [-33, -17], c \in [-75, -64], \text{ and } d \in [32, 42] \)

$40x^{3} -26 x^{2} -67 x + 35$, which corresponds to multiplying out $(2x -1)(4x + 5)(5x -7)$.
\item \( a \in [35, 45], b \in [-86, -81], c \in [16, 18], \text{ and } d \in [-38, -33] \)

$40x^{3} -86 x^{2} +17 x -35$, which corresponds to multiplying everything correctly except the constant term.
\item \( a \in [35, 45], b \in [81, 94], c \in [16, 18], \text{ and } d \in [-38, -33] \)

$40x^{3} +86 x^{2} +17 x -35$, which corresponds to multiplying out $(2x -1)(4x + 5)(5x + 7)$.
\end{enumerate}

\textbf{General Comment:} To construct the lowest-degree polynomial, you want to multiply out $(2x + 1)(4x -5)(5x -7)$
}
\litem{
Which of the following equations \textit{could} be of the graph presented below?

\begin{center}
    \includegraphics[width=0.5\textwidth]{../Figures/polyGraphToFunctionA.png}
\end{center}


The solution is \( 8(x + 2)^{8} (x + 3)^{6} (x - 1)^{6} \), which is option A.\begin{enumerate}[label=\Alph*.]
\item \( 8(x + 2)^{8} (x + 3)^{6} (x - 1)^{6} \)

* This is the correct option.
\item \( -12(x + 2)^{4} (x + 3)^{8} (x - 1)^{10} \)

This corresponds to the leading coefficient being the opposite value than it should be.
\item \( 10(x + 2)^{6} (x + 3)^{6} (x - 1)^{5} \)

The factor $(x - 1)$ should have an even power.
\item \( -13(x + 2)^{10} (x + 3)^{8} (x - 1)^{9} \)

The factor $(x - 1)$ should have an even power and the leading coefficient should be the opposite sign.
\item \( 19(x + 2)^{10} (x + 3)^{11} (x - 1)^{9} \)

The factors $(x + 3)$ and $(x - 1)$ should both have even powers.
\end{enumerate}

\textbf{General Comment:} General Comments: Draw the x-axis to determine which zeros are touching (and so have even multiplicity) or cross (and have odd multiplicity).
}
\litem{
Describe the zero behavior of the zero $x = 3$ of the polynomial below.
\[ f(x) = -2(x - 2)^{6}(x + 2)^{3}(x + 3)^{11}(x - 3)^{6} \]The solution is the graph below, which is option B.
    \begin{center}
        \includegraphics[width=0.3\textwidth]{../Figures/polyZeroBehaviorBA.png}
    \end{center}\begin{enumerate}[label=\Alph*.]
\begin{multicols}{2}
\item \includegraphics[width = 0.3\textwidth]{../Figures/polyZeroBehaviorAA.png}
\item \includegraphics[width = 0.3\textwidth]{../Figures/polyZeroBehaviorBA.png}
\item \includegraphics[width = 0.3\textwidth]{../Figures/polyZeroBehaviorCA.png}
\item \includegraphics[width = 0.3\textwidth]{../Figures/polyZeroBehaviorDA.png}
\end{multicols}\item None of the above.\end{enumerate}
\textbf{General Comment:} You will need to sketch the entire graph, then zoom in on the zero the question asks about.
}
\litem{
Describe the zero behavior of the zero $x = 2$ of the polynomial below.
\[ f(x) = -2(x - 2)^{2}(x + 2)^{7}(x + 9)^{4}(x - 9)^{8} \]The solution is the graph below, which is option B.
    \begin{center}
        \includegraphics[width=0.3\textwidth]{../Figures/polyZeroBehaviorCopyBA.png}
    \end{center}\begin{enumerate}[label=\Alph*.]
\begin{multicols}{2}
\item \includegraphics[width = 0.3\textwidth]{../Figures/polyZeroBehaviorCopyAA.png}
\item \includegraphics[width = 0.3\textwidth]{../Figures/polyZeroBehaviorCopyBA.png}
\item \includegraphics[width = 0.3\textwidth]{../Figures/polyZeroBehaviorCopyCA.png}
\item \includegraphics[width = 0.3\textwidth]{../Figures/polyZeroBehaviorCopyDA.png}
\end{multicols}\item None of the above.\end{enumerate}
\textbf{General Comment:} You will need to sketch the entire graph, then zoom in on the zero the question asks about.
}
\litem{
Construct the lowest-degree polynomial given the zeros below. Then, choose the intervals that contain the coefficients of the polynomial in the form $x^3+bx^2+cx+d$.
\[ -2 + 2 i \text{ and } 1 \]The solution is \( x^{3} +3 x^{2} +4 x -8 \), which is option C.\begin{enumerate}[label=\Alph*.]
\item \( b \in [-3.8, -1.5], c \in [2.1, 5.2], \text{ and } d \in [6.4, 9.3] \)

$x^{3} -3 x^{2} +4 x + 8$, which corresponds to multiplying out $(x-(-2 + 2 i))(x-(-2 - 2 i))(x + 1)$.
\item \( b \in [0.7, 2.7], c \in [-3.6, -0.8], \text{ and } d \in [0.3, 4.3] \)

$x^{3} + x^{2} -3 x + 2$, which corresponds to multiplying out $(x -2)(x -1)$.
\item \( b \in [2, 4.5], c \in [2.1, 5.2], \text{ and } d \in [-8.4, -7.4] \)

* $x^{3} +3 x^{2} +4 x -8$, which is the correct option.
\item \( b \in [0.7, 2.7], c \in [-0.3, 2.5], \text{ and } d \in [-4.2, -1.1] \)

$x^{3} + x^{2} +x -2$, which corresponds to multiplying out $(x + 2)(x -1)$.
\item \( \text{None of the above.} \)

This corresponds to making an unanticipated error or not understanding how to use nonreal complex numbers to create the lowest-degree polynomial. If you chose this and are not sure what you did wrong, please contact the coordinator for help.
\end{enumerate}

\textbf{General Comment:} Remember that the conjugate of $a+bi$ is $a-bi$. Since these zeros always come in pairs, we need to multiply out $(x-(-2 + 2 i))(x-(-2 - 2 i))(x-(1))$.
}
\litem{
Describe the end behavior of the polynomial below.
\[ f(x) = 6(x + 3)^{3}(x - 3)^{6}(x - 2)^{5}(x + 2)^{6} \]The solution is the graph below, which is option C.
    \begin{center}
        \includegraphics[width=0.3\textwidth]{../Figures/polyEndBehaviorCopyCA.png}
    \end{center}\begin{enumerate}[label=\Alph*.]
\begin{multicols}{2}
\item \includegraphics[width = 0.3\textwidth]{../Figures/polyEndBehaviorCopyAA.png}
\item \includegraphics[width = 0.3\textwidth]{../Figures/polyEndBehaviorCopyBA.png}
\item \includegraphics[width = 0.3\textwidth]{../Figures/polyEndBehaviorCopyCA.png}
\item \includegraphics[width = 0.3\textwidth]{../Figures/polyEndBehaviorCopyDA.png}
\end{multicols}\item None of the above.\end{enumerate}
\textbf{General Comment:} Remember that end behavior is determined by the leading coefficient AND whether the \textbf{sum} of the multiplicities is positive or negative.
}
\end{enumerate}

\end{document}