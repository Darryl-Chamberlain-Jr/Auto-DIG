\documentclass[14pt]{extbook}
\usepackage{multicol, enumerate, enumitem, hyperref, color, soul, setspace, parskip, fancyhdr} %General Packages
\usepackage{amssymb, amsthm, amsmath, latexsym, units, mathtools} %Math Packages
\everymath{\displaystyle} %All math in Display Style
% Packages with additional options
\usepackage[headsep=0.5cm,headheight=12pt, left=1 in,right= 1 in,top= 1 in,bottom= 1 in]{geometry}
\usepackage[usenames,dvipsnames]{xcolor}
\usepackage{dashrule}  % Package to use the command below to create lines between items
\newcommand{\litem}[1]{\item#1\hspace*{-1cm}\rule{\textwidth}{0.4pt}}
\pagestyle{fancy}
\lhead{Progress Quiz 5}
\chead{}
\rhead{Version C}
\lfoot{8497-6012}
\cfoot{}
\rfoot{Summer C 2021}
\begin{document}

\begin{enumerate}
\litem{
Find the inverse of the function below. Then, evaluate the inverse at $x = 8$ and choose the interval that $f^-1(8)$ belongs to.\[ f(x) = \ln{(x+5)}+2 \]\begin{enumerate}[label=\Alph*.]
\item \( f^{-1}(8) \in [22020.47, 22024.47] \)
\item \( f^{-1}(8) \in [21.09, 27.09] \)
\item \( f^{-1}(8) \in [442414.39, 442420.39] \)
\item \( f^{-1}(8) \in [405.43, 413.43] \)
\item \( f^{-1}(8) \in [388.43, 399.43] \)

\end{enumerate} }
\litem{
Add the following functions, then choose the domain of the resulting function from the list below.\[ f(x) = \sqrt{6x-28}  \text{ and } g(x) = x + 6 \]\begin{enumerate}[label=\Alph*.]
\item \( \text{ The domain is all Real numbers except } x = a, \text{ where } a \in [1.17, 5.17] \)
\item \( \text{ The domain is all Real numbers greater than or equal to } x = a, \text{ where } a \in [0.67, 5.67] \)
\item \( \text{ The domain is all Real numbers less than or equal to } x = a, \text{ where } a \in [-5.5, -0.5] \)
\item \( \text{ The domain is all Real numbers except } x = a \text{ and } x = b, \text{ where } a \in [-1.67, 4.33] \text{ and } b \in [-4.2, -3.2] \)
\item \( \text{ The domain is all Real numbers. } \)

\end{enumerate} }
\litem{
Find the inverse of the function below (if it exists). Then, evaluate the inverse at $x = -15$ and choose the interval that $f^-1(-15)$ belongs to.\[ f(x) = \sqrt[3]{3 x + 4} \]\begin{enumerate}[label=\Alph*.]
\item \( f^{-1}(-15) \in [1125.5, 1129.3] \)
\item \( f^{-1}(-15) \in [1122.7, 1126] \)
\item \( f^{-1}(-15) \in [-1125.4, -1122.4] \)
\item \( f^{-1}(-15) \in [-1129, -1126.3] \)
\item \( \text{ The function is not invertible for all Real numbers. } \)

\end{enumerate} }
\litem{
Determine whether the function below is 1-1.\[ f(x) = 9 x^2 - 39 x - 230 \]\begin{enumerate}[label=\Alph*.]
\item \( \text{No, because the domain of the function is not $(-\infty, \infty)$.} \)
\item \( \text{No, because there is a $y$-value that goes to 2 different $x$-values.} \)
\item \( \text{Yes, the function is 1-1.} \)
\item \( \text{No, because the range of the function is not $(-\infty, \infty)$.} \)
\item \( \text{No, because there is an $x$-value that goes to 2 different $y$-values.} \)

\end{enumerate} }
\litem{
Multiply the following functions, then choose the domain of the resulting function from the list below.\[ f(x) = \frac{3}{3x-16} \text{ and } g(x) = \frac{2}{3x+16} \]\begin{enumerate}[label=\Alph*.]
\item \( \text{ The domain is all Real numbers less than or equal to } x = a, \text{ where } a \in [-6.6, 5.4] \)
\item \( \text{ The domain is all Real numbers except } x = a, \text{ where } a \in [-8.25, -4.25] \)
\item \( \text{ The domain is all Real numbers greater than or equal to } x = a, \text{ where } a \in [5, 13] \)
\item \( \text{ The domain is all Real numbers except } x = a \text{ and } x = b, \text{ where } a \in [0.33, 6.33] \text{ and } b \in [-11.33, -1.33] \)
\item \( \text{ The domain is all Real numbers. } \)

\end{enumerate} }
\litem{
Find the inverse of the function below. Then, evaluate the inverse at $x = 4$ and choose the interval that $f^-1(4)$ belongs to.\[ f(x) = e^{x+2}+2 \]\begin{enumerate}[label=\Alph*.]
\item \( f^{-1}(4) \in [-0.7, 2.7] \)
\item \( f^{-1}(4) \in [-3.5, -0.7] \)
\item \( f^{-1}(4) \in [-0.7, 2.7] \)
\item \( f^{-1}(4) \in [2.8, 5.5] \)
\item \( f^{-1}(4) \in [2.8, 5.5] \)

\end{enumerate} }
\litem{
Choose the interval below that $f$ composed with $g$ at $x=-1$ is in.\[ f(x) = -4x^{3} -2 x^{2} +4 x -1 \text{ and } g(x) = -2x^{3} -2 x^{2} -x \]\begin{enumerate}[label=\Alph*.]
\item \( (f \circ g)(-1) \in [42, 47] \)
\item \( (f \circ g)(-1) \in [38, 40] \)
\item \( (f \circ g)(-1) \in [4, 6] \)
\item \( (f \circ g)(-1) \in [-12, 0] \)
\item \( \text{It is not possible to compose the two functions.} \)

\end{enumerate} }
\litem{
Find the inverse of the function below (if it exists). Then, evaluate the inverse at $x = 13$ and choose the interval that $f^-1(13)$ belongs to.\[ f(x) = \sqrt[3]{5 x + 3} \]\begin{enumerate}[label=\Alph*.]
\item \( f^{-1}(13) \in [438.67, 438.81] \)
\item \( f^{-1}(13) \in [439.45, 441.34] \)
\item \( f^{-1}(13) \in [-439.21, -438.65] \)
\item \( f^{-1}(13) \in [-440.29, -439.8] \)
\item \( \text{ The function is not invertible for all Real numbers. } \)

\end{enumerate} }
\litem{
Choose the interval below that $f$ composed with $g$ at $x=-1$ is in.\[ f(x) = -2x^{3} +3 x^{2} +4 x \text{ and } g(x) = 3x^{3} -1 x^{2} -2 x \]\begin{enumerate}[label=\Alph*.]
\item \( (f \circ g)(-1) \in [24, 37] \)
\item \( (f \circ g)(-1) \in [20, 21] \)
\item \( (f \circ g)(-1) \in [1, 14] \)
\item \( (f \circ g)(-1) \in [-5, 3] \)
\item \( \text{It is not possible to compose the two functions.} \)

\end{enumerate} }
\litem{
Determine whether the function below is 1-1.\[ f(x) = 15 x^2 - 56 x - 396 \]\begin{enumerate}[label=\Alph*.]
\item \( \text{No, because there is a $y$-value that goes to 2 different $x$-values.} \)
\item \( \text{No, because the range of the function is not $(-\infty, \infty)$.} \)
\item \( \text{No, because there is an $x$-value that goes to 2 different $y$-values.} \)
\item \( \text{No, because the domain of the function is not $(-\infty, \infty)$.} \)
\item \( \text{Yes, the function is 1-1.} \)

\end{enumerate} }
\end{enumerate}

\end{document}