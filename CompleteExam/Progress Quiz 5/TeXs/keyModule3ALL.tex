\documentclass{extbook}[14pt]
\usepackage{multicol, enumerate, enumitem, hyperref, color, soul, setspace, parskip, fancyhdr, amssymb, amsthm, amsmath, latexsym, units, mathtools}
\everymath{\displaystyle}
\usepackage[headsep=0.5cm,headheight=0cm, left=1 in,right= 1 in,top= 1 in,bottom= 1 in]{geometry}
\usepackage{dashrule}  % Package to use the command below to create lines between items
\newcommand{\litem}[1]{\item #1

\rule{\textwidth}{0.4pt}}
\pagestyle{fancy}
\lhead{}
\chead{Answer Key for Progress Quiz 5 Version ALL}
\rhead{}
\lfoot{8497-6012}
\cfoot{}
\rfoot{Summer C 2021}
\begin{document}
\textbf{This key should allow you to understand why you choose the option you did (beyond just getting a question right or wrong). \href{https://xronos.clas.ufl.edu/mac1105spring2020/courseDescriptionAndMisc/Exams/LearningFromResults}{More instructions on how to use this key can be found here}.}

\textbf{If you have a suggestion to make the keys better, \href{https://forms.gle/CZkbZmPbC9XALEE88}{please fill out the short survey here}.}

\textit{Note: This key is auto-generated and may contain issues and/or errors. The keys are reviewed after each exam to ensure grading is done accurately. If there are issues (like duplicate options), they are noted in the offline gradebook. The keys are a work-in-progress to give students as many resources to improve as possible.}

\rule{\textwidth}{0.4pt}

\begin{enumerate}\litem{
Solve the linear inequality below. Then, choose the constant and interval combination that describes the solution set.
\[ \frac{5}{7} + \frac{4}{3} x \geq \frac{10}{8} x - \frac{9}{9} \]The solution is \( [-20.571, \infty) \), which is option B.\begin{enumerate}[label=\Alph*.]
\item \( (-\infty, a], \text{ where } a \in [-21, -20.25] \)

 $(-\infty, -20.571]$, which corresponds to switching the direction of the interval. You likely did this if you did not flip the inequality when dividing by a negative!
\item \( [a, \infty), \text{ where } a \in [-22.5, -17.25] \)

* $[-20.571, \infty)$, which is the correct option.
\item \( [a, \infty), \text{ where } a \in [18.75, 22.5] \)

 $[20.571, \infty)$, which corresponds to negating the endpoint of the solution.
\item \( (-\infty, a], \text{ where } a \in [18, 23.25] \)

 $(-\infty, 20.571]$, which corresponds to switching the direction of the interval AND negating the endpoint. You likely did this if you did not flip the inequality when dividing by a negative as well as not moving values over to a side properly.
\item \( \text{None of the above}. \)

You may have chosen this if you thought the inequality did not match the ends of the intervals.
\end{enumerate}

\textbf{General Comment:} Remember that less/greater than or equal to includes the endpoint, while less/greater do not. Also, remember that you need to flip the inequality when you multiply or divide by a negative.
}
\litem{
Using an interval or intervals, describe all the $x$-values within or including a distance of the given values.
\[ \text{ More than } 9 \text{ units from the number } 4. \]The solution is \( (-\infty, -5) \cup (13, \infty) \), which is option D.\begin{enumerate}[label=\Alph*.]
\item \( (-\infty, -5] \cup [13, \infty) \)

This describes the values no less than 9 from 4
\item \( (-5, 13) \)

This describes the values less than 9 from 4
\item \( [-5, 13] \)

This describes the values no more than 9 from 4
\item \( (-\infty, -5) \cup (13, \infty) \)

This describes the values more than 9 from 4
\item \( \text{None of the above} \)

You likely thought the values in the interval were not correct.
\end{enumerate}

\textbf{General Comment:} When thinking about this language, it helps to draw a number line and try points.
}
\litem{
Solve the linear inequality below. Then, choose the constant and interval combination that describes the solution set.
\[ 5 + 6 x > 8 x \text{ or } 8 + 4 x < 5 x \]The solution is \( (-\infty, 2.5) \text{ or } (8.0, \infty) \), which is option C.\begin{enumerate}[label=\Alph*.]
\item \( (-\infty, a) \cup (b, \infty), \text{ where } a \in [-11.25, -7.5] \text{ and } b \in [-6.75, 0] \)

Corresponds to inverting the inequality and negating the solution.
\item \( (-\infty, a] \cup [b, \infty), \text{ where } a \in [-10.5, -3.75] \text{ and } b \in [-3.75, -0.75] \)

Corresponds to including the endpoints AND negating.
\item \( (-\infty, a) \cup (b, \infty), \text{ where } a \in [-3.75, 4.5] \text{ and } b \in [5.25, 12.75] \)

 * Correct option.
\item \( (-\infty, a] \cup [b, \infty), \text{ where } a \in [2.25, 4.5] \text{ and } b \in [3.75, 9.75] \)

Corresponds to including the endpoints (when they should be excluded).
\item \( (-\infty, \infty) \)

Corresponds to the variable canceling, which does not happen in this instance.
\end{enumerate}

\textbf{General Comment:} When multiplying or dividing by a negative, flip the sign.
}
\litem{
Solve the linear inequality below. Then, choose the constant and interval combination that describes the solution set.
\[ -4 - 9 x < \frac{-40 x - 8}{6} \leq 5 - 7 x \]The solution is \( (-1.14, 19.00] \), which is option A.\begin{enumerate}[label=\Alph*.]
\item \( (a, b], \text{ where } a \in [-3.97, -0.45] \text{ and } b \in [18, 20.25] \)

* $(-1.14, 19.00]$, which is the correct option.
\item \( (-\infty, a] \cup (b, \infty), \text{ where } a \in [-3.52, -0.82] \text{ and } b \in [16.5, 25.5] \)

$(-\infty, -1.14] \cup (19.00, \infty)$, which corresponds to displaying the and-inequality as an or-inequality AND flipping the inequality.
\item \( [a, b), \text{ where } a \in [-3.45, 0.22] \text{ and } b \in [18, 21.75] \)

$[-1.14, 19.00)$, which corresponds to flipping the inequality.
\item \( (-\infty, a) \cup [b, \infty), \text{ where } a \in [-4.65, -1.05] \text{ and } b \in [16.5, 26.25] \)

$(-\infty, -1.14) \cup [19.00, \infty)$, which corresponds to displaying the and-inequality as an or-inequality.
\item \( \text{None of the above.} \)


\end{enumerate}

\textbf{General Comment:} To solve, you will need to break up the compound inequality into two inequalities. Be sure to keep track of the inequality! It may be best to draw a number line and graph your solution.
}
\litem{
Solve the linear inequality below. Then, choose the constant and interval combination that describes the solution set.
\[ -7x + 3 \leq 6x + 7 \]The solution is \( [-0.308, \infty) \), which is option C.\begin{enumerate}[label=\Alph*.]
\item \( [a, \infty), \text{ where } a \in [0.05, 0.46] \)

 $[0.308, \infty)$, which corresponds to negating the endpoint of the solution.
\item \( (-\infty, a], \text{ where } a \in [-0.48, -0.29] \)

 $(-\infty, -0.308]$, which corresponds to switching the direction of the interval. You likely did this if you did not flip the inequality when dividing by a negative!
\item \( [a, \infty), \text{ where } a \in [-0.55, -0.18] \)

* $[-0.308, \infty)$, which is the correct option.
\item \( (-\infty, a], \text{ where } a \in [-0.22, 0.9] \)

 $(-\infty, 0.308]$, which corresponds to switching the direction of the interval AND negating the endpoint. You likely did this if you did not flip the inequality when dividing by a negative as well as not moving values over to a side properly.
\item \( \text{None of the above}. \)

You may have chosen this if you thought the inequality did not match the ends of the intervals.
\end{enumerate}

\textbf{General Comment:} Remember that less/greater than or equal to includes the endpoint, while less/greater do not. Also, remember that you need to flip the inequality when you multiply or divide by a negative.
}
\litem{
Solve the linear inequality below. Then, choose the constant and interval combination that describes the solution set.
\[ -3 + 9 x > 10 x \text{ or } 4 + 9 x < 12 x \]The solution is \( (-\infty, -3.0) \text{ or } (1.333, \infty) \), which is option B.\begin{enumerate}[label=\Alph*.]
\item \( (-\infty, a] \cup [b, \infty), \text{ where } a \in [-5.25, -1.5] \text{ and } b \in [-0.75, 2.25] \)

Corresponds to including the endpoints (when they should be excluded).
\item \( (-\infty, a) \cup (b, \infty), \text{ where } a \in [-4.95, -1.95] \text{ and } b \in [0.75, 2.25] \)

 * Correct option.
\item \( (-\infty, a) \cup (b, \infty), \text{ where } a \in [-1.72, -0.22] \text{ and } b \in [1.5, 6.75] \)

Corresponds to inverting the inequality and negating the solution.
\item \( (-\infty, a] \cup [b, \infty), \text{ where } a \in [-1.5, 6] \text{ and } b \in [1.5, 12] \)

Corresponds to including the endpoints AND negating.
\item \( (-\infty, \infty) \)

Corresponds to the variable canceling, which does not happen in this instance.
\end{enumerate}

\textbf{General Comment:} When multiplying or dividing by a negative, flip the sign.
}
\litem{
Solve the linear inequality below. Then, choose the constant and interval combination that describes the solution set.
\[ -8 + 8 x \leq \frac{50 x + 4}{6} < 6 + 6 x \]The solution is \( [-26.00, 2.29) \), which is option B.\begin{enumerate}[label=\Alph*.]
\item \( (-\infty, a) \cup [b, \infty), \text{ where } a \in [-28.5, -22.5] \text{ and } b \in [0.75, 3.75] \)

$(-\infty, -26.00) \cup [2.29, \infty)$, which corresponds to displaying the and-inequality as an or-inequality AND flipping the inequality.
\item \( [a, b), \text{ where } a \in [-32.25, -24] \text{ and } b \in [-1.5, 3] \)

$[-26.00, 2.29)$, which is the correct option.
\item \( (-\infty, a] \cup (b, \infty), \text{ where } a \in [-26.25, -24.75] \text{ and } b \in [-2.25, 3.75] \)

$(-\infty, -26.00] \cup (2.29, \infty)$, which corresponds to displaying the and-inequality as an or-inequality.
\item \( (a, b], \text{ where } a \in [-27, -22.5] \text{ and } b \in [-0.75, 6] \)

$(-26.00, 2.29]$, which corresponds to flipping the inequality.
\item \( \text{None of the above.} \)


\end{enumerate}

\textbf{General Comment:} To solve, you will need to break up the compound inequality into two inequalities. Be sure to keep track of the inequality! It may be best to draw a number line and graph your solution.
}
\litem{
Using an interval or intervals, describe all the $x$-values within or including a distance of the given values.
\[ \text{ Less than } 4 \text{ units from the number } 10. \]The solution is \( (6, 14) \), which is option A.\begin{enumerate}[label=\Alph*.]
\item \( (6, 14) \)

This describes the values less than 4 from 10
\item \( [6, 14] \)

This describes the values no more than 4 from 10
\item \( (-\infty, 6] \cup [14, \infty) \)

This describes the values no less than 4 from 10
\item \( (-\infty, 6) \cup (14, \infty) \)

This describes the values more than 4 from 10
\item \( \text{None of the above} \)

You likely thought the values in the interval were not correct.
\end{enumerate}

\textbf{General Comment:} When thinking about this language, it helps to draw a number line and try points.
}
\litem{
Solve the linear inequality below. Then, choose the constant and interval combination that describes the solution set.
\[ \frac{6}{5} - \frac{7}{9} x < \frac{-5}{8} x + \frac{9}{3} \]The solution is \( (-11.782, \infty) \), which is option B.\begin{enumerate}[label=\Alph*.]
\item \( (a, \infty), \text{ where } a \in [9.75, 15] \)

 $(11.782, \infty)$, which corresponds to negating the endpoint of the solution.
\item \( (a, \infty), \text{ where } a \in [-12, -8.25] \)

* $(-11.782, \infty)$, which is the correct option.
\item \( (-\infty, a), \text{ where } a \in [8.25, 13.5] \)

 $(-\infty, 11.782)$, which corresponds to switching the direction of the interval AND negating the endpoint. You likely did this if you did not flip the inequality when dividing by a negative as well as not moving values over to a side properly.
\item \( (-\infty, a), \text{ where } a \in [-13.5, -9.75] \)

 $(-\infty, -11.782)$, which corresponds to switching the direction of the interval. You likely did this if you did not flip the inequality when dividing by a negative!
\item \( \text{None of the above}. \)

You may have chosen this if you thought the inequality did not match the ends of the intervals.
\end{enumerate}

\textbf{General Comment:} Remember that less/greater than or equal to includes the endpoint, while less/greater do not. Also, remember that you need to flip the inequality when you multiply or divide by a negative.
}
\litem{
Solve the linear inequality below. Then, choose the constant and interval combination that describes the solution set.
\[ -10x -9 < -7x -8 \]The solution is \( (-0.333, \infty) \), which is option D.\begin{enumerate}[label=\Alph*.]
\item \( (a, \infty), \text{ where } a \in [-0.17, 0.68] \)

 $(0.333, \infty)$, which corresponds to negating the endpoint of the solution.
\item \( (-\infty, a), \text{ where } a \in [0.1, 1.4] \)

 $(-\infty, 0.333)$, which corresponds to switching the direction of the interval AND negating the endpoint. You likely did this if you did not flip the inequality when dividing by a negative as well as not moving values over to a side properly.
\item \( (-\infty, a), \text{ where } a \in [-0.7, -0.2] \)

 $(-\infty, -0.333)$, which corresponds to switching the direction of the interval. You likely did this if you did not flip the inequality when dividing by a negative!
\item \( (a, \infty), \text{ where } a \in [-0.48, -0.25] \)

* $(-0.333, \infty)$, which is the correct option.
\item \( \text{None of the above}. \)

You may have chosen this if you thought the inequality did not match the ends of the intervals.
\end{enumerate}

\textbf{General Comment:} Remember that less/greater than or equal to includes the endpoint, while less/greater do not. Also, remember that you need to flip the inequality when you multiply or divide by a negative.
}
\litem{
Solve the linear inequality below. Then, choose the constant and interval combination that describes the solution set.
\[ \frac{-5}{4} - \frac{9}{8} x \leq \frac{-5}{6} x + \frac{7}{9} \]The solution is \( [-6.952, \infty) \), which is option D.\begin{enumerate}[label=\Alph*.]
\item \( (-\infty, a], \text{ where } a \in [3, 7.5] \)

 $(-\infty, 6.952]$, which corresponds to switching the direction of the interval AND negating the endpoint. You likely did this if you did not flip the inequality when dividing by a negative as well as not moving values over to a side properly.
\item \( [a, \infty), \text{ where } a \in [3.75, 10.5] \)

 $[6.952, \infty)$, which corresponds to negating the endpoint of the solution.
\item \( (-\infty, a], \text{ where } a \in [-8.25, -6.75] \)

 $(-\infty, -6.952]$, which corresponds to switching the direction of the interval. You likely did this if you did not flip the inequality when dividing by a negative!
\item \( [a, \infty), \text{ where } a \in [-11.25, -4.5] \)

* $[-6.952, \infty)$, which is the correct option.
\item \( \text{None of the above}. \)

You may have chosen this if you thought the inequality did not match the ends of the intervals.
\end{enumerate}

\textbf{General Comment:} Remember that less/greater than or equal to includes the endpoint, while less/greater do not. Also, remember that you need to flip the inequality when you multiply or divide by a negative.
}
\litem{
Using an interval or intervals, describe all the $x$-values within or including a distance of the given values.
\[ \text{ Less than } 5 \text{ units from the number } -6. \]The solution is \( (-11, -1) \), which is option C.\begin{enumerate}[label=\Alph*.]
\item \( (-\infty, -11] \cup [-1, \infty) \)

This describes the values no less than 5 from -6
\item \( (-\infty, -11) \cup (-1, \infty) \)

This describes the values more than 5 from -6
\item \( (-11, -1) \)

This describes the values less than 5 from -6
\item \( [-11, -1] \)

This describes the values no more than 5 from -6
\item \( \text{None of the above} \)

You likely thought the values in the interval were not correct.
\end{enumerate}

\textbf{General Comment:} When thinking about this language, it helps to draw a number line and try points.
}
\litem{
Solve the linear inequality below. Then, choose the constant and interval combination that describes the solution set.
\[ 3 + 7 x > 10 x \text{ or } 8 + 9 x < 11 x \]The solution is \( (-\infty, 1.0) \text{ or } (4.0, \infty) \), which is option A.\begin{enumerate}[label=\Alph*.]
\item \( (-\infty, a) \cup (b, \infty), \text{ where } a \in [-2.25, 3] \text{ and } b \in [3, 5.25] \)

 * Correct option.
\item \( (-\infty, a] \cup [b, \infty), \text{ where } a \in [0.75, 5.25] \text{ and } b \in [3.75, 9] \)

Corresponds to including the endpoints (when they should be excluded).
\item \( (-\infty, a] \cup [b, \infty), \text{ where } a \in [-7.5, -3] \text{ and } b \in [-3, 1.5] \)

Corresponds to including the endpoints AND negating.
\item \( (-\infty, a) \cup (b, \infty), \text{ where } a \in [-5.25, 0.75] \text{ and } b \in [-6, 0.75] \)

Corresponds to inverting the inequality and negating the solution.
\item \( (-\infty, \infty) \)

Corresponds to the variable canceling, which does not happen in this instance.
\end{enumerate}

\textbf{General Comment:} When multiplying or dividing by a negative, flip the sign.
}
\litem{
Solve the linear inequality below. Then, choose the constant and interval combination that describes the solution set.
\[ -5 - 8 x \leq \frac{-61 x - 9}{8} < -3 - 9 x \]The solution is \( \text{None of the above.} \), which is option E.\begin{enumerate}[label=\Alph*.]
\item \( (-\infty, a] \cup (b, \infty), \text{ where } a \in [6.75, 13.5] \text{ and } b \in [-0.75, 5.25] \)

$(-\infty, 10.33] \cup (1.36, \infty)$, which corresponds to displaying the and-inequality as an or-inequality and getting negatives of the actual endpoints.
\item \( [a, b), \text{ where } a \in [5.25, 12.75] \text{ and } b \in [0.3, 2.02] \)

$[10.33, 1.36)$, which is the correct interval but negatives of the actual endpoints.
\item \( (a, b], \text{ where } a \in [9.75, 11.25] \text{ and } b \in [0, 6.75] \)

$(10.33, 1.36]$, which corresponds to flipping the inequality and getting negatives of the actual endpoints.
\item \( (-\infty, a) \cup [b, \infty), \text{ where } a \in [9, 13.5] \text{ and } b \in [-0.38, 3.38] \)

$(-\infty, 10.33) \cup [1.36, \infty)$, which corresponds to displaying the and-inequality as an or-inequality AND flipping the inequality AND getting negatives of the actual endpoints.
\item \( \text{None of the above.} \)

* This is correct as the answer should be $[-10.33, -1.36)$.
\end{enumerate}

\textbf{General Comment:} To solve, you will need to break up the compound inequality into two inequalities. Be sure to keep track of the inequality! It may be best to draw a number line and graph your solution.
}
\litem{
Solve the linear inequality below. Then, choose the constant and interval combination that describes the solution set.
\[ 6x + 9 \geq 10x -8 \]The solution is \( (-\infty, 4.25] \), which is option A.\begin{enumerate}[label=\Alph*.]
\item \( (-\infty, a], \text{ where } a \in [3.25, 10.25] \)

* $(-\infty, 4.25]$, which is the correct option.
\item \( (-\infty, a], \text{ where } a \in [-5.25, 0.75] \)

 $(-\infty, -4.25]$, which corresponds to negating the endpoint of the solution.
\item \( [a, \infty), \text{ where } a \in [4.25, 9.25] \)

 $[4.25, \infty)$, which corresponds to switching the direction of the interval. You likely did this if you did not flip the inequality when dividing by a negative!
\item \( [a, \infty), \text{ where } a \in [-10.25, 1.75] \)

 $[-4.25, \infty)$, which corresponds to switching the direction of the interval AND negating the endpoint. You likely did this if you did not flip the inequality when dividing by a negative as well as not moving values over to a side properly.
\item \( \text{None of the above}. \)

You may have chosen this if you thought the inequality did not match the ends of the intervals.
\end{enumerate}

\textbf{General Comment:} Remember that less/greater than or equal to includes the endpoint, while less/greater do not. Also, remember that you need to flip the inequality when you multiply or divide by a negative.
}
\litem{
Solve the linear inequality below. Then, choose the constant and interval combination that describes the solution set.
\[ -9 + 4 x > 5 x \text{ or } 6 + 3 x < 4 x \]The solution is \( (-\infty, -9.0) \text{ or } (6.0, \infty) \), which is option A.\begin{enumerate}[label=\Alph*.]
\item \( (-\infty, a) \cup (b, \infty), \text{ where } a \in [-9.6, -7.88] \text{ and } b \in [2.92, 7.95] \)

 * Correct option.
\item \( (-\infty, a) \cup (b, \infty), \text{ where } a \in [-7.65, -5.7] \text{ and } b \in [6.75, 10.12] \)

Corresponds to inverting the inequality and negating the solution.
\item \( (-\infty, a] \cup [b, \infty), \text{ where } a \in [-7.5, -0.75] \text{ and } b \in [6.75, 11.25] \)

Corresponds to including the endpoints AND negating.
\item \( (-\infty, a] \cup [b, \infty), \text{ where } a \in [-11.25, -8.25] \text{ and } b \in [1.5, 6.75] \)

Corresponds to including the endpoints (when they should be excluded).
\item \( (-\infty, \infty) \)

Corresponds to the variable canceling, which does not happen in this instance.
\end{enumerate}

\textbf{General Comment:} When multiplying or dividing by a negative, flip the sign.
}
\litem{
Solve the linear inequality below. Then, choose the constant and interval combination that describes the solution set.
\[ -5 - 4 x \leq \frac{-33 x - 4}{9} < 9 - 6 x \]The solution is \( \text{None of the above.} \), which is option E.\begin{enumerate}[label=\Alph*.]
\item \( [a, b), \text{ where } a \in [9.75, 19.5] \text{ and } b \in [-7.5, -1.5] \)

$[13.67, -4.05)$, which is the correct interval but negatives of the actual endpoints.
\item \( (a, b], \text{ where } a \in [10.5, 18] \text{ and } b \in [-5.25, 3] \)

$(13.67, -4.05]$, which corresponds to flipping the inequality and getting negatives of the actual endpoints.
\item \( (-\infty, a] \cup (b, \infty), \text{ where } a \in [8.25, 14.25] \text{ and } b \in [-6, -1.5] \)

$(-\infty, 13.67] \cup (-4.05, \infty)$, which corresponds to displaying the and-inequality as an or-inequality and getting negatives of the actual endpoints.
\item \( (-\infty, a) \cup [b, \infty), \text{ where } a \in [10.5, 18.75] \text{ and } b \in [-6.75, 1.5] \)

$(-\infty, 13.67) \cup [-4.05, \infty)$, which corresponds to displaying the and-inequality as an or-inequality AND flipping the inequality AND getting negatives of the actual endpoints.
\item \( \text{None of the above.} \)

* This is correct as the answer should be $[-13.67, 4.05)$.
\end{enumerate}

\textbf{General Comment:} To solve, you will need to break up the compound inequality into two inequalities. Be sure to keep track of the inequality! It may be best to draw a number line and graph your solution.
}
\litem{
Using an interval or intervals, describe all the $x$-values within or including a distance of the given values.
\[ \text{ More than } 7 \text{ units from the number } 3. \]The solution is \( \text{None of the above} \), which is option E.\begin{enumerate}[label=\Alph*.]
\item \( (4, 10) \)

This describes the values less than 3 from 7
\item \( [4, 10] \)

This describes the values no more than 3 from 7
\item \( (-\infty, 4] \cup [10, \infty) \)

This describes the values no less than 3 from 7
\item \( (-\infty, 4) \cup (10, \infty) \)

This describes the values more than 3 from 7
\item \( \text{None of the above} \)

Options A-D described the values [more/less than] 3 units from 7, which is the reverse of what the question asked.
\end{enumerate}

\textbf{General Comment:} When thinking about this language, it helps to draw a number line and try points.
}
\litem{
Solve the linear inequality below. Then, choose the constant and interval combination that describes the solution set.
\[ \frac{9}{2} - \frac{7}{4} x \leq \frac{7}{6} x - \frac{10}{3} \]The solution is \( [2.686, \infty) \), which is option B.\begin{enumerate}[label=\Alph*.]
\item \( (-\infty, a], \text{ where } a \in [-3.75, -2.25] \)

 $(-\infty, -2.686]$, which corresponds to switching the direction of the interval AND negating the endpoint. You likely did this if you did not flip the inequality when dividing by a negative as well as not moving values over to a side properly.
\item \( [a, \infty), \text{ where } a \in [1.5, 3.75] \)

* $[2.686, \infty)$, which is the correct option.
\item \( (-\infty, a], \text{ where } a \in [0, 5.25] \)

 $(-\infty, 2.686]$, which corresponds to switching the direction of the interval. You likely did this if you did not flip the inequality when dividing by a negative!
\item \( [a, \infty), \text{ where } a \in [-7.5, 0.75] \)

 $[-2.686, \infty)$, which corresponds to negating the endpoint of the solution.
\item \( \text{None of the above}. \)

You may have chosen this if you thought the inequality did not match the ends of the intervals.
\end{enumerate}

\textbf{General Comment:} Remember that less/greater than or equal to includes the endpoint, while less/greater do not. Also, remember that you need to flip the inequality when you multiply or divide by a negative.
}
\litem{
Solve the linear inequality below. Then, choose the constant and interval combination that describes the solution set.
\[ -10x + 8 < -8x + 7 \]The solution is \( (0.5, \infty) \), which is option D.\begin{enumerate}[label=\Alph*.]
\item \( (a, \infty), \text{ where } a \in [-2.1, -0.2] \)

 $(-0.5, \infty)$, which corresponds to negating the endpoint of the solution.
\item \( (-\infty, a), \text{ where } a \in [-2.98, -0.08] \)

 $(-\infty, -0.5)$, which corresponds to switching the direction of the interval AND negating the endpoint. You likely did this if you did not flip the inequality when dividing by a negative as well as not moving values over to a side properly.
\item \( (-\infty, a), \text{ where } a \in [0.18, 1.43] \)

 $(-\infty, 0.5)$, which corresponds to switching the direction of the interval. You likely did this if you did not flip the inequality when dividing by a negative!
\item \( (a, \infty), \text{ where } a \in [0.4, 4] \)

* $(0.5, \infty)$, which is the correct option.
\item \( \text{None of the above}. \)

You may have chosen this if you thought the inequality did not match the ends of the intervals.
\end{enumerate}

\textbf{General Comment:} Remember that less/greater than or equal to includes the endpoint, while less/greater do not. Also, remember that you need to flip the inequality when you multiply or divide by a negative.
}
\litem{
Solve the linear inequality below. Then, choose the constant and interval combination that describes the solution set.
\[ \frac{-6}{2} - \frac{10}{8} x \leq \frac{-9}{3} x + \frac{10}{9} \]The solution is \( (-\infty, 2.349] \), which is option B.\begin{enumerate}[label=\Alph*.]
\item \( [a, \infty), \text{ where } a \in [-3.75, 0] \)

 $[-2.349, \infty)$, which corresponds to switching the direction of the interval AND negating the endpoint. You likely did this if you did not flip the inequality when dividing by a negative as well as not moving values over to a side properly.
\item \( (-\infty, a], \text{ where } a \in [0.75, 3.75] \)

* $(-\infty, 2.349]$, which is the correct option.
\item \( [a, \infty), \text{ where } a \in [-1.5, 4.5] \)

 $[2.349, \infty)$, which corresponds to switching the direction of the interval. You likely did this if you did not flip the inequality when dividing by a negative!
\item \( (-\infty, a], \text{ where } a \in [-4.5, -0.75] \)

 $(-\infty, -2.349]$, which corresponds to negating the endpoint of the solution.
\item \( \text{None of the above}. \)

You may have chosen this if you thought the inequality did not match the ends of the intervals.
\end{enumerate}

\textbf{General Comment:} Remember that less/greater than or equal to includes the endpoint, while less/greater do not. Also, remember that you need to flip the inequality when you multiply or divide by a negative.
}
\litem{
Using an interval or intervals, describe all the $x$-values within or including a distance of the given values.
\[ \text{ More than } 8 \text{ units from the number } -7. \]The solution is \( (-\infty, -15) \cup (1, \infty) \), which is option D.\begin{enumerate}[label=\Alph*.]
\item \( (-15, 1) \)

This describes the values less than 8 from -7
\item \( [-15, 1] \)

This describes the values no more than 8 from -7
\item \( (-\infty, -15] \cup [1, \infty) \)

This describes the values no less than 8 from -7
\item \( (-\infty, -15) \cup (1, \infty) \)

This describes the values more than 8 from -7
\item \( \text{None of the above} \)

You likely thought the values in the interval were not correct.
\end{enumerate}

\textbf{General Comment:} When thinking about this language, it helps to draw a number line and try points.
}
\litem{
Solve the linear inequality below. Then, choose the constant and interval combination that describes the solution set.
\[ -5 + 9 x > 10 x \text{ or } -9 + 3 x < 6 x \]The solution is \( (-\infty, -5.0) \text{ or } (-3.0, \infty) \), which is option D.\begin{enumerate}[label=\Alph*.]
\item \( (-\infty, a] \cup [b, \infty), \text{ where } a \in [1.5, 6] \text{ and } b \in [0.75, 12.75] \)

Corresponds to including the endpoints AND negating.
\item \( (-\infty, a] \cup [b, \infty), \text{ where } a \in [-6, -0.75] \text{ and } b \in [-3.75, -2.25] \)

Corresponds to including the endpoints (when they should be excluded).
\item \( (-\infty, a) \cup (b, \infty), \text{ where } a \in [1.5, 6] \text{ and } b \in [2.25, 6] \)

Corresponds to inverting the inequality and negating the solution.
\item \( (-\infty, a) \cup (b, \infty), \text{ where } a \in [-11.25, -2.25] \text{ and } b \in [-6, -2.25] \)

 * Correct option.
\item \( (-\infty, \infty) \)

Corresponds to the variable canceling, which does not happen in this instance.
\end{enumerate}

\textbf{General Comment:} When multiplying or dividing by a negative, flip the sign.
}
\litem{
Solve the linear inequality below. Then, choose the constant and interval combination that describes the solution set.
\[ -4 + 5 x \leq \frac{42 x - 7}{8} < -5 + 4 x \]The solution is \( [-12.50, -3.30) \), which is option D.\begin{enumerate}[label=\Alph*.]
\item \( (-\infty, a) \cup [b, \infty), \text{ where } a \in [-18, -12] \text{ and } b \in [-3.75, 1.5] \)

$(-\infty, -12.50) \cup [-3.30, \infty)$, which corresponds to displaying the and-inequality as an or-inequality AND flipping the inequality.
\item \( (a, b], \text{ where } a \in [-15.75, -9.75] \text{ and } b \in [-8.25, 0.75] \)

$(-12.50, -3.30]$, which corresponds to flipping the inequality.
\item \( (-\infty, a] \cup (b, \infty), \text{ where } a \in [-17.25, -11.25] \text{ and } b \in [-5.25, -3] \)

$(-\infty, -12.50] \cup (-3.30, \infty)$, which corresponds to displaying the and-inequality as an or-inequality.
\item \( [a, b), \text{ where } a \in [-16.5, -11.25] \text{ and } b \in [-6, 0] \)

$[-12.50, -3.30)$, which is the correct option.
\item \( \text{None of the above.} \)


\end{enumerate}

\textbf{General Comment:} To solve, you will need to break up the compound inequality into two inequalities. Be sure to keep track of the inequality! It may be best to draw a number line and graph your solution.
}
\litem{
Solve the linear inequality below. Then, choose the constant and interval combination that describes the solution set.
\[ -3x -9 < 3x -10 \]The solution is \( (0.167, \infty) \), which is option C.\begin{enumerate}[label=\Alph*.]
\item \( (-\infty, a), \text{ where } a \in [-0.36, 0.02] \)

 $(-\infty, -0.167)$, which corresponds to switching the direction of the interval AND negating the endpoint. You likely did this if you did not flip the inequality when dividing by a negative as well as not moving values over to a side properly.
\item \( (-\infty, a), \text{ where } a \in [0.09, 0.46] \)

 $(-\infty, 0.167)$, which corresponds to switching the direction of the interval. You likely did this if you did not flip the inequality when dividing by a negative!
\item \( (a, \infty), \text{ where } a \in [-0.04, 0.52] \)

* $(0.167, \infty)$, which is the correct option.
\item \( (a, \infty), \text{ where } a \in [-0.83, 0.16] \)

 $(-0.167, \infty)$, which corresponds to negating the endpoint of the solution.
\item \( \text{None of the above}. \)

You may have chosen this if you thought the inequality did not match the ends of the intervals.
\end{enumerate}

\textbf{General Comment:} Remember that less/greater than or equal to includes the endpoint, while less/greater do not. Also, remember that you need to flip the inequality when you multiply or divide by a negative.
}
\litem{
Solve the linear inequality below. Then, choose the constant and interval combination that describes the solution set.
\[ -7 + 9 x > 12 x \text{ or } 3 + 7 x < 10 x \]The solution is \( (-\infty, -2.333) \text{ or } (1.0, \infty) \), which is option C.\begin{enumerate}[label=\Alph*.]
\item \( (-\infty, a] \cup [b, \infty), \text{ where } a \in [-1.5, 0.38] \text{ and } b \in [1.88, 4.2] \)

Corresponds to including the endpoints AND negating.
\item \( (-\infty, a] \cup [b, \infty), \text{ where } a \in [-2.48, -1.43] \text{ and } b \in [-1.88, 1.12] \)

Corresponds to including the endpoints (when they should be excluded).
\item \( (-\infty, a) \cup (b, \infty), \text{ where } a \in [-3.75, -2.25] \text{ and } b \in [0.85, 1.04] \)

 * Correct option.
\item \( (-\infty, a) \cup (b, \infty), \text{ where } a \in [-1.5, 2.25] \text{ and } b \in [2.31, 4.06] \)

Corresponds to inverting the inequality and negating the solution.
\item \( (-\infty, \infty) \)

Corresponds to the variable canceling, which does not happen in this instance.
\end{enumerate}

\textbf{General Comment:} When multiplying or dividing by a negative, flip the sign.
}
\litem{
Solve the linear inequality below. Then, choose the constant and interval combination that describes the solution set.
\[ -7 - 6 x < \frac{-15 x - 9}{5} \leq 7 - 4 x \]The solution is \( (-1.73, 8.80] \), which is option A.\begin{enumerate}[label=\Alph*.]
\item \( (a, b], \text{ where } a \in [-3, 0] \text{ and } b \in [7.5, 12] \)

* $(-1.73, 8.80]$, which is the correct option.
\item \( (-\infty, a) \cup [b, \infty), \text{ where } a \in [-3, 0] \text{ and } b \in [4.5, 12.75] \)

$(-\infty, -1.73) \cup [8.80, \infty)$, which corresponds to displaying the and-inequality as an or-inequality.
\item \( (-\infty, a] \cup (b, \infty), \text{ where } a \in [-2.25, 1.5] \text{ and } b \in [7.5, 15] \)

$(-\infty, -1.73] \cup (8.80, \infty)$, which corresponds to displaying the and-inequality as an or-inequality AND flipping the inequality.
\item \( [a, b), \text{ where } a \in [-3, -0.75] \text{ and } b \in [6.75, 11.25] \)

$[-1.73, 8.80)$, which corresponds to flipping the inequality.
\item \( \text{None of the above.} \)


\end{enumerate}

\textbf{General Comment:} To solve, you will need to break up the compound inequality into two inequalities. Be sure to keep track of the inequality! It may be best to draw a number line and graph your solution.
}
\litem{
Using an interval or intervals, describe all the $x$-values within or including a distance of the given values.
\[ \text{ Less than } 5 \text{ units from the number } -5. \]The solution is \( (-10, 0) \), which is option C.\begin{enumerate}[label=\Alph*.]
\item \( (-\infty, -10] \cup [0, \infty) \)

This describes the values no less than 5 from -5
\item \( (-\infty, -10) \cup (0, \infty) \)

This describes the values more than 5 from -5
\item \( (-10, 0) \)

This describes the values less than 5 from -5
\item \( [-10, 0] \)

This describes the values no more than 5 from -5
\item \( \text{None of the above} \)

You likely thought the values in the interval were not correct.
\end{enumerate}

\textbf{General Comment:} When thinking about this language, it helps to draw a number line and try points.
}
\litem{
Solve the linear inequality below. Then, choose the constant and interval combination that describes the solution set.
\[ \frac{7}{4} - \frac{4}{6} x \leq \frac{-3}{8} x - \frac{4}{5} \]The solution is \( [8.743, \infty) \), which is option C.\begin{enumerate}[label=\Alph*.]
\item \( (-\infty, a], \text{ where } a \in [6, 14.25] \)

 $(-\infty, 8.743]$, which corresponds to switching the direction of the interval. You likely did this if you did not flip the inequality when dividing by a negative!
\item \( (-\infty, a], \text{ where } a \in [-10.5, -6.75] \)

 $(-\infty, -8.743]$, which corresponds to switching the direction of the interval AND negating the endpoint. You likely did this if you did not flip the inequality when dividing by a negative as well as not moving values over to a side properly.
\item \( [a, \infty), \text{ where } a \in [8.25, 12] \)

* $[8.743, \infty)$, which is the correct option.
\item \( [a, \infty), \text{ where } a \in [-9, -6] \)

 $[-8.743, \infty)$, which corresponds to negating the endpoint of the solution.
\item \( \text{None of the above}. \)

You may have chosen this if you thought the inequality did not match the ends of the intervals.
\end{enumerate}

\textbf{General Comment:} Remember that less/greater than or equal to includes the endpoint, while less/greater do not. Also, remember that you need to flip the inequality when you multiply or divide by a negative.
}
\litem{
Solve the linear inequality below. Then, choose the constant and interval combination that describes the solution set.
\[ -4x + 6 > 3x + 3 \]The solution is \( (-\infty, 0.429) \), which is option B.\begin{enumerate}[label=\Alph*.]
\item \( (-\infty, a), \text{ where } a \in [-1.38, -0.13] \)

 $(-\infty, -0.429)$, which corresponds to negating the endpoint of the solution.
\item \( (-\infty, a), \text{ where } a \in [0.41, 1.04] \)

* $(-\infty, 0.429)$, which is the correct option.
\item \( (a, \infty), \text{ where } a \in [0.36, 0.73] \)

 $(0.429, \infty)$, which corresponds to switching the direction of the interval. You likely did this if you did not flip the inequality when dividing by a negative!
\item \( (a, \infty), \text{ where } a \in [-1.04, -0.24] \)

 $(-0.429, \infty)$, which corresponds to switching the direction of the interval AND negating the endpoint. You likely did this if you did not flip the inequality when dividing by a negative as well as not moving values over to a side properly.
\item \( \text{None of the above}. \)

You may have chosen this if you thought the inequality did not match the ends of the intervals.
\end{enumerate}

\textbf{General Comment:} Remember that less/greater than or equal to includes the endpoint, while less/greater do not. Also, remember that you need to flip the inequality when you multiply or divide by a negative.
}
\end{enumerate}

\end{document}