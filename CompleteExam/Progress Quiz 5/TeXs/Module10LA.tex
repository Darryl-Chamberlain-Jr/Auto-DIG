\documentclass[14pt]{extbook}
\usepackage{multicol, enumerate, enumitem, hyperref, color, soul, setspace, parskip, fancyhdr} %General Packages
\usepackage{amssymb, amsthm, amsmath, bbm, latexsym, units, mathtools} %Math Packages
\everymath{\displaystyle} %All math in Display Style
% Packages with additional options
\usepackage[headsep=0.5cm,headheight=12pt, left=1 in,right= 1 in,top= 1 in,bottom= 1 in]{geometry}
\usepackage[usenames,dvipsnames]{xcolor}
\usepackage{dashrule}  % Package to use the command below to create lines between items
\newcommand{\litem}[1]{\item#1\hspace*{-1cm}\rule{\textwidth}{0.4pt}}
\pagestyle{fancy}
\lhead{Progress Quiz 5}
\chead{}
\rhead{Version A}
\lfoot{9912-2038}
\cfoot{}
\rfoot{Spring 2021}
\begin{document}

\begin{enumerate}
\litem{
What are the \textit{possible Integer} roots of the polynomial below?\[ f(x) = 2x^{3} +2 x^{2} +3 x + 4 \]\begin{enumerate}[label=\Alph*.]
\item \( \pm 1,\pm 2,\pm 4 \)
\item \( \text{ All combinations of: }\frac{\pm 1,\pm 2,\pm 4}{\pm 1,\pm 2} \)
\item \( \pm 1,\pm 2 \)
\item \( \text{ All combinations of: }\frac{\pm 1,\pm 2}{\pm 1,\pm 2,\pm 4} \)
\item \( \text{There is no formula or theorem that tells us all possible Integer roots.} \)

\end{enumerate} }
\litem{
Factor the polynomial below completely. Then, choose the intervals the zeros of the polynomial belong to, where $z_1 \leq z_2 \leq z_3$. \textit{To make the problem easier, all zeros are between -5 and 5.}\[ f(x) = 25x^{3} -130 x^{2} +13 x + 60 \]\begin{enumerate}[label=\Alph*.]
\item \( z_1 \in [-5.9, -4.8], \text{   }  z_2 \in [-4.78, -3.37], \text{   and   } z_3 \in [-0.49, 0.26] \)
\item \( z_1 \in [-1.1, 0.8], \text{   }  z_2 \in [0.63, 1.01], \text{   and   } z_3 \in [4.66, 5.1] \)
\item \( z_1 \in [-5.9, -4.8], \text{   }  z_2 \in [-1.14, -0.35], \text{   and   } z_3 \in [0.38, 1.13] \)
\item \( z_1 \in [-5.9, -4.8], \text{   }  z_2 \in [-1.85, -1.24], \text{   and   } z_3 \in [1.59, 1.95] \)
\item \( z_1 \in [-2.9, -1.3], \text{   }  z_2 \in [1.02, 1.57], \text{   and   } z_3 \in [4.66, 5.1] \)

\end{enumerate} }
\litem{
Perform the division below. Then, find the intervals that correspond to the quotient in the form $ax^2+bx+c$ and remainder $r$.\[ \frac{6x^{3} -18 x + 17}{x + 2} \]\begin{enumerate}[label=\Alph*.]
\item \( a \in [-12, -6], b \in [-29, -20], c \in [-68, -62], \text{ and } r \in [-116, -114]. \)
\item \( a \in [5, 11], b \in [-15, -10], c \in [5, 14], \text{ and } r \in [5, 14]. \)
\item \( a \in [-12, -6], b \in [15, 29], c \in [-68, -62], \text{ and } r \in [146, 158]. \)
\item \( a \in [5, 11], b \in [11, 19], c \in [5, 14], \text{ and } r \in [28, 31]. \)
\item \( a \in [5, 11], b \in [-19, -17], c \in [31, 41], \text{ and } r \in [-94, -87]. \)

\end{enumerate} }
\litem{
Factor the polynomial below completely. Then, choose the intervals the zeros of the polynomial belong to, where $z_1 \leq z_2 \leq z_3$. \textit{To make the problem easier, all zeros are between -5 and 5.}\[ f(x) = 20x^{3} -113 x^{2} +142 x -40 \]\begin{enumerate}[label=\Alph*.]
\item \( z_1 \in [-4.35, -3.54], \text{   }  z_2 \in [-1.7, -0.8], \text{   and   } z_3 \in [-0.68, -0.15] \)
\item \( z_1 \in [0.7, 1.11], \text{   }  z_2 \in [1.4, 2.8], \text{   and   } z_3 \in [3.76, 4.6] \)
\item \( z_1 \in [-4.35, -3.54], \text{   }  z_2 \in [-3.1, -1.7], \text{   and   } z_3 \in [-0.85, -0.5] \)
\item \( z_1 \in [-0.32, 0.75], \text{   }  z_2 \in [1, 1.6], \text{   and   } z_3 \in [3.76, 4.6] \)
\item \( z_1 \in [-5.12, -4.89], \text{   }  z_2 \in [-5.3, -3.9], \text{   and   } z_3 \in [-0.25, 0.34] \)

\end{enumerate} }
\litem{
Perform the division below. Then, find the intervals that correspond to the quotient in the form $ax^2+bx+c$ and remainder $r$.\[ \frac{6x^{3} +18 x^{2} -26}{x + 2} \]\begin{enumerate}[label=\Alph*.]
\item \( a \in [-19, -10], b \in [41, 50], c \in [-85, -81], \text{ and } r \in [142, 144]. \)
\item \( a \in [-19, -10], b \in [-11, -4], c \in [-13, -6], \text{ and } r \in [-51, -49]. \)
\item \( a \in [3, 13], b \in [30, 31], c \in [52, 64], \text{ and } r \in [92, 97]. \)
\item \( a \in [3, 13], b \in [-3, 2], c \in [-3, 5], \text{ and } r \in [-26, -23]. \)
\item \( a \in [3, 13], b \in [2, 12], c \in [-13, -6], \text{ and } r \in [-8, 0]. \)

\end{enumerate} }
\litem{
Perform the division below. Then, find the intervals that correspond to the quotient in the form $ax^2+bx+c$ and remainder $r$.\[ \frac{12x^{3} +55 x^{2} +18 x -43}{x + 4} \]\begin{enumerate}[label=\Alph*.]
\item \( a \in [-49, -43], \text{   } b \in [-137, -134], \text{   } c \in [-536, -529], \text{   and   } r \in [-2165, -2159]. \)
\item \( a \in [10, 15], \text{   } b \in [103, 107], \text{   } c \in [426, 435], \text{   and   } r \in [1677, 1679]. \)
\item \( a \in [-49, -43], \text{   } b \in [239, 248], \text{   } c \in [-975, -969], \text{   and   } r \in [3835, 3842]. \)
\item \( a \in [10, 15], \text{   } b \in [5, 8], \text{   } c \in [-16, -3], \text{   and   } r \in [-10, 2]. \)
\item \( a \in [10, 15], \text{   } b \in [-9, -4], \text{   } c \in [41, 45], \text{   and   } r \in [-263, -257]. \)

\end{enumerate} }
\litem{
Perform the division below. Then, find the intervals that correspond to the quotient in the form $ax^2+bx+c$ and remainder $r$.\[ \frac{9x^{3} +18 x^{2} -37 x -26}{x + 3} \]\begin{enumerate}[label=\Alph*.]
\item \( a \in [3, 16], \text{   } b \in [45, 51], \text{   } c \in [92, 101], \text{   and   } r \in [262, 275]. \)
\item \( a \in [-32, -26], \text{   } b \in [97, 105], \text{   } c \in [-337, -333], \text{   and   } r \in [975, 981]. \)
\item \( a \in [3, 16], \text{   } b \in [-13, -5], \text{   } c \in [-12, -7], \text{   and   } r \in [-3, 8]. \)
\item \( a \in [3, 16], \text{   } b \in [-20, -14], \text{   } c \in [30, 36], \text{   and   } r \in [-168, -163]. \)
\item \( a \in [-32, -26], \text{   } b \in [-63, -56], \text{   } c \in [-228, -223], \text{   and   } r \in [-709, -702]. \)

\end{enumerate} }
\litem{
Factor the polynomial below completely, knowing that $x-3$ is a factor. Then, choose the intervals the zeros of the polynomial belong to, where $z_1 \leq z_2 \leq z_3 \leq z_4$. \textit{To make the problem easier, all zeros are between -5 and 5.}\[ f(x) = 16x^{4} +64 x^{3} -161 x^{2} -450 x -225 \]\begin{enumerate}[label=\Alph*.]
\item \( z_1 \in [-3.6, -2.8], \text{   }  z_2 \in [0.76, 0.89], z_3 \in [1.29, 1.34], \text{   and   } z_4 \in [4, 6] \)
\item \( z_1 \in [-3.6, -2.8], \text{   }  z_2 \in [0.07, 0.25], z_3 \in [4.99, 5.04], \text{   and   } z_4 \in [4, 6] \)
\item \( z_1 \in [-3.6, -2.8], \text{   }  z_2 \in [0.69, 0.79], z_3 \in [1.17, 1.28], \text{   and   } z_4 \in [4, 6] \)
\item \( z_1 \in [-5.5, -4.1], \text{   }  z_2 \in [-1.34, -1.26], z_3 \in [-0.82, -0.77], \text{   and   } z_4 \in [0, 4] \)
\item \( z_1 \in [-5.5, -4.1], \text{   }  z_2 \in [-1.25, -1.19], z_3 \in [-0.78, -0.72], \text{   and   } z_4 \in [0, 4] \)

\end{enumerate} }
\litem{
Factor the polynomial below completely, knowing that $x+5$ is a factor. Then, choose the intervals the zeros of the polynomial belong to, where $z_1 \leq z_2 \leq z_3 \leq z_4$. \textit{To make the problem easier, all zeros are between -5 and 5.}\[ f(x) = 12x^{4} +101 x^{3} +245 x^{2} +212 x + 60 \]\begin{enumerate}[label=\Alph*.]
\item \( z_1 \in [0.53, 1.14], \text{   }  z_2 \in [0.68, 1.03], z_3 \in [1.13, 2.11], \text{   and   } z_4 \in [4.3, 5.8] \)
\item \( z_1 \in [-5.56, -4.38], \text{   }  z_2 \in [-2.09, -1.75], z_3 \in [-1.59, -1.04], \text{   and   } z_4 \in [-1.7, -1] \)
\item \( z_1 \in [-0.44, 0.45], \text{   }  z_2 \in [1.71, 2.14], z_3 \in [1.13, 2.11], \text{   and   } z_4 \in [4.3, 5.8] \)
\item \( z_1 \in [1.08, 2.04], \text{   }  z_2 \in [1.12, 1.66], z_3 \in [1.13, 2.11], \text{   and   } z_4 \in [4.3, 5.8] \)
\item \( z_1 \in [-5.56, -4.38], \text{   }  z_2 \in [-2.09, -1.75], z_3 \in [-0.76, 0.49], \text{   and   } z_4 \in [-0.9, 0.7] \)

\end{enumerate} }
\litem{
What are the \textit{possible Rational} roots of the polynomial below?\[ f(x) = 3x^{2} +2 x + 7 \]\begin{enumerate}[label=\Alph*.]
\item \( \pm 1,\pm 3 \)
\item \( \pm 1,\pm 7 \)
\item \( \text{ All combinations of: }\frac{\pm 1,\pm 3}{\pm 1,\pm 7} \)
\item \( \text{ All combinations of: }\frac{\pm 1,\pm 7}{\pm 1,\pm 3} \)
\item \( \text{ There is no formula or theorem that tells us all possible Rational roots.} \)

\end{enumerate} }
\end{enumerate}

\end{document}