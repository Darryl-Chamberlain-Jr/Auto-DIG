\documentclass[14pt]{extbook}
\usepackage{multicol, enumerate, enumitem, hyperref, color, soul, setspace, parskip, fancyhdr} %General Packages
\usepackage{amssymb, amsthm, amsmath, latexsym, units, mathtools} %Math Packages
\everymath{\displaystyle} %All math in Display Style
% Packages with additional options
\usepackage[headsep=0.5cm,headheight=12pt, left=1 in,right= 1 in,top= 1 in,bottom= 1 in]{geometry}
\usepackage[usenames,dvipsnames]{xcolor}
\usepackage{dashrule}  % Package to use the command below to create lines between items
\newcommand{\litem}[1]{\item#1\hspace*{-1cm}\rule{\textwidth}{0.4pt}}
\pagestyle{fancy}
\lhead{Progress Quiz 5}
\chead{}
\rhead{Version A}
\lfoot{8497-6012}
\cfoot{}
\rfoot{Summer C 2021}
\begin{document}

\begin{enumerate}
\litem{
Perform the division below. Then, find the intervals that correspond to the quotient in the form $ax^2+bx+c$ and remainder $r$.\[ \frac{12x^{3} +39 x^{2} -30}{x + 3} \]\begin{enumerate}[label=\Alph*.]
\item \( a \in [-38, -33], b \in [147, 149], c \in [-441, -436], \text{ and } r \in [1291, 1296]. \)
\item \( a \in [11, 13], b \in [75, 77], c \in [220, 232], \text{ and } r \in [644, 646]. \)
\item \( a \in [11, 13], b \in [-3, 5], c \in [-12, -2], \text{ and } r \in [-7, 2]. \)
\item \( a \in [-38, -33], b \in [-70, -65], c \in [-207, -199], \text{ and } r \in [-654, -650]. \)
\item \( a \in [11, 13], b \in [-14, -8], c \in [29, 38], \text{ and } r \in [-181, -171]. \)

\end{enumerate} }
\litem{
Factor the polynomial below completely. Then, choose the intervals the zeros of the polynomial belong to, where $z_1 \leq z_2 \leq z_3$. \textit{To make the problem easier, all zeros are between -5 and 5.}\[ f(x) = 15x^{3} -44 x^{2} -79 x + 60 \]\begin{enumerate}[label=\Alph*.]
\item \( z_1 \in [-5, -2], \text{   }  z_2 \in [-0.9, -0.2], \text{   and   } z_3 \in [0.92, 1.85] \)
\item \( z_1 \in [-5, -2], \text{   }  z_2 \in [-2.2, -0.9], \text{   and   } z_3 \in [0.42, 0.84] \)
\item \( z_1 \in [-0.6, 0.4], \text{   }  z_2 \in [1.1, 2.8], \text{   and   } z_3 \in [3.7, 4.42] \)
\item \( z_1 \in [-1.67, -0.67], \text{   }  z_2 \in [-0.3, 0.8], \text{   and   } z_3 \in [3.7, 4.42] \)
\item \( z_1 \in [-5, -2], \text{   }  z_2 \in [-3.3, -2.1], \text{   and   } z_3 \in [-0.02, 0.34] \)

\end{enumerate} }
\litem{
Factor the polynomial below completely. Then, choose the intervals the zeros of the polynomial belong to, where $z_1 \leq z_2 \leq z_3$. \textit{To make the problem easier, all zeros are between -5 and 5.}\[ f(x) = 10x^{3} +3 x^{2} -79 x -60 \]\begin{enumerate}[label=\Alph*.]
\item \( z_1 \in [-3.1, -2.7], \text{   }  z_2 \in [0, 0.71], \text{   and   } z_3 \in [4.9, 5.12] \)
\item \( z_1 \in [-3.1, -2.7], \text{   }  z_2 \in [0, 0.71], \text{   and   } z_3 \in [1.12, 1.66] \)
\item \( z_1 \in [-2.8, -1.6], \text{   }  z_2 \in [-0.89, -0.5], \text{   and   } z_3 \in [2.53, 3.23] \)
\item \( z_1 \in [-3.1, -2.7], \text{   }  z_2 \in [0.74, 1.16], \text{   and   } z_3 \in [2.35, 2.78] \)
\item \( z_1 \in [-1.5, -0.9], \text{   }  z_2 \in [-0.58, -0.22], \text{   and   } z_3 \in [2.53, 3.23] \)

\end{enumerate} }
\litem{
Perform the division below. Then, find the intervals that correspond to the quotient in the form $ax^2+bx+c$ and remainder $r$.\[ \frac{4x^{3} -75 x -129}{x -5} \]\begin{enumerate}[label=\Alph*.]
\item \( a \in [2, 7], b \in [12, 18], c \in [-14, -6], \text{ and } r \in [-181, -171]. \)
\item \( a \in [2, 7], b \in [-26, -14], c \in [21, 26], \text{ and } r \in [-257, -246]. \)
\item \( a \in [17, 22], b \in [-103, -96], c \in [424, 427], \text{ and } r \in [-2255, -2252]. \)
\item \( a \in [17, 22], b \in [96, 105], c \in [424, 427], \text{ and } r \in [1992, 1999]. \)
\item \( a \in [2, 7], b \in [17, 23], c \in [21, 26], \text{ and } r \in [-7, -3]. \)

\end{enumerate} }
\litem{
Perform the division below. Then, find the intervals that correspond to the quotient in the form $ax^2+bx+c$ and remainder $r$.\[ \frac{20x^{3} +55 x^{2} -30 x -43}{x + 3} \]\begin{enumerate}[label=\Alph*.]
\item \( a \in [-62, -56], \text{   } b \in [-130, -124], \text{   } c \in [-406, -400], \text{   and   } r \in [-1263, -1256]. \)
\item \( a \in [19, 26], \text{   } b \in [111, 121], \text{   } c \in [310, 319], \text{   and   } r \in [898, 908]. \)
\item \( a \in [-62, -56], \text{   } b \in [231, 237], \text{   } c \in [-735, -733], \text{   and   } r \in [2160, 2164]. \)
\item \( a \in [19, 26], \text{   } b \in [-26, -22], \text{   } c \in [66, 73], \text{   and   } r \in [-327, -319]. \)
\item \( a \in [19, 26], \text{   } b \in [-6, -1], \text{   } c \in [-18, -14], \text{   and   } r \in [1, 8]. \)

\end{enumerate} }
\litem{
Factor the polynomial below completely, knowing that $x -5$ is a factor. Then, choose the intervals the zeros of the polynomial belong to, where $z_1 \leq z_2 \leq z_3 \leq z_4$. \textit{To make the problem easier, all zeros are between -5 and 5.}\[ f(x) = 10x^{4} -113 x^{3} +434 x^{2} -655 x + 300 \]\begin{enumerate}[label=\Alph*.]
\item \( z_1 \in [-1.5, 0.7], \text{   }  z_2 \in [0.88, 2.15], z_3 \in [2.83, 3.07], \text{   and   } z_4 \in [4.76, 5.22] \)
\item \( z_1 \in [-6.1, -4.5], \text{   }  z_2 \in [-3.06, -1.3], z_3 \in [-1.58, -0.95], \text{   and   } z_4 \in [-0.43, -0.37] \)
\item \( z_1 \in [0.5, 0.9], \text{   }  z_2 \in [2.3, 2.76], z_3 \in [2.83, 3.07], \text{   and   } z_4 \in [4.76, 5.22] \)
\item \( z_1 \in [-6.1, -4.5], \text{   }  z_2 \in [-3.06, -1.3], z_3 \in [-2.56, -2.44], \text{   and   } z_4 \in [-0.91, -0.66] \)
\item \( z_1 \in [-6.1, -4.5], \text{   }  z_2 \in [-4.78, -3.6], z_3 \in [-3.23, -2.63], \text{   and   } z_4 \in [-0.61, -0.48] \)

\end{enumerate} }
\litem{
Factor the polynomial below completely, knowing that $x -2$ is a factor. Then, choose the intervals the zeros of the polynomial belong to, where $z_1 \leq z_2 \leq z_3 \leq z_4$. \textit{To make the problem easier, all zeros are between -5 and 5.}\[ f(x) = 12x^{4} -83 x^{3} +197 x^{2} -188 x + 60 \]\begin{enumerate}[label=\Alph*.]
\item \( z_1 \in [-3.21, -2.92], \text{   }  z_2 \in [-2.11, -1.9], z_3 \in [-1.87, -1.4], \text{   and   } z_4 \in [-0.97, -0.76] \)
\item \( z_1 \in [-3.21, -2.92], \text{   }  z_2 \in [-2.11, -1.9], z_3 \in [-2.06, -1.61], \text{   and   } z_4 \in [-0.48, -0.26] \)
\item \( z_1 \in [0.79, 1.04], \text{   }  z_2 \in [1.45, 1.69], z_3 \in [1.79, 2.39], \text{   and   } z_4 \in [2.98, 3.13] \)
\item \( z_1 \in [-3.21, -2.92], \text{   }  z_2 \in [-2.11, -1.9], z_3 \in [-1.42, -1.16], \text{   and   } z_4 \in [-0.73, -0.63] \)
\item \( z_1 \in [0.42, 0.78], \text{   }  z_2 \in [0.65, 1.49], z_3 \in [1.79, 2.39], \text{   and   } z_4 \in [2.98, 3.13] \)

\end{enumerate} }
\litem{
Perform the division below. Then, find the intervals that correspond to the quotient in the form $ax^2+bx+c$ and remainder $r$.\[ \frac{10x^{3} -46 x^{2} +40 x + 22}{x -3} \]\begin{enumerate}[label=\Alph*.]
\item \( a \in [29, 35], \text{   } b \in [41, 48], \text{   } c \in [169, 174], \text{   and   } r \in [534, 540]. \)
\item \( a \in [10, 11], \text{   } b \in [-18, -9], \text{   } c \in [-8, -7], \text{   and   } r \in [-5, 2]. \)
\item \( a \in [10, 11], \text{   } b \in [-76, -75], \text{   } c \in [265, 271], \text{   and   } r \in [-787, -778]. \)
\item \( a \in [29, 35], \text{   } b \in [-137, -132], \text{   } c \in [448, 452], \text{   and   } r \in [-1326, -1315]. \)
\item \( a \in [10, 11], \text{   } b \in [-30, -22], \text{   } c \in [-12, -9], \text{   and   } r \in [-5, 2]. \)

\end{enumerate} }
\litem{
What are the \textit{possible Integer} roots of the polynomial below?\[ f(x) = 6x^{2} +7 x + 7 \]\begin{enumerate}[label=\Alph*.]
\item \( \pm 1,\pm 2,\pm 3,\pm 6 \)
\item \( \pm 1,\pm 7 \)
\item \( \text{ All combinations of: }\frac{\pm 1,\pm 2,\pm 3,\pm 6}{\pm 1,\pm 7} \)
\item \( \text{ All combinations of: }\frac{\pm 1,\pm 7}{\pm 1,\pm 2,\pm 3,\pm 6} \)
\item \( \text{There is no formula or theorem that tells us all possible Integer roots.} \)

\end{enumerate} }
\litem{
What are the \textit{possible Integer} roots of the polynomial below?\[ f(x) = 6x^{2} +6 x + 4 \]\begin{enumerate}[label=\Alph*.]
\item \( \pm 1,\pm 2,\pm 4 \)
\item \( \pm 1,\pm 2,\pm 3,\pm 6 \)
\item \( \text{ All combinations of: }\frac{\pm 1,\pm 2,\pm 3,\pm 6}{\pm 1,\pm 2,\pm 4} \)
\item \( \text{ All combinations of: }\frac{\pm 1,\pm 2,\pm 4}{\pm 1,\pm 2,\pm 3,\pm 6} \)
\item \( \text{There is no formula or theorem that tells us all possible Integer roots.} \)

\end{enumerate} }
\end{enumerate}

\end{document}