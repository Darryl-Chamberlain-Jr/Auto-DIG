\documentclass[14pt]{extbook}
\usepackage{multicol, enumerate, enumitem, hyperref, color, soul, setspace, parskip, fancyhdr} %General Packages
\usepackage{amssymb, amsthm, amsmath, latexsym, units, mathtools} %Math Packages
\everymath{\displaystyle} %All math in Display Style
% Packages with additional options
\usepackage[headsep=0.5cm,headheight=12pt, left=1 in,right= 1 in,top= 1 in,bottom= 1 in]{geometry}
\usepackage[usenames,dvipsnames]{xcolor}
\usepackage{dashrule}  % Package to use the command below to create lines between items
\newcommand{\litem}[1]{\item#1\hspace*{-1cm}\rule{\textwidth}{0.4pt}}
\pagestyle{fancy}
\lhead{Progress Quiz 5}
\chead{}
\rhead{Version B}
\lfoot{8497-6012}
\cfoot{}
\rfoot{Summer C 2021}
\begin{document}

\begin{enumerate}
\litem{
Find the inverse of the function below. Then, evaluate the inverse at $x = 7$ and choose the interval that $f^-1(7)$ belongs to.\[ f(x) = \ln{(x+2)}-5 \]\begin{enumerate}[label=\Alph*.]
\item \( f^{-1}(7) \in [162750.79, 162754.79] \)
\item \( f^{-1}(7) \in [-0.61, 7.39] \)
\item \( f^{-1}(7) \in [143.41, 144.41] \)
\item \( f^{-1}(7) \in [162753.79, 162764.79] \)
\item \( f^{-1}(7) \in [8098.08, 8102.08] \)

\end{enumerate} }
\litem{
Multiply the following functions, then choose the domain of the resulting function from the list below.\[ f(x) = \frac{2}{4x+21} \text{ and } g(x) = \frac{2}{6x-23} \]\begin{enumerate}[label=\Alph*.]
\item \( \text{ The domain is all Real numbers except } x = a, \text{ where } a \in [0.4, 11.4] \)
\item \( \text{ The domain is all Real numbers greater than or equal to } x = a, \text{ where } a \in [1, 9] \)
\item \( \text{ The domain is all Real numbers less than or equal to } x = a, \text{ where } a \in [-6.67, -2.67] \)
\item \( \text{ The domain is all Real numbers except } x = a \text{ and } x = b, \text{ where } a \in [-7.25, -4.25] \text{ and } b \in [0.83, 7.83] \)
\item \( \text{ The domain is all Real numbers. } \)

\end{enumerate} }
\litem{
Find the inverse of the function below (if it exists). Then, evaluate the inverse at $x = 11$ and choose the interval that $f^-1(11)$ belongs to.\[ f(x) = \sqrt[3]{2 x + 3} \]\begin{enumerate}[label=\Alph*.]
\item \( f^{-1}(11) \in [663.3, 664.8] \)
\item \( f^{-1}(11) \in [664.9, 667.9] \)
\item \( f^{-1}(11) \in [-664.5, -661.8] \)
\item \( f^{-1}(11) \in [-669.6, -664.4] \)
\item \( \text{ The function is not invertible for all Real numbers. } \)

\end{enumerate} }
\litem{
Determine whether the function below is 1-1.\[ f(x) = -9 x^2 + 15 x + 234 \]\begin{enumerate}[label=\Alph*.]
\item \( \text{No, because the domain of the function is not $(-\infty, \infty)$.} \)
\item \( \text{No, because there is an $x$-value that goes to 2 different $y$-values.} \)
\item \( \text{No, because the range of the function is not $(-\infty, \infty)$.} \)
\item \( \text{Yes, the function is 1-1.} \)
\item \( \text{No, because there is a $y$-value that goes to 2 different $x$-values.} \)

\end{enumerate} }
\litem{
Add the following functions, then choose the domain of the resulting function from the list below.\[ f(x) = x + 6 \text{ and } g(x) = \frac{1}{4x-13} \]\begin{enumerate}[label=\Alph*.]
\item \( \text{ The domain is all Real numbers except } x = a, \text{ where } a \in [2.25, 6.25] \)
\item \( \text{ The domain is all Real numbers less than or equal to } x = a, \text{ where } a \in [-6.4, -2.4] \)
\item \( \text{ The domain is all Real numbers greater than or equal to } x = a, \text{ where } a \in [-6.75, -2.75] \)
\item \( \text{ The domain is all Real numbers except } x = a \text{ and } x = b, \text{ where } a \in [-12.33, 2.67] \text{ and } b \in [-8.67, -3.67] \)
\item \( \text{ The domain is all Real numbers. } \)

\end{enumerate} }
\litem{
Find the inverse of the function below. Then, evaluate the inverse at $x = 7$ and choose the interval that $f^-1(7)$ belongs to.\[ f(x) = e^{x-4}+5 \]\begin{enumerate}[label=\Alph*.]
\item \( f^{-1}(7) \in [6.02, 6.21] \)
\item \( f^{-1}(7) \in [4.66, 4.73] \)
\item \( f^{-1}(7) \in [7.35, 7.45] \)
\item \( f^{-1}(7) \in [-3.34, -3.28] \)
\item \( f^{-1}(7) \in [7.41, 7.5] \)

\end{enumerate} }
\litem{
Choose the interval below that $f$ composed with $g$ at $x=1$ is in.\[ f(x) = -2x^{3} -2 x^{2} +2 x \text{ and } g(x) = -2x^{3} -3 x^{2} +3 x + 1 \]\begin{enumerate}[label=\Alph*.]
\item \( (f \circ g)(1) \in [-1.78, -0.74] \)
\item \( (f \circ g)(1) \in [2.64, 3.93] \)
\item \( (f \circ g)(1) \in [-6.26, -5.68] \)
\item \( (f \circ g)(1) \in [-2.2, -1.72] \)
\item \( \text{It is not possible to compose the two functions.} \)

\end{enumerate} }
\litem{
Find the inverse of the function below (if it exists). Then, evaluate the inverse at $x = 12$ and choose the interval that $f^-1(12)$ belongs to.\[ f(x) = \sqrt[3]{3 x + 4} \]\begin{enumerate}[label=\Alph*.]
\item \( f^{-1}(12) \in [574, 576.8] \)
\item \( f^{-1}(12) \in [577.3, 578.8] \)
\item \( f^{-1}(12) \in [-580.7, -574.7] \)
\item \( f^{-1}(12) \in [-575.6, -573.9] \)
\item \( \text{ The function is not invertible for all Real numbers. } \)

\end{enumerate} }
\litem{
Choose the interval below that $f$ composed with $g$ at $x=1$ is in.\[ f(x) = -3x^{3} -2 x^{2} +3 x + 4 \text{ and } g(x) = x^{3} -2 x^{2} +3 x \]\begin{enumerate}[label=\Alph*.]
\item \( (f \circ g)(1) \in [-30, -24] \)
\item \( (f \circ g)(1) \in [6, 11] \)
\item \( (f \circ g)(1) \in [-26, -20] \)
\item \( (f \circ g)(1) \in [-6, 1] \)
\item \( \text{It is not possible to compose the two functions.} \)

\end{enumerate} }
\litem{
Determine whether the function below is 1-1.\[ f(x) = (4 x + 13)^3 \]\begin{enumerate}[label=\Alph*.]
\item \( \text{No, because the domain of the function is not $(-\infty, \infty)$.} \)
\item \( \text{No, because there is an $x$-value that goes to 2 different $y$-values.} \)
\item \( \text{Yes, the function is 1-1.} \)
\item \( \text{No, because there is a $y$-value that goes to 2 different $x$-values.} \)
\item \( \text{No, because the range of the function is not $(-\infty, \infty)$.} \)

\end{enumerate} }
\end{enumerate}

\end{document}