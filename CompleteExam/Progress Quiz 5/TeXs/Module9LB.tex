\documentclass[14pt]{extbook}
\usepackage{multicol, enumerate, enumitem, hyperref, color, soul, setspace, parskip, fancyhdr} %General Packages
\usepackage{amssymb, amsthm, amsmath, bbm, latexsym, units, mathtools} %Math Packages
\everymath{\displaystyle} %All math in Display Style
% Packages with additional options
\usepackage[headsep=0.5cm,headheight=12pt, left=1 in,right= 1 in,top= 1 in,bottom= 1 in]{geometry}
\usepackage[usenames,dvipsnames]{xcolor}
\usepackage{dashrule}  % Package to use the command below to create lines between items
\newcommand{\litem}[1]{\item#1\hspace*{-1cm}\rule{\textwidth}{0.4pt}}
\pagestyle{fancy}
\lhead{Progress Quiz 5}
\chead{}
\rhead{Version B}
\lfoot{9912-2038}
\cfoot{}
\rfoot{Spring 2021}
\begin{document}

\begin{enumerate}
\litem{
Find the inverse of the function below (if it exists). Then, evaluate the inverse at $x = -15$ and choose the interval the $f^{-1}(-15)$ belongs to.\[ f(x) = \sqrt[3]{5 x - 2} \]\begin{enumerate}[label=\Alph*.]
\item \( f^{-1}(-15) \in [-674.9, -674.14] \)
\item \( f^{-1}(-15) \in [-675.91, -675.39] \)
\item \( f^{-1}(-15) \in [675.02, 675.54] \)
\item \( f^{-1}(-15) \in [674.57, 675.24] \)
\item \( \text{ The function is not invertible for all Real numbers. } \)

\end{enumerate} }
\litem{
Choose the interval below that $f$ composed with $g$ at $x=1$ is in.\[ f(x) = -2x^{3} +3 x^{2} +x -1 \text{ and } g(x) = 2x^{3} -1 x^{2} -2 x \]\begin{enumerate}[label=\Alph*.]
\item \( (f \circ g)(1) \in [1.72, 3.84] \)
\item \( (f \circ g)(1) \in [8.47, 9.37] \)
\item \( (f \circ g)(1) \in [6.95, 8.28] \)
\item \( (f \circ g)(1) \in [-2.32, -0.32] \)
\item \( \text{It is not possible to compose the two functions.} \)

\end{enumerate} }
\litem{
Find the inverse of the function below. Then, evaluate the inverse at $x = 7$ and choose the interval that $f^{-1}(7)$ belongs to.\[ f(x) = e^{x-5}+4 \]\begin{enumerate}[label=\Alph*.]
\item \( f^{-1}(7) \in [-3.92, -3.77] \)
\item \( f^{-1}(7) \in [6.08, 6.18] \)
\item \( f^{-1}(7) \in [4.61, 4.72] \)
\item \( f^{-1}(7) \in [6.44, 6.5] \)
\item \( f^{-1}(7) \in [6.34, 6.45] \)

\end{enumerate} }
\litem{
Determine whether the function below is 1-1.\[ f(x) = -18 x^2 + 132 x - 224 \]\begin{enumerate}[label=\Alph*.]
\item \( \text{No, because there is a $y$-value that goes to 2 different $x$-values.} \)
\item \( \text{Yes, the function is 1-1.} \)
\item \( \text{No, because the range of the function is not $(-\infty, \infty)$.} \)
\item \( \text{No, because there is an $x$-value that goes to 2 different $y$-values.} \)
\item \( \text{No, because the domain of the function is not $(-\infty, \infty)$.} \)

\end{enumerate} }
\litem{
Determine whether the function below is 1-1.\[ f(x) = \sqrt{4 x - 20} \]\begin{enumerate}[label=\Alph*.]
\item \( \text{No, because the range of the function is not $(-\infty, \infty)$.} \)
\item \( \text{No, because there is an $x$-value that goes to 2 different $y$-values.} \)
\item \( \text{No, because there is a $y$-value that goes to 2 different $x$-values.} \)
\item \( \text{Yes, the function is 1-1.} \)
\item \( \text{No, because the domain of the function is not $(-\infty, \infty)$.} \)

\end{enumerate} }
\litem{
Choose the interval below that $f$ composed with $g$ at $x=1$ is in.\[ f(x) = 2x^{3} +4 x^{2} -2 x \text{ and } g(x) = -x^{3} +3 x^{2} -2 x + 1 \]\begin{enumerate}[label=\Alph*.]
\item \( (f \circ g)(1) \in [-34.1, -31.8] \)
\item \( (f \circ g)(1) \in [-23.9, -21.6] \)
\item \( (f \circ g)(1) \in [1.9, 7.5] \)
\item \( (f \circ g)(1) \in [8.4, 10.6] \)
\item \( \text{It is not possible to compose the two functions.} \)

\end{enumerate} }
\litem{
Find the inverse of the function below (if it exists). Then, evaluate the inverse at $x = 10$ and choose the interval the $f^{-1}(10)$ belongs to.\[ f(x) = \sqrt[3]{4 x + 3} \]\begin{enumerate}[label=\Alph*.]
\item \( f^{-1}(10) \in [248.46, 249.97] \)
\item \( f^{-1}(10) \in [-249.7, -248.81] \)
\item \( f^{-1}(10) \in [-251.41, -249.48] \)
\item \( f^{-1}(10) \in [249.4, 252.77] \)
\item \( \text{ The function is not invertible for all Real numbers. } \)

\end{enumerate} }
\litem{
Find the inverse of the function below. Then, evaluate the inverse at $x = 6$ and choose the interval that $f^{-1}(6)$ belongs to.\[ f(x) = e^{x+4}+2 \]\begin{enumerate}[label=\Alph*.]
\item \( f^{-1}(6) \in [-3.24, -2.5] \)
\item \( f^{-1}(6) \in [2.44, 2.76] \)
\item \( f^{-1}(6) \in [5.09, 5.42] \)
\item \( f^{-1}(6) \in [3.8, 4.29] \)
\item \( f^{-1}(6) \in [4.12, 4.68] \)

\end{enumerate} }
\litem{
Subtract the following functions, then choose the domain of the resulting function from the list below.\[ f(x) = \sqrt{-5x-13}  \text{ and } g(x) = 4x + 6 \]\begin{enumerate}[label=\Alph*.]
\item \( \text{ The domain is all Real numbers except } x = a, \text{ where } a \in [0.17, 7.17] \)
\item \( \text{ The domain is all Real numbers greater than or equal to } x = a, \text{ where } a \in [-7.67, 0.33] \)
\item \( \text{ The domain is all Real numbers less than or equal to } x = a, \text{ where } a \in [-3.6, -0.6] \)
\item \( \text{ The domain is all Real numbers except } x = a \text{ and } x = b, \text{ where } a \in [4.33, 10.33] \text{ and } b \in [3.2, 10.2] \)
\item \( \text{ The domain is all Real numbers. } \)

\end{enumerate} }
\litem{
Subtract the following functions, then choose the domain of the resulting function from the list below.\[ f(x) = \sqrt{-5x-15}  \text{ and } g(x) = 5x^{3} +4 x^{2} +x + 2 \]\begin{enumerate}[label=\Alph*.]
\item \( \text{ The domain is all Real numbers except } x = a, \text{ where } a \in [-9.25, -5.25] \)
\item \( \text{ The domain is all Real numbers greater than or equal to } x = a, \text{ where } a \in [-5.5, -1.5] \)
\item \( \text{ The domain is all Real numbers less than or equal to } x = a, \text{ where } a \in [-5, 1] \)
\item \( \text{ The domain is all Real numbers except } x = a \text{ and } x = b, \text{ where } a \in [1.2, 10.2] \text{ and } b \in [6.33, 8.33] \)
\item \( \text{ The domain is all Real numbers. } \)

\end{enumerate} }
\end{enumerate}

\end{document}