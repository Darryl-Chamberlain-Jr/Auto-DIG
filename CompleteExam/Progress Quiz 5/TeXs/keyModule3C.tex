\documentclass{extbook}[14pt]
\usepackage{multicol, enumerate, enumitem, hyperref, color, soul, setspace, parskip, fancyhdr, amssymb, amsthm, amsmath, latexsym, units, mathtools}
\everymath{\displaystyle}
\usepackage[headsep=0.5cm,headheight=0cm, left=1 in,right= 1 in,top= 1 in,bottom= 1 in]{geometry}
\usepackage{dashrule}  % Package to use the command below to create lines between items
\newcommand{\litem}[1]{\item #1

\rule{\textwidth}{0.4pt}}
\pagestyle{fancy}
\lhead{}
\chead{Answer Key for Progress Quiz 5 Version C}
\rhead{}
\lfoot{8497-6012}
\cfoot{}
\rfoot{Summer C 2021}
\begin{document}
\textbf{This key should allow you to understand why you choose the option you did (beyond just getting a question right or wrong). \href{https://xronos.clas.ufl.edu/mac1105spring2020/courseDescriptionAndMisc/Exams/LearningFromResults}{More instructions on how to use this key can be found here}.}

\textbf{If you have a suggestion to make the keys better, \href{https://forms.gle/CZkbZmPbC9XALEE88}{please fill out the short survey here}.}

\textit{Note: This key is auto-generated and may contain issues and/or errors. The keys are reviewed after each exam to ensure grading is done accurately. If there are issues (like duplicate options), they are noted in the offline gradebook. The keys are a work-in-progress to give students as many resources to improve as possible.}

\rule{\textwidth}{0.4pt}

\begin{enumerate}\litem{
Solve the linear inequality below. Then, choose the constant and interval combination that describes the solution set.
\[ \frac{-6}{2} - \frac{10}{8} x \leq \frac{-9}{3} x + \frac{10}{9} \]The solution is \( (-\infty, 2.349] \), which is option B.\begin{enumerate}[label=\Alph*.]
\item \( [a, \infty), \text{ where } a \in [-3.75, 0] \)

 $[-2.349, \infty)$, which corresponds to switching the direction of the interval AND negating the endpoint. You likely did this if you did not flip the inequality when dividing by a negative as well as not moving values over to a side properly.
\item \( (-\infty, a], \text{ where } a \in [0.75, 3.75] \)

* $(-\infty, 2.349]$, which is the correct option.
\item \( [a, \infty), \text{ where } a \in [-1.5, 4.5] \)

 $[2.349, \infty)$, which corresponds to switching the direction of the interval. You likely did this if you did not flip the inequality when dividing by a negative!
\item \( (-\infty, a], \text{ where } a \in [-4.5, -0.75] \)

 $(-\infty, -2.349]$, which corresponds to negating the endpoint of the solution.
\item \( \text{None of the above}. \)

You may have chosen this if you thought the inequality did not match the ends of the intervals.
\end{enumerate}

\textbf{General Comment:} Remember that less/greater than or equal to includes the endpoint, while less/greater do not. Also, remember that you need to flip the inequality when you multiply or divide by a negative.
}
\litem{
Using an interval or intervals, describe all the $x$-values within or including a distance of the given values.
\[ \text{ More than } 8 \text{ units from the number } -7. \]The solution is \( (-\infty, -15) \cup (1, \infty) \), which is option D.\begin{enumerate}[label=\Alph*.]
\item \( (-15, 1) \)

This describes the values less than 8 from -7
\item \( [-15, 1] \)

This describes the values no more than 8 from -7
\item \( (-\infty, -15] \cup [1, \infty) \)

This describes the values no less than 8 from -7
\item \( (-\infty, -15) \cup (1, \infty) \)

This describes the values more than 8 from -7
\item \( \text{None of the above} \)

You likely thought the values in the interval were not correct.
\end{enumerate}

\textbf{General Comment:} When thinking about this language, it helps to draw a number line and try points.
}
\litem{
Solve the linear inequality below. Then, choose the constant and interval combination that describes the solution set.
\[ -5 + 9 x > 10 x \text{ or } -9 + 3 x < 6 x \]The solution is \( (-\infty, -5.0) \text{ or } (-3.0, \infty) \), which is option D.\begin{enumerate}[label=\Alph*.]
\item \( (-\infty, a] \cup [b, \infty), \text{ where } a \in [1.5, 6] \text{ and } b \in [0.75, 12.75] \)

Corresponds to including the endpoints AND negating.
\item \( (-\infty, a] \cup [b, \infty), \text{ where } a \in [-6, -0.75] \text{ and } b \in [-3.75, -2.25] \)

Corresponds to including the endpoints (when they should be excluded).
\item \( (-\infty, a) \cup (b, \infty), \text{ where } a \in [1.5, 6] \text{ and } b \in [2.25, 6] \)

Corresponds to inverting the inequality and negating the solution.
\item \( (-\infty, a) \cup (b, \infty), \text{ where } a \in [-11.25, -2.25] \text{ and } b \in [-6, -2.25] \)

 * Correct option.
\item \( (-\infty, \infty) \)

Corresponds to the variable canceling, which does not happen in this instance.
\end{enumerate}

\textbf{General Comment:} When multiplying or dividing by a negative, flip the sign.
}
\litem{
Solve the linear inequality below. Then, choose the constant and interval combination that describes the solution set.
\[ -4 + 5 x \leq \frac{42 x - 7}{8} < -5 + 4 x \]The solution is \( [-12.50, -3.30) \), which is option D.\begin{enumerate}[label=\Alph*.]
\item \( (-\infty, a) \cup [b, \infty), \text{ where } a \in [-18, -12] \text{ and } b \in [-3.75, 1.5] \)

$(-\infty, -12.50) \cup [-3.30, \infty)$, which corresponds to displaying the and-inequality as an or-inequality AND flipping the inequality.
\item \( (a, b], \text{ where } a \in [-15.75, -9.75] \text{ and } b \in [-8.25, 0.75] \)

$(-12.50, -3.30]$, which corresponds to flipping the inequality.
\item \( (-\infty, a] \cup (b, \infty), \text{ where } a \in [-17.25, -11.25] \text{ and } b \in [-5.25, -3] \)

$(-\infty, -12.50] \cup (-3.30, \infty)$, which corresponds to displaying the and-inequality as an or-inequality.
\item \( [a, b), \text{ where } a \in [-16.5, -11.25] \text{ and } b \in [-6, 0] \)

$[-12.50, -3.30)$, which is the correct option.
\item \( \text{None of the above.} \)


\end{enumerate}

\textbf{General Comment:} To solve, you will need to break up the compound inequality into two inequalities. Be sure to keep track of the inequality! It may be best to draw a number line and graph your solution.
}
\litem{
Solve the linear inequality below. Then, choose the constant and interval combination that describes the solution set.
\[ -3x -9 < 3x -10 \]The solution is \( (0.167, \infty) \), which is option C.\begin{enumerate}[label=\Alph*.]
\item \( (-\infty, a), \text{ where } a \in [-0.36, 0.02] \)

 $(-\infty, -0.167)$, which corresponds to switching the direction of the interval AND negating the endpoint. You likely did this if you did not flip the inequality when dividing by a negative as well as not moving values over to a side properly.
\item \( (-\infty, a), \text{ where } a \in [0.09, 0.46] \)

 $(-\infty, 0.167)$, which corresponds to switching the direction of the interval. You likely did this if you did not flip the inequality when dividing by a negative!
\item \( (a, \infty), \text{ where } a \in [-0.04, 0.52] \)

* $(0.167, \infty)$, which is the correct option.
\item \( (a, \infty), \text{ where } a \in [-0.83, 0.16] \)

 $(-0.167, \infty)$, which corresponds to negating the endpoint of the solution.
\item \( \text{None of the above}. \)

You may have chosen this if you thought the inequality did not match the ends of the intervals.
\end{enumerate}

\textbf{General Comment:} Remember that less/greater than or equal to includes the endpoint, while less/greater do not. Also, remember that you need to flip the inequality when you multiply or divide by a negative.
}
\litem{
Solve the linear inequality below. Then, choose the constant and interval combination that describes the solution set.
\[ -7 + 9 x > 12 x \text{ or } 3 + 7 x < 10 x \]The solution is \( (-\infty, -2.333) \text{ or } (1.0, \infty) \), which is option C.\begin{enumerate}[label=\Alph*.]
\item \( (-\infty, a] \cup [b, \infty), \text{ where } a \in [-1.5, 0.38] \text{ and } b \in [1.88, 4.2] \)

Corresponds to including the endpoints AND negating.
\item \( (-\infty, a] \cup [b, \infty), \text{ where } a \in [-2.48, -1.43] \text{ and } b \in [-1.88, 1.12] \)

Corresponds to including the endpoints (when they should be excluded).
\item \( (-\infty, a) \cup (b, \infty), \text{ where } a \in [-3.75, -2.25] \text{ and } b \in [0.85, 1.04] \)

 * Correct option.
\item \( (-\infty, a) \cup (b, \infty), \text{ where } a \in [-1.5, 2.25] \text{ and } b \in [2.31, 4.06] \)

Corresponds to inverting the inequality and negating the solution.
\item \( (-\infty, \infty) \)

Corresponds to the variable canceling, which does not happen in this instance.
\end{enumerate}

\textbf{General Comment:} When multiplying or dividing by a negative, flip the sign.
}
\litem{
Solve the linear inequality below. Then, choose the constant and interval combination that describes the solution set.
\[ -7 - 6 x < \frac{-15 x - 9}{5} \leq 7 - 4 x \]The solution is \( (-1.73, 8.80] \), which is option A.\begin{enumerate}[label=\Alph*.]
\item \( (a, b], \text{ where } a \in [-3, 0] \text{ and } b \in [7.5, 12] \)

* $(-1.73, 8.80]$, which is the correct option.
\item \( (-\infty, a) \cup [b, \infty), \text{ where } a \in [-3, 0] \text{ and } b \in [4.5, 12.75] \)

$(-\infty, -1.73) \cup [8.80, \infty)$, which corresponds to displaying the and-inequality as an or-inequality.
\item \( (-\infty, a] \cup (b, \infty), \text{ where } a \in [-2.25, 1.5] \text{ and } b \in [7.5, 15] \)

$(-\infty, -1.73] \cup (8.80, \infty)$, which corresponds to displaying the and-inequality as an or-inequality AND flipping the inequality.
\item \( [a, b), \text{ where } a \in [-3, -0.75] \text{ and } b \in [6.75, 11.25] \)

$[-1.73, 8.80)$, which corresponds to flipping the inequality.
\item \( \text{None of the above.} \)


\end{enumerate}

\textbf{General Comment:} To solve, you will need to break up the compound inequality into two inequalities. Be sure to keep track of the inequality! It may be best to draw a number line and graph your solution.
}
\litem{
Using an interval or intervals, describe all the $x$-values within or including a distance of the given values.
\[ \text{ Less than } 5 \text{ units from the number } -5. \]The solution is \( (-10, 0) \), which is option C.\begin{enumerate}[label=\Alph*.]
\item \( (-\infty, -10] \cup [0, \infty) \)

This describes the values no less than 5 from -5
\item \( (-\infty, -10) \cup (0, \infty) \)

This describes the values more than 5 from -5
\item \( (-10, 0) \)

This describes the values less than 5 from -5
\item \( [-10, 0] \)

This describes the values no more than 5 from -5
\item \( \text{None of the above} \)

You likely thought the values in the interval were not correct.
\end{enumerate}

\textbf{General Comment:} When thinking about this language, it helps to draw a number line and try points.
}
\litem{
Solve the linear inequality below. Then, choose the constant and interval combination that describes the solution set.
\[ \frac{7}{4} - \frac{4}{6} x \leq \frac{-3}{8} x - \frac{4}{5} \]The solution is \( [8.743, \infty) \), which is option C.\begin{enumerate}[label=\Alph*.]
\item \( (-\infty, a], \text{ where } a \in [6, 14.25] \)

 $(-\infty, 8.743]$, which corresponds to switching the direction of the interval. You likely did this if you did not flip the inequality when dividing by a negative!
\item \( (-\infty, a], \text{ where } a \in [-10.5, -6.75] \)

 $(-\infty, -8.743]$, which corresponds to switching the direction of the interval AND negating the endpoint. You likely did this if you did not flip the inequality when dividing by a negative as well as not moving values over to a side properly.
\item \( [a, \infty), \text{ where } a \in [8.25, 12] \)

* $[8.743, \infty)$, which is the correct option.
\item \( [a, \infty), \text{ where } a \in [-9, -6] \)

 $[-8.743, \infty)$, which corresponds to negating the endpoint of the solution.
\item \( \text{None of the above}. \)

You may have chosen this if you thought the inequality did not match the ends of the intervals.
\end{enumerate}

\textbf{General Comment:} Remember that less/greater than or equal to includes the endpoint, while less/greater do not. Also, remember that you need to flip the inequality when you multiply or divide by a negative.
}
\litem{
Solve the linear inequality below. Then, choose the constant and interval combination that describes the solution set.
\[ -4x + 6 > 3x + 3 \]The solution is \( (-\infty, 0.429) \), which is option B.\begin{enumerate}[label=\Alph*.]
\item \( (-\infty, a), \text{ where } a \in [-1.38, -0.13] \)

 $(-\infty, -0.429)$, which corresponds to negating the endpoint of the solution.
\item \( (-\infty, a), \text{ where } a \in [0.41, 1.04] \)

* $(-\infty, 0.429)$, which is the correct option.
\item \( (a, \infty), \text{ where } a \in [0.36, 0.73] \)

 $(0.429, \infty)$, which corresponds to switching the direction of the interval. You likely did this if you did not flip the inequality when dividing by a negative!
\item \( (a, \infty), \text{ where } a \in [-1.04, -0.24] \)

 $(-0.429, \infty)$, which corresponds to switching the direction of the interval AND negating the endpoint. You likely did this if you did not flip the inequality when dividing by a negative as well as not moving values over to a side properly.
\item \( \text{None of the above}. \)

You may have chosen this if you thought the inequality did not match the ends of the intervals.
\end{enumerate}

\textbf{General Comment:} Remember that less/greater than or equal to includes the endpoint, while less/greater do not. Also, remember that you need to flip the inequality when you multiply or divide by a negative.
}
\end{enumerate}

\end{document}