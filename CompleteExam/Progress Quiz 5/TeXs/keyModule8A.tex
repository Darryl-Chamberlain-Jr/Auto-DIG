\documentclass{extbook}[14pt]
\usepackage{multicol, enumerate, enumitem, hyperref, color, soul, setspace, parskip, fancyhdr, amssymb, amsthm, amsmath, latexsym, units, mathtools}
\everymath{\displaystyle}
\usepackage[headsep=0.5cm,headheight=0cm, left=1 in,right= 1 in,top= 1 in,bottom= 1 in]{geometry}
\usepackage{dashrule}  % Package to use the command below to create lines between items
\newcommand{\litem}[1]{\item #1

\rule{\textwidth}{0.4pt}}
\pagestyle{fancy}
\lhead{}
\chead{Answer Key for Progress Quiz 5 Version A}
\rhead{}
\lfoot{8497-6012}
\cfoot{}
\rfoot{Summer C 2021}
\begin{document}
\textbf{This key should allow you to understand why you choose the option you did (beyond just getting a question right or wrong). \href{https://xronos.clas.ufl.edu/mac1105spring2020/courseDescriptionAndMisc/Exams/LearningFromResults}{More instructions on how to use this key can be found here}.}

\textbf{If you have a suggestion to make the keys better, \href{https://forms.gle/CZkbZmPbC9XALEE88}{please fill out the short survey here}.}

\textit{Note: This key is auto-generated and may contain issues and/or errors. The keys are reviewed after each exam to ensure grading is done accurately. If there are issues (like duplicate options), they are noted in the offline gradebook. The keys are a work-in-progress to give students as many resources to improve as possible.}

\rule{\textwidth}{0.4pt}

\begin{enumerate}\litem{
Solve the equation for $x$ and choose the interval that contains the solution (if it exists).
\[ 2^{5x+2} = 343^{2x-3} \]The solution is \( x = 2.302 \), which is option A.\begin{enumerate}[label=\Alph*.]
\item \( x \in [2, 3.3] \)

* $x = 2.302$, which is the correct option.
\item \( x \in [-8.6, -4.9] \)

$x = -6.300$, which corresponds to distributing the $\ln(base)$ to the second term of the exponent only.
\item \( x \in [-2.1, -1.4] \)

$x = -1.667$, which corresponds to solving the numerators as equal while ignoring the bases are different.
\item \( x \in [-1.5, 2] \)

$x = 0.609$, which corresponds to distributing the $\ln(base)$ to the first term of the exponent only.
\item \( \text{There is no Real solution to the equation.} \)

This corresponds to believing there is no solution since the bases are not powers of each other.
\end{enumerate}

\textbf{General Comment:} \textbf{General Comments:} This question was written so that the bases could not be written the same. You will need to take the log of both sides.
}
\litem{
Solve the equation for $x$ and choose the interval that contains the solution (if it exists).
\[ \log_{3}{(4x+6)}+6 = 3 \]The solution is \( x = -1.491 \), which is option A.\begin{enumerate}[label=\Alph*.]
\item \( x \in [-2.3, -1] \)

* $x = -1.491$, which is the correct option.
\item \( x \in [-8.7, -5.9] \)

$x = -8.250$, which corresponds to reversing the base and exponent when converting.
\item \( x \in [3.5, 7] \)

$x = 5.250$, which corresponds to ignoring the vertical shift when converting to exponential form.
\item \( x \in [-5.7, -4.9] \)

$x = -5.250$, which corresponds to reversing the base and exponent when converting and reversing the value with $x$.
\item \( \text{There is no Real solution to the equation.} \)

Corresponds to believing a negative coefficient within the log equation means there is no Real solution.
\end{enumerate}

\textbf{General Comment:} \textbf{General Comments:} First, get the equation in the form $\log_b{(cx+d)} = a$. Then, convert to $b^a = cx+d$ and solve.
}
\litem{
Which of the following intervals describes the Domain of the function below?
\[ f(x) = \log_2{(x-7)}+9 \]The solution is \( (7, \infty) \), which is option B.\begin{enumerate}[label=\Alph*.]
\item \( (-\infty, a), a \in [-8.38, -6.88] \)

$(-\infty, -7)$, which corresponds to flipping the Domain. Remember: the general for is $a*\log(x-h)+k$, \textbf{where $a$ does not affect the domain}.
\item \( (a, \infty), a \in [5.99, 7.58] \)

* $(7, \infty)$, which is the correct option.
\item \( (-\infty, a], a \in [-9.2, -7.76] \)

$(-\infty, -9]$, which corresponds to using the negative vertical shift AND including the endpoint AND flipping the domain.
\item \( [a, \infty), a \in [8.71, 9.45] \)

$[9, \infty)$, which corresponds to using the vertical shift when shifting the Domain AND including the endpoint.
\item \( (-\infty, \infty) \)

This corresponds to thinking of the range of the log function (or the domain of the exponential function).
\end{enumerate}

\textbf{General Comment:} \textbf{General Comments}: The domain of a basic logarithmic function is $(0, \infty)$ and the Range is $(-\infty, \infty)$. We can use shifts when finding the Domain, but the Range will always be all Real numbers.
}
\litem{
Solve the equation for $x$ and choose the interval that contains the solution (if it exists).
\[ 3^{-4x-5} = \left(\frac{1}{64}\right)^{2x-2} \]The solution is \( x = 3.520 \), which is option A.\begin{enumerate}[label=\Alph*.]
\item \( x \in [1.8, 5.5] \)

* $x = 3.520$, which is the correct option.
\item \( x \in [-1.3, -0.3] \)

$x = -0.500$, which corresponds to solving the numerators as equal while ignoring the bases are different.
\item \( x \in [-3.4, -2.2] \)

$x = -2.302$, which corresponds to distributing the $\ln(base)$ to the second term of the exponent only.
\item \( x \in [0.6, 1.7] \)

$x = 0.765$, which corresponds to distributing the $\ln(base)$ to the first term of the exponent only.
\item \( \text{There is no Real solution to the equation.} \)

This corresponds to believing there is no solution since the bases are not powers of each other.
\end{enumerate}

\textbf{General Comment:} \textbf{General Comments:} This question was written so that the bases could not be written the same. You will need to take the log of both sides.
}
\litem{
Which of the following intervals describes the Domain of the function below?
\[ f(x) = \log_2{(x-7)}-7 \]The solution is \( (7, \infty) \), which is option C.\begin{enumerate}[label=\Alph*.]
\item \( (-\infty, a], a \in [5, 11] \)

$(-\infty, 7]$, which corresponds to using the negative vertical shift AND including the endpoint AND flipping the domain.
\item \( (-\infty, a), a \in [-9, -4] \)

$(-\infty, -7)$, which corresponds to flipping the Domain. Remember: the general for is $a*\log(x-h)+k$, \textbf{where $a$ does not affect the domain}.
\item \( (a, \infty), a \in [5, 11] \)

* $(7, \infty)$, which is the correct option.
\item \( [a, \infty), a \in [-9, -4] \)

$[-7, \infty)$, which corresponds to using the vertical shift when shifting the Domain AND including the endpoint.
\item \( (-\infty, \infty) \)

This corresponds to thinking of the range of the log function (or the domain of the exponential function).
\end{enumerate}

\textbf{General Comment:} \textbf{General Comments}: The domain of a basic logarithmic function is $(0, \infty)$ and the Range is $(-\infty, \infty)$. We can use shifts when finding the Domain, but the Range will always be all Real numbers.
}
\litem{
Which of the following intervals describes the Range of the function below?
\[ f(x) = e^{x-8}-5 \]The solution is \( (-5, \infty) \), which is option C.\begin{enumerate}[label=\Alph*.]
\item \( (-\infty, a), a \in [2, 6] \)

$(-\infty, 5)$, which corresponds to using the negative vertical shift AND flipping the Range interval.
\item \( [a, \infty), a \in [-12, 0] \)

$[-5, \infty)$, which corresponds to including the endpoint.
\item \( (a, \infty), a \in [-12, 0] \)

* $(-5, \infty)$, which is the correct option.
\item \( (-\infty, a], a \in [2, 6] \)

$(-\infty, 5]$, which corresponds to using the negative vertical shift AND flipping the Range interval AND including the endpoint.
\item \( (-\infty, \infty) \)

This corresponds to confusing range of an exponential function with the domain of an exponential function.
\end{enumerate}

\textbf{General Comment:} \textbf{General Comments}: Domain of a basic exponential function is $(-\infty, \infty)$ while the Range is $(0, \infty)$. We can shift these intervals [and even flip when $a<0$!] to find the new Domain/Range.
}
\litem{
Solve the equation for $x$ and choose the interval that contains the solution (if it exists).
\[ \log_{2}{(-4x+5)}+6 = 3 \]The solution is \( x = 1.219 \), which is option C.\begin{enumerate}[label=\Alph*.]
\item \( x \in [-4.07, -3.47] \)

$x = -3.500$, which corresponds to reversing the base and exponent when converting and reversing the value with $x$.
\item \( x \in [-0.77, -0.13] \)

$x = -0.750$, which corresponds to ignoring the vertical shift when converting to exponential form.
\item \( x \in [0.46, 1.79] \)

* $x = 1.219$, which is the correct option.
\item \( x \in [-1.26, -0.96] \)

$x = -1.000$, which corresponds to reversing the base and exponent when converting.
\item \( \text{There is no Real solution to the equation.} \)

Corresponds to believing a negative coefficient within the log equation means there is no Real solution.
\end{enumerate}

\textbf{General Comment:} \textbf{General Comments:} First, get the equation in the form $\log_b{(cx+d)} = a$. Then, convert to $b^a = cx+d$ and solve.
}
\litem{
Which of the following intervals describes the Domain of the function below?
\[ f(x) = e^{x+1}+4 \]The solution is \( (-\infty, \infty) \), which is option E.\begin{enumerate}[label=\Alph*.]
\item \( (-\infty, a], a \in [4, 8] \)

$(-\infty, 4]$, which corresponds to using the correct vertical shift *if we wanted the Range* AND including the endpoint.
\item \( (-\infty, a), a \in [4, 8] \)

$(-\infty, 4)$, which corresponds to using the correct vertical shift *if we wanted the Range*.
\item \( (a, \infty), a \in [-4, 1] \)

$(-4, \infty)$, which corresponds to using the negative vertical shift AND flipping the Range interval.
\item \( [a, \infty), a \in [-4, 1] \)

$[-4, \infty)$, which corresponds to using the negative vertical shift AND flipping the Range interval AND including the endpoint.
\item \( (-\infty, \infty) \)

* This is the correct option.
\end{enumerate}

\textbf{General Comment:} \textbf{General Comments}: Domain of a basic exponential function is $(-\infty, \infty)$ while the Range is $(0, \infty)$. We can shift these intervals [and even flip when $a<0$!] to find the new Domain/Range.
}
\litem{
 Solve the equation for $x$ and choose the interval that contains $x$ (if it exists).
\[  15 = \sqrt[7]{\frac{12}{e^{6x}}} \]The solution is \( x = -2.745 \), which is option C.\begin{enumerate}[label=\Alph*.]
\item \( x \in [-1, -0.4] \)

$x = -0.489$, which corresponds to treating any root as a square root.
\item \( x \in [-19.6, -17.4] \)

$x = -17.914$, which corresponds to thinking you don't need to take the natural log of both sides before reducing, as if the equation already had a natural log on the right side.
\item \( x \in [-4.5, -2.3] \)

* $x = -2.745$, which is the correct option.
\item \( \text{There is no Real solution to the equation.} \)

This corresponds to believing you cannot solve the equation.
\item \( \text{None of the above.} \)

This corresponds to making an unexpected error.
\end{enumerate}

\textbf{General Comment:} \textbf{General Comments}: After using the properties of logarithmic functions to break up the right-hand side, use $\ln(e) = 1$ to reduce the question to a linear function to solve. You can put $\ln(12)$ into a calculator if you are having trouble.
}
\litem{
 Solve the equation for $x$ and choose the interval that contains $x$ (if it exists).
\[  22 = \ln{\sqrt[5]{\frac{28}{e^{3x}}}} \]The solution is \( x = -35.556 \), which is option A.\begin{enumerate}[label=\Alph*.]
\item \( x \in [-38.56, -30.56] \)

* $x = -35.556$, which is the correct option.
\item \( x \in [-13.56, -10.56] \)

$x = -13.556$, which corresponds to treating any root as a square root.
\item \( x \in [-8.26, -4.26] \)

$x = -6.262$, which corresponds to thinking you need to take the natural log of on the left before reducing.
\item \( \text{There is no Real solution to the equation.} \)

This corresponds to believing you cannot solve the equation.
\item \( \text{None of the above.} \)

This corresponds to making an unexpected error.
\end{enumerate}

\textbf{General Comment:} \textbf{General Comments}: After using the properties of logarithmic functions to break up the right-hand side, use $\ln(e) = 1$ to reduce the question to a linear function to solve. You can put $\ln(28)$ into a calculator if you are having trouble.
}
\end{enumerate}

\end{document}