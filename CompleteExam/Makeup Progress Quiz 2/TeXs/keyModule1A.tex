\documentclass{extbook}[14pt]
\usepackage{multicol, enumerate, enumitem, hyperref, color, soul, setspace, parskip, fancyhdr, amssymb, amsthm, amsmath, latexsym, units, mathtools}
\everymath{\displaystyle}
\usepackage[headsep=0.5cm,headheight=0cm, left=1 in,right= 1 in,top= 1 in,bottom= 1 in]{geometry}
\usepackage{dashrule}  % Package to use the command below to create lines between items
\newcommand{\litem}[1]{\item #1

\rule{\textwidth}{0.4pt}}
\pagestyle{fancy}
\lhead{}
\chead{Answer Key for Makeup Progress Quiz 2 Version A}
\rhead{}
\lfoot{2790-1423}
\cfoot{}
\rfoot{Summer C 2021}
\begin{document}
\textbf{This key should allow you to understand why you choose the option you did (beyond just getting a question right or wrong). \href{https://xronos.clas.ufl.edu/mac1105spring2020/courseDescriptionAndMisc/Exams/LearningFromResults}{More instructions on how to use this key can be found here}.}

\textbf{If you have a suggestion to make the keys better, \href{https://forms.gle/CZkbZmPbC9XALEE88}{please fill out the short survey here}.}

\textit{Note: This key is auto-generated and may contain issues and/or errors. The keys are reviewed after each exam to ensure grading is done accurately. If there are issues (like duplicate options), they are noted in the offline gradebook. The keys are a work-in-progress to give students as many resources to improve as possible.}

\rule{\textwidth}{0.4pt}

\begin{enumerate}\litem{
Simplify the expression below into the form $a+bi$. Then, choose the intervals that $a$ and $b$ belong to.
\[ (-7 + 4 i)(10 + 3 i) \]The solution is \( -82 + 19 i \), which is option A.\begin{enumerate}[label=\Alph*.]
\item \( a \in [-86, -81] \text{ and } b \in [19, 20] \)

* $-82 + 19 i$, which is the correct option.
\item \( a \in [-61, -57] \text{ and } b \in [-65, -58] \)

 $-58 - 61 i$, which corresponds to adding a minus sign in the first term.
\item \( a \in [-86, -81] \text{ and } b \in [-19, -15] \)

 $-82 - 19 i$, which corresponds to adding a minus sign in both terms.
\item \( a \in [-71, -69] \text{ and } b \in [11, 16] \)

 $-70 + 12 i$, which corresponds to just multiplying the real terms to get the real part of the solution and the coefficients in the complex terms to get the complex part.
\item \( a \in [-61, -57] \text{ and } b \in [56, 65] \)

 $-58 + 61 i$, which corresponds to adding a minus sign in the second term.
\end{enumerate}

\textbf{General Comment:} You can treat $i$ as a variable and distribute. Just remember that $i^2=-1$, so you can continue to reduce after you distribute.
}
\litem{
Simplify the expression below into the form $a+bi$. Then, choose the intervals that $a$ and $b$ belong to.
\[ \frac{-45 + 66 i}{-3 - 8 i} \]The solution is \( -5.38  - 7.64 i \), which is option A.\begin{enumerate}[label=\Alph*.]
\item \( a \in [-6, -4.5] \text{ and } b \in [-7.9, -7.2] \)

* $-5.38  - 7.64 i$, which is the correct option.
\item \( a \in [-394, -390.5] \text{ and } b \in [-7.9, -7.2] \)

 $-393.00  - 7.64 i$, which corresponds to forgetting to multiply the conjugate by the numerator and using a plus instead of a minus in the denominator.
\item \( a \in [8, 9.5] \text{ and } b \in [2, 2.55] \)

 $9.08  + 2.22 i$, which corresponds to forgetting to multiply the conjugate by the numerator and not computing the conjugate correctly.
\item \( a \in [13.5, 17] \text{ and } b \in [-8.6, -7.9] \)

 $15.00  - 8.25 i$, which corresponds to just dividing the first term by the first term and the second by the second.
\item \( a \in [-6, -4.5] \text{ and } b \in [-558.05, -557.25] \)

 $-5.38  - 558.00 i$, which corresponds to forgetting to multiply the conjugate by the numerator.
\end{enumerate}

\textbf{General Comment:} Multiply the numerator and denominator by the *conjugate* of the denominator, then simplify. For example, if we have $2+3i$, the conjugate is $2-3i$.
}
\litem{
Choose the \textbf{smallest} set of Real numbers that the number below belongs to.
\[ -\sqrt{\frac{25921}{529}} \]The solution is \( \text{Integer} \), which is option E.\begin{enumerate}[label=\Alph*.]
\item \( \text{Whole} \)

These are the counting numbers with 0 (0, 1, 2, 3, ...)
\item \( \text{Not a Real number} \)

These are Nonreal Complex numbers \textbf{OR} things that are not numbers (e.g., dividing by 0).
\item \( \text{Rational} \)

These are numbers that can be written as fraction of Integers (e.g., -2/3)
\item \( \text{Irrational} \)

These cannot be written as a fraction of Integers.
\item \( \text{Integer} \)

* This is the correct option!
\end{enumerate}

\textbf{General Comment:} First, you \textbf{NEED} to simplify the expression. This question simplifies to $-161$. 
 
 Be sure you look at the simplified fraction and not just the decimal expansion. Numbers such as 13, 17, and 19 provide \textbf{long but repeating/terminating decimal expansions!} 
 
 The only ways to *not* be a Real number are: dividing by 0 or taking the square root of a negative number. 
 
 Irrational numbers are more than just square root of 3: adding or subtracting values from square root of 3 is also irrational.
}
\litem{
Simplify the expression below and choose the interval the simplification is contained within.
\[ 9 - 1 \div 8 * 2 - (4 * 6) \]The solution is \( -15.250 \), which is option A.\begin{enumerate}[label=\Alph*.]
\item \( [-15.72, -15.08] \)

* -15.250, which is the correct option.
\item \( [28.19, 28.67] \)

 28.500, which corresponds to not distributing a negative correctly.
\item \( [32.92, 32.95] \)

 32.938, which corresponds to not distributing addition and subtraction correctly.
\item \( [-15.17, -14.51] \)

 -15.062, which corresponds to an Order of Operations error: not reading left-to-right for multiplication/division.
\item \( \text{None of the above} \)

 You may have gotten this by making an unanticipated error. If you got a value that is not any of the others, please let the coordinator know so they can help you figure out what happened.
\end{enumerate}

\textbf{General Comment:} While you may remember (or were taught) PEMDAS is done in order, it is actually done as P/E/MD/AS. When we are at MD or AS, we read left to right.
}
\litem{
Choose the \textbf{smallest} set of Real numbers that the number below belongs to.
\[ -\sqrt{\frac{50625}{625}} \]The solution is \( \text{Integer} \), which is option C.\begin{enumerate}[label=\Alph*.]
\item \( \text{Rational} \)

These are numbers that can be written as fraction of Integers (e.g., -2/3)
\item \( \text{Whole} \)

These are the counting numbers with 0 (0, 1, 2, 3, ...)
\item \( \text{Integer} \)

* This is the correct option!
\item \( \text{Not a Real number} \)

These are Nonreal Complex numbers \textbf{OR} things that are not numbers (e.g., dividing by 0).
\item \( \text{Irrational} \)

These cannot be written as a fraction of Integers.
\end{enumerate}

\textbf{General Comment:} First, you \textbf{NEED} to simplify the expression. This question simplifies to $-225$. 
 
 Be sure you look at the simplified fraction and not just the decimal expansion. Numbers such as 13, 17, and 19 provide \textbf{long but repeating/terminating decimal expansions!} 
 
 The only ways to *not* be a Real number are: dividing by 0 or taking the square root of a negative number. 
 
 Irrational numbers are more than just square root of 3: adding or subtracting values from square root of 3 is also irrational.
}
\litem{
Simplify the expression below and choose the interval the simplification is contained within.
\[ 17 - 13^2 + 9 \div 8 * 20 \div 15 \]The solution is \( -150.500 \), which is option A.\begin{enumerate}[label=\Alph*.]
\item \( [-151.9, -148.6] \)

* -150.500, this is the correct option
\item \( [-153.6, -151.4] \)

 -151.996, which corresponds to an Order of Operations error: not reading left-to-right for multiplication/division.
\item \( [185.5, 187.3] \)

 186.004, which corresponds to two Order of Operations errors.
\item \( [186.3, 188.4] \)

 187.500, which corresponds to an Order of Operations error: multiplying by negative before squaring. For example: $(-3)^2 \neq -3^2$
\item \( \text{None of the above} \)

 You may have gotten this by making an unanticipated error. If you got a value that is not any of the others, please let the coordinator know so they can help you figure out what happened.
\end{enumerate}

\textbf{General Comment:} While you may remember (or were taught) PEMDAS is done in order, it is actually done as P/E/MD/AS. When we are at MD or AS, we read left to right.
}
\litem{
Simplify the expression below into the form $a+bi$. Then, choose the intervals that $a$ and $b$ belong to.
\[ (-2 - 6 i)(-10 + 5 i) \]The solution is \( 50 + 50 i \), which is option A.\begin{enumerate}[label=\Alph*.]
\item \( a \in [50, 52] \text{ and } b \in [50, 51] \)

* $50 + 50 i$, which is the correct option.
\item \( a \in [-15, -6] \text{ and } b \in [-72, -63] \)

 $-10 - 70 i$, which corresponds to adding a minus sign in the first term.
\item \( a \in [-15, -6] \text{ and } b \in [67, 73] \)

 $-10 + 70 i$, which corresponds to adding a minus sign in the second term.
\item \( a \in [19, 21] \text{ and } b \in [-37, -23] \)

 $20 - 30 i$, which corresponds to just multiplying the real terms to get the real part of the solution and the coefficients in the complex terms to get the complex part.
\item \( a \in [50, 52] \text{ and } b \in [-51, -49] \)

 $50 - 50 i$, which corresponds to adding a minus sign in both terms.
\end{enumerate}

\textbf{General Comment:} You can treat $i$ as a variable and distribute. Just remember that $i^2=-1$, so you can continue to reduce after you distribute.
}
\litem{
Choose the \textbf{smallest} set of Complex numbers that the number below belongs to.
\[ \sqrt{\frac{0}{7}}+\sqrt{3}i \]The solution is \( \text{Pure Imaginary} \), which is option A.\begin{enumerate}[label=\Alph*.]
\item \( \text{Pure Imaginary} \)

* This is the correct option!
\item \( \text{Not a Complex Number} \)

This is not a number. The only non-Complex number we know is dividing by 0 as this is not a number!
\item \( \text{Nonreal Complex} \)

This is a Complex number $(a+bi)$ that is not Real (has $i$ as part of the number).
\item \( \text{Rational} \)

These are numbers that can be written as fraction of Integers (e.g., -2/3 + 5)
\item \( \text{Irrational} \)

These cannot be written as a fraction of Integers. Remember: $\pi$ is not an Integer!
\end{enumerate}

\textbf{General Comment:} Be sure to simplify $i^2 = -1$. This may remove the imaginary portion for your number. If you are having trouble, you may want to look at the \textit{Subgroups of the Real Numbers} section.
}
\litem{
Simplify the expression below into the form $a+bi$. Then, choose the intervals that $a$ and $b$ belong to.
\[ \frac{-45 + 88 i}{4 - 6 i} \]The solution is \( -13.62  + 1.58 i \), which is option E.\begin{enumerate}[label=\Alph*.]
\item \( a \in [-14, -13] \text{ and } b \in [81, 82.5] \)

 $-13.62  + 82.00 i$, which corresponds to forgetting to multiply the conjugate by the numerator.
\item \( a \in [5, 7] \text{ and } b \in [11.5, 13] \)

 $6.69  + 11.96 i$, which corresponds to forgetting to multiply the conjugate by the numerator and not computing the conjugate correctly.
\item \( a \in [-708.5, -707.5] \text{ and } b \in [0.5, 2.5] \)

 $-708.00  + 1.58 i$, which corresponds to forgetting to multiply the conjugate by the numerator and using a plus instead of a minus in the denominator.
\item \( a \in [-11.5, -9.5] \text{ and } b \in [-15.5, -14] \)

 $-11.25  - 14.67 i$, which corresponds to just dividing the first term by the first term and the second by the second.
\item \( a \in [-14, -13] \text{ and } b \in [0.5, 2.5] \)

* $-13.62  + 1.58 i$, which is the correct option.
\end{enumerate}

\textbf{General Comment:} Multiply the numerator and denominator by the *conjugate* of the denominator, then simplify. For example, if we have $2+3i$, the conjugate is $2-3i$.
}
\litem{
Choose the \textbf{smallest} set of Complex numbers that the number below belongs to.
\[ \sqrt{\frac{169}{0}}+\sqrt{221} i \]The solution is \( \text{Not a Complex Number} \), which is option C.\begin{enumerate}[label=\Alph*.]
\item \( \text{Rational} \)

These are numbers that can be written as fraction of Integers (e.g., -2/3 + 5)
\item \( \text{Irrational} \)

These cannot be written as a fraction of Integers. Remember: $\pi$ is not an Integer!
\item \( \text{Not a Complex Number} \)

* This is the correct option!
\item \( \text{Pure Imaginary} \)

This is a Complex number $(a+bi)$ that \textbf{only} has an imaginary part like $2i$.
\item \( \text{Nonreal Complex} \)

This is a Complex number $(a+bi)$ that is not Real (has $i$ as part of the number).
\end{enumerate}

\textbf{General Comment:} Be sure to simplify $i^2 = -1$. This may remove the imaginary portion for your number. If you are having trouble, you may want to look at the \textit{Subgroups of the Real Numbers} section.
}
\end{enumerate}

\end{document}