\documentclass{extbook}[14pt]
\usepackage{multicol, enumerate, enumitem, hyperref, color, soul, setspace, parskip, fancyhdr, amssymb, amsthm, amsmath, latexsym, units, mathtools}
\everymath{\displaystyle}
\usepackage[headsep=0.5cm,headheight=0cm, left=1 in,right= 1 in,top= 1 in,bottom= 1 in]{geometry}
\usepackage{dashrule}  % Package to use the command below to create lines between items
\newcommand{\litem}[1]{\item #1

\rule{\textwidth}{0.4pt}}
\pagestyle{fancy}
\lhead{}
\chead{Answer Key for Makeup Progress Quiz 2 Version A}
\rhead{}
\lfoot{5763-3522}
\cfoot{}
\rfoot{Spring 2021}
\begin{document}
\textbf{This key should allow you to understand why you choose the option you did (beyond just getting a question right or wrong). \href{https://xronos.clas.ufl.edu/mac1105spring2020/courseDescriptionAndMisc/Exams/LearningFromResults}{More instructions on how to use this key can be found here}.}

\textbf{If you have a suggestion to make the keys better, \href{https://forms.gle/CZkbZmPbC9XALEE88}{please fill out the short survey here}.}

\textit{Note: This key is auto-generated and may contain issues and/or errors. The keys are reviewed after each exam to ensure grading is done accurately. If there are issues (like duplicate options), they are noted in the offline gradebook. The keys are a work-in-progress to give students as many resources to improve as possible.}

\rule{\textwidth}{0.4pt}

\begin{enumerate}\litem{
Choose the \textbf{smallest} set of Complex numbers that the number below belongs to.
\[ \sqrt{\frac{196}{529}} + 64i^2 \]The solution is \( \text{Rational} \), which is option A.\begin{enumerate}[label=\Alph*.]
\item \( \text{Rational} \)

* This is the correct option!
\item \( \text{Pure Imaginary} \)

This is a Complex number $(a+bi)$ that \textbf{only} has an imaginary part like $2i$.
\item \( \text{Nonreal Complex} \)

This is a Complex number $(a+bi)$ that is not Real (has $i$ as part of the number).
\item \( \text{Irrational} \)

These cannot be written as a fraction of Integers. Remember: $\pi$ is not an Integer!
\item \( \text{Not a Complex Number} \)

This is not a number. The only non-Complex number we know is dividing by 0 as this is not a number!
\end{enumerate}

\textbf{General Comment:} Be sure to simplify $i^2 = -1$. This may remove the imaginary portion for your number. If you are having trouble, you may want to look at the \textit{Subgroups of the Real Numbers} section.
}
\litem{
Simplify the expression below into the form $a+bi$. Then, choose the intervals that $a$ and $b$ belong to.
\[ (8 - 7 i)(3 + 6 i) \]The solution is \( 66 + 27 i \), which is option A.\begin{enumerate}[label=\Alph*.]
\item \( a \in [64, 71] \text{ and } b \in [23, 28] \)

* $66 + 27 i$, which is the correct option.
\item \( a \in [-20, -15] \text{ and } b \in [-79, -67] \)

 $-18 - 69 i$, which corresponds to adding a minus sign in the second term.
\item \( a \in [24, 30] \text{ and } b \in [-46, -37] \)

 $24 - 42 i$, which corresponds to just multiplying the real terms to get the real part of the solution and the coefficients in the complex terms to get the complex part.
\item \( a \in [-20, -15] \text{ and } b \in [66, 73] \)

 $-18 + 69 i$, which corresponds to adding a minus sign in the first term.
\item \( a \in [64, 71] \text{ and } b \in [-31, -22] \)

 $66 - 27 i$, which corresponds to adding a minus sign in both terms.
\end{enumerate}

\textbf{General Comment:} You can treat $i$ as a variable and distribute. Just remember that $i^2=-1$, so you can continue to reduce after you distribute.
}
\litem{
Simplify the expression below and choose the interval the simplification is contained within.
\[ 1 - 7^2 + 14 \div 9 * 12 \div 16 \]The solution is \( -46.833 \), which is option B.\begin{enumerate}[label=\Alph*.]
\item \( [48, 50.4] \)

 50.008, which corresponds to two Order of Operations errors.
\item \( [-47.1, -43.1] \)

* -46.833, this is the correct option
\item \( [50.1, 51.7] \)

 51.167, which corresponds to an Order of Operations error: multiplying by negative before squaring. For example: $(-3)^2 \neq -3^2$
\item \( [-48.4, -47.8] \)

 -47.992, which corresponds to an Order of Operations error: not reading left-to-right for multiplication/division.
\item \( \text{None of the above} \)

 You may have gotten this by making an unanticipated error. If you got a value that is not any of the others, please let the coordinator know so they can help you figure out what happened.
\end{enumerate}

\textbf{General Comment:} While you may remember (or were taught) PEMDAS is done in order, it is actually done as P/E/MD/AS. When we are at MD or AS, we read left to right.
}
\litem{
Simplify the expression below into the form $a+bi$. Then, choose the intervals that $a$ and $b$ belong to.
\[ \frac{-36 - 88 i}{5 - i} \]The solution is \( -3.54  - 18.31 i \), which is option E.\begin{enumerate}[label=\Alph*.]
\item \( a \in [-11.5, -9] \text{ and } b \in [-16, -15] \)

 $-10.31  - 15.54 i$, which corresponds to forgetting to multiply the conjugate by the numerator and not computing the conjugate correctly.
\item \( a \in [-4.5, -2] \text{ and } b \in [-476.5, -475] \)

 $-3.54  - 476.00 i$, which corresponds to forgetting to multiply the conjugate by the numerator.
\item \( a \in [-8.5, -7] \text{ and } b \in [86.5, 89] \)

 $-7.20  + 88.00 i$, which corresponds to just dividing the first term by the first term and the second by the second.
\item \( a \in [-93, -91] \text{ and } b \in [-19, -17.5] \)

 $-92.00  - 18.31 i$, which corresponds to forgetting to multiply the conjugate by the numerator and using a plus instead of a minus in the denominator.
\item \( a \in [-4.5, -2] \text{ and } b \in [-19, -17.5] \)

* $-3.54  - 18.31 i$, which is the correct option.
\end{enumerate}

\textbf{General Comment:} Multiply the numerator and denominator by the *conjugate* of the denominator, then simplify. For example, if we have $2+3i$, the conjugate is $2-3i$.
}
\litem{
Choose the \textbf{smallest} set of Real numbers that the number below belongs to.
\[ -\sqrt{\frac{625}{529}} \]The solution is \( \text{Rational} \), which is option C.\begin{enumerate}[label=\Alph*.]
\item \( \text{Whole} \)

These are the counting numbers with 0 (0, 1, 2, 3, ...)
\item \( \text{Integer} \)

These are the negative and positive counting numbers (..., -3, -2, -1, 0, 1, 2, 3, ...)
\item \( \text{Rational} \)

* This is the correct option!
\item \( \text{Not a Real number} \)

These are Nonreal Complex numbers \textbf{OR} things that are not numbers (e.g., dividing by 0).
\item \( \text{Irrational} \)

These cannot be written as a fraction of Integers.
\end{enumerate}

\textbf{General Comment:} First, you \textbf{NEED} to simplify the expression. This question simplifies to $-\frac{25}{23}$. 
 
 Be sure you look at the simplified fraction and not just the decimal expansion. Numbers such as 13, 17, and 19 provide \textbf{long but repeating/terminating decimal expansions!} 
 
 The only ways to *not* be a Real number are: dividing by 0 or taking the square root of a negative number. 
 
 Irrational numbers are more than just square root of 3: adding or subtracting values from square root of 3 is also irrational.
}
\litem{
Simplify the expression below into the form $a+bi$. Then, choose the intervals that $a$ and $b$ belong to.
\[ \frac{-54 - 33 i}{2 + 7 i} \]The solution is \( -6.40  + 5.89 i \), which is option B.\begin{enumerate}[label=\Alph*.]
\item \( a \in [2, 2.5] \text{ and } b \in [-9, -8] \)

 $2.32  - 8.38 i$, which corresponds to forgetting to multiply the conjugate by the numerator and not computing the conjugate correctly.
\item \( a \in [-7.5, -6] \text{ and } b \in [5, 7] \)

* $-6.40  + 5.89 i$, which is the correct option.
\item \( a \in [-28.5, -26] \text{ and } b \in [-6, -4.5] \)

 $-27.00  - 4.71 i$, which corresponds to just dividing the first term by the first term and the second by the second.
\item \( a \in [-7.5, -6] \text{ and } b \in [311, 313] \)

 $-6.40  + 312.00 i$, which corresponds to forgetting to multiply the conjugate by the numerator.
\item \( a \in [-340, -338.5] \text{ and } b \in [5, 7] \)

 $-339.00  + 5.89 i$, which corresponds to forgetting to multiply the conjugate by the numerator and using a plus instead of a minus in the denominator.
\end{enumerate}

\textbf{General Comment:} Multiply the numerator and denominator by the *conjugate* of the denominator, then simplify. For example, if we have $2+3i$, the conjugate is $2-3i$.
}
\litem{
Simplify the expression below and choose the interval the simplification is contained within.
\[ 1 - 12^2 + 6 \div 15 * 4 \div 16 \]The solution is \( -142.900 \), which is option C.\begin{enumerate}[label=\Alph*.]
\item \( [-143.02, -142.97] \)

 -142.994, which corresponds to an Order of Operations error: not reading left-to-right for multiplication/division.
\item \( [145.07, 145.16] \)

 145.100, which corresponds to an Order of Operations error: multiplying by negative before squaring. For example: $(-3)^2 \neq -3^2$
\item \( [-142.92, -142.8] \)

* -142.900, this is the correct option
\item \( [144.96, 145.02] \)

 145.006, which corresponds to two Order of Operations errors.
\item \( \text{None of the above} \)

 You may have gotten this by making an unanticipated error. If you got a value that is not any of the others, please let the coordinator know so they can help you figure out what happened.
\end{enumerate}

\textbf{General Comment:} While you may remember (or were taught) PEMDAS is done in order, it is actually done as P/E/MD/AS. When we are at MD or AS, we read left to right.
}
\litem{
Choose the \textbf{smallest} set of Complex numbers that the number below belongs to.
\[ \frac{12}{-17}+16i^2 \]The solution is \( \text{Rational} \), which is option E.\begin{enumerate}[label=\Alph*.]
\item \( \text{Pure Imaginary} \)

This is a Complex number $(a+bi)$ that \textbf{only} has an imaginary part like $2i$.
\item \( \text{Not a Complex Number} \)

This is not a number. The only non-Complex number we know is dividing by 0 as this is not a number!
\item \( \text{Irrational} \)

These cannot be written as a fraction of Integers. Remember: $\pi$ is not an Integer!
\item \( \text{Nonreal Complex} \)

This is a Complex number $(a+bi)$ that is not Real (has $i$ as part of the number).
\item \( \text{Rational} \)

* This is the correct option!
\end{enumerate}

\textbf{General Comment:} Be sure to simplify $i^2 = -1$. This may remove the imaginary portion for your number. If you are having trouble, you may want to look at the \textit{Subgroups of the Real Numbers} section.
}
\litem{
Simplify the expression below into the form $a+bi$. Then, choose the intervals that $a$ and $b$ belong to.
\[ (-6 - 3 i)(2 + 10 i) \]The solution is \( 18 - 66 i \), which is option A.\begin{enumerate}[label=\Alph*.]
\item \( a \in [13, 19] \text{ and } b \in [-66, -64] \)

* $18 - 66 i$, which is the correct option.
\item \( a \in [-43, -41] \text{ and } b \in [50, 59] \)

 $-42 + 54 i$, which corresponds to adding a minus sign in the second term.
\item \( a \in [-43, -41] \text{ and } b \in [-57, -50] \)

 $-42 - 54 i$, which corresponds to adding a minus sign in the first term.
\item \( a \in [-18, -5] \text{ and } b \in [-34, -26] \)

 $-12 - 30 i$, which corresponds to just multiplying the real terms to get the real part of the solution and the coefficients in the complex terms to get the complex part.
\item \( a \in [13, 19] \text{ and } b \in [66, 69] \)

 $18 + 66 i$, which corresponds to adding a minus sign in both terms.
\end{enumerate}

\textbf{General Comment:} You can treat $i$ as a variable and distribute. Just remember that $i^2=-1$, so you can continue to reduce after you distribute.
}
\litem{
Choose the \textbf{smallest} set of Real numbers that the number below belongs to.
\[ \sqrt{\frac{-1568}{14}} \]The solution is \( \text{Not a Real number} \), which is option A.\begin{enumerate}[label=\Alph*.]
\item \( \text{Not a Real number} \)

* This is the correct option!
\item \( \text{Irrational} \)

These cannot be written as a fraction of Integers.
\item \( \text{Whole} \)

These are the counting numbers with 0 (0, 1, 2, 3, ...)
\item \( \text{Rational} \)

These are numbers that can be written as fraction of Integers (e.g., -2/3)
\item \( \text{Integer} \)

These are the negative and positive counting numbers (..., -3, -2, -1, 0, 1, 2, 3, ...)
\end{enumerate}

\textbf{General Comment:} First, you \textbf{NEED} to simplify the expression. This question simplifies to $\sqrt{112} i$. 
 
 Be sure you look at the simplified fraction and not just the decimal expansion. Numbers such as 13, 17, and 19 provide \textbf{long but repeating/terminating decimal expansions!} 
 
 The only ways to *not* be a Real number are: dividing by 0 or taking the square root of a negative number. 
 
 Irrational numbers are more than just square root of 3: adding or subtracting values from square root of 3 is also irrational.
}
\end{enumerate}

\end{document}