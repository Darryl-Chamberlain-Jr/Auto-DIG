\documentclass{extbook}[14pt]
\usepackage{multicol, enumerate, enumitem, hyperref, color, soul, setspace, parskip, fancyhdr, amssymb, amsthm, amsmath, latexsym, units, mathtools}
\everymath{\displaystyle}
\usepackage[headsep=0.5cm,headheight=0cm, left=1 in,right= 1 in,top= 1 in,bottom= 1 in]{geometry}
\usepackage{dashrule}  % Package to use the command below to create lines between items
\newcommand{\litem}[1]{\item #1

\rule{\textwidth}{0.4pt}}
\pagestyle{fancy}
\lhead{}
\chead{Answer Key for Makeup Progress Quiz 2 Version A}
\rhead{}
\lfoot{2790-1423}
\cfoot{}
\rfoot{Summer C 2021}
\begin{document}
\textbf{This key should allow you to understand why you choose the option you did (beyond just getting a question right or wrong). \href{https://xronos.clas.ufl.edu/mac1105spring2020/courseDescriptionAndMisc/Exams/LearningFromResults}{More instructions on how to use this key can be found here}.}

\textbf{If you have a suggestion to make the keys better, \href{https://forms.gle/CZkbZmPbC9XALEE88}{please fill out the short survey here}.}

\textit{Note: This key is auto-generated and may contain issues and/or errors. The keys are reviewed after each exam to ensure grading is done accurately. If there are issues (like duplicate options), they are noted in the offline gradebook. The keys are a work-in-progress to give students as many resources to improve as possible.}

\rule{\textwidth}{0.4pt}

\begin{enumerate}\litem{
What are the \textit{possible Rational} roots of the polynomial below?
\[ f(x) = 7x^{4} +2 x^{3} +2 x^{2} +2 x + 6 \]The solution is \( \text{ All combinations of: }\frac{\pm 1,\pm 2,\pm 3,\pm 6}{\pm 1,\pm 7} \), which is option C.\begin{enumerate}[label=\Alph*.]
\item \( \pm 1,\pm 7 \)

 Distractor 1: Corresponds to the plus or minus factors of a1 only.
\item \( \text{ All combinations of: }\frac{\pm 1,\pm 7}{\pm 1,\pm 2,\pm 3,\pm 6} \)

 Distractor 3: Corresponds to the plus or minus of the inverse quotient (an/a0) of the factors. 
\item \( \text{ All combinations of: }\frac{\pm 1,\pm 2,\pm 3,\pm 6}{\pm 1,\pm 7} \)

* This is the solution \textbf{since we asked for the possible Rational roots}!
\item \( \pm 1,\pm 2,\pm 3,\pm 6 \)

This would have been the solution \textbf{if asked for the possible Integer roots}!
\item \( \text{ There is no formula or theorem that tells us all possible Rational roots.} \)

 Distractor 4: Corresponds to not recalling the theorem for rational roots of a polynomial.
\end{enumerate}

\textbf{General Comment:} We have a way to find the possible Rational roots. The possible Integer roots are the Integers in this list.
}
\litem{
What are the \textit{possible Integer} roots of the polynomial below?
\[ f(x) = 6x^{4} +6 x^{3} +3 x^{2} +2 x + 4 \]The solution is \( \pm 1,\pm 2,\pm 4 \), which is option A.\begin{enumerate}[label=\Alph*.]
\item \( \pm 1,\pm 2,\pm 4 \)

* This is the solution \textbf{since we asked for the possible Integer roots}!
\item \( \text{ All combinations of: }\frac{\pm 1,\pm 2,\pm 3,\pm 6}{\pm 1,\pm 2,\pm 4} \)

 Distractor 3: Corresponds to the plus or minus of the inverse quotient (an/a0) of the factors. 
\item \( \text{ All combinations of: }\frac{\pm 1,\pm 2,\pm 4}{\pm 1,\pm 2,\pm 3,\pm 6} \)

This would have been the solution \textbf{if asked for the possible Rational roots}!
\item \( \pm 1,\pm 2,\pm 3,\pm 6 \)

 Distractor 1: Corresponds to the plus or minus factors of a1 only.
\item \( \text{There is no formula or theorem that tells us all possible Integer roots.} \)

 Distractor 4: Corresponds to not recognizing Integers as a subset of Rationals.
\end{enumerate}

\textbf{General Comment:} We have a way to find the possible Rational roots. The possible Integer roots are the Integers in this list.
}
\litem{
Factor the polynomial below completely, knowing that $x + 5$ is a factor. Then, choose the intervals the zeros of the polynomial belong to, where $z_1 \leq z_2 \leq z_3 \leq z_4$. \textit{To make the problem easier, all zeros are between -5 and 5.}
\[ f(x) = 8x^{4} -14 x^{3} -167 x^{2} +455 x -300 \]The solution is \( [-5, 1.25, 1.5, 4] \), which is option B.\begin{enumerate}[label=\Alph*.]
\item \( z_1 \in [-4.25, -3.95], \text{   }  z_2 \in [-0.88, -0.53], z_3 \in [-0.67, -0.65], \text{   and   } z_4 \in [4.9, 5.1] \)

 Distractor 3: Corresponds to negatives of all zeros AND inversing rational roots.
\item \( z_1 \in [-5.08, -4.64], \text{   }  z_2 \in [0.79, 1.43], z_3 \in [1.47, 1.53], \text{   and   } z_4 \in [2.5, 4.4] \)

* This is the solution!
\item \( z_1 \in [-5.08, -4.64], \text{   }  z_2 \in [-0.31, 0.74], z_3 \in [0.76, 0.84], \text{   and   } z_4 \in [2.5, 4.4] \)

 Distractor 2: Corresponds to inversing rational roots.
\item \( z_1 \in [-4.25, -3.95], \text{   }  z_2 \in [-3.36, -2.87], z_3 \in [-0.63, -0.53], \text{   and   } z_4 \in [4.9, 5.1] \)

 Distractor 4: Corresponds to moving factors from one rational to another.
\item \( z_1 \in [-4.25, -3.95], \text{   }  z_2 \in [-2.41, -0.84], z_3 \in [-1.26, -1.22], \text{   and   } z_4 \in [4.9, 5.1] \)

 Distractor 1: Corresponds to negatives of all zeros.
\end{enumerate}

\textbf{General Comment:} Remember to try the middle-most integers first as these normally are the zeros. Also, once you get it to a quadratic, you can use your other factoring techniques to finish factoring.
}
\litem{
Perform the division below. Then, find the intervals that correspond to the quotient in the form $ax^2+bx+c$ and remainder $r$.
\[ \frac{4x^{3} -12 x + 6}{x + 2} \]The solution is \( 4x^{2} -8 x + 4 + \frac{-2}{x + 2} \), which is option D.\begin{enumerate}[label=\Alph*.]
\item \( a \in [3, 8], b \in [-13, -10], c \in [15, 25], \text{ and } r \in [-67, -61]. \)

 You multipled by the synthetic number and subtracted rather than adding during synthetic division.
\item \( a \in [3, 8], b \in [8, 10], c \in [-1, 8], \text{ and } r \in [8, 15]. \)

 You divided by the opposite of the factor.
\item \( a \in [-10, -4], b \in [10, 17], c \in [-48, -42], \text{ and } r \in [94, 97]. \)

 You multipled by the synthetic number rather than bringing the first factor down.
\item \( a \in [3, 8], b \in [-9, 0], c \in [-1, 8], \text{ and } r \in [-5, 4]. \)

* This is the solution!
\item \( a \in [-10, -4], b \in [-20, -15], c \in [-48, -42], \text{ and } r \in [-85, -81]. \)

 You divided by the opposite of the factor AND multipled the first factor rather than just bringing it down.
\end{enumerate}

\textbf{General Comment:} Be sure to synthetically divide by the zero of the denominator! Also, make sure to include 0 placeholders for missing terms.
}
\litem{
Perform the division below. Then, find the intervals that correspond to the quotient in the form $ax^2+bx+c$ and remainder $r$.
\[ \frac{20x^{3} -63 x^{2} + 23}{x -3} \]The solution is \( 20x^{2} -3 x -9 + \frac{-4}{x -3} \), which is option D.\begin{enumerate}[label=\Alph*.]
\item \( a \in [57, 65], b \in [113, 120], c \in [350, 355], \text{ and } r \in [1074, 1078]. \)

 You multipled by the synthetic number rather than bringing the first factor down.
\item \( a \in [17, 22], b \in [-130, -118], c \in [369, 371], \text{ and } r \in [-1085, -1082]. \)

 You divided by the opposite of the factor.
\item \( a \in [57, 65], b \in [-245, -241], c \in [729, 731], \text{ and } r \in [-2169, -2161]. \)

 You divided by the opposite of the factor AND multipled the first factor rather than just bringing it down.
\item \( a \in [17, 22], b \in [-5, 0], c \in [-13, -7], \text{ and } r \in [-6, 4]. \)

* This is the solution!
\item \( a \in [17, 22], b \in [-29, -22], c \in [-47, -42], \text{ and } r \in [-70, -68]. \)

 You multipled by the synthetic number and subtracted rather than adding during synthetic division.
\end{enumerate}

\textbf{General Comment:} Be sure to synthetically divide by the zero of the denominator! Also, make sure to include 0 placeholders for missing terms.
}
\litem{
Factor the polynomial below completely, knowing that $x -5$ is a factor. Then, choose the intervals the zeros of the polynomial belong to, where $z_1 \leq z_2 \leq z_3 \leq z_4$. \textit{To make the problem easier, all zeros are between -5 and 5.}
\[ f(x) = 12x^{4} -113 x^{3} +338 x^{2} -395 x + 150 \]The solution is \( [0.75, 1.667, 2, 5] \), which is option C.\begin{enumerate}[label=\Alph*.]
\item \( z_1 \in [-5.35, -4.91], \text{   }  z_2 \in [-6.33, -4.95], z_3 \in [-2.42, -1.69], \text{   and   } z_4 \in [-0.34, -0.14] \)

 Distractor 4: Corresponds to moving factors from one rational to another.
\item \( z_1 \in [-5.35, -4.91], \text{   }  z_2 \in [-2.13, -1.6], z_3 \in [-1.8, -1.62], \text{   and   } z_4 \in [-0.84, -0.74] \)

 Distractor 1: Corresponds to negatives of all zeros.
\item \( z_1 \in [0.72, 1.07], \text{   }  z_2 \in [1.35, 1.95], z_3 \in [1.61, 2.61], \text{   and   } z_4 \in [4.79, 5.08] \)

* This is the solution!
\item \( z_1 \in [0.59, 0.69], \text{   }  z_2 \in [1.11, 1.62], z_3 \in [1.61, 2.61], \text{   and   } z_4 \in [4.79, 5.08] \)

 Distractor 2: Corresponds to inversing rational roots.
\item \( z_1 \in [-5.35, -4.91], \text{   }  z_2 \in [-2.13, -1.6], z_3 \in [-1.44, -1.27], \text{   and   } z_4 \in [-0.68, -0.44] \)

 Distractor 3: Corresponds to negatives of all zeros AND inversing rational roots.
\end{enumerate}

\textbf{General Comment:} Remember to try the middle-most integers first as these normally are the zeros. Also, once you get it to a quadratic, you can use your other factoring techniques to finish factoring.
}
\litem{
Factor the polynomial below completely. Then, choose the intervals the zeros of the polynomial belong to, where $z_1 \leq z_2 \leq z_3$. \textit{To make the problem easier, all zeros are between -5 and 5.}
\[ f(x) = 25x^{3} -100 x^{2} -4 x + 16 \]The solution is \( [-0.4, 0.4, 4] \), which is option D.\begin{enumerate}[label=\Alph*.]
\item \( z_1 \in [-4.6, -3.3], \text{   }  z_2 \in [-2.65, -2.36], \text{   and   } z_3 \in [2.09, 3] \)

 Distractor 3: Corresponds to negatives of all zeros AND inversing rational roots.
\item \( z_1 \in [-2.9, -2.4], \text{   }  z_2 \in [1.93, 2.94], \text{   and   } z_3 \in [3.93, 4.24] \)

 Distractor 2: Corresponds to inversing rational roots.
\item \( z_1 \in [-4.6, -3.3], \text{   }  z_2 \in [-2.25, -1.76], \text{   and   } z_3 \in [-0.2, 0.15] \)

 Distractor 4: Corresponds to moving factors from one rational to another.
\item \( z_1 \in [-1.6, 0.4], \text{   }  z_2 \in [0.19, 0.79], \text{   and   } z_3 \in [3.93, 4.24] \)

* This is the solution!
\item \( z_1 \in [-4.6, -3.3], \text{   }  z_2 \in [-0.56, -0.04], \text{   and   } z_3 \in [0.11, 1.11] \)

 Distractor 1: Corresponds to negatives of all zeros.
\end{enumerate}

\textbf{General Comment:} Remember to try the middle-most integers first as these normally are the zeros. Also, once you get it to a quadratic, you can use your other factoring techniques to finish factoring.
}
\litem{
Factor the polynomial below completely. Then, choose the intervals the zeros of the polynomial belong to, where $z_1 \leq z_2 \leq z_3$. \textit{To make the problem easier, all zeros are between -5 and 5.}
\[ f(x) = 20x^{3} +31 x^{2} -38 x -40 \]The solution is \( [-2, -0.8, 1.25] \), which is option E.\begin{enumerate}[label=\Alph*.]
\item \( z_1 \in [-0.92, -0.65], \text{   }  z_2 \in [1.12, 1.47], \text{   and   } z_3 \in [1.78, 2.59] \)

 Distractor 3: Corresponds to negatives of all zeros AND inversing rational roots.
\item \( z_1 \in [-5.08, -4.9], \text{   }  z_2 \in [-0.09, 0.27], \text{   and   } z_3 \in [1.78, 2.59] \)

 Distractor 4: Corresponds to moving factors from one rational to another.
\item \( z_1 \in [-1.46, -1.02], \text{   }  z_2 \in [0.78, 1.08], \text{   and   } z_3 \in [1.78, 2.59] \)

 Distractor 1: Corresponds to negatives of all zeros.
\item \( z_1 \in [-2.2, -1.8], \text{   }  z_2 \in [-1.59, -1.18], \text{   and   } z_3 \in [0.41, 0.86] \)

 Distractor 2: Corresponds to inversing rational roots.
\item \( z_1 \in [-2.2, -1.8], \text{   }  z_2 \in [-0.86, -0.52], \text{   and   } z_3 \in [1.1, 1.49] \)

* This is the solution!
\end{enumerate}

\textbf{General Comment:} Remember to try the middle-most integers first as these normally are the zeros. Also, once you get it to a quadratic, you can use your other factoring techniques to finish factoring.
}
\litem{
Perform the division below. Then, find the intervals that correspond to the quotient in the form $ax^2+bx+c$ and remainder $r$.
\[ \frac{6x^{3} -2 x^{2} -20 x + 19}{x + 2} \]The solution is \( 6x^{2} -14 x + 8 + \frac{3}{x + 2} \), which is option E.\begin{enumerate}[label=\Alph*.]
\item \( a \in [-15, -8], \text{   } b \in [-28, -25], \text{   } c \in [-72, -68], \text{   and   } r \in [-132, -124]. \)

 You divided by the opposite of the factor AND multiplied the first factor rather than just bringing it down.
\item \( a \in [-15, -8], \text{   } b \in [22, 24], \text{   } c \in [-66, -63], \text{   and   } r \in [144, 149]. \)

 You multiplied by the synthetic number rather than bringing the first factor down.
\item \( a \in [1, 11], \text{   } b \in [-21, -19], \text{   } c \in [34, 46], \text{   and   } r \in [-103, -97]. \)

 You multiplied by the synthetic number and subtracted rather than adding during synthetic division.
\item \( a \in [1, 11], \text{   } b \in [8, 17], \text{   } c \in [-3, 4], \text{   and   } r \in [15, 20]. \)

 You divided by the opposite of the factor.
\item \( a \in [1, 11], \text{   } b \in [-14, -9], \text{   } c \in [7, 9], \text{   and   } r \in [2, 4]. \)

* This is the solution!
\end{enumerate}

\textbf{General Comment:} Be sure to synthetically divide by the zero of the denominator!
}
\litem{
Perform the division below. Then, find the intervals that correspond to the quotient in the form $ax^2+bx+c$ and remainder $r$.
\[ \frac{6x^{3} -20 x^{2} -2 x + 19}{x -3} \]The solution is \( 6x^{2} -2 x -8 + \frac{-5}{x -3} \), which is option D.\begin{enumerate}[label=\Alph*.]
\item \( a \in [5, 9], \text{   } b \in [-40, -34], \text{   } c \in [111, 115], \text{   and   } r \in [-322, -313]. \)

 You divided by the opposite of the factor.
\item \( a \in [15, 21], \text{   } b \in [-76, -72], \text{   } c \in [218, 224], \text{   and   } r \in [-646, -637]. \)

 You divided by the opposite of the factor AND multiplied the first factor rather than just bringing it down.
\item \( a \in [5, 9], \text{   } b \in [-14, -3], \text{   } c \in [-23, -16], \text{   and   } r \in [-18, -15]. \)

 You multiplied by the synthetic number and subtracted rather than adding during synthetic division.
\item \( a \in [5, 9], \text{   } b \in [-6, 6], \text{   } c \in [-15, -7], \text{   and   } r \in [-6, -3]. \)

* This is the solution!
\item \( a \in [15, 21], \text{   } b \in [29, 35], \text{   } c \in [95, 104], \text{   and   } r \in [318, 321]. \)

 You multiplied by the synthetic number rather than bringing the first factor down.
\end{enumerate}

\textbf{General Comment:} Be sure to synthetically divide by the zero of the denominator!
}
\end{enumerate}

\end{document}