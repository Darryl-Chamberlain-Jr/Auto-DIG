\documentclass{extbook}[14pt]
\usepackage{multicol, enumerate, enumitem, hyperref, color, soul, setspace, parskip, fancyhdr, amssymb, amsthm, amsmath, latexsym, units, mathtools}
\everymath{\displaystyle}
\usepackage[headsep=0.5cm,headheight=0cm, left=1 in,right= 1 in,top= 1 in,bottom= 1 in]{geometry}
\usepackage{dashrule}  % Package to use the command below to create lines between items
\newcommand{\litem}[1]{\item #1

\rule{\textwidth}{0.4pt}}
\pagestyle{fancy}
\lhead{}
\chead{Answer Key for Makeup Progress Quiz 2 Version A}
\rhead{}
\lfoot{5763-3522}
\cfoot{}
\rfoot{Spring 2021}
\begin{document}
\textbf{This key should allow you to understand why you choose the option you did (beyond just getting a question right or wrong). \href{https://xronos.clas.ufl.edu/mac1105spring2020/courseDescriptionAndMisc/Exams/LearningFromResults}{More instructions on how to use this key can be found here}.}

\textbf{If you have a suggestion to make the keys better, \href{https://forms.gle/CZkbZmPbC9XALEE88}{please fill out the short survey here}.}

\textit{Note: This key is auto-generated and may contain issues and/or errors. The keys are reviewed after each exam to ensure grading is done accurately. If there are issues (like duplicate options), they are noted in the offline gradebook. The keys are a work-in-progress to give students as many resources to improve as possible.}

\rule{\textwidth}{0.4pt}

\begin{enumerate}\litem{
Solve the linear inequality below. Then, choose the constant and interval combination that describes the solution set.
\[ -8x -3 < 3x + 6 \]The solution is \( (-0.818, \infty) \), which is option D.\begin{enumerate}[label=\Alph*.]
\item \( (-\infty, a), \text{ where } a \in [-1.9, 0.2] \)

 $(-\infty, -0.818)$, which corresponds to switching the direction of the interval. You likely did this if you did not flip the inequality when dividing by a negative!
\item \( (a, \infty), \text{ where } a \in [0.5, 1] \)

 $(0.818, \infty)$, which corresponds to negating the endpoint of the solution.
\item \( (-\infty, a), \text{ where } a \in [-0.1, 1.8] \)

 $(-\infty, 0.818)$, which corresponds to switching the direction of the interval AND negating the endpoint. You likely did this if you did not flip the inequality when dividing by a negative as well as not moving values over to a side properly.
\item \( (a, \infty), \text{ where } a \in [-1, -0.5] \)

* $(-0.818, \infty)$, which is the correct option.
\item \( \text{None of the above}. \)

You may have chosen this if you thought the inequality did not match the ends of the intervals.
\end{enumerate}

\textbf{General Comment:} Remember that less/greater than or equal to includes the endpoint, while less/greater do not. Also, remember that you need to flip the inequality when you multiply or divide by a negative.
}
\litem{
Using an interval or intervals, describe all the $x$-values within or including a distance of the given values.
\[ \text{ No less than } 10 \text{ units from the number } -9. \]The solution is \( (-\infty, -19] \cup [1, \infty) \), which is option D.\begin{enumerate}[label=\Alph*.]
\item \( [-19, 1] \)

This describes the values no more than 10 from -9
\item \( (-19, 1) \)

This describes the values less than 10 from -9
\item \( (-\infty, -19) \cup (1, \infty) \)

This describes the values more than 10 from -9
\item \( (-\infty, -19] \cup [1, \infty) \)

This describes the values no less than 10 from -9
\item \( \text{None of the above} \)

You likely thought the values in the interval were not correct.
\end{enumerate}

\textbf{General Comment:} When thinking about this language, it helps to draw a number line and try points.
}
\litem{
Solve the linear inequality below. Then, choose the constant and interval combination that describes the solution set.
\[ -8 + 9 x > 11 x \text{ or } -5 + 4 x < 7 x \]The solution is \( (-\infty, -4.0) \text{ or } (-1.667, \infty) \), which is option C.\begin{enumerate}[label=\Alph*.]
\item \( (-\infty, a) \cup (b, \infty), \text{ where } a \in [-2.33, 2.67] \text{ and } b \in [1, 9] \)

Corresponds to inverting the inequality and negating the solution.
\item \( (-\infty, a] \cup [b, \infty), \text{ where } a \in [-4, -3] \text{ and } b \in [-3.67, 3.33] \)

Corresponds to including the endpoints (when they should be excluded).
\item \( (-\infty, a) \cup (b, \infty), \text{ where } a \in [-5, -1] \text{ and } b \in [-4.67, 0.33] \)

 * Correct option.
\item \( (-\infty, a] \cup [b, \infty), \text{ where } a \in [-1.33, 7.67] \text{ and } b \in [4, 5] \)

Corresponds to including the endpoints AND negating.
\item \( (-\infty, \infty) \)

Corresponds to the variable canceling, which does not happen in this instance.
\end{enumerate}

\textbf{General Comment:} When multiplying or dividing by a negative, flip the sign.
}
\litem{
Solve the linear inequality below. Then, choose the constant and interval combination that describes the solution set.
\[ -7x + 10 > 7x + 7 \]The solution is \( (-\infty, 0.214) \), which is option B.\begin{enumerate}[label=\Alph*.]
\item \( (a, \infty), \text{ where } a \in [-0.12, 0.92] \)

 $(0.214, \infty)$, which corresponds to switching the direction of the interval. You likely did this if you did not flip the inequality when dividing by a negative!
\item \( (-\infty, a), \text{ where } a \in [-0.01, 0.24] \)

* $(-\infty, 0.214)$, which is the correct option.
\item \( (-\infty, a), \text{ where } a \in [-0.23, -0.09] \)

 $(-\infty, -0.214)$, which corresponds to negating the endpoint of the solution.
\item \( (a, \infty), \text{ where } a \in [-1.35, 0.19] \)

 $(-0.214, \infty)$, which corresponds to switching the direction of the interval AND negating the endpoint. You likely did this if you did not flip the inequality when dividing by a negative as well as not moving values over to a side properly.
\item \( \text{None of the above}. \)

You may have chosen this if you thought the inequality did not match the ends of the intervals.
\end{enumerate}

\textbf{General Comment:} Remember that less/greater than or equal to includes the endpoint, while less/greater do not. Also, remember that you need to flip the inequality when you multiply or divide by a negative.
}
\litem{
Using an interval or intervals, describe all the $x$-values within or including a distance of the given values.
\[ \text{ Less than } 4 \text{ units from the number } -6. \]The solution is \( (-10, -2) \), which is option B.\begin{enumerate}[label=\Alph*.]
\item \( [-10, -2] \)

This describes the values no more than 4 from -6
\item \( (-10, -2) \)

This describes the values less than 4 from -6
\item \( (-\infty, -10] \cup [-2, \infty) \)

This describes the values no less than 4 from -6
\item \( (-\infty, -10) \cup (-2, \infty) \)

This describes the values more than 4 from -6
\item \( \text{None of the above} \)

You likely thought the values in the interval were not correct.
\end{enumerate}

\textbf{General Comment:} When thinking about this language, it helps to draw a number line and try points.
}
\litem{
Solve the linear inequality below. Then, choose the constant and interval combination that describes the solution set.
\[ \frac{3}{9} - \frac{4}{3} x < \frac{4}{4} x - \frac{8}{2} \]The solution is \( (1.857, \infty) \), which is option C.\begin{enumerate}[label=\Alph*.]
\item \( (-\infty, a), \text{ where } a \in [-1.86, -0.86] \)

 $(-\infty, -1.857)$, which corresponds to switching the direction of the interval AND negating the endpoint. You likely did this if you did not flip the inequality when dividing by a negative as well as not moving values over to a side properly.
\item \( (a, \infty), \text{ where } a \in [-2.86, 1.14] \)

 $(-1.857, \infty)$, which corresponds to negating the endpoint of the solution.
\item \( (a, \infty), \text{ where } a \in [0.86, 4.86] \)

* $(1.857, \infty)$, which is the correct option.
\item \( (-\infty, a), \text{ where } a \in [0.86, 4.86] \)

 $(-\infty, 1.857)$, which corresponds to switching the direction of the interval. You likely did this if you did not flip the inequality when dividing by a negative!
\item \( \text{None of the above}. \)

You may have chosen this if you thought the inequality did not match the ends of the intervals.
\end{enumerate}

\textbf{General Comment:} Remember that less/greater than or equal to includes the endpoint, while less/greater do not. Also, remember that you need to flip the inequality when you multiply or divide by a negative.
}
\litem{
Solve the linear inequality below. Then, choose the constant and interval combination that describes the solution set.
\[ -6 + 9 x \leq \frac{59 x + 7}{6} < 3 + 9 x \]The solution is \( [-8.60, 2.20) \), which is option C.\begin{enumerate}[label=\Alph*.]
\item \( (a, b], \text{ where } a \in [-11.6, -7.6] \text{ and } b \in [0.2, 8.2] \)

$(-8.60, 2.20]$, which corresponds to flipping the inequality.
\item \( (-\infty, a) \cup [b, \infty), \text{ where } a \in [-10.6, -7.6] \text{ and } b \in [2.2, 3.2] \)

$(-\infty, -8.60) \cup [2.20, \infty)$, which corresponds to displaying the and-inequality as an or-inequality AND flipping the inequality.
\item \( [a, b), \text{ where } a \in [-9.6, -6.6] \text{ and } b \in [1.2, 3.2] \)

$[-8.60, 2.20)$, which is the correct option.
\item \( (-\infty, a] \cup (b, \infty), \text{ where } a \in [-11.6, -5.6] \text{ and } b \in [2.2, 4.2] \)

$(-\infty, -8.60] \cup (2.20, \infty)$, which corresponds to displaying the and-inequality as an or-inequality.
\item \( \text{None of the above.} \)


\end{enumerate}

\textbf{General Comment:} To solve, you will need to break up the compound inequality into two inequalities. Be sure to keep track of the inequality! It may be best to draw a number line and graph your solution.
}
\litem{
Solve the linear inequality below. Then, choose the constant and interval combination that describes the solution set.
\[ 9 - 6 x \leq \frac{15 x - 7}{3} < 6 + 4 x \]The solution is \( [1.03, 8.33) \), which is option A.\begin{enumerate}[label=\Alph*.]
\item \( [a, b), \text{ where } a \in [0.7, 2.3] \text{ and } b \in [4.33, 10.33] \)

$[1.03, 8.33)$, which is the correct option.
\item \( (a, b], \text{ where } a \in [0.03, 4.03] \text{ and } b \in [5.33, 16.33] \)

$(1.03, 8.33]$, which corresponds to flipping the inequality.
\item \( (-\infty, a] \cup (b, \infty), \text{ where } a \in [0.8, 3.9] \text{ and } b \in [8.33, 9.33] \)

$(-\infty, 1.03] \cup (8.33, \infty)$, which corresponds to displaying the and-inequality as an or-inequality.
\item \( (-\infty, a) \cup [b, \infty), \text{ where } a \in [1.03, 5.03] \text{ and } b \in [8.33, 13.33] \)

$(-\infty, 1.03) \cup [8.33, \infty)$, which corresponds to displaying the and-inequality as an or-inequality AND flipping the inequality.
\item \( \text{None of the above.} \)


\end{enumerate}

\textbf{General Comment:} To solve, you will need to break up the compound inequality into two inequalities. Be sure to keep track of the inequality! It may be best to draw a number line and graph your solution.
}
\litem{
Solve the linear inequality below. Then, choose the constant and interval combination that describes the solution set.
\[ -7 + 5 x > 8 x \text{ or } 4 + 8 x < 11 x \]The solution is \( (-\infty, -2.333) \text{ or } (1.333, \infty) \), which is option B.\begin{enumerate}[label=\Alph*.]
\item \( (-\infty, a] \cup [b, \infty), \text{ where } a \in [-1.5, 1.7] \text{ and } b \in [1.55, 3.25] \)

Corresponds to including the endpoints AND negating.
\item \( (-\infty, a) \cup (b, \infty), \text{ where } a \in [-4.2, -2] \text{ and } b \in [-0.37, 1.58] \)

 * Correct option.
\item \( (-\infty, a) \cup (b, \infty), \text{ where } a \in [-1.6, -0.8] \text{ and } b \in [1.68, 2.63] \)

Corresponds to inverting the inequality and negating the solution.
\item \( (-\infty, a] \cup [b, \infty), \text{ where } a \in [-2.7, -2.1] \text{ and } b \in [0.25, 1.64] \)

Corresponds to including the endpoints (when they should be excluded).
\item \( (-\infty, \infty) \)

Corresponds to the variable canceling, which does not happen in this instance.
\end{enumerate}

\textbf{General Comment:} When multiplying or dividing by a negative, flip the sign.
}
\litem{
Solve the linear inequality below. Then, choose the constant and interval combination that describes the solution set.
\[ \frac{-7}{5} + \frac{4}{7} x \leq \frac{10}{9} x - \frac{10}{3} \]The solution is \( [3.582, \infty) \), which is option C.\begin{enumerate}[label=\Alph*.]
\item \( (-\infty, a], \text{ where } a \in [-4.58, -2.58] \)

 $(-\infty, -3.582]$, which corresponds to switching the direction of the interval AND negating the endpoint. You likely did this if you did not flip the inequality when dividing by a negative as well as not moving values over to a side properly.
\item \( (-\infty, a], \text{ where } a \in [-0.42, 5.58] \)

 $(-\infty, 3.582]$, which corresponds to switching the direction of the interval. You likely did this if you did not flip the inequality when dividing by a negative!
\item \( [a, \infty), \text{ where } a \in [-0.42, 7.58] \)

* $[3.582, \infty)$, which is the correct option.
\item \( [a, \infty), \text{ where } a \in [-6.58, 1.42] \)

 $[-3.582, \infty)$, which corresponds to negating the endpoint of the solution.
\item \( \text{None of the above}. \)

You may have chosen this if you thought the inequality did not match the ends of the intervals.
\end{enumerate}

\textbf{General Comment:} Remember that less/greater than or equal to includes the endpoint, while less/greater do not. Also, remember that you need to flip the inequality when you multiply or divide by a negative.
}
\end{enumerate}

\end{document}