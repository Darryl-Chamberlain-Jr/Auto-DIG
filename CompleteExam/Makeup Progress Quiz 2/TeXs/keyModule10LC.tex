\documentclass{extbook}[14pt]
\usepackage{multicol, enumerate, enumitem, hyperref, color, soul, setspace, parskip, fancyhdr, amssymb, amsthm, amsmath, latexsym, units, mathtools}
\everymath{\displaystyle}
\usepackage[headsep=0.5cm,headheight=0cm, left=1 in,right= 1 in,top= 1 in,bottom= 1 in]{geometry}
\usepackage{dashrule}  % Package to use the command below to create lines between items
\newcommand{\litem}[1]{\item #1

\rule{\textwidth}{0.4pt}}
\pagestyle{fancy}
\lhead{}
\chead{Answer Key for Makeup Progress Quiz 2 Version C}
\rhead{}
\lfoot{2790-1423}
\cfoot{}
\rfoot{Summer C 2021}
\begin{document}
\textbf{This key should allow you to understand why you choose the option you did (beyond just getting a question right or wrong). \href{https://xronos.clas.ufl.edu/mac1105spring2020/courseDescriptionAndMisc/Exams/LearningFromResults}{More instructions on how to use this key can be found here}.}

\textbf{If you have a suggestion to make the keys better, \href{https://forms.gle/CZkbZmPbC9XALEE88}{please fill out the short survey here}.}

\textit{Note: This key is auto-generated and may contain issues and/or errors. The keys are reviewed after each exam to ensure grading is done accurately. If there are issues (like duplicate options), they are noted in the offline gradebook. The keys are a work-in-progress to give students as many resources to improve as possible.}

\rule{\textwidth}{0.4pt}

\begin{enumerate}\litem{
What are the \textit{possible Rational} roots of the polynomial below?
\[ f(x) = 5x^{2} +3 x + 6 \]The solution is \( \text{ All combinations of: }\frac{\pm 1,\pm 2,\pm 3,\pm 6}{\pm 1,\pm 5} \), which is option D.\begin{enumerate}[label=\Alph*.]
\item \( \text{ All combinations of: }\frac{\pm 1,\pm 5}{\pm 1,\pm 2,\pm 3,\pm 6} \)

 Distractor 3: Corresponds to the plus or minus of the inverse quotient (an/a0) of the factors. 
\item \( \pm 1,\pm 2,\pm 3,\pm 6 \)

This would have been the solution \textbf{if asked for the possible Integer roots}!
\item \( \pm 1,\pm 5 \)

 Distractor 1: Corresponds to the plus or minus factors of a1 only.
\item \( \text{ All combinations of: }\frac{\pm 1,\pm 2,\pm 3,\pm 6}{\pm 1,\pm 5} \)

* This is the solution \textbf{since we asked for the possible Rational roots}!
\item \( \text{ There is no formula or theorem that tells us all possible Rational roots.} \)

 Distractor 4: Corresponds to not recalling the theorem for rational roots of a polynomial.
\end{enumerate}

\textbf{General Comment:} We have a way to find the possible Rational roots. The possible Integer roots are the Integers in this list.
}
\litem{
What are the \textit{possible Integer} roots of the polynomial below?
\[ f(x) = 4x^{3} +5 x^{2} +7 x + 5 \]The solution is \( \pm 1,\pm 5 \), which is option B.\begin{enumerate}[label=\Alph*.]
\item \( \text{ All combinations of: }\frac{\pm 1,\pm 5}{\pm 1,\pm 2,\pm 4} \)

This would have been the solution \textbf{if asked for the possible Rational roots}!
\item \( \pm 1,\pm 5 \)

* This is the solution \textbf{since we asked for the possible Integer roots}!
\item \( \pm 1,\pm 2,\pm 4 \)

 Distractor 1: Corresponds to the plus or minus factors of a1 only.
\item \( \text{ All combinations of: }\frac{\pm 1,\pm 2,\pm 4}{\pm 1,\pm 5} \)

 Distractor 3: Corresponds to the plus or minus of the inverse quotient (an/a0) of the factors. 
\item \( \text{There is no formula or theorem that tells us all possible Integer roots.} \)

 Distractor 4: Corresponds to not recognizing Integers as a subset of Rationals.
\end{enumerate}

\textbf{General Comment:} We have a way to find the possible Rational roots. The possible Integer roots are the Integers in this list.
}
\litem{
Factor the polynomial below completely, knowing that $x -2$ is a factor. Then, choose the intervals the zeros of the polynomial belong to, where $z_1 \leq z_2 \leq z_3 \leq z_4$. \textit{To make the problem easier, all zeros are between -5 and 5.}
\[ f(x) = 15x^{4} -71 x^{3} +12 x^{2} +116 x + 48 \]The solution is \( [-0.667, -0.6, 2, 4] \), which is option D.\begin{enumerate}[label=\Alph*.]
\item \( z_1 \in [-5.2, -2.7], \text{   }  z_2 \in [-2.28, -1.89], z_3 \in [0.55, 0.73], \text{   and   } z_4 \in [-0.06, 1] \)

 Distractor 1: Corresponds to negatives of all zeros.
\item \( z_1 \in [-5.2, -2.7], \text{   }  z_2 \in [-2.28, -1.89], z_3 \in [1.36, 1.63], \text{   and   } z_4 \in [1.29, 2.3] \)

 Distractor 3: Corresponds to negatives of all zeros AND inversing rational roots.
\item \( z_1 \in [-5.2, -2.7], \text{   }  z_2 \in [-2.28, -1.89], z_3 \in [0.08, 0.3], \text{   and   } z_4 \in [2.98, 3.64] \)

 Distractor 4: Corresponds to moving factors from one rational to another.
\item \( z_1 \in [-0.8, -0.3], \text{   }  z_2 \in [-0.67, -0.27], z_3 \in [1.8, 2.48], \text{   and   } z_4 \in [3.99, 4.54] \)

* This is the solution!
\item \( z_1 \in [-2, -1], \text{   }  z_2 \in [-1.72, -1.24], z_3 \in [1.8, 2.48], \text{   and   } z_4 \in [3.99, 4.54] \)

 Distractor 2: Corresponds to inversing rational roots.
\end{enumerate}

\textbf{General Comment:} Remember to try the middle-most integers first as these normally are the zeros. Also, once you get it to a quadratic, you can use your other factoring techniques to finish factoring.
}
\litem{
Perform the division below. Then, find the intervals that correspond to the quotient in the form $ax^2+bx+c$ and remainder $r$.
\[ \frac{8x^{3} -24 x^{2} + 27}{x -2} \]The solution is \( 8x^{2} -8 x -16 + \frac{-5}{x -2} \), which is option C.\begin{enumerate}[label=\Alph*.]
\item \( a \in [14, 18], b \in [8, 9], c \in [16, 17], \text{ and } r \in [58, 60]. \)

 You multipled by the synthetic number rather than bringing the first factor down.
\item \( a \in [14, 18], b \in [-56, -55], c \in [109, 118], \text{ and } r \in [-197, -196]. \)

 You divided by the opposite of the factor AND multipled the first factor rather than just bringing it down.
\item \( a \in [5, 10], b \in [-11, -2], c \in [-16, -11], \text{ and } r \in [-5, -4]. \)

* This is the solution!
\item \( a \in [5, 10], b \in [-17, -12], c \in [-16, -11], \text{ and } r \in [7, 17]. \)

 You multipled by the synthetic number and subtracted rather than adding during synthetic division.
\item \( a \in [5, 10], b \in [-43, -39], c \in [74, 87], \text{ and } r \in [-135, -130]. \)

 You divided by the opposite of the factor.
\end{enumerate}

\textbf{General Comment:} Be sure to synthetically divide by the zero of the denominator! Also, make sure to include 0 placeholders for missing terms.
}
\litem{
Perform the division below. Then, find the intervals that correspond to the quotient in the form $ax^2+bx+c$ and remainder $r$.
\[ \frac{16x^{3} +84 x^{2} -97}{x + 5} \]The solution is \( 16x^{2} +4 x -20 + \frac{3}{x + 5} \), which is option B.\begin{enumerate}[label=\Alph*.]
\item \( a \in [16, 19], b \in [164, 167], c \in [820, 821], \text{ and } r \in [4001, 4004]. \)

 You divided by the opposite of the factor.
\item \( a \in [16, 19], b \in [1, 6], c \in [-20, -18], \text{ and } r \in [-1, 4]. \)

* This is the solution!
\item \( a \in [-82, -76], b \in [-320, -311], c \in [-1583, -1577], \text{ and } r \in [-7999, -7993]. \)

 You divided by the opposite of the factor AND multipled the first factor rather than just bringing it down.
\item \( a \in [-82, -76], b \in [482, 491], c \in [-2420, -2411], \text{ and } r \in [12002, 12004]. \)

 You multipled by the synthetic number rather than bringing the first factor down.
\item \( a \in [16, 19], b \in [-12, -9], c \in [69, 77], \text{ and } r \in [-535, -527]. \)

 You multipled by the synthetic number and subtracted rather than adding during synthetic division.
\end{enumerate}

\textbf{General Comment:} Be sure to synthetically divide by the zero of the denominator! Also, make sure to include 0 placeholders for missing terms.
}
\litem{
Factor the polynomial below completely, knowing that $x -5$ is a factor. Then, choose the intervals the zeros of the polynomial belong to, where $z_1 \leq z_2 \leq z_3 \leq z_4$. \textit{To make the problem easier, all zeros are between -5 and 5.}
\[ f(x) = 8x^{4} -30 x^{3} -87 x^{2} +155 x + 150 \]The solution is \( [-2.5, -0.75, 2, 5] \), which is option E.\begin{enumerate}[label=\Alph*.]
\item \( z_1 \in [-6.2, -4.4], \text{   }  z_2 \in [-2.15, -1.97], z_3 \in [0.56, 0.64], \text{   and   } z_4 \in [2.55, 3.08] \)

 Distractor 4: Corresponds to moving factors from one rational to another.
\item \( z_1 \in [-1.8, -1.1], \text{   }  z_2 \in [-0.43, -0.15], z_3 \in [1.84, 2.03], \text{   and   } z_4 \in [4.4, 5.57] \)

 Distractor 2: Corresponds to inversing rational roots.
\item \( z_1 \in [-6.2, -4.4], \text{   }  z_2 \in [-2.15, -1.97], z_3 \in [0.34, 0.55], \text{   and   } z_4 \in [0.91, 1.86] \)

 Distractor 3: Corresponds to negatives of all zeros AND inversing rational roots.
\item \( z_1 \in [-6.2, -4.4], \text{   }  z_2 \in [-2.15, -1.97], z_3 \in [0.67, 0.99], \text{   and   } z_4 \in [2.14, 2.86] \)

 Distractor 1: Corresponds to negatives of all zeros.
\item \( z_1 \in [-4, -2.2], \text{   }  z_2 \in [-0.79, -0.71], z_3 \in [1.84, 2.03], \text{   and   } z_4 \in [4.4, 5.57] \)

* This is the solution!
\end{enumerate}

\textbf{General Comment:} Remember to try the middle-most integers first as these normally are the zeros. Also, once you get it to a quadratic, you can use your other factoring techniques to finish factoring.
}
\litem{
Factor the polynomial below completely. Then, choose the intervals the zeros of the polynomial belong to, where $z_1 \leq z_2 \leq z_3$. \textit{To make the problem easier, all zeros are between -5 and 5.}
\[ f(x) = 20x^{3} -83 x^{2} -95 x + 50 \]The solution is \( [-1.25, 0.4, 5] \), which is option E.\begin{enumerate}[label=\Alph*.]
\item \( z_1 \in [-1.16, -0.26], \text{   }  z_2 \in [2.38, 3.36], \text{   and   } z_3 \in [4.32, 5.39] \)

 Distractor 2: Corresponds to inversing rational roots.
\item \( z_1 \in [-5.24, -4.84], \text{   }  z_2 \in [-2.8, -1.73], \text{   and   } z_3 \in [0.34, 0.82] \)

 Distractor 3: Corresponds to negatives of all zeros AND inversing rational roots.
\item \( z_1 \in [-5.24, -4.84], \text{   }  z_2 \in [-0.35, -0.02], \text{   and   } z_3 \in [4.32, 5.39] \)

 Distractor 4: Corresponds to moving factors from one rational to another.
\item \( z_1 \in [-5.24, -4.84], \text{   }  z_2 \in [-1.03, -0.32], \text{   and   } z_3 \in [1.09, 1.48] \)

 Distractor 1: Corresponds to negatives of all zeros.
\item \( z_1 \in [-1.45, -1.22], \text{   }  z_2 \in [-0.07, 0.58], \text{   and   } z_3 \in [4.32, 5.39] \)

* This is the solution!
\end{enumerate}

\textbf{General Comment:} Remember to try the middle-most integers first as these normally are the zeros. Also, once you get it to a quadratic, you can use your other factoring techniques to finish factoring.
}
\litem{
Factor the polynomial below completely. Then, choose the intervals the zeros of the polynomial belong to, where $z_1 \leq z_2 \leq z_3$. \textit{To make the problem easier, all zeros are between -5 and 5.}
\[ f(x) = 20x^{3} -77 x^{2} +89 x -30 \]The solution is \( [0.6, 1.25, 2] \), which is option B.\begin{enumerate}[label=\Alph*.]
\item \( z_1 \in [0.74, 0.84], \text{   }  z_2 \in [1.47, 1.68], \text{   and   } z_3 \in [1.89, 2.27] \)

 Distractor 2: Corresponds to inversing rational roots.
\item \( z_1 \in [0.49, 0.73], \text{   }  z_2 \in [1.14, 1.3], \text{   and   } z_3 \in [1.89, 2.27] \)

* This is the solution!
\item \( z_1 \in [-2.22, -1.99], \text{   }  z_2 \in [-1.85, -1.62], \text{   and   } z_3 \in [-0.9, -0.71] \)

 Distractor 3: Corresponds to negatives of all zeros AND inversing rational roots.
\item \( z_1 \in [-2.22, -1.99], \text{   }  z_2 \in [-1.28, -1.13], \text{   and   } z_3 \in [-0.63, -0.44] \)

 Distractor 1: Corresponds to negatives of all zeros.
\item \( z_1 \in [-3.19, -2.76], \text{   }  z_2 \in [-2.02, -1.87], \text{   and   } z_3 \in [-0.35, -0.18] \)

 Distractor 4: Corresponds to moving factors from one rational to another.
\end{enumerate}

\textbf{General Comment:} Remember to try the middle-most integers first as these normally are the zeros. Also, once you get it to a quadratic, you can use your other factoring techniques to finish factoring.
}
\litem{
Perform the division below. Then, find the intervals that correspond to the quotient in the form $ax^2+bx+c$ and remainder $r$.
\[ \frac{6x^{3} -46 x^{2} +88 x -43}{x -5} \]The solution is \( 6x^{2} -16 x + 8 + \frac{-3}{x -5} \), which is option E.\begin{enumerate}[label=\Alph*.]
\item \( a \in [1, 13], \text{   } b \in [-26, -19], \text{   } c \in [-4, 3], \text{   and   } r \in [-43, -40]. \)

 You multiplied by the synthetic number and subtracted rather than adding during synthetic division.
\item \( a \in [1, 13], \text{   } b \in [-77, -70], \text{   } c \in [465, 475], \text{   and   } r \in [-2386, -2380]. \)

 You divided by the opposite of the factor.
\item \( a \in [26, 32], \text{   } b \in [102, 112], \text{   } c \in [605, 610], \text{   and   } r \in [2996, 3001]. \)

 You multiplied by the synthetic number rather than bringing the first factor down.
\item \( a \in [26, 32], \text{   } b \in [-198, -189], \text{   } c \in [1064, 1072], \text{   and   } r \in [-5386, -5379]. \)

 You divided by the opposite of the factor AND multiplied the first factor rather than just bringing it down.
\item \( a \in [1, 13], \text{   } b \in [-19, -13], \text{   } c \in [8, 14], \text{   and   } r \in [-7, 2]. \)

* This is the solution!
\end{enumerate}

\textbf{General Comment:} Be sure to synthetically divide by the zero of the denominator!
}
\litem{
Perform the division below. Then, find the intervals that correspond to the quotient in the form $ax^2+bx+c$ and remainder $r$.
\[ \frac{12x^{3} -64 x^{2} +100 x -52}{x -3} \]The solution is \( 12x^{2} -28 x + 16 + \frac{-4}{x -3} \), which is option A.\begin{enumerate}[label=\Alph*.]
\item \( a \in [8, 17], \text{   } b \in [-28, -25], \text{   } c \in [14, 17], \text{   and   } r \in [-4, 0]. \)

* This is the solution!
\item \( a \in [8, 17], \text{   } b \in [-100, -98], \text{   } c \in [400, 402], \text{   and   } r \in [-1254, -1246]. \)

 You divided by the opposite of the factor.
\item \( a \in [33, 45], \text{   } b \in [39, 48], \text{   } c \in [226, 233], \text{   and   } r \in [642, 646]. \)

 You multiplied by the synthetic number rather than bringing the first factor down.
\item \( a \in [33, 45], \text{   } b \in [-175, -166], \text{   } c \in [616, 624], \text{   and   } r \in [-1903, -1893]. \)

 You divided by the opposite of the factor AND multiplied the first factor rather than just bringing it down.
\item \( a \in [8, 17], \text{   } b \in [-44, -38], \text{   } c \in [19, 21], \text{   and   } r \in [-19, -11]. \)

 You multiplied by the synthetic number and subtracted rather than adding during synthetic division.
\end{enumerate}

\textbf{General Comment:} Be sure to synthetically divide by the zero of the denominator!
}
\end{enumerate}

\end{document}