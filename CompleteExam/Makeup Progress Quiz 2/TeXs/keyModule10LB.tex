\documentclass{extbook}[14pt]
\usepackage{multicol, enumerate, enumitem, hyperref, color, soul, setspace, parskip, fancyhdr, amssymb, amsthm, amsmath, latexsym, units, mathtools}
\everymath{\displaystyle}
\usepackage[headsep=0.5cm,headheight=0cm, left=1 in,right= 1 in,top= 1 in,bottom= 1 in]{geometry}
\usepackage{dashrule}  % Package to use the command below to create lines between items
\newcommand{\litem}[1]{\item #1

\rule{\textwidth}{0.4pt}}
\pagestyle{fancy}
\lhead{}
\chead{Answer Key for Makeup Progress Quiz 2 Version B}
\rhead{}
\lfoot{5763-3522}
\cfoot{}
\rfoot{Spring 2021}
\begin{document}
\textbf{This key should allow you to understand why you choose the option you did (beyond just getting a question right or wrong). \href{https://xronos.clas.ufl.edu/mac1105spring2020/courseDescriptionAndMisc/Exams/LearningFromResults}{More instructions on how to use this key can be found here}.}

\textbf{If you have a suggestion to make the keys better, \href{https://forms.gle/CZkbZmPbC9XALEE88}{please fill out the short survey here}.}

\textit{Note: This key is auto-generated and may contain issues and/or errors. The keys are reviewed after each exam to ensure grading is done accurately. If there are issues (like duplicate options), they are noted in the offline gradebook. The keys are a work-in-progress to give students as many resources to improve as possible.}

\rule{\textwidth}{0.4pt}

\begin{enumerate}\litem{
Factor the polynomial below completely. Then, choose the intervals the zeros of the polynomial belong to, where $z_1 \leq z_2 \leq z_3$. \textit{To make the problem easier, all zeros are between -5 and 5.}
\[ f(x) = 12x^{3} -29 x^{2} -15 x + 50 \]The solution is \( [-1.25, 1.6666666666666667, 2] \), which is option A.\begin{enumerate}[label=\Alph*.]
\item \( z_1 \in [-1.61, -0.88], \text{   }  z_2 \in [1.57, 1.67], \text{   and   } z_3 \in [1.37, 2.08] \)

* This is the solution!
\item \( z_1 \in [-1.19, -0.67], \text{   }  z_2 \in [0.5, 0.8], \text{   and   } z_3 \in [1.37, 2.08] \)

 Distractor 2: Corresponds to inversing rational roots.
\item \( z_1 \in [-2.28, -1.85], \text{   }  z_2 \in [-1.78, -1.59], \text{   and   } z_3 \in [1.21, 1.48] \)

 Distractor 1: Corresponds to negatives of all zeros.
\item \( z_1 \in [-2.28, -1.85], \text{   }  z_2 \in [-0.44, -0.3], \text{   and   } z_3 \in [4.54, 5.07] \)

 Distractor 4: Corresponds to moving factors from one rational to another.
\item \( z_1 \in [-2.28, -1.85], \text{   }  z_2 \in [-0.61, -0.43], \text{   and   } z_3 \in [0.26, 1.13] \)

 Distractor 3: Corresponds to negatives of all zeros AND inversing rational roots.
\end{enumerate}

\textbf{General Comment:} Remember to try the middle-most integers first as these normally are the zeros. Also, once you get it to a quadratic, you can use your other factoring techniques to finish factoring.
}
\litem{
Factor the polynomial below completely, knowing that $x+4$ is a factor. Then, choose the intervals the zeros of the polynomial belong to, where $z_1 \leq z_2 \leq z_3 \leq z_4$. \textit{To make the problem easier, all zeros are between -5 and 5.}
\[ f(x) = 12x^{4} +13 x^{3} -253 x^{2} -512 x -240 \]The solution is \( [-4, -1.3333333333333333, -0.75, 5] \), which is option B.\begin{enumerate}[label=\Alph*.]
\item \( z_1 \in [-4.6, -3.2], \text{   }  z_2 \in [-2.95, -0.11], z_3 \in [-2, 0.6], \text{   and   } z_4 \in [4.06, 5.01] \)

 Distractor 2: Corresponds to inversing rational roots.
\item \( z_1 \in [-4.6, -3.2], \text{   }  z_2 \in [-2.95, -0.11], z_3 \in [-2, 0.6], \text{   and   } z_4 \in [4.06, 5.01] \)

* This is the solution!
\item \( z_1 \in [-6, -4.3], \text{   }  z_2 \in [0.36, 1.25], z_3 \in [1.1, 2.4], \text{   and   } z_4 \in [3.9, 4.7] \)

 Distractor 3: Corresponds to negatives of all zeros AND inversing rational roots.
\item \( z_1 \in [-6, -4.3], \text{   }  z_2 \in [-0.27, 0.43], z_3 \in [2.4, 3.3], \text{   and   } z_4 \in [3.9, 4.7] \)

 Distractor 4: Corresponds to moving factors from one rational to another.
\item \( z_1 \in [-6, -4.3], \text{   }  z_2 \in [0.36, 1.25], z_3 \in [1.1, 2.4], \text{   and   } z_4 \in [3.9, 4.7] \)

 Distractor 1: Corresponds to negatives of all zeros.
\end{enumerate}

\textbf{General Comment:} Remember to try the middle-most integers first as these normally are the zeros. Also, once you get it to a quadratic, you can use your other factoring techniques to finish factoring.
}
\litem{
Factor the polynomial below completely, knowing that $x-5$ is a factor. Then, choose the intervals the zeros of the polynomial belong to, where $z_1 \leq z_2 \leq z_3 \leq z_4$. \textit{To make the problem easier, all zeros are between -5 and 5.}
\[ f(x) = 8x^{4} -58 x^{3} +79 x^{2} +85 x -150 \]The solution is \( [-1.25, 1.5, 2, 5] \), which is option B.\begin{enumerate}[label=\Alph*.]
\item \( z_1 \in [-1.2, -0.72], \text{   }  z_2 \in [0.4, 1.1], z_3 \in [1.72, 2.14], \text{   and   } z_4 \in [4.96, 5.07] \)

 Distractor 2: Corresponds to inversing rational roots.
\item \( z_1 \in [-1.77, -1.12], \text{   }  z_2 \in [1.1, 1.9], z_3 \in [1.72, 2.14], \text{   and   } z_4 \in [4.96, 5.07] \)

* This is the solution!
\item \( z_1 \in [-5.99, -4.14], \text{   }  z_2 \in [-4.1, -2.4], z_3 \in [-2.21, -1.94], \text{   and   } z_4 \in [0.62, 0.74] \)

 Distractor 4: Corresponds to moving factors from one rational to another.
\item \( z_1 \in [-5.99, -4.14], \text{   }  z_2 \in [-2.3, -0.6], z_3 \in [-1.78, -1.34], \text{   and   } z_4 \in [1.19, 1.27] \)

 Distractor 1: Corresponds to negatives of all zeros.
\item \( z_1 \in [-5.99, -4.14], \text{   }  z_2 \in [-2.3, -0.6], z_3 \in [-0.9, -0.57], \text{   and   } z_4 \in [0.74, 0.8] \)

 Distractor 3: Corresponds to negatives of all zeros AND inversing rational roots.
\end{enumerate}

\textbf{General Comment:} Remember to try the middle-most integers first as these normally are the zeros. Also, once you get it to a quadratic, you can use your other factoring techniques to finish factoring.
}
\litem{
Perform the division below. Then, find the intervals that correspond to the quotient in the form $ax^2+bx+c$ and remainder $r$.
\[ \frac{15x^{3} +38 x^{2} -37}{x + 2} \]The solution is \( 15x^{2} +8 x -16 + \frac{-5}{x + 2} \), which is option C.\begin{enumerate}[label=\Alph*.]
\item \( a \in [12, 16], b \in [-9, -4], c \in [21, 25], \text{ and } r \in [-104, -97]. \)

 You multipled by the synthetic number and subtracted rather than adding during synthetic division.
\item \( a \in [-34, -24], b \in [98, 100], c \in [-197, -194], \text{ and } r \in [349, 357]. \)

 You multipled by the synthetic number rather than bringing the first factor down.
\item \( a \in [12, 16], b \in [4, 12], c \in [-21, -9], \text{ and } r \in [-8, -1]. \)

* This is the solution!
\item \( a \in [12, 16], b \in [60, 70], c \in [133, 144], \text{ and } r \in [234, 237]. \)

 You divided by the opposite of the factor.
\item \( a \in [-34, -24], b \in [-24, -20], c \in [-44, -42], \text{ and } r \in [-132, -121]. \)

 You divided by the opposite of the factor AND multipled the first factor rather than just bringing it down.
\end{enumerate}

\textbf{General Comment:} Be sure to synthetically divide by the zero of the denominator! Also, make sure to include 0 placeholders for missing terms.
}
\litem{
Perform the division below. Then, find the intervals that correspond to the quotient in the form $ax^2+bx+c$ and remainder $r$.
\[ \frac{8x^{3} -14 x^{2} -19 x + 25}{x -2} \]The solution is \( 8x^{2} +2 x -15 + \frac{-5}{x -2} \), which is option D.\begin{enumerate}[label=\Alph*.]
\item \( a \in [6, 10], \text{   } b \in [-8, -1], \text{   } c \in [-32, -17], \text{   and   } r \in [-3, 6]. \)

 You multiplied by the synthetic number and subtracted rather than adding during synthetic division.
\item \( a \in [6, 10], \text{   } b \in [-35, -27], \text{   } c \in [40, 43], \text{   and   } r \in [-64, -52]. \)

 You divided by the opposite of the factor.
\item \( a \in [16, 24], \text{   } b \in [-46, -38], \text{   } c \in [73, 74], \text{   and   } r \in [-124, -115]. \)

 You divided by the opposite of the factor AND multiplied the first factor rather than just bringing it down.
\item \( a \in [6, 10], \text{   } b \in [-5, 6], \text{   } c \in [-19, -9], \text{   and   } r \in [-5, -2]. \)

* This is the solution!
\item \( a \in [16, 24], \text{   } b \in [13, 19], \text{   } c \in [15, 21], \text{   and   } r \in [57, 62]. \)

 You multiplied by the synthetic number rather than bringing the first factor down.
\end{enumerate}

\textbf{General Comment:} Be sure to synthetically divide by the zero of the denominator!
}
\litem{
Factor the polynomial below completely. Then, choose the intervals the zeros of the polynomial belong to, where $z_1 \leq z_2 \leq z_3$. \textit{To make the problem easier, all zeros are between -5 and 5.}
\[ f(x) = 6x^{3} -29 x^{2} +14 x + 24 \]The solution is \( [-0.6666666666666666, 1.5, 4] \), which is option B.\begin{enumerate}[label=\Alph*.]
\item \( z_1 \in [-5.5, -3.9], \text{   }  z_2 \in [-1, -0.5], \text{   and   } z_3 \in [1.27, 1.57] \)

 Distractor 3: Corresponds to negatives of all zeros AND inversing rational roots.
\item \( z_1 \in [-1, -0.3], \text{   }  z_2 \in [1.4, 1.9], \text{   and   } z_3 \in [3.82, 4.15] \)

* This is the solution!
\item \( z_1 \in [-5.5, -3.9], \text{   }  z_2 \in [-3.9, -2.7], \text{   and   } z_3 \in [0.17, 0.38] \)

 Distractor 4: Corresponds to moving factors from one rational to another.
\item \( z_1 \in [-5.5, -3.9], \text{   }  z_2 \in [-1.9, -0.7], \text{   and   } z_3 \in [0.42, 0.76] \)

 Distractor 1: Corresponds to negatives of all zeros.
\item \( z_1 \in [-2.2, -1.2], \text{   }  z_2 \in [0.3, 0.9], \text{   and   } z_3 \in [3.82, 4.15] \)

 Distractor 2: Corresponds to inversing rational roots.
\end{enumerate}

\textbf{General Comment:} Remember to try the middle-most integers first as these normally are the zeros. Also, once you get it to a quadratic, you can use your other factoring techniques to finish factoring.
}
\litem{
What are the \textit{possible Integer} roots of the polynomial below?
\[ f(x) = 4x^{3} +2 x^{2} +3 x + 5 \]The solution is \( \pm 1,\pm 5 \), which is option A.\begin{enumerate}[label=\Alph*.]
\item \( \pm 1,\pm 5 \)

* This is the solution \textbf{since we asked for the possible Integer roots}!
\item \( \pm 1,\pm 2,\pm 4 \)

 Distractor 1: Corresponds to the plus or minus factors of a1 only.
\item \( \text{ All combinations of: }\frac{\pm 1,\pm 5}{\pm 1,\pm 2,\pm 4} \)

This would have been the solution \textbf{if asked for the possible Rational roots}!
\item \( \text{ All combinations of: }\frac{\pm 1,\pm 2,\pm 4}{\pm 1,\pm 5} \)

 Distractor 3: Corresponds to the plus or minus of the inverse quotient (an/a0) of the factors. 
\item \( \text{There is no formula or theorem that tells us all possible Integer roots.} \)

 Distractor 4: Corresponds to not recognizing Integers as a subset of Rationals.
\end{enumerate}

\textbf{General Comment:} We have a way to find the possible Rational roots. The possible Integer roots are the Integers in this list.
}
\litem{
Perform the division below. Then, find the intervals that correspond to the quotient in the form $ax^2+bx+c$ and remainder $r$.
\[ \frac{20x^{3} +65 x^{2} -41}{x + 3} \]The solution is \( 20x^{2} +5 x -15 + \frac{4}{x + 3} \), which is option A.\begin{enumerate}[label=\Alph*.]
\item \( a \in [15, 24], b \in [-1, 9], c \in [-18, -12], \text{ and } r \in [2, 8]. \)

* This is the solution!
\item \( a \in [-63, -58], b \in [-119, -114], c \in [-347, -343], \text{ and } r \in [-1078, -1070]. \)

 You divided by the opposite of the factor AND multipled the first factor rather than just bringing it down.
\item \( a \in [15, 24], b \in [-15, -10], c \in [60, 64], \text{ and } r \in [-282, -272]. \)

 You multipled by the synthetic number and subtracted rather than adding during synthetic division.
\item \( a \in [-63, -58], b \in [242, 247], c \in [-739, -729], \text{ and } r \in [2163, 2166]. \)

 You multipled by the synthetic number rather than bringing the first factor down.
\item \( a \in [15, 24], b \in [124, 129], c \in [372, 376], \text{ and } r \in [1082, 1085]. \)

 You divided by the opposite of the factor.
\end{enumerate}

\textbf{General Comment:} Be sure to synthetically divide by the zero of the denominator! Also, make sure to include 0 placeholders for missing terms.
}
\litem{
Perform the division below. Then, find the intervals that correspond to the quotient in the form $ax^2+bx+c$ and remainder $r$.
\[ \frac{20x^{3} -17 x^{2} -40 x -15}{x -2} \]The solution is \( 20x^{2} +23 x + 6 + \frac{-3}{x -2} \), which is option C.\begin{enumerate}[label=\Alph*.]
\item \( a \in [17, 28], \text{   } b \in [-58, -56], \text{   } c \in [73, 84], \text{   and   } r \in [-166, -161]. \)

 You divided by the opposite of the factor.
\item \( a \in [17, 28], \text{   } b \in [1, 4], \text{   } c \in [-38, -33], \text{   and   } r \in [-52, -50]. \)

 You multiplied by the synthetic number and subtracted rather than adding during synthetic division.
\item \( a \in [17, 28], \text{   } b \in [22, 24], \text{   } c \in [6, 10], \text{   and   } r \in [-3, -2]. \)

* This is the solution!
\item \( a \in [37, 44], \text{   } b \in [-98, -90], \text{   } c \in [151, 158], \text{   and   } r \in [-325, -322]. \)

 You divided by the opposite of the factor AND multiplied the first factor rather than just bringing it down.
\item \( a \in [37, 44], \text{   } b \in [60, 71], \text{   } c \in [82, 89], \text{   and   } r \in [153, 162]. \)

 You multiplied by the synthetic number rather than bringing the first factor down.
\end{enumerate}

\textbf{General Comment:} Be sure to synthetically divide by the zero of the denominator!
}
\litem{
What are the \textit{possible Rational} roots of the polynomial below?
\[ f(x) = 5x^{4} +5 x^{3} +7 x^{2} +6 x + 2 \]The solution is \( \text{ All combinations of: }\frac{\pm 1,\pm 2}{\pm 1,\pm 5} \), which is option B.\begin{enumerate}[label=\Alph*.]
\item \( \pm 1,\pm 5 \)

 Distractor 1: Corresponds to the plus or minus factors of a1 only.
\item \( \text{ All combinations of: }\frac{\pm 1,\pm 2}{\pm 1,\pm 5} \)

* This is the solution \textbf{since we asked for the possible Rational roots}!
\item \( \text{ All combinations of: }\frac{\pm 1,\pm 5}{\pm 1,\pm 2} \)

 Distractor 3: Corresponds to the plus or minus of the inverse quotient (an/a0) of the factors. 
\item \( \pm 1,\pm 2 \)

This would have been the solution \textbf{if asked for the possible Integer roots}!
\item \( \text{ There is no formula or theorem that tells us all possible Rational roots.} \)

 Distractor 4: Corresponds to not recalling the theorem for rational roots of a polynomial.
\end{enumerate}

\textbf{General Comment:} We have a way to find the possible Rational roots. The possible Integer roots are the Integers in this list.
}
\end{enumerate}

\end{document}