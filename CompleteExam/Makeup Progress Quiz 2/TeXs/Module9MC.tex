\documentclass[14pt]{extbook}
\usepackage{multicol, enumerate, enumitem, hyperref, color, soul, setspace, parskip, fancyhdr} %General Packages
\usepackage{amssymb, amsthm, amsmath, latexsym, units, mathtools} %Math Packages
\everymath{\displaystyle} %All math in Display Style
% Packages with additional options
\usepackage[headsep=0.5cm,headheight=12pt, left=1 in,right= 1 in,top= 1 in,bottom= 1 in]{geometry}
\usepackage[usenames,dvipsnames]{xcolor}
\usepackage{dashrule}  % Package to use the command below to create lines between items
\newcommand{\litem}[1]{\item#1\hspace*{-1cm}\rule{\textwidth}{0.4pt}}
\pagestyle{fancy}
\lhead{Makeup Progress Quiz 2}
\chead{}
\rhead{Version C}
\lfoot{2790-1423}
\cfoot{}
\rfoot{Summer C 2021}
\begin{document}

\begin{enumerate}
\litem{
What is the \textbf{best} way to describe the domain of the scenario below?
\begin{center}
    \textit{ Veronica needs to prepare 170 lbs of blended coffee beans to sell for \$4.71 per pound. She has a high-quality bean that sells for \$6.00 a pound and a low-quality been that sells for \$3.25 a pound. }
\end{center}
\begin{enumerate}[label=\Alph*.]
\item \( \text{Subset of the Rational numbers} \)
\item \( \text{Proper subset of the Real numbers} \)
\item \( \text{Subset of the Natural numbers} \)
\item \( \text{Subset of the Integers} \)
\item \( \text{There is no restricted domain in this scenario} \)

\end{enumerate} }
\litem{
Using the situation below, construct a linear model that describes the cost of the coffee beans $C(h)$ in terms of the weight of the low-quality coffee beans $h$.
\begin{center}
    \textit{ Veronica needs to prepare 110 of blended coffee beans selling for \$5.29 per pound. She has a high-quality bean that sells for \$5.85 a pound and a low-quality bean that sells for \$4.71 a pound. }
\end{center}
\begin{enumerate}[label=\Alph*.]
\item \( C(h) = 4.71 h \)
\item \( C(h) = -1.14 h + 643.50 \)
\item \( C(h) = 1.14 h + 518.10 \)
\item \( C(h) = 5.28 h \)
\item \( \text{None of the above.} \)

\end{enumerate} }
\litem{
What is the \textbf{best} way to describe the domain of the scenario below?
\begin{center}
    \textit{ Two UFPD are patrolling the campus on foot. To cover more ground, they split up and begin walking in different directions. Office A is walking at 3 mph while Office B is walking at 5 mph. }
\end{center}
\begin{enumerate}[label=\Alph*.]
\item \( \text{Subset of the Rational numbers} \)
\item \( \text{There is no restricted domain in this scenario} \)
\item \( \text{Subset of the Integers} \)
\item \( \text{Proper subset of the Real numbers} \)
\item \( \text{Subset of the Natural numbers} \)

\end{enumerate} }
\litem{
For the information provided below, construct a linear model that describes the total distance of the path, $D$, in terms of the time spent on a particular path \textit{if we know that the time spent on each path was equal}.
\begin{center}
    \textit{ A bicyclist is training for a race on a hilly path. Their bike keeps track of their speed at any time, but not the distance traveled. Their speed traveling up a hill is 5 mph, 10 mph when traveling down a hill, and 7 mph when traveling along a flat portion. }
\end{center}
\begin{enumerate}[label=\Alph*.]
\item \( 22 t \)
\item \( 350 t \)
\item \( 0.443 t \)
\item \( \text{The model can be found with the information provided, but isn't options 1-3.} \)
\item \( \text{The model cannot be found with the information provided.} \)

\end{enumerate} }
\litem{
For the information provided below, construct a linear model that describes her total income, $I$, as a function of the number of months, $x$ she is at UF.
\begin{center}
    \textit{ Aubrey is a college student going into her first year at UF. She will receive Bright Futures, which covers her tuition plus a \$1000 educational expense each year. Before college, Aubrey saved up \$6000. She knows she will need to pay \$800 in rent a month, \$50 for food a week, and \$56 in other weekly expenses. }
\end{center}
\begin{enumerate}[label=\Alph*.]
\item \( I(x) = 1000 x + 6000 \)
\item \( I(x) = 7000 \)
\item \( I(x) = 6000 x + 1000 \)
\item \( I(x) = 7000 x \)
\item \( \text{None of the above.} \)

\end{enumerate} }
\litem{
A town has an initial population of 100000. The town's population for the next 9 years is provided below. Which type of function would be most appropriate to model the town's population?

\begin{tabular}{c|c|c|c|c|c|c|c|c|c}
\textbf{Year} &1 &2 &3 &4 &5 &6 &7 &8 &9\tabularnewline \hline
\textbf{Pop} &100000 &100027 &100043 &100055 &100064 &100071 &100077 &100083 &100087\end{tabular}\begin{enumerate}[label=\Alph*.]
\item \( \text{Linear} \)
\item \( \text{Non-Linear Power} \)
\item \( \text{Logarithmic} \)
\item \( \text{Exponential} \)
\item \( \text{None of the above} \)

\end{enumerate} }
\litem{
Using the situation below, construct a linear model that describes the cost of the coffee beans $C(h)$ in terms of the weight of the low-quality coffee beans $h$.
\begin{center}
    \textit{ Veronica needs to prepare 180 of blended coffee beans selling for \$5.02 per pound. She has a high-quality bean that sells for \$6.04 a pound and a low-quality bean that sells for \$4.23 a pound. }
\end{center}
\begin{enumerate}[label=\Alph*.]
\item \( C(h) = 5.13 h \)
\item \( C(h) = 4.23 h \)
\item \( C(h) = 1.81 h + 761.40 \)
\item \( C(h) = -1.81 h + 1087.20 \)
\item \( \text{None of the above.} \)

\end{enumerate} }
\litem{
For the information provided below, construct a linear model that describes her total costs, $C$, as a function of the number of months, $x$ she is at UF. 
\begin{center}
    \textit{ Aubrey is a college student going into her first year at UF. She will receive Bright Futures, which covers her tuition plus a \$600 educational expense each year. Before college, Aubrey saved up \$8000. She knows she will need to pay \$1000 in rent a month, \$60 for food a week, and \$48 in other weekly expenses. }
\end{center}
\begin{enumerate}[label=\Alph*.]
\item \( C(x) = 8600 x \)
\item \( C(x) = 1108 \)
\item \( C(x) = 8600 \)
\item \( C(x) = 1108 x \)
\item \( \text{None of the above.} \)

\end{enumerate} }
\litem{
A town has an initial population of 70000. The town's population for the next 9 years is provided below. Which type of function would be most appropriate to model the town's population?

\begin{tabular}{c|c|c|c|c|c|c|c|c|c}
\textbf{Year} &1 &2 &3 &4 &5 &6 &7 &8 &9\tabularnewline \hline
\textbf{Pop} &70080 &70160 &70320 &70640 &71280 &72560 &75120 &80240 &90480\end{tabular}\begin{enumerate}[label=\Alph*.]
\item \( \text{Linear} \)
\item \( \text{Non-Linear Power} \)
\item \( \text{Logarithmic} \)
\item \( \text{Exponential} \)
\item \( \text{None of the above} \)

\end{enumerate} }
\litem{
For the information provided below, construct a linear model that describes the total distance of the path, $D$, in terms of the time spent on a particular path \textit{if we know that the time spent on each path was equal}.
\begin{center}
    \textit{ A bicyclist is training for a race on a hilly path. Their bike keeps track of their speed at any time, but not the distance traveled. Their speed traveling up a hill is 5 mph, 10 mph when traveling down a hill, and 8 mph when traveling along a flat portion. }
\end{center}
\begin{enumerate}[label=\Alph*.]
\item \( 0.425 t \)
\item \( 400 t \)
\item \( 23 t \)
\item \( \text{The model can be found with the information provided, but isn't options 1-3.} \)
\item \( \text{The model cannot be found with the information provided.} \)

\end{enumerate} }
\end{enumerate}

\end{document}