\documentclass{extbook}[14pt]
\usepackage{multicol, enumerate, enumitem, hyperref, color, soul, setspace, parskip, fancyhdr, amssymb, amsthm, amsmath, latexsym, units, mathtools}
\everymath{\displaystyle}
\usepackage[headsep=0.5cm,headheight=0cm, left=1 in,right= 1 in,top= 1 in,bottom= 1 in]{geometry}
\usepackage{dashrule}  % Package to use the command below to create lines between items
\newcommand{\litem}[1]{\item #1

\rule{\textwidth}{0.4pt}}
\pagestyle{fancy}
\lhead{}
\chead{Answer Key for Makeup Progress Quiz 2 Version C}
\rhead{}
\lfoot{2790-1423}
\cfoot{}
\rfoot{Summer C 2021}
\begin{document}
\textbf{This key should allow you to understand why you choose the option you did (beyond just getting a question right or wrong). \href{https://xronos.clas.ufl.edu/mac1105spring2020/courseDescriptionAndMisc/Exams/LearningFromResults}{More instructions on how to use this key can be found here}.}

\textbf{If you have a suggestion to make the keys better, \href{https://forms.gle/CZkbZmPbC9XALEE88}{please fill out the short survey here}.}

\textit{Note: This key is auto-generated and may contain issues and/or errors. The keys are reviewed after each exam to ensure grading is done accurately. If there are issues (like duplicate options), they are noted in the offline gradebook. The keys are a work-in-progress to give students as many resources to improve as possible.}

\rule{\textwidth}{0.4pt}

\begin{enumerate}\litem{
Using an interval or intervals, describe all the $x$-values within or including a distance of the given values.
\[ \text{ More than } 9 \text{ units from the number } -1. \]The solution is \( (-\infty, -10) \cup (8, \infty) \), which is option B.\begin{enumerate}[label=\Alph*.]
\item \( [-10, 8] \)

This describes the values no more than 9 from -1
\item \( (-\infty, -10) \cup (8, \infty) \)

This describes the values more than 9 from -1
\item \( (-10, 8) \)

This describes the values less than 9 from -1
\item \( (-\infty, -10] \cup [8, \infty) \)

This describes the values no less than 9 from -1
\item \( \text{None of the above} \)

You likely thought the values in the interval were not correct.
\end{enumerate}

\textbf{General Comment:} When thinking about this language, it helps to draw a number line and try points.
}
\litem{
Using an interval or intervals, describe all the $x$-values within or including a distance of the given values.
\[ \text{ More than } 5 \text{ units from the number } 7. \]The solution is \( \text{None of the above} \), which is option E.\begin{enumerate}[label=\Alph*.]
\item \( (-\infty, -2] \cup [12, \infty) \)

This describes the values no less than 7 from 5
\item \( (-2, 12) \)

This describes the values less than 7 from 5
\item \( (-\infty, -2) \cup (12, \infty) \)

This describes the values more than 7 from 5
\item \( [-2, 12] \)

This describes the values no more than 7 from 5
\item \( \text{None of the above} \)

Options A-D described the values [more/less than] 7 units from 5, which is the reverse of what the question asked.
\end{enumerate}

\textbf{General Comment:} When thinking about this language, it helps to draw a number line and try points.
}
\litem{
Solve the linear inequality below. Then, choose the constant and interval combination that describes the solution set.
\[ -8 - 6 x < \frac{-28 x - 9}{5} \leq -7 - 7 x \]The solution is \( \text{None of the above.} \), which is option E.\begin{enumerate}[label=\Alph*.]
\item \( [a, b), \text{ where } a \in [14.25, 21] \text{ and } b \in [1.5, 5.25] \)

$[15.50, 3.71)$, which corresponds to flipping the inequality and getting negatives of the actual endpoints.
\item \( (-\infty, a) \cup [b, \infty), \text{ where } a \in [13.5, 21] \text{ and } b \in [2.25, 5.25] \)

$(-\infty, 15.50) \cup [3.71, \infty)$, which corresponds to displaying the and-inequality as an or-inequality and getting negatives of the actual endpoints.
\item \( (a, b], \text{ where } a \in [12, 18.75] \text{ and } b \in [3, 6.75] \)

$(15.50, 3.71]$, which is the correct interval but negatives of the actual endpoints.
\item \( (-\infty, a] \cup (b, \infty), \text{ where } a \in [12.75, 18] \text{ and } b \in [-0.75, 11.25] \)

$(-\infty, 15.50] \cup (3.71, \infty)$, which corresponds to displaying the and-inequality as an or-inequality AND flipping the inequality AND getting negatives of the actual endpoints.
\item \( \text{None of the above.} \)

* This is correct as the answer should be $(-15.50, -3.71]$.
\end{enumerate}

\textbf{General Comment:} To solve, you will need to break up the compound inequality into two inequalities. Be sure to keep track of the inequality! It may be best to draw a number line and graph your solution.
}
\litem{
Solve the linear inequality below. Then, choose the constant and interval combination that describes the solution set.
\[ -4 - 3 x \leq \frac{-7 x - 3}{4} < 6 - 8 x \]The solution is \( \text{None of the above.} \), which is option E.\begin{enumerate}[label=\Alph*.]
\item \( [a, b), \text{ where } a \in [-2.25, 3.75] \text{ and } b \in [-2.62, -0.97] \)

$[2.60, -1.08)$, which is the correct interval but negatives of the actual endpoints.
\item \( (a, b], \text{ where } a \in [1.5, 5.25] \text{ and } b \in [-4.5, 0] \)

$(2.60, -1.08]$, which corresponds to flipping the inequality and getting negatives of the actual endpoints.
\item \( (-\infty, a) \cup [b, \infty), \text{ where } a \in [0.75, 6] \text{ and } b \in [-2.32, -0.82] \)

$(-\infty, 2.60) \cup [-1.08, \infty)$, which corresponds to displaying the and-inequality as an or-inequality AND flipping the inequality AND getting negatives of the actual endpoints.
\item \( (-\infty, a] \cup (b, \infty), \text{ where } a \in [0, 7.5] \text{ and } b \in [-1.57, -0.53] \)

$(-\infty, 2.60] \cup (-1.08, \infty)$, which corresponds to displaying the and-inequality as an or-inequality and getting negatives of the actual endpoints.
\item \( \text{None of the above.} \)

* This is correct as the answer should be $[-2.60, 1.08)$.
\end{enumerate}

\textbf{General Comment:} To solve, you will need to break up the compound inequality into two inequalities. Be sure to keep track of the inequality! It may be best to draw a number line and graph your solution.
}
\litem{
Solve the linear inequality below. Then, choose the constant and interval combination that describes the solution set.
\[ -8 + 3 x > 4 x \text{ or } -3 + 7 x < 9 x \]The solution is \( (-\infty, -8.0) \text{ or } (-1.5, \infty) \), which is option D.\begin{enumerate}[label=\Alph*.]
\item \( (-\infty, a] \cup [b, \infty), \text{ where } a \in [-9.75, -7.5] \text{ and } b \in [-3.75, -0.75] \)

Corresponds to including the endpoints (when they should be excluded).
\item \( (-\infty, a) \cup (b, \infty), \text{ where } a \in [0.75, 2.25] \text{ and } b \in [6, 12] \)

Corresponds to inverting the inequality and negating the solution.
\item \( (-\infty, a] \cup [b, \infty), \text{ where } a \in [0, 2.25] \text{ and } b \in [5.25, 11.25] \)

Corresponds to including the endpoints AND negating.
\item \( (-\infty, a) \cup (b, \infty), \text{ where } a \in [-12.75, -4.5] \text{ and } b \in [-3.75, 0] \)

 * Correct option.
\item \( (-\infty, \infty) \)

Corresponds to the variable canceling, which does not happen in this instance.
\end{enumerate}

\textbf{General Comment:} When multiplying or dividing by a negative, flip the sign.
}
\litem{
Solve the linear inequality below. Then, choose the constant and interval combination that describes the solution set.
\[ \frac{-9}{2} - \frac{9}{6} x \leq \frac{-5}{7} x + \frac{3}{4} \]The solution is \( [-6.682, \infty) \), which is option C.\begin{enumerate}[label=\Alph*.]
\item \( [a, \infty), \text{ where } a \in [4.5, 7.5] \)

 $[6.682, \infty)$, which corresponds to negating the endpoint of the solution.
\item \( (-\infty, a], \text{ where } a \in [-8.25, -0.75] \)

 $(-\infty, -6.682]$, which corresponds to switching the direction of the interval. You likely did this if you did not flip the inequality when dividing by a negative!
\item \( [a, \infty), \text{ where } a \in [-8.25, -6] \)

* $[-6.682, \infty)$, which is the correct option.
\item \( (-\infty, a], \text{ where } a \in [6, 9] \)

 $(-\infty, 6.682]$, which corresponds to switching the direction of the interval AND negating the endpoint. You likely did this if you did not flip the inequality when dividing by a negative as well as not moving values over to a side properly.
\item \( \text{None of the above}. \)

You may have chosen this if you thought the inequality did not match the ends of the intervals.
\end{enumerate}

\textbf{General Comment:} Remember that less/greater than or equal to includes the endpoint, while less/greater do not. Also, remember that you need to flip the inequality when you multiply or divide by a negative.
}
\litem{
Solve the linear inequality below. Then, choose the constant and interval combination that describes the solution set.
\[ -5 + 7 x > 8 x \text{ or } 7 + 5 x < 8 x \]The solution is \( (-\infty, -5.0) \text{ or } (2.333, \infty) \), which is option C.\begin{enumerate}[label=\Alph*.]
\item \( (-\infty, a] \cup [b, \infty), \text{ where } a \in [-3, -1.5] \text{ and } b \in [4.27, 6] \)

Corresponds to including the endpoints AND negating.
\item \( (-\infty, a] \cup [b, \infty), \text{ where } a \in [-8.25, -3.75] \text{ and } b \in [1.27, 4.42] \)

Corresponds to including the endpoints (when they should be excluded).
\item \( (-\infty, a) \cup (b, \infty), \text{ where } a \in [-6, -3] \text{ and } b \in [-2.25, 3] \)

 * Correct option.
\item \( (-\infty, a) \cup (b, \infty), \text{ where } a \in [-3, 0] \text{ and } b \in [3, 7.5] \)

Corresponds to inverting the inequality and negating the solution.
\item \( (-\infty, \infty) \)

Corresponds to the variable canceling, which does not happen in this instance.
\end{enumerate}

\textbf{General Comment:} When multiplying or dividing by a negative, flip the sign.
}
\litem{
Solve the linear inequality below. Then, choose the constant and interval combination that describes the solution set.
\[ 4x -10 \geq 7x + 5 \]The solution is \( (-\infty, -5.0] \), which is option D.\begin{enumerate}[label=\Alph*.]
\item \( [a, \infty), \text{ where } a \in [2, 6] \)

 $[5.0, \infty)$, which corresponds to switching the direction of the interval AND negating the endpoint. You likely did this if you did not flip the inequality when dividing by a negative as well as not moving values over to a side properly.
\item \( (-\infty, a], \text{ where } a \in [1, 8] \)

 $(-\infty, 5.0]$, which corresponds to negating the endpoint of the solution.
\item \( [a, \infty), \text{ where } a \in [-10, -4] \)

 $[-5.0, \infty)$, which corresponds to switching the direction of the interval. You likely did this if you did not flip the inequality when dividing by a negative!
\item \( (-\infty, a], \text{ where } a \in [-11, 0] \)

* $(-\infty, -5.0]$, which is the correct option.
\item \( \text{None of the above}. \)

You may have chosen this if you thought the inequality did not match the ends of the intervals.
\end{enumerate}

\textbf{General Comment:} Remember that less/greater than or equal to includes the endpoint, while less/greater do not. Also, remember that you need to flip the inequality when you multiply or divide by a negative.
}
\litem{
Solve the linear inequality below. Then, choose the constant and interval combination that describes the solution set.
\[ -9x -7 > -4x + 7 \]The solution is \( (-\infty, -2.8) \), which is option C.\begin{enumerate}[label=\Alph*.]
\item \( (-\infty, a), \text{ where } a \in [0.8, 5.8] \)

 $(-\infty, 2.8)$, which corresponds to negating the endpoint of the solution.
\item \( (a, \infty), \text{ where } a \in [2.8, 3.8] \)

 $(2.8, \infty)$, which corresponds to switching the direction of the interval AND negating the endpoint. You likely did this if you did not flip the inequality when dividing by a negative as well as not moving values over to a side properly.
\item \( (-\infty, a), \text{ where } a \in [-4.8, -1.8] \)

* $(-\infty, -2.8)$, which is the correct option.
\item \( (a, \infty), \text{ where } a \in [-7.8, -1.8] \)

 $(-2.8, \infty)$, which corresponds to switching the direction of the interval. You likely did this if you did not flip the inequality when dividing by a negative!
\item \( \text{None of the above}. \)

You may have chosen this if you thought the inequality did not match the ends of the intervals.
\end{enumerate}

\textbf{General Comment:} Remember that less/greater than or equal to includes the endpoint, while less/greater do not. Also, remember that you need to flip the inequality when you multiply or divide by a negative.
}
\litem{
Solve the linear inequality below. Then, choose the constant and interval combination that describes the solution set.
\[ \frac{6}{8} + \frac{6}{4} x \leq \frac{8}{9} x - \frac{3}{5} \]The solution is \( (-\infty, -2.209] \), which is option B.\begin{enumerate}[label=\Alph*.]
\item \( (-\infty, a], \text{ where } a \in [1.5, 3] \)

 $(-\infty, 2.209]$, which corresponds to negating the endpoint of the solution.
\item \( (-\infty, a], \text{ where } a \in [-4.5, -0.75] \)

* $(-\infty, -2.209]$, which is the correct option.
\item \( [a, \infty), \text{ where } a \in [0, 5.25] \)

 $[2.209, \infty)$, which corresponds to switching the direction of the interval AND negating the endpoint. You likely did this if you did not flip the inequality when dividing by a negative as well as not moving values over to a side properly.
\item \( [a, \infty), \text{ where } a \in [-3.75, 0] \)

 $[-2.209, \infty)$, which corresponds to switching the direction of the interval. You likely did this if you did not flip the inequality when dividing by a negative!
\item \( \text{None of the above}. \)

You may have chosen this if you thought the inequality did not match the ends of the intervals.
\end{enumerate}

\textbf{General Comment:} Remember that less/greater than or equal to includes the endpoint, while less/greater do not. Also, remember that you need to flip the inequality when you multiply or divide by a negative.
}
\end{enumerate}

\end{document}