\documentclass{extbook}[14pt]
\usepackage{multicol, enumerate, enumitem, hyperref, color, soul, setspace, parskip, fancyhdr, amssymb, amsthm, amsmath, latexsym, units, mathtools}
\everymath{\displaystyle}
\usepackage[headsep=0.5cm,headheight=0cm, left=1 in,right= 1 in,top= 1 in,bottom= 1 in]{geometry}
\usepackage{dashrule}  % Package to use the command below to create lines between items
\newcommand{\litem}[1]{\item #1

\rule{\textwidth}{0.4pt}}
\pagestyle{fancy}
\lhead{}
\chead{Answer Key for Makeup Progress Quiz 2 Version C}
\rhead{}
\lfoot{2790-1423}
\cfoot{}
\rfoot{Summer C 2021}
\begin{document}
\textbf{This key should allow you to understand why you choose the option you did (beyond just getting a question right or wrong). \href{https://xronos.clas.ufl.edu/mac1105spring2020/courseDescriptionAndMisc/Exams/LearningFromResults}{More instructions on how to use this key can be found here}.}

\textbf{If you have a suggestion to make the keys better, \href{https://forms.gle/CZkbZmPbC9XALEE88}{please fill out the short survey here}.}

\textit{Note: This key is auto-generated and may contain issues and/or errors. The keys are reviewed after each exam to ensure grading is done accurately. If there are issues (like duplicate options), they are noted in the offline gradebook. The keys are a work-in-progress to give students as many resources to improve as possible.}

\rule{\textwidth}{0.4pt}

\begin{enumerate}\litem{
Simplify the expression below into the form $a+bi$. Then, choose the intervals that $a$ and $b$ belong to.
\[ (6 - 10 i)(8 + 2 i) \]The solution is \( 68 - 68 i \), which is option B.\begin{enumerate}[label=\Alph*.]
\item \( a \in [67, 73] \text{ and } b \in [68, 74] \)

 $68 + 68 i$, which corresponds to adding a minus sign in both terms.
\item \( a \in [67, 73] \text{ and } b \in [-69, -62] \)

* $68 - 68 i$, which is the correct option.
\item \( a \in [26, 32] \text{ and } b \in [-95, -86] \)

 $28 - 92 i$, which corresponds to adding a minus sign in the second term.
\item \( a \in [48, 51] \text{ and } b \in [-21, -19] \)

 $48 - 20 i$, which corresponds to just multiplying the real terms to get the real part of the solution and the coefficients in the complex terms to get the complex part.
\item \( a \in [26, 32] \text{ and } b \in [89, 94] \)

 $28 + 92 i$, which corresponds to adding a minus sign in the first term.
\end{enumerate}

\textbf{General Comment:} You can treat $i$ as a variable and distribute. Just remember that $i^2=-1$, so you can continue to reduce after you distribute.
}
\litem{
Simplify the expression below into the form $a+bi$. Then, choose the intervals that $a$ and $b$ belong to.
\[ \frac{-9 - 33 i}{4 + 6 i} \]The solution is \( -4.50  - 1.50 i \), which is option D.\begin{enumerate}[label=\Alph*.]
\item \( a \in [-234.5, -233] \text{ and } b \in [-2, 0] \)

 $-234.00  - 1.50 i$, which corresponds to forgetting to multiply the conjugate by the numerator and using a plus instead of a minus in the denominator.
\item \( a \in [2, 4] \text{ and } b \in [-4, -2.5] \)

 $3.12  - 3.58 i$, which corresponds to forgetting to multiply the conjugate by the numerator and not computing the conjugate correctly.
\item \( a \in [-5, -4] \text{ and } b \in [-79, -77] \)

 $-4.50  - 78.00 i$, which corresponds to forgetting to multiply the conjugate by the numerator.
\item \( a \in [-5, -4] \text{ and } b \in [-2, 0] \)

* $-4.50  - 1.50 i$, which is the correct option.
\item \( a \in [-2.5, -2] \text{ and } b \in [-7, -4] \)

 $-2.25  - 5.50 i$, which corresponds to just dividing the first term by the first term and the second by the second.
\end{enumerate}

\textbf{General Comment:} Multiply the numerator and denominator by the *conjugate* of the denominator, then simplify. For example, if we have $2+3i$, the conjugate is $2-3i$.
}
\litem{
Choose the \textbf{smallest} set of Real numbers that the number below belongs to.
\[ -\sqrt{\frac{361}{196}} \]The solution is \( \text{Rational} \), which is option A.\begin{enumerate}[label=\Alph*.]
\item \( \text{Rational} \)

* This is the correct option!
\item \( \text{Not a Real number} \)

These are Nonreal Complex numbers \textbf{OR} things that are not numbers (e.g., dividing by 0).
\item \( \text{Whole} \)

These are the counting numbers with 0 (0, 1, 2, 3, ...)
\item \( \text{Irrational} \)

These cannot be written as a fraction of Integers.
\item \( \text{Integer} \)

These are the negative and positive counting numbers (..., -3, -2, -1, 0, 1, 2, 3, ...)
\end{enumerate}

\textbf{General Comment:} First, you \textbf{NEED} to simplify the expression. This question simplifies to $-\frac{19}{14}$. 
 
 Be sure you look at the simplified fraction and not just the decimal expansion. Numbers such as 13, 17, and 19 provide \textbf{long but repeating/terminating decimal expansions!} 
 
 The only ways to *not* be a Real number are: dividing by 0 or taking the square root of a negative number. 
 
 Irrational numbers are more than just square root of 3: adding or subtracting values from square root of 3 is also irrational.
}
\litem{
Simplify the expression below and choose the interval the simplification is contained within.
\[ 16 - 14^2 + 3 \div 10 * 20 \div 8 \]The solution is \( -179.250 \), which is option A.\begin{enumerate}[label=\Alph*.]
\item \( [-179.98, -179.06] \)

* -179.250, this is the correct option
\item \( [211.55, 212.23] \)

 212.002, which corresponds to two Order of Operations errors.
\item \( [-180.46, -179.51] \)

 -179.998, which corresponds to an Order of Operations error: not reading left-to-right for multiplication/division.
\item \( [212.7, 213.41] \)

 212.750, which corresponds to an Order of Operations error: multiplying by negative before squaring. For example: $(-3)^2 \neq -3^2$
\item \( \text{None of the above} \)

 You may have gotten this by making an unanticipated error. If you got a value that is not any of the others, please let the coordinator know so they can help you figure out what happened.
\end{enumerate}

\textbf{General Comment:} While you may remember (or were taught) PEMDAS is done in order, it is actually done as P/E/MD/AS. When we are at MD or AS, we read left to right.
}
\litem{
Choose the \textbf{smallest} set of Real numbers that the number below belongs to.
\[ \sqrt{\frac{49}{529}} \]The solution is \( \text{Rational} \), which is option A.\begin{enumerate}[label=\Alph*.]
\item \( \text{Rational} \)

* This is the correct option!
\item \( \text{Integer} \)

These are the negative and positive counting numbers (..., -3, -2, -1, 0, 1, 2, 3, ...)
\item \( \text{Irrational} \)

These cannot be written as a fraction of Integers.
\item \( \text{Not a Real number} \)

These are Nonreal Complex numbers \textbf{OR} things that are not numbers (e.g., dividing by 0).
\item \( \text{Whole} \)

These are the counting numbers with 0 (0, 1, 2, 3, ...)
\end{enumerate}

\textbf{General Comment:} First, you \textbf{NEED} to simplify the expression. This question simplifies to $\frac{7}{23}$. 
 
 Be sure you look at the simplified fraction and not just the decimal expansion. Numbers such as 13, 17, and 19 provide \textbf{long but repeating/terminating decimal expansions!} 
 
 The only ways to *not* be a Real number are: dividing by 0 or taking the square root of a negative number. 
 
 Irrational numbers are more than just square root of 3: adding or subtracting values from square root of 3 is also irrational.
}
\litem{
Simplify the expression below and choose the interval the simplification is contained within.
\[ 9 - 14^2 + 4 \div 16 * 15 \div 1 \]The solution is \( -183.250 \), which is option C.\begin{enumerate}[label=\Alph*.]
\item \( [203.8, 205.3] \)

 205.017, which corresponds to two Order of Operations errors.
\item \( [205.2, 211.4] \)

 208.750, which corresponds to an Order of Operations error: multiplying by negative before squaring. For example: $(-3)^2 \neq -3^2$
\item \( [-185.1, -182.9] \)

* -183.250, this is the correct option
\item \( [-188, -183.7] \)

 -186.983, which corresponds to an Order of Operations error: not reading left-to-right for multiplication/division.
\item \( \text{None of the above} \)

 You may have gotten this by making an unanticipated error. If you got a value that is not any of the others, please let the coordinator know so they can help you figure out what happened.
\end{enumerate}

\textbf{General Comment:} While you may remember (or were taught) PEMDAS is done in order, it is actually done as P/E/MD/AS. When we are at MD or AS, we read left to right.
}
\litem{
Simplify the expression below into the form $a+bi$. Then, choose the intervals that $a$ and $b$ belong to.
\[ (-2 - 8 i)(6 + 7 i) \]The solution is \( 44 - 62 i \), which is option C.\begin{enumerate}[label=\Alph*.]
\item \( a \in [-12, -5] \text{ and } b \in [-58, -55] \)

 $-12 - 56 i$, which corresponds to just multiplying the real terms to get the real part of the solution and the coefficients in the complex terms to get the complex part.
\item \( a \in [-70, -65] \text{ and } b \in [27, 40] \)

 $-68 + 34 i$, which corresponds to adding a minus sign in the first term.
\item \( a \in [39, 48] \text{ and } b \in [-63, -61] \)

* $44 - 62 i$, which is the correct option.
\item \( a \in [-70, -65] \text{ and } b \in [-36, -27] \)

 $-68 - 34 i$, which corresponds to adding a minus sign in the second term.
\item \( a \in [39, 48] \text{ and } b \in [59, 63] \)

 $44 + 62 i$, which corresponds to adding a minus sign in both terms.
\end{enumerate}

\textbf{General Comment:} You can treat $i$ as a variable and distribute. Just remember that $i^2=-1$, so you can continue to reduce after you distribute.
}
\litem{
Choose the \textbf{smallest} set of Complex numbers that the number below belongs to.
\[ -\sqrt{\frac{660}{6}}+5i^2 \]The solution is \( \text{Irrational} \), which is option B.\begin{enumerate}[label=\Alph*.]
\item \( \text{Nonreal Complex} \)

This is a Complex number $(a+bi)$ that is not Real (has $i$ as part of the number).
\item \( \text{Irrational} \)

* This is the correct option!
\item \( \text{Pure Imaginary} \)

This is a Complex number $(a+bi)$ that \textbf{only} has an imaginary part like $2i$.
\item \( \text{Not a Complex Number} \)

This is not a number. The only non-Complex number we know is dividing by 0 as this is not a number!
\item \( \text{Rational} \)

These are numbers that can be written as fraction of Integers (e.g., -2/3 + 5)
\end{enumerate}

\textbf{General Comment:} Be sure to simplify $i^2 = -1$. This may remove the imaginary portion for your number. If you are having trouble, you may want to look at the \textit{Subgroups of the Real Numbers} section.
}
\litem{
Simplify the expression below into the form $a+bi$. Then, choose the intervals that $a$ and $b$ belong to.
\[ \frac{72 - 55 i}{-1 - 2 i} \]The solution is \( 7.60  + 39.80 i \), which is option C.\begin{enumerate}[label=\Alph*.]
\item \( a \in [37.5, 39] \text{ and } b \in [39.5, 40.5] \)

 $38.00  + 39.80 i$, which corresponds to forgetting to multiply the conjugate by the numerator and using a plus instead of a minus in the denominator.
\item \( a \in [6.5, 8] \text{ and } b \in [198.5, 199.5] \)

 $7.60  + 199.00 i$, which corresponds to forgetting to multiply the conjugate by the numerator.
\item \( a \in [6.5, 8] \text{ and } b \in [39.5, 40.5] \)

* $7.60  + 39.80 i$, which is the correct option.
\item \( a \in [-72.5, -70.5] \text{ and } b \in [27, 28.5] \)

 $-72.00  + 27.50 i$, which corresponds to just dividing the first term by the first term and the second by the second.
\item \( a \in [-37, -35] \text{ and } b \in [-19, -17] \)

 $-36.40  - 17.80 i$, which corresponds to forgetting to multiply the conjugate by the numerator and not computing the conjugate correctly.
\end{enumerate}

\textbf{General Comment:} Multiply the numerator and denominator by the *conjugate* of the denominator, then simplify. For example, if we have $2+3i$, the conjugate is $2-3i$.
}
\litem{
Choose the \textbf{smallest} set of Complex numbers that the number below belongs to.
\[ \frac{11}{5}+\sqrt{132} i \]The solution is \( \text{Nonreal Complex} \), which is option D.\begin{enumerate}[label=\Alph*.]
\item \( \text{Irrational} \)

These cannot be written as a fraction of Integers. Remember: $\pi$ is not an Integer!
\item \( \text{Rational} \)

These are numbers that can be written as fraction of Integers (e.g., -2/3 + 5)
\item \( \text{Pure Imaginary} \)

This is a Complex number $(a+bi)$ that \textbf{only} has an imaginary part like $2i$.
\item \( \text{Nonreal Complex} \)

* This is the correct option!
\item \( \text{Not a Complex Number} \)

This is not a number. The only non-Complex number we know is dividing by 0 as this is not a number!
\end{enumerate}

\textbf{General Comment:} Be sure to simplify $i^2 = -1$. This may remove the imaginary portion for your number. If you are having trouble, you may want to look at the \textit{Subgroups of the Real Numbers} section.
}
\end{enumerate}

\end{document}