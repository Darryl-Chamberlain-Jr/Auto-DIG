\documentclass[14pt]{extbook}
\usepackage{multicol, enumerate, enumitem, hyperref, color, soul, setspace, parskip, fancyhdr} %General Packages
\usepackage{amssymb, amsthm, amsmath, latexsym, units, mathtools} %Math Packages
\everymath{\displaystyle} %All math in Display Style
% Packages with additional options
\usepackage[headsep=0.5cm,headheight=12pt, left=1 in,right= 1 in,top= 1 in,bottom= 1 in]{geometry}
\usepackage[usenames,dvipsnames]{xcolor}
\usepackage{dashrule}  % Package to use the command below to create lines between items
\newcommand{\litem}[1]{\item#1\hspace*{-1cm}\rule{\textwidth}{0.4pt}}
\pagestyle{fancy}
\lhead{Makeup Progress Quiz 2}
\chead{}
\rhead{Version C}
\lfoot{5763-3522}
\cfoot{}
\rfoot{Spring 2021}
\begin{document}

\begin{enumerate}
\litem{
Factor the polynomial below completely. Then, choose the intervals the zeros of the polynomial belong to, where $z_1 \leq z_2 \leq z_3$. \textit{To make the problem easier, all zeros are between -5 and 5.}\[ f(x) = 25x^{3} +75 x^{2} -4 x -12 \]\begin{enumerate}[label=\Alph*.]
\item \( z_1 \in [-2.84, -2.05], \text{   }  z_2 \in [1.7, 3.9], \text{   and   } z_3 \in [2.72, 3.06] \)
\item \( z_1 \in [-3.41, -2.56], \text{   }  z_2 \in [-3.2, -1.3], \text{   and   } z_3 \in [2.25, 2.51] \)
\item \( z_1 \in [-2.21, -1.96], \text{   }  z_2 \in [-0.2, 0.2], \text{   and   } z_3 \in [2.72, 3.06] \)
\item \( z_1 \in [-3.41, -2.56], \text{   }  z_2 \in [-0.5, -0.1], \text{   and   } z_3 \in [0.26, 0.46] \)
\item \( z_1 \in [-0.46, 0.08], \text{   }  z_2 \in [0.3, 1.1], \text{   and   } z_3 \in [2.72, 3.06] \)

\end{enumerate} }
\litem{
Factor the polynomial below completely, knowing that $x+2$ is a factor. Then, choose the intervals the zeros of the polynomial belong to, where $z_1 \leq z_2 \leq z_3 \leq z_4$. \textit{To make the problem easier, all zeros are between -5 and 5.}\[ f(x) = 25x^{4} +180 x^{3} +388 x^{2} +288 x + 64 \]\begin{enumerate}[label=\Alph*.]
\item \( z_1 \in [-0.04, 0.08], \text{   }  z_2 \in [1.55, 2.66], z_3 \in [3.95, 4.18], \text{   and   } z_4 \in [3.5, 5.1] \)
\item \( z_1 \in [-4.54, -3.58], \text{   }  z_2 \in [-2.78, -2.16], z_3 \in [-2.04, -1.82], \text{   and   } z_4 \in [-2.2, -0.5] \)
\item \( z_1 \in [1.11, 1.5], \text{   }  z_2 \in [1.55, 2.66], z_3 \in [2.39, 2.53], \text{   and   } z_4 \in [3.5, 5.1] \)
\item \( z_1 \in [-4.54, -3.58], \text{   }  z_2 \in [-2.13, -1.95], z_3 \in [-0.92, -0.57], \text{   and   } z_4 \in [-1.2, -0.1] \)
\item \( z_1 \in [0.38, 0.61], \text{   }  z_2 \in [0.41, 1.1], z_3 \in [1.55, 2.03], \text{   and   } z_4 \in [3.5, 5.1] \)

\end{enumerate} }
\litem{
Factor the polynomial below completely, knowing that $x+3$ is a factor. Then, choose the intervals the zeros of the polynomial belong to, where $z_1 \leq z_2 \leq z_3 \leq z_4$. \textit{To make the problem easier, all zeros are between -5 and 5.}\[ f(x) = 15x^{4} -11 x^{3} -257 x^{2} -297 x -90 \]\begin{enumerate}[label=\Alph*.]
\item \( z_1 \in [-6.9, -4.6], \text{   }  z_2 \in [1.09, 1.71], z_3 \in [1.66, 1.72], \text{   and   } z_4 \in [2.21, 3.32] \)
\item \( z_1 \in [-4.8, -1.9], \text{   }  z_2 \in [-1.79, -1.13], z_3 \in [-1.52, -1.34], \text{   and   } z_4 \in [4.95, 5.86] \)
\item \( z_1 \in [-6.9, -4.6], \text{   }  z_2 \in [-0.33, 0.51], z_3 \in [1.93, 2.22], \text{   and   } z_4 \in [2.21, 3.32] \)
\item \( z_1 \in [-6.9, -4.6], \text{   }  z_2 \in [0.58, 0.73], z_3 \in [0.49, 0.9], \text{   and   } z_4 \in [2.21, 3.32] \)
\item \( z_1 \in [-4.8, -1.9], \text{   }  z_2 \in [-0.68, -0.15], z_3 \in [-0.62, -0.59], \text{   and   } z_4 \in [4.95, 5.86] \)

\end{enumerate} }
\litem{
Perform the division below. Then, find the intervals that correspond to the quotient in the form $ax^2+bx+c$ and remainder $r$.\[ \frac{4x^{3} -49 x -57}{x -4} \]\begin{enumerate}[label=\Alph*.]
\item \( a \in [14, 18], b \in [60, 65], c \in [205, 208], \text{ and } r \in [768, 774]. \)
\item \( a \in [-3, 6], b \in [12, 15], c \in [-14, -8], \text{ and } r \in [-96, -95]. \)
\item \( a \in [-3, 6], b \in [-17, -14], c \in [12, 17], \text{ and } r \in [-119, -112]. \)
\item \( a \in [-3, 6], b \in [14, 24], c \in [12, 17], \text{ and } r \in [-2, 7]. \)
\item \( a \in [14, 18], b \in [-68, -58], c \in [205, 208], \text{ and } r \in [-885, -878]. \)

\end{enumerate} }
\litem{
Perform the division below. Then, find the intervals that correspond to the quotient in the form $ax^2+bx+c$ and remainder $r$.\[ \frac{20x^{3} -101 x^{2} -7 x + 65}{x -5} \]\begin{enumerate}[label=\Alph*.]
\item \( a \in [17, 21], \text{   } b \in [-5, 1], \text{   } c \in [-13, -2], \text{   and   } r \in [4, 6]. \)
\item \( a \in [100, 101], \text{   } b \in [395, 400], \text{   } c \in [1987, 1989], \text{   and   } r \in [10005, 10013]. \)
\item \( a \in [17, 21], \text{   } b \in [-22, -15], \text{   } c \in [-92, -88], \text{   and   } r \in [-303, -298]. \)
\item \( a \in [100, 101], \text{   } b \in [-606, -597], \text{   } c \in [2997, 3000], \text{   and   } r \in [-14929, -14923]. \)
\item \( a \in [17, 21], \text{   } b \in [-204, -198], \text{   } c \in [997, 1006], \text{   and   } r \in [-4928, -4921]. \)

\end{enumerate} }
\litem{
Factor the polynomial below completely. Then, choose the intervals the zeros of the polynomial belong to, where $z_1 \leq z_2 \leq z_3$. \textit{To make the problem easier, all zeros are between -5 and 5.}\[ f(x) = 12x^{3} -83 x^{2} +125 x -50 \]\begin{enumerate}[label=\Alph*.]
\item \( z_1 \in [-5.23, -4.58], \text{   }  z_2 \in [-1.75, -1.37], \text{   and   } z_3 \in [-1.33, -0.79] \)
\item \( z_1 \in [0.57, 0.77], \text{   }  z_2 \in [1.1, 1.45], \text{   and   } z_3 \in [4.52, 5.11] \)
\item \( z_1 \in [0.78, 0.9], \text{   }  z_2 \in [1.4, 1.6], \text{   and   } z_3 \in [4.52, 5.11] \)
\item \( z_1 \in [-5.23, -4.58], \text{   }  z_2 \in [-2.04, -1.76], \text{   and   } z_3 \in [-0.58, -0.18] \)
\item \( z_1 \in [-5.23, -4.58], \text{   }  z_2 \in [-1.33, -1.13], \text{   and   } z_3 \in [-0.67, -0.53] \)

\end{enumerate} }
\litem{
What are the \textit{possible Rational} roots of the polynomial below?\[ f(x) = 7x^{3} +5 x^{2} +5 x + 6 \]\begin{enumerate}[label=\Alph*.]
\item \( \text{ All combinations of: }\frac{\pm 1,\pm 2,\pm 3,\pm 6}{\pm 1,\pm 7} \)
\item \( \pm 1,\pm 2,\pm 3,\pm 6 \)
\item \( \text{ All combinations of: }\frac{\pm 1,\pm 7}{\pm 1,\pm 2,\pm 3,\pm 6} \)
\item \( \pm 1,\pm 7 \)
\item \( \text{ There is no formula or theorem that tells us all possible Rational roots.} \)

\end{enumerate} }
\litem{
Perform the division below. Then, find the intervals that correspond to the quotient in the form $ax^2+bx+c$ and remainder $r$.\[ \frac{15x^{3} -45 x + 28}{x + 2} \]\begin{enumerate}[label=\Alph*.]
\item \( a \in [15, 16], b \in [28, 32], c \in [14, 16], \text{ and } r \in [51, 60]. \)
\item \( a \in [15, 16], b \in [-36, -29], c \in [14, 16], \text{ and } r \in [-3, 1]. \)
\item \( a \in [-34, -23], b \in [58, 66], c \in [-166, -161], \text{ and } r \in [350, 360]. \)
\item \( a \in [15, 16], b \in [-45, -43], c \in [87, 97], \text{ and } r \in [-243, -239]. \)
\item \( a \in [-34, -23], b \in [-61, -58], c \in [-166, -161], \text{ and } r \in [-304, -300]. \)

\end{enumerate} }
\litem{
Perform the division below. Then, find the intervals that correspond to the quotient in the form $ax^2+bx+c$ and remainder $r$.\[ \frac{10x^{3} +16 x^{2} -38 x + 10}{x + 3} \]\begin{enumerate}[label=\Alph*.]
\item \( a \in [9, 11], \text{   } b \in [45, 47], \text{   } c \in [97, 101], \text{   and   } r \in [305, 312]. \)
\item \( a \in [9, 11], \text{   } b \in [-18, -12], \text{   } c \in [3, 6], \text{   and   } r \in [-7, 0]. \)
\item \( a \in [9, 11], \text{   } b \in [-25, -20], \text{   } c \in [53, 59], \text{   and   } r \in [-225, -218]. \)
\item \( a \in [-33, -24], \text{   } b \in [-77, -68], \text{   } c \in [-264, -258], \text{   and   } r \in [-773, -769]. \)
\item \( a \in [-33, -24], \text{   } b \in [103, 107], \text{   } c \in [-356, -350], \text{   and   } r \in [1077, 1083]. \)

\end{enumerate} }
\litem{
What are the \textit{possible Rational} roots of the polynomial below?\[ f(x) = 5x^{2} +2 x + 2 \]\begin{enumerate}[label=\Alph*.]
\item \( \pm 1,\pm 5 \)
\item \( \text{ All combinations of: }\frac{\pm 1,\pm 5}{\pm 1,\pm 2} \)
\item \( \pm 1,\pm 2 \)
\item \( \text{ All combinations of: }\frac{\pm 1,\pm 2}{\pm 1,\pm 5} \)
\item \( \text{ There is no formula or theorem that tells us all possible Rational roots.} \)

\end{enumerate} }
\end{enumerate}

\end{document}