\documentclass{extbook}[14pt]
\usepackage{multicol, enumerate, enumitem, hyperref, color, soul, setspace, parskip, fancyhdr, amssymb, amsthm, amsmath, latexsym, units, mathtools}
\everymath{\displaystyle}
\usepackage[headsep=0.5cm,headheight=0cm, left=1 in,right= 1 in,top= 1 in,bottom= 1 in]{geometry}
\usepackage{dashrule}  % Package to use the command below to create lines between items
\newcommand{\litem}[1]{\item #1

\rule{\textwidth}{0.4pt}}
\pagestyle{fancy}
\lhead{}
\chead{Answer Key for Makeup Progress Quiz 2 Version ALL}
\rhead{}
\lfoot{2790-1423}
\cfoot{}
\rfoot{Summer C 2021}
\begin{document}
\textbf{This key should allow you to understand why you choose the option you did (beyond just getting a question right or wrong). \href{https://xronos.clas.ufl.edu/mac1105spring2020/courseDescriptionAndMisc/Exams/LearningFromResults}{More instructions on how to use this key can be found here}.}

\textbf{If you have a suggestion to make the keys better, \href{https://forms.gle/CZkbZmPbC9XALEE88}{please fill out the short survey here}.}

\textit{Note: This key is auto-generated and may contain issues and/or errors. The keys are reviewed after each exam to ensure grading is done accurately. If there are issues (like duplicate options), they are noted in the offline gradebook. The keys are a work-in-progress to give students as many resources to improve as possible.}

\rule{\textwidth}{0.4pt}

\begin{enumerate}\litem{
What are the \textit{possible Rational} roots of the polynomial below?
\[ f(x) = 7x^{4} +2 x^{3} +2 x^{2} +2 x + 6 \]The solution is \( \text{ All combinations of: }\frac{\pm 1,\pm 2,\pm 3,\pm 6}{\pm 1,\pm 7} \), which is option C.\begin{enumerate}[label=\Alph*.]
\item \( \pm 1,\pm 7 \)

 Distractor 1: Corresponds to the plus or minus factors of a1 only.
\item \( \text{ All combinations of: }\frac{\pm 1,\pm 7}{\pm 1,\pm 2,\pm 3,\pm 6} \)

 Distractor 3: Corresponds to the plus or minus of the inverse quotient (an/a0) of the factors. 
\item \( \text{ All combinations of: }\frac{\pm 1,\pm 2,\pm 3,\pm 6}{\pm 1,\pm 7} \)

* This is the solution \textbf{since we asked for the possible Rational roots}!
\item \( \pm 1,\pm 2,\pm 3,\pm 6 \)

This would have been the solution \textbf{if asked for the possible Integer roots}!
\item \( \text{ There is no formula or theorem that tells us all possible Rational roots.} \)

 Distractor 4: Corresponds to not recalling the theorem for rational roots of a polynomial.
\end{enumerate}

\textbf{General Comment:} We have a way to find the possible Rational roots. The possible Integer roots are the Integers in this list.
}
\litem{
What are the \textit{possible Integer} roots of the polynomial below?
\[ f(x) = 6x^{4} +6 x^{3} +3 x^{2} +2 x + 4 \]The solution is \( \pm 1,\pm 2,\pm 4 \), which is option A.\begin{enumerate}[label=\Alph*.]
\item \( \pm 1,\pm 2,\pm 4 \)

* This is the solution \textbf{since we asked for the possible Integer roots}!
\item \( \text{ All combinations of: }\frac{\pm 1,\pm 2,\pm 3,\pm 6}{\pm 1,\pm 2,\pm 4} \)

 Distractor 3: Corresponds to the plus or minus of the inverse quotient (an/a0) of the factors. 
\item \( \text{ All combinations of: }\frac{\pm 1,\pm 2,\pm 4}{\pm 1,\pm 2,\pm 3,\pm 6} \)

This would have been the solution \textbf{if asked for the possible Rational roots}!
\item \( \pm 1,\pm 2,\pm 3,\pm 6 \)

 Distractor 1: Corresponds to the plus or minus factors of a1 only.
\item \( \text{There is no formula or theorem that tells us all possible Integer roots.} \)

 Distractor 4: Corresponds to not recognizing Integers as a subset of Rationals.
\end{enumerate}

\textbf{General Comment:} We have a way to find the possible Rational roots. The possible Integer roots are the Integers in this list.
}
\litem{
Factor the polynomial below completely, knowing that $x + 5$ is a factor. Then, choose the intervals the zeros of the polynomial belong to, where $z_1 \leq z_2 \leq z_3 \leq z_4$. \textit{To make the problem easier, all zeros are between -5 and 5.}
\[ f(x) = 8x^{4} -14 x^{3} -167 x^{2} +455 x -300 \]The solution is \( [-5, 1.25, 1.5, 4] \), which is option B.\begin{enumerate}[label=\Alph*.]
\item \( z_1 \in [-4.25, -3.95], \text{   }  z_2 \in [-0.88, -0.53], z_3 \in [-0.67, -0.65], \text{   and   } z_4 \in [4.9, 5.1] \)

 Distractor 3: Corresponds to negatives of all zeros AND inversing rational roots.
\item \( z_1 \in [-5.08, -4.64], \text{   }  z_2 \in [0.79, 1.43], z_3 \in [1.47, 1.53], \text{   and   } z_4 \in [2.5, 4.4] \)

* This is the solution!
\item \( z_1 \in [-5.08, -4.64], \text{   }  z_2 \in [-0.31, 0.74], z_3 \in [0.76, 0.84], \text{   and   } z_4 \in [2.5, 4.4] \)

 Distractor 2: Corresponds to inversing rational roots.
\item \( z_1 \in [-4.25, -3.95], \text{   }  z_2 \in [-3.36, -2.87], z_3 \in [-0.63, -0.53], \text{   and   } z_4 \in [4.9, 5.1] \)

 Distractor 4: Corresponds to moving factors from one rational to another.
\item \( z_1 \in [-4.25, -3.95], \text{   }  z_2 \in [-2.41, -0.84], z_3 \in [-1.26, -1.22], \text{   and   } z_4 \in [4.9, 5.1] \)

 Distractor 1: Corresponds to negatives of all zeros.
\end{enumerate}

\textbf{General Comment:} Remember to try the middle-most integers first as these normally are the zeros. Also, once you get it to a quadratic, you can use your other factoring techniques to finish factoring.
}
\litem{
Perform the division below. Then, find the intervals that correspond to the quotient in the form $ax^2+bx+c$ and remainder $r$.
\[ \frac{4x^{3} -12 x + 6}{x + 2} \]The solution is \( 4x^{2} -8 x + 4 + \frac{-2}{x + 2} \), which is option D.\begin{enumerate}[label=\Alph*.]
\item \( a \in [3, 8], b \in [-13, -10], c \in [15, 25], \text{ and } r \in [-67, -61]. \)

 You multipled by the synthetic number and subtracted rather than adding during synthetic division.
\item \( a \in [3, 8], b \in [8, 10], c \in [-1, 8], \text{ and } r \in [8, 15]. \)

 You divided by the opposite of the factor.
\item \( a \in [-10, -4], b \in [10, 17], c \in [-48, -42], \text{ and } r \in [94, 97]. \)

 You multipled by the synthetic number rather than bringing the first factor down.
\item \( a \in [3, 8], b \in [-9, 0], c \in [-1, 8], \text{ and } r \in [-5, 4]. \)

* This is the solution!
\item \( a \in [-10, -4], b \in [-20, -15], c \in [-48, -42], \text{ and } r \in [-85, -81]. \)

 You divided by the opposite of the factor AND multipled the first factor rather than just bringing it down.
\end{enumerate}

\textbf{General Comment:} Be sure to synthetically divide by the zero of the denominator! Also, make sure to include 0 placeholders for missing terms.
}
\litem{
Perform the division below. Then, find the intervals that correspond to the quotient in the form $ax^2+bx+c$ and remainder $r$.
\[ \frac{20x^{3} -63 x^{2} + 23}{x -3} \]The solution is \( 20x^{2} -3 x -9 + \frac{-4}{x -3} \), which is option D.\begin{enumerate}[label=\Alph*.]
\item \( a \in [57, 65], b \in [113, 120], c \in [350, 355], \text{ and } r \in [1074, 1078]. \)

 You multipled by the synthetic number rather than bringing the first factor down.
\item \( a \in [17, 22], b \in [-130, -118], c \in [369, 371], \text{ and } r \in [-1085, -1082]. \)

 You divided by the opposite of the factor.
\item \( a \in [57, 65], b \in [-245, -241], c \in [729, 731], \text{ and } r \in [-2169, -2161]. \)

 You divided by the opposite of the factor AND multipled the first factor rather than just bringing it down.
\item \( a \in [17, 22], b \in [-5, 0], c \in [-13, -7], \text{ and } r \in [-6, 4]. \)

* This is the solution!
\item \( a \in [17, 22], b \in [-29, -22], c \in [-47, -42], \text{ and } r \in [-70, -68]. \)

 You multipled by the synthetic number and subtracted rather than adding during synthetic division.
\end{enumerate}

\textbf{General Comment:} Be sure to synthetically divide by the zero of the denominator! Also, make sure to include 0 placeholders for missing terms.
}
\litem{
Factor the polynomial below completely, knowing that $x -5$ is a factor. Then, choose the intervals the zeros of the polynomial belong to, where $z_1 \leq z_2 \leq z_3 \leq z_4$. \textit{To make the problem easier, all zeros are between -5 and 5.}
\[ f(x) = 12x^{4} -113 x^{3} +338 x^{2} -395 x + 150 \]The solution is \( [0.75, 1.667, 2, 5] \), which is option C.\begin{enumerate}[label=\Alph*.]
\item \( z_1 \in [-5.35, -4.91], \text{   }  z_2 \in [-6.33, -4.95], z_3 \in [-2.42, -1.69], \text{   and   } z_4 \in [-0.34, -0.14] \)

 Distractor 4: Corresponds to moving factors from one rational to another.
\item \( z_1 \in [-5.35, -4.91], \text{   }  z_2 \in [-2.13, -1.6], z_3 \in [-1.8, -1.62], \text{   and   } z_4 \in [-0.84, -0.74] \)

 Distractor 1: Corresponds to negatives of all zeros.
\item \( z_1 \in [0.72, 1.07], \text{   }  z_2 \in [1.35, 1.95], z_3 \in [1.61, 2.61], \text{   and   } z_4 \in [4.79, 5.08] \)

* This is the solution!
\item \( z_1 \in [0.59, 0.69], \text{   }  z_2 \in [1.11, 1.62], z_3 \in [1.61, 2.61], \text{   and   } z_4 \in [4.79, 5.08] \)

 Distractor 2: Corresponds to inversing rational roots.
\item \( z_1 \in [-5.35, -4.91], \text{   }  z_2 \in [-2.13, -1.6], z_3 \in [-1.44, -1.27], \text{   and   } z_4 \in [-0.68, -0.44] \)

 Distractor 3: Corresponds to negatives of all zeros AND inversing rational roots.
\end{enumerate}

\textbf{General Comment:} Remember to try the middle-most integers first as these normally are the zeros. Also, once you get it to a quadratic, you can use your other factoring techniques to finish factoring.
}
\litem{
Factor the polynomial below completely. Then, choose the intervals the zeros of the polynomial belong to, where $z_1 \leq z_2 \leq z_3$. \textit{To make the problem easier, all zeros are between -5 and 5.}
\[ f(x) = 25x^{3} -100 x^{2} -4 x + 16 \]The solution is \( [-0.4, 0.4, 4] \), which is option D.\begin{enumerate}[label=\Alph*.]
\item \( z_1 \in [-4.6, -3.3], \text{   }  z_2 \in [-2.65, -2.36], \text{   and   } z_3 \in [2.09, 3] \)

 Distractor 3: Corresponds to negatives of all zeros AND inversing rational roots.
\item \( z_1 \in [-2.9, -2.4], \text{   }  z_2 \in [1.93, 2.94], \text{   and   } z_3 \in [3.93, 4.24] \)

 Distractor 2: Corresponds to inversing rational roots.
\item \( z_1 \in [-4.6, -3.3], \text{   }  z_2 \in [-2.25, -1.76], \text{   and   } z_3 \in [-0.2, 0.15] \)

 Distractor 4: Corresponds to moving factors from one rational to another.
\item \( z_1 \in [-1.6, 0.4], \text{   }  z_2 \in [0.19, 0.79], \text{   and   } z_3 \in [3.93, 4.24] \)

* This is the solution!
\item \( z_1 \in [-4.6, -3.3], \text{   }  z_2 \in [-0.56, -0.04], \text{   and   } z_3 \in [0.11, 1.11] \)

 Distractor 1: Corresponds to negatives of all zeros.
\end{enumerate}

\textbf{General Comment:} Remember to try the middle-most integers first as these normally are the zeros. Also, once you get it to a quadratic, you can use your other factoring techniques to finish factoring.
}
\litem{
Factor the polynomial below completely. Then, choose the intervals the zeros of the polynomial belong to, where $z_1 \leq z_2 \leq z_3$. \textit{To make the problem easier, all zeros are between -5 and 5.}
\[ f(x) = 20x^{3} +31 x^{2} -38 x -40 \]The solution is \( [-2, -0.8, 1.25] \), which is option E.\begin{enumerate}[label=\Alph*.]
\item \( z_1 \in [-0.92, -0.65], \text{   }  z_2 \in [1.12, 1.47], \text{   and   } z_3 \in [1.78, 2.59] \)

 Distractor 3: Corresponds to negatives of all zeros AND inversing rational roots.
\item \( z_1 \in [-5.08, -4.9], \text{   }  z_2 \in [-0.09, 0.27], \text{   and   } z_3 \in [1.78, 2.59] \)

 Distractor 4: Corresponds to moving factors from one rational to another.
\item \( z_1 \in [-1.46, -1.02], \text{   }  z_2 \in [0.78, 1.08], \text{   and   } z_3 \in [1.78, 2.59] \)

 Distractor 1: Corresponds to negatives of all zeros.
\item \( z_1 \in [-2.2, -1.8], \text{   }  z_2 \in [-1.59, -1.18], \text{   and   } z_3 \in [0.41, 0.86] \)

 Distractor 2: Corresponds to inversing rational roots.
\item \( z_1 \in [-2.2, -1.8], \text{   }  z_2 \in [-0.86, -0.52], \text{   and   } z_3 \in [1.1, 1.49] \)

* This is the solution!
\end{enumerate}

\textbf{General Comment:} Remember to try the middle-most integers first as these normally are the zeros. Also, once you get it to a quadratic, you can use your other factoring techniques to finish factoring.
}
\litem{
Perform the division below. Then, find the intervals that correspond to the quotient in the form $ax^2+bx+c$ and remainder $r$.
\[ \frac{6x^{3} -2 x^{2} -20 x + 19}{x + 2} \]The solution is \( 6x^{2} -14 x + 8 + \frac{3}{x + 2} \), which is option E.\begin{enumerate}[label=\Alph*.]
\item \( a \in [-15, -8], \text{   } b \in [-28, -25], \text{   } c \in [-72, -68], \text{   and   } r \in [-132, -124]. \)

 You divided by the opposite of the factor AND multiplied the first factor rather than just bringing it down.
\item \( a \in [-15, -8], \text{   } b \in [22, 24], \text{   } c \in [-66, -63], \text{   and   } r \in [144, 149]. \)

 You multiplied by the synthetic number rather than bringing the first factor down.
\item \( a \in [1, 11], \text{   } b \in [-21, -19], \text{   } c \in [34, 46], \text{   and   } r \in [-103, -97]. \)

 You multiplied by the synthetic number and subtracted rather than adding during synthetic division.
\item \( a \in [1, 11], \text{   } b \in [8, 17], \text{   } c \in [-3, 4], \text{   and   } r \in [15, 20]. \)

 You divided by the opposite of the factor.
\item \( a \in [1, 11], \text{   } b \in [-14, -9], \text{   } c \in [7, 9], \text{   and   } r \in [2, 4]. \)

* This is the solution!
\end{enumerate}

\textbf{General Comment:} Be sure to synthetically divide by the zero of the denominator!
}
\litem{
Perform the division below. Then, find the intervals that correspond to the quotient in the form $ax^2+bx+c$ and remainder $r$.
\[ \frac{6x^{3} -20 x^{2} -2 x + 19}{x -3} \]The solution is \( 6x^{2} -2 x -8 + \frac{-5}{x -3} \), which is option D.\begin{enumerate}[label=\Alph*.]
\item \( a \in [5, 9], \text{   } b \in [-40, -34], \text{   } c \in [111, 115], \text{   and   } r \in [-322, -313]. \)

 You divided by the opposite of the factor.
\item \( a \in [15, 21], \text{   } b \in [-76, -72], \text{   } c \in [218, 224], \text{   and   } r \in [-646, -637]. \)

 You divided by the opposite of the factor AND multiplied the first factor rather than just bringing it down.
\item \( a \in [5, 9], \text{   } b \in [-14, -3], \text{   } c \in [-23, -16], \text{   and   } r \in [-18, -15]. \)

 You multiplied by the synthetic number and subtracted rather than adding during synthetic division.
\item \( a \in [5, 9], \text{   } b \in [-6, 6], \text{   } c \in [-15, -7], \text{   and   } r \in [-6, -3]. \)

* This is the solution!
\item \( a \in [15, 21], \text{   } b \in [29, 35], \text{   } c \in [95, 104], \text{   and   } r \in [318, 321]. \)

 You multiplied by the synthetic number rather than bringing the first factor down.
\end{enumerate}

\textbf{General Comment:} Be sure to synthetically divide by the zero of the denominator!
}
\litem{
What are the \textit{possible Rational} roots of the polynomial below?
\[ f(x) = 7x^{3} +7 x^{2} +3 x + 5 \]The solution is \( \text{ All combinations of: }\frac{\pm 1,\pm 5}{\pm 1,\pm 7} \), which is option A.\begin{enumerate}[label=\Alph*.]
\item \( \text{ All combinations of: }\frac{\pm 1,\pm 5}{\pm 1,\pm 7} \)

* This is the solution \textbf{since we asked for the possible Rational roots}!
\item \( \pm 1,\pm 7 \)

 Distractor 1: Corresponds to the plus or minus factors of a1 only.
\item \( \pm 1,\pm 5 \)

This would have been the solution \textbf{if asked for the possible Integer roots}!
\item \( \text{ All combinations of: }\frac{\pm 1,\pm 7}{\pm 1,\pm 5} \)

 Distractor 3: Corresponds to the plus or minus of the inverse quotient (an/a0) of the factors. 
\item \( \text{ There is no formula or theorem that tells us all possible Rational roots.} \)

 Distractor 4: Corresponds to not recalling the theorem for rational roots of a polynomial.
\end{enumerate}

\textbf{General Comment:} We have a way to find the possible Rational roots. The possible Integer roots are the Integers in this list.
}
\litem{
What are the \textit{possible Rational} roots of the polynomial below?
\[ f(x) = 4x^{4} +3 x^{3} +3 x^{2} +3 x + 6 \]The solution is \( \text{ All combinations of: }\frac{\pm 1,\pm 2,\pm 3,\pm 6}{\pm 1,\pm 2,\pm 4} \), which is option A.\begin{enumerate}[label=\Alph*.]
\item \( \text{ All combinations of: }\frac{\pm 1,\pm 2,\pm 3,\pm 6}{\pm 1,\pm 2,\pm 4} \)

* This is the solution \textbf{since we asked for the possible Rational roots}!
\item \( \text{ All combinations of: }\frac{\pm 1,\pm 2,\pm 4}{\pm 1,\pm 2,\pm 3,\pm 6} \)

 Distractor 3: Corresponds to the plus or minus of the inverse quotient (an/a0) of the factors. 
\item \( \pm 1,\pm 2,\pm 4 \)

 Distractor 1: Corresponds to the plus or minus factors of a1 only.
\item \( \pm 1,\pm 2,\pm 3,\pm 6 \)

This would have been the solution \textbf{if asked for the possible Integer roots}!
\item \( \text{ There is no formula or theorem that tells us all possible Rational roots.} \)

 Distractor 4: Corresponds to not recalling the theorem for rational roots of a polynomial.
\end{enumerate}

\textbf{General Comment:} We have a way to find the possible Rational roots. The possible Integer roots are the Integers in this list.
}
\litem{
Factor the polynomial below completely, knowing that $x + 5$ is a factor. Then, choose the intervals the zeros of the polynomial belong to, where $z_1 \leq z_2 \leq z_3 \leq z_4$. \textit{To make the problem easier, all zeros are between -5 and 5.}
\[ f(x) = 9x^{4} +27 x^{3} -127 x^{2} -155 x + 150 \]The solution is \( [-5, -1.667, 0.667, 3] \), which is option A.\begin{enumerate}[label=\Alph*.]
\item \( z_1 \in [-5.2, -4.9], \text{   }  z_2 \in [-1.81, -1.57], z_3 \in [0.65, 0.78], \text{   and   } z_4 \in [1.9, 4.5] \)

* This is the solution!
\item \( z_1 \in [-5.2, -4.9], \text{   }  z_2 \in [-0.62, -0.5], z_3 \in [1.46, 1.56], \text{   and   } z_4 \in [1.9, 4.5] \)

 Distractor 2: Corresponds to inversing rational roots.
\item \( z_1 \in [-4.5, -1.9], \text{   }  z_2 \in [-0.8, -0.62], z_3 \in [1.62, 1.74], \text{   and   } z_4 \in [3.6, 5.7] \)

 Distractor 1: Corresponds to negatives of all zeros.
\item \( z_1 \in [-4.5, -1.9], \text{   }  z_2 \in [-0.25, -0.17], z_3 \in [4.88, 5.04], \text{   and   } z_4 \in [3.6, 5.7] \)

 Distractor 4: Corresponds to moving factors from one rational to another.
\item \( z_1 \in [-4.5, -1.9], \text{   }  z_2 \in [-1.51, -1.43], z_3 \in [0.59, 0.62], \text{   and   } z_4 \in [3.6, 5.7] \)

 Distractor 3: Corresponds to negatives of all zeros AND inversing rational roots.
\end{enumerate}

\textbf{General Comment:} Remember to try the middle-most integers first as these normally are the zeros. Also, once you get it to a quadratic, you can use your other factoring techniques to finish factoring.
}
\litem{
Perform the division below. Then, find the intervals that correspond to the quotient in the form $ax^2+bx+c$ and remainder $r$.
\[ \frac{20x^{3} +62 x^{2} -16}{x + 3} \]The solution is \( 20x^{2} +2 x -6 + \frac{2}{x + 3} \), which is option C.\begin{enumerate}[label=\Alph*.]
\item \( a \in [-60, -57], b \in [242, 243], c \in [-730, -721], \text{ and } r \in [2161, 2168]. \)

 You multipled by the synthetic number rather than bringing the first factor down.
\item \( a \in [-60, -57], b \in [-119, -110], c \in [-354, -349], \text{ and } r \in [-1080, -1074]. \)

 You divided by the opposite of the factor AND multipled the first factor rather than just bringing it down.
\item \( a \in [18, 23], b \in [-2, 7], c \in [-13, -1], \text{ and } r \in [-2, 3]. \)

* This is the solution!
\item \( a \in [18, 23], b \in [-18, -15], c \in [70, 76], \text{ and } r \in [-311, -303]. \)

 You multipled by the synthetic number and subtracted rather than adding during synthetic division.
\item \( a \in [18, 23], b \in [120, 128], c \in [364, 373], \text{ and } r \in [1076, 1088]. \)

 You divided by the opposite of the factor.
\end{enumerate}

\textbf{General Comment:} Be sure to synthetically divide by the zero of the denominator! Also, make sure to include 0 placeholders for missing terms.
}
\litem{
Perform the division below. Then, find the intervals that correspond to the quotient in the form $ax^2+bx+c$ and remainder $r$.
\[ \frac{10x^{3} -30 x^{2} + 43}{x -2} \]The solution is \( 10x^{2} -10 x -20 + \frac{3}{x -2} \), which is option D.\begin{enumerate}[label=\Alph*.]
\item \( a \in [16, 23], b \in [9, 11], c \in [12, 29], \text{ and } r \in [81, 88]. \)

 You multipled by the synthetic number rather than bringing the first factor down.
\item \( a \in [16, 23], b \in [-70, -67], c \in [139, 141], \text{ and } r \in [-239, -231]. \)

 You divided by the opposite of the factor AND multipled the first factor rather than just bringing it down.
\item \( a \in [6, 13], b \in [-20, -15], c \in [-21, -18], \text{ and } r \in [21, 24]. \)

 You multipled by the synthetic number and subtracted rather than adding during synthetic division.
\item \( a \in [6, 13], b \in [-19, -7], c \in [-21, -18], \text{ and } r \in [-1, 4]. \)

* This is the solution!
\item \( a \in [6, 13], b \in [-50, -48], c \in [95, 104], \text{ and } r \in [-158, -154]. \)

 You divided by the opposite of the factor.
\end{enumerate}

\textbf{General Comment:} Be sure to synthetically divide by the zero of the denominator! Also, make sure to include 0 placeholders for missing terms.
}
\litem{
Factor the polynomial below completely, knowing that $x + 4$ is a factor. Then, choose the intervals the zeros of the polynomial belong to, where $z_1 \leq z_2 \leq z_3 \leq z_4$. \textit{To make the problem easier, all zeros are between -5 and 5.}
\[ f(x) = 8x^{4} -10 x^{3} -101 x^{2} +238 x -120 \]The solution is \( [-4, 0.75, 2, 2.5] \), which is option B.\begin{enumerate}[label=\Alph*.]
\item \( z_1 \in [-2.15, -1.66], \text{   }  z_2 \in [-1.58, -1.33], z_3 \in [-0.52, -0.16], \text{   and   } z_4 \in [3.33, 4.3] \)

 Distractor 3: Corresponds to negatives of all zeros AND inversing rational roots.
\item \( z_1 \in [-4.34, -3.6], \text{   }  z_2 \in [0.46, 0.95], z_3 \in [1.97, 2.28], \text{   and   } z_4 \in [2.01, 2.65] \)

* This is the solution!
\item \( z_1 \in [-2.68, -2.27], \text{   }  z_2 \in [-2.16, -1.71], z_3 \in [-0.92, -0.7], \text{   and   } z_4 \in [3.33, 4.3] \)

 Distractor 1: Corresponds to negatives of all zeros.
\item \( z_1 \in [-4.34, -3.6], \text{   }  z_2 \in [0.37, 0.42], z_3 \in [1.07, 1.43], \text{   and   } z_4 \in [0.8, 2.23] \)

 Distractor 2: Corresponds to inversing rational roots.
\item \( z_1 \in [-3.35, -2.63], \text{   }  z_2 \in [-2.16, -1.71], z_3 \in [-0.64, -0.62], \text{   and   } z_4 \in [3.33, 4.3] \)

 Distractor 4: Corresponds to moving factors from one rational to another.
\end{enumerate}

\textbf{General Comment:} Remember to try the middle-most integers first as these normally are the zeros. Also, once you get it to a quadratic, you can use your other factoring techniques to finish factoring.
}
\litem{
Factor the polynomial below completely. Then, choose the intervals the zeros of the polynomial belong to, where $z_1 \leq z_2 \leq z_3$. \textit{To make the problem easier, all zeros are between -5 and 5.}
\[ f(x) = 20x^{3} -43 x^{2} -3 x + 18 \]The solution is \( [-0.6, 0.75, 2] \), which is option D.\begin{enumerate}[label=\Alph*.]
\item \( z_1 \in [-1.74, -1.55], \text{   }  z_2 \in [1.33, 1.4], \text{   and   } z_3 \in [1.74, 2.19] \)

 Distractor 2: Corresponds to inversing rational roots.
\item \( z_1 \in [-2.07, -1.88], \text{   }  z_2 \in [-0.9, -0.68], \text{   and   } z_3 \in [0.53, 0.61] \)

 Distractor 1: Corresponds to negatives of all zeros.
\item \( z_1 \in [-2.07, -1.88], \text{   }  z_2 \in [-0.47, 0.06], \text{   and   } z_3 \in [2.75, 3.01] \)

 Distractor 4: Corresponds to moving factors from one rational to another.
\item \( z_1 \in [-0.8, -0.48], \text{   }  z_2 \in [0.43, 0.86], \text{   and   } z_3 \in [1.74, 2.19] \)

* This is the solution!
\item \( z_1 \in [-2.07, -1.88], \text{   }  z_2 \in [-1.41, -1.21], \text{   and   } z_3 \in [1.07, 1.95] \)

 Distractor 3: Corresponds to negatives of all zeros AND inversing rational roots.
\end{enumerate}

\textbf{General Comment:} Remember to try the middle-most integers first as these normally are the zeros. Also, once you get it to a quadratic, you can use your other factoring techniques to finish factoring.
}
\litem{
Factor the polynomial below completely. Then, choose the intervals the zeros of the polynomial belong to, where $z_1 \leq z_2 \leq z_3$. \textit{To make the problem easier, all zeros are between -5 and 5.}
\[ f(x) = 25x^{3} +50 x^{2} -9 x -18 \]The solution is \( [-2, -0.6, 0.6] \), which is option E.\begin{enumerate}[label=\Alph*.]
\item \( z_1 \in [-3.1, -2.6], \text{   }  z_2 \in [0.06, 0.51], \text{   and   } z_3 \in [1.92, 2.63] \)

 Distractor 4: Corresponds to moving factors from one rational to another.
\item \( z_1 \in [-1.9, -0.7], \text{   }  z_2 \in [1.54, 1.86], \text{   and   } z_3 \in [1.92, 2.63] \)

 Distractor 3: Corresponds to negatives of all zeros AND inversing rational roots.
\item \( z_1 \in [-1.5, 0.1], \text{   }  z_2 \in [0.29, 1.22], \text{   and   } z_3 \in [1.92, 2.63] \)

 Distractor 1: Corresponds to negatives of all zeros.
\item \( z_1 \in [-2.1, -1.8], \text{   }  z_2 \in [-1.91, -1.44], \text{   and   } z_3 \in [1.3, 1.89] \)

 Distractor 2: Corresponds to inversing rational roots.
\item \( z_1 \in [-2.1, -1.8], \text{   }  z_2 \in [-1.34, -0.43], \text{   and   } z_3 \in [0.54, 0.97] \)

* This is the solution!
\end{enumerate}

\textbf{General Comment:} Remember to try the middle-most integers first as these normally are the zeros. Also, once you get it to a quadratic, you can use your other factoring techniques to finish factoring.
}
\litem{
Perform the division below. Then, find the intervals that correspond to the quotient in the form $ax^2+bx+c$ and remainder $r$.
\[ \frac{10x^{3} -29 x^{2} -50 x + 26}{x -4} \]The solution is \( 10x^{2} +11 x -6 + \frac{2}{x -4} \), which is option E.\begin{enumerate}[label=\Alph*.]
\item \( a \in [38, 42], \text{   } b \in [130, 134], \text{   } c \in [471, 482], \text{   and   } r \in [1920, 1925]. \)

 You multiplied by the synthetic number rather than bringing the first factor down.
\item \( a \in [5, 15], \text{   } b \in [-73, -61], \text{   } c \in [220, 231], \text{   and   } r \in [-883, -870]. \)

 You divided by the opposite of the factor.
\item \( a \in [38, 42], \text{   } b \in [-190, -187], \text{   } c \in [704, 707], \text{   and   } r \in [-2800, -2793]. \)

 You divided by the opposite of the factor AND multiplied the first factor rather than just bringing it down.
\item \( a \in [5, 15], \text{   } b \in [0, 4], \text{   } c \in [-47, -45], \text{   and   } r \in [-117, -112]. \)

 You multiplied by the synthetic number and subtracted rather than adding during synthetic division.
\item \( a \in [5, 15], \text{   } b \in [9, 14], \text{   } c \in [-13, -3], \text{   and   } r \in [0, 7]. \)

* This is the solution!
\end{enumerate}

\textbf{General Comment:} Be sure to synthetically divide by the zero of the denominator!
}
\litem{
Perform the division below. Then, find the intervals that correspond to the quotient in the form $ax^2+bx+c$ and remainder $r$.
\[ \frac{10x^{3} -85 x^{2} +200 x -129}{x -5} \]The solution is \( 10x^{2} -35 x + 25 + \frac{-4}{x -5} \), which is option C.\begin{enumerate}[label=\Alph*.]
\item \( a \in [49, 53], \text{   } b \in [-335, -331], \text{   } c \in [1872, 1879], \text{   and   } r \in [-9510, -9502]. \)

 You divided by the opposite of the factor AND multiplied the first factor rather than just bringing it down.
\item \( a \in [3, 11], \text{   } b \in [-136, -134], \text{   } c \in [875, 882], \text{   and   } r \in [-4504, -4494]. \)

 You divided by the opposite of the factor.
\item \( a \in [3, 11], \text{   } b \in [-41, -33], \text{   } c \in [25, 28], \text{   and   } r \in [-4, 1]. \)

* This is the solution!
\item \( a \in [49, 53], \text{   } b \in [164, 171], \text{   } c \in [1020, 1033], \text{   and   } r \in [4988, 5000]. \)

 You multiplied by the synthetic number rather than bringing the first factor down.
\item \( a \in [3, 11], \text{   } b \in [-45, -44], \text{   } c \in [18, 23], \text{   and   } r \in [-51, -48]. \)

 You multiplied by the synthetic number and subtracted rather than adding during synthetic division.
\end{enumerate}

\textbf{General Comment:} Be sure to synthetically divide by the zero of the denominator!
}
\litem{
What are the \textit{possible Rational} roots of the polynomial below?
\[ f(x) = 5x^{2} +3 x + 6 \]The solution is \( \text{ All combinations of: }\frac{\pm 1,\pm 2,\pm 3,\pm 6}{\pm 1,\pm 5} \), which is option D.\begin{enumerate}[label=\Alph*.]
\item \( \text{ All combinations of: }\frac{\pm 1,\pm 5}{\pm 1,\pm 2,\pm 3,\pm 6} \)

 Distractor 3: Corresponds to the plus or minus of the inverse quotient (an/a0) of the factors. 
\item \( \pm 1,\pm 2,\pm 3,\pm 6 \)

This would have been the solution \textbf{if asked for the possible Integer roots}!
\item \( \pm 1,\pm 5 \)

 Distractor 1: Corresponds to the plus or minus factors of a1 only.
\item \( \text{ All combinations of: }\frac{\pm 1,\pm 2,\pm 3,\pm 6}{\pm 1,\pm 5} \)

* This is the solution \textbf{since we asked for the possible Rational roots}!
\item \( \text{ There is no formula or theorem that tells us all possible Rational roots.} \)

 Distractor 4: Corresponds to not recalling the theorem for rational roots of a polynomial.
\end{enumerate}

\textbf{General Comment:} We have a way to find the possible Rational roots. The possible Integer roots are the Integers in this list.
}
\litem{
What are the \textit{possible Integer} roots of the polynomial below?
\[ f(x) = 4x^{3} +5 x^{2} +7 x + 5 \]The solution is \( \pm 1,\pm 5 \), which is option B.\begin{enumerate}[label=\Alph*.]
\item \( \text{ All combinations of: }\frac{\pm 1,\pm 5}{\pm 1,\pm 2,\pm 4} \)

This would have been the solution \textbf{if asked for the possible Rational roots}!
\item \( \pm 1,\pm 5 \)

* This is the solution \textbf{since we asked for the possible Integer roots}!
\item \( \pm 1,\pm 2,\pm 4 \)

 Distractor 1: Corresponds to the plus or minus factors of a1 only.
\item \( \text{ All combinations of: }\frac{\pm 1,\pm 2,\pm 4}{\pm 1,\pm 5} \)

 Distractor 3: Corresponds to the plus or minus of the inverse quotient (an/a0) of the factors. 
\item \( \text{There is no formula or theorem that tells us all possible Integer roots.} \)

 Distractor 4: Corresponds to not recognizing Integers as a subset of Rationals.
\end{enumerate}

\textbf{General Comment:} We have a way to find the possible Rational roots. The possible Integer roots are the Integers in this list.
}
\litem{
Factor the polynomial below completely, knowing that $x -2$ is a factor. Then, choose the intervals the zeros of the polynomial belong to, where $z_1 \leq z_2 \leq z_3 \leq z_4$. \textit{To make the problem easier, all zeros are between -5 and 5.}
\[ f(x) = 15x^{4} -71 x^{3} +12 x^{2} +116 x + 48 \]The solution is \( [-0.667, -0.6, 2, 4] \), which is option D.\begin{enumerate}[label=\Alph*.]
\item \( z_1 \in [-5.2, -2.7], \text{   }  z_2 \in [-2.28, -1.89], z_3 \in [0.55, 0.73], \text{   and   } z_4 \in [-0.06, 1] \)

 Distractor 1: Corresponds to negatives of all zeros.
\item \( z_1 \in [-5.2, -2.7], \text{   }  z_2 \in [-2.28, -1.89], z_3 \in [1.36, 1.63], \text{   and   } z_4 \in [1.29, 2.3] \)

 Distractor 3: Corresponds to negatives of all zeros AND inversing rational roots.
\item \( z_1 \in [-5.2, -2.7], \text{   }  z_2 \in [-2.28, -1.89], z_3 \in [0.08, 0.3], \text{   and   } z_4 \in [2.98, 3.64] \)

 Distractor 4: Corresponds to moving factors from one rational to another.
\item \( z_1 \in [-0.8, -0.3], \text{   }  z_2 \in [-0.67, -0.27], z_3 \in [1.8, 2.48], \text{   and   } z_4 \in [3.99, 4.54] \)

* This is the solution!
\item \( z_1 \in [-2, -1], \text{   }  z_2 \in [-1.72, -1.24], z_3 \in [1.8, 2.48], \text{   and   } z_4 \in [3.99, 4.54] \)

 Distractor 2: Corresponds to inversing rational roots.
\end{enumerate}

\textbf{General Comment:} Remember to try the middle-most integers first as these normally are the zeros. Also, once you get it to a quadratic, you can use your other factoring techniques to finish factoring.
}
\litem{
Perform the division below. Then, find the intervals that correspond to the quotient in the form $ax^2+bx+c$ and remainder $r$.
\[ \frac{8x^{3} -24 x^{2} + 27}{x -2} \]The solution is \( 8x^{2} -8 x -16 + \frac{-5}{x -2} \), which is option C.\begin{enumerate}[label=\Alph*.]
\item \( a \in [14, 18], b \in [8, 9], c \in [16, 17], \text{ and } r \in [58, 60]. \)

 You multipled by the synthetic number rather than bringing the first factor down.
\item \( a \in [14, 18], b \in [-56, -55], c \in [109, 118], \text{ and } r \in [-197, -196]. \)

 You divided by the opposite of the factor AND multipled the first factor rather than just bringing it down.
\item \( a \in [5, 10], b \in [-11, -2], c \in [-16, -11], \text{ and } r \in [-5, -4]. \)

* This is the solution!
\item \( a \in [5, 10], b \in [-17, -12], c \in [-16, -11], \text{ and } r \in [7, 17]. \)

 You multipled by the synthetic number and subtracted rather than adding during synthetic division.
\item \( a \in [5, 10], b \in [-43, -39], c \in [74, 87], \text{ and } r \in [-135, -130]. \)

 You divided by the opposite of the factor.
\end{enumerate}

\textbf{General Comment:} Be sure to synthetically divide by the zero of the denominator! Also, make sure to include 0 placeholders for missing terms.
}
\litem{
Perform the division below. Then, find the intervals that correspond to the quotient in the form $ax^2+bx+c$ and remainder $r$.
\[ \frac{16x^{3} +84 x^{2} -97}{x + 5} \]The solution is \( 16x^{2} +4 x -20 + \frac{3}{x + 5} \), which is option B.\begin{enumerate}[label=\Alph*.]
\item \( a \in [16, 19], b \in [164, 167], c \in [820, 821], \text{ and } r \in [4001, 4004]. \)

 You divided by the opposite of the factor.
\item \( a \in [16, 19], b \in [1, 6], c \in [-20, -18], \text{ and } r \in [-1, 4]. \)

* This is the solution!
\item \( a \in [-82, -76], b \in [-320, -311], c \in [-1583, -1577], \text{ and } r \in [-7999, -7993]. \)

 You divided by the opposite of the factor AND multipled the first factor rather than just bringing it down.
\item \( a \in [-82, -76], b \in [482, 491], c \in [-2420, -2411], \text{ and } r \in [12002, 12004]. \)

 You multipled by the synthetic number rather than bringing the first factor down.
\item \( a \in [16, 19], b \in [-12, -9], c \in [69, 77], \text{ and } r \in [-535, -527]. \)

 You multipled by the synthetic number and subtracted rather than adding during synthetic division.
\end{enumerate}

\textbf{General Comment:} Be sure to synthetically divide by the zero of the denominator! Also, make sure to include 0 placeholders for missing terms.
}
\litem{
Factor the polynomial below completely, knowing that $x -5$ is a factor. Then, choose the intervals the zeros of the polynomial belong to, where $z_1 \leq z_2 \leq z_3 \leq z_4$. \textit{To make the problem easier, all zeros are between -5 and 5.}
\[ f(x) = 8x^{4} -30 x^{3} -87 x^{2} +155 x + 150 \]The solution is \( [-2.5, -0.75, 2, 5] \), which is option E.\begin{enumerate}[label=\Alph*.]
\item \( z_1 \in [-6.2, -4.4], \text{   }  z_2 \in [-2.15, -1.97], z_3 \in [0.56, 0.64], \text{   and   } z_4 \in [2.55, 3.08] \)

 Distractor 4: Corresponds to moving factors from one rational to another.
\item \( z_1 \in [-1.8, -1.1], \text{   }  z_2 \in [-0.43, -0.15], z_3 \in [1.84, 2.03], \text{   and   } z_4 \in [4.4, 5.57] \)

 Distractor 2: Corresponds to inversing rational roots.
\item \( z_1 \in [-6.2, -4.4], \text{   }  z_2 \in [-2.15, -1.97], z_3 \in [0.34, 0.55], \text{   and   } z_4 \in [0.91, 1.86] \)

 Distractor 3: Corresponds to negatives of all zeros AND inversing rational roots.
\item \( z_1 \in [-6.2, -4.4], \text{   }  z_2 \in [-2.15, -1.97], z_3 \in [0.67, 0.99], \text{   and   } z_4 \in [2.14, 2.86] \)

 Distractor 1: Corresponds to negatives of all zeros.
\item \( z_1 \in [-4, -2.2], \text{   }  z_2 \in [-0.79, -0.71], z_3 \in [1.84, 2.03], \text{   and   } z_4 \in [4.4, 5.57] \)

* This is the solution!
\end{enumerate}

\textbf{General Comment:} Remember to try the middle-most integers first as these normally are the zeros. Also, once you get it to a quadratic, you can use your other factoring techniques to finish factoring.
}
\litem{
Factor the polynomial below completely. Then, choose the intervals the zeros of the polynomial belong to, where $z_1 \leq z_2 \leq z_3$. \textit{To make the problem easier, all zeros are between -5 and 5.}
\[ f(x) = 20x^{3} -83 x^{2} -95 x + 50 \]The solution is \( [-1.25, 0.4, 5] \), which is option E.\begin{enumerate}[label=\Alph*.]
\item \( z_1 \in [-1.16, -0.26], \text{   }  z_2 \in [2.38, 3.36], \text{   and   } z_3 \in [4.32, 5.39] \)

 Distractor 2: Corresponds to inversing rational roots.
\item \( z_1 \in [-5.24, -4.84], \text{   }  z_2 \in [-2.8, -1.73], \text{   and   } z_3 \in [0.34, 0.82] \)

 Distractor 3: Corresponds to negatives of all zeros AND inversing rational roots.
\item \( z_1 \in [-5.24, -4.84], \text{   }  z_2 \in [-0.35, -0.02], \text{   and   } z_3 \in [4.32, 5.39] \)

 Distractor 4: Corresponds to moving factors from one rational to another.
\item \( z_1 \in [-5.24, -4.84], \text{   }  z_2 \in [-1.03, -0.32], \text{   and   } z_3 \in [1.09, 1.48] \)

 Distractor 1: Corresponds to negatives of all zeros.
\item \( z_1 \in [-1.45, -1.22], \text{   }  z_2 \in [-0.07, 0.58], \text{   and   } z_3 \in [4.32, 5.39] \)

* This is the solution!
\end{enumerate}

\textbf{General Comment:} Remember to try the middle-most integers first as these normally are the zeros. Also, once you get it to a quadratic, you can use your other factoring techniques to finish factoring.
}
\litem{
Factor the polynomial below completely. Then, choose the intervals the zeros of the polynomial belong to, where $z_1 \leq z_2 \leq z_3$. \textit{To make the problem easier, all zeros are between -5 and 5.}
\[ f(x) = 20x^{3} -77 x^{2} +89 x -30 \]The solution is \( [0.6, 1.25, 2] \), which is option B.\begin{enumerate}[label=\Alph*.]
\item \( z_1 \in [0.74, 0.84], \text{   }  z_2 \in [1.47, 1.68], \text{   and   } z_3 \in [1.89, 2.27] \)

 Distractor 2: Corresponds to inversing rational roots.
\item \( z_1 \in [0.49, 0.73], \text{   }  z_2 \in [1.14, 1.3], \text{   and   } z_3 \in [1.89, 2.27] \)

* This is the solution!
\item \( z_1 \in [-2.22, -1.99], \text{   }  z_2 \in [-1.85, -1.62], \text{   and   } z_3 \in [-0.9, -0.71] \)

 Distractor 3: Corresponds to negatives of all zeros AND inversing rational roots.
\item \( z_1 \in [-2.22, -1.99], \text{   }  z_2 \in [-1.28, -1.13], \text{   and   } z_3 \in [-0.63, -0.44] \)

 Distractor 1: Corresponds to negatives of all zeros.
\item \( z_1 \in [-3.19, -2.76], \text{   }  z_2 \in [-2.02, -1.87], \text{   and   } z_3 \in [-0.35, -0.18] \)

 Distractor 4: Corresponds to moving factors from one rational to another.
\end{enumerate}

\textbf{General Comment:} Remember to try the middle-most integers first as these normally are the zeros. Also, once you get it to a quadratic, you can use your other factoring techniques to finish factoring.
}
\litem{
Perform the division below. Then, find the intervals that correspond to the quotient in the form $ax^2+bx+c$ and remainder $r$.
\[ \frac{6x^{3} -46 x^{2} +88 x -43}{x -5} \]The solution is \( 6x^{2} -16 x + 8 + \frac{-3}{x -5} \), which is option E.\begin{enumerate}[label=\Alph*.]
\item \( a \in [1, 13], \text{   } b \in [-26, -19], \text{   } c \in [-4, 3], \text{   and   } r \in [-43, -40]. \)

 You multiplied by the synthetic number and subtracted rather than adding during synthetic division.
\item \( a \in [1, 13], \text{   } b \in [-77, -70], \text{   } c \in [465, 475], \text{   and   } r \in [-2386, -2380]. \)

 You divided by the opposite of the factor.
\item \( a \in [26, 32], \text{   } b \in [102, 112], \text{   } c \in [605, 610], \text{   and   } r \in [2996, 3001]. \)

 You multiplied by the synthetic number rather than bringing the first factor down.
\item \( a \in [26, 32], \text{   } b \in [-198, -189], \text{   } c \in [1064, 1072], \text{   and   } r \in [-5386, -5379]. \)

 You divided by the opposite of the factor AND multiplied the first factor rather than just bringing it down.
\item \( a \in [1, 13], \text{   } b \in [-19, -13], \text{   } c \in [8, 14], \text{   and   } r \in [-7, 2]. \)

* This is the solution!
\end{enumerate}

\textbf{General Comment:} Be sure to synthetically divide by the zero of the denominator!
}
\litem{
Perform the division below. Then, find the intervals that correspond to the quotient in the form $ax^2+bx+c$ and remainder $r$.
\[ \frac{12x^{3} -64 x^{2} +100 x -52}{x -3} \]The solution is \( 12x^{2} -28 x + 16 + \frac{-4}{x -3} \), which is option A.\begin{enumerate}[label=\Alph*.]
\item \( a \in [8, 17], \text{   } b \in [-28, -25], \text{   } c \in [14, 17], \text{   and   } r \in [-4, 0]. \)

* This is the solution!
\item \( a \in [8, 17], \text{   } b \in [-100, -98], \text{   } c \in [400, 402], \text{   and   } r \in [-1254, -1246]. \)

 You divided by the opposite of the factor.
\item \( a \in [33, 45], \text{   } b \in [39, 48], \text{   } c \in [226, 233], \text{   and   } r \in [642, 646]. \)

 You multiplied by the synthetic number rather than bringing the first factor down.
\item \( a \in [33, 45], \text{   } b \in [-175, -166], \text{   } c \in [616, 624], \text{   and   } r \in [-1903, -1893]. \)

 You divided by the opposite of the factor AND multiplied the first factor rather than just bringing it down.
\item \( a \in [8, 17], \text{   } b \in [-44, -38], \text{   } c \in [19, 21], \text{   and   } r \in [-19, -11]. \)

 You multiplied by the synthetic number and subtracted rather than adding during synthetic division.
\end{enumerate}

\textbf{General Comment:} Be sure to synthetically divide by the zero of the denominator!
}
\end{enumerate}

\end{document}