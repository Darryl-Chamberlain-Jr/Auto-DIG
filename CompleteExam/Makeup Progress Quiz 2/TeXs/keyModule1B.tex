\documentclass{extbook}[14pt]
\usepackage{multicol, enumerate, enumitem, hyperref, color, soul, setspace, parskip, fancyhdr, amssymb, amsthm, amsmath, latexsym, units, mathtools}
\everymath{\displaystyle}
\usepackage[headsep=0.5cm,headheight=0cm, left=1 in,right= 1 in,top= 1 in,bottom= 1 in]{geometry}
\usepackage{dashrule}  % Package to use the command below to create lines between items
\newcommand{\litem}[1]{\item #1

\rule{\textwidth}{0.4pt}}
\pagestyle{fancy}
\lhead{}
\chead{Answer Key for Makeup Progress Quiz 2 Version B}
\rhead{}
\lfoot{5763-3522}
\cfoot{}
\rfoot{Spring 2021}
\begin{document}
\textbf{This key should allow you to understand why you choose the option you did (beyond just getting a question right or wrong). \href{https://xronos.clas.ufl.edu/mac1105spring2020/courseDescriptionAndMisc/Exams/LearningFromResults}{More instructions on how to use this key can be found here}.}

\textbf{If you have a suggestion to make the keys better, \href{https://forms.gle/CZkbZmPbC9XALEE88}{please fill out the short survey here}.}

\textit{Note: This key is auto-generated and may contain issues and/or errors. The keys are reviewed after each exam to ensure grading is done accurately. If there are issues (like duplicate options), they are noted in the offline gradebook. The keys are a work-in-progress to give students as many resources to improve as possible.}

\rule{\textwidth}{0.4pt}

\begin{enumerate}\litem{
Choose the \textbf{smallest} set of Complex numbers that the number below belongs to.
\[ \sqrt{\frac{-1683}{9}} i+\sqrt{156}i \]The solution is \( \text{Nonreal Complex} \), which is option C.\begin{enumerate}[label=\Alph*.]
\item \( \text{Not a Complex Number} \)

This is not a number. The only non-Complex number we know is dividing by 0 as this is not a number!
\item \( \text{Pure Imaginary} \)

This is a Complex number $(a+bi)$ that \textbf{only} has an imaginary part like $2i$.
\item \( \text{Nonreal Complex} \)

* This is the correct option!
\item \( \text{Rational} \)

These are numbers that can be written as fraction of Integers (e.g., -2/3 + 5)
\item \( \text{Irrational} \)

These cannot be written as a fraction of Integers. Remember: $\pi$ is not an Integer!
\end{enumerate}

\textbf{General Comment:} Be sure to simplify $i^2 = -1$. This may remove the imaginary portion for your number. If you are having trouble, you may want to look at the \textit{Subgroups of the Real Numbers} section.
}
\litem{
Simplify the expression below into the form $a+bi$. Then, choose the intervals that $a$ and $b$ belong to.
\[ (8 - 2 i)(3 - 6 i) \]The solution is \( 12 - 54 i \), which is option B.\begin{enumerate}[label=\Alph*.]
\item \( a \in [11, 14] \text{ and } b \in [52, 55] \)

 $12 + 54 i$, which corresponds to adding a minus sign in both terms.
\item \( a \in [11, 14] \text{ and } b \in [-58, -52] \)

* $12 - 54 i$, which is the correct option.
\item \( a \in [32, 37] \text{ and } b \in [-48, -39] \)

 $36 - 42 i$, which corresponds to adding a minus sign in the first term.
\item \( a \in [24, 29] \text{ and } b \in [10, 14] \)

 $24 + 12 i$, which corresponds to just multiplying the real terms to get the real part of the solution and the coefficients in the complex terms to get the complex part.
\item \( a \in [32, 37] \text{ and } b \in [40, 43] \)

 $36 + 42 i$, which corresponds to adding a minus sign in the second term.
\end{enumerate}

\textbf{General Comment:} You can treat $i$ as a variable and distribute. Just remember that $i^2=-1$, so you can continue to reduce after you distribute.
}
\litem{
Simplify the expression below and choose the interval the simplification is contained within.
\[ 3 - 19^2 + 5 \div 8 * 7 \div 15 \]The solution is \( -357.708 \), which is option C.\begin{enumerate}[label=\Alph*.]
\item \( [363.88, 364.06] \)

 364.006, which corresponds to two Order of Operations errors.
\item \( [-358.14, -357.93] \)

 -357.994, which corresponds to an Order of Operations error: not reading left-to-right for multiplication/division.
\item \( [-357.85, -357.68] \)

* -357.708, this is the correct option
\item \( [364.15, 364.3] \)

 364.292, which corresponds to an Order of Operations error: multiplying by negative before squaring. For example: $(-3)^2 \neq -3^2$
\item \( \text{None of the above} \)

 You may have gotten this by making an unanticipated error. If you got a value that is not any of the others, please let the coordinator know so they can help you figure out what happened.
\end{enumerate}

\textbf{General Comment:} While you may remember (or were taught) PEMDAS is done in order, it is actually done as P/E/MD/AS. When we are at MD or AS, we read left to right.
}
\litem{
Simplify the expression below into the form $a+bi$. Then, choose the intervals that $a$ and $b$ belong to.
\[ \frac{-45 + 77 i}{-6 + 2 i} \]The solution is \( 10.60  - 9.30 i \), which is option C.\begin{enumerate}[label=\Alph*.]
\item \( a \in [9.5, 11.5] \text{ and } b \in [-373, -371.5] \)

 $10.60  - 372.00 i$, which corresponds to forgetting to multiply the conjugate by the numerator.
\item \( a \in [1.5, 3.5] \text{ and } b \in [-14.5, -13.5] \)

 $2.90  - 13.80 i$, which corresponds to forgetting to multiply the conjugate by the numerator and not computing the conjugate correctly.
\item \( a \in [9.5, 11.5] \text{ and } b \in [-10, -8.5] \)

* $10.60  - 9.30 i$, which is the correct option.
\item \( a \in [423, 424.5] \text{ and } b \in [-10, -8.5] \)

 $424.00  - 9.30 i$, which corresponds to forgetting to multiply the conjugate by the numerator and using a plus instead of a minus in the denominator.
\item \( a \in [6.5, 8.5] \text{ and } b \in [37, 40.5] \)

 $7.50  + 38.50 i$, which corresponds to just dividing the first term by the first term and the second by the second.
\end{enumerate}

\textbf{General Comment:} Multiply the numerator and denominator by the *conjugate* of the denominator, then simplify. For example, if we have $2+3i$, the conjugate is $2-3i$.
}
\litem{
Choose the \textbf{smallest} set of Real numbers that the number below belongs to.
\[ \sqrt{\frac{-1170}{5}} \]The solution is \( \text{Not a Real number} \), which is option A.\begin{enumerate}[label=\Alph*.]
\item \( \text{Not a Real number} \)

* This is the correct option!
\item \( \text{Irrational} \)

These cannot be written as a fraction of Integers.
\item \( \text{Integer} \)

These are the negative and positive counting numbers (..., -3, -2, -1, 0, 1, 2, 3, ...)
\item \( \text{Rational} \)

These are numbers that can be written as fraction of Integers (e.g., -2/3)
\item \( \text{Whole} \)

These are the counting numbers with 0 (0, 1, 2, 3, ...)
\end{enumerate}

\textbf{General Comment:} First, you \textbf{NEED} to simplify the expression. This question simplifies to $\sqrt{234} i$. 
 
 Be sure you look at the simplified fraction and not just the decimal expansion. Numbers such as 13, 17, and 19 provide \textbf{long but repeating/terminating decimal expansions!} 
 
 The only ways to *not* be a Real number are: dividing by 0 or taking the square root of a negative number. 
 
 Irrational numbers are more than just square root of 3: adding or subtracting values from square root of 3 is also irrational.
}
\litem{
Simplify the expression below into the form $a+bi$. Then, choose the intervals that $a$ and $b$ belong to.
\[ \frac{-72 + 22 i}{7 + i} \]The solution is \( -9.64  + 4.52 i \), which is option C.\begin{enumerate}[label=\Alph*.]
\item \( a \in [-10.4, -9.96] \text{ and } b \in [21.5, 22.5] \)

 $-10.29  + 22.00 i$, which corresponds to just dividing the first term by the first term and the second by the second.
\item \( a \in [-482.12, -481.89] \text{ and } b \in [4, 6] \)

 $-482.00  + 4.52 i$, which corresponds to forgetting to multiply the conjugate by the numerator and using a plus instead of a minus in the denominator.
\item \( a \in [-9.87, -9.61] \text{ and } b \in [4, 6] \)

* $-9.64  + 4.52 i$, which is the correct option.
\item \( a \in [-9.87, -9.61] \text{ and } b \in [224, 228] \)

 $-9.64  + 226.00 i$, which corresponds to forgetting to multiply the conjugate by the numerator.
\item \( a \in [-10.64, -10.39] \text{ and } b \in [1, 2.5] \)

 $-10.52  + 1.64 i$, which corresponds to forgetting to multiply the conjugate by the numerator and not computing the conjugate correctly.
\end{enumerate}

\textbf{General Comment:} Multiply the numerator and denominator by the *conjugate* of the denominator, then simplify. For example, if we have $2+3i$, the conjugate is $2-3i$.
}
\litem{
Simplify the expression below and choose the interval the simplification is contained within.
\[ 7 - 13^2 + 9 \div 18 * 8 \div 10 \]The solution is \( -161.600 \), which is option C.\begin{enumerate}[label=\Alph*.]
\item \( [176.02, 176.86] \)

 176.400, which corresponds to an Order of Operations error: multiplying by negative before squaring. For example: $(-3)^2 \neq -3^2$
\item \( [-162.44, -161.93] \)

 -161.994, which corresponds to an Order of Operations error: not reading left-to-right for multiplication/division.
\item \( [-161.97, -161.41] \)

* -161.600, this is the correct option
\item \( [175.86, 176.03] \)

 176.006, which corresponds to two Order of Operations errors.
\item \( \text{None of the above} \)

 You may have gotten this by making an unanticipated error. If you got a value that is not any of the others, please let the coordinator know so they can help you figure out what happened.
\end{enumerate}

\textbf{General Comment:} While you may remember (or were taught) PEMDAS is done in order, it is actually done as P/E/MD/AS. When we are at MD or AS, we read left to right.
}
\litem{
Choose the \textbf{smallest} set of Complex numbers that the number below belongs to.
\[ \sqrt{\frac{1056}{6}}+\sqrt{70} i \]The solution is \( \text{Nonreal Complex} \), which is option A.\begin{enumerate}[label=\Alph*.]
\item \( \text{Nonreal Complex} \)

* This is the correct option!
\item \( \text{Rational} \)

These are numbers that can be written as fraction of Integers (e.g., -2/3 + 5)
\item \( \text{Pure Imaginary} \)

This is a Complex number $(a+bi)$ that \textbf{only} has an imaginary part like $2i$.
\item \( \text{Irrational} \)

These cannot be written as a fraction of Integers. Remember: $\pi$ is not an Integer!
\item \( \text{Not a Complex Number} \)

This is not a number. The only non-Complex number we know is dividing by 0 as this is not a number!
\end{enumerate}

\textbf{General Comment:} Be sure to simplify $i^2 = -1$. This may remove the imaginary portion for your number. If you are having trouble, you may want to look at the \textit{Subgroups of the Real Numbers} section.
}
\litem{
Simplify the expression below into the form $a+bi$. Then, choose the intervals that $a$ and $b$ belong to.
\[ (-3 - 6 i)(-8 - 7 i) \]The solution is \( -18 + 69 i \), which is option B.\begin{enumerate}[label=\Alph*.]
\item \( a \in [66, 73] \text{ and } b \in [24, 32] \)

 $66 + 27 i$, which corresponds to adding a minus sign in the second term.
\item \( a \in [-24, -16] \text{ and } b \in [67, 72] \)

* $-18 + 69 i$, which is the correct option.
\item \( a \in [20, 25] \text{ and } b \in [42, 43] \)

 $24 + 42 i$, which corresponds to just multiplying the real terms to get the real part of the solution and the coefficients in the complex terms to get the complex part.
\item \( a \in [66, 73] \text{ and } b \in [-27, -23] \)

 $66 - 27 i$, which corresponds to adding a minus sign in the first term.
\item \( a \in [-24, -16] \text{ and } b \in [-74, -65] \)

 $-18 - 69 i$, which corresponds to adding a minus sign in both terms.
\end{enumerate}

\textbf{General Comment:} You can treat $i$ as a variable and distribute. Just remember that $i^2=-1$, so you can continue to reduce after you distribute.
}
\litem{
Choose the \textbf{smallest} set of Real numbers that the number below belongs to.
\[ -\sqrt{\frac{455}{7}} \]The solution is \( \text{Irrational} \), which is option A.\begin{enumerate}[label=\Alph*.]
\item \( \text{Irrational} \)

* This is the correct option!
\item \( \text{Not a Real number} \)

These are Nonreal Complex numbers \textbf{OR} things that are not numbers (e.g., dividing by 0).
\item \( \text{Integer} \)

These are the negative and positive counting numbers (..., -3, -2, -1, 0, 1, 2, 3, ...)
\item \( \text{Whole} \)

These are the counting numbers with 0 (0, 1, 2, 3, ...)
\item \( \text{Rational} \)

These are numbers that can be written as fraction of Integers (e.g., -2/3)
\end{enumerate}

\textbf{General Comment:} First, you \textbf{NEED} to simplify the expression. This question simplifies to $-\sqrt{65}$. 
 
 Be sure you look at the simplified fraction and not just the decimal expansion. Numbers such as 13, 17, and 19 provide \textbf{long but repeating/terminating decimal expansions!} 
 
 The only ways to *not* be a Real number are: dividing by 0 or taking the square root of a negative number. 
 
 Irrational numbers are more than just square root of 3: adding or subtracting values from square root of 3 is also irrational.
}
\end{enumerate}

\end{document}