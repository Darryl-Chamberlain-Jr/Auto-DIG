\documentclass{extbook}[14pt]
\usepackage{multicol, enumerate, enumitem, hyperref, color, soul, setspace, parskip, fancyhdr, amssymb, amsthm, amsmath, latexsym, units, mathtools}
\everymath{\displaystyle}
\usepackage[headsep=0.5cm,headheight=0cm, left=1 in,right= 1 in,top= 1 in,bottom= 1 in]{geometry}
\usepackage{dashrule}  % Package to use the command below to create lines between items
\newcommand{\litem}[1]{\item #1

\rule{\textwidth}{0.4pt}}
\pagestyle{fancy}
\lhead{}
\chead{Answer Key for Makeup Progress Quiz 2 Version B}
\rhead{}
\lfoot{2790-1423}
\cfoot{}
\rfoot{Summer C 2021}
\begin{document}
\textbf{This key should allow you to understand why you choose the option you did (beyond just getting a question right or wrong). \href{https://xronos.clas.ufl.edu/mac1105spring2020/courseDescriptionAndMisc/Exams/LearningFromResults}{More instructions on how to use this key can be found here}.}

\textbf{If you have a suggestion to make the keys better, \href{https://forms.gle/CZkbZmPbC9XALEE88}{please fill out the short survey here}.}

\textit{Note: This key is auto-generated and may contain issues and/or errors. The keys are reviewed after each exam to ensure grading is done accurately. If there are issues (like duplicate options), they are noted in the offline gradebook. The keys are a work-in-progress to give students as many resources to improve as possible.}

\rule{\textwidth}{0.4pt}

\begin{enumerate}\litem{
Simplify the expression below into the form $a+bi$. Then, choose the intervals that $a$ and $b$ belong to.
\[ (10 - 3 i)(8 - 9 i) \]The solution is \( 53 - 114 i \), which is option D.\begin{enumerate}[label=\Alph*.]
\item \( a \in [103, 114] \text{ and } b \in [-71, -59] \)

 $107 - 66 i$, which corresponds to adding a minus sign in the first term.
\item \( a \in [75, 82] \text{ and } b \in [22, 30] \)

 $80 + 27 i$, which corresponds to just multiplying the real terms to get the real part of the solution and the coefficients in the complex terms to get the complex part.
\item \( a \in [103, 114] \text{ and } b \in [66, 68] \)

 $107 + 66 i$, which corresponds to adding a minus sign in the second term.
\item \( a \in [50, 58] \text{ and } b \in [-115, -111] \)

* $53 - 114 i$, which is the correct option.
\item \( a \in [50, 58] \text{ and } b \in [114, 116] \)

 $53 + 114 i$, which corresponds to adding a minus sign in both terms.
\end{enumerate}

\textbf{General Comment:} You can treat $i$ as a variable and distribute. Just remember that $i^2=-1$, so you can continue to reduce after you distribute.
}
\litem{
Simplify the expression below into the form $a+bi$. Then, choose the intervals that $a$ and $b$ belong to.
\[ \frac{54 - 22 i}{5 + 4 i} \]The solution is \( 4.44  - 7.95 i \), which is option B.\begin{enumerate}[label=\Alph*.]
\item \( a \in [10.5, 11] \text{ and } b \in [-6, -5] \)

 $10.80  - 5.50 i$, which corresponds to just dividing the first term by the first term and the second by the second.
\item \( a \in [4, 5] \text{ and } b \in [-9, -7] \)

* $4.44  - 7.95 i$, which is the correct option.
\item \( a \in [4, 5] \text{ and } b \in [-327, -325.5] \)

 $4.44  - 326.00 i$, which corresponds to forgetting to multiply the conjugate by the numerator.
\item \( a \in [181, 183] \text{ and } b \in [-9, -7] \)

 $182.00  - 7.95 i$, which corresponds to forgetting to multiply the conjugate by the numerator and using a plus instead of a minus in the denominator.
\item \( a \in [8, 10] \text{ and } b \in [1.5, 3.5] \)

 $8.73  + 2.59 i$, which corresponds to forgetting to multiply the conjugate by the numerator and not computing the conjugate correctly.
\end{enumerate}

\textbf{General Comment:} Multiply the numerator and denominator by the *conjugate* of the denominator, then simplify. For example, if we have $2+3i$, the conjugate is $2-3i$.
}
\litem{
Choose the \textbf{smallest} set of Real numbers that the number below belongs to.
\[ -\sqrt{\frac{484}{169}} \]The solution is \( \text{Rational} \), which is option B.\begin{enumerate}[label=\Alph*.]
\item \( \text{Not a Real number} \)

These are Nonreal Complex numbers \textbf{OR} things that are not numbers (e.g., dividing by 0).
\item \( \text{Rational} \)

* This is the correct option!
\item \( \text{Integer} \)

These are the negative and positive counting numbers (..., -3, -2, -1, 0, 1, 2, 3, ...)
\item \( \text{Irrational} \)

These cannot be written as a fraction of Integers.
\item \( \text{Whole} \)

These are the counting numbers with 0 (0, 1, 2, 3, ...)
\end{enumerate}

\textbf{General Comment:} First, you \textbf{NEED} to simplify the expression. This question simplifies to $-\frac{22}{13}$. 
 
 Be sure you look at the simplified fraction and not just the decimal expansion. Numbers such as 13, 17, and 19 provide \textbf{long but repeating/terminating decimal expansions!} 
 
 The only ways to *not* be a Real number are: dividing by 0 or taking the square root of a negative number. 
 
 Irrational numbers are more than just square root of 3: adding or subtracting values from square root of 3 is also irrational.
}
\litem{
Simplify the expression below and choose the interval the simplification is contained within.
\[ 9 - 2^2 + 10 \div 8 * 14 \div 20 \]The solution is \( 5.875 \), which is option C.\begin{enumerate}[label=\Alph*.]
\item \( [12.54, 13.05] \)

 13.004, which corresponds to two Order of Operations errors.
\item \( [13.09, 14.05] \)

 13.875, which corresponds to an Order of Operations error: multiplying by negative before squaring. For example: $(-3)^2 \neq -3^2$
\item \( [5.54, 5.99] \)

* 5.875, this is the correct option
\item \( [4.75, 5.51] \)

 5.004, which corresponds to an Order of Operations error: not reading left-to-right for multiplication/division.
\item \( \text{None of the above} \)

 You may have gotten this by making an unanticipated error. If you got a value that is not any of the others, please let the coordinator know so they can help you figure out what happened.
\end{enumerate}

\textbf{General Comment:} While you may remember (or were taught) PEMDAS is done in order, it is actually done as P/E/MD/AS. When we are at MD or AS, we read left to right.
}
\litem{
Choose the \textbf{smallest} set of Real numbers that the number below belongs to.
\[ \sqrt{\frac{144}{625}} \]The solution is \( \text{Rational} \), which is option E.\begin{enumerate}[label=\Alph*.]
\item \( \text{Integer} \)

These are the negative and positive counting numbers (..., -3, -2, -1, 0, 1, 2, 3, ...)
\item \( \text{Irrational} \)

These cannot be written as a fraction of Integers.
\item \( \text{Not a Real number} \)

These are Nonreal Complex numbers \textbf{OR} things that are not numbers (e.g., dividing by 0).
\item \( \text{Whole} \)

These are the counting numbers with 0 (0, 1, 2, 3, ...)
\item \( \text{Rational} \)

* This is the correct option!
\end{enumerate}

\textbf{General Comment:} First, you \textbf{NEED} to simplify the expression. This question simplifies to $\frac{12}{25}$. 
 
 Be sure you look at the simplified fraction and not just the decimal expansion. Numbers such as 13, 17, and 19 provide \textbf{long but repeating/terminating decimal expansions!} 
 
 The only ways to *not* be a Real number are: dividing by 0 or taking the square root of a negative number. 
 
 Irrational numbers are more than just square root of 3: adding or subtracting values from square root of 3 is also irrational.
}
\litem{
Simplify the expression below and choose the interval the simplification is contained within.
\[ 2 - 10^2 + 12 \div 18 * 16 \div 9 \]The solution is \( -96.815 \), which is option D.\begin{enumerate}[label=\Alph*.]
\item \( [101.69, 103.16] \)

 102.005, which corresponds to two Order of Operations errors.
\item \( [102.58, 103.51] \)

 103.185, which corresponds to an Order of Operations error: multiplying by negative before squaring. For example: $(-3)^2 \neq -3^2$
\item \( [-98.17, -97.88] \)

 -97.995, which corresponds to an Order of Operations error: not reading left-to-right for multiplication/division.
\item \( [-97.84, -95.36] \)

* -96.815, this is the correct option
\item \( \text{None of the above} \)

 You may have gotten this by making an unanticipated error. If you got a value that is not any of the others, please let the coordinator know so they can help you figure out what happened.
\end{enumerate}

\textbf{General Comment:} While you may remember (or were taught) PEMDAS is done in order, it is actually done as P/E/MD/AS. When we are at MD or AS, we read left to right.
}
\litem{
Simplify the expression below into the form $a+bi$. Then, choose the intervals that $a$ and $b$ belong to.
\[ (-2 + 6 i)(-3 - 4 i) \]The solution is \( 30 - 10 i \), which is option B.\begin{enumerate}[label=\Alph*.]
\item \( a \in [-18, -17] \text{ and } b \in [-27.7, -25.5] \)

 $-18 - 26 i$, which corresponds to adding a minus sign in the second term.
\item \( a \in [30, 38] \text{ and } b \in [-11.3, -7.5] \)

* $30 - 10 i$, which is the correct option.
\item \( a \in [30, 38] \text{ and } b \in [8.5, 13.4] \)

 $30 + 10 i$, which corresponds to adding a minus sign in both terms.
\item \( a \in [-18, -17] \text{ and } b \in [24.7, 27.8] \)

 $-18 + 26 i$, which corresponds to adding a minus sign in the first term.
\item \( a \in [6, 9] \text{ and } b \in [-25.9, -21.2] \)

 $6 - 24 i$, which corresponds to just multiplying the real terms to get the real part of the solution and the coefficients in the complex terms to get the complex part.
\end{enumerate}

\textbf{General Comment:} You can treat $i$ as a variable and distribute. Just remember that $i^2=-1$, so you can continue to reduce after you distribute.
}
\litem{
Choose the \textbf{smallest} set of Complex numbers that the number below belongs to.
\[ \sqrt{\frac{1820}{10}}+10i^2 \]The solution is \( \text{Irrational} \), which is option A.\begin{enumerate}[label=\Alph*.]
\item \( \text{Irrational} \)

* This is the correct option!
\item \( \text{Pure Imaginary} \)

This is a Complex number $(a+bi)$ that \textbf{only} has an imaginary part like $2i$.
\item \( \text{Rational} \)

These are numbers that can be written as fraction of Integers (e.g., -2/3 + 5)
\item \( \text{Nonreal Complex} \)

This is a Complex number $(a+bi)$ that is not Real (has $i$ as part of the number).
\item \( \text{Not a Complex Number} \)

This is not a number. The only non-Complex number we know is dividing by 0 as this is not a number!
\end{enumerate}

\textbf{General Comment:} Be sure to simplify $i^2 = -1$. This may remove the imaginary portion for your number. If you are having trouble, you may want to look at the \textit{Subgroups of the Real Numbers} section.
}
\litem{
Simplify the expression below into the form $a+bi$. Then, choose the intervals that $a$ and $b$ belong to.
\[ \frac{45 + 44 i}{-6 + 8 i} \]The solution is \( 0.82  - 6.24 i \), which is option D.\begin{enumerate}[label=\Alph*.]
\item \( a \in [-6.5, -4.5] \text{ and } b \in [-1, 1.5] \)

 $-6.22  + 0.96 i$, which corresponds to forgetting to multiply the conjugate by the numerator and not computing the conjugate correctly.
\item \( a \in [-8, -7] \text{ and } b \in [4.5, 6.5] \)

 $-7.50  + 5.50 i$, which corresponds to just dividing the first term by the first term and the second by the second.
\item \( a \in [81.5, 83] \text{ and } b \in [-7, -5.5] \)

 $82.00  - 6.24 i$, which corresponds to forgetting to multiply the conjugate by the numerator and using a plus instead of a minus in the denominator.
\item \( a \in [-1, 1.5] \text{ and } b \in [-7, -5.5] \)

* $0.82  - 6.24 i$, which is the correct option.
\item \( a \in [-1, 1.5] \text{ and } b \in [-625.5, -622.5] \)

 $0.82  - 624.00 i$, which corresponds to forgetting to multiply the conjugate by the numerator.
\end{enumerate}

\textbf{General Comment:} Multiply the numerator and denominator by the *conjugate* of the denominator, then simplify. For example, if we have $2+3i$, the conjugate is $2-3i$.
}
\litem{
Choose the \textbf{smallest} set of Complex numbers that the number below belongs to.
\[ \sqrt{\frac{36}{0}}+\sqrt{238} i \]The solution is \( \text{Not a Complex Number} \), which is option D.\begin{enumerate}[label=\Alph*.]
\item \( \text{Nonreal Complex} \)

This is a Complex number $(a+bi)$ that is not Real (has $i$ as part of the number).
\item \( \text{Rational} \)

These are numbers that can be written as fraction of Integers (e.g., -2/3 + 5)
\item \( \text{Irrational} \)

These cannot be written as a fraction of Integers. Remember: $\pi$ is not an Integer!
\item \( \text{Not a Complex Number} \)

* This is the correct option!
\item \( \text{Pure Imaginary} \)

This is a Complex number $(a+bi)$ that \textbf{only} has an imaginary part like $2i$.
\end{enumerate}

\textbf{General Comment:} Be sure to simplify $i^2 = -1$. This may remove the imaginary portion for your number. If you are having trouble, you may want to look at the \textit{Subgroups of the Real Numbers} section.
}
\end{enumerate}

\end{document}