\documentclass{extbook}[14pt]
\usepackage{multicol, enumerate, enumitem, hyperref, color, soul, setspace, parskip, fancyhdr, amssymb, amsthm, amsmath, latexsym, units, mathtools}
\everymath{\displaystyle}
\usepackage[headsep=0.5cm,headheight=0cm, left=1 in,right= 1 in,top= 1 in,bottom= 1 in]{geometry}
\usepackage{dashrule}  % Package to use the command below to create lines between items
\newcommand{\litem}[1]{\item #1

\rule{\textwidth}{0.4pt}}
\pagestyle{fancy}
\lhead{}
\chead{Answer Key for Makeup Progress Quiz 2 Version A}
\rhead{}
\lfoot{2790-1423}
\cfoot{}
\rfoot{Summer C 2021}
\begin{document}
\textbf{This key should allow you to understand why you choose the option you did (beyond just getting a question right or wrong). \href{https://xronos.clas.ufl.edu/mac1105spring2020/courseDescriptionAndMisc/Exams/LearningFromResults}{More instructions on how to use this key can be found here}.}

\textbf{If you have a suggestion to make the keys better, \href{https://forms.gle/CZkbZmPbC9XALEE88}{please fill out the short survey here}.}

\textit{Note: This key is auto-generated and may contain issues and/or errors. The keys are reviewed after each exam to ensure grading is done accurately. If there are issues (like duplicate options), they are noted in the offline gradebook. The keys are a work-in-progress to give students as many resources to improve as possible.}

\rule{\textwidth}{0.4pt}

\begin{enumerate}\litem{
 Solve the equation for $x$ and choose the interval that contains $x$ (if it exists).
\[  25 = \sqrt[7]{\frac{25}{e^{6x}}} \]The solution is \( x = -3.219 \), which is option A.\begin{enumerate}[label=\Alph*.]
\item \( x \in [-6.22, -1.22] \)

* $x = -3.219$, which is the correct option.
\item \( x \in [-0.54, 3.46] \)

$x = -0.536$, which corresponds to treating any root as a square root.
\item \( x \in [-32.7, -27.7] \)

$x = -29.703$, which corresponds to thinking you don't need to take the natural log of both sides before reducing, as if the equation already had a natural log on the right side.
\item \( \text{There is no Real solution to the equation.} \)

This corresponds to believing you cannot solve the equation.
\item \( \text{None of the above.} \)

This corresponds to making an unexpected error.
\end{enumerate}

\textbf{General Comment:} \textbf{General Comments}: After using the properties of logarithmic functions to break up the right-hand side, use $\ln(e) = 1$ to reduce the question to a linear function to solve. You can put $\ln(25)$ into a calculator if you are having trouble.
}
\litem{
Which of the following intervals describes the Domain of the function below?
\[ f(x) = -e^{x-8}+5 \]The solution is \( (-\infty, \infty) \), which is option E.\begin{enumerate}[label=\Alph*.]
\item \( (-\infty, a], a \in [3, 7] \)

$(-\infty, 5]$, which corresponds to using the correct vertical shift *if we wanted the Range* AND including the endpoint.
\item \( (a, \infty), a \in [-8, -3] \)

$(-5, \infty)$, which corresponds to using the negative vertical shift AND flipping the Range interval.
\item \( [a, \infty), a \in [-8, -3] \)

$[-5, \infty)$, which corresponds to using the negative vertical shift AND flipping the Range interval AND including the endpoint.
\item \( (-\infty, a), a \in [3, 7] \)

$(-\infty, 5)$, which corresponds to using the correct vertical shift *if we wanted the Range*.
\item \( (-\infty, \infty) \)

* This is the correct option.
\end{enumerate}

\textbf{General Comment:} \textbf{General Comments}: Domain of a basic exponential function is $(-\infty, \infty)$ while the Range is $(0, \infty)$. We can shift these intervals [and even flip when $a<0$!] to find the new Domain/Range.
}
\litem{
Which of the following intervals describes the Domain of the function below?
\[ f(x) = \log_2{(x+6)}-3 \]The solution is \( (-6, \infty) \), which is option B.\begin{enumerate}[label=\Alph*.]
\item \( (-\infty, a), a \in [4.1, 6.6] \)

$(-\infty, 6)$, which corresponds to flipping the Domain. Remember: the general for is $a*\log(x-h)+k$, \textbf{where $a$ does not affect the domain}.
\item \( (a, \infty), a \in [-6.2, -3.5] \)

* $(-6, \infty)$, which is the correct option.
\item \( [a, \infty), a \in [-4.3, 0] \)

$[-3, \infty)$, which corresponds to using the vertical shift when shifting the Domain AND including the endpoint.
\item \( (-\infty, a], a \in [0.5, 4.3] \)

$(-\infty, 3]$, which corresponds to using the negative vertical shift AND including the endpoint AND flipping the domain.
\item \( (-\infty, \infty) \)

This corresponds to thinking of the range of the log function (or the domain of the exponential function).
\end{enumerate}

\textbf{General Comment:} \textbf{General Comments}: The domain of a basic logarithmic function is $(0, \infty)$ and the Range is $(-\infty, \infty)$. We can use shifts when finding the Domain, but the Range will always be all Real numbers.
}
\litem{
Which of the following intervals describes the Range of the function below?
\[ f(x) = -e^{x+8}-4 \]The solution is \( (-\infty, -4) \), which is option A.\begin{enumerate}[label=\Alph*.]
\item \( (-\infty, a), a \in [-4, 0] \)

* $(-\infty, -4)$, which is the correct option.
\item \( (-\infty, a], a \in [-4, 0] \)

$(-\infty, -4]$, which corresponds to including the endpoint.
\item \( (a, \infty), a \in [1, 7] \)

$(4, \infty)$, which corresponds to using the negative vertical shift AND flipping the Range interval.
\item \( [a, \infty), a \in [1, 7] \)

$[4, \infty)$, which corresponds to using the negative vertical shift AND flipping the Range interval AND including the endpoint.
\item \( (-\infty, \infty) \)

This corresponds to confusing range of an exponential function with the domain of an exponential function.
\end{enumerate}

\textbf{General Comment:} \textbf{General Comments}: Domain of a basic exponential function is $(-\infty, \infty)$ while the Range is $(0, \infty)$. We can shift these intervals [and even flip when $a<0$!] to find the new Domain/Range.
}
\litem{
Solve the equation for $x$ and choose the interval that contains the solution (if it exists).
\[ 2^{3x+3} = 27^{2x-5} \]The solution is \( x = 4.113 \), which is option B.\begin{enumerate}[label=\Alph*.]
\item \( x \in [-20.9, -16.9] \)

$x = -18.559$, which corresponds to distributing the $\ln(base)$ to the second term of the exponent only.
\item \( x \in [4, 5.5] \)

* $x = 4.113$, which is the correct option.
\item \( x \in [-8.7, -7.2] \)

$x = -8.000$, which corresponds to solving the numerators as equal while ignoring the bases are different.
\item \( x \in [1.2, 2.7] \)

$x = 1.773$, which corresponds to distributing the $\ln(base)$ to the first term of the exponent only.
\item \( \text{There is no Real solution to the equation.} \)

This corresponds to believing there is no solution since the bases are not powers of each other.
\end{enumerate}

\textbf{General Comment:} \textbf{General Comments:} This question was written so that the bases could not be written the same. You will need to take the log of both sides.
}
\litem{
Solve the equation for $x$ and choose the interval that contains the solution (if it exists).
\[ \log_{4}{(-4x+6)}+5 = 2 \]The solution is \( x = 1.496 \), which is option A.\begin{enumerate}[label=\Alph*.]
\item \( x \in [0.4, 2.7] \)

* $x = 1.496$, which is the correct option.
\item \( x \in [-3.4, -1] \)

$x = -2.500$, which corresponds to ignoring the vertical shift when converting to exponential form.
\item \( x \in [-21.7, -17.7] \)

$x = -18.750$, which corresponds to reversing the base and exponent when converting.
\item \( x \in [-25.7, -20.6] \)

$x = -21.750$, which corresponds to reversing the base and exponent when converting and reversing the value with $x$.
\item \( \text{There is no Real solution to the equation.} \)

Corresponds to believing a negative coefficient within the log equation means there is no Real solution.
\end{enumerate}

\textbf{General Comment:} \textbf{General Comments:} First, get the equation in the form $\log_b{(cx+d)} = a$. Then, convert to $b^a = cx+d$ and solve.
}
\litem{
Solve the equation for $x$ and choose the interval that contains the solution (if it exists).
\[ 5^{5x+3} = 9^{4x-3} \]The solution is \( x = 15.397 \), which is option D.\begin{enumerate}[label=\Alph*.]
\item \( x \in [-11.42, -7.42] \)

$x = -11.420$, which corresponds to distributing the $\ln(base)$ to the second term of the exponent only.
\item \( x \in [6.09, 10.09] \)

$x = 8.089$, which corresponds to distributing the $\ln(base)$ to the first term of the exponent only.
\item \( x \in [-7, -3] \)

$x = -6.000$, which corresponds to solving the numerators as equal while ignoring the bases are different.
\item \( x \in [13.4, 18.4] \)

* $x = 15.397$, which is the correct option.
\item \( \text{There is no Real solution to the equation.} \)

This corresponds to believing there is no solution since the bases are not powers of each other.
\end{enumerate}

\textbf{General Comment:} \textbf{General Comments:} This question was written so that the bases could not be written the same. You will need to take the log of both sides.
}
\litem{
Which of the following intervals describes the Domain of the function below?
\[ f(x) = -\log_2{(x-3)}-4 \]The solution is \( (3, \infty) \), which is option C.\begin{enumerate}[label=\Alph*.]
\item \( [a, \infty), a \in [-4.02, -3.23] \)

$[-4, \infty)$, which corresponds to using the vertical shift when shifting the Domain AND including the endpoint.
\item \( (-\infty, a], a \in [3.65, 4.06] \)

$(-\infty, 4]$, which corresponds to using the negative vertical shift AND including the endpoint AND flipping the domain.
\item \( (a, \infty), a \in [2.45, 3.87] \)

* $(3, \infty)$, which is the correct option.
\item \( (-\infty, a), a \in [-3.14, -2.88] \)

$(-\infty, -3)$, which corresponds to flipping the Domain. Remember: the general for is $a*\log(x-h)+k$, \textbf{where $a$ does not affect the domain}.
\item \( (-\infty, \infty) \)

This corresponds to thinking of the range of the log function (or the domain of the exponential function).
\end{enumerate}

\textbf{General Comment:} \textbf{General Comments}: The domain of a basic logarithmic function is $(0, \infty)$ and the Range is $(-\infty, \infty)$. We can use shifts when finding the Domain, but the Range will always be all Real numbers.
}
\litem{
Solve the equation for $x$ and choose the interval that contains the solution (if it exists).
\[ \log_{3}{(-4x+7)}+4 = 2 \]The solution is \( x = 1.722 \), which is option A.\begin{enumerate}[label=\Alph*.]
\item \( x \in [0.56, 2.17] \)

* $x = 1.722$, which is the correct option.
\item \( x \in [-0.61, -0.3] \)

$x = -0.500$, which corresponds to ignoring the vertical shift when converting to exponential form.
\item \( x \in [3.7, 4.81] \)

$x = 3.750$, which corresponds to reversing the base and exponent when converting.
\item \( x \in [0.24, 0.81] \)

$x = 0.250$, which corresponds to reversing the base and exponent when converting and reversing the value with $x$.
\item \( \text{There is no Real solution to the equation.} \)

Corresponds to believing a negative coefficient within the log equation means there is no Real solution.
\end{enumerate}

\textbf{General Comment:} \textbf{General Comments:} First, get the equation in the form $\log_b{(cx+d)} = a$. Then, convert to $b^a = cx+d$ and solve.
}
\litem{
 Solve the equation for $x$ and choose the interval that contains $x$ (if it exists).
\[  19 = \ln{\sqrt[3]{\frac{13}{e^{7x}}}} \]The solution is \( x = -7.776 \), which is option C.\begin{enumerate}[label=\Alph*.]
\item \( x \in [-1.63, 2.37] \)

$x = -1.628$, which corresponds to thinking you need to take the natural log of on the left before reducing.
\item \( x \in [-6.06, -3.06] \)

$x = -5.062$, which corresponds to treating any root as a square root.
\item \( x \in [-9.78, -5.78] \)

* $x = -7.776$, which is the correct option.
\item \( \text{There is no Real solution to the equation.} \)

This corresponds to believing you cannot solve the equation.
\item \( \text{None of the above.} \)

This corresponds to making an unexpected error.
\end{enumerate}

\textbf{General Comment:} \textbf{General Comments}: After using the properties of logarithmic functions to break up the right-hand side, use $\ln(e) = 1$ to reduce the question to a linear function to solve. You can put $\ln(13)$ into a calculator if you are having trouble.
}
\end{enumerate}

\end{document}