\documentclass[14pt]{extbook}
\usepackage{multicol, enumerate, enumitem, hyperref, color, soul, setspace, parskip, fancyhdr} %General Packages
\usepackage{amssymb, amsthm, amsmath, latexsym, units, mathtools} %Math Packages
\everymath{\displaystyle} %All math in Display Style
% Packages with additional options
\usepackage[headsep=0.5cm,headheight=12pt, left=1 in,right= 1 in,top= 1 in,bottom= 1 in]{geometry}
\usepackage[usenames,dvipsnames]{xcolor}
\usepackage{dashrule}  % Package to use the command below to create lines between items
\newcommand{\litem}[1]{\item#1\hspace*{-1cm}\rule{\textwidth}{0.4pt}}
\pagestyle{fancy}
\lhead{Makeup Progress Quiz 2}
\chead{}
\rhead{Version A}
\lfoot{2790-1423}
\cfoot{}
\rfoot{Summer C 2021}
\begin{document}

\begin{enumerate}
\litem{
What are the \textit{possible Rational} roots of the polynomial below?\[ f(x) = 7x^{4} +2 x^{3} +2 x^{2} +2 x + 6 \]\begin{enumerate}[label=\Alph*.]
\item \( \pm 1,\pm 7 \)
\item \( \text{ All combinations of: }\frac{\pm 1,\pm 7}{\pm 1,\pm 2,\pm 3,\pm 6} \)
\item \( \text{ All combinations of: }\frac{\pm 1,\pm 2,\pm 3,\pm 6}{\pm 1,\pm 7} \)
\item \( \pm 1,\pm 2,\pm 3,\pm 6 \)
\item \( \text{ There is no formula or theorem that tells us all possible Rational roots.} \)

\end{enumerate} }
\litem{
What are the \textit{possible Integer} roots of the polynomial below?\[ f(x) = 6x^{4} +6 x^{3} +3 x^{2} +2 x + 4 \]\begin{enumerate}[label=\Alph*.]
\item \( \pm 1,\pm 2,\pm 4 \)
\item \( \text{ All combinations of: }\frac{\pm 1,\pm 2,\pm 3,\pm 6}{\pm 1,\pm 2,\pm 4} \)
\item \( \text{ All combinations of: }\frac{\pm 1,\pm 2,\pm 4}{\pm 1,\pm 2,\pm 3,\pm 6} \)
\item \( \pm 1,\pm 2,\pm 3,\pm 6 \)
\item \( \text{There is no formula or theorem that tells us all possible Integer roots.} \)

\end{enumerate} }
\litem{
Factor the polynomial below completely, knowing that $x + 5$ is a factor. Then, choose the intervals the zeros of the polynomial belong to, where $z_1 \leq z_2 \leq z_3 \leq z_4$. \textit{To make the problem easier, all zeros are between -5 and 5.}\[ f(x) = 8x^{4} -14 x^{3} -167 x^{2} +455 x -300 \]\begin{enumerate}[label=\Alph*.]
\item \( z_1 \in [-4.25, -3.95], \text{   }  z_2 \in [-0.88, -0.53], z_3 \in [-0.67, -0.65], \text{   and   } z_4 \in [4.9, 5.1] \)
\item \( z_1 \in [-5.08, -4.64], \text{   }  z_2 \in [0.79, 1.43], z_3 \in [1.47, 1.53], \text{   and   } z_4 \in [2.5, 4.4] \)
\item \( z_1 \in [-5.08, -4.64], \text{   }  z_2 \in [-0.31, 0.74], z_3 \in [0.76, 0.84], \text{   and   } z_4 \in [2.5, 4.4] \)
\item \( z_1 \in [-4.25, -3.95], \text{   }  z_2 \in [-3.36, -2.87], z_3 \in [-0.63, -0.53], \text{   and   } z_4 \in [4.9, 5.1] \)
\item \( z_1 \in [-4.25, -3.95], \text{   }  z_2 \in [-2.41, -0.84], z_3 \in [-1.26, -1.22], \text{   and   } z_4 \in [4.9, 5.1] \)

\end{enumerate} }
\litem{
Perform the division below. Then, find the intervals that correspond to the quotient in the form $ax^2+bx+c$ and remainder $r$.\[ \frac{4x^{3} -12 x + 6}{x + 2} \]\begin{enumerate}[label=\Alph*.]
\item \( a \in [3, 8], b \in [-13, -10], c \in [15, 25], \text{ and } r \in [-67, -61]. \)
\item \( a \in [3, 8], b \in [8, 10], c \in [-1, 8], \text{ and } r \in [8, 15]. \)
\item \( a \in [-10, -4], b \in [10, 17], c \in [-48, -42], \text{ and } r \in [94, 97]. \)
\item \( a \in [3, 8], b \in [-9, 0], c \in [-1, 8], \text{ and } r \in [-5, 4]. \)
\item \( a \in [-10, -4], b \in [-20, -15], c \in [-48, -42], \text{ and } r \in [-85, -81]. \)

\end{enumerate} }
\litem{
Perform the division below. Then, find the intervals that correspond to the quotient in the form $ax^2+bx+c$ and remainder $r$.\[ \frac{20x^{3} -63 x^{2} + 23}{x -3} \]\begin{enumerate}[label=\Alph*.]
\item \( a \in [57, 65], b \in [113, 120], c \in [350, 355], \text{ and } r \in [1074, 1078]. \)
\item \( a \in [17, 22], b \in [-130, -118], c \in [369, 371], \text{ and } r \in [-1085, -1082]. \)
\item \( a \in [57, 65], b \in [-245, -241], c \in [729, 731], \text{ and } r \in [-2169, -2161]. \)
\item \( a \in [17, 22], b \in [-5, 0], c \in [-13, -7], \text{ and } r \in [-6, 4]. \)
\item \( a \in [17, 22], b \in [-29, -22], c \in [-47, -42], \text{ and } r \in [-70, -68]. \)

\end{enumerate} }
\litem{
Factor the polynomial below completely, knowing that $x -5$ is a factor. Then, choose the intervals the zeros of the polynomial belong to, where $z_1 \leq z_2 \leq z_3 \leq z_4$. \textit{To make the problem easier, all zeros are between -5 and 5.}\[ f(x) = 12x^{4} -113 x^{3} +338 x^{2} -395 x + 150 \]\begin{enumerate}[label=\Alph*.]
\item \( z_1 \in [-5.35, -4.91], \text{   }  z_2 \in [-6.33, -4.95], z_3 \in [-2.42, -1.69], \text{   and   } z_4 \in [-0.34, -0.14] \)
\item \( z_1 \in [-5.35, -4.91], \text{   }  z_2 \in [-2.13, -1.6], z_3 \in [-1.8, -1.62], \text{   and   } z_4 \in [-0.84, -0.74] \)
\item \( z_1 \in [0.72, 1.07], \text{   }  z_2 \in [1.35, 1.95], z_3 \in [1.61, 2.61], \text{   and   } z_4 \in [4.79, 5.08] \)
\item \( z_1 \in [0.59, 0.69], \text{   }  z_2 \in [1.11, 1.62], z_3 \in [1.61, 2.61], \text{   and   } z_4 \in [4.79, 5.08] \)
\item \( z_1 \in [-5.35, -4.91], \text{   }  z_2 \in [-2.13, -1.6], z_3 \in [-1.44, -1.27], \text{   and   } z_4 \in [-0.68, -0.44] \)

\end{enumerate} }
\litem{
Factor the polynomial below completely. Then, choose the intervals the zeros of the polynomial belong to, where $z_1 \leq z_2 \leq z_3$. \textit{To make the problem easier, all zeros are between -5 and 5.}\[ f(x) = 25x^{3} -100 x^{2} -4 x + 16 \]\begin{enumerate}[label=\Alph*.]
\item \( z_1 \in [-4.6, -3.3], \text{   }  z_2 \in [-2.65, -2.36], \text{   and   } z_3 \in [2.09, 3] \)
\item \( z_1 \in [-2.9, -2.4], \text{   }  z_2 \in [1.93, 2.94], \text{   and   } z_3 \in [3.93, 4.24] \)
\item \( z_1 \in [-4.6, -3.3], \text{   }  z_2 \in [-2.25, -1.76], \text{   and   } z_3 \in [-0.2, 0.15] \)
\item \( z_1 \in [-1.6, 0.4], \text{   }  z_2 \in [0.19, 0.79], \text{   and   } z_3 \in [3.93, 4.24] \)
\item \( z_1 \in [-4.6, -3.3], \text{   }  z_2 \in [-0.56, -0.04], \text{   and   } z_3 \in [0.11, 1.11] \)

\end{enumerate} }
\litem{
Factor the polynomial below completely. Then, choose the intervals the zeros of the polynomial belong to, where $z_1 \leq z_2 \leq z_3$. \textit{To make the problem easier, all zeros are between -5 and 5.}\[ f(x) = 20x^{3} +31 x^{2} -38 x -40 \]\begin{enumerate}[label=\Alph*.]
\item \( z_1 \in [-0.92, -0.65], \text{   }  z_2 \in [1.12, 1.47], \text{   and   } z_3 \in [1.78, 2.59] \)
\item \( z_1 \in [-5.08, -4.9], \text{   }  z_2 \in [-0.09, 0.27], \text{   and   } z_3 \in [1.78, 2.59] \)
\item \( z_1 \in [-1.46, -1.02], \text{   }  z_2 \in [0.78, 1.08], \text{   and   } z_3 \in [1.78, 2.59] \)
\item \( z_1 \in [-2.2, -1.8], \text{   }  z_2 \in [-1.59, -1.18], \text{   and   } z_3 \in [0.41, 0.86] \)
\item \( z_1 \in [-2.2, -1.8], \text{   }  z_2 \in [-0.86, -0.52], \text{   and   } z_3 \in [1.1, 1.49] \)

\end{enumerate} }
\litem{
Perform the division below. Then, find the intervals that correspond to the quotient in the form $ax^2+bx+c$ and remainder $r$.\[ \frac{6x^{3} -2 x^{2} -20 x + 19}{x + 2} \]\begin{enumerate}[label=\Alph*.]
\item \( a \in [-15, -8], \text{   } b \in [-28, -25], \text{   } c \in [-72, -68], \text{   and   } r \in [-132, -124]. \)
\item \( a \in [-15, -8], \text{   } b \in [22, 24], \text{   } c \in [-66, -63], \text{   and   } r \in [144, 149]. \)
\item \( a \in [1, 11], \text{   } b \in [-21, -19], \text{   } c \in [34, 46], \text{   and   } r \in [-103, -97]. \)
\item \( a \in [1, 11], \text{   } b \in [8, 17], \text{   } c \in [-3, 4], \text{   and   } r \in [15, 20]. \)
\item \( a \in [1, 11], \text{   } b \in [-14, -9], \text{   } c \in [7, 9], \text{   and   } r \in [2, 4]. \)

\end{enumerate} }
\litem{
Perform the division below. Then, find the intervals that correspond to the quotient in the form $ax^2+bx+c$ and remainder $r$.\[ \frac{6x^{3} -20 x^{2} -2 x + 19}{x -3} \]\begin{enumerate}[label=\Alph*.]
\item \( a \in [5, 9], \text{   } b \in [-40, -34], \text{   } c \in [111, 115], \text{   and   } r \in [-322, -313]. \)
\item \( a \in [15, 21], \text{   } b \in [-76, -72], \text{   } c \in [218, 224], \text{   and   } r \in [-646, -637]. \)
\item \( a \in [5, 9], \text{   } b \in [-14, -3], \text{   } c \in [-23, -16], \text{   and   } r \in [-18, -15]. \)
\item \( a \in [5, 9], \text{   } b \in [-6, 6], \text{   } c \in [-15, -7], \text{   and   } r \in [-6, -3]. \)
\item \( a \in [15, 21], \text{   } b \in [29, 35], \text{   } c \in [95, 104], \text{   and   } r \in [318, 321]. \)

\end{enumerate} }
\end{enumerate}

\end{document}