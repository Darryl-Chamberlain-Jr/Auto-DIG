\documentclass{extbook}[14pt]
\usepackage{multicol, enumerate, enumitem, hyperref, color, soul, setspace, parskip, fancyhdr, amssymb, amsthm, amsmath, latexsym, units, mathtools}
\everymath{\displaystyle}
\usepackage[headsep=0.5cm,headheight=0cm, left=1 in,right= 1 in,top= 1 in,bottom= 1 in]{geometry}
\usepackage{dashrule}  % Package to use the command below to create lines between items
\newcommand{\litem}[1]{\item #1

\rule{\textwidth}{0.4pt}}
\pagestyle{fancy}
\lhead{}
\chead{Answer Key for Makeup Progress Quiz 2 Version C}
\rhead{}
\lfoot{2790-1423}
\cfoot{}
\rfoot{Summer C 2021}
\begin{document}
\textbf{This key should allow you to understand why you choose the option you did (beyond just getting a question right or wrong). \href{https://xronos.clas.ufl.edu/mac1105spring2020/courseDescriptionAndMisc/Exams/LearningFromResults}{More instructions on how to use this key can be found here}.}

\textbf{If you have a suggestion to make the keys better, \href{https://forms.gle/CZkbZmPbC9XALEE88}{please fill out the short survey here}.}

\textit{Note: This key is auto-generated and may contain issues and/or errors. The keys are reviewed after each exam to ensure grading is done accurately. If there are issues (like duplicate options), they are noted in the offline gradebook. The keys are a work-in-progress to give students as many resources to improve as possible.}

\rule{\textwidth}{0.4pt}

\begin{enumerate}\litem{
Subtract the following functions, then choose the domain of the resulting function from the list below.
\[ f(x) = 4x^{3} + x^{2} +5 x + 3 \text{ and } g(x) = \frac{4}{6x+19} \]The solution is \( \text{ The domain is all Real numbers except } x = -3.17 \), which is option A.\begin{enumerate}[label=\Alph*.]
\item \( \text{ The domain is all Real numbers except } x = a, \text{ where } a \in [-3.17, 2.83] \)


\item \( \text{ The domain is all Real numbers less than or equal to } x = a, \text{ where } a \in [-10.2, -0.2] \)


\item \( \text{ The domain is all Real numbers greater than or equal to } x = a, \text{ where } a \in [-13.5, -6.5] \)


\item \( \text{ The domain is all Real numbers except } x = a \text{ and } x = b, \text{ where } a \in [-6.6, -5.6] \text{ and } b \in [-7.33, -3.33] \)


\item \( \text{ The domain is all Real numbers. } \)


\end{enumerate}

\textbf{General Comment:} The new domain is the intersection of the previous domains.
}
\litem{
Choose the interval below that $f$ composed with $g$ at $x=-1$ is in.
\[ f(x) = 2x^{3} -3 x^{2} -4 x -1 \text{ and } g(x) = 3x^{3} -1 x^{2} -3 x + 3 \]The solution is \( -5.0 \), which is option A.\begin{enumerate}[label=\Alph*.]
\item \( (f \circ g)(-1) \in [-9, -3] \)

* This is the correct solution
\item \( (f \circ g)(-1) \in [0, 4] \)

 Distractor 2: Corresponds to being slightly off from the solution.
\item \( (f \circ g)(-1) \in [-21, -12] \)

 Distractor 1: Corresponds to reversing the composition.
\item \( (f \circ g)(-1) \in [-29, -26] \)

 Distractor 3: Corresponds to being slightly off from the solution.
\item \( \text{It is not possible to compose the two functions.} \)


\end{enumerate}

\textbf{General Comment:} $f$ composed with $g$ at $x$ means $f(g(x))$. The order matters!
}
\litem{
Choose the interval below that $f$ composed with $g$ at $x=1$ is in.
\[ f(x) = 4x^{3} -1 x^{2} -2 x \text{ and } g(x) = x^{3} + x^{2} -x + 2 \]The solution is \( 93.0 \), which is option A.\begin{enumerate}[label=\Alph*.]
\item \( (f \circ g)(1) \in [88, 97] \)

* This is the correct solution
\item \( (f \circ g)(1) \in [-5, -4] \)

 Distractor 3: Corresponds to being slightly off from the solution.
\item \( (f \circ g)(1) \in [2, 6] \)

 Distractor 1: Corresponds to reversing the composition.
\item \( (f \circ g)(1) \in [100, 103] \)

 Distractor 2: Corresponds to being slightly off from the solution.
\item \( \text{It is not possible to compose the two functions.} \)


\end{enumerate}

\textbf{General Comment:} $f$ composed with $g$ at $x$ means $f(g(x))$. The order matters!
}
\litem{
Find the inverse of the function below. Then, evaluate the inverse at $x = 8$ and choose the interval that $f^-1(8)$ belongs to.
\[ f(x) = e^{x-4}-4 \]The solution is \( f^{-1}(8) = 6.485 \), which is option C.\begin{enumerate}[label=\Alph*.]
\item \( f^{-1}(8) \in [-4.61, -1.61] \)

 This solution corresponds to distractor 4.
\item \( f^{-1}(8) \in [-1.52, -0.52] \)

 This solution corresponds to distractor 1.
\item \( f^{-1}(8) \in [1.48, 11.48] \)

 This is the solution.
\item \( f^{-1}(8) \in [-4.61, -1.61] \)

 This solution corresponds to distractor 2.
\item \( f^{-1}(8) \in [-1.52, -0.52] \)

 This solution corresponds to distractor 3.
\end{enumerate}

\textbf{General Comment:} Natural log and exponential functions always have an inverse. Once you switch the $x$ and $y$, use the conversion $ e^y = x \leftrightarrow y=\ln(x)$.
}
\litem{
Find the inverse of the function below (if it exists). Then, evaluate the inverse at $x = -13$ and choose the interval that $f^-1(-13)$ belongs to.
\[ f(x) = 4 x^2 - 3 \]The solution is \( \text{ The function is not invertible for all Real numbers. } \), which is option E.\begin{enumerate}[label=\Alph*.]
\item \( f^{-1}(-13) \in [2.52, 3.29] \)

 Distractor 3: This corresponds to finding the (nonexistent) inverse and dividing by a negative.
\item \( f^{-1}(-13) \in [3.81, 4.71] \)

 Distractor 4: This corresponds to both distractors 2 and 3.
\item \( f^{-1}(-13) \in [1.81, 2.5] \)

 Distractor 2: This corresponds to finding the (nonexistent) inverse and not subtracting by the vertical shift.
\item \( f^{-1}(-13) \in [1.15, 1.95] \)

 Distractor 1: This corresponds to trying to find the inverse even though the function is not 1-1. 
\item \( \text{ The function is not invertible for all Real numbers. } \)

* This is the correct option.
\end{enumerate}

\textbf{General Comment:} Be sure you check that the function is 1-1 before trying to find the inverse!
}
\litem{
Find the inverse of the function below. Then, evaluate the inverse at $x = 9$ and choose the interval that $f^-1(9)$ belongs to.
\[ f(x) = e^{x+5}-2 \]The solution is \( f^{-1}(9) = -2.602 \), which is option E.\begin{enumerate}[label=\Alph*.]
\item \( f^{-1}(9) \in [-0.77, -0.59] \)

 This solution corresponds to distractor 3.
\item \( f^{-1}(9) \in [7.29, 7.8] \)

 This solution corresponds to distractor 1.
\item \( f^{-1}(9) \in [0.07, 0.74] \)

 This solution corresponds to distractor 4.
\item \( f^{-1}(9) \in [-0.08, 0.22] \)

 This solution corresponds to distractor 2.
\item \( f^{-1}(9) \in [-2.69, -2.51] \)

 This is the solution.
\end{enumerate}

\textbf{General Comment:} Natural log and exponential functions always have an inverse. Once you switch the $x$ and $y$, use the conversion $ e^y = x \leftrightarrow y=\ln(x)$.
}
\litem{
Determine whether the function below is 1-1.
\[ f(x) = 9 x^2 - 60 x + 100 \]The solution is \( \text{no} \), which is option E.\begin{enumerate}[label=\Alph*.]
\item \( \text{Yes, the function is 1-1.} \)

Corresponds to believing the function passes the Horizontal Line test.
\item \( \text{No, because the range of the function is not $(-\infty, \infty)$.} \)

Corresponds to believing 1-1 means the range is all Real numbers.
\item \( \text{No, because the domain of the function is not $(-\infty, \infty)$.} \)

Corresponds to believing 1-1 means the domain is all Real numbers.
\item \( \text{No, because there is an $x$-value that goes to 2 different $y$-values.} \)

Corresponds to the Vertical Line test, which checks if an expression is a function.
\item \( \text{No, because there is a $y$-value that goes to 2 different $x$-values.} \)

* This is the solution.
\end{enumerate}

\textbf{General Comment:} There are only two valid options: The function is 1-1 OR No because there is a $y$-value that goes to 2 different $x$-values.
}
\litem{
Determine whether the function below is 1-1.
\[ f(x) = (3 x - 18)^3 \]The solution is \( \text{yes} \), which is option B.\begin{enumerate}[label=\Alph*.]
\item \( \text{No, because there is an $x$-value that goes to 2 different $y$-values.} \)

Corresponds to the Vertical Line test, which checks if an expression is a function.
\item \( \text{Yes, the function is 1-1.} \)

* This is the solution.
\item \( \text{No, because there is a $y$-value that goes to 2 different $x$-values.} \)

Corresponds to the Horizontal Line test, which this function passes.
\item \( \text{No, because the range of the function is not $(-\infty, \infty)$.} \)

Corresponds to believing 1-1 means the range is all Real numbers.
\item \( \text{No, because the domain of the function is not $(-\infty, \infty)$.} \)

Corresponds to believing 1-1 means the domain is all Real numbers.
\end{enumerate}

\textbf{General Comment:} There are only two valid options: The function is 1-1 OR No because there is a $y$-value that goes to 2 different $x$-values.
}
\litem{
Multiply the following functions, then choose the domain of the resulting function from the list below.
\[ f(x) = \frac{2}{3x-13} \text{ and } g(x) = \frac{2}{3x-19} \]The solution is \( \text{ The domain is all Real numbers except } x = 4.33 \text{ and } x = 6.33 \), which is option D.\begin{enumerate}[label=\Alph*.]
\item \( \text{ The domain is all Real numbers less than or equal to } x = a, \text{ where } a \in [-4.17, -2.17] \)


\item \( \text{ The domain is all Real numbers except } x = a, \text{ where } a \in [-8.8, -1.8] \)


\item \( \text{ The domain is all Real numbers greater than or equal to } x = a, \text{ where } a \in [3.33, 10.33] \)


\item \( \text{ The domain is all Real numbers except } x = a \text{ and } x = b, \text{ where } a \in [2.33, 8.33] \text{ and } b \in [4.33, 10.33] \)


\item \( \text{ The domain is all Real numbers. } \)


\end{enumerate}

\textbf{General Comment:} The new domain is the intersection of the previous domains.
}
\litem{
Find the inverse of the function below (if it exists). Then, evaluate the inverse at $x = -12$ and choose the interval that $f^-1(-12)$ belongs to.
\[ f(x) = 4 x^2 - 3 \]The solution is \( \text{ The function is not invertible for all Real numbers. } \), which is option E.\begin{enumerate}[label=\Alph*.]
\item \( f^{-1}(-12) \in [1.56, 2.12] \)

 Distractor 2: This corresponds to finding the (nonexistent) inverse and not subtracting by the vertical shift.
\item \( f^{-1}(-12) \in [4.48, 4.58] \)

 Distractor 3: This corresponds to finding the (nonexistent) inverse and dividing by a negative.
\item \( f^{-1}(-12) \in [7.34, 7.65] \)

 Distractor 4: This corresponds to both distractors 2 and 3.
\item \( f^{-1}(-12) \in [1.37, 1.68] \)

 Distractor 1: This corresponds to trying to find the inverse even though the function is not 1-1. 
\item \( \text{ The function is not invertible for all Real numbers. } \)

* This is the correct option.
\end{enumerate}

\textbf{General Comment:} Be sure you check that the function is 1-1 before trying to find the inverse!
}
\end{enumerate}

\end{document}