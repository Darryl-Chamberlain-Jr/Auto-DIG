\documentclass{extbook}[14pt]
\usepackage{multicol, enumerate, enumitem, hyperref, color, soul, setspace, parskip, fancyhdr, amssymb, amsthm, amsmath, latexsym, units, mathtools}
\everymath{\displaystyle}
\usepackage[headsep=0.5cm,headheight=0cm, left=1 in,right= 1 in,top= 1 in,bottom= 1 in]{geometry}
\usepackage{dashrule}  % Package to use the command below to create lines between items
\newcommand{\litem}[1]{\item #1

\rule{\textwidth}{0.4pt}}
\pagestyle{fancy}
\lhead{}
\chead{Answer Key for Makeup Progress Quiz 2 Version B}
\rhead{}
\lfoot{2790-1423}
\cfoot{}
\rfoot{Summer C 2021}
\begin{document}
\textbf{This key should allow you to understand why you choose the option you did (beyond just getting a question right or wrong). \href{https://xronos.clas.ufl.edu/mac1105spring2020/courseDescriptionAndMisc/Exams/LearningFromResults}{More instructions on how to use this key can be found here}.}

\textbf{If you have a suggestion to make the keys better, \href{https://forms.gle/CZkbZmPbC9XALEE88}{please fill out the short survey here}.}

\textit{Note: This key is auto-generated and may contain issues and/or errors. The keys are reviewed after each exam to ensure grading is done accurately. If there are issues (like duplicate options), they are noted in the offline gradebook. The keys are a work-in-progress to give students as many resources to improve as possible.}

\rule{\textwidth}{0.4pt}

\begin{enumerate}\litem{
Multiply the following functions, then choose the domain of the resulting function from the list below.
\[ f(x) = 4x^{2} +4 x + 4 \text{ and } g(x) = 5x^{2} +8 x + 2 \]The solution is \( (-\infty, \infty) \), which is option E.\begin{enumerate}[label=\Alph*.]
\item \( \text{ The domain is all Real numbers except } x = a, \text{ where } a \in [-12.67, -3.67] \)


\item \( \text{ The domain is all Real numbers greater than or equal to } x = a, \text{ where } a \in [0.25, 6.25] \)


\item \( \text{ The domain is all Real numbers less than or equal to } x = a, \text{ where } a \in [-5.4, 0.6] \)


\item \( \text{ The domain is all Real numbers except } x = a \text{ and } x = b, \text{ where } a \in [3.67, 9.67] \text{ and } b \in [-4.33, 0.67] \)


\item \( \text{ The domain is all Real numbers. } \)


\end{enumerate}

\textbf{General Comment:} The new domain is the intersection of the previous domains.
}
\litem{
Choose the interval below that $f$ composed with $g$ at $x=1$ is in.
\[ f(x) = -x^{3} -2 x^{2} +x \text{ and } g(x) = -4x^{3} -4 x^{2} +4 x + 3 \]The solution is \( -2.0 \), which is option A.\begin{enumerate}[label=\Alph*.]
\item \( (f \circ g)(1) \in [-3.2, 0.5] \)

* This is the correct solution
\item \( (f \circ g)(1) \in [9.5, 11.2] \)

 Distractor 1: Corresponds to reversing the composition.
\item \( (f \circ g)(1) \in [4.5, 8] \)

 Distractor 3: Corresponds to being slightly off from the solution.
\item \( (f \circ g)(1) \in [2.6, 4.9] \)

 Distractor 2: Corresponds to being slightly off from the solution.
\item \( \text{It is not possible to compose the two functions.} \)


\end{enumerate}

\textbf{General Comment:} $f$ composed with $g$ at $x$ means $f(g(x))$. The order matters!
}
\litem{
Choose the interval below that $f$ composed with $g$ at $x=1$ is in.
\[ f(x) = -3x^{3} +4 x^{2} +4 x \text{ and } g(x) = x^{3} -1 x^{2} -3 x + 1 \]The solution is \( 32.0 \), which is option B.\begin{enumerate}[label=\Alph*.]
\item \( (f \circ g)(1) \in [81, 89] \)

 Distractor 1: Corresponds to reversing the composition.
\item \( (f \circ g)(1) \in [28, 36] \)

* This is the correct solution
\item \( (f \circ g)(1) \in [35, 42] \)

 Distractor 2: Corresponds to being slightly off from the solution.
\item \( (f \circ g)(1) \in [88, 96] \)

 Distractor 3: Corresponds to being slightly off from the solution.
\item \( \text{It is not possible to compose the two functions.} \)


\end{enumerate}

\textbf{General Comment:} $f$ composed with $g$ at $x$ means $f(g(x))$. The order matters!
}
\litem{
Find the inverse of the function below. Then, evaluate the inverse at $x = 7$ and choose the interval that $f^-1(7)$ belongs to.
\[ f(x) = e^{x-2}-2 \]The solution is \( f^{-1}(7) = 4.197 \), which is option E.\begin{enumerate}[label=\Alph*.]
\item \( f^{-1}(7) \in [-1.06, 0.03] \)

 This solution corresponds to distractor 4.
\item \( f^{-1}(7) \in [-0.08, 0.39] \)

 This solution corresponds to distractor 1.
\item \( f^{-1}(7) \in [-1.06, 0.03] \)

 This solution corresponds to distractor 2.
\item \( f^{-1}(7) \in [-0.08, 0.39] \)

 This solution corresponds to distractor 3.
\item \( f^{-1}(7) \in [3.32, 4.75] \)

 This is the solution.
\end{enumerate}

\textbf{General Comment:} Natural log and exponential functions always have an inverse. Once you switch the $x$ and $y$, use the conversion $ e^y = x \leftrightarrow y=\ln(x)$.
}
\litem{
Find the inverse of the function below (if it exists). Then, evaluate the inverse at $x = -12$ and choose the interval that $f^-1(-12)$ belongs to.
\[ f(x) = \sqrt[3]{2 x - 5} \]The solution is \( -861.5 \), which is option A.\begin{enumerate}[label=\Alph*.]
\item \( f^{-1}(-12) \in [-863.5, -860.5] \)

* This is the correct solution.
\item \( f^{-1}(-12) \in [-866.5, -865.5] \)

 Distractor 1: This corresponds to 
\item \( f^{-1}(-12) \in [866.5, 870.5] \)

 This solution corresponds to distractor 3.
\item \( f^{-1}(-12) \in [855.5, 862.5] \)

 This solution corresponds to distractor 2.
\item \( \text{ The function is not invertible for all Real numbers. } \)

 This solution corresponds to distractor 4.
\end{enumerate}

\textbf{General Comment:} Be sure you check that the function is 1-1 before trying to find the inverse!
}
\litem{
Find the inverse of the function below. Then, evaluate the inverse at $x = 9$ and choose the interval that $f^-1(9)$ belongs to.
\[ f(x) = e^{x-5}+2 \]The solution is \( f^{-1}(9) = 6.946 \), which is option A.\begin{enumerate}[label=\Alph*.]
\item \( f^{-1}(9) \in [6.88, 7.09] \)

 This is the solution.
\item \( f^{-1}(9) \in [-3.15, -2.83] \)

 This solution corresponds to distractor 1.
\item \( f^{-1}(9) \in [4.54, 4.85] \)

 This solution corresponds to distractor 3.
\item \( f^{-1}(9) \in [4.28, 4.42] \)

 This solution corresponds to distractor 2.
\item \( f^{-1}(9) \in [3.38, 3.42] \)

 This solution corresponds to distractor 4.
\end{enumerate}

\textbf{General Comment:} Natural log and exponential functions always have an inverse. Once you switch the $x$ and $y$, use the conversion $ e^y = x \leftrightarrow y=\ln(x)$.
}
\litem{
Determine whether the function below is 1-1.
\[ f(x) = 25 x^2 - 90 x - 319 \]The solution is \( \text{no} \), which is option C.\begin{enumerate}[label=\Alph*.]
\item \( \text{No, because the range of the function is not $(-\infty, \infty)$.} \)

Corresponds to believing 1-1 means the range is all Real numbers.
\item \( \text{No, because the domain of the function is not $(-\infty, \infty)$.} \)

Corresponds to believing 1-1 means the domain is all Real numbers.
\item \( \text{No, because there is a $y$-value that goes to 2 different $x$-values.} \)

* This is the solution.
\item \( \text{No, because there is an $x$-value that goes to 2 different $y$-values.} \)

Corresponds to the Vertical Line test, which checks if an expression is a function.
\item \( \text{Yes, the function is 1-1.} \)

Corresponds to believing the function passes the Horizontal Line test.
\end{enumerate}

\textbf{General Comment:} There are only two valid options: The function is 1-1 OR No because there is a $y$-value that goes to 2 different $x$-values.
}
\litem{
Determine whether the function below is 1-1.
\[ f(x) = 9 x^2 - 60 x + 100 \]The solution is \( \text{no} \), which is option A.\begin{enumerate}[label=\Alph*.]
\item \( \text{No, because there is a $y$-value that goes to 2 different $x$-values.} \)

* This is the solution.
\item \( \text{No, because the range of the function is not $(-\infty, \infty)$.} \)

Corresponds to believing 1-1 means the range is all Real numbers.
\item \( \text{No, because the domain of the function is not $(-\infty, \infty)$.} \)

Corresponds to believing 1-1 means the domain is all Real numbers.
\item \( \text{Yes, the function is 1-1.} \)

Corresponds to believing the function passes the Horizontal Line test.
\item \( \text{No, because there is an $x$-value that goes to 2 different $y$-values.} \)

Corresponds to the Vertical Line test, which checks if an expression is a function.
\end{enumerate}

\textbf{General Comment:} There are only two valid options: The function is 1-1 OR No because there is a $y$-value that goes to 2 different $x$-values.
}
\litem{
Subtract the following functions, then choose the domain of the resulting function from the list below.
\[ f(x) = \frac{5}{3x-20} \text{ and } g(x) = 7x + 3 \]The solution is \( \text{ The domain is all Real numbers except } x = 6.67 \), which is option C.\begin{enumerate}[label=\Alph*.]
\item \( \text{ The domain is all Real numbers greater than or equal to } x = a, \text{ where } a \in [-7.4, 3.6] \)


\item \( \text{ The domain is all Real numbers less than or equal to } x = a, \text{ where } a \in [2, 4] \)


\item \( \text{ The domain is all Real numbers except } x = a, \text{ where } a \in [5.67, 7.67] \)


\item \( \text{ The domain is all Real numbers except } x = a \text{ and } x = b, \text{ where } a \in [-0.33, 8.67] \text{ and } b \in [2.83, 11.83] \)


\item \( \text{ The domain is all Real numbers. } \)


\end{enumerate}

\textbf{General Comment:} The new domain is the intersection of the previous domains.
}
\litem{
Find the inverse of the function below (if it exists). Then, evaluate the inverse at $x = 14$ and choose the interval that $f^-1(14)$ belongs to.
\[ f(x) = 5 x^2 - 4 \]The solution is \( \text{ The function is not invertible for all Real numbers. } \), which is option E.\begin{enumerate}[label=\Alph*.]
\item \( f^{-1}(14) \in [2.69, 3.04] \)

 Distractor 3: This corresponds to finding the (nonexistent) inverse and dividing by a negative.
\item \( f^{-1}(14) \in [5.7, 5.94] \)

 Distractor 4: This corresponds to both distractors 2 and 3.
\item \( f^{-1}(14) \in [0.88, 1.83] \)

 Distractor 2: This corresponds to finding the (nonexistent) inverse and not subtracting by the vertical shift.
\item \( f^{-1}(14) \in [1.81, 2.14] \)

 Distractor 1: This corresponds to trying to find the inverse even though the function is not 1-1. 
\item \( \text{ The function is not invertible for all Real numbers. } \)

* This is the correct option.
\end{enumerate}

\textbf{General Comment:} Be sure you check that the function is 1-1 before trying to find the inverse!
}
\end{enumerate}

\end{document}