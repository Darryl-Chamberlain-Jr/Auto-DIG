\documentclass[14pt]{extbook}
\usepackage{multicol, enumerate, enumitem, hyperref, color, soul, setspace, parskip, fancyhdr} %General Packages
\usepackage{amssymb, amsthm, amsmath, latexsym, units, mathtools} %Math Packages
\everymath{\displaystyle} %All math in Display Style
% Packages with additional options
\usepackage[headsep=0.5cm,headheight=12pt, left=1 in,right= 1 in,top= 1 in,bottom= 1 in]{geometry}
\usepackage[usenames,dvipsnames]{xcolor}
\usepackage{dashrule}  % Package to use the command below to create lines between items
\newcommand{\litem}[1]{\item#1\hspace*{-1cm}\rule{\textwidth}{0.4pt}}
\pagestyle{fancy}
\lhead{Makeup Progress Quiz 2}
\chead{}
\rhead{Version A}
\lfoot{2790-1423}
\cfoot{}
\rfoot{Summer C 2021}
\begin{document}

\begin{enumerate}
\litem{
Multiply the following functions, then choose the domain of the resulting function from the list below.\[ f(x) = 6x^{2} +x + 6 \text{ and } g(x) = \sqrt{4x+25}  \]\begin{enumerate}[label=\Alph*.]
\item \( \text{ The domain is all Real numbers except } x = a, \text{ where } a \in [2.4, 11.4] \)
\item \( \text{ The domain is all Real numbers greater than or equal to } x = a, \text{ where } a \in [-9.25, -4.25] \)
\item \( \text{ The domain is all Real numbers less than or equal to } x = a, \text{ where } a \in [-2.5, 10.5] \)
\item \( \text{ The domain is all Real numbers except } x = a \text{ and } x = b, \text{ where } a \in [5.67, 11.67] \text{ and } b \in [-7.83, -0.83] \)
\item \( \text{ The domain is all Real numbers. } \)

\end{enumerate} }
\litem{
Choose the interval below that $f$ composed with $g$ at $x=1$ is in.\[ f(x) = 3x^{3} +2 x^{2} -x -4 \text{ and } g(x) = 2x^{3} -4 x^{2} +2 x + 2 \]\begin{enumerate}[label=\Alph*.]
\item \( (f \circ g)(1) \in [34, 39] \)
\item \( (f \circ g)(1) \in [2, 4] \)
\item \( (f \circ g)(1) \in [8, 18] \)
\item \( (f \circ g)(1) \in [21, 31] \)
\item \( \text{It is not possible to compose the two functions.} \)

\end{enumerate} }
\litem{
Choose the interval below that $f$ composed with $g$ at $x=1$ is in.\[ f(x) = 2x^{3} +3 x^{2} -2 x \text{ and } g(x) = -3x^{3} -3 x^{2} +4 x \]\begin{enumerate}[label=\Alph*.]
\item \( (f \circ g)(1) \in [1, 11] \)
\item \( (f \circ g)(1) \in [-109, -103] \)
\item \( (f \circ g)(1) \in [-1, 1] \)
\item \( (f \circ g)(1) \in [-102, -90] \)
\item \( \text{It is not possible to compose the two functions.} \)

\end{enumerate} }
\litem{
Find the inverse of the function below. Then, evaluate the inverse at $x = 6$ and choose the interval that $f^-1(6)$ belongs to.\[ f(x) = \ln{(x+4)}+4 \]\begin{enumerate}[label=\Alph*.]
\item \( f^{-1}(6) \in [1.39, 5.39] \)
\item \( f^{-1}(6) \in [22019.47, 22024.47] \)
\item \( f^{-1}(6) \in [9.39, 14.39] \)
\item \( f^{-1}(6) \in [22029.47, 22035.47] \)
\item \( f^{-1}(6) \in [9.39, 14.39] \)

\end{enumerate} }
\litem{
Find the inverse of the function below (if it exists). Then, evaluate the inverse at $x = 15$ and choose the interval that $f^-1(15)$ belongs to.\[ f(x) = 2 x^2 - 3 \]\begin{enumerate}[label=\Alph*.]
\item \( f^{-1}(15) \in [5.93, 6.5] \)
\item \( f^{-1}(15) \in [2.58, 3.53] \)
\item \( f^{-1}(15) \in [4.86, 5.77] \)
\item \( f^{-1}(15) \in [2.06, 2.97] \)
\item \( \text{ The function is not invertible for all Real numbers. } \)

\end{enumerate} }
\litem{
Find the inverse of the function below. Then, evaluate the inverse at $x = 8$ and choose the interval that $f^-1(8)$ belongs to.\[ f(x) = e^{x-5}-2 \]\begin{enumerate}[label=\Alph*.]
\item \( f^{-1}(8) \in [-0.01, 1.1] \)
\item \( f^{-1}(8) \in [-3.28, -2.68] \)
\item \( f^{-1}(8) \in [-2.24, -0.83] \)
\item \( f^{-1}(8) \in [7.11, 8.12] \)
\item \( f^{-1}(8) \in [-0.33, 0.41] \)

\end{enumerate} }
\litem{
Determine whether the function below is 1-1.\[ f(x) = 36 x^2 - 204 x + 289 \]\begin{enumerate}[label=\Alph*.]
\item \( \text{No, because there is an $x$-value that goes to 2 different $y$-values.} \)
\item \( \text{Yes, the function is 1-1.} \)
\item \( \text{No, because there is a $y$-value that goes to 2 different $x$-values.} \)
\item \( \text{No, because the range of the function is not $(-\infty, \infty)$.} \)
\item \( \text{No, because the domain of the function is not $(-\infty, \infty)$.} \)

\end{enumerate} }
\litem{
Determine whether the function below is 1-1.\[ f(x) = 9 x^2 - 21 x - 228 \]\begin{enumerate}[label=\Alph*.]
\item \( \text{No, because the domain of the function is not $(-\infty, \infty)$.} \)
\item \( \text{No, because there is a $y$-value that goes to 2 different $x$-values.} \)
\item \( \text{No, because the range of the function is not $(-\infty, \infty)$.} \)
\item \( \text{Yes, the function is 1-1.} \)
\item \( \text{No, because there is an $x$-value that goes to 2 different $y$-values.} \)

\end{enumerate} }
\litem{
Add the following functions, then choose the domain of the resulting function from the list below.\[ f(x) = \sqrt{-6x+22}  \text{ and } g(x) = 8x + 3 \]\begin{enumerate}[label=\Alph*.]
\item \( \text{ The domain is all Real numbers except } x = a, \text{ where } a \in [-4.4, 2.6] \)
\item \( \text{ The domain is all Real numbers greater than or equal to } x = a, \text{ where } a \in [4.25, 6.25] \)
\item \( \text{ The domain is all Real numbers less than or equal to } x = a, \text{ where } a \in [1.67, 4.67] \)
\item \( \text{ The domain is all Real numbers except } x = a \text{ and } x = b, \text{ where } a \in [-5.83, -0.83] \text{ and } b \in [4.83, 12.83] \)
\item \( \text{ The domain is all Real numbers. } \)

\end{enumerate} }
\litem{
Find the inverse of the function below (if it exists). Then, evaluate the inverse at $x = 15$ and choose the interval that $f^-1(15)$ belongs to.\[ f(x) = \sqrt[3]{5 x - 2} \]\begin{enumerate}[label=\Alph*.]
\item \( f^{-1}(15) \in [-675.69, -675.24] \)
\item \( f^{-1}(15) \in [675.16, 675.65] \)
\item \( f^{-1}(15) \in [673.87, 674.67] \)
\item \( f^{-1}(15) \in [-675.01, -674.59] \)
\item \( \text{ The function is not invertible for all Real numbers. } \)

\end{enumerate} }
\end{enumerate}

\end{document}