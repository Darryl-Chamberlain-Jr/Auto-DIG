\documentclass{extbook}[14pt]
\usepackage{multicol, enumerate, enumitem, hyperref, color, soul, setspace, parskip, fancyhdr, amssymb, amsthm, amsmath, latexsym, units, mathtools}
\everymath{\displaystyle}
\usepackage[headsep=0.5cm,headheight=0cm, left=1 in,right= 1 in,top= 1 in,bottom= 1 in]{geometry}
\usepackage{dashrule}  % Package to use the command below to create lines between items
\newcommand{\litem}[1]{\item #1

\rule{\textwidth}{0.4pt}}
\pagestyle{fancy}
\lhead{}
\chead{Answer Key for Makeup Progress Quiz 2 Version B}
\rhead{}
\lfoot{5763-3522}
\cfoot{}
\rfoot{Spring 2021}
\begin{document}
\textbf{This key should allow you to understand why you choose the option you did (beyond just getting a question right or wrong). \href{https://xronos.clas.ufl.edu/mac1105spring2020/courseDescriptionAndMisc/Exams/LearningFromResults}{More instructions on how to use this key can be found here}.}

\textbf{If you have a suggestion to make the keys better, \href{https://forms.gle/CZkbZmPbC9XALEE88}{please fill out the short survey here}.}

\textit{Note: This key is auto-generated and may contain issues and/or errors. The keys are reviewed after each exam to ensure grading is done accurately. If there are issues (like duplicate options), they are noted in the offline gradebook. The keys are a work-in-progress to give students as many resources to improve as possible.}

\rule{\textwidth}{0.4pt}

\begin{enumerate}\litem{
Solve the linear inequality below. Then, choose the constant and interval combination that describes the solution set.
\[ -9x + 4 < 6x -10 \]The solution is \( (0.933, \infty) \), which is option C.\begin{enumerate}[label=\Alph*.]
\item \( (a, \infty), \text{ where } a \in [-2.03, 0.6] \)

 $(-0.933, \infty)$, which corresponds to negating the endpoint of the solution.
\item \( (-\infty, a), \text{ where } a \in [0.8, 1.4] \)

 $(-\infty, 0.933)$, which corresponds to switching the direction of the interval. You likely did this if you did not flip the inequality when dividing by a negative!
\item \( (a, \infty), \text{ where } a \in [-0.67, 1.5] \)

* $(0.933, \infty)$, which is the correct option.
\item \( (-\infty, a), \text{ where } a \in [-2.7, 0.3] \)

 $(-\infty, -0.933)$, which corresponds to switching the direction of the interval AND negating the endpoint. You likely did this if you did not flip the inequality when dividing by a negative as well as not moving values over to a side properly.
\item \( \text{None of the above}. \)

You may have chosen this if you thought the inequality did not match the ends of the intervals.
\end{enumerate}

\textbf{General Comment:} Remember that less/greater than or equal to includes the endpoint, while less/greater do not. Also, remember that you need to flip the inequality when you multiply or divide by a negative.
}
\litem{
Using an interval or intervals, describe all the $x$-values within or including a distance of the given values.
\[ \text{ No more than } 3 \text{ units from the number } -5. \]The solution is \( [-8, -2] \), which is option D.\begin{enumerate}[label=\Alph*.]
\item \( (-8, -2) \)

This describes the values less than 3 from -5
\item \( (-\infty, -8) \cup (-2, \infty) \)

This describes the values more than 3 from -5
\item \( (-\infty, -8] \cup [-2, \infty) \)

This describes the values no less than 3 from -5
\item \( [-8, -2] \)

This describes the values no more than 3 from -5
\item \( \text{None of the above} \)

You likely thought the values in the interval were not correct.
\end{enumerate}

\textbf{General Comment:} When thinking about this language, it helps to draw a number line and try points.
}
\litem{
Solve the linear inequality below. Then, choose the constant and interval combination that describes the solution set.
\[ -9 + 5 x > 7 x \text{ or } 7 + 6 x < 9 x \]The solution is \( (-\infty, -4.5) \text{ or } (2.333, \infty) \), which is option A.\begin{enumerate}[label=\Alph*.]
\item \( (-\infty, a) \cup (b, \infty), \text{ where } a \in [-6.2, -2.9] \text{ and } b \in [2.1, 2.7] \)

 * Correct option.
\item \( (-\infty, a] \cup [b, \infty), \text{ where } a \in [-6, -3.5] \text{ and } b \in [-3.67, 3.33] \)

Corresponds to including the endpoints (when they should be excluded).
\item \( (-\infty, a) \cup (b, \infty), \text{ where } a \in [-3.5, -1.6] \text{ and } b \in [3.5, 5.2] \)

Corresponds to inverting the inequality and negating the solution.
\item \( (-\infty, a] \cup [b, \infty), \text{ where } a \in [-2.8, -1.3] \text{ and } b \in [2.5, 6.5] \)

Corresponds to including the endpoints AND negating.
\item \( (-\infty, \infty) \)

Corresponds to the variable canceling, which does not happen in this instance.
\end{enumerate}

\textbf{General Comment:} When multiplying or dividing by a negative, flip the sign.
}
\litem{
Solve the linear inequality below. Then, choose the constant and interval combination that describes the solution set.
\[ -7x + 10 \geq 8x + 8 \]The solution is \( (-\infty, 0.133] \), which is option C.\begin{enumerate}[label=\Alph*.]
\item \( [a, \infty), \text{ where } a \in [0.06, 0.5] \)

 $[0.133, \infty)$, which corresponds to switching the direction of the interval. You likely did this if you did not flip the inequality when dividing by a negative!
\item \( (-\infty, a], \text{ where } a \in [-0.4, -0.03] \)

 $(-\infty, -0.133]$, which corresponds to negating the endpoint of the solution.
\item \( (-\infty, a], \text{ where } a \in [0.02, 0.49] \)

* $(-\infty, 0.133]$, which is the correct option.
\item \( [a, \infty), \text{ where } a \in [-0.3, 0.11] \)

 $[-0.133, \infty)$, which corresponds to switching the direction of the interval AND negating the endpoint. You likely did this if you did not flip the inequality when dividing by a negative as well as not moving values over to a side properly.
\item \( \text{None of the above}. \)

You may have chosen this if you thought the inequality did not match the ends of the intervals.
\end{enumerate}

\textbf{General Comment:} Remember that less/greater than or equal to includes the endpoint, while less/greater do not. Also, remember that you need to flip the inequality when you multiply or divide by a negative.
}
\litem{
Using an interval or intervals, describe all the $x$-values within or including a distance of the given values.
\[ \text{ More than } 5 \text{ units from the number } 7. \]The solution is \( (-\infty, 2) \cup (12, \infty) \), which is option D.\begin{enumerate}[label=\Alph*.]
\item \( (-\infty, 2] \cup [12, \infty) \)

This describes the values no less than 5 from 7
\item \( (2, 12) \)

This describes the values less than 5 from 7
\item \( [2, 12] \)

This describes the values no more than 5 from 7
\item \( (-\infty, 2) \cup (12, \infty) \)

This describes the values more than 5 from 7
\item \( \text{None of the above} \)

You likely thought the values in the interval were not correct.
\end{enumerate}

\textbf{General Comment:} When thinking about this language, it helps to draw a number line and try points.
}
\litem{
Solve the linear inequality below. Then, choose the constant and interval combination that describes the solution set.
\[ \frac{-9}{2} - \frac{5}{4} x \geq \frac{-4}{7} x + \frac{7}{6} \]The solution is \( (-\infty, -8.351] \), which is option C.\begin{enumerate}[label=\Alph*.]
\item \( [a, \infty), \text{ where } a \in [4.35, 11.35] \)

 $[8.351, \infty)$, which corresponds to switching the direction of the interval AND negating the endpoint. You likely did this if you did not flip the inequality when dividing by a negative as well as not moving values over to a side properly.
\item \( (-\infty, a], \text{ where } a \in [6.35, 9.35] \)

 $(-\infty, 8.351]$, which corresponds to negating the endpoint of the solution.
\item \( (-\infty, a], \text{ where } a \in [-11.35, -6.35] \)

* $(-\infty, -8.351]$, which is the correct option.
\item \( [a, \infty), \text{ where } a \in [-9.35, -5.35] \)

 $[-8.351, \infty)$, which corresponds to switching the direction of the interval. You likely did this if you did not flip the inequality when dividing by a negative!
\item \( \text{None of the above}. \)

You may have chosen this if you thought the inequality did not match the ends of the intervals.
\end{enumerate}

\textbf{General Comment:} Remember that less/greater than or equal to includes the endpoint, while less/greater do not. Also, remember that you need to flip the inequality when you multiply or divide by a negative.
}
\litem{
Solve the linear inequality below. Then, choose the constant and interval combination that describes the solution set.
\[ -4 - 7 x \leq \frac{-61 x - 5}{9} < 5 - 7 x \]The solution is \( \text{None of the above.} \), which is option E.\begin{enumerate}[label=\Alph*.]
\item \( (-\infty, a) \cup [b, \infty), \text{ where } a \in [14.5, 21.5] \text{ and } b \in [-28, -21] \)

$(-\infty, 15.50) \cup [-25.00, \infty)$, which corresponds to displaying the and-inequality as an or-inequality AND flipping the inequality AND getting negatives of the actual endpoints.
\item \( [a, b), \text{ where } a \in [15.5, 19.5] \text{ and } b \in [-27, -22] \)

$[15.50, -25.00)$, which is the correct interval but negatives of the actual endpoints.
\item \( (a, b], \text{ where } a \in [14.5, 18.5] \text{ and } b \in [-29, -17] \)

$(15.50, -25.00]$, which corresponds to flipping the inequality and getting negatives of the actual endpoints.
\item \( (-\infty, a] \cup (b, \infty), \text{ where } a \in [13.5, 19.5] \text{ and } b \in [-27, -19] \)

$(-\infty, 15.50] \cup (-25.00, \infty)$, which corresponds to displaying the and-inequality as an or-inequality and getting negatives of the actual endpoints.
\item \( \text{None of the above.} \)

* This is correct as the answer should be $[-15.50, 25.00)$.
\end{enumerate}

\textbf{General Comment:} To solve, you will need to break up the compound inequality into two inequalities. Be sure to keep track of the inequality! It may be best to draw a number line and graph your solution.
}
\litem{
Solve the linear inequality below. Then, choose the constant and interval combination that describes the solution set.
\[ 3 - 8 x < \frac{-46 x + 7}{6} \leq 4 - 8 x \]The solution is \( \text{None of the above.} \), which is option E.\begin{enumerate}[label=\Alph*.]
\item \( (a, b], \text{ where } a \in [-8.5, -3.5] \text{ and } b \in [-14.5, -4.5] \)

$(-5.50, -8.50]$, which is the correct interval but negatives of the actual endpoints.
\item \( (-\infty, a] \cup (b, \infty), \text{ where } a \in [-7.5, -0.5] \text{ and } b \in [-11.5, -2.5] \)

$(-\infty, -5.50] \cup (-8.50, \infty)$, which corresponds to displaying the and-inequality as an or-inequality AND flipping the inequality AND getting negatives of the actual endpoints.
\item \( [a, b), \text{ where } a \in [-5.5, -0.5] \text{ and } b \in [-9.5, -3.5] \)

$[-5.50, -8.50)$, which corresponds to flipping the inequality and getting negatives of the actual endpoints.
\item \( (-\infty, a) \cup [b, \infty), \text{ where } a \in [-8.5, -1.5] \text{ and } b \in [-9.5, -4.5] \)

$(-\infty, -5.50) \cup [-8.50, \infty)$, which corresponds to displaying the and-inequality as an or-inequality and getting negatives of the actual endpoints.
\item \( \text{None of the above.} \)

* This is correct as the answer should be $(5.50, 8.50]$.
\end{enumerate}

\textbf{General Comment:} To solve, you will need to break up the compound inequality into two inequalities. Be sure to keep track of the inequality! It may be best to draw a number line and graph your solution.
}
\litem{
Solve the linear inequality below. Then, choose the constant and interval combination that describes the solution set.
\[ -7 + 7 x > 8 x \text{ or } -5 + 9 x < 10 x \]The solution is \( (-\infty, -7.0) \text{ or } (-5.0, \infty) \), which is option B.\begin{enumerate}[label=\Alph*.]
\item \( (-\infty, a] \cup [b, \infty), \text{ where } a \in [4, 8] \text{ and } b \in [3, 10] \)

Corresponds to including the endpoints AND negating.
\item \( (-\infty, a) \cup (b, \infty), \text{ where } a \in [-10, -3] \text{ and } b \in [-6, 0] \)

 * Correct option.
\item \( (-\infty, a] \cup [b, \infty), \text{ where } a \in [-8, -6] \text{ and } b \in [-8, 6] \)

Corresponds to including the endpoints (when they should be excluded).
\item \( (-\infty, a) \cup (b, \infty), \text{ where } a \in [4, 9] \text{ and } b \in [7, 9] \)

Corresponds to inverting the inequality and negating the solution.
\item \( (-\infty, \infty) \)

Corresponds to the variable canceling, which does not happen in this instance.
\end{enumerate}

\textbf{General Comment:} When multiplying or dividing by a negative, flip the sign.
}
\litem{
Solve the linear inequality below. Then, choose the constant and interval combination that describes the solution set.
\[ \frac{-10}{4} + \frac{7}{6} x < \frac{9}{9} x + \frac{7}{7} \]The solution is \( (-\infty, 21.0) \), which is option C.\begin{enumerate}[label=\Alph*.]
\item \( (a, \infty), \text{ where } a \in [-24, -20] \)

 $(-21.0, \infty)$, which corresponds to switching the direction of the interval AND negating the endpoint. You likely did this if you did not flip the inequality when dividing by a negative as well as not moving values over to a side properly.
\item \( (a, \infty), \text{ where } a \in [20, 25] \)

 $(21.0, \infty)$, which corresponds to switching the direction of the interval. You likely did this if you did not flip the inequality when dividing by a negative!
\item \( (-\infty, a), \text{ where } a \in [21, 23] \)

* $(-\infty, 21.0)$, which is the correct option.
\item \( (-\infty, a), \text{ where } a \in [-21, -18] \)

 $(-\infty, -21.0)$, which corresponds to negating the endpoint of the solution.
\item \( \text{None of the above}. \)

You may have chosen this if you thought the inequality did not match the ends of the intervals.
\end{enumerate}

\textbf{General Comment:} Remember that less/greater than or equal to includes the endpoint, while less/greater do not. Also, remember that you need to flip the inequality when you multiply or divide by a negative.
}
\end{enumerate}

\end{document}