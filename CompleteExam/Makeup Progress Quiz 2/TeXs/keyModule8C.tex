\documentclass{extbook}[14pt]
\usepackage{multicol, enumerate, enumitem, hyperref, color, soul, setspace, parskip, fancyhdr, amssymb, amsthm, amsmath, latexsym, units, mathtools}
\everymath{\displaystyle}
\usepackage[headsep=0.5cm,headheight=0cm, left=1 in,right= 1 in,top= 1 in,bottom= 1 in]{geometry}
\usepackage{dashrule}  % Package to use the command below to create lines between items
\newcommand{\litem}[1]{\item #1

\rule{\textwidth}{0.4pt}}
\pagestyle{fancy}
\lhead{}
\chead{Answer Key for Makeup Progress Quiz 2 Version C}
\rhead{}
\lfoot{2790-1423}
\cfoot{}
\rfoot{Summer C 2021}
\begin{document}
\textbf{This key should allow you to understand why you choose the option you did (beyond just getting a question right or wrong). \href{https://xronos.clas.ufl.edu/mac1105spring2020/courseDescriptionAndMisc/Exams/LearningFromResults}{More instructions on how to use this key can be found here}.}

\textbf{If you have a suggestion to make the keys better, \href{https://forms.gle/CZkbZmPbC9XALEE88}{please fill out the short survey here}.}

\textit{Note: This key is auto-generated and may contain issues and/or errors. The keys are reviewed after each exam to ensure grading is done accurately. If there are issues (like duplicate options), they are noted in the offline gradebook. The keys are a work-in-progress to give students as many resources to improve as possible.}

\rule{\textwidth}{0.4pt}

\begin{enumerate}\litem{
 Solve the equation for $x$ and choose the interval that contains $x$ (if it exists).
\[  9 = \ln{\sqrt[3]{\frac{23}{e^{8x}}}} \]The solution is \( x = -2.983, \text{ which does not fit in any of the interval options.} \), which is option E.\begin{enumerate}[label=\Alph*.]
\item \( x \in [1.8, 4.3] \)

$x = 2.983$, which is the negative of the correct solution.
\item \( x \in [-2.5, -1.7] \)

$x = -1.858$, which corresponds to treating any root as a square root.
\item \( x \in [-1.7, -0.3] \)

$x = -1.216$, which corresponds to thinking you need to take the natural log of the left side before reducing.
\item \( \text{There is no Real solution to the equation.} \)

This corresponds to believing you cannot solve the equation.
\item \( \text{None of the above.} \)

*$x = -2.983$ is the correct solution and does not fit in any of the other intervals.
\end{enumerate}

\textbf{General Comment:} \textbf{General Comments}: After using the properties of logarithmic functions to break up the right-hand side, use $\ln(e) = 1$ to reduce the question to a linear function to solve. You can put $\ln(23)$ into a calculator if you are having trouble.
}
\litem{
Which of the following intervals describes the Domain of the function below?
\[ f(x) = e^{x+8}-2 \]The solution is \( (-\infty, \infty) \), which is option E.\begin{enumerate}[label=\Alph*.]
\item \( [a, \infty), a \in [1, 4] \)

$[2, \infty)$, which corresponds to using the negative vertical shift AND flipping the Range interval AND including the endpoint.
\item \( (-\infty, a), a \in [-6, 0] \)

$(-\infty, -2)$, which corresponds to using the correct vertical shift *if we wanted the Range*.
\item \( (a, \infty), a \in [1, 4] \)

$(2, \infty)$, which corresponds to using the negative vertical shift AND flipping the Range interval.
\item \( (-\infty, a], a \in [-6, 0] \)

$(-\infty, -2]$, which corresponds to using the correct vertical shift *if we wanted the Range* AND including the endpoint.
\item \( (-\infty, \infty) \)

* This is the correct option.
\end{enumerate}

\textbf{General Comment:} \textbf{General Comments}: Domain of a basic exponential function is $(-\infty, \infty)$ while the Range is $(0, \infty)$. We can shift these intervals [and even flip when $a<0$!] to find the new Domain/Range.
}
\litem{
Which of the following intervals describes the Domain of the function below?
\[ f(x) = \log_2{(x+2)}-6 \]The solution is \( (-2, \infty) \), which is option D.\begin{enumerate}[label=\Alph*.]
\item \( (-\infty, a), a \in [-1.7, 4.3] \)

$(-\infty, 2)$, which corresponds to flipping the Domain. Remember: the general for is $a*\log(x-h)+k$, \textbf{where $a$ does not affect the domain}.
\item \( (-\infty, a], a \in [3.9, 7.6] \)

$(-\infty, 6]$, which corresponds to using the negative vertical shift AND including the endpoint AND flipping the domain.
\item \( [a, \infty), a \in [-6.3, -5.2] \)

$[-6, \infty)$, which corresponds to using the vertical shift when shifting the Domain AND including the endpoint.
\item \( (a, \infty), a \in [-2.3, -1] \)

* $(-2, \infty)$, which is the correct option.
\item \( (-\infty, \infty) \)

This corresponds to thinking of the range of the log function (or the domain of the exponential function).
\end{enumerate}

\textbf{General Comment:} \textbf{General Comments}: The domain of a basic logarithmic function is $(0, \infty)$ and the Range is $(-\infty, \infty)$. We can use shifts when finding the Domain, but the Range will always be all Real numbers.
}
\litem{
Which of the following intervals describes the Range of the function below?
\[ f(x) = -e^{x+2}-3 \]The solution is \( (-\infty, -3) \), which is option D.\begin{enumerate}[label=\Alph*.]
\item \( (-\infty, a], a \in [-7, -2] \)

$(-\infty, -3]$, which corresponds to including the endpoint.
\item \( (a, \infty), a \in [3, 6] \)

$(3, \infty)$, which corresponds to using the negative vertical shift AND flipping the Range interval.
\item \( [a, \infty), a \in [3, 6] \)

$[3, \infty)$, which corresponds to using the negative vertical shift AND flipping the Range interval AND including the endpoint.
\item \( (-\infty, a), a \in [-7, -2] \)

* $(-\infty, -3)$, which is the correct option.
\item \( (-\infty, \infty) \)

This corresponds to confusing range of an exponential function with the domain of an exponential function.
\end{enumerate}

\textbf{General Comment:} \textbf{General Comments}: Domain of a basic exponential function is $(-\infty, \infty)$ while the Range is $(0, \infty)$. We can shift these intervals [and even flip when $a<0$!] to find the new Domain/Range.
}
\litem{
Solve the equation for $x$ and choose the interval that contains the solution (if it exists).
\[ 5^{3x+3} = \left(\frac{1}{343}\right)^{2x-3} \]The solution is \( x = 0.769 \), which is option D.\begin{enumerate}[label=\Alph*.]
\item \( x \in [7.68, 14.68] \)

$x = 12.685$, which corresponds to distributing the $\ln(base)$ to the second term of the exponent only.
\item \( x \in [-2.36, 0.64] \)

$x = -0.364$, which corresponds to distributing the $\ln(base)$ to the first term of the exponent only.
\item \( x \in [-8, -3] \)

$x = -6.000$, which corresponds to solving the numerators as equal while ignoring the bases are different.
\item \( x \in [-0.23, 1.77] \)

* $x = 0.769$, which is the correct option.
\item \( \text{There is no Real solution to the equation.} \)

This corresponds to believing there is no solution since the bases are not powers of each other.
\end{enumerate}

\textbf{General Comment:} \textbf{General Comments:} This question was written so that the bases could not be written the same. You will need to take the log of both sides.
}
\litem{
Solve the equation for $x$ and choose the interval that contains the solution (if it exists).
\[ \log_{3}{(-3x+6)}+4 = 2 \]The solution is \( x = 1.963 \), which is option D.\begin{enumerate}[label=\Alph*.]
\item \( x \in [-1.36, -0.16] \)

$x = -1.000$, which corresponds to ignoring the vertical shift when converting to exponential form.
\item \( x \in [-0.37, 1.1] \)

$x = 0.667$, which corresponds to reversing the base and exponent when converting and reversing the value with $x$.
\item \( x \in [4.58, 6] \)

$x = 4.667$, which corresponds to reversing the base and exponent when converting.
\item \( x \in [1.48, 2.81] \)

* $x = 1.963$, which is the correct option.
\item \( \text{There is no Real solution to the equation.} \)

Corresponds to believing a negative coefficient within the log equation means there is no Real solution.
\end{enumerate}

\textbf{General Comment:} \textbf{General Comments:} First, get the equation in the form $\log_b{(cx+d)} = a$. Then, convert to $b^a = cx+d$ and solve.
}
\litem{
Solve the equation for $x$ and choose the interval that contains the solution (if it exists).
\[ 5^{5x-4} = \left(\frac{1}{9}\right)^{-4x-3} \]The solution is \( x = -17.567 \), which is option D.\begin{enumerate}[label=\Alph*.]
\item \( x \in [-0.89, 1.11] \)

$x = 0.111$, which corresponds to solving the numerators as equal while ignoring the bases are different.
\item \( x \in [0.45, 2.45] \)

$x = 1.448$, which corresponds to distributing the $\ln(base)$ to the second term of the exponent only.
\item \( x \in [-3.35, -0.35] \)

$x = -1.348$, which corresponds to distributing the $\ln(base)$ to the first term of the exponent only.
\item \( x \in [-19.57, -14.57] \)

* $x = -17.567$, which is the correct option.
\item \( \text{There is no Real solution to the equation.} \)

This corresponds to believing there is no solution since the bases are not powers of each other.
\end{enumerate}

\textbf{General Comment:} \textbf{General Comments:} This question was written so that the bases could not be written the same. You will need to take the log of both sides.
}
\litem{
Which of the following intervals describes the Domain of the function below?
\[ f(x) = -\log_2{(x-8)}-2 \]The solution is \( (8, \infty) \), which is option B.\begin{enumerate}[label=\Alph*.]
\item \( (-\infty, a), a \in [-11.2, -6.8] \)

$(-\infty, -8)$, which corresponds to flipping the Domain. Remember: the general for is $a*\log(x-h)+k$, \textbf{where $a$ does not affect the domain}.
\item \( (a, \infty), a \in [6.5, 8.3] \)

* $(8, \infty)$, which is the correct option.
\item \( [a, \infty), a \in [-4.5, -1.7] \)

$[-2, \infty)$, which corresponds to using the vertical shift when shifting the Domain AND including the endpoint.
\item \( (-\infty, a], a \in [0.3, 4] \)

$(-\infty, 2]$, which corresponds to using the negative vertical shift AND including the endpoint AND flipping the domain.
\item \( (-\infty, \infty) \)

This corresponds to thinking of the range of the log function (or the domain of the exponential function).
\end{enumerate}

\textbf{General Comment:} \textbf{General Comments}: The domain of a basic logarithmic function is $(0, \infty)$ and the Range is $(-\infty, \infty)$. We can use shifts when finding the Domain, but the Range will always be all Real numbers.
}
\litem{
Solve the equation for $x$ and choose the interval that contains the solution (if it exists).
\[ \log_{3}{(2x+7)}+6 = 3 \]The solution is \( x = -3.481 \), which is option C.\begin{enumerate}[label=\Alph*.]
\item \( x \in [9, 12] \)

$x = 10.000$, which corresponds to ignoring the vertical shift when converting to exponential form.
\item \( x \in [-11, -9] \)

$x = -10.000$, which corresponds to reversing the base and exponent when converting and reversing the value with $x$.
\item \( x \in [-4.48, -2.48] \)

* $x = -3.481$, which is the correct option.
\item \( x \in [-18, -15] \)

$x = -17.000$, which corresponds to reversing the base and exponent when converting.
\item \( \text{There is no Real solution to the equation.} \)

Corresponds to believing a negative coefficient within the log equation means there is no Real solution.
\end{enumerate}

\textbf{General Comment:} \textbf{General Comments:} First, get the equation in the form $\log_b{(cx+d)} = a$. Then, convert to $b^a = cx+d$ and solve.
}
\litem{
 Solve the equation for $x$ and choose the interval that contains $x$ (if it exists).
\[  23 = \ln{\sqrt[5]{\frac{6}{e^{8x}}}} \]The solution is \( x = -14.151, \text{ which does not fit in any of the interval options.} \), which is option E.\begin{enumerate}[label=\Alph*.]
\item \( x \in [-4.18, -1.18] \)

$x = -2.184$, which corresponds to thinking you need to take the natural log of the left side before reducing.
\item \( x \in [-6.53, -2.53] \)

$x = -5.526$, which corresponds to treating any root as a square root.
\item \( x \in [12.15, 17.15] \)

$x = 14.151$, which is the negative of the correct solution.
\item \( \text{There is no Real solution to the equation.} \)

This corresponds to believing you cannot solve the equation.
\item \( \text{None of the above.} \)

*$x = -14.151$ is the correct solution and does not fit in any of the other intervals.
\end{enumerate}

\textbf{General Comment:} \textbf{General Comments}: After using the properties of logarithmic functions to break up the right-hand side, use $\ln(e) = 1$ to reduce the question to a linear function to solve. You can put $\ln(6)$ into a calculator if you are having trouble.
}
\end{enumerate}

\end{document}