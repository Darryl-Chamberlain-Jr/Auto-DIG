\documentclass{extbook}[14pt]
\usepackage{multicol, enumerate, enumitem, hyperref, color, soul, setspace, parskip, fancyhdr, amssymb, amsthm, amsmath, latexsym, units, mathtools}
\everymath{\displaystyle}
\usepackage[headsep=0.5cm,headheight=0cm, left=1 in,right= 1 in,top= 1 in,bottom= 1 in]{geometry}
\usepackage{dashrule}  % Package to use the command below to create lines between items
\newcommand{\litem}[1]{\item #1

\rule{\textwidth}{0.4pt}}
\pagestyle{fancy}
\lhead{}
\chead{Answer Key for Makeup Progress Quiz 2 Version B}
\rhead{}
\lfoot{2790-1423}
\cfoot{}
\rfoot{Summer C 2021}
\begin{document}
\textbf{This key should allow you to understand why you choose the option you did (beyond just getting a question right or wrong). \href{https://xronos.clas.ufl.edu/mac1105spring2020/courseDescriptionAndMisc/Exams/LearningFromResults}{More instructions on how to use this key can be found here}.}

\textbf{If you have a suggestion to make the keys better, \href{https://forms.gle/CZkbZmPbC9XALEE88}{please fill out the short survey here}.}

\textit{Note: This key is auto-generated and may contain issues and/or errors. The keys are reviewed after each exam to ensure grading is done accurately. If there are issues (like duplicate options), they are noted in the offline gradebook. The keys are a work-in-progress to give students as many resources to improve as possible.}

\rule{\textwidth}{0.4pt}

\begin{enumerate}\litem{
 Solve the equation for $x$ and choose the interval that contains $x$ (if it exists).
\[  11 = \sqrt[4]{\frac{26}{e^{6x}}} \]The solution is \( x = -1.056, \text{ which does not fit in any of the interval options.} \), which is option E.\begin{enumerate}[label=\Alph*.]
\item \( x \in [0.26, 1.95] \)

$x = 1.056$, which is the negative of the correct solution.
\item \( x \in [-8.49, -7] \)

$x = -7.876$, which corresponds to thinking you don't need to take the natural log of both sides before reducing, as if the right side already has a natural log.
\item \( x \in [-0.64, -0.01] \)

$x = -0.256$, which corresponds to treating any root as a square root.
\item \( \text{There is no Real solution to the equation.} \)

This corresponds to believing you cannot solve the equation.
\item \( \text{None of the above.} \)

* $x = -1.056$ is the correct solution and does not fit in any of the other intervals.
\end{enumerate}

\textbf{General Comment:} \textbf{General Comments}: After using the properties of logarithmic functions to break up the right-hand side, use $\ln(e) = 1$ to reduce the question to a linear function to solve. You can put $\ln(26)$ into a calculator if you are having trouble.
}
\litem{
Which of the following intervals describes the Range of the function below?
\[ f(x) = -e^{x-2}-1 \]The solution is \( (-\infty, -1) \), which is option D.\begin{enumerate}[label=\Alph*.]
\item \( (a, \infty), a \in [1, 5] \)

$(1, \infty)$, which corresponds to using the negative vertical shift AND flipping the Range interval.
\item \( (-\infty, a], a \in [-4, 0] \)

$(-\infty, -1]$, which corresponds to including the endpoint.
\item \( [a, \infty), a \in [1, 5] \)

$[1, \infty)$, which corresponds to using the negative vertical shift AND flipping the Range interval AND including the endpoint.
\item \( (-\infty, a), a \in [-4, 0] \)

* $(-\infty, -1)$, which is the correct option.
\item \( (-\infty, \infty) \)

This corresponds to confusing range of an exponential function with the domain of an exponential function.
\end{enumerate}

\textbf{General Comment:} \textbf{General Comments}: Domain of a basic exponential function is $(-\infty, \infty)$ while the Range is $(0, \infty)$. We can shift these intervals [and even flip when $a<0$!] to find the new Domain/Range.
}
\litem{
Which of the following intervals describes the Range of the function below?
\[ f(x) = -\log_2{(x+8)}+5 \]The solution is \( (\infty, \infty) \), which is option E.\begin{enumerate}[label=\Alph*.]
\item \( (-\infty, a), a \in [-6.9, -2.1] \)

$(-\infty, -5)$, which corresponds to using the using the negative of vertical shift on $(0, \infty)$.
\item \( [a, \infty), a \in [-10.3, -7.3] \)

$[5, \infty)$, which corresponds to using the flipped Domain AND including the endpoint.
\item \( (-\infty, a), a \in [3.4, 5.4] \)

$(-\infty, 5)$, which corresponds to using the vertical shift while the Range is $(-\infty, \infty)$.
\item \( [a, \infty), a \in [7.8, 9.1] \)

$[8, \infty)$, which corresponds to using the negative of the horizontal shift AND including the endpoint.
\item \( (-\infty, \infty) \)

*This is the correct option.
\end{enumerate}

\textbf{General Comment:} \textbf{General Comments}: The domain of a basic logarithmic function is $(0, \infty)$ and the Range is $(-\infty, \infty)$. We can use shifts when finding the Domain, but the Range will always be all Real numbers.
}
\litem{
Which of the following intervals describes the Domain of the function below?
\[ f(x) = e^{x-6}-4 \]The solution is \( (-\infty, \infty) \), which is option E.\begin{enumerate}[label=\Alph*.]
\item \( (a, \infty), a \in [0, 6] \)

$(4, \infty)$, which corresponds to using the negative vertical shift AND flipping the Range interval.
\item \( [a, \infty), a \in [0, 6] \)

$[4, \infty)$, which corresponds to using the negative vertical shift AND flipping the Range interval AND including the endpoint.
\item \( (-\infty, a), a \in [-8, -1] \)

$(-\infty, -4)$, which corresponds to using the correct vertical shift *if we wanted the Range*.
\item \( (-\infty, a], a \in [-8, -1] \)

$(-\infty, -4]$, which corresponds to using the correct vertical shift *if we wanted the Range* AND including the endpoint.
\item \( (-\infty, \infty) \)

* This is the correct option.
\end{enumerate}

\textbf{General Comment:} \textbf{General Comments}: Domain of a basic exponential function is $(-\infty, \infty)$ while the Range is $(0, \infty)$. We can shift these intervals [and even flip when $a<0$!] to find the new Domain/Range.
}
\litem{
Solve the equation for $x$ and choose the interval that contains the solution (if it exists).
\[ 5^{-2x+3} = 49^{-5x-5} \]The solution is \( x = -1.496 \), which is option C.\begin{enumerate}[label=\Alph*.]
\item \( x \in [-8.36, -7.3] \)

$x = -8.096$, which corresponds to distributing the $\ln(base)$ to the second term of the exponent only.
\item \( x \in [-3.24, -2.17] \)

$x = -2.667$, which corresponds to solving the numerators as equal while ignoring the bases are different.
\item \( x \in [-1.85, -0.86] \)

* $x = -1.496$, which is the correct option.
\item \( x \in [-0.86, -0.2] \)

$x = -0.493$, which corresponds to distributing the $\ln(base)$ to the first term of the exponent only.
\item \( \text{There is no Real solution to the equation.} \)

This corresponds to believing there is no solution since the bases are not powers of each other.
\end{enumerate}

\textbf{General Comment:} \textbf{General Comments:} This question was written so that the bases could not be written the same. You will need to take the log of both sides.
}
\litem{
Solve the equation for $x$ and choose the interval that contains the solution (if it exists).
\[ \log_{5}{(-3x+7)}+5 = 3 \]The solution is \( x = 2.320 \), which is option C.\begin{enumerate}[label=\Alph*.]
\item \( x \in [12, 17] \)

$x = 13.000$, which corresponds to reversing the base and exponent when converting.
\item \( x \in [-40.33, -37.33] \)

$x = -39.333$, which corresponds to ignoring the vertical shift when converting to exponential form.
\item \( x \in [-2.68, 4.32] \)

* $x = 2.320$, which is the correct option.
\item \( x \in [3.33, 10.33] \)

$x = 8.333$, which corresponds to reversing the base and exponent when converting and reversing the value with $x$.
\item \( \text{There is no Real solution to the equation.} \)

Corresponds to believing a negative coefficient within the log equation means there is no Real solution.
\end{enumerate}

\textbf{General Comment:} \textbf{General Comments:} First, get the equation in the form $\log_b{(cx+d)} = a$. Then, convert to $b^a = cx+d$ and solve.
}
\litem{
Solve the equation for $x$ and choose the interval that contains the solution (if it exists).
\[ 4^{-5x+5} = \left(\frac{1}{27}\right)^{3x-5} \]The solution is \( x = 3.230 \), which is option A.\begin{enumerate}[label=\Alph*.]
\item \( x \in [2.4, 3.4] \)

* $x = 3.230$, which is the correct option.
\item \( x \in [-2.5, -0.1] \)

$x = -1.193$, which corresponds to distributing the $\ln(base)$ to the second term of the exponent only.
\item \( x \in [0, 1.9] \)

$x = 1.250$, which corresponds to solving the numerators as equal while ignoring the bases are different.
\item \( x \in [-6.2, -2.3] \)

$x = -3.383$, which corresponds to distributing the $\ln(base)$ to the first term of the exponent only.
\item \( \text{There is no Real solution to the equation.} \)

This corresponds to believing there is no solution since the bases are not powers of each other.
\end{enumerate}

\textbf{General Comment:} \textbf{General Comments:} This question was written so that the bases could not be written the same. You will need to take the log of both sides.
}
\litem{
Which of the following intervals describes the Domain of the function below?
\[ f(x) = -\log_2{(x+1)}+2 \]The solution is \( (-1, \infty) \), which is option D.\begin{enumerate}[label=\Alph*.]
\item \( (-\infty, a), a \in [-0.51, 1.05] \)

$(-\infty, 1)$, which corresponds to flipping the Domain. Remember: the general for is $a*\log(x-h)+k$, \textbf{where $a$ does not affect the domain}.
\item \( [a, \infty), a \in [1.4, 2.04] \)

$[2, \infty)$, which corresponds to using the vertical shift when shifting the Domain AND including the endpoint.
\item \( (-\infty, a], a \in [-2.92, -1.94] \)

$(-\infty, -2]$, which corresponds to using the negative vertical shift AND including the endpoint AND flipping the domain.
\item \( (a, \infty), a \in [-1.37, 0.82] \)

* $(-1, \infty)$, which is the correct option.
\item \( (-\infty, \infty) \)

This corresponds to thinking of the range of the log function (or the domain of the exponential function).
\end{enumerate}

\textbf{General Comment:} \textbf{General Comments}: The domain of a basic logarithmic function is $(0, \infty)$ and the Range is $(-\infty, \infty)$. We can use shifts when finding the Domain, but the Range will always be all Real numbers.
}
\litem{
Solve the equation for $x$ and choose the interval that contains the solution (if it exists).
\[ \log_{2}{(4x+5)}+4 = 2 \]The solution is \( x = -1.188 \), which is option B.\begin{enumerate}[label=\Alph*.]
\item \( x \in [-0.93, -0.12] \)

$x = -0.250$, which corresponds to ignoring the vertical shift when converting to exponential form.
\item \( x \in [-2.18, -0.77] \)

* $x = -1.188$, which is the correct option.
\item \( x \in [2.16, 2.6] \)

$x = 2.250$, which corresponds to reversing the base and exponent when converting and reversing the value with $x$.
\item \( x \in [-0.93, -0.12] \)

$x = -0.250$, which corresponds to reversing the base and exponent when converting.
\item \( \text{There is no Real solution to the equation.} \)

Corresponds to believing a negative coefficient within the log equation means there is no Real solution.
\end{enumerate}

\textbf{General Comment:} \textbf{General Comments:} First, get the equation in the form $\log_b{(cx+d)} = a$. Then, convert to $b^a = cx+d$ and solve.
}
\litem{
 Solve the equation for $x$ and choose the interval that contains $x$ (if it exists).
\[  16 = \ln{\sqrt[4]{\frac{9}{e^{7x}}}} \]The solution is \( x = -8.829, \text{ which does not fit in any of the interval options.} \), which is option E.\begin{enumerate}[label=\Alph*.]
\item \( x \in [5.83, 9.83] \)

$x = 8.829$, which is the negative of the correct solution.
\item \( x \in [-7.26, -3.26] \)

$x = -4.258$, which corresponds to treating any root as a square root.
\item \( x \in [-3.9, 0.1] \)

$x = -1.898$, which corresponds to thinking you need to take the natural log of the left side before reducing.
\item \( \text{There is no Real solution to the equation.} \)

This corresponds to believing you cannot solve the equation.
\item \( \text{None of the above.} \)

*$x = -8.829$ is the correct solution and does not fit in any of the other intervals.
\end{enumerate}

\textbf{General Comment:} \textbf{General Comments}: After using the properties of logarithmic functions to break up the right-hand side, use $\ln(e) = 1$ to reduce the question to a linear function to solve. You can put $\ln(9)$ into a calculator if you are having trouble.
}
\end{enumerate}

\end{document}