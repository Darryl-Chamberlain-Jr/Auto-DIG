\documentclass[14pt]{extbook}
\usepackage{multicol, enumerate, enumitem, hyperref, color, soul, setspace, parskip, fancyhdr} %General Packages
\usepackage{amssymb, amsthm, amsmath, latexsym, units, mathtools} %Math Packages
\everymath{\displaystyle} %All math in Display Style
% Packages with additional options
\usepackage[headsep=0.5cm,headheight=12pt, left=1 in,right= 1 in,top= 1 in,bottom= 1 in]{geometry}
\usepackage[usenames,dvipsnames]{xcolor}
\usepackage{dashrule}  % Package to use the command below to create lines between items
\newcommand{\litem}[1]{\item#1\hspace*{-1cm}\rule{\textwidth}{0.4pt}}
\pagestyle{fancy}
\lhead{Makeup Progress Quiz 2}
\chead{}
\rhead{Version B}
\lfoot{2790-1423}
\cfoot{}
\rfoot{Summer C 2021}
\begin{document}

\begin{enumerate}
\litem{
What are the \textit{possible Rational} roots of the polynomial below?\[ f(x) = 7x^{3} +7 x^{2} +3 x + 5 \]\begin{enumerate}[label=\Alph*.]
\item \( \text{ All combinations of: }\frac{\pm 1,\pm 5}{\pm 1,\pm 7} \)
\item \( \pm 1,\pm 7 \)
\item \( \pm 1,\pm 5 \)
\item \( \text{ All combinations of: }\frac{\pm 1,\pm 7}{\pm 1,\pm 5} \)
\item \( \text{ There is no formula or theorem that tells us all possible Rational roots.} \)

\end{enumerate} }
\litem{
What are the \textit{possible Rational} roots of the polynomial below?\[ f(x) = 4x^{4} +3 x^{3} +3 x^{2} +3 x + 6 \]\begin{enumerate}[label=\Alph*.]
\item \( \text{ All combinations of: }\frac{\pm 1,\pm 2,\pm 3,\pm 6}{\pm 1,\pm 2,\pm 4} \)
\item \( \text{ All combinations of: }\frac{\pm 1,\pm 2,\pm 4}{\pm 1,\pm 2,\pm 3,\pm 6} \)
\item \( \pm 1,\pm 2,\pm 4 \)
\item \( \pm 1,\pm 2,\pm 3,\pm 6 \)
\item \( \text{ There is no formula or theorem that tells us all possible Rational roots.} \)

\end{enumerate} }
\litem{
Factor the polynomial below completely, knowing that $x + 5$ is a factor. Then, choose the intervals the zeros of the polynomial belong to, where $z_1 \leq z_2 \leq z_3 \leq z_4$. \textit{To make the problem easier, all zeros are between -5 and 5.}\[ f(x) = 9x^{4} +27 x^{3} -127 x^{2} -155 x + 150 \]\begin{enumerate}[label=\Alph*.]
\item \( z_1 \in [-5.2, -4.9], \text{   }  z_2 \in [-1.81, -1.57], z_3 \in [0.65, 0.78], \text{   and   } z_4 \in [1.9, 4.5] \)
\item \( z_1 \in [-5.2, -4.9], \text{   }  z_2 \in [-0.62, -0.5], z_3 \in [1.46, 1.56], \text{   and   } z_4 \in [1.9, 4.5] \)
\item \( z_1 \in [-4.5, -1.9], \text{   }  z_2 \in [-0.8, -0.62], z_3 \in [1.62, 1.74], \text{   and   } z_4 \in [3.6, 5.7] \)
\item \( z_1 \in [-4.5, -1.9], \text{   }  z_2 \in [-0.25, -0.17], z_3 \in [4.88, 5.04], \text{   and   } z_4 \in [3.6, 5.7] \)
\item \( z_1 \in [-4.5, -1.9], \text{   }  z_2 \in [-1.51, -1.43], z_3 \in [0.59, 0.62], \text{   and   } z_4 \in [3.6, 5.7] \)

\end{enumerate} }
\litem{
Perform the division below. Then, find the intervals that correspond to the quotient in the form $ax^2+bx+c$ and remainder $r$.\[ \frac{20x^{3} +62 x^{2} -16}{x + 3} \]\begin{enumerate}[label=\Alph*.]
\item \( a \in [-60, -57], b \in [242, 243], c \in [-730, -721], \text{ and } r \in [2161, 2168]. \)
\item \( a \in [-60, -57], b \in [-119, -110], c \in [-354, -349], \text{ and } r \in [-1080, -1074]. \)
\item \( a \in [18, 23], b \in [-2, 7], c \in [-13, -1], \text{ and } r \in [-2, 3]. \)
\item \( a \in [18, 23], b \in [-18, -15], c \in [70, 76], \text{ and } r \in [-311, -303]. \)
\item \( a \in [18, 23], b \in [120, 128], c \in [364, 373], \text{ and } r \in [1076, 1088]. \)

\end{enumerate} }
\litem{
Perform the division below. Then, find the intervals that correspond to the quotient in the form $ax^2+bx+c$ and remainder $r$.\[ \frac{10x^{3} -30 x^{2} + 43}{x -2} \]\begin{enumerate}[label=\Alph*.]
\item \( a \in [16, 23], b \in [9, 11], c \in [12, 29], \text{ and } r \in [81, 88]. \)
\item \( a \in [16, 23], b \in [-70, -67], c \in [139, 141], \text{ and } r \in [-239, -231]. \)
\item \( a \in [6, 13], b \in [-20, -15], c \in [-21, -18], \text{ and } r \in [21, 24]. \)
\item \( a \in [6, 13], b \in [-19, -7], c \in [-21, -18], \text{ and } r \in [-1, 4]. \)
\item \( a \in [6, 13], b \in [-50, -48], c \in [95, 104], \text{ and } r \in [-158, -154]. \)

\end{enumerate} }
\litem{
Factor the polynomial below completely, knowing that $x + 4$ is a factor. Then, choose the intervals the zeros of the polynomial belong to, where $z_1 \leq z_2 \leq z_3 \leq z_4$. \textit{To make the problem easier, all zeros are between -5 and 5.}\[ f(x) = 8x^{4} -10 x^{3} -101 x^{2} +238 x -120 \]\begin{enumerate}[label=\Alph*.]
\item \( z_1 \in [-2.15, -1.66], \text{   }  z_2 \in [-1.58, -1.33], z_3 \in [-0.52, -0.16], \text{   and   } z_4 \in [3.33, 4.3] \)
\item \( z_1 \in [-4.34, -3.6], \text{   }  z_2 \in [0.46, 0.95], z_3 \in [1.97, 2.28], \text{   and   } z_4 \in [2.01, 2.65] \)
\item \( z_1 \in [-2.68, -2.27], \text{   }  z_2 \in [-2.16, -1.71], z_3 \in [-0.92, -0.7], \text{   and   } z_4 \in [3.33, 4.3] \)
\item \( z_1 \in [-4.34, -3.6], \text{   }  z_2 \in [0.37, 0.42], z_3 \in [1.07, 1.43], \text{   and   } z_4 \in [0.8, 2.23] \)
\item \( z_1 \in [-3.35, -2.63], \text{   }  z_2 \in [-2.16, -1.71], z_3 \in [-0.64, -0.62], \text{   and   } z_4 \in [3.33, 4.3] \)

\end{enumerate} }
\litem{
Factor the polynomial below completely. Then, choose the intervals the zeros of the polynomial belong to, where $z_1 \leq z_2 \leq z_3$. \textit{To make the problem easier, all zeros are between -5 and 5.}\[ f(x) = 20x^{3} -43 x^{2} -3 x + 18 \]\begin{enumerate}[label=\Alph*.]
\item \( z_1 \in [-1.74, -1.55], \text{   }  z_2 \in [1.33, 1.4], \text{   and   } z_3 \in [1.74, 2.19] \)
\item \( z_1 \in [-2.07, -1.88], \text{   }  z_2 \in [-0.9, -0.68], \text{   and   } z_3 \in [0.53, 0.61] \)
\item \( z_1 \in [-2.07, -1.88], \text{   }  z_2 \in [-0.47, 0.06], \text{   and   } z_3 \in [2.75, 3.01] \)
\item \( z_1 \in [-0.8, -0.48], \text{   }  z_2 \in [0.43, 0.86], \text{   and   } z_3 \in [1.74, 2.19] \)
\item \( z_1 \in [-2.07, -1.88], \text{   }  z_2 \in [-1.41, -1.21], \text{   and   } z_3 \in [1.07, 1.95] \)

\end{enumerate} }
\litem{
Factor the polynomial below completely. Then, choose the intervals the zeros of the polynomial belong to, where $z_1 \leq z_2 \leq z_3$. \textit{To make the problem easier, all zeros are between -5 and 5.}\[ f(x) = 25x^{3} +50 x^{2} -9 x -18 \]\begin{enumerate}[label=\Alph*.]
\item \( z_1 \in [-3.1, -2.6], \text{   }  z_2 \in [0.06, 0.51], \text{   and   } z_3 \in [1.92, 2.63] \)
\item \( z_1 \in [-1.9, -0.7], \text{   }  z_2 \in [1.54, 1.86], \text{   and   } z_3 \in [1.92, 2.63] \)
\item \( z_1 \in [-1.5, 0.1], \text{   }  z_2 \in [0.29, 1.22], \text{   and   } z_3 \in [1.92, 2.63] \)
\item \( z_1 \in [-2.1, -1.8], \text{   }  z_2 \in [-1.91, -1.44], \text{   and   } z_3 \in [1.3, 1.89] \)
\item \( z_1 \in [-2.1, -1.8], \text{   }  z_2 \in [-1.34, -0.43], \text{   and   } z_3 \in [0.54, 0.97] \)

\end{enumerate} }
\litem{
Perform the division below. Then, find the intervals that correspond to the quotient in the form $ax^2+bx+c$ and remainder $r$.\[ \frac{10x^{3} -29 x^{2} -50 x + 26}{x -4} \]\begin{enumerate}[label=\Alph*.]
\item \( a \in [38, 42], \text{   } b \in [130, 134], \text{   } c \in [471, 482], \text{   and   } r \in [1920, 1925]. \)
\item \( a \in [5, 15], \text{   } b \in [-73, -61], \text{   } c \in [220, 231], \text{   and   } r \in [-883, -870]. \)
\item \( a \in [38, 42], \text{   } b \in [-190, -187], \text{   } c \in [704, 707], \text{   and   } r \in [-2800, -2793]. \)
\item \( a \in [5, 15], \text{   } b \in [0, 4], \text{   } c \in [-47, -45], \text{   and   } r \in [-117, -112]. \)
\item \( a \in [5, 15], \text{   } b \in [9, 14], \text{   } c \in [-13, -3], \text{   and   } r \in [0, 7]. \)

\end{enumerate} }
\litem{
Perform the division below. Then, find the intervals that correspond to the quotient in the form $ax^2+bx+c$ and remainder $r$.\[ \frac{10x^{3} -85 x^{2} +200 x -129}{x -5} \]\begin{enumerate}[label=\Alph*.]
\item \( a \in [49, 53], \text{   } b \in [-335, -331], \text{   } c \in [1872, 1879], \text{   and   } r \in [-9510, -9502]. \)
\item \( a \in [3, 11], \text{   } b \in [-136, -134], \text{   } c \in [875, 882], \text{   and   } r \in [-4504, -4494]. \)
\item \( a \in [3, 11], \text{   } b \in [-41, -33], \text{   } c \in [25, 28], \text{   and   } r \in [-4, 1]. \)
\item \( a \in [49, 53], \text{   } b \in [164, 171], \text{   } c \in [1020, 1033], \text{   and   } r \in [4988, 5000]. \)
\item \( a \in [3, 11], \text{   } b \in [-45, -44], \text{   } c \in [18, 23], \text{   and   } r \in [-51, -48]. \)

\end{enumerate} }
\end{enumerate}

\end{document}