\documentclass{extbook}[14pt]
\usepackage{multicol, enumerate, enumitem, hyperref, color, soul, setspace, parskip, fancyhdr, amssymb, amsthm, amsmath, bbm, latexsym, units, mathtools}
\everymath{\displaystyle}
\usepackage[headsep=0.5cm,headheight=0cm, left=1 in,right= 1 in,top= 1 in,bottom= 1 in]{geometry}
\pagestyle{fancy}
\lhead{}
\chead{Answer Key for Module\,10L\,-\,Synthetic\,Division Version A}
\rhead{}
\lfoot{Summer\,C\,2020}
\cfoot{}
\rfoot{}
\begin{document}
\textbf{This key should allow you to understand why you choose the option you did (beyond just getting a question right or wrong). \href{https://xronos.clas.ufl.edu/mac1105spring2020/courseDescriptionAndMisc/Exams/LearningFromResults}{More instructions on how to use this key can be found here}.}

\textbf{If you have a suggestion to make the keys better, \href{https://forms.gle/CZkbZmPbC9XALEE88}{please fill out the short survey here}.}

\textit{Note: This key is auto-generated and may contain issues and/or errors. The keys are reviewed after each exam to ensure grading is done accurately. If there are issues (like duplicate options), they are noted in the offline gradebook. The keys are a work-in-progress to give students as many resources to improve as possible.}

\rule{\textwidth}{0.4pt}

66. Perform the division below. Then, find the intervals that correspond to the quotient in the form $ax^2+bx+c$ and remainder $r$.
\[ \frac{12x^{3} -16 x^{2} -108 x -75}{x -4} \] 
The solution is $ 12x^{2} +32 x + 20 + \frac{5}{x -4} $ 

\begin{enumerate}[label=\Alph*.] 
\item $ a \in [47, 51], \text{   } b \in [-209, -207], \text{   } c \in [717, 728], \text{   and   } r \in [-2976, -2965]. $ 

  You divided by the opposite of the factor AND multiplied the first factor rather than just bringing it down. 
\item $ a \in [8, 17], \text{   } b \in [17, 21], \text{   } c \in [-52, -46], \text{   and   } r \in [-220, -216]. $ 

  You multiplied by the synthetic number and subtracted rather than adding during synthetic division. 
\item $ a \in [8, 17], \text{   } b \in [31, 40], \text{   } c \in [16, 23], \text{   and   } r \in [2, 10]. $ 

 * This is the solution! 
\item $ a \in [8, 17], \text{   } b \in [-67, -62], \text{   } c \in [145, 149], \text{   and   } r \in [-670, -662]. $ 

  You divided by the opposite of the factor. 
\item $ a \in [47, 51], \text{   } b \in [173, 177], \text{   } c \in [587, 598], \text{   and   } r \in [2305, 2313]. $ 

  You multiplied by the synthetic number rather than bringing the first factor down. 
\end{enumerate} 
 
General Comments: Be sure to synthetically divide by the zero of the denominator!

-----------------------------------------------

67. Factor the polynomial below completely. Then, choose the intervals the zeros of the polynomial belong to, where $z_1 \leq z_2 \leq z_3$. \textit{To make the problem easier, all zeros are between -5 and 5.}
\[ f(x) = 6x^{3} +13 x^{2} -40 x -75 \] 
The solution is $ [-3, -1.6666666666666667, 2.5] $ 

\begin{enumerate}[label=\Alph*.] 
\item $ z_1 \in [-1.05, -0.83], \text{   }  z_2 \in [2.61, 3.51], \text{   and   } z_3 \in [4.84, 5.5] $ 

  Distractor 4: Corresponds to moving factors from one rational to another. 
\item $ z_1 \in [-2.55, -2.4], \text{   }  z_2 \in [1, 2.02], \text{   and   } z_3 \in [2.54, 3.31] $ 

  Distractor 1: Corresponds to negatives of all zeros. 
\item $ z_1 \in [-0.59, -0.3], \text{   }  z_2 \in [0.53, 0.84], \text{   and   } z_3 \in [2.54, 3.31] $ 

  Distractor 3: Corresponds to negatives of all zeros AND inversing rational roots. 
\item $ z_1 \in [-3.02, -2.94], \text{   }  z_2 \in [-0.92, -0.12], \text{   and   } z_3 \in [0.06, 0.76] $ 

  Distractor 2: Corresponds to inversing rational roots. 
\item $ z_1 \in [-3.02, -2.94], \text{   }  z_2 \in [-2.29, -1.21], \text{   and   } z_3 \in [2.31, 2.87] $ 

 * This is the solution! 
\end{enumerate} 
 
General Comments: Remember to try the middle-most integers first as these normally are the zeros. Also, once you get it to a quadratic, you can use your other factoring techniques to finish factoring.

-----------------------------------------------

68. Factor the polynomial below completely, knowing that $x-2$ is a factor. Then, choose the intervals the zeros of the polynomial belong to, where $z_1 \leq z_2 \leq z_3 \leq z_4$. \textit{To make the problem easier, all zeros are between -5 and 5.}
\[ f(x) = 9x^{4} -63 x^{3} +74 x^{2} +112 x -160 \] 
The solution is $ [-1.3333333333333333, 1.3333333333333333, 2, 5] $ 

\begin{enumerate}[label=\Alph*.] 
\item $ z_1 \in [-6.6, -2.9], \text{   }  z_2 \in [-3.01, -1.54], z_3 \in [-0.7, -0.27], \text{   and   } z_4 \in [3.8, 4.1] $ 

  Distractor 4: Corresponds to moving factors from one rational to another. 
\item $ z_1 \in [-6.6, -2.9], \text{   }  z_2 \in [-3.01, -1.54], z_3 \in [-1.52, -0.93], \text{   and   } z_4 \in [0.9, 1.7] $ 

  Distractor 1: Corresponds to negatives of all zeros. 
\item $ z_1 \in [-2, -1.1], \text{   }  z_2 \in [1.09, 1.58], z_3 \in [1.9, 2.44], \text{   and   } z_4 \in [4.1, 5.1] $ 

 * This is the solution! 
\item $ z_1 \in [-6.6, -2.9], \text{   }  z_2 \in [-3.01, -1.54], z_3 \in [-1.03, -0.53], \text{   and   } z_4 \in [0.3, 1.3] $ 

  Distractor 3: Corresponds to negatives of all zeros AND inversing rational roots. 
\item $ z_1 \in [-0.8, 0.5], \text{   }  z_2 \in [0.42, 0.86], z_3 \in [1.9, 2.44], \text{   and   } z_4 \in [4.1, 5.1] $ 

  Distractor 2: Corresponds to inversing rational roots. 
\end{enumerate} 
 
General Comments: Remember to try the middle-most integers first as these normally are the zeros. Also, once you get it to a quadratic, you can use your other factoring techniques to finish factoring.

-----------------------------------------------

69. Perform the division below. Then, find the intervals that correspond to the quotient in the form $ax^2+bx+c$ and remainder $r$.
\[ \frac{20x^{3} -60 x + 42}{x + 2} \] 
The solution is $ 20x^{2} -40 x + 20 + \frac{2}{x + 2} $ 

\begin{enumerate}[label=\Alph*.] 
\item $ a \in [18, 24], b \in [31, 44], c \in [16, 24], \text{ and } r \in [80, 87]. $ 

  You divided by the opposite of the factor. 
\item $ a \in [18, 24], b \in [-65, -55], c \in [118, 121], \text{ and } r \in [-324, -317]. $ 

  You multipled by the synthetic number and subtracted rather than adding during synthetic division. 
\item $ a \in [-41, -37], b \in [79, 84], c \in [-222, -211], \text{ and } r \in [477, 488]. $ 

  You multipled by the synthetic number rather than bringing the first factor down. 
\item $ a \in [18, 24], b \in [-47, -37], c \in [16, 24], \text{ and } r \in [-2, 3]. $ 

 * This is the solution! 
\item $ a \in [-41, -37], b \in [-83, -76], c \in [-222, -211], \text{ and } r \in [-399, -397]. $ 

  You divided by the opposite of the factor AND multipled the first factor rather than just bringing it down. 
\end{enumerate} 
 
General Comments: Be sure to synthetically divide by the zero of the denominator! Also, make sure to include 0 placeholders for missing terms.

-----------------------------------------------

70. What are the \textit{possible Integer} roots of the polynomial below?
\[ f(x) = 6x^{2} +3 x + 3 \] 
The solution is $ \pm 1,\pm 3 $ 

\begin{enumerate}[label=\Alph*.] 
\item $ \pm 1,\pm 3 $ 

 * This is the solution \textbf{since we asked for the possible Integer roots}! 
\item $ \text{ All combinations of: }\frac{\pm 1,\pm 3}{\pm 1,\pm 2,\pm 3,\pm 6} $ 

 This would have been the solution \textbf{if asked for the possible Rational roots}! 
\item $ \pm 1,\pm 2,\pm 3,\pm 6 $ 

  Distractor 1: Corresponds to the plus or minus factors of a1 only. 
\item $ \text{ All combinations of: }\frac{\pm 1,\pm 2,\pm 3,\pm 6}{\pm 1,\pm 3} $ 

  Distractor 3: Corresponds to the plus or minus of the inverse quotient (an/a0) of the factors.  
\item $ \text{There is no formula or theorem that tells us all possible Integer roots.} $ 

  Distractor 4: Corresponds to not recognizing Integers as a subset of Rationals. 
\end{enumerate} 
 
General Comments: We have a way to find the possible Rational roots. The possible Integer roots are the Integers in this list.

-----------------------------------------------


\end{document}

