\documentclass{extbook}[14pt]
\usepackage{multicol, enumerate, enumitem, hyperref, color, soul, setspace, parskip, fancyhdr, amssymb, amsthm, amsmath, bbm, latexsym, units, mathtools}
\everymath{\displaystyle}
\usepackage[headsep=0.5cm,headheight=0cm, left=1 in,right= 1 in,top= 1 in,bottom= 1 in]{geometry}
\pagestyle{fancy}
\lhead{}
\chead{Answer Key for Module\,10L\,-\,Synthetic\,Division Version A}
\rhead{}
\lfoot{Summer\,C\,2020}
\cfoot{}
\rfoot{}
\begin{document}
\textbf{This key should allow you to understand why you choose the option you did (beyond just getting a question right or wrong). \href{https://xronos.clas.ufl.edu/mac1105spring2020/courseDescriptionAndMisc/Exams/LearningFromResults}{More instructions on how to use this key can be found here}.}

\textbf{If you have a suggestion to make the keys better, \href{https://forms.gle/CZkbZmPbC9XALEE88}{please fill out the short survey here}.}

\textit{Note: This key is auto-generated and may contain issues and/or errors. The keys are reviewed after each exam to ensure grading is done accurately. If there are issues (like duplicate options), they are noted in the offline gradebook. The keys are a work-in-progress to give students as many resources to improve as possible.}

\rule{\textwidth}{0.4pt}

66. Factor the polynomial below completely, knowing that $x+4$ is a factor. Then, choose the intervals the zeros of the polynomial belong to, where $z_1 \leq z_2 \leq z_3 \leq z_4$. \textit{To make the problem easier, all zeros are between -5 and 5.}
\[ f(x) = 10x^{4} -7 x^{3} -172 x^{2} +112 x + 192 \] 
The solution is $ [-4, -0.8, 1.5, 4] $ 

\begin{enumerate}[label=\Alph*.] 
\item $ z_1 \in [-5, -3], \text{   }  z_2 \in [-1.44, -1.13], z_3 \in [0.36, 0.71], \text{   and   } z_4 \in [2, 6] $ 

  Distractor 2: Corresponds to inversing rational roots. 
\item $ z_1 \in [-5, -3], \text{   }  z_2 \in [-0.37, -0.25], z_3 \in [3.94, 4.03], \text{   and   } z_4 \in [2, 6] $ 

  Distractor 4: Corresponds to moving factors from one rational to another. 
\item $ z_1 \in [-5, -3], \text{   }  z_2 \in [-1.65, -1.31], z_3 \in [0.67, 0.93], \text{   and   } z_4 \in [2, 6] $ 

  Distractor 1: Corresponds to negatives of all zeros. 
\item $ z_1 \in [-5, -3], \text{   }  z_2 \in [-0.99, -0.67], z_3 \in [1.33, 1.68], \text{   and   } z_4 \in [2, 6] $ 

 * This is the solution! 
\item $ z_1 \in [-5, -3], \text{   }  z_2 \in [-0.75, -0.48], z_3 \in [1.19, 1.42], \text{   and   } z_4 \in [2, 6] $ 

  Distractor 3: Corresponds to negatives of all zeros AND inversing rational roots. 
\end{enumerate} 
 
General Comments: Remember to try the middle-most integers first as these normally are the zeros. Also, once you get it to a quadratic, you can use your other factoring techniques to finish factoring.

-----------------------------------------------

67. Factor the polynomial below completely. Then, choose the intervals the zeros of the polynomial belong to, where $z_1 \leq z_2 \leq z_3$. \textit{To make the problem easier, all zeros are between -5 and 5.}
\[ f(x) = 4x^{3} -24 x^{2} +5 x + 75 \] 
The solution is $ [-1.5, 2.5, 5] $ 

\begin{enumerate}[label=\Alph*.] 
\item $ z_1 \in [-5.6, -3.7], \text{   }  z_2 \in [-2.9, -2.3], \text{   and   } z_3 \in [0.7, 2] $ 

  Distractor 1: Corresponds to negatives of all zeros. 
\item $ z_1 \in [-5.6, -3.7], \text{   }  z_2 \in [-0.8, -0.1], \text{   and   } z_3 \in [0.6, 0.9] $ 

  Distractor 3: Corresponds to negatives of all zeros AND inversing rational roots. 
\item $ z_1 \in [-1.2, 0.2], \text{   }  z_2 \in [-0.1, 1.3], \text{   and   } z_3 \in [4.3, 6.5] $ 

  Distractor 2: Corresponds to inversing rational roots. 
\item $ z_1 \in [-3.1, -0.7], \text{   }  z_2 \in [1.8, 4.1], \text{   and   } z_3 \in [4.3, 6.5] $ 

 * This is the solution! 
\item $ z_1 \in [-5.6, -3.7], \text{   }  z_2 \in [-1.8, -0.5], \text{   and   } z_3 \in [2, 4.1] $ 

  Distractor 4: Corresponds to moving factors from one rational to another. 
\end{enumerate} 
 
General Comments: Remember to try the middle-most integers first as these normally are the zeros. Also, once you get it to a quadratic, you can use your other factoring techniques to finish factoring.

-----------------------------------------------

68. Perform the division below. Then, find the intervals that correspond to the quotient in the form $ax^2+bx+c$ and remainder $r$.
\[ \frac{12x^{3} -56 x^{2} +12 x + 84}{x -4} \] 
The solution is $ 12x^{2} -8 x -20 + \frac{4}{x -4} $ 

\begin{enumerate}[label=\Alph*.] 
\item $ a \in [11, 13], \text{   } b \in [-13, -4], \text{   } c \in [-27, -16], \text{   and   } r \in [-4, 8]. $ 

 * This is the solution! 
\item $ a \in [41, 54], \text{   } b \in [132, 138], \text{   } c \in [553, 566], \text{   and   } r \in [2307, 2309]. $ 

  You multiplied by the synthetic number rather than bringing the first factor down. 
\item $ a \in [11, 13], \text{   } b \in [-24, -16], \text{   } c \in [-53, -46], \text{   and   } r \in [-62, -57]. $ 

  You multiplied by the synthetic number and subtracted rather than adding during synthetic division. 
\item $ a \in [41, 54], \text{   } b \in [-254, -243], \text{   } c \in [999, 1006], \text{   and   } r \in [-3936, -3929]. $ 

  You divided by the opposite of the factor AND multiplied the first factor rather than just bringing it down. 
\item $ a \in [11, 13], \text{   } b \in [-112, -92], \text{   } c \in [426, 437], \text{   and   } r \in [-1631, -1620]. $ 

  You divided by the opposite of the factor. 
\end{enumerate} 
 
General Comments: Be sure to synthetically divide by the zero of the denominator!

-----------------------------------------------

69. Perform the division below. Then, find the intervals that correspond to the quotient in the form $ax^2+bx+c$ and remainder $r$.
\[ \frac{15x^{3} +65 x^{2} -84}{x + 4} \] 
The solution is $ 15x^{2} +5 x -20 + \frac{-4}{x + 4} $ 

\begin{enumerate}[label=\Alph*.] 
\item $ a \in [13, 20], b \in [122, 127], c \in [492, 504], \text{ and } r \in [1910, 1918]. $ 

  You divided by the opposite of the factor. 
\item $ a \in [13, 20], b \in [4, 10], c \in [-21, -18], \text{ and } r \in [-5, 3]. $ 

 * This is the solution! 
\item $ a \in [-63, -59], b \in [301, 309], c \in [-1222, -1219], \text{ and } r \in [4792, 4797]. $ 

  You multipled by the synthetic number rather than bringing the first factor down. 
\item $ a \in [-63, -59], b \in [-181, -169], c \in [-703, -698], \text{ and } r \in [-2886, -2879]. $ 

  You divided by the opposite of the factor AND multipled the first factor rather than just bringing it down. 
\item $ a \in [13, 20], b \in [-14, -5], c \in [48, 54], \text{ and } r \in [-338, -329]. $ 

  You multipled by the synthetic number and subtracted rather than adding during synthetic division. 
\end{enumerate} 
 
General Comments: Be sure to synthetically divide by the zero of the denominator! Also, make sure to include 0 placeholders for missing terms.

-----------------------------------------------

70. What are the \textit{possible Rational} roots of the polynomial below?
\[ f(x) = 4x^{3} +3 x^{2} +2 x + 2 \] 
The solution is $ \text{ All combinations of: }\frac{\pm 1,\pm 2}{\pm 1,\pm 2,\pm 4} $ 

\begin{enumerate}[label=\Alph*.] 
\item $ \pm 1,\pm 2 $ 

 This would have been the solution \textbf{if asked for the possible Integer roots}! 
\item $ \pm 1,\pm 2,\pm 4 $ 

  Distractor 1: Corresponds to the plus or minus factors of a1 only. 
\item $ \text{ All combinations of: }\frac{\pm 1,\pm 2,\pm 4}{\pm 1,\pm 2} $ 

  Distractor 3: Corresponds to the plus or minus of the inverse quotient (an/a0) of the factors.  
\item $ \text{ All combinations of: }\frac{\pm 1,\pm 2}{\pm 1,\pm 2,\pm 4} $ 

 * This is the solution \textbf{since we asked for the possible Rational roots}! 
\item $ \text{ There is no formula or theorem that tells us all possible Rational roots.} $ 

  Distractor 4: Corresponds to not recalling the theorem for rational roots of a polynomial. 
\end{enumerate} 
 
General Comments: We have a way to find the possible Rational roots. The possible Integer roots are the Integers in this list.

-----------------------------------------------


\end{document}

