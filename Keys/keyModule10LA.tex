\documentclass{extbook}[14pt]
\usepackage{multicol, enumerate, enumitem, hyperref, color, soul, setspace, parskip, fancyhdr, amssymb, amsthm, amsmath, latexsym, units, mathtools}
\everymath{\displaystyle}
\usepackage[headsep=0.5cm,headheight=0cm, left=1 in,right= 1 in,top= 1 in,bottom= 1 in]{geometry}
\usepackage{dashrule}  % Package to use the command below to create lines between items
\newcommand{\litem}[1]{\item #1

\rule{\textwidth}{0.4pt}}
\pagestyle{fancy}
\lhead{}
\chead{Answer Key for Makeup Progress Quiz 3 Version A}
\rhead{}
\lfoot{1648-1753}
\cfoot{}
\rfoot{Summer C 2021}
\begin{document}
\textbf{This key should allow you to understand why you choose the option you did (beyond just getting a question right or wrong). \href{https://xronos.clas.ufl.edu/mac1105spring2020/courseDescriptionAndMisc/Exams/LearningFromResults}{More instructions on how to use this key can be found here}.}

\textbf{If you have a suggestion to make the keys better, \href{https://forms.gle/CZkbZmPbC9XALEE88}{please fill out the short survey here}.}

\textit{Note: This key is auto-generated and may contain issues and/or errors. The keys are reviewed after each exam to ensure grading is done accurately. If there are issues (like duplicate options), they are noted in the offline gradebook. The keys are a work-in-progress to give students as many resources to improve as possible.}

\rule{\textwidth}{0.4pt}

\begin{enumerate}\litem{
Factor the polynomial below completely. Then, choose the intervals the zeros of the polynomial belong to, where $z_1 \leq z_2 \leq z_3$. \textit{To make the problem easier, all zeros are between -5 and 5.}
\[ f(x) = 8x^{3} +14 x^{2} -63 x + 36 \]The solution is \( [-4, 0.75, 1.5] \), which is option B.\begin{enumerate}[label=\Alph*.]
\item \( z_1 \in [-3.28, -2.73], \text{   }  z_2 \in [-0.41, -0.35], \text{   and   } z_3 \in [3.73, 4.24] \)

 Distractor 4: Corresponds to moving factors from one rational to another.
\item \( z_1 \in [-4.4, -3.98], \text{   }  z_2 \in [0.74, 0.77], \text{   and   } z_3 \in [1.49, 1.67] \)

* This is the solution!
\item \( z_1 \in [-1.44, -1.26], \text{   }  z_2 \in [-0.71, -0.63], \text{   and   } z_3 \in [3.73, 4.24] \)

 Distractor 3: Corresponds to negatives of all zeros AND inversing rational roots.
\item \( z_1 \in [-1.57, -1.35], \text{   }  z_2 \in [-0.77, -0.71], \text{   and   } z_3 \in [3.73, 4.24] \)

 Distractor 1: Corresponds to negatives of all zeros.
\item \( z_1 \in [-4.4, -3.98], \text{   }  z_2 \in [0.62, 0.69], \text{   and   } z_3 \in [0.96, 1.36] \)

 Distractor 2: Corresponds to inversing rational roots.
\end{enumerate}

\textbf{General Comment:} Remember to try the middle-most integers first as these normally are the zeros. Also, once you get it to a quadratic, you can use your other factoring techniques to finish factoring.
}
\litem{
Perform the division below. Then, find the intervals that correspond to the quotient in the form $ax^2+bx+c$ and remainder $r$.
\[ \frac{8x^{3} -56 x -51}{x -3} \]The solution is \( 8x^{2} +24 x + 16 + \frac{-3}{x -3} \), which is option A.\begin{enumerate}[label=\Alph*.]
\item \( a \in [8, 12], b \in [19, 29], c \in [13, 17], \text{ and } r \in [-4, 2]. \)

* This is the solution!
\item \( a \in [23, 29], b \in [-72, -69], c \in [160, 162], \text{ and } r \in [-533, -529]. \)

 You divided by the opposite of the factor AND multipled the first factor rather than just bringing it down.
\item \( a \in [23, 29], b \in [69, 79], c \in [160, 162], \text{ and } r \in [426, 431]. \)

 You multipled by the synthetic number rather than bringing the first factor down.
\item \( a \in [8, 12], b \in [11, 19], c \in [-24, -23], \text{ and } r \in [-99, -95]. \)

 You multipled by the synthetic number and subtracted rather than adding during synthetic division.
\item \( a \in [8, 12], b \in [-24, -19], c \in [13, 17], \text{ and } r \in [-99, -95]. \)

 You divided by the opposite of the factor.
\end{enumerate}

\textbf{General Comment:} Be sure to synthetically divide by the zero of the denominator! Also, make sure to include 0 placeholders for missing terms.
}
\litem{
Perform the division below. Then, find the intervals that correspond to the quotient in the form $ax^2+bx+c$ and remainder $r$.
\[ \frac{15x^{3} +52 x^{2} -48 x -66}{x + 4} \]The solution is \( 15x^{2} -8 x -16 + \frac{-2}{x + 4} \), which is option C.\begin{enumerate}[label=\Alph*.]
\item \( a \in [7, 17], \text{   } b \in [-25, -21], \text{   } c \in [63, 72], \text{   and   } r \in [-404, -400]. \)

 You multiplied by the synthetic number and subtracted rather than adding during synthetic division.
\item \( a \in [-68, -58], \text{   } b \in [290, 297], \text{   } c \in [-1220, -1214], \text{   and   } r \in [4796, 4800]. \)

 You multiplied by the synthetic number rather than bringing the first factor down.
\item \( a \in [7, 17], \text{   } b \in [-13, -4], \text{   } c \in [-17, -14], \text{   and   } r \in [-4, 0]. \)

* This is the solution!
\item \( a \in [7, 17], \text{   } b \in [107, 113], \text{   } c \in [400, 403], \text{   and   } r \in [1534, 1539]. \)

 You divided by the opposite of the factor.
\item \( a \in [-68, -58], \text{   } b \in [-192, -185], \text{   } c \in [-800, -796], \text{   and   } r \in [-3270, -3265]. \)

 You divided by the opposite of the factor AND multiplied the first factor rather than just bringing it down.
\end{enumerate}

\textbf{General Comment:} Be sure to synthetically divide by the zero of the denominator!
}
\litem{
Factor the polynomial below completely, knowing that $x + 4$ is a factor. Then, choose the intervals the zeros of the polynomial belong to, where $z_1 \leq z_2 \leq z_3 \leq z_4$. \textit{To make the problem easier, all zeros are between -5 and 5.}
\[ f(x) = 12x^{4} +59 x^{3} -1 x^{2} -230 x -200 \]The solution is \( [-4, -1.667, -1.25, 2] \), which is option B.\begin{enumerate}[label=\Alph*.]
\item \( z_1 \in [-4.1, -3.6], \text{   }  z_2 \in [-0.85, -0.64], z_3 \in [-0.88, -0.5], \text{   and   } z_4 \in [0.6, 3] \)

 Distractor 2: Corresponds to inversing rational roots.
\item \( z_1 \in [-4.1, -3.6], \text{   }  z_2 \in [-1.72, -1.62], z_3 \in [-1.32, -1.21], \text{   and   } z_4 \in [0.6, 3] \)

* This is the solution!
\item \( z_1 \in [-3, -0.6], \text{   }  z_2 \in [0.49, 0.85], z_3 \in [0.67, 1.33], \text{   and   } z_4 \in [2.3, 4.6] \)

 Distractor 3: Corresponds to negatives of all zeros AND inversing rational roots.
\item \( z_1 \in [-3, -0.6], \text{   }  z_2 \in [1.13, 1.51], z_3 \in [1.35, 1.96], \text{   and   } z_4 \in [2.3, 4.6] \)

 Distractor 1: Corresponds to negatives of all zeros.
\item \( z_1 \in [-3, -0.6], \text{   }  z_2 \in [0.4, 0.55], z_3 \in [3.9, 4.15], \text{   and   } z_4 \in [4.3, 5.8] \)

 Distractor 4: Corresponds to moving factors from one rational to another.
\end{enumerate}

\textbf{General Comment:} Remember to try the middle-most integers first as these normally are the zeros. Also, once you get it to a quadratic, you can use your other factoring techniques to finish factoring.
}
\litem{
Factor the polynomial below completely, knowing that $x -2$ is a factor. Then, choose the intervals the zeros of the polynomial belong to, where $z_1 \leq z_2 \leq z_3 \leq z_4$. \textit{To make the problem easier, all zeros are between -5 and 5.}
\[ f(x) = 4x^{4} -24 x^{3} +29 x^{2} +51 x -90 \]The solution is \( [-1.5, 2, 2.5, 3] \), which is option D.\begin{enumerate}[label=\Alph*.]
\item \( z_1 \in [-5.77, -4.75], \text{   }  z_2 \in [-3.17, -2.57], z_3 \in [-2.16, -1.25], \text{   and   } z_4 \in [0.7, 0.83] \)

 Distractor 4: Corresponds to moving factors from one rational to another.
\item \( z_1 \in [-1.04, 0.01], \text{   }  z_2 \in [-0.19, 1.44], z_3 \in [1.27, 2.41], \text{   and   } z_4 \in [2.99, 3.06] \)

 Distractor 2: Corresponds to inversing rational roots.
\item \( z_1 \in [-3.2, -2.58], \text{   }  z_2 \in [-2.17, -1.53], z_3 \in [-0.87, 0.52], \text{   and   } z_4 \in [0.65, 0.68] \)

 Distractor 3: Corresponds to negatives of all zeros AND inversing rational roots.
\item \( z_1 \in [-1.67, -0.78], \text{   }  z_2 \in [1.55, 2.52], z_3 \in [2.15, 2.77], \text{   and   } z_4 \in [2.99, 3.06] \)

* This is the solution!
\item \( z_1 \in [-3.2, -2.58], \text{   }  z_2 \in [-2.62, -2.43], z_3 \in [-2.16, -1.25], \text{   and   } z_4 \in [1.46, 1.55] \)

 Distractor 1: Corresponds to negatives of all zeros.
\end{enumerate}

\textbf{General Comment:} Remember to try the middle-most integers first as these normally are the zeros. Also, once you get it to a quadratic, you can use your other factoring techniques to finish factoring.
}
\litem{
Perform the division below. Then, find the intervals that correspond to the quotient in the form $ax^2+bx+c$ and remainder $r$.
\[ \frac{8x^{3} +32 x^{2} -8 x -27}{x + 4} \]The solution is \( 8x^{2} -8 + \frac{5}{x + 4} \), which is option C.\begin{enumerate}[label=\Alph*.]
\item \( a \in [6, 21], \text{   } b \in [64, 68], \text{   } c \in [245, 251], \text{   and   } r \in [964, 969]. \)

 You divided by the opposite of the factor.
\item \( a \in [6, 21], \text{   } b \in [-9, -4], \text{   } c \in [28, 36], \text{   and   } r \in [-190, -185]. \)

 You multiplied by the synthetic number and subtracted rather than adding during synthetic division.
\item \( a \in [6, 21], \text{   } b \in [-3, 7], \text{   } c \in [-10, -1], \text{   and   } r \in [3, 8]. \)

* This is the solution!
\item \( a \in [-35, -28], \text{   } b \in [159, 163], \text{   } c \in [-648, -644], \text{   and   } r \in [2563, 2566]. \)

 You multiplied by the synthetic number rather than bringing the first factor down.
\item \( a \in [-35, -28], \text{   } b \in [-101, -90], \text{   } c \in [-394, -391], \text{   and   } r \in [-1596, -1594]. \)

 You divided by the opposite of the factor AND multiplied the first factor rather than just bringing it down.
\end{enumerate}

\textbf{General Comment:} Be sure to synthetically divide by the zero of the denominator!
}
\litem{
Factor the polynomial below completely. Then, choose the intervals the zeros of the polynomial belong to, where $z_1 \leq z_2 \leq z_3$. \textit{To make the problem easier, all zeros are between -5 and 5.}
\[ f(x) = 9x^{3} -54 x^{2} +35 x + 50 \]The solution is \( [-0.67, 1.67, 5] \), which is option A.\begin{enumerate}[label=\Alph*.]
\item \( z_1 \in [-1, 0.1], \text{   }  z_2 \in [1.58, 1.68], \text{   and   } z_3 \in [4.86, 5.29] \)

* This is the solution!
\item \( z_1 \in [-5.1, -4.7], \text{   }  z_2 \in [-1.67, -1.59], \text{   and   } z_3 \in [0.38, 1] \)

 Distractor 1: Corresponds to negatives of all zeros.
\item \( z_1 \in [-5.1, -4.7], \text{   }  z_2 \in [-0.61, -0.57], \text{   and   } z_3 \in [0.88, 1.74] \)

 Distractor 3: Corresponds to negatives of all zeros AND inversing rational roots.
\item \( z_1 \in [-5.1, -4.7], \text{   }  z_2 \in [-0.58, -0.54], \text{   and   } z_3 \in [1.65, 2.08] \)

 Distractor 4: Corresponds to moving factors from one rational to another.
\item \( z_1 \in [-2, -1.4], \text{   }  z_2 \in [0.57, 0.61], \text{   and   } z_3 \in [4.86, 5.29] \)

 Distractor 2: Corresponds to inversing rational roots.
\end{enumerate}

\textbf{General Comment:} Remember to try the middle-most integers first as these normally are the zeros. Also, once you get it to a quadratic, you can use your other factoring techniques to finish factoring.
}
\litem{
What are the \textit{possible Integer} roots of the polynomial below?
\[ f(x) = 2x^{3} +3 x^{2} +7 x + 6 \]The solution is \( \pm 1,\pm 2,\pm 3,\pm 6 \), which is option C.\begin{enumerate}[label=\Alph*.]
\item \( \text{ All combinations of: }\frac{\pm 1,\pm 2}{\pm 1,\pm 2,\pm 3,\pm 6} \)

 Distractor 3: Corresponds to the plus or minus of the inverse quotient (an/a0) of the factors. 
\item \( \pm 1,\pm 2 \)

 Distractor 1: Corresponds to the plus or minus factors of a1 only.
\item \( \pm 1,\pm 2,\pm 3,\pm 6 \)

* This is the solution \textbf{since we asked for the possible Integer roots}!
\item \( \text{ All combinations of: }\frac{\pm 1,\pm 2,\pm 3,\pm 6}{\pm 1,\pm 2} \)

This would have been the solution \textbf{if asked for the possible Rational roots}!
\item \( \text{There is no formula or theorem that tells us all possible Integer roots.} \)

 Distractor 4: Corresponds to not recognizing Integers as a subset of Rationals.
\end{enumerate}

\textbf{General Comment:} We have a way to find the possible Rational roots. The possible Integer roots are the Integers in this list.
}
\litem{
Perform the division below. Then, find the intervals that correspond to the quotient in the form $ax^2+bx+c$ and remainder $r$.
\[ \frac{15x^{3} -65 x^{2} + 84}{x -4} \]The solution is \( 15x^{2} -5 x -20 + \frac{4}{x -4} \), which is option C.\begin{enumerate}[label=\Alph*.]
\item \( a \in [13, 16], b \in [-20, -10], c \in [-61, -59], \text{ and } r \in [-100, -95]. \)

 You multipled by the synthetic number and subtracted rather than adding during synthetic division.
\item \( a \in [57, 64], b \in [173, 179], c \in [699, 703], \text{ and } r \in [2883, 2885]. \)

 You multipled by the synthetic number rather than bringing the first factor down.
\item \( a \in [13, 16], b \in [-7, 0], c \in [-21, -17], \text{ and } r \in [3, 5]. \)

* This is the solution!
\item \( a \in [13, 16], b \in [-127, -116], c \in [497, 510], \text{ and } r \in [-1921, -1914]. \)

 You divided by the opposite of the factor.
\item \( a \in [57, 64], b \in [-308, -301], c \in [1213, 1228], \text{ and } r \in [-4796, -4793]. \)

 You divided by the opposite of the factor AND multipled the first factor rather than just bringing it down.
\end{enumerate}

\textbf{General Comment:} Be sure to synthetically divide by the zero of the denominator! Also, make sure to include 0 placeholders for missing terms.
}
\litem{
What are the \textit{possible Rational} roots of the polynomial below?
\[ f(x) = 5x^{3} +3 x^{2} +2 x + 6 \]The solution is \( \text{ All combinations of: }\frac{\pm 1,\pm 2,\pm 3,\pm 6}{\pm 1,\pm 5} \), which is option A.\begin{enumerate}[label=\Alph*.]
\item \( \text{ All combinations of: }\frac{\pm 1,\pm 2,\pm 3,\pm 6}{\pm 1,\pm 5} \)

* This is the solution \textbf{since we asked for the possible Rational roots}!
\item \( \pm 1,\pm 5 \)

 Distractor 1: Corresponds to the plus or minus factors of a1 only.
\item \( \text{ All combinations of: }\frac{\pm 1,\pm 5}{\pm 1,\pm 2,\pm 3,\pm 6} \)

 Distractor 3: Corresponds to the plus or minus of the inverse quotient (an/a0) of the factors. 
\item \( \pm 1,\pm 2,\pm 3,\pm 6 \)

This would have been the solution \textbf{if asked for the possible Integer roots}!
\item \( \text{ There is no formula or theorem that tells us all possible Rational roots.} \)

 Distractor 4: Corresponds to not recalling the theorem for rational roots of a polynomial.
\end{enumerate}

\textbf{General Comment:} We have a way to find the possible Rational roots. The possible Integer roots are the Integers in this list.
}
\end{enumerate}

\end{document}