\documentclass{extbook}[14pt]
\usepackage{multicol, enumerate, enumitem, hyperref, color, soul, setspace, parskip, fancyhdr, amssymb, amsthm, amsmath, bbm, latexsym, units, mathtools}
\everymath{\displaystyle}
\usepackage[headsep=0.5cm,headheight=0cm, left=1 in,right= 1 in,top= 1 in,bottom= 1 in]{geometry}
\usepackage{dashrule}  % Package to use the command below to create lines between items
\newcommand{\litem}[1]{\item #1

\rule{\textwidth}{0.4pt}}
\pagestyle{fancy}
\lhead{}
\chead{Answer Key for Progress Quiz 9 Version A}
\rhead{}
\lfoot{8590-6105}
\cfoot{}
\rfoot{Fall 2020}
\begin{document}
\textbf{This key should allow you to understand why you choose the option you did (beyond just getting a question right or wrong). \href{https://xronos.clas.ufl.edu/mac1105spring2020/courseDescriptionAndMisc/Exams/LearningFromResults}{More instructions on how to use this key can be found here}.}

\textbf{If you have a suggestion to make the keys better, \href{https://forms.gle/CZkbZmPbC9XALEE88}{please fill out the short survey here}.}

\textit{Note: This key is auto-generated and may contain issues and/or errors. The keys are reviewed after each exam to ensure grading is done accurately. If there are issues (like duplicate options), they are noted in the offline gradebook. The keys are a work-in-progress to give students as many resources to improve as possible.}

\rule{\textwidth}{0.4pt}

\begin{enumerate}\litem{
What are the \textit{possible Rational} roots of the polynomial below?
\[ f(x) = 4x^{2} +6 x + 7 \]

The solution is \( \text{ All combinations of: }\frac{\pm 1,\pm 7}{\pm 1,\pm 2,\pm 4} \), which is option C.\begin{enumerate}[label=\Alph*.]
\item \( \pm 1,\pm 7 \)

This would have been the solution \textbf{if asked for the possible Integer roots}!
\item \( \text{ All combinations of: }\frac{\pm 1,\pm 2,\pm 4}{\pm 1,\pm 7} \)

 Distractor 3: Corresponds to the plus or minus of the inverse quotient (an/a0) of the factors. 
\item \( \text{ All combinations of: }\frac{\pm 1,\pm 7}{\pm 1,\pm 2,\pm 4} \)

* This is the solution \textbf{since we asked for the possible Rational roots}!
\item \( \pm 1,\pm 2,\pm 4 \)

 Distractor 1: Corresponds to the plus or minus factors of a1 only.
\item \( \text{ There is no formula or theorem that tells us all possible Rational roots.} \)

 Distractor 4: Corresponds to not recalling the theorem for rational roots of a polynomial.
\end{enumerate}

\textbf{General Comment:} We have a way to find the possible Rational roots. The possible Integer roots are the Integers in this list.
}
\litem{
Perform the division below. Then, find the intervals that correspond to the quotient in the form $ax^2+bx+c$ and remainder $r$.
\[ \frac{15x^{3} -59 x^{2} +34 x + 22}{x -3} \]

The solution is \( 15x^{2} -14 x -8 + \frac{-2}{x -3} \), which is option B.\begin{enumerate}[label=\Alph*.]
\item \( a \in [40, 51], \text{   } b \in [73, 81], \text{   } c \in [258, 269], \text{   and   } r \in [805, 813]. \)

 You multiplied by the synthetic number rather than bringing the first factor down.
\item \( a \in [7, 22], \text{   } b \in [-14, -12], \text{   } c \in [-8, -3], \text{   and   } r \in [-4, 1]. \)

* This is the solution!
\item \( a \in [7, 22], \text{   } b \in [-34, -27], \text{   } c \in [-28, -20], \text{   and   } r \in [-29, -21]. \)

 You multiplied by the synthetic number and subtracted rather than adding during synthetic division.
\item \( a \in [7, 22], \text{   } b \in [-108, -103], \text{   } c \in [341, 347], \text{   and   } r \in [-1017, -1015]. \)

 You divided by the opposite of the factor.
\item \( a \in [40, 51], \text{   } b \in [-196, -187], \text{   } c \in [610, 617], \text{   and   } r \in [-1833, -1822]. \)

 You divided by the opposite of the factor AND multiplied the first factor rather than just bringing it down.
\end{enumerate}

\textbf{General Comment:} Be sure to synthetically divide by the zero of the denominator!
}
\litem{
Factor the polynomial below completely. Then, choose the intervals the zeros of the polynomial belong to, where $z_1 \leq z_2 \leq z_3$. \textit{To make the problem easier, all zeros are between -5 and 5.}
\[ f(x) = 15x^{3} +31 x^{2} -4 x -12 \]

The solution is \( [-2, -0.6666666666666666, 0.6] \), which is option B.\begin{enumerate}[label=\Alph*.]
\item \( z_1 \in [-2.04, -1.9], \text{   }  z_2 \in [-1.77, -1.47], \text{   and   } z_3 \in [1.4, 1.95] \)

 Distractor 2: Corresponds to inversing rational roots.
\item \( z_1 \in [-2.04, -1.9], \text{   }  z_2 \in [-1.02, -0.65], \text{   and   } z_3 \in [0.42, 0.9] \)

* This is the solution!
\item \( z_1 \in [-0.79, -0.28], \text{   }  z_2 \in [0.07, 0.92], \text{   and   } z_3 \in [1.78, 2.47] \)

 Distractor 1: Corresponds to negatives of all zeros.
\item \( z_1 \in [-0.34, 0.28], \text{   }  z_2 \in [1.92, 2.25], \text{   and   } z_3 \in [1.78, 2.47] \)

 Distractor 4: Corresponds to moving factors from one rational to another.
\item \( z_1 \in [-1.73, -1.63], \text{   }  z_2 \in [0.97, 1.56], \text{   and   } z_3 \in [1.78, 2.47] \)

 Distractor 3: Corresponds to negatives of all zeros AND inversing rational roots.
\end{enumerate}

\textbf{General Comment:} Remember to try the middle-most integers first as these normally are the zeros. Also, once you get it to a quadratic, you can use your other factoring techniques to finish factoring.
}
\litem{
Factor the polynomial below completely, knowing that $x-4$ is a factor. Then, choose the intervals the zeros of the polynomial belong to, where $z_1 \leq z_2 \leq z_3 \leq z_4$. \textit{To make the problem easier, all zeros are between -5 and 5.}
\[ f(x) = 15x^{4} +2 x^{3} -248 x^{2} -32 x + 128 \]

The solution is \( [-4, -0.8, 0.6666666666666666, 4] \), which is option D.\begin{enumerate}[label=\Alph*.]
\item \( z_1 \in [-6, -2], \text{   }  z_2 \in [-0.42, -0.13], z_3 \in [3.8, 4.07], \text{   and   } z_4 \in [4, 6] \)

 Distractor 4: Corresponds to moving factors from one rational to another.
\item \( z_1 \in [-6, -2], \text{   }  z_2 \in [-0.69, -0.45], z_3 \in [0.76, 0.95], \text{   and   } z_4 \in [4, 6] \)

 Distractor 1: Corresponds to negatives of all zeros.
\item \( z_1 \in [-6, -2], \text{   }  z_2 \in [-1.56, -1.26], z_3 \in [1.15, 1.45], \text{   and   } z_4 \in [4, 6] \)

 Distractor 3: Corresponds to negatives of all zeros AND inversing rational roots.
\item \( z_1 \in [-6, -2], \text{   }  z_2 \in [-0.9, -0.68], z_3 \in [0.61, 0.68], \text{   and   } z_4 \in [4, 6] \)

* This is the solution!
\item \( z_1 \in [-6, -2], \text{   }  z_2 \in [-1.39, -1.19], z_3 \in [1.46, 1.7], \text{   and   } z_4 \in [4, 6] \)

 Distractor 2: Corresponds to inversing rational roots.
\end{enumerate}

\textbf{General Comment:} Remember to try the middle-most integers first as these normally are the zeros. Also, once you get it to a quadratic, you can use your other factoring techniques to finish factoring.
}
\litem{
Perform the division below. Then, find the intervals that correspond to the quotient in the form $ax^2+bx+c$ and remainder $r$.
\[ \frac{8x^{3} -24 x -20}{x -2} \]

The solution is \( 8x^{2} +16 x + 8 + \frac{-4}{x -2} \), which is option C.\begin{enumerate}[label=\Alph*.]
\item \( a \in [6, 14], b \in [5, 15], c \in [-18, -8], \text{ and } r \in [-36, -34]. \)

 You multipled by the synthetic number and subtracted rather than adding during synthetic division.
\item \( a \in [6, 14], b \in [-19, -14], c \in [6, 11], \text{ and } r \in [-36, -34]. \)

 You divided by the opposite of the factor.
\item \( a \in [6, 14], b \in [13, 17], c \in [6, 11], \text{ and } r \in [-9, 4]. \)

* This is the solution!
\item \( a \in [12, 18], b \in [-35, -29], c \in [37, 46], \text{ and } r \in [-101, -98]. \)

 You divided by the opposite of the factor AND multipled the first factor rather than just bringing it down.
\item \( a \in [12, 18], b \in [27, 33], c \in [37, 46], \text{ and } r \in [55, 63]. \)

 You multipled by the synthetic number rather than bringing the first factor down.
\end{enumerate}

\textbf{General Comment:} Be sure to synthetically divide by the zero of the denominator! Also, make sure to include 0 placeholders for missing terms.
}
\litem{
Factor the polynomial below completely. Then, choose the intervals the zeros of the polynomial belong to, where $z_1 \leq z_2 \leq z_3$. \textit{To make the problem easier, all zeros are between -5 and 5.}
\[ f(x) = 9x^{3} -12 x^{2} -20 x + 16 \]

The solution is \( [-1.3333333333333333, 0.6666666666666666, 2] \), which is option D.\begin{enumerate}[label=\Alph*.]
\item \( z_1 \in [-1.04, -0.1], \text{   }  z_2 \in [1.43, 2.35], \text{   and   } z_3 \in [1.4, 2.7] \)

 Distractor 2: Corresponds to inversing rational roots.
\item \( z_1 \in [-2.26, -1.53], \text{   }  z_2 \in [-1.77, -0.95], \text{   and   } z_3 \in [-0.4, 1] \)

 Distractor 3: Corresponds to negatives of all zeros AND inversing rational roots.
\item \( z_1 \in [-2.26, -1.53], \text{   }  z_2 \in [-0.29, 0.02], \text{   and   } z_3 \in [3.7, 4.9] \)

 Distractor 4: Corresponds to moving factors from one rational to another.
\item \( z_1 \in [-1.52, -1.21], \text{   }  z_2 \in [0.51, 0.98], \text{   and   } z_3 \in [1.4, 2.7] \)

* This is the solution!
\item \( z_1 \in [-2.26, -1.53], \text{   }  z_2 \in [-0.98, -0.54], \text{   and   } z_3 \in [1.2, 1.7] \)

 Distractor 1: Corresponds to negatives of all zeros.
\end{enumerate}

\textbf{General Comment:} Remember to try the middle-most integers first as these normally are the zeros. Also, once you get it to a quadratic, you can use your other factoring techniques to finish factoring.
}
\litem{
Perform the division below. Then, find the intervals that correspond to the quotient in the form $ax^2+bx+c$ and remainder $r$.
\[ \frac{15x^{3} -26 x^{2} -51 x -16}{x -3} \]

The solution is \( 15x^{2} +19 x + 6 + \frac{2}{x -3} \), which is option A.\begin{enumerate}[label=\Alph*.]
\item \( a \in [15, 19], \text{   } b \in [19, 28], \text{   } c \in [6, 8], \text{   and   } r \in [-3, 7]. \)

* This is the solution!
\item \( a \in [43, 53], \text{   } b \in [106, 113], \text{   } c \in [271, 281], \text{   and   } r \in [812, 816]. \)

 You multiplied by the synthetic number rather than bringing the first factor down.
\item \( a \in [15, 19], \text{   } b \in [-5, 5], \text{   } c \in [-48, -38], \text{   and   } r \in [-106, -101]. \)

 You multiplied by the synthetic number and subtracted rather than adding during synthetic division.
\item \( a \in [43, 53], \text{   } b \in [-164, -150], \text{   } c \in [431, 435], \text{   and   } r \in [-1312, -1303]. \)

 You divided by the opposite of the factor AND multiplied the first factor rather than just bringing it down.
\item \( a \in [15, 19], \text{   } b \in [-73, -69], \text{   } c \in [159, 164], \text{   and   } r \in [-503, -498]. \)

 You divided by the opposite of the factor.
\end{enumerate}

\textbf{General Comment:} Be sure to synthetically divide by the zero of the denominator!
}
\litem{
What are the \textit{possible Rational} roots of the polynomial below?
\[ f(x) = 2x^{3} +2 x^{2} +2 x + 3 \]

The solution is \( \text{ All combinations of: }\frac{\pm 1,\pm 3}{\pm 1,\pm 2} \), which is option B.\begin{enumerate}[label=\Alph*.]
\item \( \pm 1,\pm 3 \)

This would have been the solution \textbf{if asked for the possible Integer roots}!
\item \( \text{ All combinations of: }\frac{\pm 1,\pm 3}{\pm 1,\pm 2} \)

* This is the solution \textbf{since we asked for the possible Rational roots}!
\item \( \text{ All combinations of: }\frac{\pm 1,\pm 2}{\pm 1,\pm 3} \)

 Distractor 3: Corresponds to the plus or minus of the inverse quotient (an/a0) of the factors. 
\item \( \pm 1,\pm 2 \)

 Distractor 1: Corresponds to the plus or minus factors of a1 only.
\item \( \text{ There is no formula or theorem that tells us all possible Rational roots.} \)

 Distractor 4: Corresponds to not recalling the theorem for rational roots of a polynomial.
\end{enumerate}

\textbf{General Comment:} We have a way to find the possible Rational roots. The possible Integer roots are the Integers in this list.
}
\litem{
Perform the division below. Then, find the intervals that correspond to the quotient in the form $ax^2+bx+c$ and remainder $r$.
\[ \frac{8x^{3} -42 x -15}{x + 2} \]

The solution is \( 8x^{2} -16 x -10 + \frac{5}{x + 2} \), which is option D.\begin{enumerate}[label=\Alph*.]
\item \( a \in [-21, -12], b \in [-41, -30], c \in [-106, -103], \text{ and } r \in [-229, -223]. \)

 You divided by the opposite of the factor AND multipled the first factor rather than just bringing it down.
\item \( a \in [-21, -12], b \in [31, 34], c \in [-106, -103], \text{ and } r \in [193, 199]. \)

 You multipled by the synthetic number rather than bringing the first factor down.
\item \( a \in [0, 12], b \in [-26, -19], c \in [29, 36], \text{ and } r \in [-106, -99]. \)

 You multipled by the synthetic number and subtracted rather than adding during synthetic division.
\item \( a \in [0, 12], b \in [-18, -11], c \in [-12, -5], \text{ and } r \in [4, 9]. \)

* This is the solution!
\item \( a \in [0, 12], b \in [16, 20], c \in [-12, -5], \text{ and } r \in [-40, -31]. \)

 You divided by the opposite of the factor.
\end{enumerate}

\textbf{General Comment:} Be sure to synthetically divide by the zero of the denominator! Also, make sure to include 0 placeholders for missing terms.
}
\litem{
Factor the polynomial below completely, knowing that $x+2$ is a factor. Then, choose the intervals the zeros of the polynomial belong to, where $z_1 \leq z_2 \leq z_3 \leq z_4$. \textit{To make the problem easier, all zeros are between -5 and 5.}
\[ f(x) = 8x^{4} +42 x^{3} +19 x^{2} -102 x -72 \]

The solution is \( [-4, -2, -0.75, 1.5] \), which is option C.\begin{enumerate}[label=\Alph*.]
\item \( z_1 \in [-1.63, -1.29], \text{   }  z_2 \in [0.36, 0.88], z_3 \in [1.71, 2.43], \text{   and   } z_4 \in [2.9, 5.1] \)

 Distractor 1: Corresponds to negatives of all zeros.
\item \( z_1 \in [-0.49, 0.11], \text{   }  z_2 \in [1.74, 2.29], z_3 \in [2.58, 3.05], \text{   and   } z_4 \in [2.9, 5.1] \)

 Distractor 4: Corresponds to moving factors from one rational to another.
\item \( z_1 \in [-4, -3.77], \text{   }  z_2 \in [-2.39, -1.93], z_3 \in [-1.09, 0.14], \text{   and   } z_4 \in [0.8, 2.2] \)

* This is the solution!
\item \( z_1 \in [-0.83, -0.43], \text{   }  z_2 \in [1.3, 1.49], z_3 \in [1.71, 2.43], \text{   and   } z_4 \in [2.9, 5.1] \)

 Distractor 3: Corresponds to negatives of all zeros AND inversing rational roots.
\item \( z_1 \in [-4, -3.77], \text{   }  z_2 \in [-2.39, -1.93], z_3 \in [-2.34, -1.07], \text{   and   } z_4 \in [-0.5, 0.8] \)

 Distractor 2: Corresponds to inversing rational roots.
\end{enumerate}

\textbf{General Comment:} Remember to try the middle-most integers first as these normally are the zeros. Also, once you get it to a quadratic, you can use your other factoring techniques to finish factoring.
}
\end{enumerate}

\end{document}