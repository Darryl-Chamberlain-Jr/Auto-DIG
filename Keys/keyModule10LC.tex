\documentclass{extbook}[14pt]
\usepackage{multicol, enumerate, enumitem, hyperref, color, soul, setspace, parskip, fancyhdr, amssymb, amsthm, amsmath, bbm, latexsym, units, mathtools}
\everymath{\displaystyle}
\usepackage[headsep=0.5cm,headheight=0cm, left=1 in,right= 1 in,top= 1 in,bottom= 1 in]{geometry}
\pagestyle{fancy}
\lhead{}
\chead{Answer Key for Module\,10L\,-\,Synthetic\,Division Version C}
\rhead{}
\lfoot{Summer\,C\,2020}
\cfoot{}
\rfoot{}
\begin{document}
\textbf{This key should allow you to understand why you choose the option you did (beyond just getting a question right or wrong). \href{https://xronos.clas.ufl.edu/mac1105spring2020/courseDescriptionAndMisc/Exams/LearningFromResults}{More instructions on how to use this key can be found here}.}

\textbf{If you have a suggestion to make the keys better, \href{https://forms.gle/CZkbZmPbC9XALEE88}{please fill out the short survey here}.}

\textit{Note: This key is auto-generated and may contain issues and/or errors. The keys are reviewed after each exam to ensure grading is done accurately. If there are issues (like duplicate options), they are noted in the offline gradebook. The keys are a work-in-progress to give students as many resources to improve as possible.}

\rule{\textwidth}{0.4pt}

66. Perform the division below. Then, find the intervals that correspond to the quotient in the form $ax^2+bx+c$ and remainder $r$.
\[ \frac{9x^{3} -21 x^{2} + 10}{x -2} \] 
The solution is $ 9x^{2} -3 x -6 + \frac{-2}{x -2} $ 

\begin{enumerate}[label=\Alph*.] 
\item $ a \in [3, 12], b \in [-40, -38], c \in [73, 83], \text{ and } r \in [-151, -144]. $ 

  You divided by the opposite of the factor. 
\item $ a \in [3, 12], b \in [-16, -10], c \in [-13, -10], \text{ and } r \in [-5, 1]. $ 

  You multipled by the synthetic number and subtracted rather than adding during synthetic division. 
\item $ a \in [15, 20], b \in [14, 20], c \in [29, 38], \text{ and } r \in [62, 75]. $ 

  You multipled by the synthetic number rather than bringing the first factor down. 
\item $ a \in [15, 20], b \in [-59, -54], c \in [113, 120], \text{ and } r \in [-220, -210]. $ 

  You divided by the opposite of the factor AND multipled the first factor rather than just bringing it down. 
\item $ a \in [3, 12], b \in [-8, -2], c \in [-11, -1], \text{ and } r \in [-5, 1]. $ 

 * This is the solution! 
\end{enumerate} 
 
General Comments: Be sure to synthetically divide by the zero of the denominator! Also, make sure to include 0 placeholders for missing terms.

-----------------------------------------------

67. Factor the polynomial below completely, knowing that $x-2$ is a factor. Then, choose the intervals the zeros of the polynomial belong to, where $z_1 \leq z_2 \leq z_3 \leq z_4$. \textit{To make the problem easier, all zeros are between -5 and 5.}
\[ f(x) = 6x^{4} +31 x^{3} +5 x^{2} -122 x -120 \] 
The solution is $ [-4, -1.6666666666666667, -1.5, 2] $ 

\begin{enumerate}[label=\Alph*.] 
\item $ z_1 \in [-3, 2], \text{   }  z_2 \in [0.56, 0.81], z_3 \in [0.47, 1], \text{   and   } z_4 \in [3.2, 4.6] $ 

  Distractor 3: Corresponds to negatives of all zeros AND inversing rational roots. 
\item $ z_1 \in [-8, -3], \text{   }  z_2 \in [-0.8, -0.41], z_3 \in [-0.66, -0.15], \text{   and   } z_4 \in [1.8, 3.3] $ 

  Distractor 2: Corresponds to inversing rational roots. 
\item $ z_1 \in [-3, 2], \text{   }  z_2 \in [0.66, 0.85], z_3 \in [2.76, 4.17], \text{   and   } z_4 \in [3.2, 4.6] $ 

  Distractor 4: Corresponds to moving factors from one rational to another. 
\item $ z_1 \in [-3, 2], \text{   }  z_2 \in [1.5, 1.54], z_3 \in [0.96, 2.43], \text{   and   } z_4 \in [3.2, 4.6] $ 

  Distractor 1: Corresponds to negatives of all zeros. 
\item $ z_1 \in [-8, -3], \text{   }  z_2 \in [-1.78, -1.63], z_3 \in [-1.97, -1.17], \text{   and   } z_4 \in [1.8, 3.3] $ 

 * This is the solution! 
\end{enumerate} 
 
General Comments: Remember to try the middle-most integers first as these normally are the zeros. Also, once you get it to a quadratic, you can use your other factoring techniques to finish factoring.

-----------------------------------------------

68. Perform the division below. Then, find the intervals that correspond to the quotient in the form $ax^2+bx+c$ and remainder $r$.
\[ \frac{6x^{3} -38 x^{2} +76 x -50}{x -3} \] 
The solution is $ 6x^{2} -20 x + 16 + \frac{-2}{x -3} $ 

\begin{enumerate}[label=\Alph*.] 
\item $ a \in [15, 20], \text{   } b \in [9, 20], \text{   } c \in [123, 131], \text{   and   } r \in [319, 324]. $ 

  You multiplied by the synthetic number rather than bringing the first factor down. 
\item $ a \in [4, 8], \text{   } b \in [-29, -21], \text{   } c \in [20, 26], \text{   and   } r \in [-3, 3]. $ 

  You multiplied by the synthetic number and subtracted rather than adding during synthetic division. 
\item $ a \in [15, 20], \text{   } b \in [-99, -91], \text{   } c \in [347, 355], \text{   and   } r \in [-1112, -1102]. $ 

  You divided by the opposite of the factor AND multiplied the first factor rather than just bringing it down. 
\item $ a \in [4, 8], \text{   } b \in [-24, -19], \text{   } c \in [14, 22], \text{   and   } r \in [-3, 3]. $ 

 * This is the solution! 
\item $ a \in [4, 8], \text{   } b \in [-59, -54], \text{   } c \in [243, 247], \text{   and   } r \in [-783, -777]. $ 

  You divided by the opposite of the factor. 
\end{enumerate} 
 
General Comments: Be sure to synthetically divide by the zero of the denominator!

-----------------------------------------------

69. Factor the polynomial below completely. Then, choose the intervals the zeros of the polynomial belong to, where $z_1 \leq z_2 \leq z_3$. \textit{To make the problem easier, all zeros are between -5 and 5.}
\[ f(x) = 9x^{3} +9 x^{2} -28 x -20 \] 
The solution is $ [-2, -0.6666666666666666, 1.6666666666666667] $ 

\begin{enumerate}[label=\Alph*.] 
\item $ z_1 \in [-1.92, -1.5], \text{   }  z_2 \in [0.54, 0.69], \text{   and   } z_3 \in [1.89, 2.3] $ 

  Distractor 1: Corresponds to negatives of all zeros. 
\item $ z_1 \in [-5.17, -4.86], \text{   }  z_2 \in [0.01, 0.57], \text{   and   } z_3 \in [1.89, 2.3] $ 

  Distractor 4: Corresponds to moving factors from one rational to another. 
\item $ z_1 \in [-2.19, -1.97], \text{   }  z_2 \in [-1.64, -1.25], \text{   and   } z_3 \in [0.37, 0.66] $ 

  Distractor 2: Corresponds to inversing rational roots. 
\item $ z_1 \in [-0.88, -0.21], \text{   }  z_2 \in [1.24, 1.7], \text{   and   } z_3 \in [1.89, 2.3] $ 

  Distractor 3: Corresponds to negatives of all zeros AND inversing rational roots. 
\item $ z_1 \in [-2.19, -1.97], \text{   }  z_2 \in [-1.1, -0.25], \text{   and   } z_3 \in [1.49, 1.7] $ 

 * This is the solution! 
\end{enumerate} 
 
General Comments: Remember to try the middle-most integers first as these normally are the zeros. Also, once you get it to a quadratic, you can use your other factoring techniques to finish factoring.

-----------------------------------------------

70. What are the \textit{possible Integer} roots of the polynomial below?
\[ f(x) = 2x^{2} +5 x + 4 \] 
The solution is $ \pm 1,\pm 2,\pm 4 $ 

\begin{enumerate}[label=\Alph*.] 
\item $ \pm 1,\pm 2 $ 

  Distractor 1: Corresponds to the plus or minus factors of a1 only. 
\item $ \pm 1,\pm 2,\pm 4 $ 

 * This is the solution \textbf{since we asked for the possible Integer roots}! 
\item $ \text{ All combinations of: }\frac{\pm 1,\pm 2}{\pm 1,\pm 2,\pm 4} $ 

  Distractor 3: Corresponds to the plus or minus of the inverse quotient (an/a0) of the factors.  
\item $ \text{ All combinations of: }\frac{\pm 1,\pm 2,\pm 4}{\pm 1,\pm 2} $ 

 This would have been the solution \textbf{if asked for the possible Rational roots}! 
\item $ \text{There is no formula or theorem that tells us all possible Integer roots.} $ 

  Distractor 4: Corresponds to not recognizing Integers as a subset of Rationals. 
\end{enumerate} 
 
General Comments: We have a way to find the possible Rational roots. The possible Integer roots are the Integers in this list.

-----------------------------------------------


\end{document}

