\documentclass{extbook}[14pt]
\usepackage{multicol, enumerate, enumitem, hyperref, color, soul, setspace, parskip, fancyhdr, amssymb, amsthm, amsmath, latexsym, units, mathtools}
\everymath{\displaystyle}
\usepackage[headsep=0.5cm,headheight=0cm, left=1 in,right= 1 in,top= 1 in,bottom= 1 in]{geometry}
\usepackage{dashrule}  % Package to use the command below to create lines between items
\newcommand{\litem}[1]{\item #1

\rule{\textwidth}{0.4pt}}
\pagestyle{fancy}
\lhead{}
\chead{Answer Key for Progress Quiz 6 Version C}
\rhead{}
\lfoot{9689-6866}
\cfoot{}
\rfoot{Spring 2021}
\begin{document}
\textbf{This key should allow you to understand why you choose the option you did (beyond just getting a question right or wrong). \href{https://xronos.clas.ufl.edu/mac1105spring2020/courseDescriptionAndMisc/Exams/LearningFromResults}{More instructions on how to use this key can be found here}.}

\textbf{If you have a suggestion to make the keys better, \href{https://forms.gle/CZkbZmPbC9XALEE88}{please fill out the short survey here}.}

\textit{Note: This key is auto-generated and may contain issues and/or errors. The keys are reviewed after each exam to ensure grading is done accurately. If there are issues (like duplicate options), they are noted in the offline gradebook. The keys are a work-in-progress to give students as many resources to improve as possible.}

\rule{\textwidth}{0.4pt}

\begin{enumerate}\litem{
Perform the division below. Then, find the intervals that correspond to the quotient in the form $ax^2+bx+c$ and remainder $r$.
\[ \frac{6x^{3} +12 x^{2} -78 x + 65}{x + 5} \]The solution is \( 6x^{2} -18 x + 12 + \frac{5}{x + 5} \), which is option B.\begin{enumerate}[label=\Alph*.]
\item \( a \in [5, 8], \text{   } b \in [39, 48], \text{   } c \in [129, 136], \text{   and   } r \in [725, 729]. \)

 You divided by the opposite of the factor.
\item \( a \in [5, 8], \text{   } b \in [-18, -15], \text{   } c \in [8, 14], \text{   and   } r \in [4, 8]. \)

* This is the solution!
\item \( a \in [-36, -29], \text{   } b \in [158, 163], \text{   } c \in [-890, -883], \text{   and   } r \in [4503, 4507]. \)

 You multiplied by the synthetic number rather than bringing the first factor down.
\item \( a \in [5, 8], \text{   } b \in [-28, -19], \text{   } c \in [61, 70], \text{   and   } r \in [-335, -324]. \)

 You multiplied by the synthetic number and subtracted rather than adding during synthetic division.
\item \( a \in [-36, -29], \text{   } b \in [-140, -132], \text{   } c \in [-775, -767], \text{   and   } r \in [-3775, -3773]. \)

 You divided by the opposite of the factor AND multiplied the first factor rather than just bringing it down.
\end{enumerate}

\textbf{General Comment:} Be sure to synthetically divide by the zero of the denominator!
}
\litem{
Factor the polynomial below completely. Then, choose the intervals the zeros of the polynomial belong to, where $z_1 \leq z_2 \leq z_3$. \textit{To make the problem easier, all zeros are between -5 and 5.}
\[ f(x) = 6x^{3} +19 x^{2} -65 x -50 \]The solution is \( [-5, -0.6666666666666666, 2.5] \), which is option B.\begin{enumerate}[label=\Alph*.]
\item \( z_1 \in [-2.8, -1.1], \text{   }  z_2 \in [0.01, 1.17], \text{   and   } z_3 \in [5, 7] \)

 Distractor 1: Corresponds to negatives of all zeros.
\item \( z_1 \in [-5.7, -4.5], \text{   }  z_2 \in [-0.94, -0.62], \text{   and   } z_3 \in [2.5, 4.5] \)

* This is the solution!
\item \( z_1 \in [-5.7, -4.5], \text{   }  z_2 \in [-1.56, -1.42], \text{   and   } z_3 \in [0.4, 1.4] \)

 Distractor 2: Corresponds to inversing rational roots.
\item \( z_1 \in [-0.5, 0.4], \text{   }  z_2 \in [1.08, 1.78], \text{   and   } z_3 \in [5, 7] \)

 Distractor 3: Corresponds to negatives of all zeros AND inversing rational roots.
\item \( z_1 \in [-1.4, -0.6], \text{   }  z_2 \in [1.6, 2.33], \text{   and   } z_3 \in [5, 7] \)

 Distractor 4: Corresponds to moving factors from one rational to another.
\end{enumerate}

\textbf{General Comment:} Remember to try the middle-most integers first as these normally are the zeros. Also, once you get it to a quadratic, you can use your other factoring techniques to finish factoring.
}
\litem{
Factor the polynomial below completely, knowing that $x+5$ is a factor. Then, choose the intervals the zeros of the polynomial belong to, where $z_1 \leq z_2 \leq z_3 \leq z_4$. \textit{To make the problem easier, all zeros are between -5 and 5.}
\[ f(x) = 8x^{4} +54 x^{3} +15 x^{2} -350 x -375 \]The solution is \( [-5, -3, -1.25, 2.5] \), which is option D.\begin{enumerate}[label=\Alph*.]
\item \( z_1 \in [-3, -1.9], \text{   }  z_2 \in [1.09, 1.55], z_3 \in [2.85, 3.02], \text{   and   } z_4 \in [4.3, 5.3] \)

 Distractor 1: Corresponds to negatives of all zeros.
\item \( z_1 \in [-5.9, -3.4], \text{   }  z_2 \in [0.51, 0.66], z_3 \in [2.85, 3.02], \text{   and   } z_4 \in [4.3, 5.3] \)

 Distractor 4: Corresponds to moving factors from one rational to another.
\item \( z_1 \in [-5.9, -3.4], \text{   }  z_2 \in [-3.17, -2.87], z_3 \in [-1.05, -0.67], \text{   and   } z_4 \in [-0.7, 0.5] \)

 Distractor 2: Corresponds to inversing rational roots.
\item \( z_1 \in [-5.9, -3.4], \text{   }  z_2 \in [-3.17, -2.87], z_3 \in [-1.67, -1.07], \text{   and   } z_4 \in [2.1, 4.2] \)

* This is the solution!
\item \( z_1 \in [-0.8, -0.3], \text{   }  z_2 \in [0.77, 0.92], z_3 \in [2.85, 3.02], \text{   and   } z_4 \in [4.3, 5.3] \)

 Distractor 3: Corresponds to negatives of all zeros AND inversing rational roots.
\end{enumerate}

\textbf{General Comment:} Remember to try the middle-most integers first as these normally are the zeros. Also, once you get it to a quadratic, you can use your other factoring techniques to finish factoring.
}
\litem{
Factor the polynomial below completely, knowing that $x+5$ is a factor. Then, choose the intervals the zeros of the polynomial belong to, where $z_1 \leq z_2 \leq z_3 \leq z_4$. \textit{To make the problem easier, all zeros are between -5 and 5.}
\[ f(x) = 6x^{4} +35 x^{3} -9 x^{2} -210 x -200 \]The solution is \( [-5, -2, -1.3333333333333333, 2.5] \), which is option E.\begin{enumerate}[label=\Alph*.]
\item \( z_1 \in [-2.91, -2.04], \text{   }  z_2 \in [1.1, 1.43], z_3 \in [1.76, 2.49], \text{   and   } z_4 \in [4.7, 5.1] \)

 Distractor 1: Corresponds to negatives of all zeros.
\item \( z_1 \in [-5.06, -4.82], \text{   }  z_2 \in [-2.04, -1.88], z_3 \in [-1.2, -0.26], \text{   and   } z_4 \in [0, 1.4] \)

 Distractor 2: Corresponds to inversing rational roots.
\item \( z_1 \in [-0.93, -0.54], \text{   }  z_2 \in [1.78, 2.01], z_3 \in [3.71, 4.91], \text{   and   } z_4 \in [4.7, 5.1] \)

 Distractor 4: Corresponds to moving factors from one rational to another.
\item \( z_1 \in [-0.41, 0], \text{   }  z_2 \in [0.58, 1.14], z_3 \in [1.76, 2.49], \text{   and   } z_4 \in [4.7, 5.1] \)

 Distractor 3: Corresponds to negatives of all zeros AND inversing rational roots.
\item \( z_1 \in [-5.06, -4.82], \text{   }  z_2 \in [-2.04, -1.88], z_3 \in [-1.43, -0.99], \text{   and   } z_4 \in [2.2, 4.7] \)

* This is the solution!
\end{enumerate}

\textbf{General Comment:} Remember to try the middle-most integers first as these normally are the zeros. Also, once you get it to a quadratic, you can use your other factoring techniques to finish factoring.
}
\litem{
What are the \textit{possible Integer} roots of the polynomial below?
\[ f(x) = 3x^{4} +7 x^{3} +2 x^{2} +5 x + 7 \]The solution is \( \pm 1,\pm 7 \), which is option C.\begin{enumerate}[label=\Alph*.]
\item \( \text{ All combinations of: }\frac{\pm 1,\pm 7}{\pm 1,\pm 3} \)

This would have been the solution \textbf{if asked for the possible Rational roots}!
\item \( \text{ All combinations of: }\frac{\pm 1,\pm 3}{\pm 1,\pm 7} \)

 Distractor 3: Corresponds to the plus or minus of the inverse quotient (an/a0) of the factors. 
\item \( \pm 1,\pm 7 \)

* This is the solution \textbf{since we asked for the possible Integer roots}!
\item \( \pm 1,\pm 3 \)

 Distractor 1: Corresponds to the plus or minus factors of a1 only.
\item \( \text{There is no formula or theorem that tells us all possible Integer roots.} \)

 Distractor 4: Corresponds to not recognizing Integers as a subset of Rationals.
\end{enumerate}

\textbf{General Comment:} We have a way to find the possible Rational roots. The possible Integer roots are the Integers in this list.
}
\litem{
Factor the polynomial below completely. Then, choose the intervals the zeros of the polynomial belong to, where $z_1 \leq z_2 \leq z_3$. \textit{To make the problem easier, all zeros are between -5 and 5.}
\[ f(x) = 4x^{3} +4 x^{2} -23 x -30 \]The solution is \( [-2, -1.5, 2.5] \), which is option C.\begin{enumerate}[label=\Alph*.]
\item \( z_1 \in [-5.03, -4.83], \text{   }  z_2 \in [0.73, 0.83], \text{   and   } z_3 \in [1.36, 2.23] \)

 Distractor 4: Corresponds to moving factors from one rational to another.
\item \( z_1 \in [-2.6, -2.36], \text{   }  z_2 \in [1.48, 1.51], \text{   and   } z_3 \in [1.36, 2.23] \)

 Distractor 1: Corresponds to negatives of all zeros.
\item \( z_1 \in [-2.45, -1.97], \text{   }  z_2 \in [-1.58, -1.45], \text{   and   } z_3 \in [2.1, 2.57] \)

* This is the solution!
\item \( z_1 \in [-0.72, -0.31], \text{   }  z_2 \in [0.66, 0.68], \text{   and   } z_3 \in [1.36, 2.23] \)

 Distractor 3: Corresponds to negatives of all zeros AND inversing rational roots.
\item \( z_1 \in [-2.45, -1.97], \text{   }  z_2 \in [-0.77, -0.56], \text{   and   } z_3 \in [0.32, 0.7] \)

 Distractor 2: Corresponds to inversing rational roots.
\end{enumerate}

\textbf{General Comment:} Remember to try the middle-most integers first as these normally are the zeros. Also, once you get it to a quadratic, you can use your other factoring techniques to finish factoring.
}
\litem{
Perform the division below. Then, find the intervals that correspond to the quotient in the form $ax^2+bx+c$ and remainder $r$.
\[ \frac{8x^{3} +44 x^{2} +16 x -25}{x + 5} \]The solution is \( 8x^{2} +4 x -4 + \frac{-5}{x + 5} \), which is option E.\begin{enumerate}[label=\Alph*.]
\item \( a \in [-48, -39], \text{   } b \in [238, 247], \text{   } c \in [-1209, -1200], \text{   and   } r \in [5995, 6001]. \)

 You multiplied by the synthetic number rather than bringing the first factor down.
\item \( a \in [3, 13], \text{   } b \in [-8, 1], \text{   } c \in [32, 41], \text{   and   } r \in [-267, -261]. \)

 You multiplied by the synthetic number and subtracted rather than adding during synthetic division.
\item \( a \in [3, 13], \text{   } b \in [81, 85], \text{   } c \in [435, 440], \text{   and   } r \in [2152, 2163]. \)

 You divided by the opposite of the factor.
\item \( a \in [-48, -39], \text{   } b \in [-160, -153], \text{   } c \in [-767, -762], \text{   and   } r \in [-3847, -3841]. \)

 You divided by the opposite of the factor AND multiplied the first factor rather than just bringing it down.
\item \( a \in [3, 13], \text{   } b \in [4, 8], \text{   } c \in [-7, 1], \text{   and   } r \in [-7, -4]. \)

* This is the solution!
\end{enumerate}

\textbf{General Comment:} Be sure to synthetically divide by the zero of the denominator!
}
\litem{
What are the \textit{possible Rational} roots of the polynomial below?
\[ f(x) = 4x^{3} +5 x^{2} +4 x + 7 \]The solution is \( \text{ All combinations of: }\frac{\pm 1,\pm 7}{\pm 1,\pm 2,\pm 4} \), which is option B.\begin{enumerate}[label=\Alph*.]
\item \( \text{ All combinations of: }\frac{\pm 1,\pm 2,\pm 4}{\pm 1,\pm 7} \)

 Distractor 3: Corresponds to the plus or minus of the inverse quotient (an/a0) of the factors. 
\item \( \text{ All combinations of: }\frac{\pm 1,\pm 7}{\pm 1,\pm 2,\pm 4} \)

* This is the solution \textbf{since we asked for the possible Rational roots}!
\item \( \pm 1,\pm 2,\pm 4 \)

 Distractor 1: Corresponds to the plus or minus factors of a1 only.
\item \( \pm 1,\pm 7 \)

This would have been the solution \textbf{if asked for the possible Integer roots}!
\item \( \text{ There is no formula or theorem that tells us all possible Rational roots.} \)

 Distractor 4: Corresponds to not recalling the theorem for rational roots of a polynomial.
\end{enumerate}

\textbf{General Comment:} We have a way to find the possible Rational roots. The possible Integer roots are the Integers in this list.
}
\litem{
Perform the division below. Then, find the intervals that correspond to the quotient in the form $ax^2+bx+c$ and remainder $r$.
\[ \frac{12x^{3} -28 x^{2} + 14}{x -2} \]The solution is \( 12x^{2} -4 x -8 + \frac{-2}{x -2} \), which is option A.\begin{enumerate}[label=\Alph*.]
\item \( a \in [8, 16], b \in [-7, -2], c \in [-9, -4], \text{ and } r \in [-7, 4]. \)

* This is the solution!
\item \( a \in [17, 30], b \in [-77, -73], c \in [152, 155], \text{ and } r \in [-296, -289]. \)

 You divided by the opposite of the factor AND multipled the first factor rather than just bringing it down.
\item \( a \in [8, 16], b \in [-52, -50], c \in [102, 111], \text{ and } r \in [-197, -192]. \)

 You divided by the opposite of the factor.
\item \( a \in [8, 16], b \in [-22, -11], c \in [-20, -15], \text{ and } r \in [-7, 4]. \)

 You multipled by the synthetic number and subtracted rather than adding during synthetic division.
\item \( a \in [17, 30], b \in [15, 23], c \in [38, 42], \text{ and } r \in [91, 99]. \)

 You multipled by the synthetic number rather than bringing the first factor down.
\end{enumerate}

\textbf{General Comment:} Be sure to synthetically divide by the zero of the denominator! Also, make sure to include 0 placeholders for missing terms.
}
\litem{
Perform the division below. Then, find the intervals that correspond to the quotient in the form $ax^2+bx+c$ and remainder $r$.
\[ \frac{9x^{3} +21 x^{2} -8}{x + 2} \]The solution is \( 9x^{2} +3 x -6 + \frac{4}{x + 2} \), which is option C.\begin{enumerate}[label=\Alph*.]
\item \( a \in [6, 11], b \in [36, 46], c \in [75, 79], \text{ and } r \in [143, 151]. \)

 You divided by the opposite of the factor.
\item \( a \in [-24, -11], b \in [54, 63], c \in [-115, -110], \text{ and } r \in [218, 226]. \)

 You multipled by the synthetic number rather than bringing the first factor down.
\item \( a \in [6, 11], b \in [1, 5], c \in [-9, -2], \text{ and } r \in [0, 6]. \)

* This is the solution!
\item \( a \in [-24, -11], b \in [-17, -13], c \in [-32, -28], \text{ and } r \in [-70, -66]. \)

 You divided by the opposite of the factor AND multipled the first factor rather than just bringing it down.
\item \( a \in [6, 11], b \in [-6, 1], c \in [14, 24], \text{ and } r \in [-63, -61]. \)

 You multipled by the synthetic number and subtracted rather than adding during synthetic division.
\end{enumerate}

\textbf{General Comment:} Be sure to synthetically divide by the zero of the denominator! Also, make sure to include 0 placeholders for missing terms.
}
\end{enumerate}

\end{document}