\documentclass{extbook}[14pt]
\usepackage{multicol, enumerate, enumitem, hyperref, color, soul, setspace, parskip, fancyhdr, amssymb, amsthm, amsmath, bbm, latexsym, units, mathtools}
\everymath{\displaystyle}
\usepackage[headsep=0.5cm,headheight=0cm, left=1 in,right= 1 in,top= 1 in,bottom= 1 in]{geometry}
\usepackage{dashrule}  % Package to use the command below to create lines between items
\newcommand{\litem}[1]{\item #1

\rule{\textwidth}{0.4pt}}
\pagestyle{fancy}
\lhead{}
\chead{Answer Key for Progress Quiz 8 Version C}
\rhead{}
\lfoot{4553-3922}
\cfoot{}
\rfoot{Fall 2020}
\begin{document}
\textbf{This key should allow you to understand why you choose the option you did (beyond just getting a question right or wrong). \href{https://xronos.clas.ufl.edu/mac1105spring2020/courseDescriptionAndMisc/Exams/LearningFromResults}{More instructions on how to use this key can be found here}.}

\textbf{If you have a suggestion to make the keys better, \href{https://forms.gle/CZkbZmPbC9XALEE88}{please fill out the short survey here}.}

\textit{Note: This key is auto-generated and may contain issues and/or errors. The keys are reviewed after each exam to ensure grading is done accurately. If there are issues (like duplicate options), they are noted in the offline gradebook. The keys are a work-in-progress to give students as many resources to improve as possible.}

\rule{\textwidth}{0.4pt}

\begin{enumerate}\litem{
Perform the division below. Then, find the intervals that correspond to the quotient in the form $ax^2+bx+c$ and remainder $r$.
\[ \frac{10x^{3} +5 x^{2} -80 x -79}{x -3} \]

The solution is \( 10x^{2} +35 x + 25 + \frac{-4}{x -3} \), which is option B.\begin{enumerate}[label=\Alph*.]
\item \( a \in [8, 13], \text{   } b \in [-27, -24], \text{   } c \in [-7, -3], \text{   and   } r \in [-67, -59]. \)

 You divided by the opposite of the factor.
\item \( a \in [8, 13], \text{   } b \in [33, 37], \text{   } c \in [18, 32], \text{   and   } r \in [-7, -3]. \)

* This is the solution!
\item \( a \in [26, 34], \text{   } b \in [-87, -79], \text{   } c \in [174, 178], \text{   and   } r \in [-608, -597]. \)

 You divided by the opposite of the factor AND multiplied the first factor rather than just bringing it down.
\item \( a \in [26, 34], \text{   } b \in [91, 99], \text{   } c \in [196, 207], \text{   and   } r \in [534, 543]. \)

 You multiplied by the synthetic number rather than bringing the first factor down.
\item \( a \in [8, 13], \text{   } b \in [24, 27], \text{   } c \in [-31, -27], \text{   and   } r \in [-143, -134]. \)

 You multiplied by the synthetic number and subtracted rather than adding during synthetic division.
\end{enumerate}

\textbf{General Comment:} Be sure to synthetically divide by the zero of the denominator!
}
\litem{
What are the \textit{possible Integer} roots of the polynomial below?
\[ f(x) = 3x^{4} +2 x^{3} +3 x^{2} +7 x + 5 \]

The solution is \( \pm 1,\pm 5 \), which is option C.\begin{enumerate}[label=\Alph*.]
\item \( \pm 1,\pm 3 \)

 Distractor 1: Corresponds to the plus or minus factors of a1 only.
\item \( \text{ All combinations of: }\frac{\pm 1,\pm 3}{\pm 1,\pm 5} \)

 Distractor 3: Corresponds to the plus or minus of the inverse quotient (an/a0) of the factors. 
\item \( \pm 1,\pm 5 \)

* This is the solution \textbf{since we asked for the possible Integer roots}!
\item \( \text{ All combinations of: }\frac{\pm 1,\pm 5}{\pm 1,\pm 3} \)

This would have been the solution \textbf{if asked for the possible Rational roots}!
\item \( \text{There is no formula or theorem that tells us all possible Integer roots.} \)

 Distractor 4: Corresponds to not recognizing Integers as a subset of Rationals.
\end{enumerate}

\textbf{General Comment:} We have a way to find the possible Rational roots. The possible Integer roots are the Integers in this list.
}
\litem{
Perform the division below. Then, find the intervals that correspond to the quotient in the form $ax^2+bx+c$ and remainder $r$.
\[ \frac{16x^{3} -52 x^{2} + 33}{x -3} \]

The solution is \( 16x^{2} -4 x -12 + \frac{-3}{x -3} \), which is option D.\begin{enumerate}[label=\Alph*.]
\item \( a \in [12, 25], b \in [-20, -18], c \in [-41, -36], \text{ and } r \in [-52, -45]. \)

 You multipled by the synthetic number and subtracted rather than adding during synthetic division.
\item \( a \in [47, 51], b \in [-199, -188], c \in [586, 591], \text{ and } r \in [-1736, -1727]. \)

 You divided by the opposite of the factor AND multipled the first factor rather than just bringing it down.
\item \( a \in [47, 51], b \in [85, 98], c \in [275, 280], \text{ and } r \in [852, 865]. \)

 You multipled by the synthetic number rather than bringing the first factor down.
\item \( a \in [12, 25], b \in [-9, -2], c \in [-14, -9], \text{ and } r \in [-6, -1]. \)

* This is the solution!
\item \( a \in [12, 25], b \in [-102, -93], c \in [296, 308], \text{ and } r \in [-868, -866]. \)

 You divided by the opposite of the factor.
\end{enumerate}

\textbf{General Comment:} Be sure to synthetically divide by the zero of the denominator! Also, make sure to include 0 placeholders for missing terms.
}
\litem{
What are the \textit{possible Integer} roots of the polynomial below?
\[ f(x) = 3x^{2} +6 x + 2 \]

The solution is \( \pm 1,\pm 2 \), which is option D.\begin{enumerate}[label=\Alph*.]
\item \( \text{ All combinations of: }\frac{\pm 1,\pm 2}{\pm 1,\pm 3} \)

This would have been the solution \textbf{if asked for the possible Rational roots}!
\item \( \text{ All combinations of: }\frac{\pm 1,\pm 3}{\pm 1,\pm 2} \)

 Distractor 3: Corresponds to the plus or minus of the inverse quotient (an/a0) of the factors. 
\item \( \pm 1,\pm 3 \)

 Distractor 1: Corresponds to the plus or minus factors of a1 only.
\item \( \pm 1,\pm 2 \)

* This is the solution \textbf{since we asked for the possible Integer roots}!
\item \( \text{There is no formula or theorem that tells us all possible Integer roots.} \)

 Distractor 4: Corresponds to not recognizing Integers as a subset of Rationals.
\end{enumerate}

\textbf{General Comment:} We have a way to find the possible Rational roots. The possible Integer roots are the Integers in this list.
}
\litem{
Perform the division below. Then, find the intervals that correspond to the quotient in the form $ax^2+bx+c$ and remainder $r$.
\[ \frac{8x^{3} +4 x^{2} -28 x -27}{x -2} \]

The solution is \( 8x^{2} +20 x + 12 + \frac{-3}{x -2} \), which is option D.\begin{enumerate}[label=\Alph*.]
\item \( a \in [3, 15], \text{   } b \in [-15, -10], \text{   } c \in [-4, 1], \text{   and   } r \in [-21, -18]. \)

 You divided by the opposite of the factor.
\item \( a \in [3, 15], \text{   } b \in [8, 13], \text{   } c \in [-16, -15], \text{   and   } r \in [-51, -38]. \)

 You multiplied by the synthetic number and subtracted rather than adding during synthetic division.
\item \( a \in [13, 20], \text{   } b \in [36, 39], \text{   } c \in [38, 48], \text{   and   } r \in [60, 67]. \)

 You multiplied by the synthetic number rather than bringing the first factor down.
\item \( a \in [3, 15], \text{   } b \in [17, 21], \text{   } c \in [9, 18], \text{   and   } r \in [-6, -2]. \)

* This is the solution!
\item \( a \in [13, 20], \text{   } b \in [-31, -24], \text{   } c \in [25, 36], \text{   and   } r \in [-86, -81]. \)

 You divided by the opposite of the factor AND multiplied the first factor rather than just bringing it down.
\end{enumerate}

\textbf{General Comment:} Be sure to synthetically divide by the zero of the denominator!
}
\litem{
Perform the division below. Then, find the intervals that correspond to the quotient in the form $ax^2+bx+c$ and remainder $r$.
\[ \frac{8x^{3} -26 x^{2} + 13}{x -3} \]

The solution is \( 8x^{2} -2 x -6 + \frac{-5}{x -3} \), which is option E.\begin{enumerate}[label=\Alph*.]
\item \( a \in [22, 30], b \in [45, 50], c \in [134, 140], \text{ and } r \in [425, 428]. \)

 You multipled by the synthetic number rather than bringing the first factor down.
\item \( a \in [22, 30], b \in [-100, -94], c \in [293, 299], \text{ and } r \in [-870, -867]. \)

 You divided by the opposite of the factor AND multipled the first factor rather than just bringing it down.
\item \( a \in [2, 11], b \in [-12, -5], c \in [-25, -15], \text{ and } r \in [-32, -26]. \)

 You multipled by the synthetic number and subtracted rather than adding during synthetic division.
\item \( a \in [2, 11], b \in [-52, -45], c \in [147, 152], \text{ and } r \in [-442, -429]. \)

 You divided by the opposite of the factor.
\item \( a \in [2, 11], b \in [-7, 6], c \in [-6, -4], \text{ and } r \in [-8, 4]. \)

* This is the solution!
\end{enumerate}

\textbf{General Comment:} Be sure to synthetically divide by the zero of the denominator! Also, make sure to include 0 placeholders for missing terms.
}
\litem{
Factor the polynomial below completely. Then, choose the intervals the zeros of the polynomial belong to, where $z_1 \leq z_2 \leq z_3$. \textit{To make the problem easier, all zeros are between -5 and 5.}
\[ f(x) = 10x^{3} +49 x^{2} +68 x + 20 \]

The solution is \( [-2.5, -2, -0.4] \), which is option C.\begin{enumerate}[label=\Alph*.]
\item \( z_1 \in [-2.62, -2.31], \text{   }  z_2 \in [-3, -1], \text{   and   } z_3 \in [-0.4, 1.6] \)

 Distractor 2: Corresponds to inversing rational roots.
\item \( z_1 \in [0.36, 0.41], \text{   }  z_2 \in [2, 3], \text{   and   } z_3 \in [2.5, 4.5] \)

 Distractor 1: Corresponds to negatives of all zeros.
\item \( z_1 \in [-2.62, -2.31], \text{   }  z_2 \in [-3, -1], \text{   and   } z_3 \in [-0.4, 1.6] \)

* This is the solution!
\item \( z_1 \in [0.36, 0.41], \text{   }  z_2 \in [2, 3], \text{   and   } z_3 \in [2.5, 4.5] \)

 Distractor 3: Corresponds to negatives of all zeros AND inversing rational roots.
\item \( z_1 \in [0.18, 0.35], \text{   }  z_2 \in [2, 3], \text{   and   } z_3 \in [5, 8] \)

 Distractor 4: Corresponds to moving factors from one rational to another.
\end{enumerate}

\textbf{General Comment:} Remember to try the middle-most integers first as these normally are the zeros. Also, once you get it to a quadratic, you can use your other factoring techniques to finish factoring.
}
\litem{
Factor the polynomial below completely. Then, choose the intervals the zeros of the polynomial belong to, where $z_1 \leq z_2 \leq z_3$. \textit{To make the problem easier, all zeros are between -5 and 5.}
\[ f(x) = 25x^{3} -50 x^{2} -9 x + 18 \]

The solution is \( [-0.6, 0.6, 2] \), which is option C.\begin{enumerate}[label=\Alph*.]
\item \( z_1 \in [-1.9, -1.2], \text{   }  z_2 \in [1.33, 2.15], \text{   and   } z_3 \in [1.9, 2.26] \)

 Distractor 2: Corresponds to inversing rational roots.
\item \( z_1 \in [-2.3, -1.8], \text{   }  z_2 \in [-0.74, -0.57], \text{   and   } z_3 \in [0.47, 0.92] \)

 Distractor 1: Corresponds to negatives of all zeros.
\item \( z_1 \in [-1.1, 0.4], \text{   }  z_2 \in [0.5, 0.94], \text{   and   } z_3 \in [1.9, 2.26] \)

* This is the solution!
\item \( z_1 \in [-2.3, -1.8], \text{   }  z_2 \in [-0.25, 0.02], \text{   and   } z_3 \in [2.6, 3.82] \)

 Distractor 4: Corresponds to moving factors from one rational to another.
\item \( z_1 \in [-2.3, -1.8], \text{   }  z_2 \in [-1.74, -1.03], \text{   and   } z_3 \in [1.55, 1.94] \)

 Distractor 3: Corresponds to negatives of all zeros AND inversing rational roots.
\end{enumerate}

\textbf{General Comment:} Remember to try the middle-most integers first as these normally are the zeros. Also, once you get it to a quadratic, you can use your other factoring techniques to finish factoring.
}
\litem{
Factor the polynomial below completely, knowing that $x-2$ is a factor. Then, choose the intervals the zeros of the polynomial belong to, where $z_1 \leq z_2 \leq z_3 \leq z_4$. \textit{To make the problem easier, all zeros are between -5 and 5.}
\[ f(x) = 8x^{4} +18 x^{3} -75 x^{2} -46 x + 120 \]

The solution is \( [-4, -1.5, 1.25, 2] \), which is option B.\begin{enumerate}[label=\Alph*.]
\item \( z_1 \in [-3, 2], \text{   }  z_2 \in [-1.26, -1.24], z_3 \in [1.34, 1.54], \text{   and   } z_4 \in [3.3, 5.1] \)

 Distractor 1: Corresponds to negatives of all zeros.
\item \( z_1 \in [-4, -3], \text{   }  z_2 \in [-1.5, -1.48], z_3 \in [1.16, 1.4], \text{   and   } z_4 \in [1.1, 3.9] \)

* This is the solution!
\item \( z_1 \in [-3, 2], \text{   }  z_2 \in [-0.95, -0.78], z_3 \in [0.62, 0.76], \text{   and   } z_4 \in [3.3, 5.1] \)

 Distractor 3: Corresponds to negatives of all zeros AND inversing rational roots.
\item \( z_1 \in [-3, 2], \text{   }  z_2 \in [-0.63, -0.57], z_3 \in [2.75, 3.25], \text{   and   } z_4 \in [3.3, 5.1] \)

 Distractor 4: Corresponds to moving factors from one rational to another.
\item \( z_1 \in [-4, -3], \text{   }  z_2 \in [-0.74, -0.65], z_3 \in [0.68, 0.81], \text{   and   } z_4 \in [1.1, 3.9] \)

 Distractor 2: Corresponds to inversing rational roots.
\end{enumerate}

\textbf{General Comment:} Remember to try the middle-most integers first as these normally are the zeros. Also, once you get it to a quadratic, you can use your other factoring techniques to finish factoring.
}
\litem{
Factor the polynomial below completely, knowing that $x-2$ is a factor. Then, choose the intervals the zeros of the polynomial belong to, where $z_1 \leq z_2 \leq z_3 \leq z_4$. \textit{To make the problem easier, all zeros are between -5 and 5.}
\[ f(x) = 10x^{4} -71 x^{3} +174 x^{2} -171 x + 54 \]

The solution is \( [0.6, 1.5, 2, 3] \), which is option B.\begin{enumerate}[label=\Alph*.]
\item \( z_1 \in [0.65, 0.69], \text{   }  z_2 \in [1.66, 1.97], z_3 \in [1.59, 2.31], \text{   and   } z_4 \in [2.92, 3.18] \)

 Distractor 2: Corresponds to inversing rational roots.
\item \( z_1 \in [0.53, 0.63], \text{   }  z_2 \in [1.16, 1.61], z_3 \in [1.59, 2.31], \text{   and   } z_4 \in [2.92, 3.18] \)

* This is the solution!
\item \( z_1 \in [-3.11, -3], \text{   }  z_2 \in [-2, -1.87], z_3 \in [-1.86, -1.65], \text{   and   } z_4 \in [-0.71, -0.66] \)

 Distractor 3: Corresponds to negatives of all zeros AND inversing rational roots.
\item \( z_1 \in [-3.11, -3], \text{   }  z_2 \in [-2, -1.87], z_3 \in [-1.54, -1.11], \text{   and   } z_4 \in [-0.6, -0.49] \)

 Distractor 1: Corresponds to negatives of all zeros.
\item \( z_1 \in [-3.11, -3], \text{   }  z_2 \in [-3.1, -2.85], z_3 \in [-2.21, -1.79], \text{   and   } z_4 \in [-0.32, -0.21] \)

 Distractor 4: Corresponds to moving factors from one rational to another.
\end{enumerate}

\textbf{General Comment:} Remember to try the middle-most integers first as these normally are the zeros. Also, once you get it to a quadratic, you can use your other factoring techniques to finish factoring.
}
\end{enumerate}

\end{document}