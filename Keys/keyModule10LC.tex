\documentclass{extbook}[14pt]
\usepackage{multicol, enumerate, enumitem, hyperref, color, soul, setspace, parskip, fancyhdr, amssymb, amsthm, amsmath, bbm, latexsym, units, mathtools}
\everymath{\displaystyle}
\usepackage[headsep=0.5cm,headheight=0cm, left=1 in,right= 1 in,top= 1 in,bottom= 1 in]{geometry}
\usepackage{dashrule}  % Package to use the command below to create lines between items
\newcommand{\litem}[1]{\item #1

\rule{\textwidth}{0.4pt}}
\pagestyle{fancy}
\lhead{}
\chead{Answer Key for Progress Quiz 9 Version C}
\rhead{}
\lfoot{8590-6105}
\cfoot{}
\rfoot{Fall 2020}
\begin{document}
\textbf{This key should allow you to understand why you choose the option you did (beyond just getting a question right or wrong). \href{https://xronos.clas.ufl.edu/mac1105spring2020/courseDescriptionAndMisc/Exams/LearningFromResults}{More instructions on how to use this key can be found here}.}

\textbf{If you have a suggestion to make the keys better, \href{https://forms.gle/CZkbZmPbC9XALEE88}{please fill out the short survey here}.}

\textit{Note: This key is auto-generated and may contain issues and/or errors. The keys are reviewed after each exam to ensure grading is done accurately. If there are issues (like duplicate options), they are noted in the offline gradebook. The keys are a work-in-progress to give students as many resources to improve as possible.}

\rule{\textwidth}{0.4pt}

\begin{enumerate}\litem{
What are the \textit{possible Integer} roots of the polynomial below?
\[ f(x) = 3x^{2} +4 x + 4 \]

The solution is \( \pm 1,\pm 2,\pm 4 \), which is option A.\begin{enumerate}[label=\Alph*.]
\item \( \pm 1,\pm 2,\pm 4 \)

* This is the solution \textbf{since we asked for the possible Integer roots}!
\item \( \text{ All combinations of: }\frac{\pm 1,\pm 2,\pm 4}{\pm 1,\pm 3} \)

This would have been the solution \textbf{if asked for the possible Rational roots}!
\item \( \text{ All combinations of: }\frac{\pm 1,\pm 3}{\pm 1,\pm 2,\pm 4} \)

 Distractor 3: Corresponds to the plus or minus of the inverse quotient (an/a0) of the factors. 
\item \( \pm 1,\pm 3 \)

 Distractor 1: Corresponds to the plus or minus factors of a1 only.
\item \( \text{There is no formula or theorem that tells us all possible Integer roots.} \)

 Distractor 4: Corresponds to not recognizing Integers as a subset of Rationals.
\end{enumerate}

\textbf{General Comment:} We have a way to find the possible Rational roots. The possible Integer roots are the Integers in this list.
}
\litem{
Perform the division below. Then, find the intervals that correspond to the quotient in the form $ax^2+bx+c$ and remainder $r$.
\[ \frac{20x^{3} -111 x^{2} +136 x -46}{x -4} \]

The solution is \( 20x^{2} -31 x + 12 + \frac{2}{x -4} \), which is option A.\begin{enumerate}[label=\Alph*.]
\item \( a \in [15, 22], \text{   } b \in [-32, -22], \text{   } c \in [12, 13], \text{   and   } r \in [-3, 7]. \)

* This is the solution!
\item \( a \in [15, 22], \text{   } b \in [-52, -50], \text{   } c \in [-21, -16], \text{   and   } r \in [-99, -95]. \)

 You multiplied by the synthetic number and subtracted rather than adding during synthetic division.
\item \( a \in [78, 81], \text{   } b \in [207, 212], \text{   } c \in [968, 973], \text{   and   } r \in [3840, 3847]. \)

 You multiplied by the synthetic number rather than bringing the first factor down.
\item \( a \in [78, 81], \text{   } b \in [-435, -429], \text{   } c \in [1860, 1866], \text{   and   } r \in [-7488, -7478]. \)

 You divided by the opposite of the factor AND multiplied the first factor rather than just bringing it down.
\item \( a \in [15, 22], \text{   } b \in [-198, -186], \text{   } c \in [897, 904], \text{   and   } r \in [-3646, -3644]. \)

 You divided by the opposite of the factor.
\end{enumerate}

\textbf{General Comment:} Be sure to synthetically divide by the zero of the denominator!
}
\litem{
Factor the polynomial below completely. Then, choose the intervals the zeros of the polynomial belong to, where $z_1 \leq z_2 \leq z_3$. \textit{To make the problem easier, all zeros are between -5 and 5.}
\[ f(x) = 15x^{3} +74 x^{2} -11 x -30 \]

The solution is \( [-5, -0.6, 0.6666666666666666] \), which is option D.\begin{enumerate}[label=\Alph*.]
\item \( z_1 \in [-1.25, -0.62], \text{   }  z_2 \in [0.35, 0.72], \text{   and   } z_3 \in [4.3, 5.7] \)

 Distractor 1: Corresponds to negatives of all zeros.
\item \( z_1 \in [-1.58, -1.33], \text{   }  z_2 \in [1.07, 1.73], \text{   and   } z_3 \in [4.3, 5.7] \)

 Distractor 3: Corresponds to negatives of all zeros AND inversing rational roots.
\item \( z_1 \in [-5.29, -4.92], \text{   }  z_2 \in [-1.79, -1.65], \text{   and   } z_3 \in [1.4, 1.8] \)

 Distractor 2: Corresponds to inversing rational roots.
\item \( z_1 \in [-5.29, -4.92], \text{   }  z_2 \in [-0.96, 0.05], \text{   and   } z_3 \in [0.6, 1] \)

* This is the solution!
\item \( z_1 \in [-2.41, -1.81], \text{   }  z_2 \in [-0.16, 0.46], \text{   and   } z_3 \in [4.3, 5.7] \)

 Distractor 4: Corresponds to moving factors from one rational to another.
\end{enumerate}

\textbf{General Comment:} Remember to try the middle-most integers first as these normally are the zeros. Also, once you get it to a quadratic, you can use your other factoring techniques to finish factoring.
}
\litem{
Factor the polynomial below completely, knowing that $x-4$ is a factor. Then, choose the intervals the zeros of the polynomial belong to, where $z_1 \leq z_2 \leq z_3 \leq z_4$. \textit{To make the problem easier, all zeros are between -5 and 5.}
\[ f(x) = 10x^{4} +31 x^{3} -189 x^{2} -430 x + 200 \]

The solution is \( [-5, -2.5, 0.4, 4] \), which is option D.\begin{enumerate}[label=\Alph*.]
\item \( z_1 \in [-5.53, -4.12], \text{   }  z_2 \in [-0.63, 0.47], z_3 \in [2.43, 2.57], \text{   and   } z_4 \in [3.19, 4.43] \)

 Distractor 2: Corresponds to inversing rational roots.
\item \( z_1 \in [-4.68, -3.44], \text{   }  z_2 \in [-0.63, 0.47], z_3 \in [2.43, 2.57], \text{   and   } z_4 \in [4.32, 5.1] \)

 Distractor 1: Corresponds to negatives of all zeros.
\item \( z_1 \in [-4.68, -3.44], \text{   }  z_2 \in [-2.7, -2.28], z_3 \in [0.38, 0.44], \text{   and   } z_4 \in [4.32, 5.1] \)

 Distractor 3: Corresponds to negatives of all zeros AND inversing rational roots.
\item \( z_1 \in [-5.53, -4.12], \text{   }  z_2 \in [-2.7, -2.28], z_3 \in [0.38, 0.44], \text{   and   } z_4 \in [3.19, 4.43] \)

* This is the solution!
\item \( z_1 \in [-4.68, -3.44], \text{   }  z_2 \in [-2.27, -1.69], z_3 \in [0.43, 0.57], \text{   and   } z_4 \in [4.32, 5.1] \)

 Distractor 4: Corresponds to moving factors from one rational to another.
\end{enumerate}

\textbf{General Comment:} Remember to try the middle-most integers first as these normally are the zeros. Also, once you get it to a quadratic, you can use your other factoring techniques to finish factoring.
}
\litem{
Perform the division below. Then, find the intervals that correspond to the quotient in the form $ax^2+bx+c$ and remainder $r$.
\[ \frac{12x^{3} +39 x^{2} -30}{x + 3} \]

The solution is \( 12x^{2} +3 x -9 + \frac{-3}{x + 3} \), which is option E.\begin{enumerate}[label=\Alph*.]
\item \( a \in [9, 17], b \in [71, 78], c \in [222, 226], \text{ and } r \in [644, 648]. \)

 You divided by the opposite of the factor.
\item \( a \in [9, 17], b \in [-12, -7], c \in [36, 42], \text{ and } r \in [-181, -170]. \)

 You multipled by the synthetic number and subtracted rather than adding during synthetic division.
\item \( a \in [-39, -30], b \in [145, 149], c \in [-445, -440], \text{ and } r \in [1291, 1299]. \)

 You multipled by the synthetic number rather than bringing the first factor down.
\item \( a \in [-39, -30], b \in [-71, -65], c \in [-208, -200], \text{ and } r \in [-652, -650]. \)

 You divided by the opposite of the factor AND multipled the first factor rather than just bringing it down.
\item \( a \in [9, 17], b \in [-4, 4], c \in [-13, -2], \text{ and } r \in [-5, -2]. \)

* This is the solution!
\end{enumerate}

\textbf{General Comment:} Be sure to synthetically divide by the zero of the denominator! Also, make sure to include 0 placeholders for missing terms.
}
\litem{
Factor the polynomial below completely. Then, choose the intervals the zeros of the polynomial belong to, where $z_1 \leq z_2 \leq z_3$. \textit{To make the problem easier, all zeros are between -5 and 5.}
\[ f(x) = 20x^{3} -63 x^{2} -78 x + 40 \]

The solution is \( [-1.25, 0.4, 4] \), which is option A.\begin{enumerate}[label=\Alph*.]
\item \( z_1 \in [-1.8, -1.17], \text{   }  z_2 \in [0.21, 0.57], \text{   and   } z_3 \in [3.67, 4.79] \)

* This is the solution!
\item \( z_1 \in [-4.14, -3.54], \text{   }  z_2 \in [-0.19, -0.04], \text{   and   } z_3 \in [4.3, 5.34] \)

 Distractor 4: Corresponds to moving factors from one rational to another.
\item \( z_1 \in [-4.14, -3.54], \text{   }  z_2 \in [-2.73, -2.44], \text{   and   } z_3 \in [0.54, 0.84] \)

 Distractor 3: Corresponds to negatives of all zeros AND inversing rational roots.
\item \( z_1 \in [-0.91, -0.54], \text{   }  z_2 \in [2.4, 2.52], \text{   and   } z_3 \in [3.67, 4.79] \)

 Distractor 2: Corresponds to inversing rational roots.
\item \( z_1 \in [-4.14, -3.54], \text{   }  z_2 \in [-0.45, -0.35], \text{   and   } z_3 \in [1.04, 1.87] \)

 Distractor 1: Corresponds to negatives of all zeros.
\end{enumerate}

\textbf{General Comment:} Remember to try the middle-most integers first as these normally are the zeros. Also, once you get it to a quadratic, you can use your other factoring techniques to finish factoring.
}
\litem{
Perform the division below. Then, find the intervals that correspond to the quotient in the form $ax^2+bx+c$ and remainder $r$.
\[ \frac{10x^{3} -28 x^{2} -14 x + 19}{x -3} \]

The solution is \( 10x^{2} +2 x -8 + \frac{-5}{x -3} \), which is option B.\begin{enumerate}[label=\Alph*.]
\item \( a \in [29, 37], \text{   } b \in [-118, -112], \text{   } c \in [338, 341], \text{   and   } r \in [-1002, -999]. \)

 You divided by the opposite of the factor AND multiplied the first factor rather than just bringing it down.
\item \( a \in [10, 12], \text{   } b \in [-3, 9], \text{   } c \in [-8, -3], \text{   and   } r \in [-6, -4]. \)

* This is the solution!
\item \( a \in [10, 12], \text{   } b \in [-62, -52], \text{   } c \in [155, 164], \text{   and   } r \in [-462, -460]. \)

 You divided by the opposite of the factor.
\item \( a \in [29, 37], \text{   } b \in [61, 70], \text{   } c \in [169, 177], \text{   and   } r \in [531, 539]. \)

 You multiplied by the synthetic number rather than bringing the first factor down.
\item \( a \in [10, 12], \text{   } b \in [-14, -6], \text{   } c \in [-30, -20], \text{   and   } r \in [-46, -36]. \)

 You multiplied by the synthetic number and subtracted rather than adding during synthetic division.
\end{enumerate}

\textbf{General Comment:} Be sure to synthetically divide by the zero of the denominator!
}
\litem{
What are the \textit{possible Integer} roots of the polynomial below?
\[ f(x) = 2x^{2} +2 x + 3 \]

The solution is \( \pm 1,\pm 3 \), which is option B.\begin{enumerate}[label=\Alph*.]
\item \( \pm 1,\pm 2 \)

 Distractor 1: Corresponds to the plus or minus factors of a1 only.
\item \( \pm 1,\pm 3 \)

* This is the solution \textbf{since we asked for the possible Integer roots}!
\item \( \text{ All combinations of: }\frac{\pm 1,\pm 3}{\pm 1,\pm 2} \)

This would have been the solution \textbf{if asked for the possible Rational roots}!
\item \( \text{ All combinations of: }\frac{\pm 1,\pm 2}{\pm 1,\pm 3} \)

 Distractor 3: Corresponds to the plus or minus of the inverse quotient (an/a0) of the factors. 
\item \( \text{There is no formula or theorem that tells us all possible Integer roots.} \)

 Distractor 4: Corresponds to not recognizing Integers as a subset of Rationals.
\end{enumerate}

\textbf{General Comment:} We have a way to find the possible Rational roots. The possible Integer roots are the Integers in this list.
}
\litem{
Perform the division below. Then, find the intervals that correspond to the quotient in the form $ax^2+bx+c$ and remainder $r$.
\[ \frac{25x^{3} +105 x^{2} -82}{x + 4} \]

The solution is \( 25x^{2} +5 x -20 + \frac{-2}{x + 4} \), which is option A.\begin{enumerate}[label=\Alph*.]
\item \( a \in [16, 31], b \in [5, 8], c \in [-22, -14], \text{ and } r \in [-2, 3]. \)

* This is the solution!
\item \( a \in [16, 31], b \in [203, 207], c \in [816, 821], \text{ and } r \in [3198, 3200]. \)

 You divided by the opposite of the factor.
\item \( a \in [-106, -96], b \in [-299, -289], c \in [-1184, -1174], \text{ and } r \in [-4805, -4799]. \)

 You divided by the opposite of the factor AND multipled the first factor rather than just bringing it down.
\item \( a \in [-106, -96], b \in [503, 513], c \in [-2021, -2019], \text{ and } r \in [7998, 7999]. \)

 You multipled by the synthetic number rather than bringing the first factor down.
\item \( a \in [16, 31], b \in [-21, -19], c \in [97, 107], \text{ and } r \in [-582, -579]. \)

 You multipled by the synthetic number and subtracted rather than adding during synthetic division.
\end{enumerate}

\textbf{General Comment:} Be sure to synthetically divide by the zero of the denominator! Also, make sure to include 0 placeholders for missing terms.
}
\litem{
Factor the polynomial below completely, knowing that $x+3$ is a factor. Then, choose the intervals the zeros of the polynomial belong to, where $z_1 \leq z_2 \leq z_3 \leq z_4$. \textit{To make the problem easier, all zeros are between -5 and 5.}
\[ f(x) = 12x^{4} +95 x^{3} +152 x^{2} -175 x -300 \]

The solution is \( [-5, -3, -1.25, 1.3333333333333333] \), which is option E.\begin{enumerate}[label=\Alph*.]
\item \( z_1 \in [-5, -4.72], \text{   }  z_2 \in [-5.1, -1.9], z_3 \in [-0.98, -0.28], \text{   and   } z_4 \in [0.59, 0.79] \)

 Distractor 2: Corresponds to inversing rational roots.
\item \( z_1 \in [-0.94, -0.69], \text{   }  z_2 \in [0.6, 1.1], z_3 \in [2.84, 3.23], \text{   and   } z_4 \in [4.78, 5.41] \)

 Distractor 3: Corresponds to negatives of all zeros AND inversing rational roots.
\item \( z_1 \in [-0.54, -0.28], \text{   }  z_2 \in [2.3, 3.2], z_3 \in [4.71, 5.08], \text{   and   } z_4 \in [4.78, 5.41] \)

 Distractor 4: Corresponds to moving factors from one rational to another.
\item \( z_1 \in [-1.79, -1.21], \text{   }  z_2 \in [1, 2.4], z_3 \in [2.84, 3.23], \text{   and   } z_4 \in [4.78, 5.41] \)

 Distractor 1: Corresponds to negatives of all zeros.
\item \( z_1 \in [-5, -4.72], \text{   }  z_2 \in [-5.1, -1.9], z_3 \in [-2.16, -0.94], \text{   and   } z_4 \in [0.95, 1.49] \)

* This is the solution!
\end{enumerate}

\textbf{General Comment:} Remember to try the middle-most integers first as these normally are the zeros. Also, once you get it to a quadratic, you can use your other factoring techniques to finish factoring.
}
\end{enumerate}

\end{document}