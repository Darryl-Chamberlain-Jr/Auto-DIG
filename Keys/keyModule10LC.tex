\documentclass{extbook}[14pt]
\usepackage{multicol, enumerate, enumitem, hyperref, color, soul, setspace, parskip, fancyhdr, amssymb, amsthm, amsmath, bbm, latexsym, units, mathtools}
\everymath{\displaystyle}
\usepackage[headsep=0.5cm,headheight=0cm, left=1 in,right= 1 in,top= 1 in,bottom= 1 in]{geometry}
\pagestyle{fancy}
\lhead{}
\chead{Answer Key for Module\,10L\,-\,Synthetic\,Division Version C}
\rhead{}
\lfoot{Summer\,C\,2020}
\cfoot{}
\rfoot{}
\begin{document}
\textbf{This key should allow you to understand why you choose the option you did (beyond just getting a question right or wrong). \href{https://xronos.clas.ufl.edu/mac1105spring2020/courseDescriptionAndMisc/Exams/LearningFromResults}{More instructions on how to use this key can be found here}.}

\textbf{If you have a suggestion to make the keys better, \href{https://forms.gle/CZkbZmPbC9XALEE88}{please fill out the short survey here}.}

\textit{Note: This key is auto-generated and may contain issues and/or errors. The keys are reviewed after each exam to ensure grading is done accurately. If there are issues (like duplicate options), they are noted in the offline gradebook. The keys are a work-in-progress to give students as many resources to improve as possible.}

\rule{\textwidth}{0.4pt}

66. Factor the polynomial below completely, knowing that $x+3$ is a factor. Then, choose the intervals the zeros of the polynomial belong to, where $z_1 \leq z_2 \leq z_3 \leq z_4$. \textit{To make the problem easier, all zeros are between -5 and 5.}
\[ f(x) = 20x^{4} -11 x^{3} -269 x^{2} -88 x + 240 \] 
The solution is $ [-3, -1.25, 0.8, 4] $ 

\begin{enumerate}[label=\Alph*.] 
\item $ z_1 \in [-5, -3.7], \text{   }  z_2 \in [-1.13, -0.52], z_3 \in [0.84, 1.84], \text{   and   } z_4 \in [0.8, 3.8] $ 

  Distractor 1: Corresponds to negatives of all zeros. 
\item $ z_1 \in [-3.3, -2.1], \text{   }  z_2 \in [-1.13, -0.52], z_3 \in [0.84, 1.84], \text{   and   } z_4 \in [3.8, 4.7] $ 

  Distractor 2: Corresponds to inversing rational roots. 
\item $ z_1 \in [-3.3, -2.1], \text{   }  z_2 \in [-2.29, -0.99], z_3 \in [0.4, 1.19], \text{   and   } z_4 \in [3.8, 4.7] $ 

 * This is the solution! 
\item $ z_1 \in [-5, -3.7], \text{   }  z_2 \in [-2.29, -0.99], z_3 \in [0.4, 1.19], \text{   and   } z_4 \in [0.8, 3.8] $ 

  Distractor 3: Corresponds to negatives of all zeros AND inversing rational roots. 
\item $ z_1 \in [-5, -3.7], \text{   }  z_2 \in [-0.68, 0.52], z_3 \in [2.79, 4.39], \text{   and   } z_4 \in [4.8, 5.5] $ 

  Distractor 4: Corresponds to moving factors from one rational to another. 
\end{enumerate} 
 
General Comments: Remember to try the middle-most integers first as these normally are the zeros. Also, once you get it to a quadratic, you can use your other factoring techniques to finish factoring.

-----------------------------------------------

67. Factor the polynomial below completely. Then, choose the intervals the zeros of the polynomial belong to, where $z_1 \leq z_2 \leq z_3$. \textit{To make the problem easier, all zeros are between -5 and 5.}
\[ f(x) = 10x^{3} +21 x^{2} -91 x -60 \] 
The solution is $ [-4, -0.6, 2.5] $ 

\begin{enumerate}[label=\Alph*.] 
\item $ z_1 \in [-4.14, -3.89], \text{   }  z_2 \in [-1.85, -1.26], \text{   and   } z_3 \in [0, 1.1] $ 

  Distractor 2: Corresponds to inversing rational roots. 
\item $ z_1 \in [-0.55, -0.44], \text{   }  z_2 \in [2.7, 3.22], \text{   and   } z_3 \in [3.3, 4.6] $ 

  Distractor 4: Corresponds to moving factors from one rational to another. 
\item $ z_1 \in [-4.14, -3.89], \text{   }  z_2 \in [-0.78, -0.39], \text{   and   } z_3 \in [1.6, 3.2] $ 

 * This is the solution! 
\item $ z_1 \in [-0.42, -0.25], \text{   }  z_2 \in [1.64, 1.96], \text{   and   } z_3 \in [3.3, 4.6] $ 

  Distractor 3: Corresponds to negatives of all zeros AND inversing rational roots. 
\item $ z_1 \in [-2.64, -2.46], \text{   }  z_2 \in [0.58, 0.68], \text{   and   } z_3 \in [3.3, 4.6] $ 

  Distractor 1: Corresponds to negatives of all zeros. 
\end{enumerate} 
 
General Comments: Remember to try the middle-most integers first as these normally are the zeros. Also, once you get it to a quadratic, you can use your other factoring techniques to finish factoring.

-----------------------------------------------

68. What are the \textit{possible Rational} roots of the polynomial below?
\[ f(x) = 2x^{3} +4 x^{2} +2 x + 7 \] 
The solution is $ \text{ All combinations of: }\frac{\pm 1,\pm 7}{\pm 1,\pm 2} $ 

\begin{enumerate}[label=\Alph*.] 
\item $ \pm 1,\pm 7 $ 

 This would have been the solution \textbf{if asked for the possible Integer roots}! 
\item $ \text{ All combinations of: }\frac{\pm 1,\pm 7}{\pm 1,\pm 2} $ 

 * This is the solution \textbf{since we asked for the possible Rational roots}! 
\item $ \pm 1,\pm 2 $ 

  Distractor 1: Corresponds to the plus or minus factors of a1 only. 
\item $ \text{ All combinations of: }\frac{\pm 1,\pm 2}{\pm 1,\pm 7} $ 

  Distractor 3: Corresponds to the plus or minus of the inverse quotient (an/a0) of the factors.  
\item $ \text{ There is no formula or theorem that tells us all possible Rational roots.} $ 

  Distractor 4: Corresponds to not recalling the theorem for rational roots of a polynomial. 
\end{enumerate} 
 
General Comments: We have a way to find the possible Rational roots. The possible Integer roots are the Integers in this list.

-----------------------------------------------

69. Perform the division below. Then, find the intervals that correspond to the quotient in the form $ax^2+bx+c$ and remainder $r$.
\[ \frac{12x^{3} +28 x^{2} -19}{x + 2} \] 
The solution is $ 12x^{2} +4 x -8 + \frac{-3}{x + 2} $ 

\begin{enumerate}[label=\Alph*.] 
\item $ a \in [-27, -22], b \in [75, 83], c \in [-153, -149], \text{ and } r \in [283, 288]. $ 

  You multipled by the synthetic number rather than bringing the first factor down. 
\item $ a \in [5, 15], b \in [48, 54], c \in [103, 106], \text{ and } r \in [188, 191]. $ 

  You divided by the opposite of the factor. 
\item $ a \in [-27, -22], b \in [-21, -17], c \in [-43, -39], \text{ and } r \in [-101, -92]. $ 

  You divided by the opposite of the factor AND multipled the first factor rather than just bringing it down. 
\item $ a \in [5, 15], b \in [-17, -6], c \in [21, 25], \text{ and } r \in [-92, -89]. $ 

  You multipled by the synthetic number and subtracted rather than adding during synthetic division. 
\item $ a \in [5, 15], b \in [-1, 12], c \in [-13, -5], \text{ and } r \in [-5, 7]. $ 

 * This is the solution! 
\end{enumerate} 
 
General Comments: Be sure to synthetically divide by the zero of the denominator! Also, make sure to include 0 placeholders for missing terms.

-----------------------------------------------

70. Perform the division below. Then, find the intervals that correspond to the quotient in the form $ax^2+bx+c$ and remainder $r$.
\[ \frac{10x^{3} +33 x^{2} -105 x -102}{x + 5} \] 
The solution is $ 10x^{2} -17 x -20 + \frac{-2}{x + 5} $ 

\begin{enumerate}[label=\Alph*.] 
\item $ a \in [-51, -48], \text{   } b \in [279, 288], \text{   } c \in [-1521, -1518], \text{   and   } r \in [7495, 7506]. $ 

  You multiplied by the synthetic number rather than bringing the first factor down. 
\item $ a \in [8, 11], \text{   } b \in [-28, -23], \text{   } c \in [56, 63], \text{   and   } r \in [-445, -438]. $ 

  You multiplied by the synthetic number and subtracted rather than adding during synthetic division. 
\item $ a \in [-51, -48], \text{   } b \in [-221, -216], \text{   } c \in [-1196, -1184], \text{   and   } r \in [-6057, -6048]. $ 

  You divided by the opposite of the factor AND multiplied the first factor rather than just bringing it down. 
\item $ a \in [8, 11], \text{   } b \in [80, 92], \text{   } c \in [306, 313], \text{   and   } r \in [1445, 1449]. $ 

  You divided by the opposite of the factor. 
\item $ a \in [8, 11], \text{   } b \in [-25, -12], \text{   } c \in [-26, -11], \text{   and   } r \in [-5, 1]. $ 

 * This is the solution! 
\end{enumerate} 
 
General Comments: Be sure to synthetically divide by the zero of the denominator!

-----------------------------------------------


\end{document}

