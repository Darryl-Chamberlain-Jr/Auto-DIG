\documentclass{extbook}[14pt]
\usepackage{multicol, enumerate, enumitem, hyperref, color, soul, setspace, parskip, fancyhdr, amssymb, amsthm, amsmath, bbm, latexsym, units, mathtools}
\everymath{\displaystyle}
\usepackage[headsep=0.5cm,headheight=0cm, left=1 in,right= 1 in,top= 1 in,bottom= 1 in]{geometry}
\pagestyle{fancy}
\lhead{}
\chead{Answer Key for Module\,10L\,-\,Synthetic\,Division Version B}
\rhead{}
\lfoot{Summer\,C\,2020}
\cfoot{}
\rfoot{}
\begin{document}
\textbf{This key should allow you to understand why you choose the option you did (beyond just getting a question right or wrong). \href{https://xronos.clas.ufl.edu/mac1105spring2020/courseDescriptionAndMisc/Exams/LearningFromResults}{More instructions on how to use this key can be found here}.}

\textbf{If you have a suggestion to make the keys better, \href{https://forms.gle/CZkbZmPbC9XALEE88}{please fill out the short survey here}.}

\textit{Note: This key is auto-generated and may contain issues and/or errors. The keys are reviewed after each exam to ensure grading is done accurately. If there are issues (like duplicate options), they are noted in the offline gradebook. The keys are a work-in-progress to give students as many resources to improve as possible.}

\rule{\textwidth}{0.4pt}

1. Perform the division below. Then, find the intervals that correspond to the quotient in the form $ax^2+bx+c$ and remainder $r$.
\[ \frac{20x^{3} -42 x^{2} + 12}{x -2} \] 
The solution is $ 20x^{2} -2 x -4 + \frac{4}{x -2} $ 

\begin{enumerate}[label=\Alph*.] 
\item $ a \in [39, 47], b \in [-123, -118], c \in [242, 249], \text{ and } r \in [-477, -473]. $ 

  You divided by the opposite of the factor AND multipled the first factor rather than just bringing it down. 
\item $ a \in [19, 26], b \in [-23, -19], c \in [-23, -20], \text{ and } r \in [-12, -7]. $ 

  You multipled by the synthetic number and subtracted rather than adding during synthetic division. 
\item $ a \in [39, 47], b \in [33, 40], c \in [75, 79], \text{ and } r \in [157, 165]. $ 

  You multipled by the synthetic number rather than bringing the first factor down. 
\item $ a \in [19, 26], b \in [-87, -75], c \in [160, 170], \text{ and } r \in [-318, -311]. $ 

  You divided by the opposite of the factor. 
\item $ a \in [19, 26], b \in [-9, 4], c \in [-11, -2], \text{ and } r \in [1, 7]. $ 

 * This is the solution! 
\end{enumerate} 
 
\textbf{General Comment:} General Comments: Be sure to synthetically divide by the zero of the denominator! Also, make sure to include 0 placeholders for missing terms. 

-----------------------------------------------

2. Factor the polynomial below completely, knowing that $x-4$ is a factor. Then, choose the intervals the zeros of the polynomial belong to, where $z_1 \leq z_2 \leq z_3 \leq z_4$. \textit{To make the problem easier, all zeros are between -5 and 5.}
\[ f(x) = 8x^{4} -34 x^{3} -43 x^{2} +159 x + 180 \] 
The solution is $ [-1.5, -1.25, 3, 4] $ 

\begin{enumerate}[label=\Alph*.] 
\item $ z_1 \in [-4.4, -3.21], \text{   }  z_2 \in [-3.17, -2.23], z_3 \in [0.03, 0.59], \text{   and   } z_4 \in [4.45, 5.57] $ 

  Distractor 4: Corresponds to moving factors from one rational to another. 
\item $ z_1 \in [-4.4, -3.21], \text{   }  z_2 \in [-3.17, -2.23], z_3 \in [0.42, 1.09], \text{   and   } z_4 \in [0.39, 1.06] $ 

  Distractor 3: Corresponds to negatives of all zeros AND inversing rational roots. 
\item $ z_1 \in [-1.15, -0.69], \text{   }  z_2 \in [-0.72, 0.12], z_3 \in [2.92, 3.06], \text{   and   } z_4 \in [3.29, 4.26] $ 

  Distractor 2: Corresponds to inversing rational roots. 
\item $ z_1 \in [-2.15, -1.26], \text{   }  z_2 \in [-1.84, -0.97], z_3 \in [2.92, 3.06], \text{   and   } z_4 \in [3.29, 4.26] $ 

 * This is the solution! 
\item $ z_1 \in [-4.4, -3.21], \text{   }  z_2 \in [-3.17, -2.23], z_3 \in [0.94, 1.4], \text{   and   } z_4 \in [0.91, 2.01] $ 

  Distractor 1: Corresponds to negatives of all zeros. 
\end{enumerate} 
 
\textbf{General Comment:} General Comments: Remember to try the middle-most integers first as these normally are the zeros. Also, once you get it to a quadratic, you can use your other factoring techniques to finish factoring. 

-----------------------------------------------

3. Perform the division below. Then, find the intervals that correspond to the quotient in the form $ax^2+bx+c$ and remainder $r$.
\[ \frac{8x^{3} +16 x^{2} -110 x + 55}{x + 5} \] 
The solution is $ 8x^{2} -24 x + 10 + \frac{5}{x + 5} $ 

\begin{enumerate}[label=\Alph*.] 
\item $ a \in [-43, -35], \text{   } b \in [214, 222], \text{   } c \in [-1193, -1188], \text{   and   } r \in [6003, 6006]. $ 

  You multiplied by the synthetic number rather than bringing the first factor down. 
\item $ a \in [4, 9], \text{   } b \in [-35, -30], \text{   } c \in [79, 83], \text{   and   } r \in [-440, -436]. $ 

  You multiplied by the synthetic number and subtracted rather than adding during synthetic division. 
\item $ a \in [4, 9], \text{   } b \in [-27, -16], \text{   } c \in [7, 11], \text{   and   } r \in [2, 6]. $ 

 * This is the solution! 
\item $ a \in [-43, -35], \text{   } b \in [-189, -183], \text{   } c \in [-1035, -1027], \text{   and   } r \in [-5096, -5092]. $ 

  You divided by the opposite of the factor AND multiplied the first factor rather than just bringing it down. 
\item $ a \in [4, 9], \text{   } b \in [52, 60], \text{   } c \in [166, 175], \text{   and   } r \in [900, 908]. $ 

  You divided by the opposite of the factor. 
\end{enumerate} 
 
\textbf{General Comment:} General Comments: Be sure to synthetically divide by the zero of the denominator! 

-----------------------------------------------

4. What are the \textit{possible Rational} roots of the polynomial below?
\[ f(x) = 3x^{4} +6 x^{3} +7 x^{2} +2 x + 2 \] 
The solution is $ \text{ All combinations of: }\frac{\pm 1,\pm 2}{\pm 1,\pm 3} $ 

\begin{enumerate}[label=\Alph*.] 
\item $ \pm 1,\pm 2 $ 

 This would have been the solution \textbf{if asked for the possible Integer roots}! 
\item $ \text{ All combinations of: }\frac{\pm 1,\pm 2}{\pm 1,\pm 3} $ 

 * This is the solution \textbf{since we asked for the possible Rational roots}! 
\item $ \pm 1,\pm 3 $ 

  Distractor 1: Corresponds to the plus or minus factors of a1 only. 
\item $ \text{ All combinations of: }\frac{\pm 1,\pm 3}{\pm 1,\pm 2} $ 

  Distractor 3: Corresponds to the plus or minus of the inverse quotient (an/a0) of the factors.  
\item $ \text{ There is no formula or theorem that tells us all possible Rational roots.} $ 

  Distractor 4: Corresponds to not recalling the theorem for rational roots of a polynomial. 
\end{enumerate} 
 
\textbf{General Comment:} General Comments: We have a way to find the possible Rational roots. The possible Integer roots are the Integers in this list. 

-----------------------------------------------

0. Factor the polynomial below completely. Then, choose the intervals the zeros of the polynomial belong to, where $z_1 \leq z_2 \leq z_3$. \textit{To make the problem easier, all zeros are between -5 and 5.}
\[ f(x) = 12x^{3} -53 x^{2} +57 x -18 \] 
The solution is $ [0.6666666666666666, 0.75, 3] $ 

\begin{enumerate}[label=\Alph*.] 
\item $ z_1 \in [-3.6, -1.8], \text{   }  z_2 \in [-1.77, -1.24], \text{   and   } z_3 \in [-1.69, -1.02] $ 

  Distractor 3: Corresponds to negatives of all zeros AND inversing rational roots. 
\item $ z_1 \in [1.2, 2.7], \text{   }  z_2 \in [1.14, 2.28], \text{   and   } z_3 \in [2.54, 3.2] $ 

  Distractor 2: Corresponds to inversing rational roots. 
\item $ z_1 \in [-3.6, -1.8], \text{   }  z_2 \in [-1.2, -0.08], \text{   and   } z_3 \in [-0.85, -0.27] $ 

  Distractor 1: Corresponds to negatives of all zeros. 
\item $ z_1 \in [-3.6, -1.8], \text{   }  z_2 \in [-3.31, -2.54], \text{   and   } z_3 \in [-0.22, 0.29] $ 

  Distractor 4: Corresponds to moving factors from one rational to another. 
\item $ z_1 \in [0.5, 0.9], \text{   }  z_2 \in [0.42, 1.13], \text{   and   } z_3 \in [2.54, 3.2] $ 

 * This is the solution! 
\end{enumerate} 
 
\textbf{General Comment:} General Comments: Remember to try the middle-most integers first as these normally are the zeros. Also, once you get it to a quadratic, you can use your other factoring techniques to finish factoring. 

-----------------------------------------------


\end{document}

