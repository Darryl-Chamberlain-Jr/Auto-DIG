\documentclass{extbook}[14pt]
\usepackage{multicol, enumerate, enumitem, hyperref, color, soul, setspace, parskip, fancyhdr, amssymb, amsthm, amsmath, bbm, latexsym, units, mathtools}
\everymath{\displaystyle}
\usepackage[headsep=0.5cm,headheight=0cm, left=1 in,right= 1 in,top= 1 in,bottom= 1 in]{geometry}
\usepackage{dashrule}  % Package to use the command below to create lines between items
\newcommand{\litem}[1]{\item #1

\rule{\textwidth}{0.4pt}}
\pagestyle{fancy}
\lhead{}
\chead{Answer Key for Progress Quiz 8 Version B}
\rhead{}
\lfoot{4553-3922}
\cfoot{}
\rfoot{Fall 2020}
\begin{document}
\textbf{This key should allow you to understand why you choose the option you did (beyond just getting a question right or wrong). \href{https://xronos.clas.ufl.edu/mac1105spring2020/courseDescriptionAndMisc/Exams/LearningFromResults}{More instructions on how to use this key can be found here}.}

\textbf{If you have a suggestion to make the keys better, \href{https://forms.gle/CZkbZmPbC9XALEE88}{please fill out the short survey here}.}

\textit{Note: This key is auto-generated and may contain issues and/or errors. The keys are reviewed after each exam to ensure grading is done accurately. If there are issues (like duplicate options), they are noted in the offline gradebook. The keys are a work-in-progress to give students as many resources to improve as possible.}

\rule{\textwidth}{0.4pt}

\begin{enumerate}\litem{
Perform the division below. Then, find the intervals that correspond to the quotient in the form $ax^2+bx+c$ and remainder $r$.
\[ \frac{25x^{3} -85 x^{2} +82 x -20}{x -2} \]

The solution is \( 25x^{2} -35 x + 12 + \frac{4}{x -2} \), which is option C.\begin{enumerate}[label=\Alph*.]
\item \( a \in [18, 29], \text{   } b \in [-141, -129], \text{   } c \in [348, 356], \text{   and   } r \in [-724.68, -722.52]. \)

 You divided by the opposite of the factor.
\item \( a \in [18, 29], \text{   } b \in [-60, -57], \text{   } c \in [16, 28], \text{   and   } r \in [1.08, 3.15]. \)

 You multiplied by the synthetic number and subtracted rather than adding during synthetic division.
\item \( a \in [18, 29], \text{   } b \in [-42, -33], \text{   } c \in [4, 14], \text{   and   } r \in [3.6, 5.08]. \)

* This is the solution!
\item \( a \in [48, 58], \text{   } b \in [-191, -182], \text{   } c \in [452, 458], \text{   and   } r \in [-925.19, -923.81]. \)

 You divided by the opposite of the factor AND multiplied the first factor rather than just bringing it down.
\item \( a \in [48, 58], \text{   } b \in [13, 20], \text{   } c \in [109, 116], \text{   and   } r \in [203.09, 204.16]. \)

 You multiplied by the synthetic number rather than bringing the first factor down.
\end{enumerate}

\textbf{General Comment:} Be sure to synthetically divide by the zero of the denominator!
}
\litem{
What are the \textit{possible Integer} roots of the polynomial below?
\[ f(x) = 2x^{4} +7 x^{3} +2 x^{2} +3 x + 7 \]

The solution is \( \pm 1,\pm 7 \), which is option A.\begin{enumerate}[label=\Alph*.]
\item \( \pm 1,\pm 7 \)

* This is the solution \textbf{since we asked for the possible Integer roots}!
\item \( \pm 1,\pm 2 \)

 Distractor 1: Corresponds to the plus or minus factors of a1 only.
\item \( \text{ All combinations of: }\frac{\pm 1,\pm 7}{\pm 1,\pm 2} \)

This would have been the solution \textbf{if asked for the possible Rational roots}!
\item \( \text{ All combinations of: }\frac{\pm 1,\pm 2}{\pm 1,\pm 7} \)

 Distractor 3: Corresponds to the plus or minus of the inverse quotient (an/a0) of the factors. 
\item \( \text{There is no formula or theorem that tells us all possible Integer roots.} \)

 Distractor 4: Corresponds to not recognizing Integers as a subset of Rationals.
\end{enumerate}

\textbf{General Comment:} We have a way to find the possible Rational roots. The possible Integer roots are the Integers in this list.
}
\litem{
Perform the division below. Then, find the intervals that correspond to the quotient in the form $ax^2+bx+c$ and remainder $r$.
\[ \frac{9x^{3} -28 x -14}{x -2} \]

The solution is \( 9x^{2} +18 x + 8 + \frac{2}{x -2} \), which is option C.\begin{enumerate}[label=\Alph*.]
\item \( a \in [14, 22], b \in [30, 40], c \in [42, 45], \text{ and } r \in [72, 77]. \)

 You multipled by the synthetic number rather than bringing the first factor down.
\item \( a \in [7, 13], b \in [8, 14], c \in [-19, -16], \text{ and } r \in [-36, -31]. \)

 You multipled by the synthetic number and subtracted rather than adding during synthetic division.
\item \( a \in [7, 13], b \in [10, 22], c \in [4, 15], \text{ and } r \in [0, 10]. \)

* This is the solution!
\item \( a \in [7, 13], b \in [-18, -11], c \in [4, 15], \text{ and } r \in [-32, -25]. \)

 You divided by the opposite of the factor.
\item \( a \in [14, 22], b \in [-38, -30], c \in [42, 45], \text{ and } r \in [-105, -97]. \)

 You divided by the opposite of the factor AND multipled the first factor rather than just bringing it down.
\end{enumerate}

\textbf{General Comment:} Be sure to synthetically divide by the zero of the denominator! Also, make sure to include 0 placeholders for missing terms.
}
\litem{
What are the \textit{possible Rational} roots of the polynomial below?
\[ f(x) = 7x^{4} +3 x^{3} +5 x^{2} +5 x + 2 \]

The solution is \( \text{ All combinations of: }\frac{\pm 1,\pm 2}{\pm 1,\pm 7} \), which is option B.\begin{enumerate}[label=\Alph*.]
\item \( \pm 1,\pm 2 \)

This would have been the solution \textbf{if asked for the possible Integer roots}!
\item \( \text{ All combinations of: }\frac{\pm 1,\pm 2}{\pm 1,\pm 7} \)

* This is the solution \textbf{since we asked for the possible Rational roots}!
\item \( \text{ All combinations of: }\frac{\pm 1,\pm 7}{\pm 1,\pm 2} \)

 Distractor 3: Corresponds to the plus or minus of the inverse quotient (an/a0) of the factors. 
\item \( \pm 1,\pm 7 \)

 Distractor 1: Corresponds to the plus or minus factors of a1 only.
\item \( \text{ There is no formula or theorem that tells us all possible Rational roots.} \)

 Distractor 4: Corresponds to not recalling the theorem for rational roots of a polynomial.
\end{enumerate}

\textbf{General Comment:} We have a way to find the possible Rational roots. The possible Integer roots are the Integers in this list.
}
\litem{
Perform the division below. Then, find the intervals that correspond to the quotient in the form $ax^2+bx+c$ and remainder $r$.
\[ \frac{10x^{3} +31 x^{2} -45 x -32}{x + 4} \]

The solution is \( 10x^{2} -9 x -9 + \frac{4}{x + 4} \), which is option A.\begin{enumerate}[label=\Alph*.]
\item \( a \in [5, 14], \text{   } b \in [-16, -7], \text{   } c \in [-13, -1], \text{   and   } r \in [-3, 7]. \)

* This is the solution!
\item \( a \in [-41, -36], \text{   } b \in [-131, -128], \text{   } c \in [-563, -559], \text{   and   } r \in [-2285, -2267]. \)

 You divided by the opposite of the factor AND multiplied the first factor rather than just bringing it down.
\item \( a \in [5, 14], \text{   } b \in [-26, -15], \text{   } c \in [45, 57], \text{   and   } r \in [-289, -275]. \)

 You multiplied by the synthetic number and subtracted rather than adding during synthetic division.
\item \( a \in [5, 14], \text{   } b \in [70, 72], \text{   } c \in [232, 244], \text{   and   } r \in [920, 926]. \)

 You divided by the opposite of the factor.
\item \( a \in [-41, -36], \text{   } b \in [187, 196], \text{   } c \in [-810, -807], \text{   and   } r \in [3203, 3209]. \)

 You multiplied by the synthetic number rather than bringing the first factor down.
\end{enumerate}

\textbf{General Comment:} Be sure to synthetically divide by the zero of the denominator!
}
\litem{
Perform the division below. Then, find the intervals that correspond to the quotient in the form $ax^2+bx+c$ and remainder $r$.
\[ \frac{8x^{3} -26 x^{2} + 15}{x -3} \]

The solution is \( 8x^{2} -2 x -6 + \frac{-3}{x -3} \), which is option C.\begin{enumerate}[label=\Alph*.]
\item \( a \in [5, 10], b \in [-52, -44], c \in [150, 153], \text{ and } r \in [-437, -434]. \)

 You divided by the opposite of the factor.
\item \( a \in [5, 10], b \in [-15, -5], c \in [-26, -16], \text{ and } r \in [-25, -22]. \)

 You multipled by the synthetic number and subtracted rather than adding during synthetic division.
\item \( a \in [5, 10], b \in [-7, -1], c \in [-8, -4], \text{ and } r \in [-9, -1]. \)

* This is the solution!
\item \( a \in [19, 27], b \in [-104, -92], c \in [293, 295], \text{ and } r \in [-868, -865]. \)

 You divided by the opposite of the factor AND multipled the first factor rather than just bringing it down.
\item \( a \in [19, 27], b \in [43, 49], c \in [138, 143], \text{ and } r \in [423, 432]. \)

 You multipled by the synthetic number rather than bringing the first factor down.
\end{enumerate}

\textbf{General Comment:} Be sure to synthetically divide by the zero of the denominator! Also, make sure to include 0 placeholders for missing terms.
}
\litem{
Factor the polynomial below completely. Then, choose the intervals the zeros of the polynomial belong to, where $z_1 \leq z_2 \leq z_3$. \textit{To make the problem easier, all zeros are between -5 and 5.}
\[ f(x) = 20x^{3} +29 x^{2} -81 x + 36 \]

The solution is \( [-3, 0.75, 0.8] \), which is option B.\begin{enumerate}[label=\Alph*.]
\item \( z_1 \in [-4.26, -3.91], \text{   }  z_2 \in [-0.21, 0.28], \text{   and   } z_3 \in [2.72, 3.15] \)

 Distractor 4: Corresponds to moving factors from one rational to another.
\item \( z_1 \in [-3.55, -2.84], \text{   }  z_2 \in [0.39, 0.8], \text{   and   } z_3 \in [0.52, 1.13] \)

* This is the solution!
\item \( z_1 \in [-1.41, -1.2], \text{   }  z_2 \in [-1.38, -1.15], \text{   and   } z_3 \in [2.72, 3.15] \)

 Distractor 3: Corresponds to negatives of all zeros AND inversing rational roots.
\item \( z_1 \in [-0.99, -0.41], \text{   }  z_2 \in [-0.99, -0.26], \text{   and   } z_3 \in [2.72, 3.15] \)

 Distractor 1: Corresponds to negatives of all zeros.
\item \( z_1 \in [-3.55, -2.84], \text{   }  z_2 \in [0.85, 1.65], \text{   and   } z_3 \in [1.17, 1.6] \)

 Distractor 2: Corresponds to inversing rational roots.
\end{enumerate}

\textbf{General Comment:} Remember to try the middle-most integers first as these normally are the zeros. Also, once you get it to a quadratic, you can use your other factoring techniques to finish factoring.
}
\litem{
Factor the polynomial below completely. Then, choose the intervals the zeros of the polynomial belong to, where $z_1 \leq z_2 \leq z_3$. \textit{To make the problem easier, all zeros are between -5 and 5.}
\[ f(x) = 20x^{3} +31 x^{2} -38 x -40 \]

The solution is \( [-2, -0.8, 1.25] \), which is option A.\begin{enumerate}[label=\Alph*.]
\item \( z_1 \in [-2.05, -1.76], \text{   }  z_2 \in [-0.86, -0.28], \text{   and   } z_3 \in [1.08, 1.34] \)

* This is the solution!
\item \( z_1 \in [-2.05, -1.76], \text{   }  z_2 \in [-1.7, -1.15], \text{   and   } z_3 \in [0.44, 1.08] \)

 Distractor 2: Corresponds to inversing rational roots.
\item \( z_1 \in [-0.51, -0.23], \text{   }  z_2 \in [1.94, 2.57], \text{   and   } z_3 \in [3.41, 4.51] \)

 Distractor 4: Corresponds to moving factors from one rational to another.
\item \( z_1 \in [-0.84, -0.78], \text{   }  z_2 \in [0.9, 1.41], \text{   and   } z_3 \in [1.62, 2.38] \)

 Distractor 3: Corresponds to negatives of all zeros AND inversing rational roots.
\item \( z_1 \in [-1.42, -1.14], \text{   }  z_2 \in [0.26, 1], \text{   and   } z_3 \in [1.62, 2.38] \)

 Distractor 1: Corresponds to negatives of all zeros.
\end{enumerate}

\textbf{General Comment:} Remember to try the middle-most integers first as these normally are the zeros. Also, once you get it to a quadratic, you can use your other factoring techniques to finish factoring.
}
\litem{
Factor the polynomial below completely, knowing that $x+2$ is a factor. Then, choose the intervals the zeros of the polynomial belong to, where $z_1 \leq z_2 \leq z_3 \leq z_4$. \textit{To make the problem easier, all zeros are between -5 and 5.}
\[ f(x) = 8x^{4} -26 x^{3} -69 x^{2} +130 x + 200 \]

The solution is \( [-2, -1.25, 2.5, 4] \), which is option D.\begin{enumerate}[label=\Alph*.]
\item \( z_1 \in [-3.6, -0.1], \text{   }  z_2 \in [-0.83, -0.64], z_3 \in [0.22, 0.54], \text{   and   } z_4 \in [3.28, 4.12] \)

 Distractor 2: Corresponds to inversing rational roots.
\item \( z_1 \in [-4.7, -2.9], \text{   }  z_2 \in [-2.58, -2.37], z_3 \in [1.06, 1.66], \text{   and   } z_4 \in [1.27, 2.9] \)

 Distractor 1: Corresponds to negatives of all zeros.
\item \( z_1 \in [-4.7, -2.9], \text{   }  z_2 \in [-0.4, -0.15], z_3 \in [0.66, 1.04], \text{   and   } z_4 \in [1.27, 2.9] \)

 Distractor 3: Corresponds to negatives of all zeros AND inversing rational roots.
\item \( z_1 \in [-3.6, -0.1], \text{   }  z_2 \in [-1.33, -1.01], z_3 \in [2.41, 2.53], \text{   and   } z_4 \in [3.28, 4.12] \)

* This is the solution!
\item \( z_1 \in [-4.7, -2.9], \text{   }  z_2 \in [-0.73, -0.57], z_3 \in [1.86, 2.33], \text{   and   } z_4 \in [4.81, 5.51] \)

 Distractor 4: Corresponds to moving factors from one rational to another.
\end{enumerate}

\textbf{General Comment:} Remember to try the middle-most integers first as these normally are the zeros. Also, once you get it to a quadratic, you can use your other factoring techniques to finish factoring.
}
\litem{
Factor the polynomial below completely, knowing that $x+3$ is a factor. Then, choose the intervals the zeros of the polynomial belong to, where $z_1 \leq z_2 \leq z_3 \leq z_4$. \textit{To make the problem easier, all zeros are between -5 and 5.}
\[ f(x) = 10x^{4} -39 x^{3} -127 x^{2} +315 x + 225 \]

The solution is \( [-3, -0.6, 2.5, 5] \), which is option D.\begin{enumerate}[label=\Alph*.]
\item \( z_1 \in [-6.3, -3.9], \text{   }  z_2 \in [-2.7, -1.91], z_3 \in [0.49, 0.71], \text{   and   } z_4 \in [2.6, 3.9] \)

 Distractor 1: Corresponds to negatives of all zeros.
\item \( z_1 \in [-3.3, 0.2], \text{   }  z_2 \in [-2.23, -1.58], z_3 \in [0.34, 0.41], \text{   and   } z_4 \in [3.8, 5.4] \)

 Distractor 2: Corresponds to inversing rational roots.
\item \( z_1 \in [-6.3, -3.9], \text{   }  z_2 \in [-5.42, -5], z_3 \in [0.11, 0.38], \text{   and   } z_4 \in [2.6, 3.9] \)

 Distractor 4: Corresponds to moving factors from one rational to another.
\item \( z_1 \in [-3.3, 0.2], \text{   }  z_2 \in [-0.66, -0.58], z_3 \in [2.41, 2.52], \text{   and   } z_4 \in [3.8, 5.4] \)

* This is the solution!
\item \( z_1 \in [-6.3, -3.9], \text{   }  z_2 \in [-0.55, 0.6], z_3 \in [1.58, 1.88], \text{   and   } z_4 \in [2.6, 3.9] \)

 Distractor 3: Corresponds to negatives of all zeros AND inversing rational roots.
\end{enumerate}

\textbf{General Comment:} Remember to try the middle-most integers first as these normally are the zeros. Also, once you get it to a quadratic, you can use your other factoring techniques to finish factoring.
}
\end{enumerate}

\end{document}