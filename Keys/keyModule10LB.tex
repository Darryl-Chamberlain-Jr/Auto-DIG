\documentclass{extbook}[14pt]
\usepackage{multicol, enumerate, enumitem, hyperref, color, soul, setspace, parskip, fancyhdr, amssymb, amsthm, amsmath, bbm, latexsym, units, mathtools}
\everymath{\displaystyle}
\usepackage[headsep=0.5cm,headheight=0cm, left=1 in,right= 1 in,top= 1 in,bottom= 1 in]{geometry}
\pagestyle{fancy}
\lhead{}
\chead{Answer Key for Module\,10L\,-\,Synthetic\,Division Version B}
\rhead{}
\lfoot{Summer\,C\,2020}
\cfoot{}
\rfoot{}
\begin{document}
\textbf{This key should allow you to understand why you choose the option you did (beyond just getting a question right or wrong). \href{https://xronos.clas.ufl.edu/mac1105spring2020/courseDescriptionAndMisc/Exams/LearningFromResults}{More instructions on how to use this key can be found here}.}

\textbf{If you have a suggestion to make the keys better, \href{https://forms.gle/CZkbZmPbC9XALEE88}{please fill out the short survey here}.}

\textit{Note: This key is auto-generated and may contain issues and/or errors. The keys are reviewed after each exam to ensure grading is done accurately. If there are issues (like duplicate options), they are noted in the offline gradebook. The keys are a work-in-progress to give students as many resources to improve as possible.}

\rule{\textwidth}{0.4pt}

66. Factor the polynomial below completely. Then, choose the intervals the zeros of the polynomial belong to, where $z_1 \leq z_2 \leq z_3$. \textit{To make the problem easier, all zeros are between -5 and 5.}
\[ f(x) = 6x^{3} -35 x^{2} +19 x + 30 \] 
The solution is $ [-0.6666666666666666, 1.5, 5] $ 

\begin{enumerate}[label=\Alph*.] 
\item $ z_1 \in [-6.9, -4.8], \text{   }  z_2 \in [-1.6, -1], \text{   and   } z_3 \in [0.55, 0.79] $ 

  Distractor 1: Corresponds to negatives of all zeros. 
\item $ z_1 \in [-6.9, -4.8], \text{   }  z_2 \in [-0.8, 0.1], \text{   and   } z_3 \in [1.17, 1.62] $ 

  Distractor 3: Corresponds to negatives of all zeros AND inversing rational roots. 
\item $ z_1 \in [-6.9, -4.8], \text{   }  z_2 \in [-3.3, -2.4], \text{   and   } z_3 \in [0.12, 0.51] $ 

  Distractor 4: Corresponds to moving factors from one rational to another. 
\item $ z_1 \in [-0.8, 0], \text{   }  z_2 \in [1, 2], \text{   and   } z_3 \in [4.83, 5.67] $ 

 * This is the solution! 
\item $ z_1 \in [-2.6, -0.9], \text{   }  z_2 \in [0.6, 1.3], \text{   and   } z_3 \in [4.83, 5.67] $ 

  Distractor 2: Corresponds to inversing rational roots. 
\end{enumerate} 
 
General Comments: Remember to try the middle-most integers first as these normally are the zeros. Also, once you get it to a quadratic, you can use your other factoring techniques to finish factoring.

-----------------------------------------------

67. Factor the polynomial below completely, knowing that $x+2$ is a factor. Then, choose the intervals the zeros of the polynomial belong to, where $z_1 \leq z_2 \leq z_3 \leq z_4$. \textit{To make the problem easier, all zeros are between -5 and 5.}
\[ f(x) = 12x^{4} +89 x^{3} +204 x^{2} +172 x + 48 \] 
The solution is $ [-4, -2, -0.75, -0.6666666666666666] $ 

\begin{enumerate}[label=\Alph*.] 
\item $ z_1 \in [0.58, 0.81], \text{   }  z_2 \in [0.39, 0.88], z_3 \in [1.13, 2.21], \text{   and   } z_4 \in [3.58, 4.06] $ 

  Distractor 1: Corresponds to negatives of all zeros. 
\item $ z_1 \in [-4.27, -3.23], \text{   }  z_2 \in [-2.33, -1.98], z_3 \in [-1.09, 0.1], \text{   and   } z_4 \in [-1.03, -0.6] $ 

 * This is the solution! 
\item $ z_1 \in [0.88, 1.82], \text{   }  z_2 \in [1.25, 1.93], z_3 \in [1.13, 2.21], \text{   and   } z_4 \in [3.58, 4.06] $ 

  Distractor 3: Corresponds to negatives of all zeros AND inversing rational roots. 
\item $ z_1 \in [-0.39, 0.57], \text{   }  z_2 \in [1.97, 2.62], z_3 \in [1.13, 2.21], \text{   and   } z_4 \in [3.58, 4.06] $ 

  Distractor 4: Corresponds to moving factors from one rational to another. 
\item $ z_1 \in [-4.27, -3.23], \text{   }  z_2 \in [-2.33, -1.98], z_3 \in [-1.52, -1.19], \text{   and   } z_4 \in [-2.73, -0.78] $ 

  Distractor 2: Corresponds to inversing rational roots. 
\end{enumerate} 
 
General Comments: Remember to try the middle-most integers first as these normally are the zeros. Also, once you get it to a quadratic, you can use your other factoring techniques to finish factoring.

-----------------------------------------------

68. What are the \textit{possible Integer} roots of the polynomial below?
\[ f(x) = 4x^{4} +3 x^{3} +7 x^{2} +3 x + 6 \] 
The solution is $ \pm 1,\pm 2,\pm 3,\pm 6 $ 

\begin{enumerate}[label=\Alph*.] 
\item $ \text{ All combinations of: }\frac{\pm 1,\pm 2,\pm 3,\pm 6}{\pm 1,\pm 2,\pm 4} $ 

 This would have been the solution \textbf{if asked for the possible Rational roots}! 
\item $ \text{ All combinations of: }\frac{\pm 1,\pm 2,\pm 4}{\pm 1,\pm 2,\pm 3,\pm 6} $ 

  Distractor 3: Corresponds to the plus or minus of the inverse quotient (an/a0) of the factors.  
\item $ \pm 1,\pm 2,\pm 4 $ 

  Distractor 1: Corresponds to the plus or minus factors of a1 only. 
\item $ \pm 1,\pm 2,\pm 3,\pm 6 $ 

 * This is the solution \textbf{since we asked for the possible Integer roots}! 
\item $ \text{There is no formula or theorem that tells us all possible Integer roots.} $ 

  Distractor 4: Corresponds to not recognizing Integers as a subset of Rationals. 
\end{enumerate} 
 
General Comments: We have a way to find the possible Rational roots. The possible Integer roots are the Integers in this list.

-----------------------------------------------

69. Perform the division below. Then, find the intervals that correspond to the quotient in the form $ax^2+bx+c$ and remainder $r$.
\[ \frac{12x^{3} -59 x^{2} -25 x + 104}{x -5} \] 
The solution is $ 12x^{2} +x -20 + \frac{4}{x -5} $ 

\begin{enumerate}[label=\Alph*.] 
\item $ a \in [11, 14], \text{   } b \in [-121, -116], \text{   } c \in [562, 576], \text{   and   } r \in [-2753, -2741]. $ 

  You divided by the opposite of the factor. 
\item $ a \in [57, 63], \text{   } b \in [-362, -352], \text{   } c \in [1769, 1775], \text{   and   } r \in [-8753, -8742]. $ 

  You divided by the opposite of the factor AND multiplied the first factor rather than just bringing it down. 
\item $ a \in [11, 14], \text{   } b \in [-16, -9], \text{   } c \in [-70, -64], \text{   and   } r \in [-173, -171]. $ 

  You multiplied by the synthetic number and subtracted rather than adding during synthetic division. 
\item $ a \in [57, 63], \text{   } b \in [236, 244], \text{   } c \in [1175, 1183], \text{   and   } r \in [5999, 6009]. $ 

  You multiplied by the synthetic number rather than bringing the first factor down. 
\item $ a \in [11, 14], \text{   } b \in [-1, 2], \text{   } c \in [-25, -12], \text{   and   } r \in [3, 6]. $ 

 * This is the solution! 
\end{enumerate} 
 
General Comments: Be sure to synthetically divide by the zero of the denominator!

-----------------------------------------------

70. Perform the division below. Then, find the intervals that correspond to the quotient in the form $ax^2+bx+c$ and remainder $r$.
\[ \frac{9x^{3} -27 x -16}{x -2} \] 
The solution is $ 9x^{2} +18 x + 9 + \frac{2}{x -2} $ 

\begin{enumerate}[label=\Alph*.] 
\item $ a \in [4, 11], b \in [15, 22], c \in [4, 10], \text{ and } r \in [-1, 3]. $ 

 * This is the solution! 
\item $ a \in [4, 11], b \in [-21, -16], c \in [4, 10], \text{ and } r \in [-37, -33]. $ 

  You divided by the opposite of the factor. 
\item $ a \in [13, 22], b \in [35, 40], c \in [43, 52], \text{ and } r \in [67, 75]. $ 

  You multipled by the synthetic number rather than bringing the first factor down. 
\item $ a \in [13, 22], b \in [-40, -35], c \in [43, 52], \text{ and } r \in [-109, -101]. $ 

  You divided by the opposite of the factor AND multipled the first factor rather than just bringing it down. 
\item $ a \in [4, 11], b \in [6, 10], c \in [-23, -16], \text{ and } r \in [-37, -33]. $ 

  You multipled by the synthetic number and subtracted rather than adding during synthetic division. 
\end{enumerate} 
 
General Comments: Be sure to synthetically divide by the zero of the denominator! Also, make sure to include 0 placeholders for missing terms.

-----------------------------------------------


\end{document}

