\documentclass{extbook}[14pt]
\usepackage{multicol, enumerate, enumitem, hyperref, color, soul, setspace, parskip, fancyhdr, amssymb, amsthm, amsmath, bbm, latexsym, units, mathtools}
\everymath{\displaystyle}
\usepackage[headsep=0.5cm,headheight=0cm, left=1 in,right= 1 in,top= 1 in,bottom= 1 in]{geometry}
\usepackage{dashrule}  % Package to use the command below to create lines between items
\newcommand{\litem}[1]{\item #1

\rule{\textwidth}{0.4pt}}
\pagestyle{fancy}
\lhead{}
\chead{Answer Key for Progress Quiz 9 Version B}
\rhead{}
\lfoot{8590-6105}
\cfoot{}
\rfoot{Fall 2020}
\begin{document}
\textbf{This key should allow you to understand why you choose the option you did (beyond just getting a question right or wrong). \href{https://xronos.clas.ufl.edu/mac1105spring2020/courseDescriptionAndMisc/Exams/LearningFromResults}{More instructions on how to use this key can be found here}.}

\textbf{If you have a suggestion to make the keys better, \href{https://forms.gle/CZkbZmPbC9XALEE88}{please fill out the short survey here}.}

\textit{Note: This key is auto-generated and may contain issues and/or errors. The keys are reviewed after each exam to ensure grading is done accurately. If there are issues (like duplicate options), they are noted in the offline gradebook. The keys are a work-in-progress to give students as many resources to improve as possible.}

\rule{\textwidth}{0.4pt}

\begin{enumerate}\litem{
What are the \textit{possible Integer} roots of the polynomial below?
\[ f(x) = 5x^{3} +7 x^{2} +5 x + 2 \]

The solution is \( \pm 1,\pm 2 \), which is option B.\begin{enumerate}[label=\Alph*.]
\item \( \pm 1,\pm 5 \)

 Distractor 1: Corresponds to the plus or minus factors of a1 only.
\item \( \pm 1,\pm 2 \)

* This is the solution \textbf{since we asked for the possible Integer roots}!
\item \( \text{ All combinations of: }\frac{\pm 1,\pm 5}{\pm 1,\pm 2} \)

 Distractor 3: Corresponds to the plus or minus of the inverse quotient (an/a0) of the factors. 
\item \( \text{ All combinations of: }\frac{\pm 1,\pm 2}{\pm 1,\pm 5} \)

This would have been the solution \textbf{if asked for the possible Rational roots}!
\item \( \text{There is no formula or theorem that tells us all possible Integer roots.} \)

 Distractor 4: Corresponds to not recognizing Integers as a subset of Rationals.
\end{enumerate}

\textbf{General Comment:} We have a way to find the possible Rational roots. The possible Integer roots are the Integers in this list.
}
\litem{
Perform the division below. Then, find the intervals that correspond to the quotient in the form $ax^2+bx+c$ and remainder $r$.
\[ \frac{20x^{3} +29 x^{2} -81 x + 31}{x + 3} \]

The solution is \( 20x^{2} -31 x + 12 + \frac{-5}{x + 3} \), which is option E.\begin{enumerate}[label=\Alph*.]
\item \( a \in [14, 25], \text{   } b \in [-54, -49], \text{   } c \in [121, 126], \text{   and   } r \in [-463, -460]. \)

 You multiplied by the synthetic number and subtracted rather than adding during synthetic division.
\item \( a \in [14, 25], \text{   } b \in [78, 93], \text{   } c \in [185, 193], \text{   and   } r \in [589, 590]. \)

 You divided by the opposite of the factor.
\item \( a \in [-63, -58], \text{   } b \in [-155, -147], \text{   } c \in [-536, -531], \text{   and   } r \in [-1575, -1568]. \)

 You divided by the opposite of the factor AND multiplied the first factor rather than just bringing it down.
\item \( a \in [-63, -58], \text{   } b \in [204, 212], \text{   } c \in [-714, -704], \text{   and   } r \in [2152, 2156]. \)

 You multiplied by the synthetic number rather than bringing the first factor down.
\item \( a \in [14, 25], \text{   } b \in [-35, -29], \text{   } c \in [6, 16], \text{   and   } r \in [-6, -1]. \)

* This is the solution!
\end{enumerate}

\textbf{General Comment:} Be sure to synthetically divide by the zero of the denominator!
}
\litem{
Factor the polynomial below completely. Then, choose the intervals the zeros of the polynomial belong to, where $z_1 \leq z_2 \leq z_3$. \textit{To make the problem easier, all zeros are between -5 and 5.}
\[ f(x) = 15x^{3} +79 x^{2} +82 x + 24 \]

The solution is \( [-4, -0.6666666666666666, -0.6] \), which is option E.\begin{enumerate}[label=\Alph*.]
\item \( z_1 \in [1.36, 1.69], \text{   }  z_2 \in [1.61, 1.89], \text{   and   } z_3 \in [3.75, 4.1] \)

 Distractor 3: Corresponds to negatives of all zeros AND inversing rational roots.
\item \( z_1 \in [-4.03, -3.6], \text{   }  z_2 \in [-1.92, -1.22], \text{   and   } z_3 \in [-1.51, -1] \)

 Distractor 2: Corresponds to inversing rational roots.
\item \( z_1 \in [0.41, 0.72], \text{   }  z_2 \in [0.31, 0.74], \text{   and   } z_3 \in [3.75, 4.1] \)

 Distractor 1: Corresponds to negatives of all zeros.
\item \( z_1 \in [-0.12, 0.24], \text{   }  z_2 \in [1.73, 2.38], \text{   and   } z_3 \in [3.75, 4.1] \)

 Distractor 4: Corresponds to moving factors from one rational to another.
\item \( z_1 \in [-4.03, -3.6], \text{   }  z_2 \in [-1.02, -0.07], \text{   and   } z_3 \in [-0.98, -0.18] \)

* This is the solution!
\end{enumerate}

\textbf{General Comment:} Remember to try the middle-most integers first as these normally are the zeros. Also, once you get it to a quadratic, you can use your other factoring techniques to finish factoring.
}
\litem{
Factor the polynomial below completely, knowing that $x-4$ is a factor. Then, choose the intervals the zeros of the polynomial belong to, where $z_1 \leq z_2 \leq z_3 \leq z_4$. \textit{To make the problem easier, all zeros are between -5 and 5.}
\[ f(x) = 25x^{4} -175 x^{3} +356 x^{2} -236 x + 48 \]

The solution is \( [0.4, 0.6, 2, 4] \), which is option E.\begin{enumerate}[label=\Alph*.]
\item \( z_1 \in [-5.1, -2.9], \text{   }  z_2 \in [-3.21, -2.94], z_3 \in [-2.25, -1.81], \text{   and   } z_4 \in [-0.17, 0.28] \)

 Distractor 4: Corresponds to moving factors from one rational to another.
\item \( z_1 \in [1.3, 2.1], \text{   }  z_2 \in [1.12, 2.02], z_3 \in [2.2, 3.35], \text{   and   } z_4 \in [3.14, 4.24] \)

 Distractor 2: Corresponds to inversing rational roots.
\item \( z_1 \in [-5.1, -2.9], \text{   }  z_2 \in [-2.49, -1.62], z_3 \in [-0.69, -0.01], \text{   and   } z_4 \in [-0.61, -0.26] \)

 Distractor 1: Corresponds to negatives of all zeros.
\item \( z_1 \in [-5.1, -2.9], \text{   }  z_2 \in [-2.72, -2.43], z_3 \in [-2.25, -1.81], \text{   and   } z_4 \in [-2.25, -1.63] \)

 Distractor 3: Corresponds to negatives of all zeros AND inversing rational roots.
\item \( z_1 \in [0.1, 0.8], \text{   }  z_2 \in [0.28, 1.28], z_3 \in [1.76, 2.02], \text{   and   } z_4 \in [3.14, 4.24] \)

* This is the solution!
\end{enumerate}

\textbf{General Comment:} Remember to try the middle-most integers first as these normally are the zeros. Also, once you get it to a quadratic, you can use your other factoring techniques to finish factoring.
}
\litem{
Perform the division below. Then, find the intervals that correspond to the quotient in the form $ax^2+bx+c$ and remainder $r$.
\[ \frac{12x^{3} +28 x^{2} -14}{x + 2} \]

The solution is \( 12x^{2} +4 x -8 + \frac{2}{x + 2} \), which is option E.\begin{enumerate}[label=\Alph*.]
\item \( a \in [-28, -19], b \in [67, 81], c \in [-153, -146], \text{ and } r \in [290, 293]. \)

 You multipled by the synthetic number rather than bringing the first factor down.
\item \( a \in [9, 21], b \in [-8, -3], c \in [22, 33], \text{ and } r \in [-87, -83]. \)

 You multipled by the synthetic number and subtracted rather than adding during synthetic division.
\item \( a \in [9, 21], b \in [48, 56], c \in [104, 105], \text{ and } r \in [193, 196]. \)

 You divided by the opposite of the factor.
\item \( a \in [-28, -19], b \in [-26, -18], c \in [-41, -33], \text{ and } r \in [-96, -90]. \)

 You divided by the opposite of the factor AND multipled the first factor rather than just bringing it down.
\item \( a \in [9, 21], b \in [3, 8], c \in [-8, -1], \text{ and } r \in [-4, 6]. \)

* This is the solution!
\end{enumerate}

\textbf{General Comment:} Be sure to synthetically divide by the zero of the denominator! Also, make sure to include 0 placeholders for missing terms.
}
\litem{
Factor the polynomial below completely. Then, choose the intervals the zeros of the polynomial belong to, where $z_1 \leq z_2 \leq z_3$. \textit{To make the problem easier, all zeros are between -5 and 5.}
\[ f(x) = 20x^{3} +121 x^{2} +184 x + 80 \]

The solution is \( [-4, -1.25, -0.8] \), which is option C.\begin{enumerate}[label=\Alph*.]
\item \( z_1 \in [0.74, 1.49], \text{   }  z_2 \in [1.25, 2.25], \text{   and   } z_3 \in [3, 5] \)

 Distractor 3: Corresponds to negatives of all zeros AND inversing rational roots.
\item \( z_1 \in [0.74, 1.49], \text{   }  z_2 \in [1.25, 2.25], \text{   and   } z_3 \in [3, 5] \)

 Distractor 1: Corresponds to negatives of all zeros.
\item \( z_1 \in [-4.53, -3.89], \text{   }  z_2 \in [-2.25, 0.75], \text{   and   } z_3 \in [-0.8, 2.2] \)

* This is the solution!
\item \( z_1 \in [-4.53, -3.89], \text{   }  z_2 \in [-2.25, 0.75], \text{   and   } z_3 \in [-0.8, 2.2] \)

 Distractor 2: Corresponds to inversing rational roots.
\item \( z_1 \in [0.1, 0.49], \text{   }  z_2 \in [4, 7], \text{   and   } z_3 \in [3, 5] \)

 Distractor 4: Corresponds to moving factors from one rational to another.
\end{enumerate}

\textbf{General Comment:} Remember to try the middle-most integers first as these normally are the zeros. Also, once you get it to a quadratic, you can use your other factoring techniques to finish factoring.
}
\litem{
Perform the division below. Then, find the intervals that correspond to the quotient in the form $ax^2+bx+c$ and remainder $r$.
\[ \frac{10x^{3} +85 x^{2} +200 x + 130}{x + 5} \]

The solution is \( 10x^{2} +35 x + 25 + \frac{5}{x + 5} \), which is option A.\begin{enumerate}[label=\Alph*.]
\item \( a \in [6, 11], \text{   } b \in [32, 39], \text{   } c \in [25, 28], \text{   and   } r \in [5, 12]. \)

* This is the solution!
\item \( a \in [-52, -47], \text{   } b \in [-166, -161], \text{   } c \in [-629, -620], \text{   and   } r \in [-2999, -2993]. \)

 You divided by the opposite of the factor AND multiplied the first factor rather than just bringing it down.
\item \( a \in [6, 11], \text{   } b \in [24, 29], \text{   } c \in [41, 55], \text{   and   } r \in [-172, -162]. \)

 You multiplied by the synthetic number and subtracted rather than adding during synthetic division.
\item \( a \in [-52, -47], \text{   } b \in [331, 340], \text{   } c \in [-1477, -1472], \text{   and   } r \in [7503, 7507]. \)

 You multiplied by the synthetic number rather than bringing the first factor down.
\item \( a \in [6, 11], \text{   } b \in [132, 142], \text{   } c \in [867, 878], \text{   and   } r \in [4503, 4509]. \)

 You divided by the opposite of the factor.
\end{enumerate}

\textbf{General Comment:} Be sure to synthetically divide by the zero of the denominator!
}
\litem{
What are the \textit{possible Rational} roots of the polynomial below?
\[ f(x) = 6x^{4} +7 x^{3} +3 x^{2} +6 x + 3 \]

The solution is \( \text{ All combinations of: }\frac{\pm 1,\pm 3}{\pm 1,\pm 2,\pm 3,\pm 6} \), which is option D.\begin{enumerate}[label=\Alph*.]
\item \( \pm 1,\pm 2,\pm 3,\pm 6 \)

 Distractor 1: Corresponds to the plus or minus factors of a1 only.
\item \( \pm 1,\pm 3 \)

This would have been the solution \textbf{if asked for the possible Integer roots}!
\item \( \text{ All combinations of: }\frac{\pm 1,\pm 2,\pm 3,\pm 6}{\pm 1,\pm 3} \)

 Distractor 3: Corresponds to the plus or minus of the inverse quotient (an/a0) of the factors. 
\item \( \text{ All combinations of: }\frac{\pm 1,\pm 3}{\pm 1,\pm 2,\pm 3,\pm 6} \)

* This is the solution \textbf{since we asked for the possible Rational roots}!
\item \( \text{ There is no formula or theorem that tells us all possible Rational roots.} \)

 Distractor 4: Corresponds to not recalling the theorem for rational roots of a polynomial.
\end{enumerate}

\textbf{General Comment:} We have a way to find the possible Rational roots. The possible Integer roots are the Integers in this list.
}
\litem{
Perform the division below. Then, find the intervals that correspond to the quotient in the form $ax^2+bx+c$ and remainder $r$.
\[ \frac{20x^{3} +105 x^{2} -120}{x + 5} \]

The solution is \( 20x^{2} +5 x -25 + \frac{5}{x + 5} \), which is option C.\begin{enumerate}[label=\Alph*.]
\item \( a \in [-104, -91], b \in [603, 611], c \in [-3028, -3019], \text{ and } r \in [15003, 15007]. \)

 You multipled by the synthetic number rather than bringing the first factor down.
\item \( a \in [20, 22], b \in [204, 210], c \in [1018, 1028], \text{ and } r \in [5000, 5011]. \)

 You divided by the opposite of the factor.
\item \( a \in [20, 22], b \in [4, 7], c \in [-25, -23], \text{ and } r \in [2, 7]. \)

* This is the solution!
\item \( a \in [20, 22], b \in [-19, -12], c \in [85, 92], \text{ and } r \in [-662, -657]. \)

 You multipled by the synthetic number and subtracted rather than adding during synthetic division.
\item \( a \in [-104, -91], b \in [-403, -390], c \in [-1978, -1968], \text{ and } r \in [-9996, -9992]. \)

 You divided by the opposite of the factor AND multipled the first factor rather than just bringing it down.
\end{enumerate}

\textbf{General Comment:} Be sure to synthetically divide by the zero of the denominator! Also, make sure to include 0 placeholders for missing terms.
}
\litem{
Factor the polynomial below completely, knowing that $x-5$ is a factor. Then, choose the intervals the zeros of the polynomial belong to, where $z_1 \leq z_2 \leq z_3 \leq z_4$. \textit{To make the problem easier, all zeros are between -5 and 5.}
\[ f(x) = 12x^{4} -19 x^{3} -215 x^{2} -25 x + 375 \]

The solution is \( [-3, -1.6666666666666667, 1.25, 5] \), which is option C.\begin{enumerate}[label=\Alph*.]
\item \( z_1 \in [-4.8, -2], \text{   }  z_2 \in [-0.75, -0.48], z_3 \in [0.79, 1.02], \text{   and   } z_4 \in [5, 8] \)

 Distractor 2: Corresponds to inversing rational roots.
\item \( z_1 \in [-6.3, -4.2], \text{   }  z_2 \in [-1.15, -0.71], z_3 \in [0.59, 0.65], \text{   and   } z_4 \in [3, 4] \)

 Distractor 3: Corresponds to negatives of all zeros AND inversing rational roots.
\item \( z_1 \in [-4.8, -2], \text{   }  z_2 \in [-2.09, -1.49], z_3 \in [1.19, 1.37], \text{   and   } z_4 \in [5, 8] \)

* This is the solution!
\item \( z_1 \in [-6.3, -4.2], \text{   }  z_2 \in [-5.09, -4.63], z_3 \in [0.41, 0.46], \text{   and   } z_4 \in [3, 4] \)

 Distractor 4: Corresponds to moving factors from one rational to another.
\item \( z_1 \in [-6.3, -4.2], \text{   }  z_2 \in [-1.58, -1.01], z_3 \in [1.4, 1.78], \text{   and   } z_4 \in [3, 4] \)

 Distractor 1: Corresponds to negatives of all zeros.
\end{enumerate}

\textbf{General Comment:} Remember to try the middle-most integers first as these normally are the zeros. Also, once you get it to a quadratic, you can use your other factoring techniques to finish factoring.
}
\end{enumerate}

\end{document}