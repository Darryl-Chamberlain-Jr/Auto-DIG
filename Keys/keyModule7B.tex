\documentclass{extbook}[14pt]
\usepackage{multicol, enumerate, enumitem, hyperref, color, soul, setspace, parskip, fancyhdr, amssymb, amsthm, amsmath, bbm, latexsym, units, mathtools}
\everymath{\displaystyle}
\usepackage[headsep=0.5cm,headheight=0cm, left=1 in,right= 1 in,top= 1 in,bottom= 1 in]{geometry}
\usepackage{dashrule}  % Package to use the command below to create lines between items
\newcommand{\litem}[1]{\item #1

\rule{\textwidth}{0.4pt}}
\pagestyle{fancy}
\lhead{}
\chead{Answer Key for Progress Quiz 8 Version B}
\rhead{}
\lfoot{4553-3922}
\cfoot{}
\rfoot{Fall 2020}
\begin{document}
\textbf{This key should allow you to understand why you choose the option you did (beyond just getting a question right or wrong). \href{https://xronos.clas.ufl.edu/mac1105spring2020/courseDescriptionAndMisc/Exams/LearningFromResults}{More instructions on how to use this key can be found here}.}

\textbf{If you have a suggestion to make the keys better, \href{https://forms.gle/CZkbZmPbC9XALEE88}{please fill out the short survey here}.}

\textit{Note: This key is auto-generated and may contain issues and/or errors. The keys are reviewed after each exam to ensure grading is done accurately. If there are issues (like duplicate options), they are noted in the offline gradebook. The keys are a work-in-progress to give students as many resources to improve as possible.}

\rule{\textwidth}{0.4pt}

\begin{enumerate}\litem{
Determine the domain of the function below.
\[ f(x) = \frac{3}{12x^{2} -32 x + 20} \]

The solution is \( \text{All Real numbers except } x = 1.000 \text{ and } x = 1.667. \), which is option E.\begin{enumerate}[label=\Alph*.]
\item \( \text{All Real numbers except } x = a \text{ and } x = b, \text{ where } a \in [13.79, 15.39] \text{ and } b \in [15.86, 16.46] \)

All Real numbers except $x = 15.000$ and $x = 16.000$, which corresponds to not factoring the denominator correctly.
\item \( \text{All Real numbers.} \)

This corresponds to thinking the denominator has complex roots or that rational functions have a domain of all Real numbers.
\item \( \text{All Real numbers except } x = a, \text{ where } a \in [0.09, 1.33] \)

All Real numbers except $x = 1.000$, which corresponds to removing only 1 value from the denominator.
\item \( \text{All Real numbers except } x = a, \text{ where } a \in [13.79, 15.39] \)

All Real numbers except $x = 15.000$, which corresponds to removing a distractor value from the denominator.
\item \( \text{All Real numbers except } x = a \text{ and } x = b, \text{ where } a \in [0.09, 1.33] \text{ and } b \in [1.55, 1.9] \)

All Real numbers except $x = 1.000$ and $x = 1.667$, which is the correct option.
\end{enumerate}

\textbf{General Comment:} Recall that dividing by zero is not a real number. Therefore the domain is all real numbers \textbf{except} those that make the denominator 0.
}
\litem{
Solve the rational equation below. Then, choose the interval(s) that the solution(s) belongs to.
\[ \frac{-50}{-50x + 30} + 1 = \frac{-50}{-50x + 30} \]

The solution is \( \text{all solutions are invalid or lead to complex values in the equation.} \), which is option B.\begin{enumerate}[label=\Alph*.]
\item \( x_1 \in [-0.4, 1.6] \text{ and } x_2 \in [0.6,3.6] \)

$x = 0.600 \text{ and } x = 0.600$, which corresponds to getting the correct solution and believing there should be a second solution to the equation.
\item \( \text{All solutions lead to invalid or complex values in the equation.} \)

*$x = 0.600$ leads to dividing by 0 in the original equation and thus is not a valid solution, which is the correct option.
\item \( x \in [-1.6,0.4] \)

$x = -0.600$, which corresponds to not distributing the factor $-50x + 30$ correctly when trying to eliminate the fraction.
\item \( x \in [0.6,1.6] \)

$x = 0.600$, which corresponds to not checking if this value leads to dividing by 0 in the original equation and thus is not a valid solution.
\item \( x_1 \in [-1.6, 0.4] \text{ and } x_2 \in [0.6,3.6] \)

$x = -0.600 \text{ and } x = 0.600$, which corresponds to getting the correct solution and believing there should be a second solution to the equation.
\end{enumerate}

\textbf{General Comment:} Distractors are different based on the number of solutions. Remember that after solving, we need to make sure our solution does not make the original equation divide by zero!
}
\litem{
Solve the rational equation below. Then, choose the interval(s) that the solution(s) belongs to.
\[ \frac{3x}{2x + 6} + \frac{-3x^{2}}{12x^{2} +50 x + 42} = \frac{3}{6x + 7} \]

The solution is \( \text{There are two solutions: } x = 0.704 \text{ and } x = -1.704 \), which is option B.\begin{enumerate}[label=\Alph*.]
\item \( x_1 \in [0.04, 1.11] \text{ and } x_2 \in [-3.91,-1.76] \)


\item \( x_1 \in [0.04, 1.11] \text{ and } x_2 \in [-2.04,-1.39] \)

* $x = 0.704 \text{ and } x = -1.704$, which is the correct option.
\item \( \text{All solutions lead to invalid or complex values in the equation.} \)


\item \( x \in [-1.62,-0.61] \)


\item \( x \in [-2.44,-1.5] \)


\end{enumerate}

\textbf{General Comment:} Distractors are different based on the number of solutions. Remember that after solving, we need to make sure our solution does not make the original equation divide by zero!
}
\litem{
Choose the graph of the equation below.
\[ f(x) = \frac{1}{x - 1} - 1 \]

The solution is the graph below, which is option A.
\begin{center}
    \includegraphics[width=0.3\textwidth]{../Figures/rationalEquationToGraphCopyAB.png}
\end{center}\begin{enumerate}[label=\Alph*.]
\begin{multicols}{2}
\item \includegraphics[width = 0.3\textwidth]{../Figures/rationalEquationToGraphCopyAB.png}
\item \includegraphics[width = 0.3\textwidth]{../Figures/rationalEquationToGraphCopyBB.png}
\item \includegraphics[width = 0.3\textwidth]{../Figures/rationalEquationToGraphCopyCB.png}
\item \includegraphics[width = 0.3\textwidth]{../Figures/rationalEquationToGraphCopyDB.png}
\end{multicols}\item None of the above.\end{enumerate}
\textbf{General Comment:} Remember that the general form of a basic rational equation is $ f(x) = \frac{a}{(x-h)^n} + k$, where $a$ is the leading coefficient (and in this case, we assume is either $1$ or $-1$), $n$ is the degree (in this case, either $1$ or $2$), and $(h, k)$ is the intersection of the asymptotes.
}
\litem{
Choose the equation of the function graphed below.

\begin{center}
    \includegraphics[width=0.5\textwidth]{../Figures/rationalGraphToEquationB.png}
\end{center}




The solution is \( f(x) = \frac{1}{(x - 1)^2} - 3 \), which is option A.\begin{enumerate}[label=\Alph*.]
\item \( f(x) = \frac{1}{(x - 1)^2} - 3 \)

This is the correct option.
\item \( f(x) = \frac{-1}{(x + 1)^2} - 3 \)

Corresponds to using the general form $f(x) = \frac{a}{(x+h)^2}+k$ and the opposite leading coefficient.
\item \( f(x) = \frac{-1}{x + 1} - 3 \)

Corresponds to thinking the graph was a shifted version of $\frac{1}{x}$, using the general form $f(x) = \frac{a}{(x+h)^2}+k$, and the opposite leading coefficient.
\item \( f(x) = \frac{1}{x - 1} - 3 \)

Corresponds to thinking the graph was a shifted version of $\frac{1}{x}$.
\item \( \text{None of the above} \)

This corresponds to believing the vertex of the graph was not correct.
\end{enumerate}

\textbf{General Comment:} Remember that the general form of a basic rational equation is $ f(x) = \frac{a}{(x-h)^n} + k$, where $a$ is the leading coefficient (and in this case, we assume is either $1$ or $-1$), $n$ is the degree (in this case, either $1$ or $2$), and $(h, k)$ is the intersection of the asymptotes.
}
\litem{
Choose the graph of the equation below.
\[ f(x) = \frac{1}{(x - 2)^2} + 1 \]

The solution is the graph below, which is option E.
\begin{center}
    \includegraphics[width=0.3\textwidth]{../Figures/rationalEquationToGraphEB.png}
\end{center}\begin{enumerate}[label=\Alph*.]
\begin{multicols}{2}
\item \includegraphics[width = 0.3\textwidth]{../Figures/rationalEquationToGraphAB.png}
\item \includegraphics[width = 0.3\textwidth]{../Figures/rationalEquationToGraphBB.png}
\item \includegraphics[width = 0.3\textwidth]{../Figures/rationalEquationToGraphCB.png}
\item \includegraphics[width = 0.3\textwidth]{../Figures/rationalEquationToGraphDB.png}
\end{multicols}\item None of the above.\end{enumerate}
\textbf{General Comment:} Remember that the general form of a basic rational equation is $ f(x) = \frac{a}{(x-h)^n} + k$, where $a$ is the leading coefficient (and in this case, we assume is either $1$ or $-1$), $n$ is the degree (in this case, either $1$ or $2$), and $(h, k)$ is the intersection of the asymptotes.
}
\litem{
Solve the rational equation below. Then, choose the interval(s) that the solution(s) belongs to.
\[ \frac{5x}{-7x -2} + \frac{-4x^{2}}{21x^{2} +34 x + 8} = \frac{6}{-3x -4} \]

The solution is \( \text{There are two solutions: } x = -0.404 \text{ and } x = 1.562 \), which is option C.\begin{enumerate}[label=\Alph*.]
\item \( x \in [-1.7,-1.32] \)


\item \( x \in [1.21,2.75] \)


\item \( x_1 \in [-0.75, 0.03] \text{ and } x_2 \in [-0.2,4.1] \)

* $x = -0.404 \text{ and } x = 1.562$, which is the correct option.
\item \( \text{All solutions lead to invalid or complex values in the equation.} \)


\item \( x_1 \in [-0.75, 0.03] \text{ and } x_2 \in [-2.5,0.7] \)


\end{enumerate}

\textbf{General Comment:} Distractors are different based on the number of solutions. Remember that after solving, we need to make sure our solution does not make the original equation divide by zero!
}
\litem{
Choose the equation of the function graphed below.

\begin{center}
    \includegraphics[width=0.5\textwidth]{../Figures/rationalGraphToEquationCopyB.png}
\end{center}




The solution is \( \text{None of the above as it should be } f(x) = \frac{-1}{(x + 2)^2} - 3 \), which is option E.\begin{enumerate}[label=\Alph*.]
\item \( f(x) = \frac{-1}{x - 2} - 8 \)

Corresponds to thinking the graph was a shifted version of $\frac{1}{x}$ AND not noticing the $y$-value was wrong.
\item \( f(x) = \frac{1}{x + 2} - 8 \)

Corresponds to thinking the graph was a shifted version of $\frac{1}{x}$, using the general form $f(x) = \frac{a}{(x-h)^2}+k$, the opposite leading coefficient, AND not noticing the $y$-value was wrong.
\item \( f(x) = \frac{-1}{(x - 2)^2} - 8 \)

The $x$- and $y$-value of the equation does not match the graph.
\item \( f(x) = \frac{1}{(x + 2)^2} - 8 \)

Corresponds to using the general form $f(x) = \frac{a}{(x-h)^2}+k$, the opposite leading coefficient, AND not noticing the $y$-value was wrong.
\item \( \text{None of the above} \)

None of the equation options were the correct equation.
\end{enumerate}

\textbf{General Comment:} Remember that the general form of a basic rational equation is $ f(x) = \frac{a}{(x-h)^n} + k$, where $a$ is the leading coefficient (and in this case, we assume is either $1$ or $-1$), $n$ is the degree (in this case, either $1$ or $2$), and $(h, k)$ is the intersection of the asymptotes.
}
\litem{
Determine the domain of the function below.
\[ f(x) = \frac{6}{20x^{2} -49 x + 30} \]

The solution is \( \text{All Real numbers except } x = 1.200 \text{ and } x = 1.250. \), which is option C.\begin{enumerate}[label=\Alph*.]
\item \( \text{All Real numbers.} \)

This corresponds to thinking the denominator has complex roots or that rational functions have a domain of all Real numbers.
\item \( \text{All Real numbers except } x = a \text{ and } x = b, \text{ where } a \in [19.93, 20.12] \text{ and } b \in [29.94, 30.1] \)

All Real numbers except $x = 20.000$ and $x = 30.000$, which corresponds to not factoring the denominator correctly.
\item \( \text{All Real numbers except } x = a \text{ and } x = b, \text{ where } a \in [1.17, 1.21] \text{ and } b \in [1.25, 1.36] \)

All Real numbers except $x = 1.200$ and $x = 1.250$, which is the correct option.
\item \( \text{All Real numbers except } x = a, \text{ where } a \in [19.93, 20.12] \)

All Real numbers except $x = 20.000$, which corresponds to removing a distractor value from the denominator.
\item \( \text{All Real numbers except } x = a, \text{ where } a \in [1.17, 1.21] \)

All Real numbers except $x = 1.200$, which corresponds to removing only 1 value from the denominator.
\end{enumerate}

\textbf{General Comment:} Recall that dividing by zero is not a real number. Therefore the domain is all real numbers \textbf{except} those that make the denominator 0.
}
\litem{
Solve the rational equation below. Then, choose the interval(s) that the solution(s) belongs to.
\[ \frac{4}{9x -6} + 2 = \frac{6}{-27x + 18} \]

The solution is \( x = 0.333 \), which is option A.\begin{enumerate}[label=\Alph*.]
\item \( x \in [0.33,2.33] \)

* $x = 0.333$, which is the correct option.
\item \( \text{All solutions lead to invalid or complex values in the equation.} \)

This corresponds to thinking $x = 0.333$ leads to dividing by zero in the original equation, which it does not.
\item \( x \in [-1,0] \)

$x = -1.000$, which corresponds to not distributing the factor $9x -6$ correctly when trying to eliminate the fraction.
\item \( x_1 \in [-1, 0] \text{ and } x_2 \in [-0.8,0.4] \)

$x = -1.000 \text{ and } x = 0.333$, which corresponds to getting the correct solution and believing there should be a second solution to the equation.
\item \( x_1 \in [-0.67, 2.33] \text{ and } x_2 \in [0.7,2.2] \)

$x = 0.333 \text{ and } x = 0.778$, which corresponds to getting the correct solution and believing there should be a second solution to the equation.
\end{enumerate}

\textbf{General Comment:} Distractors are different based on the number of solutions. Remember that after solving, we need to make sure our solution does not make the original equation divide by zero!
}
\end{enumerate}

\end{document}