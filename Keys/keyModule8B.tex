\documentclass{extbook}[14pt]
\usepackage{multicol, enumerate, enumitem, hyperref, color, soul, setspace, parskip, fancyhdr, amssymb, amsthm, amsmath, bbm, latexsym, units, mathtools}
\everymath{\displaystyle}
\usepackage[headsep=0.5cm,headheight=0cm, left=1 in,right= 1 in,top= 1 in,bottom= 1 in]{geometry}
\usepackage{dashrule}  % Package to use the command below to create lines between items
\newcommand{\litem}[1]{\item #1

\rule{\textwidth}{0.4pt}}
\pagestyle{fancy}
\lhead{}
\chead{Answer Key for Progress Quiz 8 Version B}
\rhead{}
\lfoot{4553-3922}
\cfoot{}
\rfoot{Fall 2020}
\begin{document}
\textbf{This key should allow you to understand why you choose the option you did (beyond just getting a question right or wrong). \href{https://xronos.clas.ufl.edu/mac1105spring2020/courseDescriptionAndMisc/Exams/LearningFromResults}{More instructions on how to use this key can be found here}.}

\textbf{If you have a suggestion to make the keys better, \href{https://forms.gle/CZkbZmPbC9XALEE88}{please fill out the short survey here}.}

\textit{Note: This key is auto-generated and may contain issues and/or errors. The keys are reviewed after each exam to ensure grading is done accurately. If there are issues (like duplicate options), they are noted in the offline gradebook. The keys are a work-in-progress to give students as many resources to improve as possible.}

\rule{\textwidth}{0.4pt}

\begin{enumerate}\litem{
Which of the following intervals describes the Range of the function below?
\[ f(x) = \log_2{(x+2)}-3 \]

The solution is \( (\infty, \infty) \), which is option E.\begin{enumerate}[label=\Alph*.]
\item \( (-\infty, a), a \in [2.75, 3.43] \)

$(-\infty, 3)$, which corresponds to using the using the negative of vertical shift on $(0, \infty)$.
\item \( [a, \infty), a \in [-2.02, -1.88] \)

$[-3, \infty)$, which corresponds to using the flipped Domain AND including the endpoint.
\item \( [a, \infty), a \in [1.72, 2.56] \)

$[2, \infty)$, which corresponds to using the negative of the horizontal shift AND including the endpoint.
\item \( (-\infty, a), a \in [-3.22, -2.99] \)

$(-\infty, -3)$, which corresponds to using the vertical shift while the Range is $(-\infty, \infty)$.
\item \( (-\infty, \infty) \)

*This is the correct option.
\end{enumerate}

\textbf{General Comment:} \textbf{General Comments}: The domain of a basic logarithmic function is $(0, \infty)$ and the Range is $(-\infty, \infty)$. We can use shifts when finding the Domain, but the Range will always be all Real numbers.
}
\litem{
Solve the equation for $x$ and choose the interval that contains the solution (if it exists).
\[ 4^{-5x-2} = 49^{-4x+5} \]

The solution is \( x = 2.574 \), which is option C.\begin{enumerate}[label=\Alph*.]
\item \( x \in [-8, -5.2] \)

$x = -7.000$, which corresponds to solving the numerators as equal while ignoring the bases are different.
\item \( x \in [-22.6, -20.7] \)

$x = -22.232$, which corresponds to distributing the $\ln(base)$ to the second term of the exponent only.
\item \( x \in [1.3, 2.8] \)

* $x = 2.574$, which is the correct option.
\item \( x \in [-1.7, 2] \)

$x = 0.811$, which corresponds to distributing the $\ln(base)$ to the first term of the exponent only.
\item \( \text{There is no Real solution to the equation.} \)

This corresponds to believing there is no solution since the bases are not powers of each other.
\end{enumerate}

\textbf{General Comment:} \textbf{General Comments:} This question was written so that the bases could not be written the same. You will need to take the log of both sides.
}
\litem{
 Solve the equation for $x$ and choose the interval that contains $x$ (if it exists).
\[  21 = \sqrt[6]{\frac{15}{e^{4x}}} \]

The solution is \( x = -3.89 \), which is option A.\begin{enumerate}[label=\Alph*.]
\item \( x \in [-5.89, -1.89] \)

* $x = -3.890$, which is the correct option.
\item \( x \in [-0.85, 0.15] \)

$x = -0.845$, which corresponds to treating any root as a square root.
\item \( x \in [-34.18, -27.18] \)

$x = -32.177$, which corresponds to thinking you don't need to take the natural log of both sides before reducing, as if the equation already had a natural log on the right side.
\item \( \text{There is no Real solution to the equation.} \)

This corresponds to believing you cannot solve the equation.
\item \( \text{None of the above.} \)

This corresponds to making an unexpected error.
\end{enumerate}

\textbf{General Comment:} \textbf{General Comments}: After using the properties of logarithmic functions to break up the right-hand side, use $\ln(e) = 1$ to reduce the question to a linear function to solve. You can put $\ln(15)$ into a calculator if you are having trouble.
}
\litem{
Solve the equation for $x$ and choose the interval that contains the solution (if it exists).
\[ 2^{-4x-2} = 9^{-2x+4} \]

The solution is \( x = 6.274 \), which is option C.\begin{enumerate}[label=\Alph*.]
\item \( x \in [0.7, 5.7] \)

$x = 3.699$, which corresponds to distributing the $\ln(base)$ to the first term of the exponent only.
\item \( x \in [-10.09, -4.09] \)

$x = -5.088$, which corresponds to distributing the $\ln(base)$ to the second term of the exponent only.
\item \( x \in [6.27, 8.27] \)

* $x = 6.274$, which is the correct option.
\item \( x \in [-4, -1] \)

$x = -3.000$, which corresponds to solving the numerators as equal while ignoring the bases are different.
\item \( \text{There is no Real solution to the equation.} \)

This corresponds to believing there is no solution since the bases are not powers of each other.
\end{enumerate}

\textbf{General Comment:} \textbf{General Comments:} This question was written so that the bases could not be written the same. You will need to take the log of both sides.
}
\litem{
Which of the following intervals describes the Domain of the function below?
\[ f(x) = \log_2{(x-4)}-2 \]

The solution is \( (4, \infty) \), which is option C.\begin{enumerate}[label=\Alph*.]
\item \( [a, \infty), a \in [-2.8, -0.3] \)

$[-2, \infty)$, which corresponds to using the vertical shift when shifting the Domain AND including the endpoint.
\item \( (-\infty, a], a \in [1.6, 3.7] \)

$(-\infty, 2]$, which corresponds to using the negative vertical shift AND including the endpoint AND flipping the domain.
\item \( (a, \infty), a \in [2.7, 5.4] \)

* $(4, \infty)$, which is the correct option.
\item \( (-\infty, a), a \in [-5.5, -3.4] \)

$(-\infty, -4)$, which corresponds to flipping the Domain. Remember: the general for is $a*\log(x-h)+k$, \textbf{where $a$ does not affect the domain}.
\item \( (-\infty, \infty) \)

This corresponds to thinking of the range of the log function (or the domain of the exponential function).
\end{enumerate}

\textbf{General Comment:} \textbf{General Comments}: The domain of a basic logarithmic function is $(0, \infty)$ and the Range is $(-\infty, \infty)$. We can use shifts when finding the Domain, but the Range will always be all Real numbers.
}
\litem{
Which of the following intervals describes the Domain of the function below?
\[ f(x) = -e^{x-7}+9 \]

The solution is \( (-\infty, \infty) \), which is option E.\begin{enumerate}[label=\Alph*.]
\item \( (a, \infty), a \in [-9, -6] \)

$(-9, \infty)$, which corresponds to using the negative vertical shift AND flipping the Range interval.
\item \( [a, \infty), a \in [-9, -6] \)

$[-9, \infty)$, which corresponds to using the negative vertical shift AND flipping the Range interval AND including the endpoint.
\item \( (-\infty, a], a \in [3, 10] \)

$(-\infty, 9]$, which corresponds to using the correct vertical shift *if we wanted the Range* AND including the endpoint.
\item \( (-\infty, a), a \in [3, 10] \)

$(-\infty, 9)$, which corresponds to using the correct vertical shift *if we wanted the Range*.
\item \( (-\infty, \infty) \)

* This is the correct option.
\end{enumerate}

\textbf{General Comment:} \textbf{General Comments}: Domain of a basic exponential function is $(-\infty, \infty)$ while the Range is $(0, \infty)$. We can shift these intervals [and even flip when $a<0$!] to find the new Domain/Range.
}
\litem{
 Solve the equation for $x$ and choose the interval that contains $x$ (if it exists).
\[  25 = \sqrt[7]{\frac{14}{e^{6x}}} \]

The solution is \( x = -3.316, \text{ which does not fit in any of the interval options.} \), which is option E.\begin{enumerate}[label=\Alph*.]
\item \( x \in [-1.63, 0.37] \)

$x = -0.633$, which corresponds to treating any root as a square root.
\item \( x \in [2.32, 7.32] \)

$x = 3.316$, which is the negative of the correct solution.
\item \( x \in [-30.61, -26.61] \)

$x = -29.607$, which corresponds to thinking you don't need to take the natural log of both sides before reducing, as if the right side already has a natural log.
\item \( \text{There is no Real solution to the equation.} \)

This corresponds to believing you cannot solve the equation.
\item \( \text{None of the above.} \)

* $x = -3.316$ is the correct solution and does not fit in any of the other intervals.
\end{enumerate}

\textbf{General Comment:} \textbf{General Comments}: After using the properties of logarithmic functions to break up the right-hand side, use $\ln(e) = 1$ to reduce the question to a linear function to solve. You can put $\ln(14)$ into a calculator if you are having trouble.
}
\litem{
Solve the equation for $x$ and choose the interval that contains the solution (if it exists).
\[ \log_{5}{(-3x+7)}+5 = 3 \]

The solution is \( x = 2.320 \), which is option A.\begin{enumerate}[label=\Alph*.]
\item \( x \in [1.32, 6.32] \)

* $x = 2.320$, which is the correct option.
\item \( x \in [10, 14] \)

$x = 13.000$, which corresponds to reversing the base and exponent when converting.
\item \( x \in [-45.33, -35.33] \)

$x = -39.333$, which corresponds to ignoring the vertical shift when converting to exponential form.
\item \( x \in [6.33, 12.33] \)

$x = 8.333$, which corresponds to reversing the base and exponent when converting and reversing the value with $x$.
\item \( \text{There is no Real solution to the equation.} \)

Corresponds to believing a negative coefficient within the log equation means there is no Real solution.
\end{enumerate}

\textbf{General Comment:} \textbf{General Comments:} First, get the equation in the form $\log_b{(cx+d)} = a$. Then, convert to $b^a = cx+d$ and solve.
}
\litem{
Solve the equation for $x$ and choose the interval that contains the solution (if it exists).
\[ \log_{4}{(-3x+7)}+5 = 3 \]

The solution is \( x = 2.312 \), which is option C.\begin{enumerate}[label=\Alph*.]
\item \( x \in [-22, -17] \)

$x = -19.000$, which corresponds to ignoring the vertical shift when converting to exponential form.
\item \( x \in [-8.67, -3.67] \)

$x = -7.667$, which corresponds to reversing the base and exponent when converting and reversing the value with $x$.
\item \( x \in [-0.69, 8.31] \)

* $x = 2.312$, which is the correct option.
\item \( x \in [-5, 2] \)

$x = -3.000$, which corresponds to reversing the base and exponent when converting.
\item \( \text{There is no Real solution to the equation.} \)

Corresponds to believing a negative coefficient within the log equation means there is no Real solution.
\end{enumerate}

\textbf{General Comment:} \textbf{General Comments:} First, get the equation in the form $\log_b{(cx+d)} = a$. Then, convert to $b^a = cx+d$ and solve.
}
\litem{
Which of the following intervals describes the Domain of the function below?
\[ f(x) = e^{x-6}-5 \]

The solution is \( (-\infty, \infty) \), which is option E.\begin{enumerate}[label=\Alph*.]
\item \( (-\infty, a), a \in [-6, -1] \)

$(-\infty, -5)$, which corresponds to using the correct vertical shift *if we wanted the Range*.
\item \( (a, \infty), a \in [0, 7] \)

$(5, \infty)$, which corresponds to using the negative vertical shift AND flipping the Range interval.
\item \( (-\infty, a], a \in [-6, -1] \)

$(-\infty, -5]$, which corresponds to using the correct vertical shift *if we wanted the Range* AND including the endpoint.
\item \( [a, \infty), a \in [0, 7] \)

$[5, \infty)$, which corresponds to using the negative vertical shift AND flipping the Range interval AND including the endpoint.
\item \( (-\infty, \infty) \)

* This is the correct option.
\end{enumerate}

\textbf{General Comment:} \textbf{General Comments}: Domain of a basic exponential function is $(-\infty, \infty)$ while the Range is $(0, \infty)$. We can shift these intervals [and even flip when $a<0$!] to find the new Domain/Range.
}
\end{enumerate}

\end{document}