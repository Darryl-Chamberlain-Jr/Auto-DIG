\documentclass{extbook}[14pt]
\usepackage{multicol, enumerate, enumitem, hyperref, color, soul, setspace, parskip, fancyhdr, amssymb, amsthm, amsmath, bbm, latexsym, units, mathtools}
\everymath{\displaystyle}
\usepackage[headsep=0.5cm,headheight=0cm, left=1 in,right= 1 in,top= 1 in,bottom= 1 in]{geometry}
\usepackage{dashrule}  % Package to use the command below to create lines between items
\newcommand{\litem}[1]{\item #1

\rule{\textwidth}{0.4pt}}
\pagestyle{fancy}
\lhead{}
\chead{Answer Key for Progress Quiz 4 Version B}
\rhead{}
\lfoot{8448-1521}
\cfoot{}
\rfoot{Fall 2020}
\begin{document}
\textbf{This key should allow you to understand why you choose the option you did (beyond just getting a question right or wrong). \href{https://xronos.clas.ufl.edu/mac1105spring2020/courseDescriptionAndMisc/Exams/LearningFromResults}{More instructions on how to use this key can be found here}.}

\textbf{If you have a suggestion to make the keys better, \href{https://forms.gle/CZkbZmPbC9XALEE88}{please fill out the short survey here}.}

\textit{Note: This key is auto-generated and may contain issues and/or errors. The keys are reviewed after each exam to ensure grading is done accurately. If there are issues (like duplicate options), they are noted in the offline gradebook. The keys are a work-in-progress to give students as many resources to improve as possible.}

\rule{\textwidth}{0.4pt}

\begin{enumerate}\litem{
Solve the equation for $x$ and choose the interval that contains the solution (if it exists).
\[ 2^{-5x+2} = \left(\frac{1}{125}\right)^{-3x-5} \]
The solution is \( x = -1.268 \), which is option D.\begin{enumerate}[label=\Alph*.]
\item \( x \in [-11.38, -7.38] \)

$x = -11.378$, which corresponds to distributing the $\ln(base)$ to the second term of the exponent only.
\item \( x \in [3.5, 5.5] \)

$x = 3.500$, which corresponds to solving the numerators as equal while ignoring the bases are different.
\item \( x \in [-0.61, 2.39] \)

$x = 0.390$, which corresponds to distributing the $\ln(base)$ to the first term of the exponent only.
\item \( x \in [-4.27, -0.27] \)

* $x = -1.268$, which is the correct option.
\item \( \text{There is no Real solution to the equation.} \)

This corresponds to believing there is no solution since the bases are not powers of each other.
\end{enumerate}

\textbf{General Comment:} \textbf{General Comments:} This question was written so that the bases could not be written the same. You will need to take the log of both sides.
}
\litem{
Solve the equation for $x$ and choose the interval that contains the solution (if it exists).
\[ \log_{3}{(-2x+6)}+6 = 3 \]
The solution is \( x = 2.981 \), which is option D.\begin{enumerate}[label=\Alph*.]
\item \( x \in [14.5, 18.5] \)

$x = 16.500$, which corresponds to reversing the base and exponent when converting.
\item \( x \in [-10.5, -9.5] \)

$x = -10.500$, which corresponds to ignoring the vertical shift when converting to exponential form.
\item \( x \in [8.5, 12.5] \)

$x = 10.500$, which corresponds to reversing the base and exponent when converting and reversing the value with $x$.
\item \( x \in [2.98, 6.98] \)

* $x = 2.981$, which is the correct option.
\item \( \text{There is no Real solution to the equation.} \)

Corresponds to believing a negative coefficient within the log equation means there is no Real solution.
\end{enumerate}

\textbf{General Comment:} \textbf{General Comments:} First, get the equation in the form $\log_b{(cx+d)} = a$. Then, convert to $b^a = cx+d$ and solve.
}
\litem{
Which of the following intervals describes the Domain of the function below?
\[ f(x) = \log_2{(x+4)}-8 \]
The solution is \( (-4, \infty) \), which is option D.\begin{enumerate}[label=\Alph*.]
\item \( (-\infty, a], a \in [5.9, 8.7] \)

$(-\infty, 8]$, which corresponds to using the negative vertical shift AND including the endpoint AND flipping the domain.
\item \( [a, \infty), a \in [-12.5, -7.2] \)

$[-8, \infty)$, which corresponds to using the vertical shift when shifting the Domain AND including the endpoint.
\item \( (-\infty, a), a \in [1.1, 4.1] \)

$(-\infty, 4)$, which corresponds to flipping the Domain. Remember: the general for is $a*\log(x-h)+k$, \textbf{where $a$ does not affect the domain}.
\item \( (a, \infty), a \in [-6.3, -1.8] \)

* $(-4, \infty)$, which is the correct option.
\item \( (-\infty, \infty) \)

This corresponds to thinking of the range of the log function (or the domain of the exponential function).
\end{enumerate}

\textbf{General Comment:} \textbf{General Comments}: The domain of a basic logarithmic function is $(0, \infty)$ and the Range is $(-\infty, \infty)$. We can use shifts when finding the Domain, but the Range will always be all Real numbers.
}
\litem{
Which of the following intervals describes the Domain of the function below?
\[ f(x) = \log_2{(x+5)}+4 \]
The solution is \( (-5, \infty) \), which is option B.\begin{enumerate}[label=\Alph*.]
\item \( [a, \infty), a \in [3.79, 4.47] \)

$[4, \infty)$, which corresponds to using the vertical shift when shifting the Domain AND including the endpoint.
\item \( (a, \infty), a \in [-5.17, -4.79] \)

* $(-5, \infty)$, which is the correct option.
\item \( (-\infty, a], a \in [-4.95, -3.6] \)

$(-\infty, -4]$, which corresponds to using the negative vertical shift AND including the endpoint AND flipping the domain.
\item \( (-\infty, a), a \in [4.35, 5.47] \)

$(-\infty, 5)$, which corresponds to flipping the Domain. Remember: the general for is $a*\log(x-h)+k$, \textbf{where $a$ does not affect the domain}.
\item \( (-\infty, \infty) \)

This corresponds to thinking of the range of the log function (or the domain of the exponential function).
\end{enumerate}

\textbf{General Comment:} \textbf{General Comments}: The domain of a basic logarithmic function is $(0, \infty)$ and the Range is $(-\infty, \infty)$. We can use shifts when finding the Domain, but the Range will always be all Real numbers.
}
\litem{
Which of the following intervals describes the Domain of the function below?
\[ f(x) = e^{x+6}-4 \]
The solution is \( (-\infty, \infty) \), which is option E.\begin{enumerate}[label=\Alph*.]
\item \( (a, \infty), a \in [-1, 9] \)

$(4, \infty)$, which corresponds to using the negative vertical shift AND flipping the Range interval.
\item \( (-\infty, a), a \in [-8, -3] \)

$(-\infty, -4)$, which corresponds to using the correct vertical shift *if we wanted the Range*.
\item \( [a, \infty), a \in [-1, 9] \)

$[4, \infty)$, which corresponds to using the negative vertical shift AND flipping the Range interval AND including the endpoint.
\item \( (-\infty, a], a \in [-8, -3] \)

$(-\infty, -4]$, which corresponds to using the correct vertical shift *if we wanted the Range* AND including the endpoint.
\item \( (-\infty, \infty) \)

* This is the correct option.
\end{enumerate}

\textbf{General Comment:} \textbf{General Comments}: Domain of a basic exponential function is $(-\infty, \infty)$ while the Range is $(0, \infty)$. We can shift these intervals [and even flip when $a<0$!] to find the new Domain/Range.
}
\litem{
 Solve the equation for $x$ and choose the interval that contains $x$ (if it exists).
\[  13 = \sqrt[3]{\frac{30}{e^{3x}}} \]
The solution is \( x = -1.431, \text{ which does not fit in any of the interval options.} \), which is option E.\begin{enumerate}[label=\Alph*.]
\item \( x \in [-1, 0.7] \)

$x = -0.576$, which corresponds to treating any root as a square root.
\item \( x \in [-0.1, 3] \)

$x = 1.431$, which is the negative of the correct solution.
\item \( x \in [-15.2, -14] \)

$x = -14.134$, which corresponds to thinking you don't need to take the natural log of both sides before reducing, as if the right side already has a natural log.
\item \( \text{There is no Real solution to the equation.} \)

This corresponds to believing you cannot solve the equation.
\item \( \text{None of the above.} \)

* $x = -1.431$ is the correct solution and does not fit in any of the other intervals.
\end{enumerate}

\textbf{General Comment:} \textbf{General Comments}: After using the properties of logarithmic functions to break up the right-hand side, use $\ln(e) = 1$ to reduce the question to a linear function to solve. You can put $\ln(30)$ into a calculator if you are having trouble.
}
\litem{
Solve the equation for $x$ and choose the interval that contains the solution (if it exists).
\[ 4^{-3x+5} = 9^{-5x-5} \]
The solution is \( x = -2.624 \), which is option A.\begin{enumerate}[label=\Alph*.]
\item \( x \in [-3.19, -1.95] \)

* $x = -2.624$, which is the correct option.
\item \( x \in [-1.81, -0.72] \)

$x = -1.465$, which corresponds to distributing the $\ln(base)$ to the first term of the exponent only.
\item \( x \in [-5.68, -4.83] \)

$x = -5.000$, which corresponds to solving the numerators as equal while ignoring the bases are different.
\item \( x \in [-10.42, -8.54] \)

$x = -8.959$, which corresponds to distributing the $\ln(base)$ to the second term of the exponent only.
\item \( \text{There is no Real solution to the equation.} \)

This corresponds to believing there is no solution since the bases are not powers of each other.
\end{enumerate}

\textbf{General Comment:} \textbf{General Comments:} This question was written so that the bases could not be written the same. You will need to take the log of both sides.
}
\litem{
 Solve the equation for $x$ and choose the interval that contains $x$ (if it exists).
\[  23 = \sqrt[5]{\frac{5}{e^{9x}}} \]
The solution is \( x = -1.563, \text{ which does not fit in any of the interval options.} \), which is option E.\begin{enumerate}[label=\Alph*.]
\item \( x \in [1.1, 3.7] \)

$x = 1.563$, which is the negative of the correct solution.
\item \( x \in [-13.9, -12.6] \)

$x = -12.957$, which corresponds to thinking you don't need to take the natural log of both sides before reducing, as if the right side already has a natural log.
\item \( x \in [-0.9, -0.2] \)

$x = -0.518$, which corresponds to treating any root as a square root.
\item \( \text{There is no Real solution to the equation.} \)

This corresponds to believing you cannot solve the equation.
\item \( \text{None of the above.} \)

* $x = -1.563$ is the correct solution and does not fit in any of the other intervals.
\end{enumerate}

\textbf{General Comment:} \textbf{General Comments}: After using the properties of logarithmic functions to break up the right-hand side, use $\ln(e) = 1$ to reduce the question to a linear function to solve. You can put $\ln(5)$ into a calculator if you are having trouble.
}
\litem{
Solve the equation for $x$ and choose the interval that contains the solution (if it exists).
\[ \log_{4}{(4x+8)}+6 = 2 \]
The solution is \( x = -1.999 \), which is option D.\begin{enumerate}[label=\Alph*.]
\item \( x \in [64, 71] \)

$x = 66.000$, which corresponds to reversing the base and exponent when converting and reversing the value with $x$.
\item \( x \in [1, 6] \)

$x = 2.000$, which corresponds to ignoring the vertical shift when converting to exponential form.
\item \( x \in [60, 64] \)

$x = 62.000$, which corresponds to reversing the base and exponent when converting.
\item \( x \in [-4, 1] \)

* $x = -1.999$, which is the correct option.
\item \( \text{There is no Real solution to the equation.} \)

Corresponds to believing a negative coefficient within the log equation means there is no Real solution.
\end{enumerate}

\textbf{General Comment:} \textbf{General Comments:} First, get the equation in the form $\log_b{(cx+d)} = a$. Then, convert to $b^a = cx+d$ and solve.
}
\litem{
Which of the following intervals describes the Range of the function below?
\[ f(x) = -e^{x+3}-3 \]
The solution is \( (-\infty, -3) \), which is option C.\begin{enumerate}[label=\Alph*.]
\item \( (-\infty, a], a \in [-3, 2] \)

$(-\infty, -3]$, which corresponds to including the endpoint.
\item \( [a, \infty), a \in [0, 7] \)

$[3, \infty)$, which corresponds to using the negative vertical shift AND flipping the Range interval AND including the endpoint.
\item \( (-\infty, a), a \in [-3, 2] \)

* $(-\infty, -3)$, which is the correct option.
\item \( (a, \infty), a \in [0, 7] \)

$(3, \infty)$, which corresponds to using the negative vertical shift AND flipping the Range interval.
\item \( (-\infty, \infty) \)

This corresponds to confusing range of an exponential function with the domain of an exponential function.
\end{enumerate}

\textbf{General Comment:} \textbf{General Comments}: Domain of a basic exponential function is $(-\infty, \infty)$ while the Range is $(0, \infty)$. We can shift these intervals [and even flip when $a<0$!] to find the new Domain/Range.
}
\end{enumerate}

\end{document}