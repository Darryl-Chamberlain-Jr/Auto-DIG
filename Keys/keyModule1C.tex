\documentclass{extbook}[14pt]
\usepackage{multicol, enumerate, enumitem, hyperref, color, soul, setspace, parskip, fancyhdr, amssymb, amsthm, amsmath, bbm, latexsym, units, mathtools}
\everymath{\displaystyle}
\usepackage[headsep=0.5cm,headheight=0cm, left=1 in,right= 1 in,top= 1 in,bottom= 1 in]{geometry}
\usepackage{dashrule}  % Package to use the command below to create lines between items
\newcommand{\litem}[1]{\item #1

\rule{\textwidth}{0.4pt}}
\pagestyle{fancy}
\lhead{}
\chead{Answer Key for Progress Quiz 8 Version C}
\rhead{}
\lfoot{4553-3922}
\cfoot{}
\rfoot{Fall 2020}
\begin{document}
\textbf{This key should allow you to understand why you choose the option you did (beyond just getting a question right or wrong). \href{https://xronos.clas.ufl.edu/mac1105spring2020/courseDescriptionAndMisc/Exams/LearningFromResults}{More instructions on how to use this key can be found here}.}

\textbf{If you have a suggestion to make the keys better, \href{https://forms.gle/CZkbZmPbC9XALEE88}{please fill out the short survey here}.}

\textit{Note: This key is auto-generated and may contain issues and/or errors. The keys are reviewed after each exam to ensure grading is done accurately. If there are issues (like duplicate options), they are noted in the offline gradebook. The keys are a work-in-progress to give students as many resources to improve as possible.}

\rule{\textwidth}{0.4pt}

\begin{enumerate}\litem{
Simplify the expression below into the form $a+bi$. Then, choose the intervals that $a$ and $b$ belong to.
\[ \frac{-54 + 77 i}{5 - i} \]

The solution is \( -13.35  + 12.73 i \), which is option A.\begin{enumerate}[label=\Alph*.]
\item \( a \in [-14, -12.5] \text{ and } b \in [11.5, 13] \)

* $-13.35  + 12.73 i$, which is the correct option.
\item \( a \in [-12, -10.5] \text{ and } b \in [-77.5, -76.5] \)

 $-10.80  - 77.00 i$, which corresponds to just dividing the first term by the first term and the second by the second.
\item \( a \in [-14, -12.5] \text{ and } b \in [330.5, 331.5] \)

 $-13.35  + 331.00 i$, which corresponds to forgetting to multiply the conjugate by the numerator.
\item \( a \in [-8, -6.5] \text{ and } b \in [15.5, 18.5] \)

 $-7.42  + 16.88 i$, which corresponds to forgetting to multiply the conjugate by the numerator and not computing the conjugate correctly.
\item \( a \in [-348, -346.5] \text{ and } b \in [11.5, 13] \)

 $-347.00  + 12.73 i$, which corresponds to forgetting to multiply the conjugate by the numerator and using a plus instead of a minus in the denominator.
\end{enumerate}

\textbf{General Comment:} Multiply the numerator and denominator by the *conjugate* of the denominator, then simplify. For example, if we have $2+3i$, the conjugate is $2-3i$.
}
\litem{
Simplify the expression below into the form $a+bi$. Then, choose the intervals that $a$ and $b$ belong to.
\[ \frac{63 - 55 i}{-8 + i} \]

The solution is \( -8.60  + 5.80 i \), which is option B.\begin{enumerate}[label=\Alph*.]
\item \( a \in [-9.2, -8.2] \text{ and } b \in [376, 377.5] \)

 $-8.60  + 377.00 i$, which corresponds to forgetting to multiply the conjugate by the numerator.
\item \( a \in [-9.2, -8.2] \text{ and } b \in [5, 6] \)

* $-8.60  + 5.80 i$, which is the correct option.
\item \( a \in [-8.2, -7.35] \text{ and } b \in [-56, -54.5] \)

 $-7.88  - 55.00 i$, which corresponds to just dividing the first term by the first term and the second by the second.
\item \( a \in [-7.5, -6.55] \text{ and } b \in [7.5, 8] \)

 $-6.91  + 7.74 i$, which corresponds to forgetting to multiply the conjugate by the numerator and not computing the conjugate correctly.
\item \( a \in [-559.1, -558.35] \text{ and } b \in [5, 6] \)

 $-559.00  + 5.80 i$, which corresponds to forgetting to multiply the conjugate by the numerator and using a plus instead of a minus in the denominator.
\end{enumerate}

\textbf{General Comment:} Multiply the numerator and denominator by the *conjugate* of the denominator, then simplify. For example, if we have $2+3i$, the conjugate is $2-3i$.
}
\litem{
Simplify the expression below into the form $a+bi$. Then, choose the intervals that $a$ and $b$ belong to.
\[ (-7 + 8 i)(3 + 10 i) \]

The solution is \( -101 - 46 i \), which is option B.\begin{enumerate}[label=\Alph*.]
\item \( a \in [59, 63] \text{ and } b \in [-96, -90] \)

 $59 - 94 i$, which corresponds to adding a minus sign in the first term.
\item \( a \in [-109, -100] \text{ and } b \in [-49, -44] \)

* $-101 - 46 i$, which is the correct option.
\item \( a \in [-109, -100] \text{ and } b \in [46, 48] \)

 $-101 + 46 i$, which corresponds to adding a minus sign in both terms.
\item \( a \in [-31, -19] \text{ and } b \in [76, 86] \)

 $-21 + 80 i$, which corresponds to just multiplying the real terms to get the real part of the solution and the coefficients in the complex terms to get the complex part.
\item \( a \in [59, 63] \text{ and } b \in [94, 98] \)

 $59 + 94 i$, which corresponds to adding a minus sign in the second term.
\end{enumerate}

\textbf{General Comment:} You can treat $i$ as a variable and distribute. Just remember that $i^2=-1$, so you can continue to reduce after you distribute.
}
\litem{
Choose the \textbf{smallest} set of Real numbers that the number below belongs to.
\[ \sqrt{\frac{1716}{11}} \]

The solution is \( \text{Irrational} \), which is option B.\begin{enumerate}[label=\Alph*.]
\item \( \text{Whole} \)

These are the counting numbers with 0 (0, 1, 2, 3, ...)
\item \( \text{Irrational} \)

* This is the correct option!
\item \( \text{Rational} \)

These are numbers that can be written as fraction of Integers (e.g., -2/3)
\item \( \text{Not a Real number} \)

These are Nonreal Complex numbers \textbf{OR} things that are not numbers (e.g., dividing by 0).
\item \( \text{Integer} \)

These are the negative and positive counting numbers (..., -3, -2, -1, 0, 1, 2, 3, ...)
\end{enumerate}

\textbf{General Comment:} First, you \textbf{NEED} to simplify the expression. This question simplifies to $\sqrt{156}$. 
 
 Be sure you look at the simplified fraction and not just the decimal expansion. Numbers such as 13, 17, and 19 provide \textbf{long but repeating/terminating decimal expansions!} 
 
 The only ways to *not* be a Real number are: dividing by 0 or taking the square root of a negative number. 
 
 Irrational numbers are more than just square root of 3: adding or subtracting values from square root of 3 is also irrational.
}
\litem{
Simplify the expression below and choose the interval the simplification is contained within.
\[ 17 - 20^2 + 16 \div 12 * 5 \div 13 \]

The solution is \( -382.487 \), which is option A.\begin{enumerate}[label=\Alph*.]
\item \( [-382.5, -382.32] \)

* -382.487, this is the correct option
\item \( [-383.45, -382.66] \)

 -382.979, which corresponds to an Order of Operations error: not reading left-to-right for multiplication/division.
\item \( [416.92, 417.21] \)

 417.021, which corresponds to two Order of Operations errors.
\item \( [417.51, 417.62] \)

 417.513, which corresponds to an Order of Operations error: multiplying by negative before squaring. For example: $(-3)^2 \neq -3^2$
\item \( \text{None of the above} \)

 You may have gotten this by making an unanticipated error. If you got a value that is not any of the others, please let the coordinator know so they can help you figure out what happened.
\end{enumerate}

\textbf{General Comment:} While you may remember (or were taught) PEMDAS is done in order, it is actually done as P/E/MD/AS. When we are at MD or AS, we read left to right.
}
\litem{
Simplify the expression below and choose the interval the simplification is contained within.
\[ 15 - 9^2 + 20 \div 6 * 10 \div 4 \]

The solution is \( -57.667 \), which is option C.\begin{enumerate}[label=\Alph*.]
\item \( [95.08, 101.08] \)

 96.083, which corresponds to two Order of Operations errors.
\item \( [-65.92, -64.92] \)

 -65.917, which corresponds to an Order of Operations error: not reading left-to-right for multiplication/division.
\item \( [-58.67, -55.67] \)

* -57.667, this is the correct option
\item \( [97.33, 111.33] \)

 104.333, which corresponds to an Order of Operations error: multiplying by negative before squaring. For example: $(-3)^2 \neq -3^2$
\item \( \text{None of the above} \)

 You may have gotten this by making an unanticipated error. If you got a value that is not any of the others, please let the coordinator know so they can help you figure out what happened.
\end{enumerate}

\textbf{General Comment:} While you may remember (or were taught) PEMDAS is done in order, it is actually done as P/E/MD/AS. When we are at MD or AS, we read left to right.
}
\litem{
Choose the \textbf{smallest} set of Complex numbers that the number below belongs to.
\[ \sqrt{\frac{-1980}{12}}+\sqrt{0}i \]

The solution is \( \text{Pure Imaginary} \), which is option E.\begin{enumerate}[label=\Alph*.]
\item \( \text{Rational} \)

These are numbers that can be written as fraction of Integers (e.g., -2/3 + 5)
\item \( \text{Nonreal Complex} \)

This is a Complex number $(a+bi)$ that is not Real (has $i$ as part of the number).
\item \( \text{Irrational} \)

These cannot be written as a fraction of Integers. Remember: $\pi$ is not an Integer!
\item \( \text{Not a Complex Number} \)

This is not a number. The only non-Complex number we know is dividing by 0 as this is not a number!
\item \( \text{Pure Imaginary} \)

* This is the correct option!
\end{enumerate}

\textbf{General Comment:} Be sure to simplify $i^2 = -1$. This may remove the imaginary portion for your number. If you are having trouble, you may want to look at the \textit{Subgroups of the Real Numbers} section.
}
\litem{
Choose the \textbf{smallest} set of Complex numbers that the number below belongs to.
\[ \frac{\sqrt{154}}{13}+\sqrt{-7}i \]

The solution is \( \text{Irrational} \), which is option A.\begin{enumerate}[label=\Alph*.]
\item \( \text{Irrational} \)

* This is the correct option!
\item \( \text{Rational} \)

These are numbers that can be written as fraction of Integers (e.g., -2/3 + 5)
\item \( \text{Nonreal Complex} \)

This is a Complex number $(a+bi)$ that is not Real (has $i$ as part of the number).
\item \( \text{Pure Imaginary} \)

This is a Complex number $(a+bi)$ that \textbf{only} has an imaginary part like $2i$.
\item \( \text{Not a Complex Number} \)

This is not a number. The only non-Complex number we know is dividing by 0 as this is not a number!
\end{enumerate}

\textbf{General Comment:} Be sure to simplify $i^2 = -1$. This may remove the imaginary portion for your number. If you are having trouble, you may want to look at the \textit{Subgroups of the Real Numbers} section.
}
\litem{
Simplify the expression below into the form $a+bi$. Then, choose the intervals that $a$ and $b$ belong to.
\[ (6 + 7 i)(-9 - 5 i) \]

The solution is \( -19 - 93 i \), which is option D.\begin{enumerate}[label=\Alph*.]
\item \( a \in [-25, -18] \text{ and } b \in [90.8, 93.8] \)

 $-19 + 93 i$, which corresponds to adding a minus sign in both terms.
\item \( a \in [-89, -84] \text{ and } b \in [31, 36.1] \)

 $-89 + 33 i$, which corresponds to adding a minus sign in the first term.
\item \( a \in [-89, -84] \text{ and } b \in [-34.1, -31.3] \)

 $-89 - 33 i$, which corresponds to adding a minus sign in the second term.
\item \( a \in [-25, -18] \text{ and } b \in [-95.7, -90.9] \)

* $-19 - 93 i$, which is the correct option.
\item \( a \in [-60, -49] \text{ and } b \in [-38.1, -34] \)

 $-54 - 35 i$, which corresponds to just multiplying the real terms to get the real part of the solution and the coefficients in the complex terms to get the complex part.
\end{enumerate}

\textbf{General Comment:} You can treat $i$ as a variable and distribute. Just remember that $i^2=-1$, so you can continue to reduce after you distribute.
}
\litem{
Choose the \textbf{smallest} set of Real numbers that the number below belongs to.
\[ -\sqrt{\frac{256}{289}} \]

The solution is \( \text{Rational} \), which is option A.\begin{enumerate}[label=\Alph*.]
\item \( \text{Rational} \)

* This is the correct option!
\item \( \text{Integer} \)

These are the negative and positive counting numbers (..., -3, -2, -1, 0, 1, 2, 3, ...)
\item \( \text{Irrational} \)

These cannot be written as a fraction of Integers.
\item \( \text{Not a Real number} \)

These are Nonreal Complex numbers \textbf{OR} things that are not numbers (e.g., dividing by 0).
\item \( \text{Whole} \)

These are the counting numbers with 0 (0, 1, 2, 3, ...)
\end{enumerate}

\textbf{General Comment:} First, you \textbf{NEED} to simplify the expression. This question simplifies to $-\frac{16}{17}$. 
 
 Be sure you look at the simplified fraction and not just the decimal expansion. Numbers such as 13, 17, and 19 provide \textbf{long but repeating/terminating decimal expansions!} 
 
 The only ways to *not* be a Real number are: dividing by 0 or taking the square root of a negative number. 
 
 Irrational numbers are more than just square root of 3: adding or subtracting values from square root of 3 is also irrational.
}
\end{enumerate}

\end{document}