\documentclass{extbook}[14pt]
\usepackage{multicol, enumerate, enumitem, hyperref, color, soul, setspace, parskip, fancyhdr, amssymb, amsthm, amsmath, latexsym, units, mathtools}
\everymath{\displaystyle}
\usepackage[headsep=0.5cm,headheight=0cm, left=1 in,right= 1 in,top= 1 in,bottom= 1 in]{geometry}
\usepackage{dashrule}  % Package to use the command below to create lines between items
\newcommand{\litem}[1]{\item #1

\rule{\textwidth}{0.4pt}}
\pagestyle{fancy}
\lhead{}
\chead{Answer Key for Makeup Progress Quiz 3 Version C}
\rhead{}
\lfoot{1648-1753}
\cfoot{}
\rfoot{Summer C 2021}
\begin{document}
\textbf{This key should allow you to understand why you choose the option you did (beyond just getting a question right or wrong). \href{https://xronos.clas.ufl.edu/mac1105spring2020/courseDescriptionAndMisc/Exams/LearningFromResults}{More instructions on how to use this key can be found here}.}

\textbf{If you have a suggestion to make the keys better, \href{https://forms.gle/CZkbZmPbC9XALEE88}{please fill out the short survey here}.}

\textit{Note: This key is auto-generated and may contain issues and/or errors. The keys are reviewed after each exam to ensure grading is done accurately. If there are issues (like duplicate options), they are noted in the offline gradebook. The keys are a work-in-progress to give students as many resources to improve as possible.}

\rule{\textwidth}{0.4pt}

\begin{enumerate}\litem{
Choose the \textbf{smallest} set of Complex numbers that the number below belongs to.
\[ \sqrt{\frac{-605}{11}}+\sqrt{0}i \]The solution is \( \text{Pure Imaginary} \), which is option D.\begin{enumerate}[label=\Alph*.]
\item \( \text{Not a Complex Number} \)

This is not a number. The only non-Complex number we know is dividing by 0 as this is not a number!
\item \( \text{Irrational} \)

These cannot be written as a fraction of Integers. Remember: $\pi$ is not an Integer!
\item \( \text{Nonreal Complex} \)

This is a Complex number $(a+bi)$ that is not Real (has $i$ as part of the number).
\item \( \text{Pure Imaginary} \)

* This is the correct option!
\item \( \text{Rational} \)

These are numbers that can be written as fraction of Integers (e.g., -2/3 + 5)
\end{enumerate}

\textbf{General Comment:} Be sure to simplify $i^2 = -1$. This may remove the imaginary portion for your number. If you are having trouble, you may want to look at the \textit{Subgroups of the Real Numbers} section.
}
\litem{
Choose the \textbf{smallest} set of Real numbers that the number below belongs to.
\[ -\sqrt{\frac{100}{529}} \]The solution is \( \text{Rational} \), which is option A.\begin{enumerate}[label=\Alph*.]
\item \( \text{Rational} \)

* This is the correct option!
\item \( \text{Not a Real number} \)

These are Nonreal Complex numbers \textbf{OR} things that are not numbers (e.g., dividing by 0).
\item \( \text{Integer} \)

These are the negative and positive counting numbers (..., -3, -2, -1, 0, 1, 2, 3, ...)
\item \( \text{Whole} \)

These are the counting numbers with 0 (0, 1, 2, 3, ...)
\item \( \text{Irrational} \)

These cannot be written as a fraction of Integers.
\end{enumerate}

\textbf{General Comment:} First, you \textbf{NEED} to simplify the expression. This question simplifies to $-\frac{10}{23}$. 
 
 Be sure you look at the simplified fraction and not just the decimal expansion. Numbers such as 13, 17, and 19 provide \textbf{long but repeating/terminating decimal expansions!} 
 
 The only ways to *not* be a Real number are: dividing by 0 or taking the square root of a negative number. 
 
 Irrational numbers are more than just square root of 3: adding or subtracting values from square root of 3 is also irrational.
}
\litem{
Choose the \textbf{smallest} set of Real numbers that the number below belongs to.
\[ -\sqrt{\frac{25}{361}} \]The solution is \( \text{Rational} \), which is option D.\begin{enumerate}[label=\Alph*.]
\item \( \text{Irrational} \)

These cannot be written as a fraction of Integers.
\item \( \text{Not a Real number} \)

These are Nonreal Complex numbers \textbf{OR} things that are not numbers (e.g., dividing by 0).
\item \( \text{Integer} \)

These are the negative and positive counting numbers (..., -3, -2, -1, 0, 1, 2, 3, ...)
\item \( \text{Rational} \)

* This is the correct option!
\item \( \text{Whole} \)

These are the counting numbers with 0 (0, 1, 2, 3, ...)
\end{enumerate}

\textbf{General Comment:} First, you \textbf{NEED} to simplify the expression. This question simplifies to $-\frac{5}{19}$. 
 
 Be sure you look at the simplified fraction and not just the decimal expansion. Numbers such as 13, 17, and 19 provide \textbf{long but repeating/terminating decimal expansions!} 
 
 The only ways to *not* be a Real number are: dividing by 0 or taking the square root of a negative number. 
 
 Irrational numbers are more than just square root of 3: adding or subtracting values from square root of 3 is also irrational.
}
\litem{
Simplify the expression below into the form $a+bi$. Then, choose the intervals that $a$ and $b$ belong to.
\[ \frac{-9 + 66 i}{-3 - 8 i} \]The solution is \( -6.86  - 3.70 i \), which is option C.\begin{enumerate}[label=\Alph*.]
\item \( a \in [-7, -6.5] \text{ and } b \in [-270.5, -269.5] \)

 $-6.86  - 270.00 i$, which corresponds to forgetting to multiply the conjugate by the numerator.
\item \( a \in [7, 8] \text{ and } b \in [-3, -0.5] \)

 $7.60  - 1.73 i$, which corresponds to forgetting to multiply the conjugate by the numerator and not computing the conjugate correctly.
\item \( a \in [-7, -6.5] \text{ and } b \in [-5, -2.5] \)

* $-6.86  - 3.70 i$, which is the correct option.
\item \( a \in [-502, -500.5] \text{ and } b \in [-5, -2.5] \)

 $-501.00  - 3.70 i$, which corresponds to forgetting to multiply the conjugate by the numerator and using a plus instead of a minus in the denominator.
\item \( a \in [1.5, 4.5] \text{ and } b \in [-9, -8] \)

 $3.00  - 8.25 i$, which corresponds to just dividing the first term by the first term and the second by the second.
\end{enumerate}

\textbf{General Comment:} Multiply the numerator and denominator by the *conjugate* of the denominator, then simplify. For example, if we have $2+3i$, the conjugate is $2-3i$.
}
\litem{
Simplify the expression below and choose the interval the simplification is contained within.
\[ 4 - 7 \div 17 * 2 - (6 * 19) \]The solution is \( -110.824 \), which is option B.\begin{enumerate}[label=\Alph*.]
\item \( [-53.69, -53.25] \)

 -53.647, which corresponds to not distributing a negative correctly.
\item \( [-111.58, -110.57] \)

* -110.824, which is the correct option.
\item \( [-110.65, -109.21] \)

 -110.206, which corresponds to an Order of Operations error: not reading left-to-right for multiplication/division.
\item \( [117.44, 117.85] \)

 117.794, which corresponds to not distributing addition and subtraction correctly.
\item \( \text{None of the above} \)

 You may have gotten this by making an unanticipated error. If you got a value that is not any of the others, please let the coordinator know so they can help you figure out what happened.
\end{enumerate}

\textbf{General Comment:} While you may remember (or were taught) PEMDAS is done in order, it is actually done as P/E/MD/AS. When we are at MD or AS, we read left to right.
}
\litem{
Simplify the expression below into the form $a+bi$. Then, choose the intervals that $a$ and $b$ belong to.
\[ \frac{27 - 77 i}{6 - 5 i} \]The solution is \( 8.97  - 5.36 i \), which is option E.\begin{enumerate}[label=\Alph*.]
\item \( a \in [3.5, 5] \text{ and } b \in [15, 16.5] \)

 $4.50  + 15.40 i$, which corresponds to just dividing the first term by the first term and the second by the second.
\item \( a \in [546, 547.5] \text{ and } b \in [-6.5, -5] \)

 $547.00  - 5.36 i$, which corresponds to forgetting to multiply the conjugate by the numerator and using a plus instead of a minus in the denominator.
\item \( a \in [-4, -3] \text{ and } b \in [-11, -9.5] \)

 $-3.66  - 9.79 i$, which corresponds to forgetting to multiply the conjugate by the numerator and not computing the conjugate correctly.
\item \( a \in [7.5, 9.5] \text{ and } b \in [-328.5, -326.5] \)

 $8.97  - 327.00 i$, which corresponds to forgetting to multiply the conjugate by the numerator.
\item \( a \in [7.5, 9.5] \text{ and } b \in [-6.5, -5] \)

* $8.97  - 5.36 i$, which is the correct option.
\end{enumerate}

\textbf{General Comment:} Multiply the numerator and denominator by the *conjugate* of the denominator, then simplify. For example, if we have $2+3i$, the conjugate is $2-3i$.
}
\litem{
Choose the \textbf{smallest} set of Complex numbers that the number below belongs to.
\[ \sqrt{\frac{-567}{9}}+\sqrt{0}i \]The solution is \( \text{Pure Imaginary} \), which is option D.\begin{enumerate}[label=\Alph*.]
\item \( \text{Not a Complex Number} \)

This is not a number. The only non-Complex number we know is dividing by 0 as this is not a number!
\item \( \text{Nonreal Complex} \)

This is a Complex number $(a+bi)$ that is not Real (has $i$ as part of the number).
\item \( \text{Irrational} \)

These cannot be written as a fraction of Integers. Remember: $\pi$ is not an Integer!
\item \( \text{Pure Imaginary} \)

* This is the correct option!
\item \( \text{Rational} \)

These are numbers that can be written as fraction of Integers (e.g., -2/3 + 5)
\end{enumerate}

\textbf{General Comment:} Be sure to simplify $i^2 = -1$. This may remove the imaginary portion for your number. If you are having trouble, you may want to look at the \textit{Subgroups of the Real Numbers} section.
}
\litem{
Simplify the expression below and choose the interval the simplification is contained within.
\[ 1 - 19 \div 20 * 13 - (16 * 2) \]The solution is \( -43.350 \), which is option D.\begin{enumerate}[label=\Alph*.]
\item \( [-34.07, -25.07] \)

 -31.073, which corresponds to an Order of Operations error: not reading left-to-right for multiplication/division.
\item \( [30.93, 34.93] \)

 32.927, which corresponds to not distributing addition and subtraction correctly.
\item \( [-57.7, -48.7] \)

 -54.700, which corresponds to not distributing a negative correctly.
\item \( [-49.35, -36.35] \)

* -43.350, which is the correct option.
\item \( \text{None of the above} \)

 You may have gotten this by making an unanticipated error. If you got a value that is not any of the others, please let the coordinator know so they can help you figure out what happened.
\end{enumerate}

\textbf{General Comment:} While you may remember (or were taught) PEMDAS is done in order, it is actually done as P/E/MD/AS. When we are at MD or AS, we read left to right.
}
\litem{
Simplify the expression below into the form $a+bi$. Then, choose the intervals that $a$ and $b$ belong to.
\[ (7 - 3 i)(-5 - 2 i) \]The solution is \( -41 + i \), which is option C.\begin{enumerate}[label=\Alph*.]
\item \( a \in [-42, -36] \text{ and } b \in [-1.04, -0.04] \)

 $-41 - i$, which corresponds to adding a minus sign in both terms.
\item \( a \in [-33, -27] \text{ and } b \in [27.6, 29.73] \)

 $-29 + 29 i$, which corresponds to adding a minus sign in the second term.
\item \( a \in [-42, -36] \text{ and } b \in [0.39, 2.65] \)

* $-41 + i$, which is the correct option.
\item \( a \in [-37, -32] \text{ and } b \in [4.68, 7.26] \)

 $-35 + 6 i$, which corresponds to just multiplying the real terms to get the real part of the solution and the coefficients in the complex terms to get the complex part.
\item \( a \in [-33, -27] \text{ and } b \in [-29.61, -28.87] \)

 $-29 - 29 i$, which corresponds to adding a minus sign in the first term.
\end{enumerate}

\textbf{General Comment:} You can treat $i$ as a variable and distribute. Just remember that $i^2=-1$, so you can continue to reduce after you distribute.
}
\litem{
Simplify the expression below into the form $a+bi$. Then, choose the intervals that $a$ and $b$ belong to.
\[ (-8 - 10 i)(9 + 5 i) \]The solution is \( -22 - 130 i \), which is option A.\begin{enumerate}[label=\Alph*.]
\item \( a \in [-23, -17] \text{ and } b \in [-132, -123] \)

* $-22 - 130 i$, which is the correct option.
\item \( a \in [-75, -64] \text{ and } b \in [-55, -49] \)

 $-72 - 50 i$, which corresponds to just multiplying the real terms to get the real part of the solution and the coefficients in the complex terms to get the complex part.
\item \( a \in [-23, -17] \text{ and } b \in [127, 131] \)

 $-22 + 130 i$, which corresponds to adding a minus sign in both terms.
\item \( a \in [-123, -118] \text{ and } b \in [45, 51] \)

 $-122 + 50 i$, which corresponds to adding a minus sign in the first term.
\item \( a \in [-123, -118] \text{ and } b \in [-55, -49] \)

 $-122 - 50 i$, which corresponds to adding a minus sign in the second term.
\end{enumerate}

\textbf{General Comment:} You can treat $i$ as a variable and distribute. Just remember that $i^2=-1$, so you can continue to reduce after you distribute.
}
\end{enumerate}

\end{document}