\documentclass{extbook}[14pt]
\usepackage{multicol, enumerate, enumitem, hyperref, color, soul, setspace, parskip, fancyhdr, amssymb, amsthm, amsmath, bbm, latexsym, units, mathtools}
\everymath{\displaystyle}
\usepackage[headsep=0.5cm,headheight=0cm, left=1 in,right= 1 in,top= 1 in,bottom= 1 in]{geometry}
\usepackage{dashrule}  % Package to use the command below to create lines between items
\newcommand{\litem}[1]{\item #1

\rule{\textwidth}{0.4pt}}
\pagestyle{fancy}
\lhead{}
\chead{Answer Key for Progress Quiz 4 Version A}
\rhead{}
\lfoot{6286-1986}
\cfoot{}
\rfoot{Fall 2020}
\begin{document}
\textbf{This key should allow you to understand why you choose the option you did (beyond just getting a question right or wrong). \href{https://xronos.clas.ufl.edu/mac1105spring2020/courseDescriptionAndMisc/Exams/LearningFromResults}{More instructions on how to use this key can be found here}.}

\textbf{If you have a suggestion to make the keys better, \href{https://forms.gle/CZkbZmPbC9XALEE88}{please fill out the short survey here}.}

\textit{Note: This key is auto-generated and may contain issues and/or errors. The keys are reviewed after each exam to ensure grading is done accurately. If there are issues (like duplicate options), they are noted in the offline gradebook. The keys are a work-in-progress to give students as many resources to improve as possible.}

\rule{\textwidth}{0.4pt}

\begin{enumerate}\litem{
Choose the graph of the equation below.
\[ f(x) = \frac{1}{(x - 1)^2} + 1 \]
The solution is the graph below, which is option E.
\begin{center}
    \includegraphics[width=0.3\textwidth]{../Figures/rationalEquationToGraphEA.png}
\end{center}\begin{enumerate}[label=\Alph*.]
\begin{multicols}{2}
\item \includegraphics[width = 0.3\textwidth]{../Figures/rationalEquationToGraphAA.png}
\item \includegraphics[width = 0.3\textwidth]{../Figures/rationalEquationToGraphBA.png}
\item \includegraphics[width = 0.3\textwidth]{../Figures/rationalEquationToGraphCA.png}
\item \includegraphics[width = 0.3\textwidth]{../Figures/rationalEquationToGraphDA.png}
\end{multicols}\item None of the above.\end{enumerate}
\textbf{General Comment:} Remember that the general form of a basic rational equation is $ f(x) = \frac{a}{(x-h)^n} + k$, where $a$ is the leading coefficient (and in this case, we assume is either $1$ or $-1$), $n$ is the degree (in this case, either $1$ or $2$), and $(h, k)$ is the intersection of the asymptotes.
}
\litem{
Choose the equation of the function graphed below.

\begin{center}
    \includegraphics[width=0.5\textwidth]{../Figures/rationalGraphToEquationA.png}
\end{center}



The solution is \( \text{None of the above as it should be } f(x) = \frac{1}{x + 3} + 2 \), which is option E.\begin{enumerate}[label=\Alph*.]
\item \( f(x) = \frac{-1}{(x + 3)^2} - 1 \)

Corresponds to thinking the graph was a shifted version of $\frac{1}{x^2}$, using the general form $f(x) = \frac{a}{x-h}+k$, the opposite leading coefficient, AND not noticing the $y$-value was wrong.
\item \( f(x) = \frac{-1}{x + 3} - 1 \)

Corresponds to using the general form $f(x) = \frac{a}{x-h}+k$, the opposite leading coefficient AND not noticing the $y$-value was wrong.
\item \( f(x) = \frac{1}{x - 3} - 1 \)

The $x$- and $y$-value of the equation does not match the graph.
\item \( f(x) = \frac{1}{(x - 3)^2} - 1 \)

Corresponds to thinking the graph was a shifted version of $\frac{1}{x^2}$ not noticing the $y$-value was wrong.
\item \( \text{None of the above} \)

None of the equation options were the correct equation.
\end{enumerate}

\textbf{General Comment:} Remember that the general form of a basic rational equation is $ f(x) = \frac{a}{(x-h)^n} + k$, where $a$ is the leading coefficient (and in this case, we assume is either $1$ or $-1$), $n$ is the degree (in this case, either $1$ or $2$), and $(h, k)$ is the intersection of the asymptotes.
}
\litem{
Solve the rational equation below. Then, choose the interval(s) that the solution(s) belongs to.
\[ \frac{60}{-72x -60} + 1 = \frac{60}{-72x -60} \]
The solution is \( \text{all solutions are invalid or lead to complex values in the equation.} \), which is option E.\begin{enumerate}[label=\Alph*.]
\item \( x \in [0.4,2.8] \)

$x = 0.833$, which corresponds to not distributing the factor $-72x -60$ correctly when trying to eliminate the fraction.
\item \( x_1 \in [-1.5, -0.1] \text{ and } x_2 \in [-1.83,0.17] \)

$x = -0.833 \text{ and } x = -0.833$, which corresponds to getting the correct solution and believing there should be a second solution to the equation.
\item \( x_1 \in [-1.5, -0.1] \text{ and } x_2 \in [-0.17,3.83] \)

$x = -0.833 \text{ and } x = 0.833$, which corresponds to getting the correct solution and believing there should be a second solution to the equation.
\item \( x \in [-0.83,0.17] \)

$x = -0.833$, which corresponds to not checking if this value leads to dividing by 0 in the original equation and thus is not a valid solution.
\item \( \text{All solutions lead to invalid or complex values in the equation.} \)

*$x = -0.833$ leads to dividing by 0 in the original equation and thus is not a valid solution, which is the correct option.
\end{enumerate}

\textbf{General Comment:} Distractors are different based on the number of solutions. Remember that after solving, we need to make sure our solution does not make the original equation divide by zero!
}
\litem{
Determine the domain of the function below.
\[ f(x) = \frac{4}{20x^{2} +x -30} \]
The solution is \( \text{All Real numbers except } x = -1.250 \text{ and } x = 1.200. \), which is option B.\begin{enumerate}[label=\Alph*.]
\item \( \text{All Real numbers.} \)

This corresponds to thinking the denominator has complex roots or that rational functions have a domain of all Real numbers.
\item \( \text{All Real numbers except } x = a \text{ and } x = b, \text{ where } a \in [-4.25, -0.25] \text{ and } b \in [1.2, 2.2] \)

All Real numbers except $x = -1.250$ and $x = 1.200$, which is the correct option.
\item \( \text{All Real numbers except } x = a \text{ and } x = b, \text{ where } a \in [-20, -17] \text{ and } b \in [30, 32] \)

All Real numbers except $x = -20.000$ and $x = 30.000$, which corresponds to not factoring the denominator correctly.
\item \( \text{All Real numbers except } x = a, \text{ where } a \in [-20, -17] \)

All Real numbers except $x = -20.000$, which corresponds to removing a distractor value from the denominator.
\item \( \text{All Real numbers except } x = a, \text{ where } a \in [-4.25, -0.25] \)

All Real numbers except $x = -1.250$, which corresponds to removing only 1 value from the denominator.
\end{enumerate}

\textbf{General Comment:} Recall that dividing by zero is not a real number. Therefore the domain is all real numbers \textbf{except} those that make the denominator 0.
}
\litem{
Solve the rational equation below. Then, choose the interval(s) that the solution(s) belongs to.
\[ \frac{6x}{5x -5} + \frac{-4x^{2}}{20x^{2} -35 x + 15} = \frac{-7}{4x -3} \]
The solution is \( \text{There are two solutions: } x = 0.964 \text{ and } x = -1.814 \), which is option B.\begin{enumerate}[label=\Alph*.]
\item \( x \in [-1.83,-1.71] \)


\item \( x_1 \in [0.84, 1.05] \text{ and } x_2 \in [-4.81,0.19] \)

* $x = 0.964 \text{ and } x = -1.814$, which is the correct option.
\item \( x_1 \in [0.84, 1.05] \text{ and } x_2 \in [-1,4] \)


\item \( x \in [0.7,0.85] \)


\item \( \text{All solutions lead to invalid or complex values in the equation.} \)


\end{enumerate}

\textbf{General Comment:} Distractors are different based on the number of solutions. Remember that after solving, we need to make sure our solution does not make the original equation divide by zero!
}
\litem{
Choose the equation of the function graphed below.

\begin{center}
    \includegraphics[width=0.5\textwidth]{../Figures/rationalGraphToEquationCopyA.png}
\end{center}



The solution is \( \text{None of the above as it should be } f(x) = \frac{-1}{(x - 2)^2} + 3 \), which is option E.\begin{enumerate}[label=\Alph*.]
\item \( f(x) = \frac{-1}{x + 2} + 3 \)

Corresponds to thinking the graph was a shifted version of $\frac{1}{x}$.
\item \( f(x) = \frac{1}{(x - 2)^2} + 3 \)

Corresponds to using the general form $f(x) = \frac{a}{(x-h)^2}+k$ and the opposite leading coefficient.
\item \( f(x) = \frac{-1}{(x + 2)^2} + 3 \)

The $x$-value of the equation does not match the graph.
\item \( f(x) = \frac{1}{x - 2} + 3 \)

Corresponds to thinking the graph was a shifted version of $\frac{1}{x}$, using the general form $f(x) = \frac{a}{(x-h)^2}+k$, and the opposite leading coefficient.
\item \( \text{None of the above} \)

None of the equation options were the correct equation.
\end{enumerate}

\textbf{General Comment:} Remember that the general form of a basic rational equation is $ f(x) = \frac{a}{(x-h)^n} + k$, where $a$ is the leading coefficient (and in this case, we assume is either $1$ or $-1$), $n$ is the degree (in this case, either $1$ or $2$), and $(h, k)$ is the intersection of the asymptotes.
}
\litem{
Choose the graph of the equation below.
\[ f(x) = \frac{-1}{(x + 3)^2} + 3 \]
The solution is the graph below, which is option A.
\begin{center}
    \includegraphics[width=0.3\textwidth]{../Figures/rationalEquationToGraphCopyAA.png}
\end{center}\begin{enumerate}[label=\Alph*.]
\begin{multicols}{2}
\item \includegraphics[width = 0.3\textwidth]{../Figures/rationalEquationToGraphCopyAA.png}
\item \includegraphics[width = 0.3\textwidth]{../Figures/rationalEquationToGraphCopyBA.png}
\item \includegraphics[width = 0.3\textwidth]{../Figures/rationalEquationToGraphCopyCA.png}
\item \includegraphics[width = 0.3\textwidth]{../Figures/rationalEquationToGraphCopyDA.png}
\end{multicols}\item None of the above.\end{enumerate}
\textbf{General Comment:} Remember that the general form of a basic rational equation is $ f(x) = \frac{a}{(x-h)^n} + k$, where $a$ is the leading coefficient (and in this case, we assume is either $1$ or $-1$), $n$ is the degree (in this case, either $1$ or $2$), and $(h, k)$ is the intersection of the asymptotes.
}
\litem{
Solve the rational equation below. Then, choose the interval(s) that the solution(s) belongs to.
\[ \frac{9}{5x -8} + -5 = \frac{6}{-45x + 72} \]
The solution is \( x = 1.987 \), which is option E.\begin{enumerate}[label=\Alph*.]
\item \( x_1 \in [-1.43, -0.9] \text{ and } x_2 \in [0.99,4.99] \)

$x = -1.213 \text{ and } x = 1.987$, which corresponds to getting the correct solution and believing there should be a second solution to the equation.
\item \( x \in [-1.43,-0.9] \)

$x = -1.213$, which corresponds to not distributing the factor $5x -8$ correctly when trying to eliminate the fraction.
\item \( x_1 \in [1.62, 1.9] \text{ and } x_2 \in [0.99,4.99] \)

$x = 1.720 \text{ and } x = 1.987$, which corresponds to getting the correct solution and believing there should be a second solution to the equation.
\item \( \text{All solutions lead to invalid or complex values in the equation.} \)

This corresponds to thinking $x = 1.987$ leads to dividing by zero in the original equation, which it does not.
\item \( x \in [1.99,2.99] \)

* $x = 1.987$, which is the correct option.
\end{enumerate}

\textbf{General Comment:} Distractors are different based on the number of solutions. Remember that after solving, we need to make sure our solution does not make the original equation divide by zero!
}
\litem{
Solve the rational equation below. Then, choose the interval(s) that the solution(s) belongs to.
\[ \frac{2x}{3x + 5} + \frac{-7x^{2}}{12x^{2} +29 x + 15} = \frac{4}{4x + 3} \]
The solution is \( \text{There are two solutions: } x = -2.385 \text{ and } x = 8.385 \), which is option C.\begin{enumerate}[label=\Alph*.]
\item \( \text{All solutions lead to invalid or complex values in the equation.} \)


\item \( x \in [7.7,9.8] \)


\item \( x_1 \in [-3.1, -1.3] \text{ and } x_2 \in [6.38,9.38] \)

* $x = -2.385 \text{ and } x = 8.385$, which is the correct option.
\item \( x \in [-2.1,3.3] \)


\item \( x_1 \in [-3.1, -1.3] \text{ and } x_2 \in [-7.67,-0.67] \)


\end{enumerate}

\textbf{General Comment:} Distractors are different based on the number of solutions. Remember that after solving, we need to make sure our solution does not make the original equation divide by zero!
}
\litem{
Determine the domain of the function below.
\[ f(x) = \frac{6}{24x^{2} -48 x + 18} \]
The solution is \( \text{All Real numbers except } x = 0.500 \text{ and } x = 1.500. \), which is option B.\begin{enumerate}[label=\Alph*.]
\item \( \text{All Real numbers except } x = a, \text{ where } a \in [17, 18.3] \)

All Real numbers except $x = 18.000$, which corresponds to removing a distractor value from the denominator.
\item \( \text{All Real numbers except } x = a \text{ and } x = b, \text{ where } a \in [-0.1, 0.8] \text{ and } b \in [1, 2.2] \)

All Real numbers except $x = 0.500$ and $x = 1.500$, which is the correct option.
\item \( \text{All Real numbers except } x = a, \text{ where } a \in [-0.1, 0.8] \)

All Real numbers except $x = 0.500$, which corresponds to removing only 1 value from the denominator.
\item \( \text{All Real numbers except } x = a \text{ and } x = b, \text{ where } a \in [17, 18.3] \text{ and } b \in [23.8, 25.4] \)

All Real numbers except $x = 18.000$ and $x = 24.000$, which corresponds to not factoring the denominator correctly.
\item \( \text{All Real numbers.} \)

This corresponds to thinking the denominator has complex roots or that rational functions have a domain of all Real numbers.
\end{enumerate}

\textbf{General Comment:} Recall that dividing by zero is not a real number. Therefore the domain is all real numbers \textbf{except} those that make the denominator 0.
}
\end{enumerate}

\end{document}