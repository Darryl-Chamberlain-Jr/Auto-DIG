
\documentclass{article}[14pt]
\usepackage{multicol, enumerate, enumitem, hyperref, color, soul, setspace, parskip, fancyhdr, amssymb, amsthm, amsmath, bbm, latexsym, units, mathtools}
\everymath{\displaystyle}
\usepackage[headsep=0.5cm,headheight=0cm, left=1 in,right= 1 in,top= 1 in,bottom= 1 in]{geometry}
\pagestyle{fancy}
\lhead{}
\chead{Answer Key for Module\,6\,-\,Polynomial\,Functions Version C}
\rhead{}
\lfoot{debug}
\cfoot{}
\rfoot{}
\begin{document}
\textbf{This key should allow you to understand why you choose the option you did (beyond just getting a question right or wrong). \href{https://xronos.clas.ufl.edu/mac1105spring2020/courseDescriptionAndMisc/Exams/LearningFromResults}{More instructions on how to use this key can be found here}.}

\textbf{If you have a suggestion to make the keys better, \href{https://forms.gle/CZkbZmPbC9XALEE88}{please fill out the short survey here}.}

\textit{Note: This key is auto-generated and may contain issues and/or errors. The keys are reviewed after each exam to ensure grading is done accurately. If there are issues (like duplicate options), they are noted in the offline gradebook. The keys are a work-in-progress to give students as many resources to improve as possible.}

\rule{\textwidth}{0.4pt}

26. Which of the following equations \textit{could} be of the graph presented below?
\begin{center} \includegraphics[width=0.3\textwidth]{../Figures/polyGraphToFunctionC.png} \end{center} 

The solution is $ -13(x - 3)^{6} (x + 1)^{10} (x - 1)^{8} $ 

\begin{enumerate}[label=\Alph*.] 
\item $ -3(x - 3)^{4} (x + 1)^{7} (x - 1)^{7} $ 

 The factors $(x + 1)$ and $(x - 1)$ should both have even powers. 
\item $ 20(x - 3)^{4} (x + 1)^{6} (x - 1)^{5} $ 

 The factor $(x - 1)$ should have an even power and the leading coefficient should be the opposite sign. 
\item $ -13(x - 3)^{6} (x + 1)^{10} (x - 1)^{8} $ 

 * This is the correct option. 
\item $ -10(x - 3)^{6} (x + 1)^{8} (x - 1)^{11} $ 

 The factor $(x - 1)$ should have an even power. 
\item $ 13(x - 3)^{8} (x + 1)^{8} (x - 1)^{8} $ 

 This corresponds to the leading coefficient being the opposite value than it should be. 
\end{enumerate} 
 
General Comments: Draw the x-axis to determine which zeros are touching (and so have even multiplicity) or cross (and have odd multiplicity).

-----------------------------------------------

27. Describe the zero behavior of the zero $x = -9$ of the polynomial below.
$$ f(x) = 2(x - 2)^{10}(x + 2)^{6}(x - 9)^{9}(x + 9)^{8} $$ 

 
 The solution is  
 \begin{center} \includegraphics[width=0.3\textwidth]{../Figures/zeroBehaviorNegativeEvenC.png} \end{center}\begin{tabular}{|c|c|} 
\hline 
 & \tabularnewline 
 \textbf{A.} \includegraphics[width=0.3\textwidth]{../Figures/zeroBehaviorNegativeOddC.png} & \textbf{B.} \includegraphics[width=0.3\textwidth]{../Figures/zeroBehaviorNegativeEvenC.png} \tabularnewline 
\hline 
 & \tabularnewline 
 \textbf{C.} \includegraphics[width=0.3\textwidth]{../Figures/zeroBehaviorPositiveEvenC.png} & \textbf{D.} \includegraphics[width=0.3\textwidth]{../Figures/zeroBehaviorPositiveOddC.png} \tabularnewline 
\hline 
 E. None of the figures above. & \tabularnewline 
\hline 
 \end{tabular} 
 
\textbf{General Comments:} You will need to sketch the entire graph, then zoom in on the zero the question asks about.

-----------------------------------------------

28. Construct the lowest-degree polynomial given the zeros below. Then, choose the intervals that contain the coefficients of the polynomial in the form $x^3+bx^2+cx+d$.
$$ 2 - 5i \text{ and } -3 $$ 
The solution is $ x^{3} -1 x^{2} +17 x + 87 $ 

\begin{enumerate}[label=\Alph*.] 
\item $ b \in [0.2, 1.4], c \in [10, 19], \text{ and } d \in [-89, -83] $ 

 $x^{3} + x^{2} +17 x -87$, which corresponds to multiplying out $(x-(2 - 5i))(x-(2 + 5i))(x -3)$. 
\item $ b \in [0.2, 1.4], c \in [7, 14], \text{ and } d \in [10, 18] $ 

 $x^{3} + x^{2} +8 x + 15$, which corresponds to multiplying out $(x + 5)(x + 3)$. 
\item $ b \in [0.2, 1.4], c \in [-1, 4], \text{ and } d \in [-8, -3] $ 

 $x^{3} + x^{2} +x -6$, which corresponds to multiplying out $(x -2)(x + 3)$. 
\item $ b \in [-2, -0.7], c \in [10, 19], \text{ and } d \in [79, 89] $ 

 * $x^{3} -1 x^{2} +17 x + 87$, which is the correct option. 
\item $ \text{None of the above.} $ 

 This corresponds to making an unanticipated error or not understanding how to use nonreal complex numbers to create the lowest-degree polynomial. If you chose this and are not sure what you did wrong, please contact the coordinator for help. 
\end{enumerate} 
 
General Comments: Remember that the conjugate of $a+bi$ is $a-bi$. Since these zeros always come in pairs, we need to multiply out $(x-(2 - 5i))(x-(2 + 5i))(x-(-3))$.

-----------------------------------------------

29. Describe the end behavior of the polynomial below.
$$ f(x) = 6(x - 2)^{5}(x + 2)^{8}(x + 9)^{2}(x - 9)^{4} $$ 

 
 The solution is  
 \begin{center} \includegraphics[width=0.3\textwidth]{../Figures/endBehaviorPositiveOddC.png} \end{center}\begin{tabular}{|c|c|} 
\hline 
 & \tabularnewline 
 \textbf{A.} \includegraphics[width=0.3\textwidth]{../Figures/endBehaviorNegativeOddC.png} & \textbf{B.} \includegraphics[width=0.3\textwidth]{../Figures/endBehaviorNegativeEvenC.png} \tabularnewline 
\hline 
 & \tabularnewline 
 \textbf{C.} \includegraphics[width=0.3\textwidth]{../Figures/endBehaviorPositiveEvenC.png} & \textbf{D.} \includegraphics[width=0.3\textwidth]{../Figures/endBehaviorPositiveOddC.png} \tabularnewline 
\hline 
 E. None of the figures above. & \tabularnewline 
\hline 
 \end{tabular} 
 
\textbf{General Comments:} Remember that end behavior is determined by the leading coefficient AND whether the \textbf{sum} of the multiplicities is positive or negative.

-----------------------------------------------

30. Construct the lowest-degree polynomial given the zeros below. Then, choose the intervals that contain the coefficients of the polynomial in the form $ax^3+bx^2+cx+d$.
$$ \frac{5}{2}, \frac{-4}{5}, \text{ and } \frac{2}{5} $$ 
The solution is $ 50x^{3} -105 x^{2} -66 x + 40 $ 

\begin{enumerate}[label=\Alph*.] 
\item $ a \in [45, 62], b \in [101, 107], c \in [-71, -61], \text{ and } d \in [-41, -36] $ 

 $50x^{3} +105 x^{2} -66 x -40$, which corresponds to multiplying out $(2x + 5)(5x -4)(5x + 2)$. 
\item $ a \in [45, 62], b \in [-113, -96], c \in [-71, -61], \text{ and } d \in [-41, -36] $ 

 $50x^{3} -105 x^{2} -66 x -40$, which corresponds to multiplying everything correctly except the constant term. 
\item $ a \in [45, 62], b \in [62, 69], c \in [-136, -133], \text{ and } d \in [30, 46] $ 

 $50x^{3} +65 x^{2} -134 x + 40$, which corresponds to multiplying out $(2x + 2)(5x + 5)(5x -5)$. 
\item $ a \in [45, 62], b \in [141, 146], c \in [31, 45], \text{ and } d \in [-41, -36] $ 

 $50x^{3} +145 x^{2} +34 x -40$, which corresponds to multiplying out $(2x + 2)(5x -5)(5x -5)$. 
\item $ a \in [45, 62], b \in [-113, -96], c \in [-71, -61], \text{ and } d \in [30, 46] $ 

 * $50x^{3} -105 x^{2} -66 x + 40$, which is the correct option. 
\end{enumerate} 
 
General Comments: To construct the lowest-degree polynomial, you want to multiply out $(2x -5)(5x + 4)(5x -2)$

-----------------------------------------------


\end{document}

