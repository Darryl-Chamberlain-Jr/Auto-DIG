\documentclass{extbook}[14pt]
\usepackage{multicol, enumerate, enumitem, hyperref, color, soul, setspace, parskip, fancyhdr, amssymb, amsthm, amsmath, bbm, latexsym, units, mathtools}
\everymath{\displaystyle}
\usepackage[headsep=0.5cm,headheight=0cm, left=1 in,right= 1 in,top= 1 in,bottom= 1 in]{geometry}
\pagestyle{fancy}
\lhead{}
\chead{Answer Key for Module\,1\,-\,Real\,and\,Complex\,Numbers Version B}
\rhead{}
\lfoot{Summer\,C\,2020}
\cfoot{}
\rfoot{}
\begin{document}
\textbf{This key should allow you to understand why you choose the option you did (beyond just getting a question right or wrong). \href{https://xronos.clas.ufl.edu/mac1105spring2020/courseDescriptionAndMisc/Exams/LearningFromResults}{More instructions on how to use this key can be found here}.}

\textbf{If you have a suggestion to make the keys better, \href{https://forms.gle/CZkbZmPbC9XALEE88}{please fill out the short survey here}.}

\textit{Note: This key is auto-generated and may contain issues and/or errors. The keys are reviewed after each exam to ensure grading is done accurately. If there are issues (like duplicate options), they are noted in the offline gradebook. The keys are a work-in-progress to give students as many resources to improve as possible.}

\rule{\textwidth}{0.4pt}

1. Choose the \textbf{smallest} set of Real numbers that the number below belongs to.
\[ -\sqrt{\frac{-2548}{14}} \] 
The solution is $ \text{Not a Real number} $ 

\begin{enumerate}[label=\Alph*.] 
\item $ \text{Not a Real number} $ 

 * This is the correct option! 
\item $ \text{Whole} $ 

 These are the counting numbers with 0 (0, 1, 2, 3, ...) 
\item $ \text{Rational} $ 

 These are numbers that can be written as fraction of Integers (e.g., -2/3) 
\item $ \text{Irrational} $ 

 These cannot be written as a fraction of Integers. 
\item $ \text{Integer} $ 

 These are the negative and positive counting numbers (..., -3, -2, -1, 0, 1, 2, 3, ...) 
\end{enumerate} 
 
\textbf{General Comment:} First, you \textbf{NEED} to simplify the expression. This question simplifies to $-\sqrt{182} i$. 
 
 Be sure you look at the simplified fraction and not just the decimal expansion. Numbers such as 13, 17, and 19 provide \textbf{long but repeating/terminating decimal expansions!} 
 
 The only ways to *not* be a Real number are: dividing by 0 or taking the square root of a negative number. 
 
 Irrational numbers are more than just square root of 3: adding or subtracting values from square root of 3 is also irrational. 

-----------------------------------------------

2. Simplify the expression below into the form $a+bi$. Then, choose the intervals that $a$ and $b$ belong to.
\[ \frac{-18 + 33 i}{6 + 4 i} \] 
The solution is $ 0.46  + 5.19 i $ 

\begin{enumerate}[label=\Alph*.] 
\item $ a \in [-1, 3] \text{ and } b \in [3.4, 5.6] $ 

 * $0.46  + 5.19 i$, which is the correct option. 
\item $ a \in [-1, 3] \text{ and } b \in [268.2, 270.5] $ 

  $0.46  + 270.00 i$, which corresponds to forgetting to multiply the conjugate by the numerator. 
\item $ a \in [-4, 0] \text{ and } b \in [5.9, 9.6] $ 

  $-3.00  + 8.25 i$, which corresponds to just dividing the first term by the first term and the second by the second. 
\item $ a \in [-7, -4] \text{ and } b \in [-1.6, 4.3] $ 

  $-4.62  + 2.42 i$, which corresponds to forgetting to multiply the conjugate by the numerator and not computing the conjugate correctly. 
\item $ a \in [20, 26] \text{ and } b \in [3.4, 5.6] $ 

  $24.00  + 5.19 i$, which corresponds to forgetting to multiply the conjugate by the numerator and using a plus instead of a minus in the denominator. 
\end{enumerate} 
 
\textbf{General Comment:} Multiply the numerator and denominator by the *conjugate* of the denominator, then simplify. For example, if we have $2+3i$, the conjugate is $2-3i$. 

-----------------------------------------------

3. Simplify the expression below and choose the interval the simplification is contained within.
\[ 19 - 9^2 + 18 \div 15 * 20 \div 7 \] 
The solution is $ -58.571 $ 

\begin{enumerate}[label=\Alph*.] 
\item $ [-62.1, -59.9] $ 

  -61.991, which corresponds to an Order of Operations error: not reading left-to-right for multiplication/division. 
\item $ [102.7, 105.8] $ 

  103.429, which corresponds to an Order of Operations error: multiplying by negative before squaring. For example: $(-3)^2 \neq -3^2$ 
\item $ [98.5, 100.2] $ 

  100.009, which corresponds to two Order of Operations errors. 
\item $ [-61.2, -56.5] $ 

 * -58.571, this is the correct option 
\item $ \text{None of the above} $ 

  You may have gotten this by making an unanticipated error. If you got a value that is not any of the others, please let the coordinator know so they can help you figure out what happened. 
\end{enumerate} 
 
\textbf{General Comment:} While you may remember (or were taught) PEMDAS is done in order, it is actually done as P/E/MD/AS. When we are at MD or AS, we read left to right. 

-----------------------------------------------

4. Simplify the expression below into the form $a+bi$. Then, choose the intervals that $a$ and $b$ belong to.
\[ (-6 + 8 i)(-2 + 5 i) \] 
The solution is $ -28 - 46 i $ 

\begin{enumerate}[label=\Alph*.] 
\item $ a \in [4, 15] \text{ and } b \in [38, 45] $ 

  $12 + 40 i$, which corresponds to just multiplying the real terms to get the real part of the solution and the coefficients in the complex terms to get the complex part. 
\item $ a \in [-36, -24] \text{ and } b \in [42, 48] $ 

  $-28 + 46 i$, which corresponds to adding a minus sign in both terms. 
\item $ a \in [-36, -24] \text{ and } b \in [-47, -39] $ 

 * $-28 - 46 i$, which is the correct option. 
\item $ a \in [48, 61] \text{ and } b \in [10, 17] $ 

  $52 + 14 i$, which corresponds to adding a minus sign in the second term. 
\item $ a \in [48, 61] \text{ and } b \in [-22, -10] $ 

  $52 - 14 i$, which corresponds to adding a minus sign in the first term. 
\end{enumerate} 
 
\textbf{General Comment:} You can treat $i$ as a variable and distribute. Just remember that $i^2=-1$, so you can continue to reduce after you distribute. 

-----------------------------------------------

0. Choose the \textbf{smallest} set of Complex numbers that the number below belongs to.
\[ \sqrt{\frac{-825}{0}} i+\sqrt{112}i \] 
The solution is $ \text{Not a Complex Number} $ 

\begin{enumerate}[label=\Alph*.] 
\item $ \text{Nonreal Complex} $ 

 This is a Complex number $(a+bi)$ that is not Real (has $i$ as part of the number). 
\item $ \text{Not a Complex Number} $ 

 * This is the correct option! 
\item $ \text{Rational} $ 

 These are numbers that can be written as fraction of Integers (e.g., -2/3 + 5) 
\item $ \text{Irrational} $ 

 These cannot be written as a fraction of Integers. Remember: $\pi$ is not an Integer! 
\item $ \text{Pure Imaginary} $ 

 This is a Complex number $(a+bi)$ that \textbf{only} has an imaginary part like $2i$. 
\end{enumerate} 
 
\textbf{General Comment:} Be sure to simplify $i^2 = -1$. This may remove the imaginary portion for your number. If you are having trouble, you may want to look at the \textit{Subgroups of the Real Numbers} section. 

-----------------------------------------------


\end{document}

