\documentclass{extbook}[14pt]
\usepackage{multicol, enumerate, enumitem, hyperref, color, soul, setspace, parskip, fancyhdr, amssymb, amsthm, amsmath, bbm, latexsym, units, mathtools}
\everymath{\displaystyle}
\usepackage[headsep=0.5cm,headheight=0cm, left=1 in,right= 1 in,top= 1 in,bottom= 1 in]{geometry}
\usepackage{dashrule}  % Package to use the command below to create lines between items
\newcommand{\litem}[1]{\item #1

\rule{\textwidth}{0.4pt}}
\pagestyle{fancy}
\lhead{}
\chead{Answer Key for Progress Quiz 8 Version B}
\rhead{}
\lfoot{4553-3922}
\cfoot{}
\rfoot{Fall 2020}
\begin{document}
\textbf{This key should allow you to understand why you choose the option you did (beyond just getting a question right or wrong). \href{https://xronos.clas.ufl.edu/mac1105spring2020/courseDescriptionAndMisc/Exams/LearningFromResults}{More instructions on how to use this key can be found here}.}

\textbf{If you have a suggestion to make the keys better, \href{https://forms.gle/CZkbZmPbC9XALEE88}{please fill out the short survey here}.}

\textit{Note: This key is auto-generated and may contain issues and/or errors. The keys are reviewed after each exam to ensure grading is done accurately. If there are issues (like duplicate options), they are noted in the offline gradebook. The keys are a work-in-progress to give students as many resources to improve as possible.}

\rule{\textwidth}{0.4pt}

\begin{enumerate}\litem{
Simplify the expression below into the form $a+bi$. Then, choose the intervals that $a$ and $b$ belong to.
\[ \frac{-18 - 77 i}{-5 + i} \]

The solution is \( 0.50  + 15.50 i \), which is option A.\begin{enumerate}[label=\Alph*.]
\item \( a \in [0, 1.5] \text{ and } b \in [14.9, 15.7] \)

* $0.50  + 15.50 i$, which is the correct option.
\item \( a \in [2.5, 4.5] \text{ and } b \in [-77.55, -76.8] \)

 $3.60  - 77.00 i$, which corresponds to just dividing the first term by the first term and the second by the second.
\item \( a \in [0, 1.5] \text{ and } b \in [402.9, 403.2] \)

 $0.50  + 403.00 i$, which corresponds to forgetting to multiply the conjugate by the numerator.
\item \( a \in [11.5, 13.5] \text{ and } b \in [14.9, 15.7] \)

 $13.00  + 15.50 i$, which corresponds to forgetting to multiply the conjugate by the numerator and using a plus instead of a minus in the denominator.
\item \( a \in [5, 7] \text{ and } b \in [14, 14.55] \)

 $6.42  + 14.12 i$, which corresponds to forgetting to multiply the conjugate by the numerator and not computing the conjugate correctly.
\end{enumerate}

\textbf{General Comment:} Multiply the numerator and denominator by the *conjugate* of the denominator, then simplify. For example, if we have $2+3i$, the conjugate is $2-3i$.
}
\litem{
Simplify the expression below into the form $a+bi$. Then, choose the intervals that $a$ and $b$ belong to.
\[ \frac{45 - 88 i}{-4 + 7 i} \]

The solution is \( -12.25  + 0.57 i \), which is option B.\begin{enumerate}[label=\Alph*.]
\item \( a \in [-11.5, -10] \text{ and } b \in [-13.5, -11.5] \)

 $-11.25  - 12.57 i$, which corresponds to just dividing the first term by the first term and the second by the second.
\item \( a \in [-13.5, -12] \text{ and } b \in [-0.5, 2] \)

* $-12.25  + 0.57 i$, which is the correct option.
\item \( a \in [-797, -795] \text{ and } b \in [-0.5, 2] \)

 $-796.00  + 0.57 i$, which corresponds to forgetting to multiply the conjugate by the numerator and using a plus instead of a minus in the denominator.
\item \( a \in [5.5, 8] \text{ and } b \in [9, 11] \)

 $6.71  + 10.26 i$, which corresponds to forgetting to multiply the conjugate by the numerator and not computing the conjugate correctly.
\item \( a \in [-13.5, -12] \text{ and } b \in [36.5, 38.5] \)

 $-12.25  + 37.00 i$, which corresponds to forgetting to multiply the conjugate by the numerator.
\end{enumerate}

\textbf{General Comment:} Multiply the numerator and denominator by the *conjugate* of the denominator, then simplify. For example, if we have $2+3i$, the conjugate is $2-3i$.
}
\litem{
Simplify the expression below into the form $a+bi$. Then, choose the intervals that $a$ and $b$ belong to.
\[ (4 - 6 i)(3 + 7 i) \]

The solution is \( 54 + 10 i \), which is option D.\begin{enumerate}[label=\Alph*.]
\item \( a \in [-31, -28] \text{ and } b \in [-54, -44] \)

 $-30 - 46 i$, which corresponds to adding a minus sign in the second term.
\item \( a \in [6, 15] \text{ and } b \in [-44, -41] \)

 $12 - 42 i$, which corresponds to just multiplying the real terms to get the real part of the solution and the coefficients in the complex terms to get the complex part.
\item \( a \in [-31, -28] \text{ and } b \in [42, 48] \)

 $-30 + 46 i$, which corresponds to adding a minus sign in the first term.
\item \( a \in [48, 57] \text{ and } b \in [4, 13] \)

* $54 + 10 i$, which is the correct option.
\item \( a \in [48, 57] \text{ and } b \in [-10, -9] \)

 $54 - 10 i$, which corresponds to adding a minus sign in both terms.
\end{enumerate}

\textbf{General Comment:} You can treat $i$ as a variable and distribute. Just remember that $i^2=-1$, so you can continue to reduce after you distribute.
}
\litem{
Choose the \textbf{smallest} set of Real numbers that the number below belongs to.
\[ -\sqrt{\frac{-525}{5}} \]

The solution is \( \text{Not a Real number} \), which is option C.\begin{enumerate}[label=\Alph*.]
\item \( \text{Whole} \)

These are the counting numbers with 0 (0, 1, 2, 3, ...)
\item \( \text{Integer} \)

These are the negative and positive counting numbers (..., -3, -2, -1, 0, 1, 2, 3, ...)
\item \( \text{Not a Real number} \)

* This is the correct option!
\item \( \text{Rational} \)

These are numbers that can be written as fraction of Integers (e.g., -2/3)
\item \( \text{Irrational} \)

These cannot be written as a fraction of Integers.
\end{enumerate}

\textbf{General Comment:} First, you \textbf{NEED} to simplify the expression. This question simplifies to $-\sqrt{105} i$. 
 
 Be sure you look at the simplified fraction and not just the decimal expansion. Numbers such as 13, 17, and 19 provide \textbf{long but repeating/terminating decimal expansions!} 
 
 The only ways to *not* be a Real number are: dividing by 0 or taking the square root of a negative number. 
 
 Irrational numbers are more than just square root of 3: adding or subtracting values from square root of 3 is also irrational.
}
\litem{
Simplify the expression below and choose the interval the simplification is contained within.
\[ 15 - 14^2 + 5 \div 17 * 18 \div 8 \]

The solution is \( -180.338 \), which is option B.\begin{enumerate}[label=\Alph*.]
\item \( [211.12, 211.99] \)

 211.662, which corresponds to an Order of Operations error: multiplying by negative before squaring. For example: $(-3)^2 \neq -3^2$
\item \( [-180.44, -180.08] \)

* -180.338, this is the correct option
\item \( [-181.33, -180.83] \)

 -180.998, which corresponds to an Order of Operations error: not reading left-to-right for multiplication/division.
\item \( [210.75, 211.12] \)

 211.002, which corresponds to two Order of Operations errors.
\item \( \text{None of the above} \)

 You may have gotten this by making an unanticipated error. If you got a value that is not any of the others, please let the coordinator know so they can help you figure out what happened.
\end{enumerate}

\textbf{General Comment:} While you may remember (or were taught) PEMDAS is done in order, it is actually done as P/E/MD/AS. When we are at MD or AS, we read left to right.
}
\litem{
Simplify the expression below and choose the interval the simplification is contained within.
\[ 14 - 7 \div 20 * 6 - (10 * 12) \]

The solution is \( -108.100 \), which is option C.\begin{enumerate}[label=\Alph*.]
\item \( [20.1, 24.9] \)

 22.800, which corresponds to not distributing a negative correctly.
\item \( [133, 136.2] \)

 133.942, which corresponds to not distributing addition and subtraction correctly.
\item \( [-109.2, -108] \)

* -108.100, which is the correct option.
\item \( [-107.1, -103.2] \)

 -106.058, which corresponds to an Order of Operations error: not reading left-to-right for multiplication/division.
\item \( \text{None of the above} \)

 You may have gotten this by making an unanticipated error. If you got a value that is not any of the others, please let the coordinator know so they can help you figure out what happened.
\end{enumerate}

\textbf{General Comment:} While you may remember (or were taught) PEMDAS is done in order, it is actually done as P/E/MD/AS. When we are at MD or AS, we read left to right.
}
\litem{
Choose the \textbf{smallest} set of Complex numbers that the number below belongs to.
\[ \sqrt{\frac{-2496}{0}}+\sqrt{60} \]

The solution is \( \text{Not a Complex Number} \), which is option C.\begin{enumerate}[label=\Alph*.]
\item \( \text{Irrational} \)

These cannot be written as a fraction of Integers. Remember: $\pi$ is not an Integer!
\item \( \text{Pure Imaginary} \)

This is a Complex number $(a+bi)$ that \textbf{only} has an imaginary part like $2i$.
\item \( \text{Not a Complex Number} \)

* This is the correct option!
\item \( \text{Rational} \)

These are numbers that can be written as fraction of Integers (e.g., -2/3 + 5)
\item \( \text{Nonreal Complex} \)

This is a Complex number $(a+bi)$ that is not Real (has $i$ as part of the number).
\end{enumerate}

\textbf{General Comment:} Be sure to simplify $i^2 = -1$. This may remove the imaginary portion for your number. If you are having trouble, you may want to look at the \textit{Subgroups of the Real Numbers} section.
}
\litem{
Choose the \textbf{smallest} set of Complex numbers that the number below belongs to.
\[ \frac{18}{-18}+4i^2 \]

The solution is \( \text{Rational} \), which is option A.\begin{enumerate}[label=\Alph*.]
\item \( \text{Rational} \)

* This is the correct option!
\item \( \text{Nonreal Complex} \)

This is a Complex number $(a+bi)$ that is not Real (has $i$ as part of the number).
\item \( \text{Not a Complex Number} \)

This is not a number. The only non-Complex number we know is dividing by 0 as this is not a number!
\item \( \text{Pure Imaginary} \)

This is a Complex number $(a+bi)$ that \textbf{only} has an imaginary part like $2i$.
\item \( \text{Irrational} \)

These cannot be written as a fraction of Integers. Remember: $\pi$ is not an Integer!
\end{enumerate}

\textbf{General Comment:} Be sure to simplify $i^2 = -1$. This may remove the imaginary portion for your number. If you are having trouble, you may want to look at the \textit{Subgroups of the Real Numbers} section.
}
\litem{
Simplify the expression below into the form $a+bi$. Then, choose the intervals that $a$ and $b$ belong to.
\[ (3 - 9 i)(-10 + 5 i) \]

The solution is \( 15 + 105 i \), which is option D.\begin{enumerate}[label=\Alph*.]
\item \( a \in [-75, -73] \text{ and } b \in [-82, -74] \)

 $-75 - 75 i$, which corresponds to adding a minus sign in the first term.
\item \( a \in [14, 17] \text{ and } b \in [-109, -98] \)

 $15 - 105 i$, which corresponds to adding a minus sign in both terms.
\item \( a \in [-75, -73] \text{ and } b \in [73, 76] \)

 $-75 + 75 i$, which corresponds to adding a minus sign in the second term.
\item \( a \in [14, 17] \text{ and } b \in [104, 107] \)

* $15 + 105 i$, which is the correct option.
\item \( a \in [-34, -27] \text{ and } b \in [-47, -43] \)

 $-30 - 45 i$, which corresponds to just multiplying the real terms to get the real part of the solution and the coefficients in the complex terms to get the complex part.
\end{enumerate}

\textbf{General Comment:} You can treat $i$ as a variable and distribute. Just remember that $i^2=-1$, so you can continue to reduce after you distribute.
}
\litem{
Choose the \textbf{smallest} set of Real numbers that the number below belongs to.
\[ -\sqrt{\frac{1134}{9}} \]

The solution is \( \text{Irrational} \), which is option B.\begin{enumerate}[label=\Alph*.]
\item \( \text{Not a Real number} \)

These are Nonreal Complex numbers \textbf{OR} things that are not numbers (e.g., dividing by 0).
\item \( \text{Irrational} \)

* This is the correct option!
\item \( \text{Rational} \)

These are numbers that can be written as fraction of Integers (e.g., -2/3)
\item \( \text{Whole} \)

These are the counting numbers with 0 (0, 1, 2, 3, ...)
\item \( \text{Integer} \)

These are the negative and positive counting numbers (..., -3, -2, -1, 0, 1, 2, 3, ...)
\end{enumerate}

\textbf{General Comment:} First, you \textbf{NEED} to simplify the expression. This question simplifies to $-\sqrt{126}$. 
 
 Be sure you look at the simplified fraction and not just the decimal expansion. Numbers such as 13, 17, and 19 provide \textbf{long but repeating/terminating decimal expansions!} 
 
 The only ways to *not* be a Real number are: dividing by 0 or taking the square root of a negative number. 
 
 Irrational numbers are more than just square root of 3: adding or subtracting values from square root of 3 is also irrational.
}
\end{enumerate}

\end{document}