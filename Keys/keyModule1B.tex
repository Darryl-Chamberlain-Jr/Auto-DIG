\documentclass{extbook}[14pt]
\usepackage{multicol, enumerate, enumitem, hyperref, color, soul, setspace, parskip, fancyhdr, amssymb, amsthm, amsmath, bbm, latexsym, units, mathtools}
\everymath{\displaystyle}
\usepackage[headsep=0.5cm,headheight=0cm, left=1 in,right= 1 in,top= 1 in,bottom= 1 in]{geometry}
\usepackage{dashrule}  % Package to use the command below to create lines between items
\newcommand{\litem}[1]{\item #1

\rule{\textwidth}{0.4pt}}
\pagestyle{fancy}
\lhead{}
\chead{Answer Key for Progress Quiz 4 Version B}
\rhead{}
\lfoot{6286-1986}
\cfoot{}
\rfoot{Fall 2020}
\begin{document}
\textbf{This key should allow you to understand why you choose the option you did (beyond just getting a question right or wrong). \href{https://xronos.clas.ufl.edu/mac1105spring2020/courseDescriptionAndMisc/Exams/LearningFromResults}{More instructions on how to use this key can be found here}.}

\textbf{If you have a suggestion to make the keys better, \href{https://forms.gle/CZkbZmPbC9XALEE88}{please fill out the short survey here}.}

\textit{Note: This key is auto-generated and may contain issues and/or errors. The keys are reviewed after each exam to ensure grading is done accurately. If there are issues (like duplicate options), they are noted in the offline gradebook. The keys are a work-in-progress to give students as many resources to improve as possible.}

\rule{\textwidth}{0.4pt}

\begin{enumerate}\litem{
Simplify the expression below into the form $a+bi$. Then, choose the intervals that $a$ and $b$ belong to.
\[ (-3 + 7 i)(10 - 6 i) \]
The solution is \( 12 + 88 i \), which is option C.\begin{enumerate}[label=\Alph*.]
\item \( a \in [-74, -69] \text{ and } b \in [-58, -48] \)

 $-72 - 52 i$, which corresponds to adding a minus sign in the first term.
\item \( a \in [-74, -69] \text{ and } b \in [51, 53] \)

 $-72 + 52 i$, which corresponds to adding a minus sign in the second term.
\item \( a \in [9, 18] \text{ and } b \in [80, 92] \)

* $12 + 88 i$, which is the correct option.
\item \( a \in [-31, -24] \text{ and } b \in [-46, -39] \)

 $-30 - 42 i$, which corresponds to just multiplying the real terms to get the real part of the solution and the coefficients in the complex terms to get the complex part.
\item \( a \in [9, 18] \text{ and } b \in [-90, -84] \)

 $12 - 88 i$, which corresponds to adding a minus sign in both terms.
\end{enumerate}

\textbf{General Comment:} You can treat $i$ as a variable and distribute. Just remember that $i^2=-1$, so you can continue to reduce after you distribute.
}
\litem{
Choose the \textbf{smallest} set of Complex numbers that the number below belongs to.
\[ \frac{\sqrt{208}}{17}+8i^2 \]
The solution is \( \text{Irrational} \), which is option A.\begin{enumerate}[label=\Alph*.]
\item \( \text{Irrational} \)

* This is the correct option!
\item \( \text{Pure Imaginary} \)

This is a Complex number $(a+bi)$ that \textbf{only} has an imaginary part like $2i$.
\item \( \text{Not a Complex Number} \)

This is not a number. The only non-Complex number we know is dividing by 0 as this is not a number!
\item \( \text{Rational} \)

These are numbers that can be written as fraction of Integers (e.g., -2/3 + 5)
\item \( \text{Nonreal Complex} \)

This is a Complex number $(a+bi)$ that is not Real (has $i$ as part of the number).
\end{enumerate}

\textbf{General Comment:} Be sure to simplify $i^2 = -1$. This may remove the imaginary portion for your number. If you are having trouble, you may want to look at the \textit{Subgroups of the Real Numbers} section.
}
\litem{
Simplify the expression below into the form $a+bi$. Then, choose the intervals that $a$ and $b$ belong to.
\[ \frac{-9 - 22 i}{-4 + 8 i} \]
The solution is \( -1.75  + 2.00 i \), which is option E.\begin{enumerate}[label=\Alph*.]
\item \( a \in [1.89, 2.48] \text{ and } b \in [-3.5, -2] \)

 $2.25  - 2.75 i$, which corresponds to just dividing the first term by the first term and the second by the second.
\item \( a \in [-1.86, -1.46] \text{ and } b \in [159.5, 160.5] \)

 $-1.75  + 160.00 i$, which corresponds to forgetting to multiply the conjugate by the numerator.
\item \( a \in [-140.08, -139.66] \text{ and } b \in [1.5, 3] \)

 $-140.00  + 2.00 i$, which corresponds to forgetting to multiply the conjugate by the numerator and using a plus instead of a minus in the denominator.
\item \( a \in [2.6, 2.89] \text{ and } b \in [0, 1] \)

 $2.65  + 0.20 i$, which corresponds to forgetting to multiply the conjugate by the numerator and not computing the conjugate correctly.
\item \( a \in [-1.86, -1.46] \text{ and } b \in [1.5, 3] \)

* $-1.75  + 2.00 i$, which is the correct option.
\end{enumerate}

\textbf{General Comment:} Multiply the numerator and denominator by the *conjugate* of the denominator, then simplify. For example, if we have $2+3i$, the conjugate is $2-3i$.
}
\litem{
Choose the \textbf{smallest} set of Real numbers that the number below belongs to.
\[ \sqrt{\frac{576}{25}} \]
The solution is \( \text{Rational} \), which is option B.\begin{enumerate}[label=\Alph*.]
\item \( \text{Integer} \)

These are the negative and positive counting numbers (..., -3, -2, -1, 0, 1, 2, 3, ...)
\item \( \text{Rational} \)

* This is the correct option!
\item \( \text{Irrational} \)

These cannot be written as a fraction of Integers.
\item \( \text{Whole} \)

These are the counting numbers with 0 (0, 1, 2, 3, ...)
\item \( \text{Not a Real number} \)

These are Nonreal Complex numbers \textbf{OR} things that are not numbers (e.g., dividing by 0).
\end{enumerate}

\textbf{General Comment:} First, you \textbf{NEED} to simplify the expression. This question simplifies to $\frac{24}{5}$. 
 
 Be sure you look at the simplified fraction and not just the decimal expansion. Numbers such as 13, 17, and 19 provide \textbf{long but repeating/terminating decimal expansions!} 
 
 The only ways to *not* be a Real number are: dividing by 0 or taking the square root of a negative number. 
 
 Irrational numbers are more than just square root of 3: adding or subtracting values from square root of 3 is also irrational.
}
\litem{
Simplify the expression below into the form $a+bi$. Then, choose the intervals that $a$ and $b$ belong to.
\[ \frac{-72 - 66 i}{-1 + 3 i} \]
The solution is \( -12.60  + 28.20 i \), which is option E.\begin{enumerate}[label=\Alph*.]
\item \( a \in [-13.5, -11] \text{ and } b \in [281, 283.5] \)

 $-12.60  + 282.00 i$, which corresponds to forgetting to multiply the conjugate by the numerator.
\item \( a \in [25.5, 28.5] \text{ and } b \in [-16, -14.5] \)

 $27.00  - 15.00 i$, which corresponds to forgetting to multiply the conjugate by the numerator and not computing the conjugate correctly.
\item \( a \in [-128, -125.5] \text{ and } b \in [27.5, 29] \)

 $-126.00  + 28.20 i$, which corresponds to forgetting to multiply the conjugate by the numerator and using a plus instead of a minus in the denominator.
\item \( a \in [71, 72.5] \text{ and } b \in [-23.5, -21.5] \)

 $72.00  - 22.00 i$, which corresponds to just dividing the first term by the first term and the second by the second.
\item \( a \in [-13.5, -11] \text{ and } b \in [27.5, 29] \)

* $-12.60  + 28.20 i$, which is the correct option.
\end{enumerate}

\textbf{General Comment:} Multiply the numerator and denominator by the *conjugate* of the denominator, then simplify. For example, if we have $2+3i$, the conjugate is $2-3i$.
}
\litem{
Simplify the expression below and choose the interval the simplification is contained within.
\[ 9 - 15 \div 19 * 20 - (12 * 5) \]
The solution is \( -66.789 \), which is option D.\begin{enumerate}[label=\Alph*.]
\item \( [-93.95, -90.95] \)

 -93.947, which corresponds to not distributing a negative correctly.
\item \( [66.96, 70.96] \)

 68.961, which corresponds to not distributing addition and subtraction correctly.
\item \( [-52.04, -46.04] \)

 -51.039, which corresponds to an Order of Operations error: not reading left-to-right for multiplication/division.
\item \( [-68.79, -63.79] \)

* -66.789, which is the correct option.
\item \( \text{None of the above} \)

 You may have gotten this by making an unanticipated error. If you got a value that is not any of the others, please let the coordinator know so they can help you figure out what happened.
\end{enumerate}

\textbf{General Comment:} While you may remember (or were taught) PEMDAS is done in order, it is actually done as P/E/MD/AS. When we are at MD or AS, we read left to right.
}
\litem{
Choose the \textbf{smallest} set of Real numbers that the number below belongs to.
\[ \sqrt{\frac{22500}{36}} \]
The solution is \( \text{Whole} \), which is option C.\begin{enumerate}[label=\Alph*.]
\item \( \text{Not a Real number} \)

These are Nonreal Complex numbers \textbf{OR} things that are not numbers (e.g., dividing by 0).
\item \( \text{Rational} \)

These are numbers that can be written as fraction of Integers (e.g., -2/3)
\item \( \text{Whole} \)

* This is the correct option!
\item \( \text{Irrational} \)

These cannot be written as a fraction of Integers.
\item \( \text{Integer} \)

These are the negative and positive counting numbers (..., -3, -2, -1, 0, 1, 2, 3, ...)
\end{enumerate}

\textbf{General Comment:} First, you \textbf{NEED} to simplify the expression. This question simplifies to $150$. 
 
 Be sure you look at the simplified fraction and not just the decimal expansion. Numbers such as 13, 17, and 19 provide \textbf{long but repeating/terminating decimal expansions!} 
 
 The only ways to *not* be a Real number are: dividing by 0 or taking the square root of a negative number. 
 
 Irrational numbers are more than just square root of 3: adding or subtracting values from square root of 3 is also irrational.
}
\litem{
Simplify the expression below and choose the interval the simplification is contained within.
\[ 17 - 9^2 + 19 \div 12 * 10 \div 2 \]
The solution is \( -56.083 \), which is option B.\begin{enumerate}[label=\Alph*.]
\item \( [104.92, 111.92] \)

 105.917, which corresponds to an Order of Operations error: multiplying by negative before squaring. For example: $(-3)^2 \neq -3^2$
\item \( [-58.08, -55.08] \)

* -56.083, this is the correct option
\item \( [-64.92, -62.92] \)

 -63.921, which corresponds to an Order of Operations error: not reading left-to-right for multiplication/division.
\item \( [95.08, 103.08] \)

 98.079, which corresponds to two Order of Operations errors.
\item \( \text{None of the above} \)

 You may have gotten this by making an unanticipated error. If you got a value that is not any of the others, please let the coordinator know so they can help you figure out what happened.
\end{enumerate}

\textbf{General Comment:} While you may remember (or were taught) PEMDAS is done in order, it is actually done as P/E/MD/AS. When we are at MD or AS, we read left to right.
}
\litem{
Choose the \textbf{smallest} set of Complex numbers that the number below belongs to.
\[ -\sqrt{\frac{49}{121}} + 25i^2 \]
The solution is \( \text{Rational} \), which is option C.\begin{enumerate}[label=\Alph*.]
\item \( \text{Pure Imaginary} \)

This is a Complex number $(a+bi)$ that \textbf{only} has an imaginary part like $2i$.
\item \( \text{Irrational} \)

These cannot be written as a fraction of Integers. Remember: $\pi$ is not an Integer!
\item \( \text{Rational} \)

* This is the correct option!
\item \( \text{Nonreal Complex} \)

This is a Complex number $(a+bi)$ that is not Real (has $i$ as part of the number).
\item \( \text{Not a Complex Number} \)

This is not a number. The only non-Complex number we know is dividing by 0 as this is not a number!
\end{enumerate}

\textbf{General Comment:} Be sure to simplify $i^2 = -1$. This may remove the imaginary portion for your number. If you are having trouble, you may want to look at the \textit{Subgroups of the Real Numbers} section.
}
\litem{
Simplify the expression below into the form $a+bi$. Then, choose the intervals that $a$ and $b$ belong to.
\[ (8 + 7 i)(3 + 5 i) \]
The solution is \( -11 + 61 i \), which is option D.\begin{enumerate}[label=\Alph*.]
\item \( a \in [24, 29] \text{ and } b \in [34, 37] \)

 $24 + 35 i$, which corresponds to just multiplying the real terms to get the real part of the solution and the coefficients in the complex terms to get the complex part.
\item \( a \in [54, 62] \text{ and } b \in [17, 22] \)

 $59 + 19 i$, which corresponds to adding a minus sign in the first term.
\item \( a \in [-16, -9] \text{ and } b \in [-62, -54] \)

 $-11 - 61 i$, which corresponds to adding a minus sign in both terms.
\item \( a \in [-16, -9] \text{ and } b \in [60, 62] \)

* $-11 + 61 i$, which is the correct option.
\item \( a \in [54, 62] \text{ and } b \in [-22, -12] \)

 $59 - 19 i$, which corresponds to adding a minus sign in the second term.
\end{enumerate}

\textbf{General Comment:} You can treat $i$ as a variable and distribute. Just remember that $i^2=-1$, so you can continue to reduce after you distribute.
}
\end{enumerate}

\end{document}