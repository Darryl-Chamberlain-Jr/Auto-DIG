\documentclass{extbook}[14pt]
\usepackage{multicol, enumerate, enumitem, hyperref, color, soul, setspace, parskip, fancyhdr, amssymb, amsthm, amsmath, latexsym, units, mathtools}
\everymath{\displaystyle}
\usepackage[headsep=0.5cm,headheight=0cm, left=1 in,right= 1 in,top= 1 in,bottom= 1 in]{geometry}
\usepackage{dashrule}  % Package to use the command below to create lines between items
\newcommand{\litem}[1]{\item #1

\rule{\textwidth}{0.4pt}}
\pagestyle{fancy}
\lhead{}
\chead{Answer Key for Progress Quiz 6 Version B}
\rhead{}
\lfoot{9689-6866}
\cfoot{}
\rfoot{Spring 2021}
\begin{document}
\textbf{This key should allow you to understand why you choose the option you did (beyond just getting a question right or wrong). \href{https://xronos.clas.ufl.edu/mac1105spring2020/courseDescriptionAndMisc/Exams/LearningFromResults}{More instructions on how to use this key can be found here}.}

\textbf{If you have a suggestion to make the keys better, \href{https://forms.gle/CZkbZmPbC9XALEE88}{please fill out the short survey here}.}

\textit{Note: This key is auto-generated and may contain issues and/or errors. The keys are reviewed after each exam to ensure grading is done accurately. If there are issues (like duplicate options), they are noted in the offline gradebook. The keys are a work-in-progress to give students as many resources to improve as possible.}

\rule{\textwidth}{0.4pt}

\begin{enumerate}\litem{
Simplify the expression below into the form $a+bi$. Then, choose the intervals that $a$ and $b$ belong to.
\[ \frac{72 - 44 i}{2 + 3 i} \]The solution is \( 0.92  - 23.38 i \), which is option C.\begin{enumerate}[label=\Alph*.]
\item \( a \in [34.5, 37] \text{ and } b \in [-16, -14] \)

 $36.00  - 14.67 i$, which corresponds to just dividing the first term by the first term and the second by the second.
\item \( a \in [0, 1.5] \text{ and } b \in [-304.5, -303] \)

 $0.92  - 304.00 i$, which corresponds to forgetting to multiply the conjugate by the numerator.
\item \( a \in [0, 1.5] \text{ and } b \in [-24, -23] \)

* $0.92  - 23.38 i$, which is the correct option.
\item \( a \in [11.5, 12.5] \text{ and } b \in [-24, -23] \)

 $12.00  - 23.38 i$, which corresponds to forgetting to multiply the conjugate by the numerator and using a plus instead of a minus in the denominator.
\item \( a \in [21, 22] \text{ and } b \in [9.5, 10.5] \)

 $21.23  + 9.85 i$, which corresponds to forgetting to multiply the conjugate by the numerator and not computing the conjugate correctly.
\end{enumerate}

\textbf{General Comment:} Multiply the numerator and denominator by the *conjugate* of the denominator, then simplify. For example, if we have $2+3i$, the conjugate is $2-3i$.
}
\litem{
Simplify the expression below and choose the interval the simplification is contained within.
\[ 17 - 10 \div 19 * 5 - (2 * 16) \]The solution is \( -17.632 \), which is option C.\begin{enumerate}[label=\Alph*.]
\item \( [47.5, 49.7] \)

 48.895, which corresponds to not distributing addition and subtraction correctly.
\item \( [196, 200.9] \)

 197.895, which corresponds to not distributing a negative correctly.
\item \( [-17.9, -17.4] \)

* -17.632, which is the correct option.
\item \( [-16.4, -12.7] \)

 -15.105, which corresponds to an Order of Operations error: not reading left-to-right for multiplication/division.
\item \( \text{None of the above} \)

 You may have gotten this by making an unanticipated error. If you got a value that is not any of the others, please let the coordinator know so they can help you figure out what happened.
\end{enumerate}

\textbf{General Comment:} While you may remember (or were taught) PEMDAS is done in order, it is actually done as P/E/MD/AS. When we are at MD or AS, we read left to right.
}
\litem{
Simplify the expression below into the form $a+bi$. Then, choose the intervals that $a$ and $b$ belong to.
\[ (-3 - 2 i)(7 - 10 i) \]The solution is \( -41 + 16 i \), which is option C.\begin{enumerate}[label=\Alph*.]
\item \( a \in [-48, -40] \text{ and } b \in [-19, -9] \)

 $-41 - 16 i$, which corresponds to adding a minus sign in both terms.
\item \( a \in [-3, 1] \text{ and } b \in [-45, -43] \)

 $-1 - 44 i$, which corresponds to adding a minus sign in the second term.
\item \( a \in [-48, -40] \text{ and } b \in [16, 18] \)

* $-41 + 16 i$, which is the correct option.
\item \( a \in [-3, 1] \text{ and } b \in [40, 49] \)

 $-1 + 44 i$, which corresponds to adding a minus sign in the first term.
\item \( a \in [-24, -18] \text{ and } b \in [18, 21] \)

 $-21 + 20 i$, which corresponds to just multiplying the real terms to get the real part of the solution and the coefficients in the complex terms to get the complex part.
\end{enumerate}

\textbf{General Comment:} You can treat $i$ as a variable and distribute. Just remember that $i^2=-1$, so you can continue to reduce after you distribute.
}
\litem{
Choose the \textbf{smallest} set of Complex numbers that the number below belongs to.
\[ \frac{\sqrt{187}}{6}+\sqrt{-9}i \]The solution is \( \text{Irrational} \), which is option B.\begin{enumerate}[label=\Alph*.]
\item \( \text{Nonreal Complex} \)

This is a Complex number $(a+bi)$ that is not Real (has $i$ as part of the number).
\item \( \text{Irrational} \)

* This is the correct option!
\item \( \text{Rational} \)

These are numbers that can be written as fraction of Integers (e.g., -2/3 + 5)
\item \( \text{Pure Imaginary} \)

This is a Complex number $(a+bi)$ that \textbf{only} has an imaginary part like $2i$.
\item \( \text{Not a Complex Number} \)

This is not a number. The only non-Complex number we know is dividing by 0 as this is not a number!
\end{enumerate}

\textbf{General Comment:} Be sure to simplify $i^2 = -1$. This may remove the imaginary portion for your number. If you are having trouble, you may want to look at the \textit{Subgroups of the Real Numbers} section.
}
\litem{
Choose the \textbf{smallest} set of Real numbers that the number below belongs to.
\[ -\sqrt{\frac{115600}{400}} \]The solution is \( \text{Integer} \), which is option B.\begin{enumerate}[label=\Alph*.]
\item \( \text{Irrational} \)

These cannot be written as a fraction of Integers.
\item \( \text{Integer} \)

* This is the correct option!
\item \( \text{Not a Real number} \)

These are Nonreal Complex numbers \textbf{OR} things that are not numbers (e.g., dividing by 0).
\item \( \text{Rational} \)

These are numbers that can be written as fraction of Integers (e.g., -2/3)
\item \( \text{Whole} \)

These are the counting numbers with 0 (0, 1, 2, 3, ...)
\end{enumerate}

\textbf{General Comment:} First, you \textbf{NEED} to simplify the expression. This question simplifies to $-340$. 
 
 Be sure you look at the simplified fraction and not just the decimal expansion. Numbers such as 13, 17, and 19 provide \textbf{long but repeating/terminating decimal expansions!} 
 
 The only ways to *not* be a Real number are: dividing by 0 or taking the square root of a negative number. 
 
 Irrational numbers are more than just square root of 3: adding or subtracting values from square root of 3 is also irrational.
}
\litem{
Simplify the expression below into the form $a+bi$. Then, choose the intervals that $a$ and $b$ belong to.
\[ \frac{9 - 22 i}{4 - 8 i} \]The solution is \( 2.65  - 0.20 i \), which is option B.\begin{enumerate}[label=\Alph*.]
\item \( a \in [1.4, 2.3] \text{ and } b \in [2, 3] \)

 $2.25  + 2.75 i$, which corresponds to just dividing the first term by the first term and the second by the second.
\item \( a \in [2.5, 3.05] \text{ and } b \in [-1, 1] \)

* $2.65  - 0.20 i$, which is the correct option.
\item \( a \in [-1.9, -1.2] \text{ and } b \in [-2.5, -0.5] \)

 $-1.75  - 2.00 i$, which corresponds to forgetting to multiply the conjugate by the numerator and not computing the conjugate correctly.
\item \( a \in [211.9, 212.1] \text{ and } b \in [-1, 1] \)

 $212.00  - 0.20 i$, which corresponds to forgetting to multiply the conjugate by the numerator and using a plus instead of a minus in the denominator.
\item \( a \in [2.5, 3.05] \text{ and } b \in [-17, -15.5] \)

 $2.65  - 16.00 i$, which corresponds to forgetting to multiply the conjugate by the numerator.
\end{enumerate}

\textbf{General Comment:} Multiply the numerator and denominator by the *conjugate* of the denominator, then simplify. For example, if we have $2+3i$, the conjugate is $2-3i$.
}
\litem{
Choose the \textbf{smallest} set of Complex numbers that the number below belongs to.
\[ \sqrt{\frac{0}{15}}+\sqrt{3}i \]The solution is \( \text{Pure Imaginary} \), which is option B.\begin{enumerate}[label=\Alph*.]
\item \( \text{Irrational} \)

These cannot be written as a fraction of Integers. Remember: $\pi$ is not an Integer!
\item \( \text{Pure Imaginary} \)

* This is the correct option!
\item \( \text{Nonreal Complex} \)

This is a Complex number $(a+bi)$ that is not Real (has $i$ as part of the number).
\item \( \text{Not a Complex Number} \)

This is not a number. The only non-Complex number we know is dividing by 0 as this is not a number!
\item \( \text{Rational} \)

These are numbers that can be written as fraction of Integers (e.g., -2/3 + 5)
\end{enumerate}

\textbf{General Comment:} Be sure to simplify $i^2 = -1$. This may remove the imaginary portion for your number. If you are having trouble, you may want to look at the \textit{Subgroups of the Real Numbers} section.
}
\litem{
Choose the \textbf{smallest} set of Real numbers that the number below belongs to.
\[ \sqrt{\frac{2210}{10}} \]The solution is \( \text{Irrational} \), which is option D.\begin{enumerate}[label=\Alph*.]
\item \( \text{Not a Real number} \)

These are Nonreal Complex numbers \textbf{OR} things that are not numbers (e.g., dividing by 0).
\item \( \text{Rational} \)

These are numbers that can be written as fraction of Integers (e.g., -2/3)
\item \( \text{Whole} \)

These are the counting numbers with 0 (0, 1, 2, 3, ...)
\item \( \text{Irrational} \)

* This is the correct option!
\item \( \text{Integer} \)

These are the negative and positive counting numbers (..., -3, -2, -1, 0, 1, 2, 3, ...)
\end{enumerate}

\textbf{General Comment:} First, you \textbf{NEED} to simplify the expression. This question simplifies to $\sqrt{221}$. 
 
 Be sure you look at the simplified fraction and not just the decimal expansion. Numbers such as 13, 17, and 19 provide \textbf{long but repeating/terminating decimal expansions!} 
 
 The only ways to *not* be a Real number are: dividing by 0 or taking the square root of a negative number. 
 
 Irrational numbers are more than just square root of 3: adding or subtracting values from square root of 3 is also irrational.
}
\litem{
Simplify the expression below and choose the interval the simplification is contained within.
\[ 6 - 2^2 + 3 \div 7 * 16 \div 1 \]The solution is \( 8.857 \), which is option C.\begin{enumerate}[label=\Alph*.]
\item \( [16.67, 18.75] \)

 16.857, which corresponds to an Order of Operations error: multiplying by negative before squaring. For example: $(-3)^2 \neq -3^2$
\item \( [0.61, 2.57] \)

 2.027, which corresponds to an Order of Operations error: not reading left-to-right for multiplication/division.
\item \( [7.88, 9.48] \)

* 8.857, this is the correct option
\item \( [9.08, 11.1] \)

 10.027, which corresponds to two Order of Operations errors.
\item \( \text{None of the above} \)

 You may have gotten this by making an unanticipated error. If you got a value that is not any of the others, please let the coordinator know so they can help you figure out what happened.
\end{enumerate}

\textbf{General Comment:} While you may remember (or were taught) PEMDAS is done in order, it is actually done as P/E/MD/AS. When we are at MD or AS, we read left to right.
}
\litem{
Simplify the expression below into the form $a+bi$. Then, choose the intervals that $a$ and $b$ belong to.
\[ (-8 - 6 i)(7 + 9 i) \]The solution is \( -2 - 114 i \), which is option E.\begin{enumerate}[label=\Alph*.]
\item \( a \in [-2, -1] \text{ and } b \in [111, 116] \)

 $-2 + 114 i$, which corresponds to adding a minus sign in both terms.
\item \( a \in [-111, -108] \text{ and } b \in [29, 31] \)

 $-110 + 30 i$, which corresponds to adding a minus sign in the second term.
\item \( a \in [-57, -54] \text{ and } b \in [-56, -51] \)

 $-56 - 54 i$, which corresponds to just multiplying the real terms to get the real part of the solution and the coefficients in the complex terms to get the complex part.
\item \( a \in [-111, -108] \text{ and } b \in [-32, -26] \)

 $-110 - 30 i$, which corresponds to adding a minus sign in the first term.
\item \( a \in [-2, -1] \text{ and } b \in [-119, -111] \)

* $-2 - 114 i$, which is the correct option.
\end{enumerate}

\textbf{General Comment:} You can treat $i$ as a variable and distribute. Just remember that $i^2=-1$, so you can continue to reduce after you distribute.
}
\end{enumerate}

\end{document}