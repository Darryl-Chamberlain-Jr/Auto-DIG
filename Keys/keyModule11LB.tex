\documentclass{extbook}[14pt]
\usepackage{multicol, enumerate, enumitem, hyperref, color, soul, setspace, parskip, fancyhdr, amssymb, amsthm, amsmath, latexsym, units, mathtools}
\everymath{\displaystyle}
\usepackage[headsep=0.5cm,headheight=0cm, left=1 in,right= 1 in,top= 1 in,bottom= 1 in]{geometry}
\usepackage{dashrule}  % Package to use the command below to create lines between items
\newcommand{\litem}[1]{\item #1

\rule{\textwidth}{0.4pt}}
\pagestyle{fancy}
\lhead{}
\chead{Answer Key for Progress Quiz 6 Version B}
\rhead{}
\lfoot{9689-6866}
\cfoot{}
\rfoot{Spring 2021}
\begin{document}
\textbf{This key should allow you to understand why you choose the option you did (beyond just getting a question right or wrong). \href{https://xronos.clas.ufl.edu/mac1105spring2020/courseDescriptionAndMisc/Exams/LearningFromResults}{More instructions on how to use this key can be found here}.}

\textbf{If you have a suggestion to make the keys better, \href{https://forms.gle/CZkbZmPbC9XALEE88}{please fill out the short survey here}.}

\textit{Note: This key is auto-generated and may contain issues and/or errors. The keys are reviewed after each exam to ensure grading is done accurately. If there are issues (like duplicate options), they are noted in the offline gradebook. The keys are a work-in-progress to give students as many resources to improve as possible.}

\rule{\textwidth}{0.4pt}

\begin{enumerate}\litem{
Evaluate the limit below, if possible.
\[ \lim_{x \rightarrow 5} \frac{\sqrt{8x - 15} - 5}{3x - 15} \]The solution is \( 0.267 \), which is option C.\begin{enumerate}[label=\Alph*.]
\item \( 0.033 \)

You likely learned L'Hospital's Rule in a previous course, but misapplied it here.
\item \( 0.100 \)

You likely memorized how to solve the similar homework problem and used the same formula here.
\item \( 0.267 \)


\item \( \infty \)

You likely believed that since the denominator is equal to 0, the limit is infinity.
\item \( \text{None of the above} \)

If you got a limit that does not match any of the above, please contact the coordinator.
\end{enumerate}

\textbf{General Comment:} \textbf{General comments:} It is difficult to imagine the graph of this function, so you need to test values close to $x = 5$.
}
\litem{
Evaluate the one-sided limit of the function $f(x)$ below, if possible.
\[ \lim_{x \rightarrow -8^+} \frac{4}{(x-8)^4}+1 \]The solution is \( f(-8) \), which is option B.\begin{enumerate}[label=\Alph*.]
\item \( -\infty \)


\item \( f(-8) \)


\item \( \infty \)


\item \( \text{The limit does not exist} \)


\item \( \text{None of the above} \)


\end{enumerate}

\textbf{General Comment:} \textbf{General comments:} You should be able to graph the rational function displayed. If not, go back to Module 7 to learn about the general shape of rational functions.
}
\litem{
Based on the information below, which of the following statements is always true?

\begin{center}
    \textit{ As $x$ approaches $8$, $f(x)$ approaches $\infty$. }
\end{center}
The solution is \( f(x) \text{ is undefined when } x \text{ is close to or exactly } 8. \), which is option A.\begin{enumerate}[label=\Alph*.]
\item \( f(x) \text{ is undefined when } x \text{ is close to or exactly } 8. \)


\item \( f(x) \text{ is close to or exactly } \infty \text{ when } x \text{ is large enough}. \)


\item \( f(x) \text{ is close to or exactly } 8 \text{ when } x \text{ is large enough}. \)


\item \( x \text{ is undefined when } f(x) \text{ is close to or exactly } \infty. \)


\item \( \text{None of the above are always true.} \)


\end{enumerate}

\textbf{General Comment:} The limit tells you what happens as the $x$-values approach $8$. It says \textbf{absolutely nothing} about what is happening exactly at $f(8)$!
}
\litem{
Based on the information below, which of the following statements is always true?

\begin{center}
    \textit{ As $x$ approaches $1$, $f(x)$ approaches $9.895$. }
\end{center}
The solution is \( f(x) \text{ is close to or exactly } 9.895 \text{ when } x \text{ is close to } 1 \), which is option C.\begin{enumerate}[label=\Alph*.]
\item \( f(x) \text{ is close to or exactly } 1 \text{ when } x \text{ is close to } 9.895 \)


\item \( f(x) = 9.895 \text{ when } x \text{ is close to } 1 \)


\item \( f(x) \text{ is close to or exactly } 9.895 \text{ when } x \text{ is close to } 1 \)


\item \( f(x) = 1 \text{ when } x \text{ is close to } 9.895 \)


\item \( \text{None of the above are always true.} \)


\end{enumerate}

\textbf{General Comment:} The limit tells you what happens as the $x$-values approach $1$. It says \textbf{absolutely nothing} about what is happening exactly at $f(1)$!
}
\litem{
To estimate the one-sided limit of the function below as $x$ approaches 5 from the right, which of the following sets of numbers should you use?
\[ \frac{\frac{5}{x} - 1}{x - 5} \]The solution is \( \{ 5.1000, 5.0100, 5.0010, 5.0001 \} \), which is option A.\begin{enumerate}[label=\Alph*.]
\item \( \{ 5.1000, 5.0100, 5.0010, 5.0001 \} \)

This is correct!
\item \( \{ 4.9000, 4.9900, 4.9990, 4.9999 \} \)

These values would estimate the limit of 5 on the left.
\item \( \{ 4.9000, 4.9900, 5.0100, 5.1000 \} \)

These values would estimate the limit at the point and not a one-sided limit.
\item \( \{ 5.0000, 4.9000, 4.9900, 4.9990 \} \)

If we get $\frac{0}{0}$ or $\frac{\infty}{\infty}$, the value 5 doesn't help us estimate the limit.
\item \( \{ 5.0000, 5.1000, 5.0100, 5.0010 \} \)

If we get $\frac{0}{0}$ or $\frac{\infty}{\infty}$, the value 5 doesn't help us estimate the limit.
\end{enumerate}

\textbf{General Comment:} \textbf{General Comments:} To evaluate a one-sided limit, we want to put numbers close to the limit. We can't use the limit value itself if it results in $\frac{0}{0}$ or $\frac{\infty}{\infty}$
}
\litem{
Evaluate the limit below, if possible.
\[ \lim_{x \rightarrow 8} \frac{\sqrt{8x - 15} - 7}{7x - 56} \]The solution is \( 0.082 \), which is option A.\begin{enumerate}[label=\Alph*.]
\item \( 0.082 \)

* This is the correct option.
\item \( 0.404 \)

You likely tried to use a shortcut to find the limit of a function that only works for when the numerator/denominator are polynomials.
\item \( 0.071 \)

You likely memorized how to solve the similar homework problem and used the same formula here.
\item \( \infty \)

You likely believed that since the denominator is equal to 0, the limit is infinity.
\item \( \text{None of the above} \)

If you got a limit that does not match any of the above, please contact the coordinator.
\end{enumerate}

\textbf{General Comment:} \textbf{General comments:} It is difficult to imagine the graph of this function, so you need to test values close to $x = 8$.
}
\litem{
For the graph below, evaluate the limit: $ \displaystyle \lim_{x \rightarrow -4} f(x)$.

\begin{center}
    \includegraphics[width=0.5\textwidth]{../Figures/evaluateLimitGraphicallyCopyB.png}
\end{center}


The solution is \( 0 \), which is option C.\begin{enumerate}[label=\Alph*.]
\item \( -\infty \)


\item \( -6 \)


\item \( 0 \)


\item \( \text{The limit does not exist} \)


\item \( \text{None of the above} \)


\end{enumerate}

\textbf{General Comment:} \textbf{General Comments:} Remember that the limit does not exist if the left-hand and right-hand limits do not match.
}
\litem{
For the graph below, evaluate the limit: $ \displaystyle \lim_{x \rightarrow -4} f(x)$.

\begin{center}
    \includegraphics[width=0.5\textwidth]{../Figures/evaluateLimitGraphicallyB.png}
\end{center}


The solution is \( 0 \), which is option A.\begin{enumerate}[label=\Alph*.]
\item \( 0 \)


\item \( -\infty \)


\item \( -6 \)


\item \( \text{The limit does not exist} \)


\item \( \text{None of the above} \)


\end{enumerate}

\textbf{General Comment:} \textbf{General Comments:} Remember that the limit does not exist if the left-hand and right-hand limits do not match.
}
\litem{
To estimate the one-sided limit of the function below as $x$ approaches 5 from the right, which of the following sets of numbers should you use?
\[ \frac{\frac{5}{x} - 1}{x - 5} \]The solution is \( \{ 5.1000, 5.0100, 5.0010, 5.0001 \} \), which is option A.\begin{enumerate}[label=\Alph*.]
\item \( \{ 5.1000, 5.0100, 5.0010, 5.0001 \} \)

This is correct!
\item \( \{ 5.0000, 5.1000, 5.0100, 5.0010 \} \)

If we get $\frac{0}{0}$ or $\frac{\infty}{\infty}$, the value 5 doesn't help us estimate the limit.
\item \( \{ 5.0000, 4.9000, 4.9900, 4.9990 \} \)

If we get $\frac{0}{0}$ or $\frac{\infty}{\infty}$, the value 5 doesn't help us estimate the limit.
\item \( \{ 4.9000, 4.9900, 4.9990, 4.9999 \} \)

These values would estimate the limit of 5 on the left.
\item \( \{ 4.9000, 4.9900, 5.0100, 5.1000 \} \)

These values would estimate the limit at the point and not a one-sided limit.
\end{enumerate}

\textbf{General Comment:} \textbf{General Comments:} To evaluate a one-sided limit, we want to put numbers close to the limit. We can't use the limit value itself if it results in $\frac{0}{0}$ or $\frac{\infty}{\infty}$
}
\litem{
Evaluate the one-sided limit of the function $f(x)$ below, if possible.
\[ \lim_{x \rightarrow 4^+} \frac{-1}{(x+4)^8}+7 \]The solution is \( f(4) \), which is option B.\begin{enumerate}[label=\Alph*.]
\item \( \infty \)


\item \( f(4) \)


\item \( -\infty \)


\item \( \text{The limit does not exist} \)


\item \( \text{None of the above} \)


\end{enumerate}

\textbf{General Comment:} \textbf{General comments:} You should be able to graph the rational function displayed. If not, go back to Module 7 to learn about the general shape of rational functions.
}
\end{enumerate}

\end{document}