\documentclass{extbook}[14pt]
\usepackage{multicol, enumerate, enumitem, hyperref, color, soul, setspace, parskip, fancyhdr, amssymb, amsthm, amsmath, bbm, latexsym, units, mathtools}
\everymath{\displaystyle}
\usepackage[headsep=0.5cm,headheight=0cm, left=1 in,right= 1 in,top= 1 in,bottom= 1 in]{geometry}
\pagestyle{fancy}
\lhead{}
\chead{Answer Key for Module\,11L\,-\,Introduction\,to\,Limits Version B}
\rhead{}
\lfoot{Summer\,C\,2020}
\cfoot{}
\rfoot{}
\begin{document}
\textbf{This key should allow you to understand why you choose the option you did (beyond just getting a question right or wrong). \href{https://xronos.clas.ufl.edu/mac1105spring2020/courseDescriptionAndMisc/Exams/LearningFromResults}{More instructions on how to use this key can be found here}.}

\textbf{If you have a suggestion to make the keys better, \href{https://forms.gle/CZkbZmPbC9XALEE88}{please fill out the short survey here}.}

\textit{Note: This key is auto-generated and may contain issues and/or errors. The keys are reviewed after each exam to ensure grading is done accurately. If there are issues (like duplicate options), they are noted in the offline gradebook. The keys are a work-in-progress to give students as many resources to improve as possible.}

\rule{\textwidth}{0.4pt}

1. Evaluate the limit below, if possible.
\[ \lim_{x \rightarrow 8} \frac{\sqrt{6x - 12} - 6}{2x - 16} \] 
The solution is $ 0.250 $ 

\begin{enumerate}[label=\Alph*.] 
\item $ 0.083 $ 

 You likely memorized how to solve the similar homework problem and used the same formula here. 
\item $ \infty $ 

 You likely believed that since the denominator is equal to 0, the limit is infinity. 
\item $ 1.225 $ 

 You likely tried to use a shortcut to find the limit of a function that only works for when the numerator/denominator are polynomials. 
\item $ 0.250 $ 

 * This is the correct option. 
\item $ \text{None of the above} $ 

 If you got a limit that does not match any of the above, please contact the coordinator. 
\end{enumerate} 
 
\textbf{General Comment:} \textbf{General comments:} It is difficult to imagine the graph of this function, so you need to test values close to $x = 8$. 

-----------------------------------------------

2. For the graph below, find the value(s) $a$ that makes the limit true: $ \displaystyle \lim_{x \rightarrow a} f(x) = -\infty$.
\begin{center} \includegraphics[width=0.3\textwidth]{../Figures/evaluateLimitGraphicallyB.png} \end{center} 

The solution is $ \text{Multiple } a \text{ make the limit true}. $ 

\begin{enumerate}[label=\Alph*.] 
\item $ -2 $ 

  
\item $ -\infty $ 

  
\item $ 3 $ 

  
\item $ \text{Multiple } a \text{ make the limit true}. $ 

  
\item $ \text{No } a \text{ make the limit true}. $ 

  
\end{enumerate} 
 
\textbf{General Comment:} \textbf{General Comments:} There can be multiple $a$ values that make the limit true! For the limit, draw a horizontal line and determine if an $x$ value makes the limit true. 

-----------------------------------------------

3. To estimate the one-sided limit of the function below as $x$ approaches 7 from the right, which of the following sets of numbers should you use?
\[ \frac{\frac{7}{x} - 1}{x - 7} \] 
The solution is $ \{ 7.1000, 7.0100, 7.0010, 7.0001 \} $ 

\begin{enumerate}[label=\Alph*.] 
\item $ \{ 7.0000, 6.9000, 6.9900, 6.9990 \} $ 

 If we get $\frac{0}{0}$ or $\frac{\infty}{\infty}$, the value 7 doesn't help us estimate the limit. 
\item $ \{ 6.9000, 6.9900, 6.9990, 6.9999 \} $ 

 These values would estimate the limit of 7 on the left. 
\item $ \{ 7.1000, 7.0100, 7.0010, 7.0001 \} $ 

 This is correct! 
\item $ \{ 7.0000, 7.1000, 7.0100, 7.0010 \} $ 

 If we get $\frac{0}{0}$ or $\frac{\infty}{\infty}$, the value 7 doesn't help us estimate the limit. 
\item $ \{ 6.9000, 6.9900, 7.0100, 7.1000 \} $ 

 These values would estimate the limit at the point and not a one-sided limit. 
\end{enumerate} 
 
\textbf{General Comment:} \textbf{General Comments:} To evaluate a one-sided limit, we want to put numbers close to the limit. We can't use the limit value itself if it results in $\frac{0}{0}$ or $\frac{\infty}{\infty}$ 

-----------------------------------------------

4. Evaluate the one-sided limit of the function $f(x)$ below, if possible.
\[ \lim_{x \rightarrow -4^+} \frac{-6}{(x-4)^9}+2 \] 
The solution is $ f(-4) $ 

\begin{enumerate}[label=\Alph*.] 
\item $ \infty $ 

  
\item $ f(-4) $ 

  
\item $ -\infty $ 

  
\item $ \text{The limit does not exist} $ 

  
\item $ \text{None of the above} $ 

  
\end{enumerate} 
 
\textbf{General Comment:} \textbf{General comments:} You should be able to graph the rational function displayed. If not, go back to Module 7 to learn about the general shape of rational functions. 

-----------------------------------------------

0. Based on the information below, which of the following statements is always true?
As $x$ approaches $2$, $f(x)$ approaches $16.325$. 
The solution is $ \text{None of the above are always true.} $ 

\begin{enumerate}[label=\Alph*.] 
\item $ f(16) = 2 $ 

  
\item $ f(2) = 16 $ 

  
\item $ f(16) \text{ is close to or exactly } 2 $ 

  
\item $ f(2) \text{ is close to or exactly } 16 $ 

  
\item $ \text{None of the above are always true.} $ 

  
\end{enumerate} 
 
\textbf{General Comment:} \textbf{General comments:} The limit tells you what happens as the $x$-values approach $2$. It says \textbf{absolutely nothing} about what is happening exactly at $f(x)$! 

-----------------------------------------------


\end{document}

