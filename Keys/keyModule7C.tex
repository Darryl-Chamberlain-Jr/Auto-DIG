\documentclass{extbook}[14pt]
\usepackage{multicol, enumerate, enumitem, hyperref, color, soul, setspace, parskip, fancyhdr, amssymb, amsthm, amsmath, bbm, latexsym, units, mathtools}
\everymath{\displaystyle}
\usepackage[headsep=0.5cm,headheight=0cm, left=1 in,right= 1 in,top= 1 in,bottom= 1 in]{geometry}
\pagestyle{fancy}
\lhead{}
\chead{Answer Key for Module\,7\,-\,Rational\,Functions Version C}
\rhead{}
\lfoot{Summer\,C\,2020}
\cfoot{}
\rfoot{}
\begin{document}
\textbf{This key should allow you to understand why you choose the option you did (beyond just getting a question right or wrong). \href{https://xronos.clas.ufl.edu/mac1105spring2020/courseDescriptionAndMisc/Exams/LearningFromResults}{More instructions on how to use this key can be found here}.}

\textbf{If you have a suggestion to make the keys better, \href{https://forms.gle/CZkbZmPbC9XALEE88}{please fill out the short survey here}.}

\textit{Note: This key is auto-generated and may contain issues and/or errors. The keys are reviewed after each exam to ensure grading is done accurately. If there are issues (like duplicate options), they are noted in the offline gradebook. The keys are a work-in-progress to give students as many resources to improve as possible.}

\rule{\textwidth}{0.4pt}

1. Choose the graph of the equation below.
\[ f(x) = \frac{1}{(x - 2)^2} - 1 \] 

 
 The solution is  
 \begin{center} \includegraphics[width=0.3\textwidth]{../Figures/rationalEquationToGraphEC.png} \end{center}\begin{tabular}{|c|c|} 
\hline 
 & \tabularnewline 
 \textbf{A.} \includegraphics[width=0.3\textwidth]{../Figures/rationalEquationToGraphAC.png} & \textbf{B.} \includegraphics[width=0.3\textwidth]{../Figures/rationalEquationToGraphBC.png} \tabularnewline 
\hline 
 & \tabularnewline 
 \textbf{C.} \includegraphics[width=0.3\textwidth]{../Figures/rationalEquationToGraphCC.png} & \textbf{D.} \includegraphics[width=0.3\textwidth]{../Figures/rationalEquationToGraphDC.png} \tabularnewline 
\hline 
 E. None of the figures above. & \tabularnewline 
\hline 
 \end{tabular} 
 
\begin{enumerate}[label=\Alph*.] 
\item Incorrect due to $y$-value.  
\item Corresponds to using the general form $f(x) = \frac{a}{(x+h)^2}+k$ and the opposite leading coefficient.  
\item Corresponds to thinking the graph was a shifted version of $\frac{1}{x}$.  
\item Corresponds to thinking the graph was a shifted version of $\frac{1}{x}$, using the general form $f(x) = \frac{a}{(x+h)^2}+k$, and the opposite leading coefficient.  
\end{enumerate} 
 
\textbf{General Comment:} General Comments: Remember that the general form of a basic rational equation is $ f(x) = \frac{a}{(x-h)^n} + k$, where $a$ is the leading coefficient (and in this case, we assume is either $1$ or $-1$), $n$ is the degree (in this case, either $1$ or $2$), and $(h, k)$ is the intersection of the asymptotes. 

-----------------------------------------------

2. Choose the equation of the function graphed below.
\begin{center} \includegraphics[width=0.3\textwidth]{../Figures/rationalGraphToEquationC.png} \end{center} 

The solution is $ f(x) = \frac{1}{(x + 3)^2} + 1 $ 

\begin{enumerate}[label=\Alph*.] 
\item $ f(x) = \frac{1}{(x + 3)^2} + 1 $ 

 This is the correct option. 
\item $ f(x) = \frac{1}{x + 3} + 1 $ 

 Corresponds to thinking the graph was a shifted version of $\frac{1}{x}$. 
\item $ f(x) = \frac{-1}{(x - 3)^2} + 1 $ 

 Corresponds to using the general form $f(x) = \frac{a}{(x+h)^2}+k$ and the opposite leading coefficient. 
\item $ f(x) = \frac{-1}{x - 3} + 1 $ 

 Corresponds to thinking the graph was a shifted version of $\frac{1}{x}$, using the general form $f(x) = \frac{a}{(x+h)^2}+k$, and the opposite leading coefficient. 
\item $ \text{None of the above} $ 

 This corresponds to believing the vertex of the graph was not correct. 
\end{enumerate} 
 
\textbf{General Comment:} General Comments: Remember that the general form of a basic rational equation is $ f(x) = \frac{a}{(x-h)^n} + k$, where $a$ is the leading coefficient (and in this case, we assume is either $1$ or $-1$), $n$ is the degree (in this case, either $1$ or $2$), and $(h, k)$ is the intersection of the asymptotes. 

-----------------------------------------------

3. Determine the domain of the function below.
\[ f(x) = \frac{3}{36x^{2} -36} \] 
The solution is $ \text{All Real numbers except } x = -1.000 \text{ and } x = 1.000. $ 

\begin{enumerate}[label=\Alph*.] 
\item $ \text{All Real numbers except } x = a \text{ and } x = b, \text{ where } a \in [-36.2, -35.1] \text{ and } b \in [35.1, 36.3] $ 

 All Real numbers except $x = -36.000$ and $x = 36.000$, which corresponds to not factoring the denominator correctly. 
\item $ \text{All Real numbers except } x = a, \text{ where } a \in [-36.2, -35.1] $ 

 All Real numbers except $x = -36.000$, which corresponds to removing a distractor value from the denominator. 
\item $ \text{All Real numbers.} $ 

 This corresponds to thinking the denominator has complex roots or that rational functions have a domain of all Real numbers. 
\item $ \text{All Real numbers except } x = a, \text{ where } a \in [-1.4, 0.6] $ 

 All Real numbers except $x = -1.000$, which corresponds to removing only 1 value from the denominator. 
\item $ \text{All Real numbers except } x = a \text{ and } x = b, \text{ where } a \in [-1.4, 0.6] \text{ and } b \in [0.2, 3.6] $ 

 All Real numbers except $x = -1.000$ and $x = 1.000$, which is the correct option. 
\end{enumerate} 
 
\textbf{General Comment:} General Comments: The new domain is the intersection of the previous domains. 

-----------------------------------------------

4. Solve the rational equation below. Then, choose the interval(s) that the solution(s) belongs to.
\[ \frac{-3}{2x + 8} + 5 = \frac{7}{-16x -64} \] 
The solution is $ x = -3.788 $ 

\begin{enumerate}[label=\Alph*.] 
\item $ \text{All solutions lead to invalid or complex values in the equation.} $ 

 This corresponds to thinking $x = -3.788$ leads to dividing by zero in the original equation, which it does not. 
\item $ x_1 \in [-8, -3] \text{ and } x_2 \in [-5,0] $ 

 $x = -3.788 \text{ and } x = -3.000$, which corresponds to getting the correct solution and believing there should be a second solution to the equation. 
\item $ x_1 \in [-8, -3] \text{ and } x_2 \in [2,7] $ 

 $x = -3.788 \text{ and } x = 4.213$, which corresponds to getting the correct solution and believing there should be a second solution to the equation. 
\item $ x \in [2,7] $ 

 $x = 4.213$, which corresponds to not distributing the factor $2x + 8$ correctly when trying to eliminate the fraction. 
\item $ x \in [-3.79,-1.79] $ 

 * $x = -3.788$, which is the correct option. 
\end{enumerate} 
 
\textbf{General Comment:} General Comments: Distractors are different based on the number of solutions. Remember that after solving, we need to make sure our solution does not make the original equation divide by zero! 

-----------------------------------------------

0. Solve the rational equation below. Then, choose the interval(s) that the solution(s) belongs to.
\[ \frac{5x}{3x -3} + \frac{-4x^{2}}{-12x^{2} -9 x + 21} = \frac{4}{-4x -7} \] 
The solution is $ \text{There are two solutions: } x = 0.229 \text{ and } x = -2.187 $ 

\begin{enumerate}[label=\Alph*.] 
\item $ x_1 \in [-0.87, 1.07] \text{ and } x_2 \in [-0.4,2.4] $ 

  
\item $ x_1 \in [-0.87, 1.07] \text{ and } x_2 \in [-6,-1.4] $ 

 * $x = 0.229 \text{ and } x = -2.187$, which is the correct option. 
\item $ x \in [-2.56,-1.85] $ 

  
\item $ x \in [-1.83,-0.66] $ 

  
\item $ \text{All solutions lead to invalid or complex values in the equation.} $ 

  
\end{enumerate} 
 
\textbf{General Comment:} General Comments: Distractors are different based on the number of solutions. Remember that after solving, we need to make sure our solution does not make the original equation divide by zero! 

-----------------------------------------------


\end{document}

