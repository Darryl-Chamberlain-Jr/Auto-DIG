\documentclass{extbook}[14pt]
\usepackage{multicol, enumerate, enumitem, hyperref, color, soul, setspace, parskip, fancyhdr, amssymb, amsthm, amsmath, bbm, latexsym, units, mathtools}
\everymath{\displaystyle}
\usepackage[headsep=0.5cm,headheight=0cm, left=1 in,right= 1 in,top= 1 in,bottom= 1 in]{geometry}
\pagestyle{fancy}
\lhead{}
\chead{Answer Key for Module\,7\,-\,Rational\,Functions Version C}
\rhead{}
\lfoot{Summer\,C\,2020}
\cfoot{}
\rfoot{}
\begin{document}
\textbf{This key should allow you to understand why you choose the option you did (beyond just getting a question right or wrong). \href{https://xronos.clas.ufl.edu/mac1105spring2020/courseDescriptionAndMisc/Exams/LearningFromResults}{More instructions on how to use this key can be found here}.}

\textbf{If you have a suggestion to make the keys better, \href{https://forms.gle/CZkbZmPbC9XALEE88}{please fill out the short survey here}.}

\textit{Note: This key is auto-generated and may contain issues and/or errors. The keys are reviewed after each exam to ensure grading is done accurately. If there are issues (like duplicate options), they are noted in the offline gradebook. The keys are a work-in-progress to give students as many resources to improve as possible.}

\rule{\textwidth}{0.4pt}

31. Determine the domain of the function below.
\[ f(x) = \frac{4}{15x^{2} -8 x -16} \] 
The solution is $ \text{All Real numbers except } x = -0.800 \text{ and } x = 1.333. $ 

\begin{enumerate}[label=\Alph*.] 
\item $ \text{All Real numbers except } x = a, \text{ where } a \in [-2, 1] $ 

 All Real numbers except $x = -0.800$, which corresponds to removing only 1 value from the denominator. 
\item $ \text{All Real numbers.} $ 

 This corresponds to thinking the denominator has complex roots or that rational functions have a domain of all Real numbers. 
\item $ \text{All Real numbers except } x = a \text{ and } x = b, \text{ where } a \in [-2, 1] \text{ and } b \in [0, 4] $ 

 All Real numbers except $x = -0.800$ and $x = 1.333$, which is the correct option. 
\item $ \text{All Real numbers except } x = a \text{ and } x = b, \text{ where } a \in [-21, -14] \text{ and } b \in [8, 16] $ 

 All Real numbers except $x = -20.000$ and $x = 12.000$, which corresponds to not factoring the denominator correctly. 
\item $ \text{All Real numbers except } x = a, \text{ where } a \in [-21, -14] $ 

 All Real numbers except $x = -20.000$, which corresponds to removing a distractor value from the denominator. 
\end{enumerate} 
 
General Comments: The new domain is the intersection of the previous domains.

-----------------------------------------------

32. Solve the rational equation below. Then, choose the interval(s) that the solution(s) belongs to.
\[ \frac{2x}{5x + 4} + \frac{-6x^{2}}{-25x^{2} -30 x -8} = \frac{-4}{-5x -2} \] 
The solution is $ \text{There are two solutions: } x = 1.618 \text{ and } x = -0.618 $ 

\begin{enumerate}[label=\Alph*.] 
\item $ x_1 \in [1.42, 2.07] \text{ and } x_2 \in [-0.76,-0.53] $ 

 * $x = 1.618 \text{ and } x = -0.618$, which is the correct option. 
\item $ x \in [-0.58,-0.38] $ 

  
\item $ \text{All solutions lead to invalid or complex values in the equation.} $ 

  
\item $ x \in [-0.97,-0.57] $ 

  
\item $ x_1 \in [1.42, 2.07] \text{ and } x_2 \in [-1.22,-0.7] $ 

  
\end{enumerate} 
 
General Comments: Distractors are different based on the number of solutions. Remember that after solving, we need to make sure our solution does not make the original equation divide by zero!

-----------------------------------------------

33. Choose the graph of the equation below.
\[ f(x) = \frac{-1}{x + 3} - 3 \] 

 
 The solution is  
 \begin{center} \includegraphics[width=0.3\textwidth]{../Figures/rationalEquationToGraphCD.png} \end{center}\begin{tabular}{|c|c|} 
\hline 
 & \tabularnewline 
 \textbf{A.} \includegraphics[width=0.3\textwidth]{../Figures/rationalEquationToGraphAC.png} & \textbf{B.} \includegraphics[width=0.3\textwidth]{../Figures/rationalEquationToGraphBC.png} \tabularnewline 
\hline 
 & \tabularnewline 
 \textbf{C.} \includegraphics[width=0.3\textwidth]{../Figures/rationalEquationToGraphCC.png} & \textbf{D.} \includegraphics[width=0.3\textwidth]{../Figures/rationalEquationToGraphDC.png} \tabularnewline 
\hline 
 E. None of the figures above. & \tabularnewline 
\hline 
 \end{tabular} 
 
\begin{enumerate}[label=\Alph*.] 
\item This is the correct option.  
\item Corresponds to using the general form $f(x) = \frac{a}{x+h}+k$ and the opposite leading coefficient.  
\item Corresponds to thinking the graph was a shifted version of $\frac{1}{x^2}$.  
\item Corresponds to thinking the graph was a shifted version of $\frac{1}{x^2}$, using the general form $f(x) = \frac{a}{x+h}+k$, and the opposite leading coefficient.  
\end{enumerate} 
 
General Comments: Remember that the general form of a basic rational equation is $ f(x) = \frac{a}{(x-h)^n} + k$, where $a$ is the leading coefficient (and in this case, we assume is either $1$ or $-1$), $n$ is the degree (in this case, either $1$ or $2$), and $(h, k)$ is the intersection of the asymptotes.

-----------------------------------------------

34. Choose the equation of the function graphed below.
\begin{center} \includegraphics[width=0.3\textwidth]{../Figures/rationalGraphToEquationC.png} \end{center} 

The solution is $ \text{None of the above as it should be } f(x) = \frac{1}{(x - 2)^2} - 1 $ 

\begin{enumerate}[label=\Alph*.] 
\item $ f(x) = \frac{-1}{x - 2} + 2 $ 

 Corresponds to thinking the graph was a shifted version of $\frac{1}{x}$, using the general form $f(x) = \frac{a}{(x-h)^2}+k$, the opposite leading coefficient, AND not noticing the $y$-value was wrong. 
\item $ f(x) = \frac{1}{(x + 2)^2} + 2 $ 

 The $x$- and $y$-value of the equation does not match the graph. 
\item $ f(x) = \frac{-1}{(x - 2)^2} + 2 $ 

 Corresponds to using the general form $f(x) = \frac{a}{(x-h)^2}+k$, the opposite leading coefficient, AND not noticing the $y$-value was wrong. 
\item $ f(x) = \frac{1}{x + 2} + 2 $ 

 Corresponds to thinking the graph was a shifted version of $\frac{1}{x}$ AND not noticing the $y$-value was wrong. 
\item $ \text{None of the above} $ 

 None of the equation options were the correct equation. 
\end{enumerate} 
 
General Comments: Remember that the general form of a basic rational equation is $ f(x) = \frac{a}{(x-h)^n} + k$, where $a$ is the leading coefficient (and in this case, we assume is either $1$ or $-1$), $n$ is the degree (in this case, either $1$ or $2$), and $(h, k)$ is the intersection of the asymptotes.

-----------------------------------------------

35. Solve the rational equation below. Then, choose the interval(s) that the solution(s) belongs to.
\[ \frac{-9}{-9x + 4} + -4 = \frac{5}{81x -36} \] 
The solution is $ x = 0.679 $ 

\begin{enumerate}[label=\Alph*.] 
\item $ x \in [-1.4,0] $ 

 $x = -0.210$, which corresponds to not distributing the factor $-9x + 4$ correctly when trying to eliminate the fraction. 
\item $ x_1 \in [-1.4, 0] \text{ and } x_2 \in [0.6,0.8] $ 

 $x = -0.210 \text{ and } x = 0.679$, which corresponds to getting the correct solution and believing there should be a second solution to the equation. 
\item $ x_1 \in [0.4, 1.1] \text{ and } x_2 \in [0.72,1.95] $ 

 $x = 0.679 \text{ and } x = 0.833$, which corresponds to getting the correct solution and believing there should be a second solution to the equation. 
\item $ \text{All solutions lead to invalid or complex values in the equation.} $ 

 This corresponds to thinking $x = 0.679$ leads to dividing by zero in the original equation, which it does not. 
\item $ x \in [-0.32,2.68] $ 

 * $x = 0.679$, which is the correct option. 
\end{enumerate} 
 
General Comments: Distractors are different based on the number of solutions. Remember that after solving, we need to make sure our solution does not make the original equation divide by zero!

-----------------------------------------------


\end{document}

