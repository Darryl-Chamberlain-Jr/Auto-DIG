\documentclass{extbook}[14pt]
\usepackage{multicol, enumerate, enumitem, hyperref, color, soul, setspace, parskip, fancyhdr, amssymb, amsthm, amsmath, latexsym, units, mathtools}
\everymath{\displaystyle}
\usepackage[headsep=0.5cm,headheight=0cm, left=1 in,right= 1 in,top= 1 in,bottom= 1 in]{geometry}
\usepackage{dashrule}  % Package to use the command below to create lines between items
\newcommand{\litem}[1]{\item #1

\rule{\textwidth}{0.4pt}}
\pagestyle{fancy}
\lhead{}
\chead{Answer Key for Makeup Progress Quiz 3 Version C}
\rhead{}
\lfoot{1648-1753}
\cfoot{}
\rfoot{Summer C 2021}
\begin{document}
\textbf{This key should allow you to understand why you choose the option you did (beyond just getting a question right or wrong). \href{https://xronos.clas.ufl.edu/mac1105spring2020/courseDescriptionAndMisc/Exams/LearningFromResults}{More instructions on how to use this key can be found here}.}

\textbf{If you have a suggestion to make the keys better, \href{https://forms.gle/CZkbZmPbC9XALEE88}{please fill out the short survey here}.}

\textit{Note: This key is auto-generated and may contain issues and/or errors. The keys are reviewed after each exam to ensure grading is done accurately. If there are issues (like duplicate options), they are noted in the offline gradebook. The keys are a work-in-progress to give students as many resources to improve as possible.}

\rule{\textwidth}{0.4pt}

\begin{enumerate}\litem{
Determine the domain of the function below.
\[ f(x) = \frac{6}{30x^{2} -7 x -15} \]The solution is \( \text{All Real numbers except } x = -0.600 \text{ and } x = 0.833. \), which is option B.\begin{enumerate}[label=\Alph*.]
\item \( \text{All Real numbers.} \)

This corresponds to thinking the denominator has complex roots or that rational functions have a domain of all Real numbers.
\item \( \text{All Real numbers except } x = a \text{ and } x = b, \text{ where } a \in [-1.42, -0.37] \text{ and } b \in [0.6, 1.37] \)

All Real numbers except $x = -0.600$ and $x = 0.833$, which is the correct option.
\item \( \text{All Real numbers except } x = a, \text{ where } a \in [-1.42, -0.37] \)

All Real numbers except $x = -0.600$, which corresponds to removing only 1 value from the denominator.
\item \( \text{All Real numbers except } x = a \text{ and } x = b, \text{ where } a \in [-16.01, -14.82] \text{ and } b \in [29.58, 30.49] \)

All Real numbers except $x = -15.000$ and $x = 30.000$, which corresponds to not factoring the denominator correctly.
\item \( \text{All Real numbers except } x = a, \text{ where } a \in [-16.01, -14.82] \)

All Real numbers except $x = -15.000$, which corresponds to removing a distractor value from the denominator.
\end{enumerate}

\textbf{General Comment:} Recall that dividing by zero is not a real number. Therefore the domain is all real numbers \textbf{except} those that make the denominator 0.
}
\litem{
Solve the rational equation below. Then, choose the interval(s) that the solution(s) belongs to.
\[ \frac{-63}{-35x + 14} + 1 = \frac{-63}{-35x + 14} \]The solution is \( \text{all solutions are invalid or lead to complex values in the equation.} \), which is option C.\begin{enumerate}[label=\Alph*.]
\item \( x_1 \in [-0.3, 1] \text{ and } x_2 \in [-3.6,1.4] \)

$x = 0.400 \text{ and } x = 0.400$, which corresponds to getting the correct solution and believing there should be a second solution to the equation.
\item \( x_1 \in [-0.8, 0] \text{ and } x_2 \in [-3.6,1.4] \)

$x = -0.400 \text{ and } x = 0.400$, which corresponds to getting the correct solution and believing there should be a second solution to the equation.
\item \( \text{All solutions lead to invalid or complex values in the equation.} \)

*$x = 0.400$ leads to dividing by 0 in the original equation and thus is not a valid solution, which is the correct option.
\item \( x \in [0.4,1.4] \)

$x = 0.400$, which corresponds to not checking if this value leads to dividing by 0 in the original equation and thus is not a valid solution.
\item \( x \in [-0.8,0] \)

$x = -0.400$, which corresponds to not distributing the factor $-35x + 14$ correctly when trying to eliminate the fraction.
\end{enumerate}

\textbf{General Comment:} Distractors are different based on the number of solutions. Remember that after solving, we need to make sure our solution does not make the original equation divide by zero!
}
\litem{
Choose the equation of the function graphed below.

\begin{center}
    \includegraphics[width=0.5\textwidth]{../Figures/rationalGraphToEquationCopyC.png}
\end{center}


The solution is \( \text{None of the above as it should be } f(x) = \frac{-1}{(x - 1)^2} - 3 \), which is option E.\begin{enumerate}[label=\Alph*.]
\item \( f(x) = \frac{-1}{x + 1} + 4 \)

Corresponds to thinking the graph was a shifted version of $\frac{1}{x}$ AND not noticing the $y$-value was wrong.
\item \( f(x) = \frac{-1}{(x + 1)^2} + 4 \)

The $x$- and $y$-value of the equation does not match the graph.
\item \( f(x) = \frac{1}{x - 1} + 4 \)

Corresponds to thinking the graph was a shifted version of $\frac{1}{x}$, using the general form $f(x) = \frac{a}{(x-h)^2}+k$, the opposite leading coefficient, AND not noticing the $y$-value was wrong.
\item \( f(x) = \frac{1}{(x - 1)^2} + 4 \)

Corresponds to using the general form $f(x) = \frac{a}{(x-h)^2}+k$, the opposite leading coefficient, AND not noticing the $y$-value was wrong.
\item \( \text{None of the above} \)

None of the equation options were the correct equation.
\end{enumerate}

\textbf{General Comment:} Remember that the general form of a basic rational equation is $ f(x) = \frac{a}{(x-h)^n} + k$, where $a$ is the leading coefficient (and in this case, we assume is either $1$ or $-1$), $n$ is the degree (in this case, either $1$ or $2$), and $(h, k)$ is the intersection of the asymptotes.
}
\litem{
Choose the graph of the equation below.
\[ f(x) = \frac{1}{x - 2} + 1 \]The solution is the graph below, which is option A.
    \begin{center}
        \includegraphics[width=0.3\textwidth]{../Figures/rationalEquationToGraphCopyAC.png}
    \end{center}\begin{enumerate}[label=\Alph*.]
\begin{multicols}{2}
\item \includegraphics[width = 0.3\textwidth]{../Figures/rationalEquationToGraphCopyAC.png}
\item \includegraphics[width = 0.3\textwidth]{../Figures/rationalEquationToGraphCopyBC.png}
\item \includegraphics[width = 0.3\textwidth]{../Figures/rationalEquationToGraphCopyCC.png}
\item \includegraphics[width = 0.3\textwidth]{../Figures/rationalEquationToGraphCopyDC.png}
\end{multicols}\item None of the above.\end{enumerate}
\textbf{General Comment:} Remember that the general form of a basic rational equation is $ f(x) = \frac{a}{(x-h)^n} + k$, where $a$ is the leading coefficient (and in this case, we assume is either $1$ or $-1$), $n$ is the degree (in this case, either $1$ or $2$), and $(h, k)$ is the intersection of the asymptotes.
}
\litem{
Determine the domain of the function below.
\[ f(x) = \frac{6}{15x^{2} -38 x + 24} \]The solution is \( \text{All Real numbers except } x = 1.200 \text{ and } x = 1.333. \), which is option E.\begin{enumerate}[label=\Alph*.]
\item \( \text{All Real numbers except } x = a, \text{ where } a \in [11.98, 12.15] \)

All Real numbers except $x = 12.000$, which corresponds to removing a distractor value from the denominator.
\item \( \text{All Real numbers.} \)

This corresponds to thinking the denominator has complex roots or that rational functions have a domain of all Real numbers.
\item \( \text{All Real numbers except } x = a \text{ and } x = b, \text{ where } a \in [11.98, 12.15] \text{ and } b \in [29.98, 30.12] \)

All Real numbers except $x = 12.000$ and $x = 30.000$, which corresponds to not factoring the denominator correctly.
\item \( \text{All Real numbers except } x = a, \text{ where } a \in [1.2, 1.24] \)

All Real numbers except $x = 1.200$, which corresponds to removing only 1 value from the denominator.
\item \( \text{All Real numbers except } x = a \text{ and } x = b, \text{ where } a \in [1.2, 1.24] \text{ and } b \in [1.31, 1.43] \)

All Real numbers except $x = 1.200$ and $x = 1.333$, which is the correct option.
\end{enumerate}

\textbf{General Comment:} Recall that dividing by zero is not a real number. Therefore the domain is all real numbers \textbf{except} those that make the denominator 0.
}
\litem{
Solve the rational equation below. Then, choose the interval(s) that the solution(s) belongs to.
\[ \frac{-8}{-5x + 2} + -9 = \frac{-9}{-45x + 18} \]The solution is \( x = 0.556 \), which is option A.\begin{enumerate}[label=\Alph*.]
\item \( x \in [-1.44,1.56] \)

* $x = 0.556$, which is the correct option.
\item \( x_1 \in [0.23, 0.43] \text{ and } x_2 \in [-0.44,3.56] \)

$x = 0.378 \text{ and } x = 0.556$, which corresponds to getting the correct solution and believing there should be a second solution to the equation.
\item \( \text{All solutions lead to invalid or complex values in the equation.} \)

This corresponds to thinking $x = 0.556$ leads to dividing by zero in the original equation, which it does not.
\item \( x \in [-0.37,-0.15] \)

$x = -0.244$, which corresponds to not distributing the factor $-5x + 2$ correctly when trying to eliminate the fraction.
\item \( x_1 \in [-0.37, -0.15] \text{ and } x_2 \in [-0.44,3.56] \)

$x = -0.244 \text{ and } x = 0.556$, which corresponds to getting the correct solution and believing there should be a second solution to the equation.
\end{enumerate}

\textbf{General Comment:} Distractors are different based on the number of solutions. Remember that after solving, we need to make sure our solution does not make the original equation divide by zero!
}
\litem{
Choose the graph of the equation below.
\[ f(x) = \frac{1}{x + 2} + 3 \]The solution is the graph below, which is option A.
    \begin{center}
        \includegraphics[width=0.3\textwidth]{../Figures/rationalEquationToGraphAC.png}
    \end{center}\begin{enumerate}[label=\Alph*.]
\begin{multicols}{2}
\item \includegraphics[width = 0.3\textwidth]{../Figures/rationalEquationToGraphAC.png}
\item \includegraphics[width = 0.3\textwidth]{../Figures/rationalEquationToGraphBC.png}
\item \includegraphics[width = 0.3\textwidth]{../Figures/rationalEquationToGraphCC.png}
\item \includegraphics[width = 0.3\textwidth]{../Figures/rationalEquationToGraphDC.png}
\end{multicols}\item None of the above.\end{enumerate}
\textbf{General Comment:} Remember that the general form of a basic rational equation is $ f(x) = \frac{a}{(x-h)^n} + k$, where $a$ is the leading coefficient (and in this case, we assume is either $1$ or $-1$), $n$ is the degree (in this case, either $1$ or $2$), and $(h, k)$ is the intersection of the asymptotes.
}
\litem{
Choose the equation of the function graphed below.

\begin{center}
    \includegraphics[width=0.5\textwidth]{../Figures/rationalGraphToEquationC.png}
\end{center}


The solution is \( \text{None of the above as it should be } f(x) = \frac{1}{(x - 1)^2} + 2 \), which is option E.\begin{enumerate}[label=\Alph*.]
\item \( f(x) = \frac{-1}{x - 1} + 0 \)

Corresponds to thinking the graph was a shifted version of $\frac{1}{x}$, using the general form $f(x) = \frac{a}{(x-h)^2}+k$, the opposite leading coefficient, AND not noticing the $y$-value was wrong.
\item \( f(x) = \frac{1}{x + 1} + 0 \)

Corresponds to thinking the graph was a shifted version of $\frac{1}{x}$ AND not noticing the $y$-value was wrong.
\item \( f(x) = \frac{-1}{(x - 1)^2} + 0 \)

Corresponds to using the general form $f(x) = \frac{a}{(x-h)^2}+k$, the opposite leading coefficient, AND not noticing the $y$-value was wrong.
\item \( f(x) = \frac{1}{(x + 1)^2} + 0 \)

The $x$- and $y$-value of the equation does not match the graph.
\item \( \text{None of the above} \)

None of the equation options were the correct equation.
\end{enumerate}

\textbf{General Comment:} Remember that the general form of a basic rational equation is $ f(x) = \frac{a}{(x-h)^n} + k$, where $a$ is the leading coefficient (and in this case, we assume is either $1$ or $-1$), $n$ is the degree (in this case, either $1$ or $2$), and $(h, k)$ is the intersection of the asymptotes.
}
\litem{
Solve the rational equation below. Then, choose the interval(s) that the solution(s) belongs to.
\[ \frac{-7x}{2x -7} + \frac{-7x^{2}}{10x^{2} -39 x + 14} = \frac{-4}{5x -2} \]The solution is \( \text{All solutions are invalid or lead to complex values in the equation.} \), which is option B.\begin{enumerate}[label=\Alph*.]
\item \( x \in [-0.24,0.83] \)

$x = 0.400$, which corresponds to solving $5x -2 = 0$ and treating it as a solution to the equation.
\item \( \text{All solutions lead to invalid or complex values in the equation.} \)

* The equation leads to solving $-28x^{2} +22 x -28=0$, which leads to complex solutions. This is the correct option.
\item \( x_1 \in [2.32, 5.4] \text{ and } x_2 \in [-0.04,0.83] \)

$x = 3.500 \text{ and } x = 0.400$, which corresponds to solving $2x -7 = 0$ and $5x -2 = 0$ and treating them as solutions to the equation.
\item \( x_1 \in [0.7, 2.1] \text{ and } x_2 \in [-0.72,0.38] \)

$x = 1.312 \text{ and } x = -0.527$, which corresponds to making the discriminant from the Quadratic Formula positive to avoid complex solutions.
\item \( x \in [2.32,5.4] \)

$x = 3.500$, which corresponds to solving $2x -7 = 0$ and treating it as a solution to the equation.
\end{enumerate}

\textbf{General Comment:} Distractors are different based on the number of solutions. Remember that after solving, we need to make sure our solution does not make the original equation divide by zero!
}
\litem{
Solve the rational equation below. Then, choose the interval(s) that the solution(s) belongs to.
\[ \frac{-6x}{-5x + 4} + \frac{-4x^{2}}{-20x^{2} -4 x + 16} = \frac{-2}{4x + 4} \]The solution is \( \text{There are two solutions: } x = 0.202 \text{ and } x = -1.416 \), which is option E.\begin{enumerate}[label=\Alph*.]
\item \( x_1 \in [-0.08, 0.53] \text{ and } x_2 \in [-1.2,6.8] \)


\item \( x \in [-1.06,-0.81] \)


\item \( \text{All solutions lead to invalid or complex values in the equation.} \)


\item \( x \in [-1.47,-1.3] \)


\item \( x_1 \in [-0.08, 0.53] \text{ and } x_2 \in [-1.42,0.58] \)

* $x = 0.202 \text{ and } x = -1.416$, which is the correct option.
\end{enumerate}

\textbf{General Comment:} Distractors are different based on the number of solutions. Remember that after solving, we need to make sure our solution does not make the original equation divide by zero!
}
\end{enumerate}

\end{document}