\documentclass{extbook}[14pt]
\usepackage{multicol, enumerate, enumitem, hyperref, color, soul, setspace, parskip, fancyhdr, amssymb, amsthm, amsmath, bbm, latexsym, units, mathtools}
\everymath{\displaystyle}
\usepackage[headsep=0.5cm,headheight=0cm, left=1 in,right= 1 in,top= 1 in,bottom= 1 in]{geometry}
\usepackage{dashrule}  % Package to use the command below to create lines between items
\newcommand{\litem}[1]{\item #1

\rule{\textwidth}{0.4pt}}
\pagestyle{fancy}
\lhead{}
\chead{Answer Key for Progress Quiz 5 Version A}
\rhead{}
\lfoot{9912-2038}
\cfoot{}
\rfoot{Spring 2021}
\begin{document}
\textbf{This key should allow you to understand why you choose the option you did (beyond just getting a question right or wrong). \href{https://xronos.clas.ufl.edu/mac1105spring2020/courseDescriptionAndMisc/Exams/LearningFromResults}{More instructions on how to use this key can be found here}.}

\textbf{If you have a suggestion to make the keys better, \href{https://forms.gle/CZkbZmPbC9XALEE88}{please fill out the short survey here}.}

\textit{Note: This key is auto-generated and may contain issues and/or errors. The keys are reviewed after each exam to ensure grading is done accurately. If there are issues (like duplicate options), they are noted in the offline gradebook. The keys are a work-in-progress to give students as many resources to improve as possible.}

\rule{\textwidth}{0.4pt}

\begin{enumerate}\litem{
Solve the linear inequality below. Then, choose the constant and interval combination that describes the solution set.
\[ -5 + 3 x > 5 x \text{ or } 7 + 6 x < 8 x \]The solution is \( (-\infty, -2.5) \text{ or } (3.5, \infty) \), which is option B.\begin{enumerate}[label=\Alph*.]
\item \( (-\infty, a] \cup [b, \infty), \text{ where } a \in [-2.56, -2.25] \text{ and } b \in [3.39, 3.77] \)

Corresponds to including the endpoints (when they should be excluded).
\item \( (-\infty, a) \cup (b, \infty), \text{ where } a \in [-2.9, -1.8] \text{ and } b \in [2.63, 4.43] \)

 * Correct option.
\item \( (-\infty, a] \cup [b, \infty), \text{ where } a \in [-3.75, -3.25] \text{ and } b \in [2.41, 2.71] \)

Corresponds to including the endpoints AND negating.
\item \( (-\infty, a) \cup (b, \infty), \text{ where } a \in [-5.1, -3.1] \text{ and } b \in [2.37, 3.21] \)

Corresponds to inverting the inequality and negating the solution.
\item \( (-\infty, \infty) \)

Corresponds to the variable canceling, which does not happen in this instance.
\end{enumerate}

\textbf{General Comment:} When multiplying or dividing by a negative, flip the sign.
}
\litem{
Solve the linear inequality below. Then, choose the constant and interval combination that describes the solution set.
\[ \frac{-9}{2} - \frac{9}{7} x \geq \frac{7}{9} x + \frac{5}{5} \]The solution is \( (-\infty, -2.665] \), which is option A.\begin{enumerate}[label=\Alph*.]
\item \( (-\infty, a], \text{ where } a \in [-3.67, 0.33] \)

* $(-\infty, -2.665]$, which is the correct option.
\item \( (-\infty, a], \text{ where } a \in [-0.33, 3.67] \)

 $(-\infty, 2.665]$, which corresponds to negating the endpoint of the solution.
\item \( [a, \infty), \text{ where } a \in [-6.67, 1.33] \)

 $[-2.665, \infty)$, which corresponds to switching the direction of the interval. You likely did this if you did not flip the inequality when dividing by a negative!
\item \( [a, \infty), \text{ where } a \in [1.67, 7.67] \)

 $[2.665, \infty)$, which corresponds to switching the direction of the interval AND negating the endpoint. You likely did this if you did not flip the inequality when dividing by a negative as well as not moving values over to a side properly.
\item \( \text{None of the above}. \)

You may have chosen this if you thought the inequality did not match the ends of the intervals.
\end{enumerate}

\textbf{General Comment:} Remember that less/greater than or equal to includes the endpoint, while less/greater do not. Also, remember that you need to flip the inequality when you multiply or divide by a negative.
}
\litem{
Solve the linear inequality below. Then, choose the constant and interval combination that describes the solution set.
\[ -9 - 4 x < \frac{-30 x - 6}{8} \leq -3 - 5 x \]The solution is \( (-33.00, -1.80] \), which is option B.\begin{enumerate}[label=\Alph*.]
\item \( (-\infty, a) \cup [b, \infty), \text{ where } a \in [-36, -30] \text{ and } b \in [-1.8, 0.2] \)

$(-\infty, -33.00) \cup [-1.80, \infty)$, which corresponds to displaying the and-inequality as an or-inequality.
\item \( (a, b], \text{ where } a \in [-36, -30] \text{ and } b \in [-4.8, 0.2] \)

* $(-33.00, -1.80]$, which is the correct option.
\item \( [a, b), \text{ where } a \in [-37, -32] \text{ and } b \in [-5.8, 1.2] \)

$[-33.00, -1.80)$, which corresponds to flipping the inequality.
\item \( (-\infty, a] \cup (b, \infty), \text{ where } a \in [-36, -29] \text{ and } b \in [-1.8, 1.2] \)

$(-\infty, -33.00] \cup (-1.80, \infty)$, which corresponds to displaying the and-inequality as an or-inequality AND flipping the inequality.
\item \( \text{None of the above.} \)


\end{enumerate}

\textbf{General Comment:} To solve, you will need to break up the compound inequality into two inequalities. Be sure to keep track of the inequality! It may be best to draw a number line and graph your solution.
}
\litem{
Solve the linear inequality below. Then, choose the constant and interval combination that describes the solution set.
\[ 8x -7 < 10x + 10 \]The solution is \( (-8.5, \infty) \), which is option D.\begin{enumerate}[label=\Alph*.]
\item \( (-\infty, a), \text{ where } a \in [5.5, 10.5] \)

 $(-\infty, 8.5)$, which corresponds to switching the direction of the interval AND negating the endpoint. You likely did this if you did not flip the inequality when dividing by a negative as well as not moving values over to a side properly.
\item \( (-\infty, a), \text{ where } a \in [-15.5, -3.5] \)

 $(-\infty, -8.5)$, which corresponds to switching the direction of the interval. You likely did this if you did not flip the inequality when dividing by a negative!
\item \( (a, \infty), \text{ where } a \in [5.5, 12.5] \)

 $(8.5, \infty)$, which corresponds to negating the endpoint of the solution.
\item \( (a, \infty), \text{ where } a \in [-10.5, -2.5] \)

* $(-8.5, \infty)$, which is the correct option.
\item \( \text{None of the above}. \)

You may have chosen this if you thought the inequality did not match the ends of the intervals.
\end{enumerate}

\textbf{General Comment:} Remember that less/greater than or equal to includes the endpoint, while less/greater do not. Also, remember that you need to flip the inequality when you multiply or divide by a negative.
}
\litem{
Using an interval or intervals, describe all the $x$-values within or including a distance of the given values.
\[ \text{ No more than } 4 \text{ units from the number } 9. \]The solution is \( [5, 13] \), which is option A.\begin{enumerate}[label=\Alph*.]
\item \( [5, 13] \)

This describes the values no more than 4 from 9
\item \( (-\infty, 5] \cup [13, \infty) \)

This describes the values no less than 4 from 9
\item \( (-\infty, 5) \cup (13, \infty) \)

This describes the values more than 4 from 9
\item \( (5, 13) \)

This describes the values less than 4 from 9
\item \( \text{None of the above} \)

You likely thought the values in the interval were not correct.
\end{enumerate}

\textbf{General Comment:} When thinking about this language, it helps to draw a number line and try points.
}
\litem{
Solve the linear inequality below. Then, choose the constant and interval combination that describes the solution set.
\[ -10x -4 > -6x -7 \]The solution is \( (-\infty, 0.75) \), which is option C.\begin{enumerate}[label=\Alph*.]
\item \( (a, \infty), \text{ where } a \in [-0.96, -0.47] \)

 $(-0.75, \infty)$, which corresponds to switching the direction of the interval AND negating the endpoint. You likely did this if you did not flip the inequality when dividing by a negative as well as not moving values over to a side properly.
\item \( (-\infty, a), \text{ where } a \in [-4.6, 0.1] \)

 $(-\infty, -0.75)$, which corresponds to negating the endpoint of the solution.
\item \( (-\infty, a), \text{ where } a \in [0.6, 3.1] \)

* $(-\infty, 0.75)$, which is the correct option.
\item \( (a, \infty), \text{ where } a \in [0.62, 0.86] \)

 $(0.75, \infty)$, which corresponds to switching the direction of the interval. You likely did this if you did not flip the inequality when dividing by a negative!
\item \( \text{None of the above}. \)

You may have chosen this if you thought the inequality did not match the ends of the intervals.
\end{enumerate}

\textbf{General Comment:} Remember that less/greater than or equal to includes the endpoint, while less/greater do not. Also, remember that you need to flip the inequality when you multiply or divide by a negative.
}
\litem{
Solve the linear inequality below. Then, choose the constant and interval combination that describes the solution set.
\[ -7 - 8 x < \frac{-22 x - 5}{3} \leq -5 - 8 x \]The solution is \( (-8.00, -5.00] \), which is option A.\begin{enumerate}[label=\Alph*.]
\item \( (a, b], \text{ where } a \in [-8, -3] \text{ and } b \in [-7, 0] \)

* $(-8.00, -5.00]$, which is the correct option.
\item \( [a, b), \text{ where } a \in [-11, -3] \text{ and } b \in [-7, -1] \)

$[-8.00, -5.00)$, which corresponds to flipping the inequality.
\item \( (-\infty, a) \cup [b, \infty), \text{ where } a \in [-13, -3] \text{ and } b \in [-6, -2] \)

$(-\infty, -8.00) \cup [-5.00, \infty)$, which corresponds to displaying the and-inequality as an or-inequality.
\item \( (-\infty, a] \cup (b, \infty), \text{ where } a \in [-9, -7] \text{ and } b \in [-7, -1] \)

$(-\infty, -8.00] \cup (-5.00, \infty)$, which corresponds to displaying the and-inequality as an or-inequality AND flipping the inequality.
\item \( \text{None of the above.} \)


\end{enumerate}

\textbf{General Comment:} To solve, you will need to break up the compound inequality into two inequalities. Be sure to keep track of the inequality! It may be best to draw a number line and graph your solution.
}
\litem{
Solve the linear inequality below. Then, choose the constant and interval combination that describes the solution set.
\[ 3 + 6 x > 9 x \text{ or } 6 + 6 x < 8 x \]The solution is \( (-\infty, 1.0) \text{ or } (3.0, \infty) \), which is option A.\begin{enumerate}[label=\Alph*.]
\item \( (-\infty, a) \cup (b, \infty), \text{ where } a \in [1, 2] \text{ and } b \in [1, 6] \)

 * Correct option.
\item \( (-\infty, a] \cup [b, \infty), \text{ where } a \in [0, 4] \text{ and } b \in [3, 4] \)

Corresponds to including the endpoints (when they should be excluded).
\item \( (-\infty, a) \cup (b, \infty), \text{ where } a \in [-4, -2] \text{ and } b \in [-1, 0] \)

Corresponds to inverting the inequality and negating the solution.
\item \( (-\infty, a] \cup [b, \infty), \text{ where } a \in [-3, 0] \text{ and } b \in [-4, 2] \)

Corresponds to including the endpoints AND negating.
\item \( (-\infty, \infty) \)

Corresponds to the variable canceling, which does not happen in this instance.
\end{enumerate}

\textbf{General Comment:} When multiplying or dividing by a negative, flip the sign.
}
\litem{
Solve the linear inequality below. Then, choose the constant and interval combination that describes the solution set.
\[ \frac{10}{9} + \frac{5}{3} x \leq \frac{10}{5} x - \frac{3}{7} \]The solution is \( [4.619, \infty) \), which is option B.\begin{enumerate}[label=\Alph*.]
\item \( [a, \infty), \text{ where } a \in [-4.62, -2.62] \)

 $[-4.619, \infty)$, which corresponds to negating the endpoint of the solution.
\item \( [a, \infty), \text{ where } a \in [2.62, 7.62] \)

* $[4.619, \infty)$, which is the correct option.
\item \( (-\infty, a], \text{ where } a \in [-7.62, -3.62] \)

 $(-\infty, -4.619]$, which corresponds to switching the direction of the interval AND negating the endpoint. You likely did this if you did not flip the inequality when dividing by a negative as well as not moving values over to a side properly.
\item \( (-\infty, a], \text{ where } a \in [3.62, 5.62] \)

 $(-\infty, 4.619]$, which corresponds to switching the direction of the interval. You likely did this if you did not flip the inequality when dividing by a negative!
\item \( \text{None of the above}. \)

You may have chosen this if you thought the inequality did not match the ends of the intervals.
\end{enumerate}

\textbf{General Comment:} Remember that less/greater than or equal to includes the endpoint, while less/greater do not. Also, remember that you need to flip the inequality when you multiply or divide by a negative.
}
\litem{
Using an interval or intervals, describe all the $x$-values within or including a distance of the given values.
\[ \text{ No more than } 9 \text{ units from the number } 5. \]The solution is \( [-4, 14] \), which is option B.\begin{enumerate}[label=\Alph*.]
\item \( (-\infty, -4] \cup [14, \infty) \)

This describes the values no less than 9 from 5
\item \( [-4, 14] \)

This describes the values no more than 9 from 5
\item \( (-\infty, -4) \cup (14, \infty) \)

This describes the values more than 9 from 5
\item \( (-4, 14) \)

This describes the values less than 9 from 5
\item \( \text{None of the above} \)

You likely thought the values in the interval were not correct.
\end{enumerate}

\textbf{General Comment:} When thinking about this language, it helps to draw a number line and try points.
}
\end{enumerate}

\end{document}