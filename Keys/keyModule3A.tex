\documentclass{extbook}[14pt]
\usepackage{multicol, enumerate, enumitem, hyperref, color, soul, setspace, parskip, fancyhdr, amssymb, amsthm, amsmath, bbm, latexsym, units, mathtools}
\everymath{\displaystyle}
\usepackage[headsep=0.5cm,headheight=0cm, left=1 in,right= 1 in,top= 1 in,bottom= 1 in]{geometry}
\usepackage{dashrule}  % Package to use the command below to create lines between items
\newcommand{\litem}[1]{\item #1

\rule{\textwidth}{0.4pt}}
\pagestyle{fancy}
\lhead{}
\chead{Answer Key for Progress Quiz 2 Version A}
\rhead{}
\lfoot{7862-5421}
\cfoot{}
\rfoot{Spring 2021}
\begin{document}
\textbf{This key should allow you to understand why you choose the option you did (beyond just getting a question right or wrong). \href{https://xronos.clas.ufl.edu/mac1105spring2020/courseDescriptionAndMisc/Exams/LearningFromResults}{More instructions on how to use this key can be found here}.}

\textbf{If you have a suggestion to make the keys better, \href{https://forms.gle/CZkbZmPbC9XALEE88}{please fill out the short survey here}.}

\textit{Note: This key is auto-generated and may contain issues and/or errors. The keys are reviewed after each exam to ensure grading is done accurately. If there are issues (like duplicate options), they are noted in the offline gradebook. The keys are a work-in-progress to give students as many resources to improve as possible.}

\rule{\textwidth}{0.4pt}

\begin{enumerate}\litem{
Solve the linear inequality below. Then, choose the constant and interval combination that describes the solution set.
\[ -5 - 9 x \leq \frac{-76 x + 5}{9} < 3 - 9 x \]

The solution is \( \text{None of the above.} \), which is option E.\begin{enumerate}[label=\Alph*.]
\item \( (-\infty, a) \cup [b, \infty), \text{ where } a \in [8, 14] \text{ and } b \in [-6.4, -0.4] \)

$(-\infty, 10.00) \cup [-4.40, \infty)$, which corresponds to displaying the and-inequality as an or-inequality AND flipping the inequality AND getting negatives of the actual endpoints.
\item \( (-\infty, a] \cup (b, \infty), \text{ where } a \in [9, 12] \text{ and } b \in [-7.4, -2.4] \)

$(-\infty, 10.00] \cup (-4.40, \infty)$, which corresponds to displaying the and-inequality as an or-inequality and getting negatives of the actual endpoints.
\item \( (a, b], \text{ where } a \in [8, 14] \text{ and } b \in [-5.4, -3.4] \)

$(10.00, -4.40]$, which corresponds to flipping the inequality and getting negatives of the actual endpoints.
\item \( [a, b), \text{ where } a \in [7, 12] \text{ and } b \in [-9.4, -2.4] \)

$[10.00, -4.40)$, which is the correct interval but negatives of the actual endpoints.
\item \( \text{None of the above.} \)

* This is correct as the answer should be $[-10.00, 4.40)$.
\end{enumerate}

\textbf{General Comment:} To solve, you will need to break up the compound inequality into two inequalities. Be sure to keep track of the inequality! It may be best to draw a number line and graph your solution.
}
\litem{
Solve the linear inequality below. Then, choose the constant and interval combination that describes the solution set.
\[ -10x -4 > 5x -5 \]

The solution is \( (-\infty, 0.067) \), which is option D.\begin{enumerate}[label=\Alph*.]
\item \( (-\infty, a), \text{ where } a \in [-0.44, 0.04] \)

 $(-\infty, -0.067)$, which corresponds to negating the endpoint of the solution.
\item \( (a, \infty), \text{ where } a \in [-0.89, -0.01] \)

 $(-0.067, \infty)$, which corresponds to switching the direction of the interval AND negating the endpoint. You likely did this if you did not flip the inequality when dividing by a negative as well as not moving values over to a side properly.
\item \( (a, \infty), \text{ where } a \in [0.06, 0.19] \)

 $(0.067, \infty)$, which corresponds to switching the direction of the interval. You likely did this if you did not flip the inequality when dividing by a negative!
\item \( (-\infty, a), \text{ where } a \in [-0.03, 0.14] \)

* $(-\infty, 0.067)$, which is the correct option.
\item \( \text{None of the above}. \)

You may have chosen this if you thought the inequality did not match the ends of the intervals.
\end{enumerate}

\textbf{General Comment:} Remember that less/greater than or equal to includes the endpoint, while less/greater do not. Also, remember that you need to flip the inequality when you multiply or divide by a negative.
}
\litem{
Solve the linear inequality below. Then, choose the constant and interval combination that describes the solution set.
\[ \frac{-9}{7} + \frac{6}{4} x \geq \frac{8}{6} x + \frac{7}{9} \]

The solution is \( [12.381, \infty) \), which is option D.\begin{enumerate}[label=\Alph*.]
\item \( [a, \infty), \text{ where } a \in [-14.38, -7.38] \)

 $[-12.381, \infty)$, which corresponds to negating the endpoint of the solution.
\item \( (-\infty, a], \text{ where } a \in [11.38, 14.38] \)

 $(-\infty, 12.381]$, which corresponds to switching the direction of the interval. You likely did this if you did not flip the inequality when dividing by a negative!
\item \( (-\infty, a], \text{ where } a \in [-13.38, -8.38] \)

 $(-\infty, -12.381]$, which corresponds to switching the direction of the interval AND negating the endpoint. You likely did this if you did not flip the inequality when dividing by a negative as well as not moving values over to a side properly.
\item \( [a, \infty), \text{ where } a \in [12.38, 15.38] \)

* $[12.381, \infty)$, which is the correct option.
\item \( \text{None of the above}. \)

You may have chosen this if you thought the inequality did not match the ends of the intervals.
\end{enumerate}

\textbf{General Comment:} Remember that less/greater than or equal to includes the endpoint, while less/greater do not. Also, remember that you need to flip the inequality when you multiply or divide by a negative.
}
\litem{
Using an interval or intervals, describe all the $x$-values within or including a distance of the given values.
\[ \text{ No less than } 3 \text{ units from the number } 10. \]

The solution is \( \text{None of the above} \), which is option E.\begin{enumerate}[label=\Alph*.]
\item \( (-\infty, -7] \cup [13, \infty) \)

This describes the values no less than 10 from 3
\item \( (-7, 13) \)

This describes the values less than 10 from 3
\item \( [-7, 13] \)

This describes the values no more than 10 from 3
\item \( (-\infty, -7) \cup (13, \infty) \)

This describes the values more than 10 from 3
\item \( \text{None of the above} \)

Options A-D described the values [more/less than] 10 units from 3, which is the reverse of what the question asked.
\end{enumerate}

\textbf{General Comment:} When thinking about this language, it helps to draw a number line and try points.
}
\litem{
Solve the linear inequality below. Then, choose the constant and interval combination that describes the solution set.
\[ -7 + 7 x > 8 x \text{ or } -6 + 6 x < 8 x \]

The solution is \( (-\infty, -7.0) \text{ or } (-3.0, \infty) \), which is option A.\begin{enumerate}[label=\Alph*.]
\item \( (-\infty, a) \cup (b, \infty), \text{ where } a \in [-7, -6] \text{ and } b \in [-4, 1] \)

 * Correct option.
\item \( (-\infty, a] \cup [b, \infty), \text{ where } a \in [3, 4] \text{ and } b \in [5, 11] \)

Corresponds to including the endpoints AND negating.
\item \( (-\infty, a] \cup [b, \infty), \text{ where } a \in [-12, -6] \text{ and } b \in [-5, -1] \)

Corresponds to including the endpoints (when they should be excluded).
\item \( (-\infty, a) \cup (b, \infty), \text{ where } a \in [1, 6] \text{ and } b \in [0, 8] \)

Corresponds to inverting the inequality and negating the solution.
\item \( (-\infty, \infty) \)

Corresponds to the variable canceling, which does not happen in this instance.
\end{enumerate}

\textbf{General Comment:} When multiplying or dividing by a negative, flip the sign.
}
\litem{
Solve the linear inequality below. Then, choose the constant and interval combination that describes the solution set.
\[ -9x + 5 \leq -5x -8 \]

The solution is \( [3.25, \infty) \), which is option B.\begin{enumerate}[label=\Alph*.]
\item \( (-\infty, a], \text{ where } a \in [0.25, 6.25] \)

 $(-\infty, 3.25]$, which corresponds to switching the direction of the interval. You likely did this if you did not flip the inequality when dividing by a negative!
\item \( [a, \infty), \text{ where } a \in [-0.75, 4.25] \)

* $[3.25, \infty)$, which is the correct option.
\item \( [a, \infty), \text{ where } a \in [-3.25, 1.75] \)

 $[-3.25, \infty)$, which corresponds to negating the endpoint of the solution.
\item \( (-\infty, a], \text{ where } a \in [-4.25, -1.25] \)

 $(-\infty, -3.25]$, which corresponds to switching the direction of the interval AND negating the endpoint. You likely did this if you did not flip the inequality when dividing by a negative as well as not moving values over to a side properly.
\item \( \text{None of the above}. \)

You may have chosen this if you thought the inequality did not match the ends of the intervals.
\end{enumerate}

\textbf{General Comment:} Remember that less/greater than or equal to includes the endpoint, while less/greater do not. Also, remember that you need to flip the inequality when you multiply or divide by a negative.
}
\litem{
Solve the linear inequality below. Then, choose the constant and interval combination that describes the solution set.
\[ \frac{10}{2} - \frac{9}{6} x \leq \frac{-5}{5} x + \frac{3}{4} \]

The solution is \( [8.5, \infty) \), which is option A.\begin{enumerate}[label=\Alph*.]
\item \( [a, \infty), \text{ where } a \in [7.5, 10.5] \)

* $[8.5, \infty)$, which is the correct option.
\item \( [a, \infty), \text{ where } a \in [-9.5, -7.5] \)

 $[-8.5, \infty)$, which corresponds to negating the endpoint of the solution.
\item \( (-\infty, a], \text{ where } a \in [6.5, 16.5] \)

 $(-\infty, 8.5]$, which corresponds to switching the direction of the interval. You likely did this if you did not flip the inequality when dividing by a negative!
\item \( (-\infty, a], \text{ where } a \in [-10.5, -5.5] \)

 $(-\infty, -8.5]$, which corresponds to switching the direction of the interval AND negating the endpoint. You likely did this if you did not flip the inequality when dividing by a negative as well as not moving values over to a side properly.
\item \( \text{None of the above}. \)

You may have chosen this if you thought the inequality did not match the ends of the intervals.
\end{enumerate}

\textbf{General Comment:} Remember that less/greater than or equal to includes the endpoint, while less/greater do not. Also, remember that you need to flip the inequality when you multiply or divide by a negative.
}
\litem{
Using an interval or intervals, describe all the $x$-values within or including a distance of the given values.
\[ \text{ More than } 6 \text{ units from the number } 3. \]

The solution is \( \text{None of the above} \), which is option E.\begin{enumerate}[label=\Alph*.]
\item \( (3, 9) \)

This describes the values less than 3 from 6
\item \( (-\infty, 3] \cup [9, \infty) \)

This describes the values no less than 3 from 6
\item \( (-\infty, 3) \cup (9, \infty) \)

This describes the values more than 3 from 6
\item \( [3, 9] \)

This describes the values no more than 3 from 6
\item \( \text{None of the above} \)

Options A-D described the values [more/less than] 3 units from 6, which is the reverse of what the question asked.
\end{enumerate}

\textbf{General Comment:} When thinking about this language, it helps to draw a number line and try points.
}
\litem{
Solve the linear inequality below. Then, choose the constant and interval combination that describes the solution set.
\[ -7 - 5 x < \frac{-40 x - 3}{9} \leq 9 - 8 x \]

The solution is \( (-12.00, 2.62] \), which is option C.\begin{enumerate}[label=\Alph*.]
\item \( [a, b), \text{ where } a \in [-18, -9] \text{ and } b \in [0.62, 7.62] \)

$[-12.00, 2.62)$, which corresponds to flipping the inequality.
\item \( (-\infty, a] \cup (b, \infty), \text{ where } a \in [-13, -8] \text{ and } b \in [-0.38, 4.62] \)

$(-\infty, -12.00] \cup (2.62, \infty)$, which corresponds to displaying the and-inequality as an or-inequality AND flipping the inequality.
\item \( (a, b], \text{ where } a \in [-13, -7] \text{ and } b \in [0.62, 3.62] \)

* $(-12.00, 2.62]$, which is the correct option.
\item \( (-\infty, a) \cup [b, \infty), \text{ where } a \in [-15, -10] \text{ and } b \in [1.62, 3.62] \)

$(-\infty, -12.00) \cup [2.62, \infty)$, which corresponds to displaying the and-inequality as an or-inequality.
\item \( \text{None of the above.} \)


\end{enumerate}

\textbf{General Comment:} To solve, you will need to break up the compound inequality into two inequalities. Be sure to keep track of the inequality! It may be best to draw a number line and graph your solution.
}
\litem{
Solve the linear inequality below. Then, choose the constant and interval combination that describes the solution set.
\[ -4 + 3 x > 5 x \text{ or } 8 + 7 x < 8 x \]

The solution is \( (-\infty, -2.0) \text{ or } (8.0, \infty) \), which is option C.\begin{enumerate}[label=\Alph*.]
\item \( (-\infty, a] \cup [b, \infty), \text{ where } a \in [-5, 0] \text{ and } b \in [6, 9] \)

Corresponds to including the endpoints (when they should be excluded).
\item \( (-\infty, a] \cup [b, \infty), \text{ where } a \in [-11, -7] \text{ and } b \in [-3, 6] \)

Corresponds to including the endpoints AND negating.
\item \( (-\infty, a) \cup (b, \infty), \text{ where } a \in [-6, 4] \text{ and } b \in [6, 12] \)

 * Correct option.
\item \( (-\infty, a) \cup (b, \infty), \text{ where } a \in [-9, -6] \text{ and } b \in [2, 5] \)

Corresponds to inverting the inequality and negating the solution.
\item \( (-\infty, \infty) \)

Corresponds to the variable canceling, which does not happen in this instance.
\end{enumerate}

\textbf{General Comment:} When multiplying or dividing by a negative, flip the sign.
}
\end{enumerate}

\end{document}