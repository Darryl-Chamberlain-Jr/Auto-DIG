\documentclass{extbook}[14pt]
\usepackage{multicol, enumerate, enumitem, hyperref, color, soul, setspace, parskip, fancyhdr, amssymb, amsthm, amsmath, latexsym, units, mathtools}
\everymath{\displaystyle}
\usepackage[headsep=0.5cm,headheight=0cm, left=1 in,right= 1 in,top= 1 in,bottom= 1 in]{geometry}
\usepackage{dashrule}  % Package to use the command below to create lines between items
\newcommand{\litem}[1]{\item #1

\rule{\textwidth}{0.4pt}}
\pagestyle{fancy}
\lhead{}
\chead{Answer Key for Progress Quiz 6 Version A}
\rhead{}
\lfoot{9689-6866}
\cfoot{}
\rfoot{Spring 2021}
\begin{document}
\textbf{This key should allow you to understand why you choose the option you did (beyond just getting a question right or wrong). \href{https://xronos.clas.ufl.edu/mac1105spring2020/courseDescriptionAndMisc/Exams/LearningFromResults}{More instructions on how to use this key can be found here}.}

\textbf{If you have a suggestion to make the keys better, \href{https://forms.gle/CZkbZmPbC9XALEE88}{please fill out the short survey here}.}

\textit{Note: This key is auto-generated and may contain issues and/or errors. The keys are reviewed after each exam to ensure grading is done accurately. If there are issues (like duplicate options), they are noted in the offline gradebook. The keys are a work-in-progress to give students as many resources to improve as possible.}

\rule{\textwidth}{0.4pt}

\begin{enumerate}\litem{
Solve the linear inequality below. Then, choose the constant and interval combination that describes the solution set.
\[ -4x -3 \geq 6x + 3 \]The solution is \( (-\infty, -0.6] \), which is option C.\begin{enumerate}[label=\Alph*.]
\item \( [a, \infty), \text{ where } a \in [-1.6, 0.4] \)

 $[-0.6, \infty)$, which corresponds to switching the direction of the interval. You likely did this if you did not flip the inequality when dividing by a negative!
\item \( (-\infty, a], \text{ where } a \in [0.17, 1.19] \)

 $(-\infty, 0.6]$, which corresponds to negating the endpoint of the solution.
\item \( (-\infty, a], \text{ where } a \in [-1.12, -0.25] \)

* $(-\infty, -0.6]$, which is the correct option.
\item \( [a, \infty), \text{ where } a \in [0.6, 3.6] \)

 $[0.6, \infty)$, which corresponds to switching the direction of the interval AND negating the endpoint. You likely did this if you did not flip the inequality when dividing by a negative as well as not moving values over to a side properly.
\item \( \text{None of the above}. \)

You may have chosen this if you thought the inequality did not match the ends of the intervals.
\end{enumerate}

\textbf{General Comment:} Remember that less/greater than or equal to includes the endpoint, while less/greater do not. Also, remember that you need to flip the inequality when you multiply or divide by a negative.
}
\litem{
Solve the linear inequality below. Then, choose the constant and interval combination that describes the solution set.
\[ -6 + 3 x > 6 x \text{ or } 3 + 7 x < 8 x \]The solution is \( (-\infty, -2.0) \text{ or } (3.0, \infty) \), which is option B.\begin{enumerate}[label=\Alph*.]
\item \( (-\infty, a] \cup [b, \infty), \text{ where } a \in [-2.05, -1.97] \text{ and } b \in [2.37, 3.17] \)

Corresponds to including the endpoints (when they should be excluded).
\item \( (-\infty, a) \cup (b, \infty), \text{ where } a \in [-2.16, -0.58] \text{ and } b \in [2.34, 3.44] \)

 * Correct option.
\item \( (-\infty, a) \cup (b, \infty), \text{ where } a \in [-3.91, -2.75] \text{ and } b \in [1.99, 2.05] \)

Corresponds to inverting the inequality and negating the solution.
\item \( (-\infty, a] \cup [b, \infty), \text{ where } a \in [-3.24, -2.72] \text{ and } b \in [0.99, 2.39] \)

Corresponds to including the endpoints AND negating.
\item \( (-\infty, \infty) \)

Corresponds to the variable canceling, which does not happen in this instance.
\end{enumerate}

\textbf{General Comment:} When multiplying or dividing by a negative, flip the sign.
}
\litem{
Solve the linear inequality below. Then, choose the constant and interval combination that describes the solution set.
\[ -6 + 6 x > 7 x \text{ or } -5 + 8 x < 9 x \]The solution is \( (-\infty, -6.0) \text{ or } (-5.0, \infty) \), which is option D.\begin{enumerate}[label=\Alph*.]
\item \( (-\infty, a] \cup [b, \infty), \text{ where } a \in [5, 10] \text{ and } b \in [5, 7] \)

Corresponds to including the endpoints AND negating.
\item \( (-\infty, a] \cup [b, \infty), \text{ where } a \in [-10, -3] \text{ and } b \in [-12, -4] \)

Corresponds to including the endpoints (when they should be excluded).
\item \( (-\infty, a) \cup (b, \infty), \text{ where } a \in [2, 13] \text{ and } b \in [6, 14] \)

Corresponds to inverting the inequality and negating the solution.
\item \( (-\infty, a) \cup (b, \infty), \text{ where } a \in [-9, -5] \text{ and } b \in [-5, 1] \)

 * Correct option.
\item \( (-\infty, \infty) \)

Corresponds to the variable canceling, which does not happen in this instance.
\end{enumerate}

\textbf{General Comment:} When multiplying or dividing by a negative, flip the sign.
}
\litem{
Solve the linear inequality below. Then, choose the constant and interval combination that describes the solution set.
\[ \frac{-7}{3} - \frac{8}{7} x \geq \frac{6}{5} x + \frac{3}{6} \]The solution is \( (-\infty, -1.209] \), which is option B.\begin{enumerate}[label=\Alph*.]
\item \( [a, \infty), \text{ where } a \in [-0.79, 2.21] \)

 $[1.209, \infty)$, which corresponds to switching the direction of the interval AND negating the endpoint. You likely did this if you did not flip the inequality when dividing by a negative as well as not moving values over to a side properly.
\item \( (-\infty, a], \text{ where } a \in [-4.21, 0.79] \)

* $(-\infty, -1.209]$, which is the correct option.
\item \( [a, \infty), \text{ where } a \in [-1.21, -0.21] \)

 $[-1.209, \infty)$, which corresponds to switching the direction of the interval. You likely did this if you did not flip the inequality when dividing by a negative!
\item \( (-\infty, a], \text{ where } a \in [0.21, 3.21] \)

 $(-\infty, 1.209]$, which corresponds to negating the endpoint of the solution.
\item \( \text{None of the above}. \)

You may have chosen this if you thought the inequality did not match the ends of the intervals.
\end{enumerate}

\textbf{General Comment:} Remember that less/greater than or equal to includes the endpoint, while less/greater do not. Also, remember that you need to flip the inequality when you multiply or divide by a negative.
}
\litem{
Solve the linear inequality below. Then, choose the constant and interval combination that describes the solution set.
\[ -10x -10 \leq 10x + 7 \]The solution is \( [-0.85, \infty) \), which is option B.\begin{enumerate}[label=\Alph*.]
\item \( (-\infty, a], \text{ where } a \in [-2.7, 0.4] \)

 $(-\infty, -0.85]$, which corresponds to switching the direction of the interval. You likely did this if you did not flip the inequality when dividing by a negative!
\item \( [a, \infty), \text{ where } a \in [-2.5, 0.2] \)

* $[-0.85, \infty)$, which is the correct option.
\item \( (-\infty, a], \text{ where } a \in [-0.5, 2.4] \)

 $(-\infty, 0.85]$, which corresponds to switching the direction of the interval AND negating the endpoint. You likely did this if you did not flip the inequality when dividing by a negative as well as not moving values over to a side properly.
\item \( [a, \infty), \text{ where } a \in [-0.8, 2.4] \)

 $[0.85, \infty)$, which corresponds to negating the endpoint of the solution.
\item \( \text{None of the above}. \)

You may have chosen this if you thought the inequality did not match the ends of the intervals.
\end{enumerate}

\textbf{General Comment:} Remember that less/greater than or equal to includes the endpoint, while less/greater do not. Also, remember that you need to flip the inequality when you multiply or divide by a negative.
}
\litem{
Using an interval or intervals, describe all the $x$-values within or including a distance of the given values.
\[ \text{ Less than } 8 \text{ units from the number } -3. \]The solution is \( (-11, 5) \), which is option B.\begin{enumerate}[label=\Alph*.]
\item \( [-11, 5] \)

This describes the values no more than 8 from -3
\item \( (-11, 5) \)

This describes the values less than 8 from -3
\item \( (-\infty, -11] \cup [5, \infty) \)

This describes the values no less than 8 from -3
\item \( (-\infty, -11) \cup (5, \infty) \)

This describes the values more than 8 from -3
\item \( \text{None of the above} \)

You likely thought the values in the interval were not correct.
\end{enumerate}

\textbf{General Comment:} When thinking about this language, it helps to draw a number line and try points.
}
\litem{
Solve the linear inequality below. Then, choose the constant and interval combination that describes the solution set.
\[ -8 + 3 x \leq \frac{23 x - 4}{7} < -6 + 3 x \]The solution is \( \text{None of the above.} \), which is option E.\begin{enumerate}[label=\Alph*.]
\item \( (a, b], \text{ where } a \in [26, 27] \text{ and } b \in [19, 21] \)

$(26.00, 19.00]$, which corresponds to flipping the inequality and getting negatives of the actual endpoints.
\item \( [a, b), \text{ where } a \in [26, 27] \text{ and } b \in [19, 24] \)

$[26.00, 19.00)$, which is the correct interval but negatives of the actual endpoints.
\item \( (-\infty, a) \cup [b, \infty), \text{ where } a \in [23, 28] \text{ and } b \in [19, 21] \)

$(-\infty, 26.00) \cup [19.00, \infty)$, which corresponds to displaying the and-inequality as an or-inequality AND flipping the inequality AND getting negatives of the actual endpoints.
\item \( (-\infty, a] \cup (b, \infty), \text{ where } a \in [23, 29] \text{ and } b \in [19, 23] \)

$(-\infty, 26.00] \cup (19.00, \infty)$, which corresponds to displaying the and-inequality as an or-inequality and getting negatives of the actual endpoints.
\item \( \text{None of the above.} \)

* This is correct as the answer should be $[-26.00, -19.00)$.
\end{enumerate}

\textbf{General Comment:} To solve, you will need to break up the compound inequality into two inequalities. Be sure to keep track of the inequality! It may be best to draw a number line and graph your solution.
}
\litem{
Solve the linear inequality below. Then, choose the constant and interval combination that describes the solution set.
\[ -7 + 9 x \leq \frac{84 x + 3}{9} < -9 + 5 x \]The solution is \( [-22.00, -2.15) \), which is option C.\begin{enumerate}[label=\Alph*.]
\item \( (-\infty, a) \cup [b, \infty), \text{ where } a \in [-22, -17] \text{ and } b \in [-5.15, 0.85] \)

$(-\infty, -22.00) \cup [-2.15, \infty)$, which corresponds to displaying the and-inequality as an or-inequality AND flipping the inequality.
\item \( (a, b], \text{ where } a \in [-26, -21] \text{ and } b \in [-3.15, 1.85] \)

$(-22.00, -2.15]$, which corresponds to flipping the inequality.
\item \( [a, b), \text{ where } a \in [-23, -19] \text{ and } b \in [-4.15, -0.15] \)

$[-22.00, -2.15)$, which is the correct option.
\item \( (-\infty, a] \cup (b, \infty), \text{ where } a \in [-22, -19] \text{ and } b \in [-8.15, 1.85] \)

$(-\infty, -22.00] \cup (-2.15, \infty)$, which corresponds to displaying the and-inequality as an or-inequality.
\item \( \text{None of the above.} \)


\end{enumerate}

\textbf{General Comment:} To solve, you will need to break up the compound inequality into two inequalities. Be sure to keep track of the inequality! It may be best to draw a number line and graph your solution.
}
\litem{
Solve the linear inequality below. Then, choose the constant and interval combination that describes the solution set.
\[ \frac{-10}{8} - \frac{10}{9} x > \frac{-9}{4} x + \frac{6}{6} \]The solution is \( (1.976, \infty) \), which is option A.\begin{enumerate}[label=\Alph*.]
\item \( (a, \infty), \text{ where } a \in [0.98, 5.98] \)

* $(1.976, \infty)$, which is the correct option.
\item \( (-\infty, a), \text{ where } a \in [-0.02, 6.98] \)

 $(-\infty, 1.976)$, which corresponds to switching the direction of the interval. You likely did this if you did not flip the inequality when dividing by a negative!
\item \( (a, \infty), \text{ where } a \in [-2.98, 1.02] \)

 $(-1.976, \infty)$, which corresponds to negating the endpoint of the solution.
\item \( (-\infty, a), \text{ where } a \in [-3.98, -0.98] \)

 $(-\infty, -1.976)$, which corresponds to switching the direction of the interval AND negating the endpoint. You likely did this if you did not flip the inequality when dividing by a negative as well as not moving values over to a side properly.
\item \( \text{None of the above}. \)

You may have chosen this if you thought the inequality did not match the ends of the intervals.
\end{enumerate}

\textbf{General Comment:} Remember that less/greater than or equal to includes the endpoint, while less/greater do not. Also, remember that you need to flip the inequality when you multiply or divide by a negative.
}
\litem{
Using an interval or intervals, describe all the $x$-values within or including a distance of the given values.
\[ \text{ More than } 7 \text{ units from the number } -9. \]The solution is \( (-\infty, -16) \cup (-2, \infty) \), which is option C.\begin{enumerate}[label=\Alph*.]
\item \( (-16, -2) \)

This describes the values less than 7 from -9
\item \( [-16, -2] \)

This describes the values no more than 7 from -9
\item \( (-\infty, -16) \cup (-2, \infty) \)

This describes the values more than 7 from -9
\item \( (-\infty, -16] \cup [-2, \infty) \)

This describes the values no less than 7 from -9
\item \( \text{None of the above} \)

You likely thought the values in the interval were not correct.
\end{enumerate}

\textbf{General Comment:} When thinking about this language, it helps to draw a number line and try points.
}
\end{enumerate}

\end{document}