\documentclass{extbook}[14pt]
\usepackage{multicol, enumerate, enumitem, hyperref, color, soul, setspace, parskip, fancyhdr, amssymb, amsthm, amsmath, bbm, latexsym, units, mathtools}
\everymath{\displaystyle}
\usepackage[headsep=0.5cm,headheight=0cm, left=1 in,right= 1 in,top= 1 in,bottom= 1 in]{geometry}
\pagestyle{fancy}
\lhead{}
\chead{Answer Key for Module\,3\,-\,Inequalities Version A}
\rhead{}
\lfoot{Summer\,C\,2020}
\cfoot{}
\rfoot{}
\begin{document}
\textbf{This key should allow you to understand why you choose the option you did (beyond just getting a question right or wrong). \href{https://xronos.clas.ufl.edu/mac1105spring2020/courseDescriptionAndMisc/Exams/LearningFromResults}{More instructions on how to use this key can be found here}.}

\textbf{If you have a suggestion to make the keys better, \href{https://forms.gle/CZkbZmPbC9XALEE88}{please fill out the short survey here}.}

\textit{Note: This key is auto-generated and may contain issues and/or errors. The keys are reviewed after each exam to ensure grading is done accurately. If there are issues (like duplicate options), they are noted in the offline gradebook. The keys are a work-in-progress to give students as many resources to improve as possible.}

\rule{\textwidth}{0.4pt}

11. Solve the linear inequality below. Then, choose the constant and interval combination that describes the solution set.
\[ -9x + 7 \leq 7x + 9 \] 
The solution is $ [-0.125, \infty) $ 

\begin{enumerate}[label=\Alph*.] 
\item $ (-\infty, a], \text{ where } a \in [0.04, 0.4] $ 

  $(-\infty, 0.125]$, which corresponds to switching the direction of the interval AND negating the endpoint. You likely did this if you did not flip the inequality when dividing by a negative as well as not moving values over to a side properly. 
\item $ (-\infty, a], \text{ where } a \in [-0.17, 0.04] $ 

  $(-\infty, -0.125]$, which corresponds to switching the direction of the interval. You likely did this if you did not flip the inequality when dividing by a negative! 
\item $ [a, \infty), \text{ where } a \in [-0.18, -0.06] $ 

 * $[-0.125, \infty)$, which is the correct option. 
\item $ [a, \infty), \text{ where } a \in [0.04, 0.23] $ 

  $[0.125, \infty)$, which corresponds to negating the endpoint of the solution. 
\item $ \text{None of the above}. $ 

 You may have chosen this if you thought the inequality did not match the ends of the intervals. 
\end{enumerate} 
 
General Comments: Remember that less/greater than or equal to includes the endpoint, while less/greater do not. Also, remember that you need to flip the inequality when you multiply or divide by a negative.

-----------------------------------------------

12. Solve the linear inequality below. Then, choose the constant and interval combination that describes the solution set.
\[ -3 + 4 x > 6 x \text{ or } 7 + 7 x < 9 x \] 
The solution is $ (-\infty, -1.5) \text{ or } (3.5, \infty) $ 

\begin{enumerate}[label=\Alph*.] 
\item $ (-\infty, a] \cup [b, \infty), \text{ where } a \in [-7, -2] \text{ and } b \in [0.8, 2.5] $ 

 Corresponds to including the endpoints AND negating. 
\item $ (-\infty, a] \cup [b, \infty), \text{ where } a \in [-2, 0] \text{ and } b \in [3.1, 6] $ 

 Corresponds to including the endpoints (when they should be excluded). 
\item $ (-\infty, a) \cup (b, \infty), \text{ where } a \in [-5, -2] \text{ and } b \in [0, 2] $ 

 Corresponds to inverting the inequality and negating the solution. 
\item $ (-\infty, a) \cup (b, \infty), \text{ where } a \in [-3, 0] \text{ and } b \in [3, 5] $ 

  * Correct option. 
\item $ (-\infty, \infty) $ 

 Corresponds to the variable canceling, which does not happen in this instance. 
\end{enumerate} 
 
General Comments: When multiplying or dividing by a negative, flip the sign.

-----------------------------------------------

13. Solve the linear inequality below. Then, choose the constant and interval combination that describes the solution set.
\[ -9 + 5 x < \frac{23 x + 3}{4} \leq 7 + 4 x \] 
The solution is $ \text{None of the above.} $ 

\begin{enumerate}[label=\Alph*.] 
\item $ (a, b], \text{ where } a \in [7, 18] \text{ and } b \in [-4, -1] $ 

 $(13.00, -3.57]$, which is the correct interval but negatives of the actual endpoints. 
\item $ [a, b), \text{ where } a \in [10, 19] \text{ and } b \in [-5, 0] $ 

 $[13.00, -3.57)$, which corresponds to flipping the inequality and getting negatives of the actual endpoints. 
\item $ (-\infty, a) \cup [b, \infty), \text{ where } a \in [11, 14] \text{ and } b \in [-8, 0] $ 

 $(-\infty, 13.00) \cup [-3.57, \infty)$, which corresponds to displaying the and-inequality as an or-inequality and getting negatives of the actual endpoints. 
\item $ (-\infty, a] \cup (b, \infty), \text{ where } a \in [12, 18] \text{ and } b \in [-5, 1] $ 

 $(-\infty, 13.00] \cup (-3.57, \infty)$, which corresponds to displaying the and-inequality as an or-inequality AND flipping the inequality AND getting negatives of the actual endpoints. 
\item $ \text{None of the above.} $ 

 * This is correct as the answer should be $(-13.00, 3.57]$. 
\end{enumerate} 
 
To solve, you will need to break up the compound inequality into two inequalities. Be sure to keep track of the inequality! It may be best to draw a number line and graph your solution.

-----------------------------------------------

14. Solve the linear inequality below. Then, choose the constant and interval combination that describes the solution set.
\[ \frac{8}{6} - \frac{9}{5} x > \frac{5}{9} x - \frac{10}{3} \] 
The solution is $ (-\infty, 1.981) $ 

\begin{enumerate}[label=\Alph*.] 
\item $ (a, \infty), \text{ where } a \in [0, 4] $ 

  $(1.981, \infty)$, which corresponds to switching the direction of the interval. You likely did this if you did not flip the inequality when dividing by a negative! 
\item $ (-\infty, a), \text{ where } a \in [0, 4] $ 

 * $(-\infty, 1.981)$, which is the correct option. 
\item $ (-\infty, a), \text{ where } a \in [-3, 0] $ 

  $(-\infty, -1.981)$, which corresponds to negating the endpoint of the solution. 
\item $ (a, \infty), \text{ where } a \in [-3, 1] $ 

  $(-1.981, \infty)$, which corresponds to switching the direction of the interval AND negating the endpoint. You likely did this if you did not flip the inequality when dividing by a negative as well as not moving values over to a side properly. 
\item $ \text{None of the above}. $ 

 You may have chosen this if you thought the inequality did not match the ends of the intervals. 
\end{enumerate} 
 
General Comments: Remember that less/greater than or equal to includes the endpoint, while less/greater do not. Also, remember that you need to flip the inequality when you multiply or divide by a negative.

-----------------------------------------------

15. Using an interval or intervals, describe all the $x$-values within or including a distance of the given values.
\[ \text{ More than } 6 \text{ units from the number } 1. \] 
The solution is $ \text{None of the above} $ 

\begin{enumerate}[label=\Alph*.] 
\item $ (-\infty, 5) \cup (7, \infty) $ 

 This describes the values more than 1 from 6 
\item $ (5, 7) $ 

 This describes the values less than 1 from 6 
\item $ [5, 7] $ 

 This describes the values no more than 1 from 6 
\item $ (-\infty, 5] \cup [7, \infty) $ 

 This describes the values no less than 1 from 6 
\item $ \text{None of the above} $ 

 Options A-D described the values [more/less than] 1 units from 6, which is the reverse of what the question asked. 
\end{enumerate} 
 
General Comments: When thinking about this language, it helps to draw a number line and try points.

-----------------------------------------------


\end{document}

