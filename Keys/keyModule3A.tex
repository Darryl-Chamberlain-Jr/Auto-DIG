\documentclass{extbook}[14pt]
\usepackage{multicol, enumerate, enumitem, hyperref, color, soul, setspace, parskip, fancyhdr, amssymb, amsthm, amsmath, latexsym, units, mathtools}
\everymath{\displaystyle}
\usepackage[headsep=0.5cm,headheight=0cm, left=1 in,right= 1 in,top= 1 in,bottom= 1 in]{geometry}
\usepackage{dashrule}  % Package to use the command below to create lines between items
\newcommand{\litem}[1]{\item #1

\rule{\textwidth}{0.4pt}}
\pagestyle{fancy}
\lhead{}
\chead{Answer Key for Makeup Progress Quiz 3 Version A}
\rhead{}
\lfoot{1648-1753}
\cfoot{}
\rfoot{Summer C 2021}
\begin{document}
\textbf{This key should allow you to understand why you choose the option you did (beyond just getting a question right or wrong). \href{https://xronos.clas.ufl.edu/mac1105spring2020/courseDescriptionAndMisc/Exams/LearningFromResults}{More instructions on how to use this key can be found here}.}

\textbf{If you have a suggestion to make the keys better, \href{https://forms.gle/CZkbZmPbC9XALEE88}{please fill out the short survey here}.}

\textit{Note: This key is auto-generated and may contain issues and/or errors. The keys are reviewed after each exam to ensure grading is done accurately. If there are issues (like duplicate options), they are noted in the offline gradebook. The keys are a work-in-progress to give students as many resources to improve as possible.}

\rule{\textwidth}{0.4pt}

\begin{enumerate}\litem{
Using an interval or intervals, describe all the $x$-values within or including a distance of the given values.
\[ \text{ More than } 10 \text{ units from the number } 3. \]The solution is \( (-\infty, -7) \cup (13, \infty) \), which is option C.\begin{enumerate}[label=\Alph*.]
\item \( [-7, 13] \)

This describes the values no more than 10 from 3
\item \( (-\infty, -7] \cup [13, \infty) \)

This describes the values no less than 10 from 3
\item \( (-\infty, -7) \cup (13, \infty) \)

This describes the values more than 10 from 3
\item \( (-7, 13) \)

This describes the values less than 10 from 3
\item \( \text{None of the above} \)

You likely thought the values in the interval were not correct.
\end{enumerate}

\textbf{General Comment:} When thinking about this language, it helps to draw a number line and try points.
}
\litem{
Solve the linear inequality below. Then, choose the constant and interval combination that describes the solution set.
\[ -9x + 5 \leq 5x + 4 \]The solution is \( [0.071, \infty) \), which is option B.\begin{enumerate}[label=\Alph*.]
\item \( (-\infty, a], \text{ where } a \in [-0.63, 0.04] \)

 $(-\infty, -0.071]$, which corresponds to switching the direction of the interval AND negating the endpoint. You likely did this if you did not flip the inequality when dividing by a negative as well as not moving values over to a side properly.
\item \( [a, \infty), \text{ where } a \in [0.03, 0.34] \)

* $[0.071, \infty)$, which is the correct option.
\item \( [a, \infty), \text{ where } a \in [-0.29, -0.03] \)

 $[-0.071, \infty)$, which corresponds to negating the endpoint of the solution.
\item \( (-\infty, a], \text{ where } a \in [0.05, 0.16] \)

 $(-\infty, 0.071]$, which corresponds to switching the direction of the interval. You likely did this if you did not flip the inequality when dividing by a negative!
\item \( \text{None of the above}. \)

You may have chosen this if you thought the inequality did not match the ends of the intervals.
\end{enumerate}

\textbf{General Comment:} Remember that less/greater than or equal to includes the endpoint, while less/greater do not. Also, remember that you need to flip the inequality when you multiply or divide by a negative.
}
\litem{
Solve the linear inequality below. Then, choose the constant and interval combination that describes the solution set.
\[ -9 + 4 x < \frac{37 x - 8}{8} \leq -7 + 3 x \]The solution is \( \text{None of the above.} \), which is option E.\begin{enumerate}[label=\Alph*.]
\item \( (-\infty, a] \cup (b, \infty), \text{ where } a \in [8.25, 14.25] \text{ and } b \in [2.25, 8.25] \)

$(-\infty, 12.80] \cup (3.69, \infty)$, which corresponds to displaying the and-inequality as an or-inequality AND flipping the inequality AND getting negatives of the actual endpoints.
\item \( (a, b], \text{ where } a \in [8.25, 17.25] \text{ and } b \in [3, 5.25] \)

$(12.80, 3.69]$, which is the correct interval but negatives of the actual endpoints.
\item \( [a, b), \text{ where } a \in [9.75, 15] \text{ and } b \in [2.25, 4.5] \)

$[12.80, 3.69)$, which corresponds to flipping the inequality and getting negatives of the actual endpoints.
\item \( (-\infty, a) \cup [b, \infty), \text{ where } a \in [11.25, 14.25] \text{ and } b \in [-1.5, 7.5] \)

$(-\infty, 12.80) \cup [3.69, \infty)$, which corresponds to displaying the and-inequality as an or-inequality and getting negatives of the actual endpoints.
\item \( \text{None of the above.} \)

* This is correct as the answer should be $(-12.80, -3.69]$.
\end{enumerate}

\textbf{General Comment:} To solve, you will need to break up the compound inequality into two inequalities. Be sure to keep track of the inequality! It may be best to draw a number line and graph your solution.
}
\litem{
Solve the linear inequality below. Then, choose the constant and interval combination that describes the solution set.
\[ -7 + 7 x > 9 x \text{ or } 9 + 3 x < 4 x \]The solution is \( (-\infty, -3.5) \text{ or } (9.0, \infty) \), which is option C.\begin{enumerate}[label=\Alph*.]
\item \( (-\infty, a] \cup [b, \infty), \text{ where } a \in [-10.5, -6] \text{ and } b \in [2.25, 7.5] \)

Corresponds to including the endpoints AND negating.
\item \( (-\infty, a] \cup [b, \infty), \text{ where } a \in [-4.5, 0] \text{ and } b \in [8.25, 9.75] \)

Corresponds to including the endpoints (when they should be excluded).
\item \( (-\infty, a) \cup (b, \infty), \text{ where } a \in [-6.75, -3] \text{ and } b \in [8.25, 10.5] \)

 * Correct option.
\item \( (-\infty, a) \cup (b, \infty), \text{ where } a \in [-13.5, -7.5] \text{ and } b \in [0, 4.5] \)

Corresponds to inverting the inequality and negating the solution.
\item \( (-\infty, \infty) \)

Corresponds to the variable canceling, which does not happen in this instance.
\end{enumerate}

\textbf{General Comment:} When multiplying or dividing by a negative, flip the sign.
}
\litem{
Solve the linear inequality below. Then, choose the constant and interval combination that describes the solution set.
\[ \frac{-7}{2} - \frac{9}{8} x \leq \frac{5}{9} x - \frac{4}{5} \]The solution is \( [-1.607, \infty) \), which is option A.\begin{enumerate}[label=\Alph*.]
\item \( [a, \infty), \text{ where } a \in [-4.5, 0.75] \)

* $[-1.607, \infty)$, which is the correct option.
\item \( (-\infty, a], \text{ where } a \in [0.75, 5.25] \)

 $(-\infty, 1.607]$, which corresponds to switching the direction of the interval AND negating the endpoint. You likely did this if you did not flip the inequality when dividing by a negative as well as not moving values over to a side properly.
\item \( [a, \infty), \text{ where } a \in [-1.5, 3] \)

 $[1.607, \infty)$, which corresponds to negating the endpoint of the solution.
\item \( (-\infty, a], \text{ where } a \in [-6, -0.75] \)

 $(-\infty, -1.607]$, which corresponds to switching the direction of the interval. You likely did this if you did not flip the inequality when dividing by a negative!
\item \( \text{None of the above}. \)

You may have chosen this if you thought the inequality did not match the ends of the intervals.
\end{enumerate}

\textbf{General Comment:} Remember that less/greater than or equal to includes the endpoint, while less/greater do not. Also, remember that you need to flip the inequality when you multiply or divide by a negative.
}
\litem{
Solve the linear inequality below. Then, choose the constant and interval combination that describes the solution set.
\[ \frac{-8}{7} + \frac{6}{2} x > \frac{7}{9} x + \frac{7}{5} \]The solution is \( (1.144, \infty) \), which is option A.\begin{enumerate}[label=\Alph*.]
\item \( (a, \infty), \text{ where } a \in [0, 4.5] \)

* $(1.144, \infty)$, which is the correct option.
\item \( (-\infty, a), \text{ where } a \in [-0.75, 4.5] \)

 $(-\infty, 1.144)$, which corresponds to switching the direction of the interval. You likely did this if you did not flip the inequality when dividing by a negative!
\item \( (a, \infty), \text{ where } a \in [-3, -0.75] \)

 $(-1.144, \infty)$, which corresponds to negating the endpoint of the solution.
\item \( (-\infty, a), \text{ where } a \in [-4.5, 0] \)

 $(-\infty, -1.144)$, which corresponds to switching the direction of the interval AND negating the endpoint. You likely did this if you did not flip the inequality when dividing by a negative as well as not moving values over to a side properly.
\item \( \text{None of the above}. \)

You may have chosen this if you thought the inequality did not match the ends of the intervals.
\end{enumerate}

\textbf{General Comment:} Remember that less/greater than or equal to includes the endpoint, while less/greater do not. Also, remember that you need to flip the inequality when you multiply or divide by a negative.
}
\litem{
Solve the linear inequality below. Then, choose the constant and interval combination that describes the solution set.
\[ -4 + 6 x > 8 x \text{ or } 6 + 9 x < 10 x \]The solution is \( (-\infty, -2.0) \text{ or } (6.0, \infty) \), which is option B.\begin{enumerate}[label=\Alph*.]
\item \( (-\infty, a] \cup [b, \infty), \text{ where } a \in [-4.5, 2.25] \text{ and } b \in [5.25, 12] \)

Corresponds to including the endpoints (when they should be excluded).
\item \( (-\infty, a) \cup (b, \infty), \text{ where } a \in [-4.5, -0.75] \text{ and } b \in [3.75, 9] \)

 * Correct option.
\item \( (-\infty, a) \cup (b, \infty), \text{ where } a \in [-9.75, -5.25] \text{ and } b \in [0.75, 2.25] \)

Corresponds to inverting the inequality and negating the solution.
\item \( (-\infty, a] \cup [b, \infty), \text{ where } a \in [-8.25, -3.75] \text{ and } b \in [0, 2.25] \)

Corresponds to including the endpoints AND negating.
\item \( (-\infty, \infty) \)

Corresponds to the variable canceling, which does not happen in this instance.
\end{enumerate}

\textbf{General Comment:} When multiplying or dividing by a negative, flip the sign.
}
\litem{
Using an interval or intervals, describe all the $x$-values within or including a distance of the given values.
\[ \text{ No less than } 6 \text{ units from the number } 5. \]The solution is \( \text{None of the above} \), which is option E.\begin{enumerate}[label=\Alph*.]
\item \( [1, 11] \)

This describes the values no more than 5 from 6
\item \( (-\infty, 1] \cup [11, \infty) \)

This describes the values no less than 5 from 6
\item \( (1, 11) \)

This describes the values less than 5 from 6
\item \( (-\infty, 1) \cup (11, \infty) \)

This describes the values more than 5 from 6
\item \( \text{None of the above} \)

Options A-D described the values [more/less than] 5 units from 6, which is the reverse of what the question asked.
\end{enumerate}

\textbf{General Comment:} When thinking about this language, it helps to draw a number line and try points.
}
\litem{
Solve the linear inequality below. Then, choose the constant and interval combination that describes the solution set.
\[ -4 + 5 x \leq \frac{25 x - 8}{4} < 6 + 4 x \]The solution is \( [-1.60, 3.56) \), which is option A.\begin{enumerate}[label=\Alph*.]
\item \( [a, b), \text{ where } a \in [-2.4, -1.05] \text{ and } b \in [0, 6] \)

$[-1.60, 3.56)$, which is the correct option.
\item \( (-\infty, a) \cup [b, \infty), \text{ where } a \in [-2.25, 1.5] \text{ and } b \in [-1.5, 6] \)

$(-\infty, -1.60) \cup [3.56, \infty)$, which corresponds to displaying the and-inequality as an or-inequality AND flipping the inequality.
\item \( (a, b], \text{ where } a \in [-2.25, -0.75] \text{ and } b \in [0, 4.5] \)

$(-1.60, 3.56]$, which corresponds to flipping the inequality.
\item \( (-\infty, a] \cup (b, \infty), \text{ where } a \in [-5.25, 0] \text{ and } b \in [1.5, 8.25] \)

$(-\infty, -1.60] \cup (3.56, \infty)$, which corresponds to displaying the and-inequality as an or-inequality.
\item \( \text{None of the above.} \)


\end{enumerate}

\textbf{General Comment:} To solve, you will need to break up the compound inequality into two inequalities. Be sure to keep track of the inequality! It may be best to draw a number line and graph your solution.
}
\litem{
Solve the linear inequality below. Then, choose the constant and interval combination that describes the solution set.
\[ 3x + 8 \geq 5x -10 \]The solution is \( (-\infty, 9.0] \), which is option B.\begin{enumerate}[label=\Alph*.]
\item \( [a, \infty), \text{ where } a \in [9, 13] \)

 $[9.0, \infty)$, which corresponds to switching the direction of the interval. You likely did this if you did not flip the inequality when dividing by a negative!
\item \( (-\infty, a], \text{ where } a \in [6, 10] \)

* $(-\infty, 9.0]$, which is the correct option.
\item \( (-\infty, a], \text{ where } a \in [-9, -3] \)

 $(-\infty, -9.0]$, which corresponds to negating the endpoint of the solution.
\item \( [a, \infty), \text{ where } a \in [-9, -7] \)

 $[-9.0, \infty)$, which corresponds to switching the direction of the interval AND negating the endpoint. You likely did this if you did not flip the inequality when dividing by a negative as well as not moving values over to a side properly.
\item \( \text{None of the above}. \)

You may have chosen this if you thought the inequality did not match the ends of the intervals.
\end{enumerate}

\textbf{General Comment:} Remember that less/greater than or equal to includes the endpoint, while less/greater do not. Also, remember that you need to flip the inequality when you multiply or divide by a negative.
}
\end{enumerate}

\end{document}