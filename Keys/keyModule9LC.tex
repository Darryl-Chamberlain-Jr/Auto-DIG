\documentclass{extbook}[14pt]
\usepackage{multicol, enumerate, enumitem, hyperref, color, soul, setspace, parskip, fancyhdr, amssymb, amsthm, amsmath, bbm, latexsym, units, mathtools}
\everymath{\displaystyle}
\usepackage[headsep=0.5cm,headheight=0cm, left=1 in,right= 1 in,top= 1 in,bottom= 1 in]{geometry}
\usepackage{dashrule}  % Package to use the command below to create lines between items
\newcommand{\litem}[1]{\item #1

\rule{\textwidth}{0.4pt}}
\pagestyle{fancy}
\lhead{}
\chead{Answer Key for Progress Quiz 9 Version C}
\rhead{}
\lfoot{8590-6105}
\cfoot{}
\rfoot{Fall 2020}
\begin{document}
\textbf{This key should allow you to understand why you choose the option you did (beyond just getting a question right or wrong). \href{https://xronos.clas.ufl.edu/mac1105spring2020/courseDescriptionAndMisc/Exams/LearningFromResults}{More instructions on how to use this key can be found here}.}

\textbf{If you have a suggestion to make the keys better, \href{https://forms.gle/CZkbZmPbC9XALEE88}{please fill out the short survey here}.}

\textit{Note: This key is auto-generated and may contain issues and/or errors. The keys are reviewed after each exam to ensure grading is done accurately. If there are issues (like duplicate options), they are noted in the offline gradebook. The keys are a work-in-progress to give students as many resources to improve as possible.}

\rule{\textwidth}{0.4pt}

\begin{enumerate}\litem{
Subtract the following functions, then choose the domain of the resulting function from the list below.
\[ f(x) = \sqrt{-5x+18}  \text{ and } g(x) = 8x^{4} +5 x^{3} +3 x^{2} +9 x + 5 \]

The solution is \( \text{ The domain is all Real numbers less than or equal to} x = 3.6. \), which is option B.\begin{enumerate}[label=\Alph*.]
\item \( \text{ The domain is all Real numbers except } x = a, \text{ where } a \in [0.4, 8.4] \)


\item \( \text{ The domain is all Real numbers less than or equal to } x = a, \text{ where } a \in [1.6, 4.6] \)


\item \( \text{ The domain is all Real numbers greater than or equal to } x = a, \text{ where } a \in [-7.75, 1.25] \)


\item \( \text{ The domain is all Real numbers except } x = a \text{ and } x = b, \text{ where } a \in [-8.4, -2.4] \text{ and } b \in [1.83, 8.83] \)


\item \( \text{ The domain is all Real numbers. } \)


\end{enumerate}

\textbf{General Comment:} The new domain is the intersection of the previous domains.
}
\litem{
Find the inverse of the function below (if it exists). Then, evaluate the inverse at $x = -10$ and choose the interval the $f^{-1}(-10)$ belongs to.
\[ f(x) = \sqrt[3]{2 x + 4} \]

The solution is \( -502.0 \), which is option C.\begin{enumerate}[label=\Alph*.]
\item \( f^{-1}(-10) \in [501.5, 506.2] \)

 This solution corresponds to distractor 2.
\item \( f^{-1}(-10) \in [496.7, 498.8] \)

 This solution corresponds to distractor 3.
\item \( f^{-1}(-10) \in [-503, -499.6] \)

* This is the correct solution.
\item \( f^{-1}(-10) \in [-500.5, -497.4] \)

 Distractor 1: This corresponds to 
\item \( \text{ The function is not invertible for all Real numbers. } \)

 This solution corresponds to distractor 4.
\end{enumerate}

\textbf{General Comment:} Be sure you check that the function is 1-1 before trying to find the inverse!
}
\litem{
Determine whether the function below is 1-1.
\[ f(x) = -18 x^2 + 93 x + 870 \]

The solution is \( \text{no} \), which is option C.\begin{enumerate}[label=\Alph*.]
\item \( \text{No, because the domain of the function is not $(-\infty, \infty)$.} \)

Corresponds to believing 1-1 means the domain is all Real numbers.
\item \( \text{No, because there is an $x$-value that goes to 2 different $y$-values.} \)

Corresponds to the Vertical Line test, which checks if an expression is a function.
\item \( \text{No, because there is a $y$-value that goes to 2 different $x$-values.} \)

* This is the solution.
\item \( \text{Yes, the function is 1-1.} \)

Corresponds to believing the function passes the Horizontal Line test.
\item \( \text{No, because the range of the function is not $(-\infty, \infty)$.} \)

Corresponds to believing 1-1 means the range is all Real numbers.
\end{enumerate}

\textbf{General Comment:} There are only two valid options: The function is 1-1 OR No because there is a $y$-value that goes to 2 different $x$-values.
}
\litem{
Find the inverse of the function below (if it exists). Then, evaluate the inverse at $x = 10$ and choose the interval the $f^{-1}(10)$ belongs to.
\[ f(x) = \sqrt[3]{2 x + 4} \]

The solution is \( 498.0 \), which is option D.\begin{enumerate}[label=\Alph*.]
\item \( f^{-1}(10) \in [501.1, 505.8] \)

 Distractor 1: This corresponds to 
\item \( f^{-1}(10) \in [-504.2, -498.6] \)

 This solution corresponds to distractor 3.
\item \( f^{-1}(10) \in [-498.9, -495.7] \)

 This solution corresponds to distractor 2.
\item \( f^{-1}(10) \in [495.5, 500.2] \)

* This is the correct solution.
\item \( \text{ The function is not invertible for all Real numbers. } \)

 This solution corresponds to distractor 4.
\end{enumerate}

\textbf{General Comment:} Be sure you check that the function is 1-1 before trying to find the inverse!
}
\litem{
Find the inverse of the function below. Then, evaluate the inverse at $x = 9$ and choose the interval that $f^{-1}(9)$ belongs to.
\[ f(x) = \ln{(x-4)}+5 \]

The solution is \( f^{-1}(9) = 58.598 \), which is option B.\begin{enumerate}[label=\Alph*.]
\item \( f^{-1}(9) \in [442412.39, 442427.39] \)

 This solution corresponds to distractor 2.
\item \( f^{-1}(9) \in [56.6, 60.6] \)

 This is the solution.
\item \( f^{-1}(9) \in [45.6, 55.6] \)

 This solution corresponds to distractor 3.
\item \( f^{-1}(9) \in [152.41, 155.41] \)

 This solution corresponds to distractor 4.
\item \( f^{-1}(9) \in [1202605.28, 1202611.28] \)

 This solution corresponds to distractor 1.
\end{enumerate}

\textbf{General Comment:} Natural log and exponential functions always have an inverse. Once you switch the $x$ and $y$, use the conversion $ e^y = x \leftrightarrow y=\ln(x)$.
}
\litem{
Choose the interval below that $f$ composed with $g$ at $x=-1$ is in.
\[ f(x) = 2x^{3} -1 x^{2} -4 x + 4 \text{ and } g(x) = -x^{3} +4 x^{2} +3 x -3 \]

The solution is \( 5.0 \), which is option B.\begin{enumerate}[label=\Alph*.]
\item \( (f \circ g)(-1) \in [-15, -9] \)

 Distractor 1: Corresponds to reversing the composition.
\item \( (f \circ g)(-1) \in [3, 9] \)

* This is the correct solution
\item \( (f \circ g)(-1) \in [14, 15] \)

 Distractor 2: Corresponds to being slightly off from the solution.
\item \( (f \circ g)(-1) \in [-21, -15] \)

 Distractor 3: Corresponds to being slightly off from the solution.
\item \( \text{It is not possible to compose the two functions.} \)


\end{enumerate}

\textbf{General Comment:} $f$ composed with $g$ at $x$ means $f(g(x))$. The order matters!
}
\litem{
Choose the interval below that $f$ composed with $g$ at $x=-1$ is in.
\[ f(x) = -x^{3} +2 x^{2} +x \text{ and } g(x) = 2x^{3} +3 x^{2} -x + 3 \]

The solution is \( -70.0 \), which is option B.\begin{enumerate}[label=\Alph*.]
\item \( (f \circ g)(-1) \in [19, 27] \)

 Distractor 3: Corresponds to being slightly off from the solution.
\item \( (f \circ g)(-1) \in [-73, -65] \)

* This is the correct solution
\item \( (f \circ g)(-1) \in [28, 32] \)

 Distractor 1: Corresponds to reversing the composition.
\item \( (f \circ g)(-1) \in [-66, -56] \)

 Distractor 2: Corresponds to being slightly off from the solution.
\item \( \text{It is not possible to compose the two functions.} \)


\end{enumerate}

\textbf{General Comment:} $f$ composed with $g$ at $x$ means $f(g(x))$. The order matters!
}
\litem{
Find the inverse of the function below. Then, evaluate the inverse at $x = 7$ and choose the interval that $f^{-1}(7)$ belongs to.
\[ f(x) = e^{x-2}-5 \]

The solution is \( f^{-1}(7) = 4.485 \), which is option A.\begin{enumerate}[label=\Alph*.]
\item \( f^{-1}(7) \in [4.31, 4.6] \)

 This is the solution.
\item \( f^{-1}(7) \in [-3.01, -2.63] \)

 This solution corresponds to distractor 3.
\item \( f^{-1}(7) \in [-3.43, -3.34] \)

 This solution corresponds to distractor 4.
\item \( f^{-1}(7) \in [0.43, 0.88] \)

 This solution corresponds to distractor 1.
\item \( f^{-1}(7) \in [-4.6, -4.26] \)

 This solution corresponds to distractor 2.
\end{enumerate}

\textbf{General Comment:} Natural log and exponential functions always have an inverse. Once you switch the $x$ and $y$, use the conversion $ e^y = x \leftrightarrow y=\ln(x)$.
}
\litem{
Multiply the following functions, then choose the domain of the resulting function from the list below.
\[ f(x) = \sqrt{-5x-27}  \text{ and } g(x) = 2x^{2} \]

The solution is \( \text{ The domain is all Real numbers less than or equal to} x = -5.4. \), which is option A.\begin{enumerate}[label=\Alph*.]
\item \( \text{ The domain is all Real numbers less than or equal to } x = a, \text{ where } a \in [-10.4, 0.6] \)


\item \( \text{ The domain is all Real numbers except } x = a, \text{ where } a \in [-10.25, -5.25] \)


\item \( \text{ The domain is all Real numbers greater than or equal to } x = a, \text{ where } a \in [-9, -2] \)


\item \( \text{ The domain is all Real numbers except } x = a \text{ and } x = b, \text{ where } a \in [-8.8, -4.8] \text{ and } b \in [-3.8, 0.2] \)


\item \( \text{ The domain is all Real numbers. } \)


\end{enumerate}

\textbf{General Comment:} The new domain is the intersection of the previous domains.
}
\litem{
Determine whether the function below is 1-1.
\[ f(x) = -24 x^2 - 176 x - 306 \]

The solution is \( \text{no} \), which is option C.\begin{enumerate}[label=\Alph*.]
\item \( \text{No, because the domain of the function is not $(-\infty, \infty)$.} \)

Corresponds to believing 1-1 means the domain is all Real numbers.
\item \( \text{Yes, the function is 1-1.} \)

Corresponds to believing the function passes the Horizontal Line test.
\item \( \text{No, because there is a $y$-value that goes to 2 different $x$-values.} \)

* This is the solution.
\item \( \text{No, because the range of the function is not $(-\infty, \infty)$.} \)

Corresponds to believing 1-1 means the range is all Real numbers.
\item \( \text{No, because there is an $x$-value that goes to 2 different $y$-values.} \)

Corresponds to the Vertical Line test, which checks if an expression is a function.
\end{enumerate}

\textbf{General Comment:} There are only two valid options: The function is 1-1 OR No because there is a $y$-value that goes to 2 different $x$-values.
}
\end{enumerate}

\end{document}