\documentclass{extbook}[14pt]
\usepackage{multicol, enumerate, enumitem, hyperref, color, soul, setspace, parskip, fancyhdr, amssymb, amsthm, amsmath, bbm, latexsym, units, mathtools}
\everymath{\displaystyle}
\usepackage[headsep=0.5cm,headheight=0cm, left=1 in,right= 1 in,top= 1 in,bottom= 1 in]{geometry}
\pagestyle{fancy}
\lhead{}
\chead{Answer Key for Module\,9L\,-\,Operations\,on\,Functions Version C}
\rhead{}
\lfoot{Summer\,C\,2020}
\cfoot{}
\rfoot{}
\begin{document}
\textbf{This key should allow you to understand why you choose the option you did (beyond just getting a question right or wrong). \href{https://xronos.clas.ufl.edu/mac1105spring2020/courseDescriptionAndMisc/Exams/LearningFromResults}{More instructions on how to use this key can be found here}.}

\textbf{If you have a suggestion to make the keys better, \href{https://forms.gle/CZkbZmPbC9XALEE88}{please fill out the short survey here}.}

\textit{Note: This key is auto-generated and may contain issues and/or errors. The keys are reviewed after each exam to ensure grading is done accurately. If there are issues (like duplicate options), they are noted in the offline gradebook. The keys are a work-in-progress to give students as many resources to improve as possible.}

\rule{\textwidth}{0.4pt}

1. Multiply the following functions, then choose the domain of the resulting function from the list below.
\[ f(x) = \sqrt{5x-33}  \text{ and } g(x) = 8x^{3} +4 x^{2} +9 x + 8 \] 
The solution is $ \text{ The domain is all Real numbers greater than or equal to} x = 6.6. $ 

\begin{enumerate}[label=\Alph*.] 
\item $ \text{ The domain is all Real numbers greater than or equal to } x = a, \text{ where } a \in [-3, 9] $ 

  
\item $ \text{ The domain is all Real numbers except } x = a, \text{ where } a \in [-7, 2] $ 

  
\item $ \text{ The domain is all Real numbers less than or equal to } x = a, \text{ where } a \in [0, 11] $ 

  
\item $ \text{ The domain is all Real numbers except } x = a \text{ and } x = b, \text{ where } a \in [2, 9] \text{ and } b \in [-3, 6] $ 

  
\item $ \text{ The domain is all Real numbers. } $ 

  
\end{enumerate} 
 
\textbf{General Comment:} General Comments: The new domain is the intersection of the previous domains. 

-----------------------------------------------

2. Find the inverse of the function below. Then, evaluate the inverse at $x = 9$ and choose the interval that $f^{-1}(9)$ belongs to.
\[ f(x) = e^{x-2}+4 \] 
The solution is $ f^{-1}(9) = 3.609 $ 

\begin{enumerate}[label=\Alph*.] 
\item $ f^{-1}(9) \in [3.53, 3.72] $ 

  This is the solution. 
\item $ f^{-1}(9) \in [6.43, 6.63] $ 

  This solution corresponds to distractor 2. 
\item $ f^{-1}(9) \in [-0.44, -0.35] $ 

  This solution corresponds to distractor 1. 
\item $ f^{-1}(9) \in [5.9, 6.01] $ 

  This solution corresponds to distractor 4. 
\item $ f^{-1}(9) \in [6.33, 6.53] $ 

  This solution corresponds to distractor 3. 
\end{enumerate} 
 
\textbf{General Comment:} Natural log and exponential functions always have an inverse. Once you switch the $x$ and $y$, use the conversion $ e^y = x \leftrightarrow y=\ln(x)$. 

-----------------------------------------------

3. Find the inverse of the function below (if it exists). Then, evaluate the inverse at $x = 12$ and choose the interval the $f^{-1}(12)$ belongs to.
\[ f(x) = \sqrt[3]{5 x - 3} \] 
The solution is $ 346.2 $ 

\begin{enumerate}[label=\Alph*.] 
\item $ f^{-1}(12) \in [-345.76, -344.27] $ 

  This solution corresponds to distractor 3. 
\item $ f^{-1}(12) \in [346.12, 346.26] $ 

 * This is the correct solution. 
\item $ f^{-1}(12) \in [-347.17, -345.73] $ 

  This solution corresponds to distractor 2. 
\item $ f^{-1}(12) \in [344.5, 345.61] $ 

  Distractor 1: This corresponds to  
\item $ \text{ The function is not invertible for all Real numbers. } $ 

  This solution corresponds to distractor 4. 
\end{enumerate} 
 
\textbf{General Comment:} General Comments: Be sure you check that the function is 1-1 before trying to find the inverse! 

-----------------------------------------------

4. Determine whether the function below is 1-1.
\[ f(x) = (3 x - 15)^3 \] 
The solution is $ \text{yes} $ 

\begin{enumerate}[label=\Alph*.] 
\item $ \text{No, because the domain of the function is not $(-\infty, \infty)$.} $ 

 Corresponds to believing 1-1 means the domain is all Real numbers. 
\item $ \text{No, because the range of the function is not $(-\infty, \infty)$.} $ 

 Corresponds to believing 1-1 means the range is all Real numbers. 
\item $ \text{No, because there is a $y$-value that goes to 2 different $x$-values.} $ 

 Corresponds to the Horizontal Line test, which this function passes. 
\item $ \text{No, because there is an $x$-value that goes to 2 different $y$-values.} $ 

 Corresponds to the Vertical Line test, which checks if an expression is a function. 
\item $ \text{Yes, the function is 1-1.} $ 

 * This is the solution. 
\end{enumerate} 
 
\textbf{General Comment:} \textbf{General Comments:} There are only two valid options: The function is 1-1 OR No because there is a $y$-value that goes to 2 different $x$-values. 

-----------------------------------------------

0. Choose the interval below that $f$ composed with $g$ at $x=-1$ is in.
\[ f(x) = -3x^{3} -3 x^{2} +x \text{ and } g(x) = 3x^{3} -3 x^{2} -3 x \] 
The solution is $ 51.0 $ 

\begin{enumerate}[label=\Alph*.] 
\item $ (f \circ g)(-1) \in [45, 52] $ 

 * This is the correct solution 
\item $ (f \circ g)(-1) \in [53, 59] $ 

  Distractor 2: Corresponds to being slightly off from the solution. 
\item $ (f \circ g)(-1) \in [-5, -2] $ 

  Distractor 1: Corresponds to reversing the composition. 
\item $ (f \circ g)(-1) \in [-2, 5] $ 

  Distractor 3: Corresponds to being slightly off from the solution. 
\item $ \text{It is not possible to compose the two functions.} $ 

  
\end{enumerate} 
 
\textbf{General Comment:} General Comments: $f$ composed with $g$ at $x$ means $f(g(x))$. The order matters! 

-----------------------------------------------


\end{document}

