\documentclass{extbook}[14pt]
\usepackage{multicol, enumerate, enumitem, hyperref, color, soul, setspace, parskip, fancyhdr, amssymb, amsthm, amsmath, latexsym, units, mathtools}
\everymath{\displaystyle}
\usepackage[headsep=0.5cm,headheight=0cm, left=1 in,right= 1 in,top= 1 in,bottom= 1 in]{geometry}
\usepackage{dashrule}  % Package to use the command below to create lines between items
\newcommand{\litem}[1]{\item #1

\rule{\textwidth}{0.4pt}}
\pagestyle{fancy}
\lhead{}
\chead{Answer Key for Makeup Progress Quiz 3 Version ALL}
\rhead{}
\lfoot{1648-1753}
\cfoot{}
\rfoot{Summer C 2021}
\begin{document}
\textbf{This key should allow you to understand why you choose the option you did (beyond just getting a question right or wrong). \href{https://xronos.clas.ufl.edu/mac1105spring2020/courseDescriptionAndMisc/Exams/LearningFromResults}{More instructions on how to use this key can be found here}.}

\textbf{If you have a suggestion to make the keys better, \href{https://forms.gle/CZkbZmPbC9XALEE88}{please fill out the short survey here}.}

\textit{Note: This key is auto-generated and may contain issues and/or errors. The keys are reviewed after each exam to ensure grading is done accurately. If there are issues (like duplicate options), they are noted in the offline gradebook. The keys are a work-in-progress to give students as many resources to improve as possible.}

\rule{\textwidth}{0.4pt}

\begin{enumerate}\litem{
Determine whether the function below is 1-1.
\[ f(x) = -24 x^2 - 12 x + 336 \]The solution is \( \text{no} \), which is option C.\begin{enumerate}[label=\Alph*.]
\item \( \text{No, because there is an $x$-value that goes to 2 different $y$-values.} \)

Corresponds to the Vertical Line test, which checks if an expression is a function.
\item \( \text{No, because the domain of the function is not $(-\infty, \infty)$.} \)

Corresponds to believing 1-1 means the domain is all Real numbers.
\item \( \text{No, because there is a $y$-value that goes to 2 different $x$-values.} \)

* This is the solution.
\item \( \text{No, because the range of the function is not $(-\infty, \infty)$.} \)

Corresponds to believing 1-1 means the range is all Real numbers.
\item \( \text{Yes, the function is 1-1.} \)

Corresponds to believing the function passes the Horizontal Line test.
\end{enumerate}

\textbf{General Comment:} There are only two valid options: The function is 1-1 OR No because there is a $y$-value that goes to 2 different $x$-values.
}
\litem{
Determine whether the function below is 1-1.
\[ f(x) = 36 x^2 + 480 x + 1600 \]The solution is \( \text{no} \), which is option E.\begin{enumerate}[label=\Alph*.]
\item \( \text{No, because the domain of the function is not $(-\infty, \infty)$.} \)

Corresponds to believing 1-1 means the domain is all Real numbers.
\item \( \text{No, because there is an $x$-value that goes to 2 different $y$-values.} \)

Corresponds to the Vertical Line test, which checks if an expression is a function.
\item \( \text{Yes, the function is 1-1.} \)

Corresponds to believing the function passes the Horizontal Line test.
\item \( \text{No, because the range of the function is not $(-\infty, \infty)$.} \)

Corresponds to believing 1-1 means the range is all Real numbers.
\item \( \text{No, because there is a $y$-value that goes to 2 different $x$-values.} \)

* This is the solution.
\end{enumerate}

\textbf{General Comment:} There are only two valid options: The function is 1-1 OR No because there is a $y$-value that goes to 2 different $x$-values.
}
\litem{
Find the inverse of the function below (if it exists). Then, evaluate the inverse at $x = -10$ and choose the interval that $f^-1(-10)$ belongs to.
\[ f(x) = \sqrt[3]{4 x + 5} \]The solution is \( -251.25 \), which is option B.\begin{enumerate}[label=\Alph*.]
\item \( f^{-1}(-10) \in [249.3, 253.6] \)

 This solution corresponds to distractor 2.
\item \( f^{-1}(-10) \in [-253.5, -249.2] \)

* This is the correct solution.
\item \( f^{-1}(-10) \in [246.5, 250.6] \)

 This solution corresponds to distractor 3.
\item \( f^{-1}(-10) \in [-250.2, -248.6] \)

 Distractor 1: This corresponds to 
\item \( \text{ The function is not invertible for all Real numbers. } \)

 This solution corresponds to distractor 4.
\end{enumerate}

\textbf{General Comment:} Be sure you check that the function is 1-1 before trying to find the inverse!
}
\litem{
Multiply the following functions, then choose the domain of the resulting function from the list below.
\[ f(x) = 3x^{2} +x + 5 \text{ and } g(x) = 8x^{3} +5 x^{2} +5 x \]The solution is \( (-\infty, \infty) \), which is option E.\begin{enumerate}[label=\Alph*.]
\item \( \text{ The domain is all Real numbers except } x = a, \text{ where } a \in [-10.25, 1.75] \)


\item \( \text{ The domain is all Real numbers less than or equal to } x = a, \text{ where } a \in [5.33, 12.33] \)


\item \( \text{ The domain is all Real numbers greater than or equal to } x = a, \text{ where } a \in [-13.67, -2.67] \)


\item \( \text{ The domain is all Real numbers except } x = a \text{ and } x = b, \text{ where } a \in [5.83, 7.83] \text{ and } b \in [4.67, 6.67] \)


\item \( \text{ The domain is all Real numbers. } \)


\end{enumerate}

\textbf{General Comment:} The new domain is the intersection of the previous domains.
}
\litem{
Choose the interval below that $f$ composed with $g$ at $x=1$ is in.
\[ f(x) = 2x^{3} -4 x^{2} +4 x \text{ and } g(x) = -2x^{3} +4 x^{2} +x + 1 \]The solution is \( 80.0 \), which is option D.\begin{enumerate}[label=\Alph*.]
\item \( (f \circ g)(1) \in [-8, 2] \)

 Distractor 3: Corresponds to being slightly off from the solution.
\item \( (f \circ g)(1) \in [88, 95] \)

 Distractor 2: Corresponds to being slightly off from the solution.
\item \( (f \circ g)(1) \in [1, 5] \)

 Distractor 1: Corresponds to reversing the composition.
\item \( (f \circ g)(1) \in [77, 87] \)

* This is the correct solution
\item \( \text{It is not possible to compose the two functions.} \)


\end{enumerate}

\textbf{General Comment:} $f$ composed with $g$ at $x$ means $f(g(x))$. The order matters!
}
\litem{
Find the inverse of the function below. Then, evaluate the inverse at $x = 7$ and choose the interval that $f^-1(7)$ belongs to.
\[ f(x) = e^{x-5}+3 \]The solution is \( f^{-1}(7) = 6.386 \), which is option E.\begin{enumerate}[label=\Alph*.]
\item \( f^{-1}(7) \in [2.62, 3.88] \)

 This solution corresponds to distractor 4.
\item \( f^{-1}(7) \in [-4.27, -3.07] \)

 This solution corresponds to distractor 1.
\item \( f^{-1}(7) \in [5.41, 5.89] \)

 This solution corresponds to distractor 3.
\item \( f^{-1}(7) \in [4.86, 5.34] \)

 This solution corresponds to distractor 2.
\item \( f^{-1}(7) \in [6.08, 7.06] \)

 This is the solution.
\end{enumerate}

\textbf{General Comment:} Natural log and exponential functions always have an inverse. Once you switch the $x$ and $y$, use the conversion $ e^y = x \leftrightarrow y=\ln(x)$.
}
\litem{
Choose the interval below that $f$ composed with $g$ at $x=1$ is in.
\[ f(x) = -2x^{3} + x^{2} -x \text{ and } g(x) = -2x^{3} -1 x^{2} -x + 4 \]The solution is \( 0.0 \), which is option C.\begin{enumerate}[label=\Alph*.]
\item \( (f \circ g)(1) \in [23.1, 25.2] \)

 Distractor 3: Corresponds to being slightly off from the solution.
\item \( (f \circ g)(1) \in [8.9, 9.9] \)

 Distractor 2: Corresponds to being slightly off from the solution.
\item \( (f \circ g)(1) \in [-1.3, 3.9] \)

* This is the correct solution
\item \( (f \circ g)(1) \in [17.6, 18.8] \)

 Distractor 1: Corresponds to reversing the composition.
\item \( \text{It is not possible to compose the two functions.} \)


\end{enumerate}

\textbf{General Comment:} $f$ composed with $g$ at $x$ means $f(g(x))$. The order matters!
}
\litem{
Multiply the following functions, then choose the domain of the resulting function from the list below.
\[ f(x) = 5x^{2} +8 x + 9 \text{ and } g(x) = 2x^{3} +4 x^{2} +x + 8 \]The solution is \( (-\infty, \infty) \), which is option E.\begin{enumerate}[label=\Alph*.]
\item \( \text{ The domain is all Real numbers less than or equal to } x = a, \text{ where } a \in [-7.75, 2.25] \)


\item \( \text{ The domain is all Real numbers except } x = a, \text{ where } a \in [1.67, 10.67] \)


\item \( \text{ The domain is all Real numbers greater than or equal to } x = a, \text{ where } a \in [3.5, 8.5] \)


\item \( \text{ The domain is all Real numbers except } x = a \text{ and } x = b, \text{ where } a \in [3.2, 10.2] \text{ and } b \in [-8.67, -4.67] \)


\item \( \text{ The domain is all Real numbers. } \)


\end{enumerate}

\textbf{General Comment:} The new domain is the intersection of the previous domains.
}
\litem{
Find the inverse of the function below (if it exists). Then, evaluate the inverse at $x = -10$ and choose the interval that $f^-1(-10)$ belongs to.
\[ f(x) = 3 x^2 - 5 \]The solution is \( \text{ The function is not invertible for all Real numbers. } \), which is option E.\begin{enumerate}[label=\Alph*.]
\item \( f^{-1}(-10) \in [1.29, 1.31] \)

 Distractor 1: This corresponds to trying to find the inverse even though the function is not 1-1. 
\item \( f^{-1}(-10) \in [2.28, 2.31] \)

 Distractor 3: This corresponds to finding the (nonexistent) inverse and dividing by a negative.
\item \( f^{-1}(-10) \in [3.27, 3.35] \)

 Distractor 4: This corresponds to both distractors 2 and 3.
\item \( f^{-1}(-10) \in [2.18, 2.29] \)

 Distractor 2: This corresponds to finding the (nonexistent) inverse and not subtracting by the vertical shift.
\item \( \text{ The function is not invertible for all Real numbers. } \)

* This is the correct option.
\end{enumerate}

\textbf{General Comment:} Be sure you check that the function is 1-1 before trying to find the inverse!
}
\litem{
Find the inverse of the function below. Then, evaluate the inverse at $x = 9$ and choose the interval that $f^-1(9)$ belongs to.
\[ f(x) = e^{x-5}+3 \]The solution is \( f^{-1}(9) = 6.792 \), which is option E.\begin{enumerate}[label=\Alph*.]
\item \( f^{-1}(9) \in [5.57, 5.67] \)

 This solution corresponds to distractor 3.
\item \( f^{-1}(9) \in [4.16, 4.4] \)

 This solution corresponds to distractor 4.
\item \( f^{-1}(9) \in [5.3, 5.53] \)

 This solution corresponds to distractor 2.
\item \( f^{-1}(9) \in [-3.25, -2.83] \)

 This solution corresponds to distractor 1.
\item \( f^{-1}(9) \in [6.79, 7.24] \)

 This is the solution.
\end{enumerate}

\textbf{General Comment:} Natural log and exponential functions always have an inverse. Once you switch the $x$ and $y$, use the conversion $ e^y = x \leftrightarrow y=\ln(x)$.
}
\litem{
Determine whether the function below is 1-1.
\[ f(x) = 36 x^2 - 252 x + 441 \]The solution is \( \text{no} \), which is option C.\begin{enumerate}[label=\Alph*.]
\item \( \text{No, because the domain of the function is not $(-\infty, \infty)$.} \)

Corresponds to believing 1-1 means the domain is all Real numbers.
\item \( \text{No, because there is an $x$-value that goes to 2 different $y$-values.} \)

Corresponds to the Vertical Line test, which checks if an expression is a function.
\item \( \text{No, because there is a $y$-value that goes to 2 different $x$-values.} \)

* This is the solution.
\item \( \text{Yes, the function is 1-1.} \)

Corresponds to believing the function passes the Horizontal Line test.
\item \( \text{No, because the range of the function is not $(-\infty, \infty)$.} \)

Corresponds to believing 1-1 means the range is all Real numbers.
\end{enumerate}

\textbf{General Comment:} There are only two valid options: The function is 1-1 OR No because there is a $y$-value that goes to 2 different $x$-values.
}
\litem{
Determine whether the function below is 1-1.
\[ f(x) = (3 x + 19)^3 \]The solution is \( \text{yes} \), which is option B.\begin{enumerate}[label=\Alph*.]
\item \( \text{No, because the range of the function is not $(-\infty, \infty)$.} \)

Corresponds to believing 1-1 means the range is all Real numbers.
\item \( \text{Yes, the function is 1-1.} \)

* This is the solution.
\item \( \text{No, because there is a $y$-value that goes to 2 different $x$-values.} \)

Corresponds to the Horizontal Line test, which this function passes.
\item \( \text{No, because there is an $x$-value that goes to 2 different $y$-values.} \)

Corresponds to the Vertical Line test, which checks if an expression is a function.
\item \( \text{No, because the domain of the function is not $(-\infty, \infty)$.} \)

Corresponds to believing 1-1 means the domain is all Real numbers.
\end{enumerate}

\textbf{General Comment:} There are only two valid options: The function is 1-1 OR No because there is a $y$-value that goes to 2 different $x$-values.
}
\litem{
Find the inverse of the function below (if it exists). Then, evaluate the inverse at $x = 13$ and choose the interval that $f^-1(13)$ belongs to.
\[ f(x) = 3 x^2 - 2 \]The solution is \( \text{ The function is not invertible for all Real numbers. } \), which is option E.\begin{enumerate}[label=\Alph*.]
\item \( f^{-1}(13) \in [2.17, 2.58] \)

 Distractor 1: This corresponds to trying to find the inverse even though the function is not 1-1. 
\item \( f^{-1}(13) \in [4.98, 5.3] \)

 Distractor 4: This corresponds to both distractors 2 and 3.
\item \( f^{-1}(13) \in [1.51, 2.17] \)

 Distractor 2: This corresponds to finding the (nonexistent) inverse and not subtracting by the vertical shift.
\item \( f^{-1}(13) \in [3.23, 3.35] \)

 Distractor 3: This corresponds to finding the (nonexistent) inverse and dividing by a negative.
\item \( \text{ The function is not invertible for all Real numbers. } \)

* This is the correct option.
\end{enumerate}

\textbf{General Comment:} Be sure you check that the function is 1-1 before trying to find the inverse!
}
\litem{
Subtract the following functions, then choose the domain of the resulting function from the list below.
\[ f(x) = x^{2} +5 x + 8 \text{ and } g(x) = \frac{3}{4x-21} \]The solution is \( \text{ The domain is all Real numbers except } x = 5.25 \), which is option C.\begin{enumerate}[label=\Alph*.]
\item \( \text{ The domain is all Real numbers less than or equal to } x = a, \text{ where } a \in [-0.25, 6.75] \)


\item \( \text{ The domain is all Real numbers greater than or equal to } x = a, \text{ where } a \in [4, 9] \)


\item \( \text{ The domain is all Real numbers except } x = a, \text{ where } a \in [4.25, 9.25] \)


\item \( \text{ The domain is all Real numbers except } x = a \text{ and } x = b, \text{ where } a \in [-5.4, -2.4] \text{ and } b \in [1.25, 9.25] \)


\item \( \text{ The domain is all Real numbers. } \)


\end{enumerate}

\textbf{General Comment:} The new domain is the intersection of the previous domains.
}
\litem{
Choose the interval below that $f$ composed with $g$ at $x=-1$ is in.
\[ f(x) = -3x^{3} -2 x^{2} -2 x -2 \text{ and } g(x) = -3x^{3} -2 x^{2} +4 x \]The solution is \( 67.0 \), which is option D.\begin{enumerate}[label=\Alph*.]
\item \( (f \circ g)(-1) \in [5, 14] \)

 Distractor 3: Corresponds to being slightly off from the solution.
\item \( (f \circ g)(-1) \in [58, 65] \)

 Distractor 2: Corresponds to being slightly off from the solution.
\item \( (f \circ g)(-1) \in [-3, 5] \)

 Distractor 1: Corresponds to reversing the composition.
\item \( (f \circ g)(-1) \in [65, 70] \)

* This is the correct solution
\item \( \text{It is not possible to compose the two functions.} \)


\end{enumerate}

\textbf{General Comment:} $f$ composed with $g$ at $x$ means $f(g(x))$. The order matters!
}
\litem{
Find the inverse of the function below. Then, evaluate the inverse at $x = 8$ and choose the interval that $f^-1(8)$ belongs to.
\[ f(x) = \ln{(x+3)}+3 \]The solution is \( f^{-1}(8) = 145.413 \), which is option D.\begin{enumerate}[label=\Alph*.]
\item \( f^{-1}(8) \in [151.41, 152.41] \)

 This solution corresponds to distractor 3.
\item \( f^{-1}(8) \in [59871.14, 59872.14] \)

 This solution corresponds to distractor 1.
\item \( f^{-1}(8) \in [151.41, 152.41] \)

 This solution corresponds to distractor 2.
\item \( f^{-1}(8) \in [141.41, 150.41] \)

 This is the solution.
\item \( f^{-1}(8) \in [59877.14, 59879.14] \)

 This solution corresponds to distractor 4.
\end{enumerate}

\textbf{General Comment:} Natural log and exponential functions always have an inverse. Once you switch the $x$ and $y$, use the conversion $ e^y = x \leftrightarrow y=\ln(x)$.
}
\litem{
Choose the interval below that $f$ composed with $g$ at $x=-1$ is in.
\[ f(x) = x^{3} -4 x^{2} -2 x + 1 \text{ and } g(x) = -3x^{3} -4 x^{2} +2 x \]The solution is \( -56.0 \), which is option A.\begin{enumerate}[label=\Alph*.]
\item \( (f \circ g)(-1) \in [-59, -51] \)

* This is the correct solution
\item \( (f \circ g)(-1) \in [4, 10] \)

 Distractor 1: Corresponds to reversing the composition.
\item \( (f \circ g)(-1) \in [6, 16] \)

 Distractor 3: Corresponds to being slightly off from the solution.
\item \( (f \circ g)(-1) \in [-66, -64] \)

 Distractor 2: Corresponds to being slightly off from the solution.
\item \( \text{It is not possible to compose the two functions.} \)


\end{enumerate}

\textbf{General Comment:} $f$ composed with $g$ at $x$ means $f(g(x))$. The order matters!
}
\litem{
Multiply the following functions, then choose the domain of the resulting function from the list below.
\[ f(x) = 4x^{4} +6 x^{3} +7 x^{2} +7 x + 8 \text{ and } g(x) = x^{2} +7 x + 1 \]The solution is \( (-\infty, \infty) \), which is option E.\begin{enumerate}[label=\Alph*.]
\item \( \text{ The domain is all Real numbers greater than or equal to } x = a, \text{ where } a \in [2.25, 7.25] \)


\item \( \text{ The domain is all Real numbers less than or equal to } x = a, \text{ where } a \in [0.25, 8.25] \)


\item \( \text{ The domain is all Real numbers except } x = a, \text{ where } a \in [-5.75, -3.75] \)


\item \( \text{ The domain is all Real numbers except } x = a \text{ and } x = b, \text{ where } a \in [-10.33, -1.33] \text{ and } b \in [3.2, 6.2] \)


\item \( \text{ The domain is all Real numbers. } \)


\end{enumerate}

\textbf{General Comment:} The new domain is the intersection of the previous domains.
}
\litem{
Find the inverse of the function below (if it exists). Then, evaluate the inverse at $x = -11$ and choose the interval that $f^-1(-11)$ belongs to.
\[ f(x) = \sqrt[3]{2 x + 3} \]The solution is \( -667.0 \), which is option B.\begin{enumerate}[label=\Alph*.]
\item \( f^{-1}(-11) \in [-665, -662.1] \)

 Distractor 1: This corresponds to 
\item \( f^{-1}(-11) \in [-669.5, -665.9] \)

* This is the correct solution.
\item \( f^{-1}(-11) \in [663.5, 665.9] \)

 This solution corresponds to distractor 3.
\item \( f^{-1}(-11) \in [666.6, 667.4] \)

 This solution corresponds to distractor 2.
\item \( \text{ The function is not invertible for all Real numbers. } \)

 This solution corresponds to distractor 4.
\end{enumerate}

\textbf{General Comment:} Be sure you check that the function is 1-1 before trying to find the inverse!
}
\litem{
Find the inverse of the function below. Then, evaluate the inverse at $x = 9$ and choose the interval that $f^-1(9)$ belongs to.
\[ f(x) = e^{x+3}-5 \]The solution is \( f^{-1}(9) = -0.361 \), which is option D.\begin{enumerate}[label=\Alph*.]
\item \( f^{-1}(9) \in [-3.85, -3.55] \)

 This solution corresponds to distractor 2.
\item \( f^{-1}(9) \in [-2.87, -2.39] \)

 This solution corresponds to distractor 4.
\item \( f^{-1}(9) \in [5.29, 5.97] \)

 This solution corresponds to distractor 1.
\item \( f^{-1}(9) \in [-0.68, -0.23] \)

 This is the solution.
\item \( f^{-1}(9) \in [-3.3, -3.08] \)

 This solution corresponds to distractor 3.
\end{enumerate}

\textbf{General Comment:} Natural log and exponential functions always have an inverse. Once you switch the $x$ and $y$, use the conversion $ e^y = x \leftrightarrow y=\ln(x)$.
}
\litem{
Determine whether the function below is 1-1.
\[ f(x) = (5 x - 16)^3 \]The solution is \( \text{yes} \), which is option E.\begin{enumerate}[label=\Alph*.]
\item \( \text{No, because the domain of the function is not $(-\infty, \infty)$.} \)

Corresponds to believing 1-1 means the domain is all Real numbers.
\item \( \text{No, because there is an $x$-value that goes to 2 different $y$-values.} \)

Corresponds to the Vertical Line test, which checks if an expression is a function.
\item \( \text{No, because the range of the function is not $(-\infty, \infty)$.} \)

Corresponds to believing 1-1 means the range is all Real numbers.
\item \( \text{No, because there is a $y$-value that goes to 2 different $x$-values.} \)

Corresponds to the Horizontal Line test, which this function passes.
\item \( \text{Yes, the function is 1-1.} \)

* This is the solution.
\end{enumerate}

\textbf{General Comment:} There are only two valid options: The function is 1-1 OR No because there is a $y$-value that goes to 2 different $x$-values.
}
\litem{
Determine whether the function below is 1-1.
\[ f(x) = 9 x^2 - 30 x + 25 \]The solution is \( \text{no} \), which is option E.\begin{enumerate}[label=\Alph*.]
\item \( \text{Yes, the function is 1-1.} \)

Corresponds to believing the function passes the Horizontal Line test.
\item \( \text{No, because the range of the function is not $(-\infty, \infty)$.} \)

Corresponds to believing 1-1 means the range is all Real numbers.
\item \( \text{No, because there is an $x$-value that goes to 2 different $y$-values.} \)

Corresponds to the Vertical Line test, which checks if an expression is a function.
\item \( \text{No, because the domain of the function is not $(-\infty, \infty)$.} \)

Corresponds to believing 1-1 means the domain is all Real numbers.
\item \( \text{No, because there is a $y$-value that goes to 2 different $x$-values.} \)

* This is the solution.
\end{enumerate}

\textbf{General Comment:} There are only two valid options: The function is 1-1 OR No because there is a $y$-value that goes to 2 different $x$-values.
}
\litem{
Find the inverse of the function below (if it exists). Then, evaluate the inverse at $x = 12$ and choose the interval that $f^-1(12)$ belongs to.
\[ f(x) = 3 x^2 + 2 \]The solution is \( \text{ The function is not invertible for all Real numbers. } \), which is option E.\begin{enumerate}[label=\Alph*.]
\item \( f^{-1}(12) \in [4.74, 5.36] \)

 Distractor 3: This corresponds to finding the (nonexistent) inverse and dividing by a negative.
\item \( f^{-1}(12) \in [2.03, 4.21] \)

 Distractor 2: This corresponds to finding the (nonexistent) inverse and not subtracting by the vertical shift.
\item \( f^{-1}(12) \in [6.45, 8.02] \)

 Distractor 4: This corresponds to both distractors 2 and 3.
\item \( f^{-1}(12) \in [1.74, 1.9] \)

 Distractor 1: This corresponds to trying to find the inverse even though the function is not 1-1. 
\item \( \text{ The function is not invertible for all Real numbers. } \)

* This is the correct option.
\end{enumerate}

\textbf{General Comment:} Be sure you check that the function is 1-1 before trying to find the inverse!
}
\litem{
Subtract the following functions, then choose the domain of the resulting function from the list below.
\[ f(x) = 6x^{4} +6 x^{2} +7 x + 7 \text{ and } g(x) = \sqrt{-5x-15}  \]The solution is \( \text{ The domain is all Real numbers less than or equal to} x = -3.0. \), which is option B.\begin{enumerate}[label=\Alph*.]
\item \( \text{ The domain is all Real numbers except } x = a, \text{ where } a \in [-14.4, 0.6] \)


\item \( \text{ The domain is all Real numbers less than or equal to } x = a, \text{ where } a \in [-4, 0] \)


\item \( \text{ The domain is all Real numbers greater than or equal to } x = a, \text{ where } a \in [-5.5, -0.5] \)


\item \( \text{ The domain is all Real numbers except } x = a \text{ and } x = b, \text{ where } a \in [-9.67, -4.67] \text{ and } b \in [-8.83, -4.83] \)


\item \( \text{ The domain is all Real numbers. } \)


\end{enumerate}

\textbf{General Comment:} The new domain is the intersection of the previous domains.
}
\litem{
Choose the interval below that $f$ composed with $g$ at $x=1$ is in.
\[ f(x) = -2x^{3} +2 x^{2} +x \text{ and } g(x) = 4x^{3} -2 x^{2} -2 x \]The solution is \( 0.0 \), which is option B.\begin{enumerate}[label=\Alph*.]
\item \( (f \circ g)(1) \in [-1.42, 0.53] \)

 Distractor 1: Corresponds to reversing the composition.
\item \( (f \circ g)(1) \in [-1.42, 0.53] \)

* This is the correct solution
\item \( (f \circ g)(1) \in [4.53, 5.32] \)

 Distractor 3: Corresponds to being slightly off from the solution.
\item \( (f \circ g)(1) \in [5.81, 6.62] \)

 Distractor 2: Corresponds to being slightly off from the solution.
\item \( \text{It is not possible to compose the two functions.} \)


\end{enumerate}

\textbf{General Comment:} $f$ composed with $g$ at $x$ means $f(g(x))$. The order matters!
}
\litem{
Find the inverse of the function below. Then, evaluate the inverse at $x = 7$ and choose the interval that $f^-1(7)$ belongs to.
\[ f(x) = e^{x-3}-3 \]The solution is \( f^{-1}(7) = 5.303 \), which is option A.\begin{enumerate}[label=\Alph*.]
\item \( f^{-1}(7) \in [5.1, 6.59] \)

 This is the solution.
\item \( f^{-1}(7) \in [-1.53, 0.15] \)

 This solution corresponds to distractor 1.
\item \( f^{-1}(7) \in [-1.53, 0.15] \)

 This solution corresponds to distractor 3.
\item \( f^{-1}(7) \in [-1.96, -1.38] \)

 This solution corresponds to distractor 4.
\item \( f^{-1}(7) \in [-1.96, -1.38] \)

 This solution corresponds to distractor 2.
\end{enumerate}

\textbf{General Comment:} Natural log and exponential functions always have an inverse. Once you switch the $x$ and $y$, use the conversion $ e^y = x \leftrightarrow y=\ln(x)$.
}
\litem{
Choose the interval below that $f$ composed with $g$ at $x=1$ is in.
\[ f(x) = x^{3} -1 x^{2} -3 x + 1 \text{ and } g(x) = 3x^{3} -3 x^{2} +2 x \]The solution is \( -1.0 \), which is option C.\begin{enumerate}[label=\Alph*.]
\item \( (f \circ g)(1) \in [5, 13] \)

 Distractor 2: Corresponds to being slightly off from the solution.
\item \( (f \circ g)(1) \in [-49, -41] \)

 Distractor 3: Corresponds to being slightly off from the solution.
\item \( (f \circ g)(1) \in [-4, 2] \)

* This is the correct solution
\item \( (f \circ g)(1) \in [-43, -39] \)

 Distractor 1: Corresponds to reversing the composition.
\item \( \text{It is not possible to compose the two functions.} \)


\end{enumerate}

\textbf{General Comment:} $f$ composed with $g$ at $x$ means $f(g(x))$. The order matters!
}
\litem{
Multiply the following functions, then choose the domain of the resulting function from the list below.
\[ f(x) = \sqrt{4x-26}  \text{ and } g(x) = 6x^{3} +9 x^{2} +8 x + 1 \]The solution is \( \text{ The domain is all Real numbers greater than or equal to} x = 6.5. \), which is option C.\begin{enumerate}[label=\Alph*.]
\item \( \text{ The domain is all Real numbers except } x = a, \text{ where } a \in [3.75, 11.75] \)


\item \( \text{ The domain is all Real numbers less than or equal to } x = a, \text{ where } a \in [2.67, 12.67] \)


\item \( \text{ The domain is all Real numbers greater than or equal to } x = a, \text{ where } a \in [4.5, 10.5] \)


\item \( \text{ The domain is all Real numbers except } x = a \text{ and } x = b, \text{ where } a \in [4.2, 8.2] \text{ and } b \in [-7.67, 0.33] \)


\item \( \text{ The domain is all Real numbers. } \)


\end{enumerate}

\textbf{General Comment:} The new domain is the intersection of the previous domains.
}
\litem{
Find the inverse of the function below (if it exists). Then, evaluate the inverse at $x = 14$ and choose the interval that $f^-1(14)$ belongs to.
\[ f(x) = 5 x^2 + 3 \]The solution is \( \text{ The function is not invertible for all Real numbers. } \), which is option E.\begin{enumerate}[label=\Alph*.]
\item \( f^{-1}(14) \in [1.74, 1.9] \)

 Distractor 2: This corresponds to finding the (nonexistent) inverse and not subtracting by the vertical shift.
\item \( f^{-1}(14) \in [3.47, 3.7] \)

 Distractor 3: This corresponds to finding the (nonexistent) inverse and dividing by a negative.
\item \( f^{-1}(14) \in [4.46, 4.89] \)

 Distractor 4: This corresponds to both distractors 2 and 3.
\item \( f^{-1}(14) \in [1.48, 1.69] \)

 Distractor 1: This corresponds to trying to find the inverse even though the function is not 1-1. 
\item \( \text{ The function is not invertible for all Real numbers. } \)

* This is the correct option.
\end{enumerate}

\textbf{General Comment:} Be sure you check that the function is 1-1 before trying to find the inverse!
}
\litem{
Find the inverse of the function below. Then, evaluate the inverse at $x = 8$ and choose the interval that $f^-1(8)$ belongs to.
\[ f(x) = e^{x+4}-2 \]The solution is \( f^{-1}(8) = -1.697 \), which is option B.\begin{enumerate}[label=\Alph*.]
\item \( f^{-1}(8) \in [0.47, 0.65] \)

 This solution corresponds to distractor 4.
\item \( f^{-1}(8) \in [-1.94, -1.15] \)

 This is the solution.
\item \( f^{-1}(8) \in [-1.35, -0.49] \)

 This solution corresponds to distractor 3.
\item \( f^{-1}(8) \in [6.13, 6.88] \)

 This solution corresponds to distractor 1.
\item \( f^{-1}(8) \in [-0.27, 0.15] \)

 This solution corresponds to distractor 2.
\end{enumerate}

\textbf{General Comment:} Natural log and exponential functions always have an inverse. Once you switch the $x$ and $y$, use the conversion $ e^y = x \leftrightarrow y=\ln(x)$.
}
\end{enumerate}

\end{document}