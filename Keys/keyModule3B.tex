\documentclass{extbook}[14pt]
\usepackage{multicol, enumerate, enumitem, hyperref, color, soul, setspace, parskip, fancyhdr, amssymb, amsthm, amsmath, latexsym, units, mathtools}
\everymath{\displaystyle}
\usepackage[headsep=0.5cm,headheight=0cm, left=1 in,right= 1 in,top= 1 in,bottom= 1 in]{geometry}
\usepackage{dashrule}  % Package to use the command below to create lines between items
\newcommand{\litem}[1]{\item #1

\rule{\textwidth}{0.4pt}}
\pagestyle{fancy}
\lhead{}
\chead{Answer Key for Progress Quiz 6 Version B}
\rhead{}
\lfoot{9689-6866}
\cfoot{}
\rfoot{Spring 2021}
\begin{document}
\textbf{This key should allow you to understand why you choose the option you did (beyond just getting a question right or wrong). \href{https://xronos.clas.ufl.edu/mac1105spring2020/courseDescriptionAndMisc/Exams/LearningFromResults}{More instructions on how to use this key can be found here}.}

\textbf{If you have a suggestion to make the keys better, \href{https://forms.gle/CZkbZmPbC9XALEE88}{please fill out the short survey here}.}

\textit{Note: This key is auto-generated and may contain issues and/or errors. The keys are reviewed after each exam to ensure grading is done accurately. If there are issues (like duplicate options), they are noted in the offline gradebook. The keys are a work-in-progress to give students as many resources to improve as possible.}

\rule{\textwidth}{0.4pt}

\begin{enumerate}\litem{
Solve the linear inequality below. Then, choose the constant and interval combination that describes the solution set.
\[ 4x -6 \leq 9x + 6 \]The solution is \( [-2.4, \infty) \), which is option C.\begin{enumerate}[label=\Alph*.]
\item \( [a, \infty), \text{ where } a \in [2.4, 5.4] \)

 $[2.4, \infty)$, which corresponds to negating the endpoint of the solution.
\item \( (-\infty, a], \text{ where } a \in [-3.4, 1.6] \)

 $(-\infty, -2.4]$, which corresponds to switching the direction of the interval. You likely did this if you did not flip the inequality when dividing by a negative!
\item \( [a, \infty), \text{ where } a \in [-2.4, -0.4] \)

* $[-2.4, \infty)$, which is the correct option.
\item \( (-\infty, a], \text{ where } a \in [1.4, 4.4] \)

 $(-\infty, 2.4]$, which corresponds to switching the direction of the interval AND negating the endpoint. You likely did this if you did not flip the inequality when dividing by a negative as well as not moving values over to a side properly.
\item \( \text{None of the above}. \)

You may have chosen this if you thought the inequality did not match the ends of the intervals.
\end{enumerate}

\textbf{General Comment:} Remember that less/greater than or equal to includes the endpoint, while less/greater do not. Also, remember that you need to flip the inequality when you multiply or divide by a negative.
}
\litem{
Solve the linear inequality below. Then, choose the constant and interval combination that describes the solution set.
\[ 3 + 5 x > 7 x \text{ or } 9 + 4 x < 7 x \]The solution is \( (-\infty, 1.5) \text{ or } (3.0, \infty) \), which is option C.\begin{enumerate}[label=\Alph*.]
\item \( (-\infty, a] \cup [b, \infty), \text{ where } a \in [-5, -2] \text{ and } b \in [-1.5, -0.5] \)

Corresponds to including the endpoints AND negating.
\item \( (-\infty, a) \cup (b, \infty), \text{ where } a \in [-4, 0] \text{ and } b \in [-2.5, 0.5] \)

Corresponds to inverting the inequality and negating the solution.
\item \( (-\infty, a) \cup (b, \infty), \text{ where } a \in [-0.5, 4.5] \text{ and } b \in [0, 4] \)

 * Correct option.
\item \( (-\infty, a] \cup [b, \infty), \text{ where } a \in [-2.5, 4.5] \text{ and } b \in [3, 7] \)

Corresponds to including the endpoints (when they should be excluded).
\item \( (-\infty, \infty) \)

Corresponds to the variable canceling, which does not happen in this instance.
\end{enumerate}

\textbf{General Comment:} When multiplying or dividing by a negative, flip the sign.
}
\litem{
Solve the linear inequality below. Then, choose the constant and interval combination that describes the solution set.
\[ -5 + 9 x > 11 x \text{ or } 3 + 9 x < 11 x \]The solution is \( (-\infty, -2.5) \text{ or } (1.5, \infty) \), which is option A.\begin{enumerate}[label=\Alph*.]
\item \( (-\infty, a) \cup (b, \infty), \text{ where } a \in [-3.01, -2.47] \text{ and } b \in [1.4, 1.86] \)

 * Correct option.
\item \( (-\infty, a] \cup [b, \infty), \text{ where } a \in [-1.8, 0.1] \text{ and } b \in [2.06, 3.51] \)

Corresponds to including the endpoints AND negating.
\item \( (-\infty, a) \cup (b, \infty), \text{ where } a \in [-1.81, -0.61] \text{ and } b \in [1.91, 2.59] \)

Corresponds to inverting the inequality and negating the solution.
\item \( (-\infty, a] \cup [b, \infty), \text{ where } a \in [-2.7, -2.19] \text{ and } b \in [1.38, 1.72] \)

Corresponds to including the endpoints (when they should be excluded).
\item \( (-\infty, \infty) \)

Corresponds to the variable canceling, which does not happen in this instance.
\end{enumerate}

\textbf{General Comment:} When multiplying or dividing by a negative, flip the sign.
}
\litem{
Solve the linear inequality below. Then, choose the constant and interval combination that describes the solution set.
\[ \frac{6}{3} - \frac{10}{6} x \geq \frac{-9}{8} x - \frac{8}{9} \]The solution is \( (-\infty, 5.333] \), which is option A.\begin{enumerate}[label=\Alph*.]
\item \( (-\infty, a], \text{ where } a \in [4.33, 8.33] \)

* $(-\infty, 5.333]$, which is the correct option.
\item \( (-\infty, a], \text{ where } a \in [-6.33, -4.33] \)

 $(-\infty, -5.333]$, which corresponds to negating the endpoint of the solution.
\item \( [a, \infty), \text{ where } a \in [-7.33, -1.33] \)

 $[-5.333, \infty)$, which corresponds to switching the direction of the interval AND negating the endpoint. You likely did this if you did not flip the inequality when dividing by a negative as well as not moving values over to a side properly.
\item \( [a, \infty), \text{ where } a \in [4.33, 8.33] \)

 $[5.333, \infty)$, which corresponds to switching the direction of the interval. You likely did this if you did not flip the inequality when dividing by a negative!
\item \( \text{None of the above}. \)

You may have chosen this if you thought the inequality did not match the ends of the intervals.
\end{enumerate}

\textbf{General Comment:} Remember that less/greater than or equal to includes the endpoint, while less/greater do not. Also, remember that you need to flip the inequality when you multiply or divide by a negative.
}
\litem{
Solve the linear inequality below. Then, choose the constant and interval combination that describes the solution set.
\[ -3x + 10 \leq 10x -4 \]The solution is \( [1.077, \infty) \), which is option A.\begin{enumerate}[label=\Alph*.]
\item \( [a, \infty), \text{ where } a \in [1, 2.7] \)

* $[1.077, \infty)$, which is the correct option.
\item \( [a, \infty), \text{ where } a \in [-3.1, 0.6] \)

 $[-1.077, \infty)$, which corresponds to negating the endpoint of the solution.
\item \( (-\infty, a], \text{ where } a \in [-3.08, -0.08] \)

 $(-\infty, -1.077]$, which corresponds to switching the direction of the interval AND negating the endpoint. You likely did this if you did not flip the inequality when dividing by a negative as well as not moving values over to a side properly.
\item \( (-\infty, a], \text{ where } a \in [-0.92, 5.08] \)

 $(-\infty, 1.077]$, which corresponds to switching the direction of the interval. You likely did this if you did not flip the inequality when dividing by a negative!
\item \( \text{None of the above}. \)

You may have chosen this if you thought the inequality did not match the ends of the intervals.
\end{enumerate}

\textbf{General Comment:} Remember that less/greater than or equal to includes the endpoint, while less/greater do not. Also, remember that you need to flip the inequality when you multiply or divide by a negative.
}
\litem{
Using an interval or intervals, describe all the $x$-values within or including a distance of the given values.
\[ \text{ No more than } 8 \text{ units from the number } -4. \]The solution is \( [-12, 4] \), which is option B.\begin{enumerate}[label=\Alph*.]
\item \( (-\infty, -12) \cup (4, \infty) \)

This describes the values more than 8 from -4
\item \( [-12, 4] \)

This describes the values no more than 8 from -4
\item \( (-\infty, -12] \cup [4, \infty) \)

This describes the values no less than 8 from -4
\item \( (-12, 4) \)

This describes the values less than 8 from -4
\item \( \text{None of the above} \)

You likely thought the values in the interval were not correct.
\end{enumerate}

\textbf{General Comment:} When thinking about this language, it helps to draw a number line and try points.
}
\litem{
Solve the linear inequality below. Then, choose the constant and interval combination that describes the solution set.
\[ -7 - 8 x < \frac{-36 x - 4}{5} \leq 9 - 8 x \]The solution is \( \text{None of the above.} \), which is option E.\begin{enumerate}[label=\Alph*.]
\item \( (a, b], \text{ where } a \in [3.75, 12.75] \text{ and } b \in [-15.25, -10.25] \)

$(7.75, -12.25]$, which is the correct interval but negatives of the actual endpoints.
\item \( [a, b), \text{ where } a \in [7.75, 11.75] \text{ and } b \in [-16.25, -9.25] \)

$[7.75, -12.25)$, which corresponds to flipping the inequality and getting negatives of the actual endpoints.
\item \( (-\infty, a] \cup (b, \infty), \text{ where } a \in [7.75, 8.75] \text{ and } b \in [-13.25, -9.25] \)

$(-\infty, 7.75] \cup (-12.25, \infty)$, which corresponds to displaying the and-inequality as an or-inequality AND flipping the inequality AND getting negatives of the actual endpoints.
\item \( (-\infty, a) \cup [b, \infty), \text{ where } a \in [7.75, 10.75] \text{ and } b \in [-14.25, -11.25] \)

$(-\infty, 7.75) \cup [-12.25, \infty)$, which corresponds to displaying the and-inequality as an or-inequality and getting negatives of the actual endpoints.
\item \( \text{None of the above.} \)

* This is correct as the answer should be $(-7.75, 12.25]$.
\end{enumerate}

\textbf{General Comment:} To solve, you will need to break up the compound inequality into two inequalities. Be sure to keep track of the inequality! It may be best to draw a number line and graph your solution.
}
\litem{
Solve the linear inequality below. Then, choose the constant and interval combination that describes the solution set.
\[ -8 - 5 x \leq \frac{-42 x + 6}{9} < 9 - 9 x \]The solution is \( \text{None of the above.} \), which is option E.\begin{enumerate}[label=\Alph*.]
\item \( (-\infty, a] \cup (b, \infty), \text{ where } a \in [22, 27] \text{ and } b \in [-5.92, -0.92] \)

$(-\infty, 26.00] \cup (-1.92, \infty)$, which corresponds to displaying the and-inequality as an or-inequality and getting negatives of the actual endpoints.
\item \( (-\infty, a) \cup [b, \infty), \text{ where } a \in [26, 28] \text{ and } b \in [-4.92, -0.92] \)

$(-\infty, 26.00) \cup [-1.92, \infty)$, which corresponds to displaying the and-inequality as an or-inequality AND flipping the inequality AND getting negatives of the actual endpoints.
\item \( (a, b], \text{ where } a \in [23, 29] \text{ and } b \in [-3.92, 1.08] \)

$(26.00, -1.92]$, which corresponds to flipping the inequality and getting negatives of the actual endpoints.
\item \( [a, b), \text{ where } a \in [25, 32] \text{ and } b \in [-4.92, -0.92] \)

$[26.00, -1.92)$, which is the correct interval but negatives of the actual endpoints.
\item \( \text{None of the above.} \)

* This is correct as the answer should be $[-26.00, 1.92)$.
\end{enumerate}

\textbf{General Comment:} To solve, you will need to break up the compound inequality into two inequalities. Be sure to keep track of the inequality! It may be best to draw a number line and graph your solution.
}
\litem{
Solve the linear inequality below. Then, choose the constant and interval combination that describes the solution set.
\[ \frac{-3}{8} + \frac{6}{3} x < \frac{8}{9} x + \frac{10}{6} \]The solution is \( (-\infty, 1.837) \), which is option D.\begin{enumerate}[label=\Alph*.]
\item \( (a, \infty), \text{ where } a \in [0.84, 2.84] \)

 $(1.837, \infty)$, which corresponds to switching the direction of the interval. You likely did this if you did not flip the inequality when dividing by a negative!
\item \( (a, \infty), \text{ where } a \in [-2.84, 0.16] \)

 $(-1.837, \infty)$, which corresponds to switching the direction of the interval AND negating the endpoint. You likely did this if you did not flip the inequality when dividing by a negative as well as not moving values over to a side properly.
\item \( (-\infty, a), \text{ where } a \in [-1.84, 1.16] \)

 $(-\infty, -1.837)$, which corresponds to negating the endpoint of the solution.
\item \( (-\infty, a), \text{ where } a \in [-0.16, 6.84] \)

* $(-\infty, 1.837)$, which is the correct option.
\item \( \text{None of the above}. \)

You may have chosen this if you thought the inequality did not match the ends of the intervals.
\end{enumerate}

\textbf{General Comment:} Remember that less/greater than or equal to includes the endpoint, while less/greater do not. Also, remember that you need to flip the inequality when you multiply or divide by a negative.
}
\litem{
Using an interval or intervals, describe all the $x$-values within or including a distance of the given values.
\[ \text{ Less than } 7 \text{ units from the number } 1. \]The solution is \( \text{None of the above} \), which is option E.\begin{enumerate}[label=\Alph*.]
\item \( [6, 8] \)

This describes the values no more than 1 from 7
\item \( (-\infty, 6) \cup (8, \infty) \)

This describes the values more than 1 from 7
\item \( (6, 8) \)

This describes the values less than 1 from 7
\item \( (-\infty, 6] \cup [8, \infty) \)

This describes the values no less than 1 from 7
\item \( \text{None of the above} \)

Options A-D described the values [more/less than] 1 units from 7, which is the reverse of what the question asked.
\end{enumerate}

\textbf{General Comment:} When thinking about this language, it helps to draw a number line and try points.
}
\end{enumerate}

\end{document}