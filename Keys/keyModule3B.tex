\documentclass{extbook}[14pt]
\usepackage{multicol, enumerate, enumitem, hyperref, color, soul, setspace, parskip, fancyhdr, amssymb, amsthm, amsmath, bbm, latexsym, units, mathtools}
\everymath{\displaystyle}
\usepackage[headsep=0.5cm,headheight=0cm, left=1 in,right= 1 in,top= 1 in,bottom= 1 in]{geometry}
\usepackage{dashrule}  % Package to use the command below to create lines between items
\newcommand{\litem}[1]{\item #1

\rule{\textwidth}{0.4pt}}
\pagestyle{fancy}
\lhead{}
\chead{Answer Key for Progress Quiz 4 Version B}
\rhead{}
\lfoot{8448-1521}
\cfoot{}
\rfoot{Fall 2020}
\begin{document}
\textbf{This key should allow you to understand why you choose the option you did (beyond just getting a question right or wrong). \href{https://xronos.clas.ufl.edu/mac1105spring2020/courseDescriptionAndMisc/Exams/LearningFromResults}{More instructions on how to use this key can be found here}.}

\textbf{If you have a suggestion to make the keys better, \href{https://forms.gle/CZkbZmPbC9XALEE88}{please fill out the short survey here}.}

\textit{Note: This key is auto-generated and may contain issues and/or errors. The keys are reviewed after each exam to ensure grading is done accurately. If there are issues (like duplicate options), they are noted in the offline gradebook. The keys are a work-in-progress to give students as many resources to improve as possible.}

\rule{\textwidth}{0.4pt}

\begin{enumerate}\litem{
Using an interval or intervals, describe all the $x$-values within or including a distance of the given values.
\[ \text{ More than } 4 \text{ units from the number } 3. \]
The solution is \( \text{None of the above} \), which is option E.\begin{enumerate}[label=\Alph*.]
\item \( (1, 7) \)

This describes the values less than 3 from 4
\item \( [1, 7] \)

This describes the values no more than 3 from 4
\item \( (-\infty, 1] \cup [7, \infty) \)

This describes the values no less than 3 from 4
\item \( (-\infty, 1) \cup (7, \infty) \)

This describes the values more than 3 from 4
\item \( \text{None of the above} \)

Options A-D described the values [more/less than] 3 units from 4, which is the reverse of what the question asked.
\end{enumerate}

\textbf{General Comment:} When thinking about this language, it helps to draw a number line and try points.
}
\litem{
Solve the linear inequality below. Then, choose the constant and interval combination that describes the solution set.
\[ -10x -6 \leq -4x + 9 \]
The solution is \( [-2.5, \infty) \), which is option A.\begin{enumerate}[label=\Alph*.]
\item \( [a, \infty), \text{ where } a \in [-5.5, -0.5] \)

* $[-2.5, \infty)$, which is the correct option.
\item \( (-\infty, a], \text{ where } a \in [1.5, 5.8] \)

 $(-\infty, 2.5]$, which corresponds to switching the direction of the interval AND negating the endpoint. You likely did this if you did not flip the inequality when dividing by a negative as well as not moving values over to a side properly.
\item \( [a, \infty), \text{ where } a \in [1.5, 8.5] \)

 $[2.5, \infty)$, which corresponds to negating the endpoint of the solution.
\item \( (-\infty, a], \text{ where } a \in [-4.4, -0.7] \)

 $(-\infty, -2.5]$, which corresponds to switching the direction of the interval. You likely did this if you did not flip the inequality when dividing by a negative!
\item \( \text{None of the above}. \)

You may have chosen this if you thought the inequality did not match the ends of the intervals.
\end{enumerate}

\textbf{General Comment:} Remember that less/greater than or equal to includes the endpoint, while less/greater do not. Also, remember that you need to flip the inequality when you multiply or divide by a negative.
}
\litem{
Solve the linear inequality below. Then, choose the constant and interval combination that describes the solution set.
\[ \frac{-6}{9} + \frac{5}{5} x > \frac{8}{6} x - \frac{10}{2} \]
The solution is \( (-\infty, 13.0) \), which is option C.\begin{enumerate}[label=\Alph*.]
\item \( (-\infty, a), \text{ where } a \in [-14, -7] \)

 $(-\infty, -13.0)$, which corresponds to negating the endpoint of the solution.
\item \( (a, \infty), \text{ where } a \in [-15, -12] \)

 $(-13.0, \infty)$, which corresponds to switching the direction of the interval AND negating the endpoint. You likely did this if you did not flip the inequality when dividing by a negative as well as not moving values over to a side properly.
\item \( (-\infty, a), \text{ where } a \in [11, 15] \)

* $(-\infty, 13.0)$, which is the correct option.
\item \( (a, \infty), \text{ where } a \in [10, 19] \)

 $(13.0, \infty)$, which corresponds to switching the direction of the interval. You likely did this if you did not flip the inequality when dividing by a negative!
\item \( \text{None of the above}. \)

You may have chosen this if you thought the inequality did not match the ends of the intervals.
\end{enumerate}

\textbf{General Comment:} Remember that less/greater than or equal to includes the endpoint, while less/greater do not. Also, remember that you need to flip the inequality when you multiply or divide by a negative.
}
\litem{
Solve the linear inequality below. Then, choose the constant and interval combination that describes the solution set.
\[ 3 + 4 x > 7 x \text{ or } 6 + 6 x < 7 x \]
The solution is \( (-\infty, 1.0) \text{ or } (6.0, \infty) \), which is option A.\begin{enumerate}[label=\Alph*.]
\item \( (-\infty, a) \cup (b, \infty), \text{ where } a \in [1, 4] \text{ and } b \in [4, 11] \)

 * Correct option.
\item \( (-\infty, a] \cup [b, \infty), \text{ where } a \in [-10, -5] \text{ and } b \in [-2, 0] \)

Corresponds to including the endpoints AND negating.
\item \( (-\infty, a] \cup [b, \infty), \text{ where } a \in [-2, 2] \text{ and } b \in [6, 7] \)

Corresponds to including the endpoints (when they should be excluded).
\item \( (-\infty, a) \cup (b, \infty), \text{ where } a \in [-6, -2] \text{ and } b \in [-4, 3] \)

Corresponds to inverting the inequality and negating the solution.
\item \( (-\infty, \infty) \)

Corresponds to the variable canceling, which does not happen in this instance.
\end{enumerate}

\textbf{General Comment:} When multiplying or dividing by a negative, flip the sign.
}
\litem{
Solve the linear inequality below. Then, choose the constant and interval combination that describes the solution set.
\[ -7 - 4 x \leq \frac{-16 x - 4}{6} < 8 - 3 x \]
The solution is \( [-4.75, 26.00) \), which is option B.\begin{enumerate}[label=\Alph*.]
\item \( (-\infty, a) \cup [b, \infty), \text{ where } a \in [-5.75, 3.25] \text{ and } b \in [22, 31] \)

$(-\infty, -4.75) \cup [26.00, \infty)$, which corresponds to displaying the and-inequality as an or-inequality AND flipping the inequality.
\item \( [a, b), \text{ where } a \in [-7.75, -1.75] \text{ and } b \in [21, 27] \)

$[-4.75, 26.00)$, which is the correct option.
\item \( (-\infty, a] \cup (b, \infty), \text{ where } a \in [-10.75, -3.75] \text{ and } b \in [25, 29] \)

$(-\infty, -4.75] \cup (26.00, \infty)$, which corresponds to displaying the and-inequality as an or-inequality.
\item \( (a, b], \text{ where } a \in [-7.75, -3.75] \text{ and } b \in [26, 29] \)

$(-4.75, 26.00]$, which corresponds to flipping the inequality.
\item \( \text{None of the above.} \)


\end{enumerate}

\textbf{General Comment:} To solve, you will need to break up the compound inequality into two inequalities. Be sure to keep track of the inequality! It may be best to draw a number line and graph your solution.
}
\litem{
Using an interval or intervals, describe all the $x$-values within or including a distance of the given values.
\[ \text{ More than } 10 \text{ units from the number } 7. \]
The solution is \( (-\infty, -3) \cup (17, \infty) \), which is option B.\begin{enumerate}[label=\Alph*.]
\item \( (-3, 17) \)

This describes the values less than 10 from 7
\item \( (-\infty, -3) \cup (17, \infty) \)

This describes the values more than 10 from 7
\item \( (-\infty, -3] \cup [17, \infty) \)

This describes the values no less than 10 from 7
\item \( [-3, 17] \)

This describes the values no more than 10 from 7
\item \( \text{None of the above} \)

You likely thought the values in the interval were not correct.
\end{enumerate}

\textbf{General Comment:} When thinking about this language, it helps to draw a number line and try points.
}
\litem{
Solve the linear inequality below. Then, choose the constant and interval combination that describes the solution set.
\[ \frac{-6}{4} - \frac{4}{6} x > \frac{8}{7} x + \frac{4}{5} \]
The solution is \( (-\infty, -1.271) \), which is option A.\begin{enumerate}[label=\Alph*.]
\item \( (-\infty, a), \text{ where } a \in [-5.27, 0.73] \)

* $(-\infty, -1.271)$, which is the correct option.
\item \( (a, \infty), \text{ where } a \in [-2.27, 0.73] \)

 $(-1.271, \infty)$, which corresponds to switching the direction of the interval. You likely did this if you did not flip the inequality when dividing by a negative!
\item \( (a, \infty), \text{ where } a \in [0.27, 2.27] \)

 $(1.271, \infty)$, which corresponds to switching the direction of the interval AND negating the endpoint. You likely did this if you did not flip the inequality when dividing by a negative as well as not moving values over to a side properly.
\item \( (-\infty, a), \text{ where } a \in [1.27, 4.27] \)

 $(-\infty, 1.271)$, which corresponds to negating the endpoint of the solution.
\item \( \text{None of the above}. \)

You may have chosen this if you thought the inequality did not match the ends of the intervals.
\end{enumerate}

\textbf{General Comment:} Remember that less/greater than or equal to includes the endpoint, while less/greater do not. Also, remember that you need to flip the inequality when you multiply or divide by a negative.
}
\litem{
Solve the linear inequality below. Then, choose the constant and interval combination that describes the solution set.
\[ -4 + 8 x > 10 x \text{ or } 9 + 3 x < 5 x \]
The solution is \( (-\infty, -2.0) \text{ or } (4.5, \infty) \), which is option B.\begin{enumerate}[label=\Alph*.]
\item \( (-\infty, a] \cup [b, \infty), \text{ where } a \in [-9.5, -3.5] \text{ and } b \in [0, 3] \)

Corresponds to including the endpoints AND negating.
\item \( (-\infty, a) \cup (b, \infty), \text{ where } a \in [-3, 1] \text{ and } b \in [4.1, 6] \)

 * Correct option.
\item \( (-\infty, a] \cup [b, \infty), \text{ where } a \in [-2, 0] \text{ and } b \in [3.5, 5.5] \)

Corresponds to including the endpoints (when they should be excluded).
\item \( (-\infty, a) \cup (b, \infty), \text{ where } a \in [-4.5, -3.5] \text{ and } b \in [1.4, 4.2] \)

Corresponds to inverting the inequality and negating the solution.
\item \( (-\infty, \infty) \)

Corresponds to the variable canceling, which does not happen in this instance.
\end{enumerate}

\textbf{General Comment:} When multiplying or dividing by a negative, flip the sign.
}
\litem{
Solve the linear inequality below. Then, choose the constant and interval combination that describes the solution set.
\[ 9x + 9 < 10x + 5 \]
The solution is \( (4.0, \infty) \), which is option D.\begin{enumerate}[label=\Alph*.]
\item \( (-\infty, a), \text{ where } a \in [-1, 7] \)

 $(-\infty, 4.0)$, which corresponds to switching the direction of the interval. You likely did this if you did not flip the inequality when dividing by a negative!
\item \( (-\infty, a), \text{ where } a \in [-9, 1] \)

 $(-\infty, -4.0)$, which corresponds to switching the direction of the interval AND negating the endpoint. You likely did this if you did not flip the inequality when dividing by a negative as well as not moving values over to a side properly.
\item \( (a, \infty), \text{ where } a \in [-11, -3] \)

 $(-4.0, \infty)$, which corresponds to negating the endpoint of the solution.
\item \( (a, \infty), \text{ where } a \in [0, 7] \)

* $(4.0, \infty)$, which is the correct option.
\item \( \text{None of the above}. \)

You may have chosen this if you thought the inequality did not match the ends of the intervals.
\end{enumerate}

\textbf{General Comment:} Remember that less/greater than or equal to includes the endpoint, while less/greater do not. Also, remember that you need to flip the inequality when you multiply or divide by a negative.
}
\litem{
Solve the linear inequality below. Then, choose the constant and interval combination that describes the solution set.
\[ -4 + 7 x < \frac{67 x + 4}{9} \leq 4 + 5 x \]
The solution is \( \text{None of the above.} \), which is option E.\begin{enumerate}[label=\Alph*.]
\item \( (a, b], \text{ where } a \in [9, 13] \text{ and } b \in [-2.1, 0.6] \)

$(10.00, -1.45]$, which is the correct interval but negatives of the actual endpoints.
\item \( (-\infty, a] \cup (b, \infty), \text{ where } a \in [8, 15] \text{ and } b \in [-1.6, 0.2] \)

$(-\infty, 10.00] \cup (-1.45, \infty)$, which corresponds to displaying the and-inequality as an or-inequality AND flipping the inequality AND getting negatives of the actual endpoints.
\item \( [a, b), \text{ where } a \in [10, 11] \text{ and } b \in [-3, 0.2] \)

$[10.00, -1.45)$, which corresponds to flipping the inequality and getting negatives of the actual endpoints.
\item \( (-\infty, a) \cup [b, \infty), \text{ where } a \in [8, 14] \text{ and } b \in [-3.45, 0.55] \)

$(-\infty, 10.00) \cup [-1.45, \infty)$, which corresponds to displaying the and-inequality as an or-inequality and getting negatives of the actual endpoints.
\item \( \text{None of the above.} \)

* This is correct as the answer should be $(-10.00, 1.45]$.
\end{enumerate}

\textbf{General Comment:} To solve, you will need to break up the compound inequality into two inequalities. Be sure to keep track of the inequality! It may be best to draw a number line and graph your solution.
}
\end{enumerate}

\end{document}