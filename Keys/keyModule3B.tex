\documentclass{extbook}[14pt]
\usepackage{multicol, enumerate, enumitem, hyperref, color, soul, setspace, parskip, fancyhdr, amssymb, amsthm, amsmath, bbm, latexsym, units, mathtools}
\everymath{\displaystyle}
\usepackage[headsep=0.5cm,headheight=0cm, left=1 in,right= 1 in,top= 1 in,bottom= 1 in]{geometry}
\usepackage{dashrule}  % Package to use the command below to create lines between items
\newcommand{\litem}[1]{\item #1

\rule{\textwidth}{0.4pt}}
\pagestyle{fancy}
\lhead{}
\chead{Answer Key for Progress Quiz 4 Version B}
\rhead{}
\lfoot{6286-1986}
\cfoot{}
\rfoot{Fall 2020}
\begin{document}
\textbf{This key should allow you to understand why you choose the option you did (beyond just getting a question right or wrong). \href{https://xronos.clas.ufl.edu/mac1105spring2020/courseDescriptionAndMisc/Exams/LearningFromResults}{More instructions on how to use this key can be found here}.}

\textbf{If you have a suggestion to make the keys better, \href{https://forms.gle/CZkbZmPbC9XALEE88}{please fill out the short survey here}.}

\textit{Note: This key is auto-generated and may contain issues and/or errors. The keys are reviewed after each exam to ensure grading is done accurately. If there are issues (like duplicate options), they are noted in the offline gradebook. The keys are a work-in-progress to give students as many resources to improve as possible.}

\rule{\textwidth}{0.4pt}

\begin{enumerate}\litem{
Solve the linear inequality below. Then, choose the constant and interval combination that describes the solution set.
\[ -8 + 9 x > 11 x \text{ or } -8 - 3 x < 5 x \]
The solution is \( (-\infty, -4.0) \text{ or } (-1.0, \infty) \), which is option C.\begin{enumerate}[label=\Alph*.]
\item \( (-\infty, a) \cup (b, \infty), \text{ where } a \in [1, 4] \text{ and } b \in [4, 9] \)

Corresponds to inverting the inequality and negating the solution.
\item \( (-\infty, a] \cup [b, \infty), \text{ where } a \in [-2, 4] \text{ and } b \in [0, 6] \)

Corresponds to including the endpoints AND negating.
\item \( (-\infty, a) \cup (b, \infty), \text{ where } a \in [-4, -2] \text{ and } b \in [-1, 1] \)

 * Correct option.
\item \( (-\infty, a] \cup [b, \infty), \text{ where } a \in [-7, -3] \text{ and } b \in [-1, 1] \)

Corresponds to including the endpoints (when they should be excluded).
\item \( (-\infty, \infty) \)

Corresponds to the variable canceling, which does not happen in this instance.
\end{enumerate}

\textbf{General Comment:} When multiplying or dividing by a negative, flip the sign.
}
\litem{
Using an interval or intervals, describe all the $x$-values within or including a distance of the given values.
\[ \text{ More than } 4 \text{ units from the number } 6. \]
The solution is \( \text{None of the above} \), which is option E.\begin{enumerate}[label=\Alph*.]
\item \( (-2, 10) \)

This describes the values less than 6 from 4
\item \( (-\infty, -2] \cup [10, \infty) \)

This describes the values no less than 6 from 4
\item \( (-\infty, -2) \cup (10, \infty) \)

This describes the values more than 6 from 4
\item \( [-2, 10] \)

This describes the values no more than 6 from 4
\item \( \text{None of the above} \)

Options A-D described the values [more/less than] 6 units from 4, which is the reverse of what the question asked.
\end{enumerate}

\textbf{General Comment:} When thinking about this language, it helps to draw a number line and try points.
}
\litem{
Solve the linear inequality below. Then, choose the constant and interval combination that describes the solution set.
\[ -3x + 9 \leq 7x -4 \]
The solution is \( [1.3, \infty) \), which is option A.\begin{enumerate}[label=\Alph*.]
\item \( [a, \infty), \text{ where } a \in [-0.7, 5.3] \)

* $[1.3, \infty)$, which is the correct option.
\item \( (-\infty, a], \text{ where } a \in [-1.6, -1.2] \)

 $(-\infty, -1.3]$, which corresponds to switching the direction of the interval AND negating the endpoint. You likely did this if you did not flip the inequality when dividing by a negative as well as not moving values over to a side properly.
\item \( (-\infty, a], \text{ where } a \in [-0.7, 4.4] \)

 $(-\infty, 1.3]$, which corresponds to switching the direction of the interval. You likely did this if you did not flip the inequality when dividing by a negative!
\item \( [a, \infty), \text{ where } a \in [-8.3, -0.3] \)

 $[-1.3, \infty)$, which corresponds to negating the endpoint of the solution.
\item \( \text{None of the above}. \)

You may have chosen this if you thought the inequality did not match the ends of the intervals.
\end{enumerate}

\textbf{General Comment:} Remember that less/greater than or equal to includes the endpoint, while less/greater do not. Also, remember that you need to flip the inequality when you multiply or divide by a negative.
}
\litem{
Using an interval or intervals, describe all the $x$-values within or including a distance of the given values.
\[ \text{ Less than } 3 \text{ units from the number } 8. \]
The solution is \( (5, 11) \), which is option D.\begin{enumerate}[label=\Alph*.]
\item \( (-\infty, 5] \cup [11, \infty) \)

This describes the values no less than 3 from 8
\item \( (-\infty, 5) \cup (11, \infty) \)

This describes the values more than 3 from 8
\item \( [5, 11] \)

This describes the values no more than 3 from 8
\item \( (5, 11) \)

This describes the values less than 3 from 8
\item \( \text{None of the above} \)

You likely thought the values in the interval were not correct.
\end{enumerate}

\textbf{General Comment:} When thinking about this language, it helps to draw a number line and try points.
}
\litem{
Solve the linear inequality below. Then, choose the constant and interval combination that describes the solution set.
\[ \frac{4}{7} - \frac{8}{3} x \geq \frac{-4}{9} x - \frac{9}{2} \]
The solution is \( (-\infty, 2.282] \), which is option D.\begin{enumerate}[label=\Alph*.]
\item \( [a, \infty), \text{ where } a \in [1.28, 6.28] \)

 $[2.282, \infty)$, which corresponds to switching the direction of the interval. You likely did this if you did not flip the inequality when dividing by a negative!
\item \( (-\infty, a], \text{ where } a \in [-4.28, 0.72] \)

 $(-\infty, -2.282]$, which corresponds to negating the endpoint of the solution.
\item \( [a, \infty), \text{ where } a \in [-3.28, -1.28] \)

 $[-2.282, \infty)$, which corresponds to switching the direction of the interval AND negating the endpoint. You likely did this if you did not flip the inequality when dividing by a negative as well as not moving values over to a side properly.
\item \( (-\infty, a], \text{ where } a \in [2.28, 3.28] \)

* $(-\infty, 2.282]$, which is the correct option.
\item \( \text{None of the above}. \)

You may have chosen this if you thought the inequality did not match the ends of the intervals.
\end{enumerate}

\textbf{General Comment:} Remember that less/greater than or equal to includes the endpoint, while less/greater do not. Also, remember that you need to flip the inequality when you multiply or divide by a negative.
}
\litem{
Solve the linear inequality below. Then, choose the constant and interval combination that describes the solution set.
\[ 9 - 3 x > 6 x \text{ or } 8 + 7 x < 10 x \]
The solution is \( (-\infty, 1.0) \text{ or } (2.667, \infty) \), which is option A.\begin{enumerate}[label=\Alph*.]
\item \( (-\infty, a) \cup (b, \infty), \text{ where } a \in [1, 3] \text{ and } b \in [0.67, 4.67] \)

 * Correct option.
\item \( (-\infty, a] \cup [b, \infty), \text{ where } a \in [-5.67, 0.33] \text{ and } b \in [-3, 2] \)

Corresponds to including the endpoints AND negating.
\item \( (-\infty, a) \cup (b, \infty), \text{ where } a \in [-2.67, -1.67] \text{ and } b \in [-2, 1] \)

Corresponds to inverting the inequality and negating the solution.
\item \( (-\infty, a] \cup [b, \infty), \text{ where } a \in [0, 3] \text{ and } b \in [0.67, 5.67] \)

Corresponds to including the endpoints (when they should be excluded).
\item \( (-\infty, \infty) \)

Corresponds to the variable canceling, which does not happen in this instance.
\end{enumerate}

\textbf{General Comment:} When multiplying or dividing by a negative, flip the sign.
}
\litem{
Solve the linear inequality below. Then, choose the constant and interval combination that describes the solution set.
\[ \frac{-10}{2} - \frac{9}{4} x > \frac{-6}{7} x - \frac{8}{6} \]
The solution is \( (-\infty, -2.632) \), which is option C.\begin{enumerate}[label=\Alph*.]
\item \( (a, \infty), \text{ where } a \in [-3.63, -1.63] \)

 $(-2.632, \infty)$, which corresponds to switching the direction of the interval. You likely did this if you did not flip the inequality when dividing by a negative!
\item \( (-\infty, a), \text{ where } a \in [2.63, 6.63] \)

 $(-\infty, 2.632)$, which corresponds to negating the endpoint of the solution.
\item \( (-\infty, a), \text{ where } a \in [-4.63, -0.63] \)

* $(-\infty, -2.632)$, which is the correct option.
\item \( (a, \infty), \text{ where } a \in [1.63, 4.63] \)

 $(2.632, \infty)$, which corresponds to switching the direction of the interval AND negating the endpoint. You likely did this if you did not flip the inequality when dividing by a negative as well as not moving values over to a side properly.
\item \( \text{None of the above}. \)

You may have chosen this if you thought the inequality did not match the ends of the intervals.
\end{enumerate}

\textbf{General Comment:} Remember that less/greater than or equal to includes the endpoint, while less/greater do not. Also, remember that you need to flip the inequality when you multiply or divide by a negative.
}
\litem{
Solve the linear inequality below. Then, choose the constant and interval combination that describes the solution set.
\[ -10x -9 \geq -5x + 8 \]
The solution is \( (-\infty, -3.4] \), which is option A.\begin{enumerate}[label=\Alph*.]
\item \( (-\infty, a], \text{ where } a \in [-3.4, -0.4] \)

* $(-\infty, -3.4]$, which is the correct option.
\item \( [a, \infty), \text{ where } a \in [0.4, 4.4] \)

 $[3.4, \infty)$, which corresponds to switching the direction of the interval AND negating the endpoint. You likely did this if you did not flip the inequality when dividing by a negative as well as not moving values over to a side properly.
\item \( (-\infty, a], \text{ where } a \in [2.4, 10.4] \)

 $(-\infty, 3.4]$, which corresponds to negating the endpoint of the solution.
\item \( [a, \infty), \text{ where } a \in [-11.4, -0.4] \)

 $[-3.4, \infty)$, which corresponds to switching the direction of the interval. You likely did this if you did not flip the inequality when dividing by a negative!
\item \( \text{None of the above}. \)

You may have chosen this if you thought the inequality did not match the ends of the intervals.
\end{enumerate}

\textbf{General Comment:} Remember that less/greater than or equal to includes the endpoint, while less/greater do not. Also, remember that you need to flip the inequality when you multiply or divide by a negative.
}
\litem{
Solve the linear inequality below. Then, choose the constant and interval combination that describes the solution set.
\[ -4 - 8 x \leq \frac{-22 x + 4}{4} < 6 - 6 x \]
The solution is \( [-2.00, 10.00) \), which is option C.\begin{enumerate}[label=\Alph*.]
\item \( (a, b], \text{ where } a \in [-4, 0] \text{ and } b \in [10, 12] \)

$(-2.00, 10.00]$, which corresponds to flipping the inequality.
\item \( (-\infty, a) \cup [b, \infty), \text{ where } a \in [-4, 0] \text{ and } b \in [9, 13] \)

$(-\infty, -2.00) \cup [10.00, \infty)$, which corresponds to displaying the and-inequality as an or-inequality AND flipping the inequality.
\item \( [a, b), \text{ where } a \in [-4, 1] \text{ and } b \in [6, 12] \)

$[-2.00, 10.00)$, which is the correct option.
\item \( (-\infty, a] \cup (b, \infty), \text{ where } a \in [-3, 1] \text{ and } b \in [5, 11] \)

$(-\infty, -2.00] \cup (10.00, \infty)$, which corresponds to displaying the and-inequality as an or-inequality.
\item \( \text{None of the above.} \)


\end{enumerate}

\textbf{General Comment:} To solve, you will need to break up the compound inequality into two inequalities. Be sure to keep track of the inequality! It may be best to draw a number line and graph your solution.
}
\litem{
Solve the linear inequality below. Then, choose the constant and interval combination that describes the solution set.
\[ 4 + 6 x \leq \frac{44 x - 4}{6} < 7 + 7 x \]
The solution is \( \text{None of the above.} \), which is option E.\begin{enumerate}[label=\Alph*.]
\item \( (-\infty, a] \cup (b, \infty), \text{ where } a \in [-4.5, -0.5] \text{ and } b \in [-27, -22] \)

$(-\infty, -3.50] \cup (-23.00, \infty)$, which corresponds to displaying the and-inequality as an or-inequality and getting negatives of the actual endpoints.
\item \( (-\infty, a) \cup [b, \infty), \text{ where } a \in [-6.5, -2.5] \text{ and } b \in [-26, -21] \)

$(-\infty, -3.50) \cup [-23.00, \infty)$, which corresponds to displaying the and-inequality as an or-inequality AND flipping the inequality AND getting negatives of the actual endpoints.
\item \( (a, b], \text{ where } a \in [-7.5, -2.5] \text{ and } b \in [-23, -22] \)

$(-3.50, -23.00]$, which corresponds to flipping the inequality and getting negatives of the actual endpoints.
\item \( [a, b), \text{ where } a \in [-3.5, 0.5] \text{ and } b \in [-24, -21] \)

$[-3.50, -23.00)$, which is the correct interval but negatives of the actual endpoints.
\item \( \text{None of the above.} \)

* This is correct as the answer should be $[3.50, 23.00)$.
\end{enumerate}

\textbf{General Comment:} To solve, you will need to break up the compound inequality into two inequalities. Be sure to keep track of the inequality! It may be best to draw a number line and graph your solution.
}
\end{enumerate}

\end{document}