\documentclass{extbook}[14pt]
\usepackage{multicol, enumerate, enumitem, hyperref, color, soul, setspace, parskip, fancyhdr, amssymb, amsthm, amsmath, bbm, latexsym, units, mathtools}
\everymath{\displaystyle}
\usepackage[headsep=0.5cm,headheight=0cm, left=1 in,right= 1 in,top= 1 in,bottom= 1 in]{geometry}
\usepackage{dashrule}  % Package to use the command below to create lines between items
\newcommand{\litem}[1]{\item #1

\rule{\textwidth}{0.4pt}}
\pagestyle{fancy}
\lhead{}
\chead{Answer Key for Progress Quiz 9 Version B}
\rhead{}
\lfoot{8590-6105}
\cfoot{}
\rfoot{Fall 2020}
\begin{document}
\textbf{This key should allow you to understand why you choose the option you did (beyond just getting a question right or wrong). \href{https://xronos.clas.ufl.edu/mac1105spring2020/courseDescriptionAndMisc/Exams/LearningFromResults}{More instructions on how to use this key can be found here}.}

\textbf{If you have a suggestion to make the keys better, \href{https://forms.gle/CZkbZmPbC9XALEE88}{please fill out the short survey here}.}

\textit{Note: This key is auto-generated and may contain issues and/or errors. The keys are reviewed after each exam to ensure grading is done accurately. If there are issues (like duplicate options), they are noted in the offline gradebook. The keys are a work-in-progress to give students as many resources to improve as possible.}

\rule{\textwidth}{0.4pt}

\begin{enumerate}\litem{
Solve the linear inequality below. Then, choose the constant and interval combination that describes the solution set.
\[ -6x + 3 \leq 8x -7 \]

The solution is \( [0.714, \infty) \), which is option A.\begin{enumerate}[label=\Alph*.]
\item \( [a, \infty), \text{ where } a \in [-0.6, 0.94] \)

* $[0.714, \infty)$, which is the correct option.
\item \( (-\infty, a], \text{ where } a \in [-5.71, 0.29] \)

 $(-\infty, -0.714]$, which corresponds to switching the direction of the interval AND negating the endpoint. You likely did this if you did not flip the inequality when dividing by a negative as well as not moving values over to a side properly.
\item \( [a, \infty), \text{ where } a \in [-1.54, -0.24] \)

 $[-0.714, \infty)$, which corresponds to negating the endpoint of the solution.
\item \( (-\infty, a], \text{ where } a \in [0.71, 1.71] \)

 $(-\infty, 0.714]$, which corresponds to switching the direction of the interval. You likely did this if you did not flip the inequality when dividing by a negative!
\item \( \text{None of the above}. \)

You may have chosen this if you thought the inequality did not match the ends of the intervals.
\end{enumerate}

\textbf{General Comment:} Remember that less/greater than or equal to includes the endpoint, while less/greater do not. Also, remember that you need to flip the inequality when you multiply or divide by a negative.
}
\litem{
Solve the linear inequality below. Then, choose the constant and interval combination that describes the solution set.
\[ -5x -6 \leq 10x -3 \]

The solution is \( [-0.2, \infty) \), which is option A.\begin{enumerate}[label=\Alph*.]
\item \( [a, \infty), \text{ where } a \in [-0.64, -0.18] \)

* $[-0.2, \infty)$, which is the correct option.
\item \( (-\infty, a], \text{ where } a \in [-0.82, 0.04] \)

 $(-\infty, -0.2]$, which corresponds to switching the direction of the interval. You likely did this if you did not flip the inequality when dividing by a negative!
\item \( [a, \infty), \text{ where } a \in [-0.1, 0.68] \)

 $[0.2, \infty)$, which corresponds to negating the endpoint of the solution.
\item \( (-\infty, a], \text{ where } a \in [0.02, 0.52] \)

 $(-\infty, 0.2]$, which corresponds to switching the direction of the interval AND negating the endpoint. You likely did this if you did not flip the inequality when dividing by a negative as well as not moving values over to a side properly.
\item \( \text{None of the above}. \)

You may have chosen this if you thought the inequality did not match the ends of the intervals.
\end{enumerate}

\textbf{General Comment:} Remember that less/greater than or equal to includes the endpoint, while less/greater do not. Also, remember that you need to flip the inequality when you multiply or divide by a negative.
}
\litem{
Solve the linear inequality below. Then, choose the constant and interval combination that describes the solution set.
\[ -9 + 8 x > 10 x \text{ or } -8 + 8 x < 11 x \]

The solution is \( (-\infty, -4.5) \text{ or } (-2.667, \infty) \), which is option B.\begin{enumerate}[label=\Alph*.]
\item \( (-\infty, a) \cup (b, \infty), \text{ where } a \in [0.67, 3.67] \text{ and } b \in [1.5, 6.5] \)

Corresponds to inverting the inequality and negating the solution.
\item \( (-\infty, a) \cup (b, \infty), \text{ where } a \in [-6.5, -2.5] \text{ and } b \in [-8.67, 0.33] \)

 * Correct option.
\item \( (-\infty, a] \cup [b, \infty), \text{ where } a \in [-4.5, -3.5] \text{ and } b \in [-2.67, 1.33] \)

Corresponds to including the endpoints (when they should be excluded).
\item \( (-\infty, a] \cup [b, \infty), \text{ where } a \in [1.67, 4.67] \text{ and } b \in [3.5, 5.5] \)

Corresponds to including the endpoints AND negating.
\item \( (-\infty, \infty) \)

Corresponds to the variable canceling, which does not happen in this instance.
\end{enumerate}

\textbf{General Comment:} When multiplying or dividing by a negative, flip the sign.
}
\litem{
Solve the linear inequality below. Then, choose the constant and interval combination that describes the solution set.
\[ 6 - 8 x < \frac{-43 x + 8}{6} \leq 8 - 8 x \]

The solution is \( (5.60, 8.00] \), which is option B.\begin{enumerate}[label=\Alph*.]
\item \( (-\infty, a] \cup (b, \infty), \text{ where } a \in [5.6, 7.6] \text{ and } b \in [5, 12] \)

$(-\infty, 5.60] \cup (8.00, \infty)$, which corresponds to displaying the and-inequality as an or-inequality AND flipping the inequality.
\item \( (a, b], \text{ where } a \in [2.6, 7.6] \text{ and } b \in [8, 10] \)

* $(5.60, 8.00]$, which is the correct option.
\item \( [a, b), \text{ where } a \in [-0.4, 7.6] \text{ and } b \in [5, 10] \)

$[5.60, 8.00)$, which corresponds to flipping the inequality.
\item \( (-\infty, a) \cup [b, \infty), \text{ where } a \in [2.6, 6.6] \text{ and } b \in [8, 9] \)

$(-\infty, 5.60) \cup [8.00, \infty)$, which corresponds to displaying the and-inequality as an or-inequality.
\item \( \text{None of the above.} \)


\end{enumerate}

\textbf{General Comment:} To solve, you will need to break up the compound inequality into two inequalities. Be sure to keep track of the inequality! It may be best to draw a number line and graph your solution.
}
\litem{
Using an interval or intervals, describe all the $x$-values within or including a distance of the given values.
\[ \text{ No more than } 6 \text{ units from the number } -1. \]

The solution is \( [-7, 5] \), which is option A.\begin{enumerate}[label=\Alph*.]
\item \( [-7, 5] \)

This describes the values no more than 6 from -1
\item \( (-\infty, -7) \cup (5, \infty) \)

This describes the values more than 6 from -1
\item \( (-\infty, -7] \cup [5, \infty) \)

This describes the values no less than 6 from -1
\item \( (-7, 5) \)

This describes the values less than 6 from -1
\item \( \text{None of the above} \)

You likely thought the values in the interval were not correct.
\end{enumerate}

\textbf{General Comment:} When thinking about this language, it helps to draw a number line and try points.
}
\litem{
Using an interval or intervals, describe all the $x$-values within or including a distance of the given values.
\[ \text{ No more than } 7 \text{ units from the number } -8. \]

The solution is \( [-15, -1] \), which is option A.\begin{enumerate}[label=\Alph*.]
\item \( [-15, -1] \)

This describes the values no more than 7 from -8
\item \( (-15, -1) \)

This describes the values less than 7 from -8
\item \( (-\infty, -15] \cup [-1, \infty) \)

This describes the values no less than 7 from -8
\item \( (-\infty, -15) \cup (-1, \infty) \)

This describes the values more than 7 from -8
\item \( \text{None of the above} \)

You likely thought the values in the interval were not correct.
\end{enumerate}

\textbf{General Comment:} When thinking about this language, it helps to draw a number line and try points.
}
\litem{
Solve the linear inequality below. Then, choose the constant and interval combination that describes the solution set.
\[ -8 + 3 x > 5 x \text{ or } -6 + 6 x < 9 x \]

The solution is \( (-\infty, -4.0) \text{ or } (-2.0, \infty) \), which is option C.\begin{enumerate}[label=\Alph*.]
\item \( (-\infty, a] \cup [b, \infty), \text{ where } a \in [-6, -2] \text{ and } b \in [-7, 0] \)

Corresponds to including the endpoints (when they should be excluded).
\item \( (-\infty, a) \cup (b, \infty), \text{ where } a \in [0, 7] \text{ and } b \in [1, 7] \)

Corresponds to inverting the inequality and negating the solution.
\item \( (-\infty, a) \cup (b, \infty), \text{ where } a \in [-4, -2] \text{ and } b \in [-5, 3] \)

 * Correct option.
\item \( (-\infty, a] \cup [b, \infty), \text{ where } a \in [-2, 3] \text{ and } b \in [3, 10] \)

Corresponds to including the endpoints AND negating.
\item \( (-\infty, \infty) \)

Corresponds to the variable canceling, which does not happen in this instance.
\end{enumerate}

\textbf{General Comment:} When multiplying or dividing by a negative, flip the sign.
}
\litem{
Solve the linear inequality below. Then, choose the constant and interval combination that describes the solution set.
\[ \frac{-4}{2} + \frac{5}{3} x \leq \frac{7}{6} x + \frac{4}{9} \]

The solution is \( (-\infty, 4.889] \), which is option D.\begin{enumerate}[label=\Alph*.]
\item \( (-\infty, a], \text{ where } a \in [-7.89, -0.89] \)

 $(-\infty, -4.889]$, which corresponds to negating the endpoint of the solution.
\item \( [a, \infty), \text{ where } a \in [-7.89, -3.89] \)

 $[-4.889, \infty)$, which corresponds to switching the direction of the interval AND negating the endpoint. You likely did this if you did not flip the inequality when dividing by a negative as well as not moving values over to a side properly.
\item \( [a, \infty), \text{ where } a \in [2.89, 5.89] \)

 $[4.889, \infty)$, which corresponds to switching the direction of the interval. You likely did this if you did not flip the inequality when dividing by a negative!
\item \( (-\infty, a], \text{ where } a \in [3.89, 5.89] \)

* $(-\infty, 4.889]$, which is the correct option.
\item \( \text{None of the above}. \)

You may have chosen this if you thought the inequality did not match the ends of the intervals.
\end{enumerate}

\textbf{General Comment:} Remember that less/greater than or equal to includes the endpoint, while less/greater do not. Also, remember that you need to flip the inequality when you multiply or divide by a negative.
}
\litem{
Solve the linear inequality below. Then, choose the constant and interval combination that describes the solution set.
\[ -4 + 8 x \leq \frac{68 x + 6}{8} < 8 + 6 x \]

The solution is \( \text{None of the above.} \), which is option E.\begin{enumerate}[label=\Alph*.]
\item \( (-\infty, a) \cup [b, \infty), \text{ where } a \in [7.5, 12.5] \text{ and } b \in [-4.9, 0.1] \)

$(-\infty, 9.50) \cup [-2.90, \infty)$, which corresponds to displaying the and-inequality as an or-inequality AND flipping the inequality AND getting negatives of the actual endpoints.
\item \( (-\infty, a] \cup (b, \infty), \text{ where } a \in [8.5, 10.5] \text{ and } b \in [-4.9, -1.9] \)

$(-\infty, 9.50] \cup (-2.90, \infty)$, which corresponds to displaying the and-inequality as an or-inequality and getting negatives of the actual endpoints.
\item \( [a, b), \text{ where } a \in [6.5, 10.5] \text{ and } b \in [-2.9, -1.9] \)

$[9.50, -2.90)$, which is the correct interval but negatives of the actual endpoints.
\item \( (a, b], \text{ where } a \in [9.5, 16.5] \text{ and } b \in [-3.9, 2.1] \)

$(9.50, -2.90]$, which corresponds to flipping the inequality and getting negatives of the actual endpoints.
\item \( \text{None of the above.} \)

* This is correct as the answer should be $[-9.50, 2.90)$.
\end{enumerate}

\textbf{General Comment:} To solve, you will need to break up the compound inequality into two inequalities. Be sure to keep track of the inequality! It may be best to draw a number line and graph your solution.
}
\litem{
Solve the linear inequality below. Then, choose the constant and interval combination that describes the solution set.
\[ \frac{-7}{2} - \frac{5}{3} x \leq \frac{3}{6} x + \frac{3}{4} \]

The solution is \( [-1.962, \infty) \), which is option C.\begin{enumerate}[label=\Alph*.]
\item \( (-\infty, a], \text{ where } a \in [1.96, 5.96] \)

 $(-\infty, 1.962]$, which corresponds to switching the direction of the interval AND negating the endpoint. You likely did this if you did not flip the inequality when dividing by a negative as well as not moving values over to a side properly.
\item \( [a, \infty), \text{ where } a \in [0.96, 4.96] \)

 $[1.962, \infty)$, which corresponds to negating the endpoint of the solution.
\item \( [a, \infty), \text{ where } a \in [-2.96, -0.96] \)

* $[-1.962, \infty)$, which is the correct option.
\item \( (-\infty, a], \text{ where } a \in [-2.96, -0.96] \)

 $(-\infty, -1.962]$, which corresponds to switching the direction of the interval. You likely did this if you did not flip the inequality when dividing by a negative!
\item \( \text{None of the above}. \)

You may have chosen this if you thought the inequality did not match the ends of the intervals.
\end{enumerate}

\textbf{General Comment:} Remember that less/greater than or equal to includes the endpoint, while less/greater do not. Also, remember that you need to flip the inequality when you multiply or divide by a negative.
}
\end{enumerate}

\end{document}