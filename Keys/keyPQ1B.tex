\documentclass{extbook}[14pt]
\usepackage{multicol, enumerate, enumitem, hyperref, color, soul, setspace, parskip, fancyhdr, amssymb, amsthm, amsmath, bbm, latexsym, units, mathtools}
\everymath{\displaystyle}
\usepackage[headsep=0.5cm,headheight=0cm, left=1 in,right= 1 in,top= 1 in,bottom= 1 in]{geometry}
\usepackage{dashrule}  % Package to use the command below to create lines between items
\newcommand{\litem}[1]{\item#1\hspace*{-1cm}\rule{\textwidth}{0.4pt}}
\pagestyle{fancy}
\lhead{}
\chead{Answer Key for Progress Quiz 1 Version B}
\rhead{}
\lfoot{3882-3329}
\cfoot{}
\rfoot{Fall 2020}
\begin{document}
\textbf{This key should allow you to understand why you choose the option you did (beyond just getting a question right or wrong). \href{https://xronos.clas.ufl.edu/mac1105spring2020/courseDescriptionAndMisc/Exams/LearningFromResults}{More instructions on how to use this key can be found here}.}

\textbf{If you have a suggestion to make the keys better, \href{https://forms.gle/CZkbZmPbC9XALEE88}{please fill out the short survey here}.}

\textit{Note: This key is auto-generated and may contain issues and/or errors. The keys are reviewed after each exam to ensure grading is done accurately. If there are issues (like duplicate options), they are noted in the offline gradebook. The keys are a work-in-progress to give students as many resources to improve as possible.}

\rule{\textwidth}{0.4pt}

\begin{enumerate}\litem{



\textbf{General Comment:} None
}
\litem{



\textbf{General Comment:} None
}
\litem{



\textbf{General Comment:} None
}
\litem{
Simplify the expression below into the form $a+bi$. Then, choose the intervals that $a$ and $b$ belong to.
\[ \frac{-45 - 11 i}{-4 - 6 i} \]
The solution is \( [4.730769230769231, -4.346153846153846] \), which is option C.\begin{enumerate}[label=\Alph*.]
\item \( a \in [1.5, 3.5] \text{ and } b \in [4.5, 6.5] \) $2.19  + 6.04 i$, which corresponds to forgetting to multiply the conjugate by the numerator and not computing the conjugate correctly.
\item \( a \in [245, 247] \text{ and } b \in [-4.5, -4] \) $246.00  - 4.35 i$, which corresponds to forgetting to multiply the conjugate by the numerator and using a plus instead of a minus in the denominator.
\item \( a \in [3.5, 5] \text{ and } b \in [-4.5, -4] \)* $4.73  - 4.35 i$, which is the correct option.
\item \( a \in [10, 11.5] \text{ and } b \in [1, 3] \) $11.25  + 1.83 i$, which corresponds to just dividing the first term by the first term and the second by the second.
\item \( a \in [3.5, 5] \text{ and } b \in [-226.5, -225.5] \) $4.73  - 226.00 i$, which corresponds to forgetting to multiply the conjugate by the numerator.
\end{enumerate}

\textbf{General Comment:} Multiply the numerator and denominator by the *conjugate* of the denominator, then simplify. For example, if we have $2+3i$, the conjugate is $2-3i$.
}
\litem{
Simplify the expression below into the form $a+bi$. Then, choose the intervals that $a$ and $b$ belong to.
\[ (10 - 6 i)(3 + 4 i) \]
The solution is \( 54 + 22 i \), which is option E.\begin{enumerate}[label=\Alph*.]
\item \( a \in [-1, 7] \text{ and } b \in [-61.2, -57.5] \) $6 - 58 i$, which corresponds to adding a minus sign in the second term.
\item \( a \in [30, 32] \text{ and } b \in [-24.2, -22.6] \) $30 - 24 i$, which corresponds to just multiplying the real terms to get the real part of the solution and the coefficients in the complex terms to get the complex part.
\item \( a \in [52, 59] \text{ and } b \in [-23.2, -20.1] \) $54 - 22 i$, which corresponds to adding a minus sign in both terms.
\item \( a \in [-1, 7] \text{ and } b \in [55.7, 60.7] \) $6 + 58 i$, which corresponds to adding a minus sign in the first term.
\item \( a \in [52, 59] \text{ and } b \in [19.1, 22.2] \)* $54 + 22 i$, which is the correct option.
\end{enumerate}

\textbf{General Comment:} You can treat $i$ as a variable and distribute. Just remember that $i^2=-1$, so you can continue to reduce after you distribute.
}
\end{enumerate}

\end{document}