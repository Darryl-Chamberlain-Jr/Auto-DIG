\documentclass{extbook}[14pt]
\usepackage{multicol, enumerate, enumitem, hyperref, color, soul, setspace, parskip, fancyhdr, amssymb, amsthm, amsmath, bbm, latexsym, units, mathtools}
\everymath{\displaystyle}
\usepackage[headsep=0.5cm,headheight=0cm, left=1 in,right= 1 in,top= 1 in,bottom= 1 in]{geometry}
\usepackage{dashrule}  % Package to use the command below to create lines between items
\newcommand{\litem}[1]{\item #1

\rule{\textwidth}{0.4pt}}
\pagestyle{fancy}
\lhead{}
\chead{Answer Key for Progress Quiz 8 Version C}
\rhead{}
\lfoot{4553-3922}
\cfoot{}
\rfoot{Fall 2020}
\begin{document}
\textbf{This key should allow you to understand why you choose the option you did (beyond just getting a question right or wrong). \href{https://xronos.clas.ufl.edu/mac1105spring2020/courseDescriptionAndMisc/Exams/LearningFromResults}{More instructions on how to use this key can be found here}.}

\textbf{If you have a suggestion to make the keys better, \href{https://forms.gle/CZkbZmPbC9XALEE88}{please fill out the short survey here}.}

\textit{Note: This key is auto-generated and may contain issues and/or errors. The keys are reviewed after each exam to ensure grading is done accurately. If there are issues (like duplicate options), they are noted in the offline gradebook. The keys are a work-in-progress to give students as many resources to improve as possible.}

\rule{\textwidth}{0.4pt}

\begin{enumerate}\litem{
Solve the linear inequality below. Then, choose the constant and interval combination that describes the solution set.
\[ -10x -3 < 8x + 5 \]

The solution is \( (-0.444, \infty) \), which is option B.\begin{enumerate}[label=\Alph*.]
\item \( (a, \infty), \text{ where } a \in [-0.12, 1.06] \)

 $(0.444, \infty)$, which corresponds to negating the endpoint of the solution.
\item \( (a, \infty), \text{ where } a \in [-0.47, 0.28] \)

* $(-0.444, \infty)$, which is the correct option.
\item \( (-\infty, a), \text{ where } a \in [0.16, 1.2] \)

 $(-\infty, 0.444)$, which corresponds to switching the direction of the interval AND negating the endpoint. You likely did this if you did not flip the inequality when dividing by a negative as well as not moving values over to a side properly.
\item \( (-\infty, a), \text{ where } a \in [-0.5, -0.11] \)

 $(-\infty, -0.444)$, which corresponds to switching the direction of the interval. You likely did this if you did not flip the inequality when dividing by a negative!
\item \( \text{None of the above}. \)

You may have chosen this if you thought the inequality did not match the ends of the intervals.
\end{enumerate}

\textbf{General Comment:} Remember that less/greater than or equal to includes the endpoint, while less/greater do not. Also, remember that you need to flip the inequality when you multiply or divide by a negative.
}
\litem{
Using an interval or intervals, describe all the $x$-values within or including a distance of the given values.
\[ \text{ No more than } 2 \text{ units from the number } -7. \]

The solution is \( [-9, -5] \), which is option D.\begin{enumerate}[label=\Alph*.]
\item \( (-9, -5) \)

This describes the values less than 2 from -7
\item \( (-\infty, -9) \cup (-5, \infty) \)

This describes the values more than 2 from -7
\item \( (-\infty, -9] \cup [-5, \infty) \)

This describes the values no less than 2 from -7
\item \( [-9, -5] \)

This describes the values no more than 2 from -7
\item \( \text{None of the above} \)

You likely thought the values in the interval were not correct.
\end{enumerate}

\textbf{General Comment:} When thinking about this language, it helps to draw a number line and try points.
}
\litem{
Solve the linear inequality below. Then, choose the constant and interval combination that describes the solution set.
\[ -6 + 3 x < \frac{38 x + 3}{9} \leq 4 + 4 x \]

The solution is \( \text{None of the above.} \), which is option E.\begin{enumerate}[label=\Alph*.]
\item \( [a, b), \text{ where } a \in [3.18, 6.18] \text{ and } b \in [-19.5, -15.5] \)

$[5.18, -16.50)$, which corresponds to flipping the inequality and getting negatives of the actual endpoints.
\item \( (-\infty, a) \cup [b, \infty), \text{ where } a \in [4.18, 8.18] \text{ and } b \in [-16.5, -12.5] \)

$(-\infty, 5.18) \cup [-16.50, \infty)$, which corresponds to displaying the and-inequality as an or-inequality and getting negatives of the actual endpoints.
\item \( (a, b], \text{ where } a \in [3.18, 7.18] \text{ and } b \in [-16.5, -8.5] \)

$(5.18, -16.50]$, which is the correct interval but negatives of the actual endpoints.
\item \( (-\infty, a] \cup (b, \infty), \text{ where } a \in [5.18, 11.18] \text{ and } b \in [-18.5, -14.5] \)

$(-\infty, 5.18] \cup (-16.50, \infty)$, which corresponds to displaying the and-inequality as an or-inequality AND flipping the inequality AND getting negatives of the actual endpoints.
\item \( \text{None of the above.} \)

* This is correct as the answer should be $(-5.18, 16.50]$.
\end{enumerate}

\textbf{General Comment:} To solve, you will need to break up the compound inequality into two inequalities. Be sure to keep track of the inequality! It may be best to draw a number line and graph your solution.
}
\litem{
Solve the linear inequality below. Then, choose the constant and interval combination that describes the solution set.
\[ \frac{-10}{2} - \frac{5}{8} x \geq \frac{10}{4} x - \frac{9}{6} \]

The solution is \( (-\infty, -1.12] \), which is option C.\begin{enumerate}[label=\Alph*.]
\item \( (-\infty, a], \text{ where } a \in [1.12, 3.12] \)

 $(-\infty, 1.12]$, which corresponds to negating the endpoint of the solution.
\item \( [a, \infty), \text{ where } a \in [-1.3, -0.4] \)

 $[-1.12, \infty)$, which corresponds to switching the direction of the interval. You likely did this if you did not flip the inequality when dividing by a negative!
\item \( (-\infty, a], \text{ where } a \in [-3.12, -0.12] \)

* $(-\infty, -1.12]$, which is the correct option.
\item \( [a, \infty), \text{ where } a \in [0.6, 1.9] \)

 $[1.12, \infty)$, which corresponds to switching the direction of the interval AND negating the endpoint. You likely did this if you did not flip the inequality when dividing by a negative as well as not moving values over to a side properly.
\item \( \text{None of the above}. \)

You may have chosen this if you thought the inequality did not match the ends of the intervals.
\end{enumerate}

\textbf{General Comment:} Remember that less/greater than or equal to includes the endpoint, while less/greater do not. Also, remember that you need to flip the inequality when you multiply or divide by a negative.
}
\litem{
Solve the linear inequality below. Then, choose the constant and interval combination that describes the solution set.
\[ -7 + 7 x > 8 x \text{ or } 4 + 9 x < 10 x \]

The solution is \( (-\infty, -7.0) \text{ or } (4.0, \infty) \), which is option D.\begin{enumerate}[label=\Alph*.]
\item \( (-\infty, a] \cup [b, \infty), \text{ where } a \in [-4, -1] \text{ and } b \in [6, 10] \)

Corresponds to including the endpoints AND negating.
\item \( (-\infty, a) \cup (b, \infty), \text{ where } a \in [-6, 0] \text{ and } b \in [4.4, 9.1] \)

Corresponds to inverting the inequality and negating the solution.
\item \( (-\infty, a] \cup [b, \infty), \text{ where } a \in [-10, -5] \text{ and } b \in [4, 5] \)

Corresponds to including the endpoints (when they should be excluded).
\item \( (-\infty, a) \cup (b, \infty), \text{ where } a \in [-7, -6] \text{ and } b \in [2.5, 4.4] \)

 * Correct option.
\item \( (-\infty, \infty) \)

Corresponds to the variable canceling, which does not happen in this instance.
\end{enumerate}

\textbf{General Comment:} When multiplying or dividing by a negative, flip the sign.
}
\litem{
Solve the linear inequality below. Then, choose the constant and interval combination that describes the solution set.
\[ -9 + 9 x > 10 x \text{ or } 5 + 8 x < 10 x \]

The solution is \( (-\infty, -9.0) \text{ or } (2.5, \infty) \), which is option A.\begin{enumerate}[label=\Alph*.]
\item \( (-\infty, a) \cup (b, \infty), \text{ where } a \in [-15, -7] \text{ and } b \in [2.5, 3.5] \)

 * Correct option.
\item \( (-\infty, a] \cup [b, \infty), \text{ where } a \in [-9, -8] \text{ and } b \in [2.5, 3.5] \)

Corresponds to including the endpoints (when they should be excluded).
\item \( (-\infty, a] \cup [b, \infty), \text{ where } a \in [-4.5, 0.5] \text{ and } b \in [7, 10] \)

Corresponds to including the endpoints AND negating.
\item \( (-\infty, a) \cup (b, \infty), \text{ where } a \in [-3.5, 0.5] \text{ and } b \in [8, 14] \)

Corresponds to inverting the inequality and negating the solution.
\item \( (-\infty, \infty) \)

Corresponds to the variable canceling, which does not happen in this instance.
\end{enumerate}

\textbf{General Comment:} When multiplying or dividing by a negative, flip the sign.
}
\litem{
Solve the linear inequality below. Then, choose the constant and interval combination that describes the solution set.
\[ -6 + 6 x \leq \frac{28 x + 8}{4} < 5 + 6 x \]

The solution is \( [-8.00, 3.00) \), which is option C.\begin{enumerate}[label=\Alph*.]
\item \( (a, b], \text{ where } a \in [-8, -3] \text{ and } b \in [1, 7] \)

$(-8.00, 3.00]$, which corresponds to flipping the inequality.
\item \( (-\infty, a) \cup [b, \infty), \text{ where } a \in [-8, -6] \text{ and } b \in [2, 6] \)

$(-\infty, -8.00) \cup [3.00, \infty)$, which corresponds to displaying the and-inequality as an or-inequality AND flipping the inequality.
\item \( [a, b), \text{ where } a \in [-8, -7] \text{ and } b \in [-1, 7] \)

$[-8.00, 3.00)$, which is the correct option.
\item \( (-\infty, a] \cup (b, \infty), \text{ where } a \in [-11, -7] \text{ and } b \in [1, 4] \)

$(-\infty, -8.00] \cup (3.00, \infty)$, which corresponds to displaying the and-inequality as an or-inequality.
\item \( \text{None of the above.} \)


\end{enumerate}

\textbf{General Comment:} To solve, you will need to break up the compound inequality into two inequalities. Be sure to keep track of the inequality! It may be best to draw a number line and graph your solution.
}
\litem{
Using an interval or intervals, describe all the $x$-values within or including a distance of the given values.
\[ \text{ More than } 7 \text{ units from the number } 10. \]

The solution is \( (-\infty, 3) \cup (17, \infty) \), which is option A.\begin{enumerate}[label=\Alph*.]
\item \( (-\infty, 3) \cup (17, \infty) \)

This describes the values more than 7 from 10
\item \( (3, 17) \)

This describes the values less than 7 from 10
\item \( [3, 17] \)

This describes the values no more than 7 from 10
\item \( (-\infty, 3] \cup [17, \infty) \)

This describes the values no less than 7 from 10
\item \( \text{None of the above} \)

You likely thought the values in the interval were not correct.
\end{enumerate}

\textbf{General Comment:} When thinking about this language, it helps to draw a number line and try points.
}
\litem{
Solve the linear inequality below. Then, choose the constant and interval combination that describes the solution set.
\[ \frac{-8}{9} + \frac{4}{4} x \leq \frac{5}{8} x - \frac{10}{2} \]

The solution is \( (-\infty, -10.963] \), which is option B.\begin{enumerate}[label=\Alph*.]
\item \( [a, \infty), \text{ where } a \in [8.96, 13.96] \)

 $[10.963, \infty)$, which corresponds to switching the direction of the interval AND negating the endpoint. You likely did this if you did not flip the inequality when dividing by a negative as well as not moving values over to a side properly.
\item \( (-\infty, a], \text{ where } a \in [-13.96, -6.96] \)

* $(-\infty, -10.963]$, which is the correct option.
\item \( (-\infty, a], \text{ where } a \in [10.96, 11.96] \)

 $(-\infty, 10.963]$, which corresponds to negating the endpoint of the solution.
\item \( [a, \infty), \text{ where } a \in [-11.96, -8.96] \)

 $[-10.963, \infty)$, which corresponds to switching the direction of the interval. You likely did this if you did not flip the inequality when dividing by a negative!
\item \( \text{None of the above}. \)

You may have chosen this if you thought the inequality did not match the ends of the intervals.
\end{enumerate}

\textbf{General Comment:} Remember that less/greater than or equal to includes the endpoint, while less/greater do not. Also, remember that you need to flip the inequality when you multiply or divide by a negative.
}
\litem{
Solve the linear inequality below. Then, choose the constant and interval combination that describes the solution set.
\[ -3x -9 < 10x + 3 \]

The solution is \( (-0.923, \infty) \), which is option C.\begin{enumerate}[label=\Alph*.]
\item \( (a, \infty), \text{ where } a \in [-0.2, 3.1] \)

 $(0.923, \infty)$, which corresponds to negating the endpoint of the solution.
\item \( (-\infty, a), \text{ where } a \in [-2.5, 0] \)

 $(-\infty, -0.923)$, which corresponds to switching the direction of the interval. You likely did this if you did not flip the inequality when dividing by a negative!
\item \( (a, \infty), \text{ where } a \in [-2.3, 0.5] \)

* $(-0.923, \infty)$, which is the correct option.
\item \( (-\infty, a), \text{ where } a \in [0.8, 1.7] \)

 $(-\infty, 0.923)$, which corresponds to switching the direction of the interval AND negating the endpoint. You likely did this if you did not flip the inequality when dividing by a negative as well as not moving values over to a side properly.
\item \( \text{None of the above}. \)

You may have chosen this if you thought the inequality did not match the ends of the intervals.
\end{enumerate}

\textbf{General Comment:} Remember that less/greater than or equal to includes the endpoint, while less/greater do not. Also, remember that you need to flip the inequality when you multiply or divide by a negative.
}
\end{enumerate}

\end{document}