\documentclass{extbook}[14pt]
\usepackage{multicol, enumerate, enumitem, hyperref, color, soul, setspace, parskip, fancyhdr, amssymb, amsthm, amsmath, latexsym, units, mathtools}
\everymath{\displaystyle}
\usepackage[headsep=0.5cm,headheight=0cm, left=1 in,right= 1 in,top= 1 in,bottom= 1 in]{geometry}
\usepackage{dashrule}  % Package to use the command below to create lines between items
\newcommand{\litem}[1]{\item #1

\rule{\textwidth}{0.4pt}}
\pagestyle{fancy}
\lhead{}
\chead{Answer Key for Makeup Progress Quiz 3 Version C}
\rhead{}
\lfoot{1648-1753}
\cfoot{}
\rfoot{Summer C 2021}
\begin{document}
\textbf{This key should allow you to understand why you choose the option you did (beyond just getting a question right or wrong). \href{https://xronos.clas.ufl.edu/mac1105spring2020/courseDescriptionAndMisc/Exams/LearningFromResults}{More instructions on how to use this key can be found here}.}

\textbf{If you have a suggestion to make the keys better, \href{https://forms.gle/CZkbZmPbC9XALEE88}{please fill out the short survey here}.}

\textit{Note: This key is auto-generated and may contain issues and/or errors. The keys are reviewed after each exam to ensure grading is done accurately. If there are issues (like duplicate options), they are noted in the offline gradebook. The keys are a work-in-progress to give students as many resources to improve as possible.}

\rule{\textwidth}{0.4pt}

\begin{enumerate}\litem{
Using an interval or intervals, describe all the $x$-values within or including a distance of the given values.
\[ \text{ Less than } 10 \text{ units from the number } -3. \]The solution is \( (-13, 7) \), which is option A.\begin{enumerate}[label=\Alph*.]
\item \( (-13, 7) \)

This describes the values less than 10 from -3
\item \( [-13, 7] \)

This describes the values no more than 10 from -3
\item \( (-\infty, -13) \cup (7, \infty) \)

This describes the values more than 10 from -3
\item \( (-\infty, -13] \cup [7, \infty) \)

This describes the values no less than 10 from -3
\item \( \text{None of the above} \)

You likely thought the values in the interval were not correct.
\end{enumerate}

\textbf{General Comment:} When thinking about this language, it helps to draw a number line and try points.
}
\litem{
Solve the linear inequality below. Then, choose the constant and interval combination that describes the solution set.
\[ -10x + 3 > -5x -6 \]The solution is \( (-\infty, 1.8) \), which is option A.\begin{enumerate}[label=\Alph*.]
\item \( (-\infty, a), \text{ where } a \in [0.8, 6.8] \)

* $(-\infty, 1.8)$, which is the correct option.
\item \( (a, \infty), \text{ where } a \in [-2.8, -0.8] \)

 $(-1.8, \infty)$, which corresponds to switching the direction of the interval AND negating the endpoint. You likely did this if you did not flip the inequality when dividing by a negative as well as not moving values over to a side properly.
\item \( (-\infty, a), \text{ where } a \in [-2.8, 0.2] \)

 $(-\infty, -1.8)$, which corresponds to negating the endpoint of the solution.
\item \( (a, \infty), \text{ where } a \in [1.8, 3.8] \)

 $(1.8, \infty)$, which corresponds to switching the direction of the interval. You likely did this if you did not flip the inequality when dividing by a negative!
\item \( \text{None of the above}. \)

You may have chosen this if you thought the inequality did not match the ends of the intervals.
\end{enumerate}

\textbf{General Comment:} Remember that less/greater than or equal to includes the endpoint, while less/greater do not. Also, remember that you need to flip the inequality when you multiply or divide by a negative.
}
\litem{
Solve the linear inequality below. Then, choose the constant and interval combination that describes the solution set.
\[ 8 + 5 x \leq \frac{77 x - 4}{9} < 7 + 8 x \]The solution is \( \text{None of the above.} \), which is option E.\begin{enumerate}[label=\Alph*.]
\item \( [a, b), \text{ where } a \in [-3, 1.5] \text{ and } b \in [-17.25, -6.75] \)

$[-2.38, -13.40)$, which is the correct interval but negatives of the actual endpoints.
\item \( (-\infty, a) \cup [b, \infty), \text{ where } a \in [-3.75, 0] \text{ and } b \in [-15.75, -12.75] \)

$(-\infty, -2.38) \cup [-13.40, \infty)$, which corresponds to displaying the and-inequality as an or-inequality AND flipping the inequality AND getting negatives of the actual endpoints.
\item \( (-\infty, a] \cup (b, \infty), \text{ where } a \in [-5.25, -0.75] \text{ and } b \in [-15.75, -8.25] \)

$(-\infty, -2.38] \cup (-13.40, \infty)$, which corresponds to displaying the and-inequality as an or-inequality and getting negatives of the actual endpoints.
\item \( (a, b], \text{ where } a \in [-4.5, 0.75] \text{ and } b \in [-16.5, -11.25] \)

$(-2.38, -13.40]$, which corresponds to flipping the inequality and getting negatives of the actual endpoints.
\item \( \text{None of the above.} \)

* This is correct as the answer should be $[2.38, 13.40)$.
\end{enumerate}

\textbf{General Comment:} To solve, you will need to break up the compound inequality into two inequalities. Be sure to keep track of the inequality! It may be best to draw a number line and graph your solution.
}
\litem{
Solve the linear inequality below. Then, choose the constant and interval combination that describes the solution set.
\[ -9 + 3 x > 4 x \text{ or } 8 + 4 x < 7 x \]The solution is \( (-\infty, -9.0) \text{ or } (2.667, \infty) \), which is option D.\begin{enumerate}[label=\Alph*.]
\item \( (-\infty, a] \cup [b, \infty), \text{ where } a \in [-3, -0.75] \text{ and } b \in [3.75, 14.25] \)

Corresponds to including the endpoints AND negating.
\item \( (-\infty, a) \cup (b, \infty), \text{ where } a \in [-7.5, 1.5] \text{ and } b \in [8.25, 12] \)

Corresponds to inverting the inequality and negating the solution.
\item \( (-\infty, a] \cup [b, \infty), \text{ where } a \in [-9.75, -8.25] \text{ and } b \in [-0.75, 6] \)

Corresponds to including the endpoints (when they should be excluded).
\item \( (-\infty, a) \cup (b, \infty), \text{ where } a \in [-11.25, -5.25] \text{ and } b \in [0, 5.25] \)

 * Correct option.
\item \( (-\infty, \infty) \)

Corresponds to the variable canceling, which does not happen in this instance.
\end{enumerate}

\textbf{General Comment:} When multiplying or dividing by a negative, flip the sign.
}
\litem{
Solve the linear inequality below. Then, choose the constant and interval combination that describes the solution set.
\[ \frac{-5}{5} + \frac{3}{4} x < \frac{6}{6} x + \frac{10}{9} \]The solution is \( (-8.444, \infty) \), which is option A.\begin{enumerate}[label=\Alph*.]
\item \( (a, \infty), \text{ where } a \in [-10.5, -4.5] \)

* $(-8.444, \infty)$, which is the correct option.
\item \( (-\infty, a), \text{ where } a \in [-10.5, -4.5] \)

 $(-\infty, -8.444)$, which corresponds to switching the direction of the interval. You likely did this if you did not flip the inequality when dividing by a negative!
\item \( (-\infty, a), \text{ where } a \in [5.25, 12] \)

 $(-\infty, 8.444)$, which corresponds to switching the direction of the interval AND negating the endpoint. You likely did this if you did not flip the inequality when dividing by a negative as well as not moving values over to a side properly.
\item \( (a, \infty), \text{ where } a \in [6, 11.25] \)

 $(8.444, \infty)$, which corresponds to negating the endpoint of the solution.
\item \( \text{None of the above}. \)

You may have chosen this if you thought the inequality did not match the ends of the intervals.
\end{enumerate}

\textbf{General Comment:} Remember that less/greater than or equal to includes the endpoint, while less/greater do not. Also, remember that you need to flip the inequality when you multiply or divide by a negative.
}
\litem{
Solve the linear inequality below. Then, choose the constant and interval combination that describes the solution set.
\[ \frac{7}{2} + \frac{6}{6} x > \frac{10}{3} x - \frac{3}{9} \]The solution is \( (-\infty, 1.643) \), which is option B.\begin{enumerate}[label=\Alph*.]
\item \( (a, \infty), \text{ where } a \in [-0.75, 3.75] \)

 $(1.643, \infty)$, which corresponds to switching the direction of the interval. You likely did this if you did not flip the inequality when dividing by a negative!
\item \( (-\infty, a), \text{ where } a \in [-1.5, 7.5] \)

* $(-\infty, 1.643)$, which is the correct option.
\item \( (-\infty, a), \text{ where } a \in [-3.75, 0] \)

 $(-\infty, -1.643)$, which corresponds to negating the endpoint of the solution.
\item \( (a, \infty), \text{ where } a \in [-3.75, 1.5] \)

 $(-1.643, \infty)$, which corresponds to switching the direction of the interval AND negating the endpoint. You likely did this if you did not flip the inequality when dividing by a negative as well as not moving values over to a side properly.
\item \( \text{None of the above}. \)

You may have chosen this if you thought the inequality did not match the ends of the intervals.
\end{enumerate}

\textbf{General Comment:} Remember that less/greater than or equal to includes the endpoint, while less/greater do not. Also, remember that you need to flip the inequality when you multiply or divide by a negative.
}
\litem{
Solve the linear inequality below. Then, choose the constant and interval combination that describes the solution set.
\[ -9 + 6 x > 9 x \text{ or } 9 + 9 x < 11 x \]The solution is \( (-\infty, -3.0) \text{ or } (4.5, \infty) \), which is option B.\begin{enumerate}[label=\Alph*.]
\item \( (-\infty, a) \cup (b, \infty), \text{ where } a \in [-6.15, -4.35] \text{ and } b \in [1.65, 3.97] \)

Corresponds to inverting the inequality and negating the solution.
\item \( (-\infty, a) \cup (b, \infty), \text{ where } a \in [-3.82, -2.92] \text{ and } b \in [4.35, 6.3] \)

 * Correct option.
\item \( (-\infty, a] \cup [b, \infty), \text{ where } a \in [-4.35, -2.77] \text{ and } b \in [3.3, 6] \)

Corresponds to including the endpoints (when they should be excluded).
\item \( (-\infty, a] \cup [b, \infty), \text{ where } a \in [-4.72, -3.3] \text{ and } b \in [0.97, 3.3] \)

Corresponds to including the endpoints AND negating.
\item \( (-\infty, \infty) \)

Corresponds to the variable canceling, which does not happen in this instance.
\end{enumerate}

\textbf{General Comment:} When multiplying or dividing by a negative, flip the sign.
}
\litem{
Using an interval or intervals, describe all the $x$-values within or including a distance of the given values.
\[ \text{ No less than } 10 \text{ units from the number } 7. \]The solution is \( (-\infty, -3] \cup [17, \infty) \), which is option C.\begin{enumerate}[label=\Alph*.]
\item \( [-3, 17] \)

This describes the values no more than 10 from 7
\item \( (-\infty, -3) \cup (17, \infty) \)

This describes the values more than 10 from 7
\item \( (-\infty, -3] \cup [17, \infty) \)

This describes the values no less than 10 from 7
\item \( (-3, 17) \)

This describes the values less than 10 from 7
\item \( \text{None of the above} \)

You likely thought the values in the interval were not correct.
\end{enumerate}

\textbf{General Comment:} When thinking about this language, it helps to draw a number line and try points.
}
\litem{
Solve the linear inequality below. Then, choose the constant and interval combination that describes the solution set.
\[ -5 - 7 x \leq \frac{-37 x + 5}{6} < 4 - 8 x \]The solution is \( [-7.00, 1.73) \), which is option B.\begin{enumerate}[label=\Alph*.]
\item \( (-\infty, a) \cup [b, \infty), \text{ where } a \in [-8.25, -3.75] \text{ and } b \in [0.75, 5.25] \)

$(-\infty, -7.00) \cup [1.73, \infty)$, which corresponds to displaying the and-inequality as an or-inequality AND flipping the inequality.
\item \( [a, b), \text{ where } a \in [-9, -5.25] \text{ and } b \in [0.75, 4.5] \)

$[-7.00, 1.73)$, which is the correct option.
\item \( (-\infty, a] \cup (b, \infty), \text{ where } a \in [-10.5, -3.75] \text{ and } b \in [-0.75, 9.75] \)

$(-\infty, -7.00] \cup (1.73, \infty)$, which corresponds to displaying the and-inequality as an or-inequality.
\item \( (a, b], \text{ where } a \in [-9, -2.25] \text{ and } b \in [-1.5, 2.25] \)

$(-7.00, 1.73]$, which corresponds to flipping the inequality.
\item \( \text{None of the above.} \)


\end{enumerate}

\textbf{General Comment:} To solve, you will need to break up the compound inequality into two inequalities. Be sure to keep track of the inequality! It may be best to draw a number line and graph your solution.
}
\litem{
Solve the linear inequality below. Then, choose the constant and interval combination that describes the solution set.
\[ -7x + 10 \leq -6x + 4 \]The solution is \( [6.0, \infty) \), which is option A.\begin{enumerate}[label=\Alph*.]
\item \( [a, \infty), \text{ where } a \in [3, 7] \)

* $[6.0, \infty)$, which is the correct option.
\item \( [a, \infty), \text{ where } a \in [-9, -4] \)

 $[-6.0, \infty)$, which corresponds to negating the endpoint of the solution.
\item \( (-\infty, a], \text{ where } a \in [-8, -4] \)

 $(-\infty, -6.0]$, which corresponds to switching the direction of the interval AND negating the endpoint. You likely did this if you did not flip the inequality when dividing by a negative as well as not moving values over to a side properly.
\item \( (-\infty, a], \text{ where } a \in [2, 12] \)

 $(-\infty, 6.0]$, which corresponds to switching the direction of the interval. You likely did this if you did not flip the inequality when dividing by a negative!
\item \( \text{None of the above}. \)

You may have chosen this if you thought the inequality did not match the ends of the intervals.
\end{enumerate}

\textbf{General Comment:} Remember that less/greater than or equal to includes the endpoint, while less/greater do not. Also, remember that you need to flip the inequality when you multiply or divide by a negative.
}
\end{enumerate}

\end{document}