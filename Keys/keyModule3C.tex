\documentclass{extbook}[14pt]
\usepackage{multicol, enumerate, enumitem, hyperref, color, soul, setspace, parskip, fancyhdr, amssymb, amsthm, amsmath, latexsym, units, mathtools}
\everymath{\displaystyle}
\usepackage[headsep=0.5cm,headheight=0cm, left=1 in,right= 1 in,top= 1 in,bottom= 1 in]{geometry}
\usepackage{dashrule}  % Package to use the command below to create lines between items
\newcommand{\litem}[1]{\item #1

\rule{\textwidth}{0.4pt}}
\pagestyle{fancy}
\lhead{}
\chead{Answer Key for Progress Quiz 6 Version C}
\rhead{}
\lfoot{9689-6866}
\cfoot{}
\rfoot{Spring 2021}
\begin{document}
\textbf{This key should allow you to understand why you choose the option you did (beyond just getting a question right or wrong). \href{https://xronos.clas.ufl.edu/mac1105spring2020/courseDescriptionAndMisc/Exams/LearningFromResults}{More instructions on how to use this key can be found here}.}

\textbf{If you have a suggestion to make the keys better, \href{https://forms.gle/CZkbZmPbC9XALEE88}{please fill out the short survey here}.}

\textit{Note: This key is auto-generated and may contain issues and/or errors. The keys are reviewed after each exam to ensure grading is done accurately. If there are issues (like duplicate options), they are noted in the offline gradebook. The keys are a work-in-progress to give students as many resources to improve as possible.}

\rule{\textwidth}{0.4pt}

\begin{enumerate}\litem{
Solve the linear inequality below. Then, choose the constant and interval combination that describes the solution set.
\[ -10x -9 > -4x -8 \]The solution is \( (-\infty, -0.167) \), which is option A.\begin{enumerate}[label=\Alph*.]
\item \( (-\infty, a), \text{ where } a \in [-0.57, -0.06] \)

* $(-\infty, -0.167)$, which is the correct option.
\item \( (-\infty, a), \text{ where } a \in [0.11, 0.28] \)

 $(-\infty, 0.167)$, which corresponds to negating the endpoint of the solution.
\item \( (a, \infty), \text{ where } a \in [-0.31, -0.04] \)

 $(-0.167, \infty)$, which corresponds to switching the direction of the interval. You likely did this if you did not flip the inequality when dividing by a negative!
\item \( (a, \infty), \text{ where } a \in [0.03, 0.19] \)

 $(0.167, \infty)$, which corresponds to switching the direction of the interval AND negating the endpoint. You likely did this if you did not flip the inequality when dividing by a negative as well as not moving values over to a side properly.
\item \( \text{None of the above}. \)

You may have chosen this if you thought the inequality did not match the ends of the intervals.
\end{enumerate}

\textbf{General Comment:} Remember that less/greater than or equal to includes the endpoint, while less/greater do not. Also, remember that you need to flip the inequality when you multiply or divide by a negative.
}
\litem{
Solve the linear inequality below. Then, choose the constant and interval combination that describes the solution set.
\[ -3 + 5 x > 8 x \text{ or } 7 + 3 x < 6 x \]The solution is \( (-\infty, -1.0) \text{ or } (2.333, \infty) \), which is option A.\begin{enumerate}[label=\Alph*.]
\item \( (-\infty, a) \cup (b, \infty), \text{ where } a \in [-2, 4] \text{ and } b \in [1.6, 3.9] \)

 * Correct option.
\item \( (-\infty, a) \cup (b, \infty), \text{ where } a \in [-4.33, -1.33] \text{ and } b \in [-3.2, 1.3] \)

Corresponds to inverting the inequality and negating the solution.
\item \( (-\infty, a] \cup [b, \infty), \text{ where } a \in [-1.66, -0.62] \text{ and } b \in [1.33, 5.33] \)

Corresponds to including the endpoints (when they should be excluded).
\item \( (-\infty, a] \cup [b, \infty), \text{ where } a \in [-4.61, -1.9] \text{ and } b \in [-1, 2] \)

Corresponds to including the endpoints AND negating.
\item \( (-\infty, \infty) \)

Corresponds to the variable canceling, which does not happen in this instance.
\end{enumerate}

\textbf{General Comment:} When multiplying or dividing by a negative, flip the sign.
}
\litem{
Solve the linear inequality below. Then, choose the constant and interval combination that describes the solution set.
\[ -8 + 9 x > 11 x \text{ or } 6 + 9 x < 12 x \]The solution is \( (-\infty, -4.0) \text{ or } (2.0, \infty) \), which is option B.\begin{enumerate}[label=\Alph*.]
\item \( (-\infty, a] \cup [b, \infty), \text{ where } a \in [-3.2, -0.1] \text{ and } b \in [3.1, 6.8] \)

Corresponds to including the endpoints AND negating.
\item \( (-\infty, a) \cup (b, \infty), \text{ where } a \in [-5, -3] \text{ and } b \in [0.64, 2.89] \)

 * Correct option.
\item \( (-\infty, a] \cup [b, \infty), \text{ where } a \in [-5, -2.7] \text{ and } b \in [-0.7, 3.7] \)

Corresponds to including the endpoints (when they should be excluded).
\item \( (-\infty, a) \cup (b, \infty), \text{ where } a \in [-3, -1] \text{ and } b \in [3.85, 4.43] \)

Corresponds to inverting the inequality and negating the solution.
\item \( (-\infty, \infty) \)

Corresponds to the variable canceling, which does not happen in this instance.
\end{enumerate}

\textbf{General Comment:} When multiplying or dividing by a negative, flip the sign.
}
\litem{
Solve the linear inequality below. Then, choose the constant and interval combination that describes the solution set.
\[ \frac{-8}{3} + \frac{6}{2} x > \frac{8}{9} x + \frac{3}{6} \]The solution is \( (1.5, \infty) \), which is option A.\begin{enumerate}[label=\Alph*.]
\item \( (a, \infty), \text{ where } a \in [0.5, 4.5] \)

* $(1.5, \infty)$, which is the correct option.
\item \( (a, \infty), \text{ where } a \in [-1.5, -0.5] \)

 $(-1.5, \infty)$, which corresponds to negating the endpoint of the solution.
\item \( (-\infty, a), \text{ where } a \in [-0.5, 2.5] \)

 $(-\infty, 1.5)$, which corresponds to switching the direction of the interval. You likely did this if you did not flip the inequality when dividing by a negative!
\item \( (-\infty, a), \text{ where } a \in [-1.5, 0.5] \)

 $(-\infty, -1.5)$, which corresponds to switching the direction of the interval AND negating the endpoint. You likely did this if you did not flip the inequality when dividing by a negative as well as not moving values over to a side properly.
\item \( \text{None of the above}. \)

You may have chosen this if you thought the inequality did not match the ends of the intervals.
\end{enumerate}

\textbf{General Comment:} Remember that less/greater than or equal to includes the endpoint, while less/greater do not. Also, remember that you need to flip the inequality when you multiply or divide by a negative.
}
\litem{
Solve the linear inequality below. Then, choose the constant and interval combination that describes the solution set.
\[ -5x -8 < 3x + 4 \]The solution is \( (-1.5, \infty) \), which is option C.\begin{enumerate}[label=\Alph*.]
\item \( (-\infty, a), \text{ where } a \in [-4.5, 0.5] \)

 $(-\infty, -1.5)$, which corresponds to switching the direction of the interval. You likely did this if you did not flip the inequality when dividing by a negative!
\item \( (-\infty, a), \text{ where } a \in [-0.5, 2.5] \)

 $(-\infty, 1.5)$, which corresponds to switching the direction of the interval AND negating the endpoint. You likely did this if you did not flip the inequality when dividing by a negative as well as not moving values over to a side properly.
\item \( (a, \infty), \text{ where } a \in [-3.5, 0.5] \)

* $(-1.5, \infty)$, which is the correct option.
\item \( (a, \infty), \text{ where } a \in [1.5, 5.5] \)

 $(1.5, \infty)$, which corresponds to negating the endpoint of the solution.
\item \( \text{None of the above}. \)

You may have chosen this if you thought the inequality did not match the ends of the intervals.
\end{enumerate}

\textbf{General Comment:} Remember that less/greater than or equal to includes the endpoint, while less/greater do not. Also, remember that you need to flip the inequality when you multiply or divide by a negative.
}
\litem{
Using an interval or intervals, describe all the $x$-values within or including a distance of the given values.
\[ \text{ More than } 4 \text{ units from the number } 6. \]The solution is \( \text{None of the above} \), which is option E.\begin{enumerate}[label=\Alph*.]
\item \( (-2, 10) \)

This describes the values less than 6 from 4
\item \( (-\infty, -2) \cup (10, \infty) \)

This describes the values more than 6 from 4
\item \( [-2, 10] \)

This describes the values no more than 6 from 4
\item \( (-\infty, -2] \cup [10, \infty) \)

This describes the values no less than 6 from 4
\item \( \text{None of the above} \)

Options A-D described the values [more/less than] 6 units from 4, which is the reverse of what the question asked.
\end{enumerate}

\textbf{General Comment:} When thinking about this language, it helps to draw a number line and try points.
}
\litem{
Solve the linear inequality below. Then, choose the constant and interval combination that describes the solution set.
\[ -3 + 5 x < \frac{46 x - 9}{7} \leq 7 + 6 x \]The solution is \( (-1.09, 14.50] \), which is option B.\begin{enumerate}[label=\Alph*.]
\item \( [a, b), \text{ where } a \in [-5.09, -0.09] \text{ and } b \in [9.5, 17.5] \)

$[-1.09, 14.50)$, which corresponds to flipping the inequality.
\item \( (a, b], \text{ where } a \in [-4.09, 0.91] \text{ and } b \in [12.5, 15.5] \)

* $(-1.09, 14.50]$, which is the correct option.
\item \( (-\infty, a] \cup (b, \infty), \text{ where } a \in [-3.09, -0.09] \text{ and } b \in [12.5, 19.5] \)

$(-\infty, -1.09] \cup (14.50, \infty)$, which corresponds to displaying the and-inequality as an or-inequality AND flipping the inequality.
\item \( (-\infty, a) \cup [b, \infty), \text{ where } a \in [-1.2, 0.5] \text{ and } b \in [12.5, 20.5] \)

$(-\infty, -1.09) \cup [14.50, \infty)$, which corresponds to displaying the and-inequality as an or-inequality.
\item \( \text{None of the above.} \)


\end{enumerate}

\textbf{General Comment:} To solve, you will need to break up the compound inequality into two inequalities. Be sure to keep track of the inequality! It may be best to draw a number line and graph your solution.
}
\litem{
Solve the linear inequality below. Then, choose the constant and interval combination that describes the solution set.
\[ -8 + 4 x < \frac{16 x - 3}{3} \leq 4 + 3 x \]The solution is \( \text{None of the above.} \), which is option E.\begin{enumerate}[label=\Alph*.]
\item \( (-\infty, a] \cup (b, \infty), \text{ where } a \in [0.25, 7.25] \text{ and } b \in [-5.14, 0.86] \)

$(-\infty, 5.25] \cup (-2.14, \infty)$, which corresponds to displaying the and-inequality as an or-inequality AND flipping the inequality AND getting negatives of the actual endpoints.
\item \( (-\infty, a) \cup [b, \infty), \text{ where } a \in [3.25, 6.25] \text{ and } b \in [-5.14, 0.86] \)

$(-\infty, 5.25) \cup [-2.14, \infty)$, which corresponds to displaying the and-inequality as an or-inequality and getting negatives of the actual endpoints.
\item \( [a, b), \text{ where } a \in [5.25, 6.25] \text{ and } b \in [-6.14, -0.14] \)

$[5.25, -2.14)$, which corresponds to flipping the inequality and getting negatives of the actual endpoints.
\item \( (a, b], \text{ where } a \in [4.25, 11.25] \text{ and } b \in [-4.14, -0.14] \)

$(5.25, -2.14]$, which is the correct interval but negatives of the actual endpoints.
\item \( \text{None of the above.} \)

* This is correct as the answer should be $(-5.25, 2.14]$.
\end{enumerate}

\textbf{General Comment:} To solve, you will need to break up the compound inequality into two inequalities. Be sure to keep track of the inequality! It may be best to draw a number line and graph your solution.
}
\litem{
Solve the linear inequality below. Then, choose the constant and interval combination that describes the solution set.
\[ \frac{-5}{8} - \frac{8}{5} x < \frac{-4}{6} x + \frac{4}{7} \]The solution is \( (-1.282, \infty) \), which is option A.\begin{enumerate}[label=\Alph*.]
\item \( (a, \infty), \text{ where } a \in [-2.28, -0.28] \)

* $(-1.282, \infty)$, which is the correct option.
\item \( (-\infty, a), \text{ where } a \in [-2.28, 0.72] \)

 $(-\infty, -1.282)$, which corresponds to switching the direction of the interval. You likely did this if you did not flip the inequality when dividing by a negative!
\item \( (a, \infty), \text{ where } a \in [1.28, 4.28] \)

 $(1.282, \infty)$, which corresponds to negating the endpoint of the solution.
\item \( (-\infty, a), \text{ where } a \in [0.28, 3.28] \)

 $(-\infty, 1.282)$, which corresponds to switching the direction of the interval AND negating the endpoint. You likely did this if you did not flip the inequality when dividing by a negative as well as not moving values over to a side properly.
\item \( \text{None of the above}. \)

You may have chosen this if you thought the inequality did not match the ends of the intervals.
\end{enumerate}

\textbf{General Comment:} Remember that less/greater than or equal to includes the endpoint, while less/greater do not. Also, remember that you need to flip the inequality when you multiply or divide by a negative.
}
\litem{
Using an interval or intervals, describe all the $x$-values within or including a distance of the given values.
\[ \text{ Less than } 9 \text{ units from the number } 2. \]The solution is \( \text{None of the above} \), which is option E.\begin{enumerate}[label=\Alph*.]
\item \( [7, 11] \)

This describes the values no more than 2 from 9
\item \( (-\infty, 7) \cup (11, \infty) \)

This describes the values more than 2 from 9
\item \( (-\infty, 7] \cup [11, \infty) \)

This describes the values no less than 2 from 9
\item \( (7, 11) \)

This describes the values less than 2 from 9
\item \( \text{None of the above} \)

Options A-D described the values [more/less than] 2 units from 9, which is the reverse of what the question asked.
\end{enumerate}

\textbf{General Comment:} When thinking about this language, it helps to draw a number line and try points.
}
\end{enumerate}

\end{document}