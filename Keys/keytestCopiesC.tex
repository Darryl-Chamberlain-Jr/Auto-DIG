\documentclass{extbook}[14pt]
\usepackage{multicol, enumerate, enumitem, hyperref, color, soul, setspace, parskip, fancyhdr, amssymb, amsthm, amsmath, latexsym, units, mathtools}
\everymath{\displaystyle}
\usepackage[headsep=0.5cm,headheight=0cm, left=1 in,right= 1 in,top= 1 in,bottom= 1 in]{geometry}
\usepackage{dashrule}  % Package to use the command below to create lines between items
\newcommand{\litem}[1]{\item #1

\rule{\textwidth}{0.4pt}}
\pagestyle{fancy}
\lhead{}
\chead{Answer Key for test copies Version C}
\rhead{}
\lfoot{5370-9939}
\cfoot{}
\rfoot{test}
\begin{document}
\textbf{This key should allow you to understand why you choose the option you did (beyond just getting a question right or wrong). \href{https://xronos.clas.ufl.edu/mac1105spring2020/courseDescriptionAndMisc/Exams/LearningFromResults}{More instructions on how to use this key can be found here}.}

\textbf{If you have a suggestion to make the keys better, \href{https://forms.gle/CZkbZmPbC9XALEE88}{please fill out the short survey here}.}

\textit{Note: This key is auto-generated and may contain issues and/or errors. The keys are reviewed after each exam to ensure grading is done accurately. If there are issues (like duplicate options), they are noted in the offline gradebook. The keys are a work-in-progress to give students as many resources to improve as possible.}

\rule{\textwidth}{0.4pt}

\begin{enumerate}\litem{
Simplify the expression below into the form $a+bi$. Then, choose the intervals that $a$ and $b$ belong to.
\[ (5 + 4 i)(-3 - 10 i) \]The solution is \( 25 - 62 i \), which is option B.\begin{enumerate}[label=\Alph*.]
\item \( a \in [-57, -52] \text{ and } b \in [-38.8, -36.1] \)

 $-55 - 38 i$, which corresponds to adding a minus sign in the first term.
\item \( a \in [22, 28] \text{ and } b \in [-64.3, -59.9] \)

* $25 - 62 i$, which is the correct option.
\item \( a \in [22, 28] \text{ and } b \in [60.9, 64.4] \)

 $25 + 62 i$, which corresponds to adding a minus sign in both terms.
\item \( a \in [-21, -14] \text{ and } b \in [-42.4, -39.1] \)

 $-15 - 40 i$, which corresponds to just multiplying the real terms to get the real part of the solution and the coefficients in the complex terms to get the complex part.
\item \( a \in [-57, -52] \text{ and } b \in [36.8, 38.8] \)

 $-55 + 38 i$, which corresponds to adding a minus sign in the second term.
\end{enumerate}

\textbf{General Comment:} You can treat $i$ as a variable and distribute. Just remember that $i^2=-1$, so you can continue to reduce after you distribute.
}
\litem{
Simplify the expression below into the form $a+bi$. Then, choose the intervals that $a$ and $b$ belong to.
\[ \frac{72 - 77 i}{4 + 5 i} \]The solution is \( -2.37  - 16.29 i \), which is option D.\begin{enumerate}[label=\Alph*.]
\item \( a \in [16, 17] \text{ and } b \in [1, 2.5] \)

 $16.41  + 1.27 i$, which corresponds to forgetting to multiply the conjugate by the numerator and not computing the conjugate correctly.
\item \( a \in [17, 19] \text{ and } b \in [-16, -14.5] \)

 $18.00  - 15.40 i$, which corresponds to just dividing the first term by the first term and the second by the second.
\item \( a \in [-97.5, -96] \text{ and } b \in [-16.5, -16] \)

 $-97.00  - 16.29 i$, which corresponds to forgetting to multiply the conjugate by the numerator and using a plus instead of a minus in the denominator.
\item \( a \in [-4, -2] \text{ and } b \in [-16.5, -16] \)

* $-2.37  - 16.29 i$, which is the correct option.
\item \( a \in [-4, -2] \text{ and } b \in [-669, -667.5] \)

 $-2.37  - 668.00 i$, which corresponds to forgetting to multiply the conjugate by the numerator.
\end{enumerate}

\textbf{General Comment:} Multiply the numerator and denominator by the *conjugate* of the denominator, then simplify. For example, if we have $2+3i$, the conjugate is $2-3i$.
}
\litem{
Simplify the expression below into the form $a+bi$. Then, choose the intervals that $a$ and $b$ belong to.
\[ \frac{54 - 33 i}{-8 - 5 i} \]The solution is \( -3.00  + 6.00 i \), which is option B.\begin{enumerate}[label=\Alph*.]
\item \( a \in [-3.05, -2.97] \text{ and } b \in [533.85, 534.2] \)

 $-3.00  + 534.00 i$, which corresponds to forgetting to multiply the conjugate by the numerator.
\item \( a \in [-3.05, -2.97] \text{ and } b \in [5.75, 6.4] \)

* $-3.00  + 6.00 i$, which is the correct option.
\item \( a \in [-267.09, -266.92] \text{ and } b \in [5.75, 6.4] \)

 $-267.00  + 6.00 i$, which corresponds to forgetting to multiply the conjugate by the numerator and using a plus instead of a minus in the denominator.
\item \( a \in [-6.72, -6.69] \text{ and } b \in [-0.45, 0.2] \)

 $-6.71  - 0.07 i$, which corresponds to forgetting to multiply the conjugate by the numerator and not computing the conjugate correctly.
\item \( a \in [-6.8, -6.72] \text{ and } b \in [6.25, 7.05] \)

 $-6.75  + 6.60 i$, which corresponds to just dividing the first term by the first term and the second by the second.
\end{enumerate}

\textbf{General Comment:} Multiply the numerator and denominator by the *conjugate* of the denominator, then simplify. For example, if we have $2+3i$, the conjugate is $2-3i$.
}
\litem{
Simplify the expression below into the form $a+bi$. Then, choose the intervals that $a$ and $b$ belong to.
\[ (-8 - 2 i)(-6 - 5 i) \]The solution is \( 38 + 52 i \), which is option B.\begin{enumerate}[label=\Alph*.]
\item \( a \in [35, 40] \text{ and } b \in [-59, -49] \)

 $38 - 52 i$, which corresponds to adding a minus sign in both terms.
\item \( a \in [35, 40] \text{ and } b \in [51, 57] \)

* $38 + 52 i$, which is the correct option.
\item \( a \in [52, 63] \text{ and } b \in [28, 31] \)

 $58 + 28 i$, which corresponds to adding a minus sign in the first term.
\item \( a \in [45, 49] \text{ and } b \in [7, 12] \)

 $48 + 10 i$, which corresponds to just multiplying the real terms to get the real part of the solution and the coefficients in the complex terms to get the complex part.
\item \( a \in [52, 63] \text{ and } b \in [-28, -24] \)

 $58 - 28 i$, which corresponds to adding a minus sign in the second term.
\end{enumerate}

\textbf{General Comment:} You can treat $i$ as a variable and distribute. Just remember that $i^2=-1$, so you can continue to reduce after you distribute.
}
\end{enumerate}

\end{document}