\documentclass{extbook}[14pt]
\usepackage{multicol, enumerate, enumitem, hyperref, color, soul, setspace, parskip, fancyhdr, amssymb, amsthm, amsmath, bbm, latexsym, units, mathtools}
\everymath{\displaystyle}
\usepackage[headsep=0.5cm,headheight=0cm, left=1 in,right= 1 in,top= 1 in,bottom= 1 in]{geometry}
\usepackage{dashrule}  % Package to use the command below to create lines between items
\newcommand{\litem}[1]{\item #1

\rule{\textwidth}{0.4pt}}
\pagestyle{fancy}
\lhead{}
\chead{Answer Key for Progress Quiz 8 Version A}
\rhead{}
\lfoot{4553-3922}
\cfoot{}
\rfoot{Fall 2020}
\begin{document}
\textbf{This key should allow you to understand why you choose the option you did (beyond just getting a question right or wrong). \href{https://xronos.clas.ufl.edu/mac1105spring2020/courseDescriptionAndMisc/Exams/LearningFromResults}{More instructions on how to use this key can be found here}.}

\textbf{If you have a suggestion to make the keys better, \href{https://forms.gle/CZkbZmPbC9XALEE88}{please fill out the short survey here}.}

\textit{Note: This key is auto-generated and may contain issues and/or errors. The keys are reviewed after each exam to ensure grading is done accurately. If there are issues (like duplicate options), they are noted in the offline gradebook. The keys are a work-in-progress to give students as many resources to improve as possible.}

\rule{\textwidth}{0.4pt}

\begin{enumerate}\litem{
Simplify the expression below into the form $a+bi$. Then, choose the intervals that $a$ and $b$ belong to.
\[ \frac{-18 - 44 i}{-3 + 8 i} \]

The solution is \( -4.08  + 3.78 i \), which is option B.\begin{enumerate}[label=\Alph*.]
\item \( a \in [5.9, 6.8] \text{ and } b \in [-6, -4] \)

 $6.00  - 5.50 i$, which corresponds to just dividing the first term by the first term and the second by the second.
\item \( a \in [-4.2, -4] \text{ and } b \in [2.5, 4.5] \)

* $-4.08  + 3.78 i$, which is the correct option.
\item \( a \in [-4.2, -4] \text{ and } b \in [275.5, 277] \)

 $-4.08  + 276.00 i$, which corresponds to forgetting to multiply the conjugate by the numerator.
\item \( a \in [-298.35, -297.9] \text{ and } b \in [2.5, 4.5] \)

 $-298.00  + 3.78 i$, which corresponds to forgetting to multiply the conjugate by the numerator and using a plus instead of a minus in the denominator.
\item \( a \in [5, 5.7] \text{ and } b \in [-1, 0.5] \)

 $5.56  - 0.16 i$, which corresponds to forgetting to multiply the conjugate by the numerator and not computing the conjugate correctly.
\end{enumerate}

\textbf{General Comment:} Multiply the numerator and denominator by the *conjugate* of the denominator, then simplify. For example, if we have $2+3i$, the conjugate is $2-3i$.
}
\litem{
Simplify the expression below into the form $a+bi$. Then, choose the intervals that $a$ and $b$ belong to.
\[ \frac{18 + 11 i}{7 - 8 i} \]

The solution is \( 0.34  + 1.96 i \), which is option B.\begin{enumerate}[label=\Alph*.]
\item \( a \in [1.45, 2.15] \text{ and } b \in [-0.65, -0.2] \)

 $1.89  - 0.59 i$, which corresponds to forgetting to multiply the conjugate by the numerator and not computing the conjugate correctly.
\item \( a \in [0.2, 0.45] \text{ and } b \in [1.9, 2.15] \)

* $0.34  + 1.96 i$, which is the correct option.
\item \( a \in [0.2, 0.45] \text{ and } b \in [220.8, 221.2] \)

 $0.34  + 221.00 i$, which corresponds to forgetting to multiply the conjugate by the numerator.
\item \( a \in [2.4, 3.1] \text{ and } b \in [-1.55, -0.95] \)

 $2.57  - 1.38 i$, which corresponds to just dividing the first term by the first term and the second by the second.
\item \( a \in [37.35, 38.35] \text{ and } b \in [1.9, 2.15] \)

 $38.00  + 1.96 i$, which corresponds to forgetting to multiply the conjugate by the numerator and using a plus instead of a minus in the denominator.
\end{enumerate}

\textbf{General Comment:} Multiply the numerator and denominator by the *conjugate* of the denominator, then simplify. For example, if we have $2+3i$, the conjugate is $2-3i$.
}
\litem{
Simplify the expression below into the form $a+bi$. Then, choose the intervals that $a$ and $b$ belong to.
\[ (7 - 4 i)(-5 - 10 i) \]

The solution is \( -75 - 50 i \), which is option B.\begin{enumerate}[label=\Alph*.]
\item \( a \in [0, 13] \text{ and } b \in [85, 94] \)

 $5 + 90 i$, which corresponds to adding a minus sign in the second term.
\item \( a \in [-83, -74] \text{ and } b \in [-55, -49] \)

* $-75 - 50 i$, which is the correct option.
\item \( a \in [-40, -29] \text{ and } b \in [38, 43] \)

 $-35 + 40 i$, which corresponds to just multiplying the real terms to get the real part of the solution and the coefficients in the complex terms to get the complex part.
\item \( a \in [-83, -74] \text{ and } b \in [50, 57] \)

 $-75 + 50 i$, which corresponds to adding a minus sign in both terms.
\item \( a \in [0, 13] \text{ and } b \in [-93, -88] \)

 $5 - 90 i$, which corresponds to adding a minus sign in the first term.
\end{enumerate}

\textbf{General Comment:} You can treat $i$ as a variable and distribute. Just remember that $i^2=-1$, so you can continue to reduce after you distribute.
}
\litem{
Choose the \textbf{smallest} set of Real numbers that the number below belongs to.
\[ \sqrt{\frac{39204}{324}} \]

The solution is \( \text{Whole} \), which is option D.\begin{enumerate}[label=\Alph*.]
\item \( \text{Integer} \)

These are the negative and positive counting numbers (..., -3, -2, -1, 0, 1, 2, 3, ...)
\item \( \text{Rational} \)

These are numbers that can be written as fraction of Integers (e.g., -2/3)
\item \( \text{Not a Real number} \)

These are Nonreal Complex numbers \textbf{OR} things that are not numbers (e.g., dividing by 0).
\item \( \text{Whole} \)

* This is the correct option!
\item \( \text{Irrational} \)

These cannot be written as a fraction of Integers.
\end{enumerate}

\textbf{General Comment:} First, you \textbf{NEED} to simplify the expression. This question simplifies to $198$. 
 
 Be sure you look at the simplified fraction and not just the decimal expansion. Numbers such as 13, 17, and 19 provide \textbf{long but repeating/terminating decimal expansions!} 
 
 The only ways to *not* be a Real number are: dividing by 0 or taking the square root of a negative number. 
 
 Irrational numbers are more than just square root of 3: adding or subtracting values from square root of 3 is also irrational.
}
\litem{
Simplify the expression below and choose the interval the simplification is contained within.
\[ 9 - 7^2 + 12 \div 17 * 8 \div 13 \]

The solution is \( -39.566 \), which is option D.\begin{enumerate}[label=\Alph*.]
\item \( [58.41, 58.54] \)

 58.434, which corresponds to an Order of Operations error: multiplying by negative before squaring. For example: $(-3)^2 \neq -3^2$
\item \( [-40.59, -39.77] \)

 -39.993, which corresponds to an Order of Operations error: not reading left-to-right for multiplication/division.
\item \( [57.88, 58.26] \)

 58.007, which corresponds to two Order of Operations errors.
\item \( [-39.73, -39.06] \)

* -39.566, this is the correct option
\item \( \text{None of the above} \)

 You may have gotten this by making an unanticipated error. If you got a value that is not any of the others, please let the coordinator know so they can help you figure out what happened.
\end{enumerate}

\textbf{General Comment:} While you may remember (or were taught) PEMDAS is done in order, it is actually done as P/E/MD/AS. When we are at MD or AS, we read left to right.
}
\litem{
Simplify the expression below and choose the interval the simplification is contained within.
\[ 12 - 1^2 + 18 \div 3 * 5 \div 14 \]

The solution is \( 13.143 \), which is option B.\begin{enumerate}[label=\Alph*.]
\item \( [15.06, 15.19] \)

 15.143, which corresponds to an Order of Operations error: multiplying by negative before squaring. For example: $(-3)^2 \neq -3^2$
\item \( [13.11, 13.2] \)

* 13.143, this is the correct option
\item \( [13, 13.14] \)

 13.086, which corresponds to two Order of Operations errors.
\item \( [11.03, 11.11] \)

 11.086, which corresponds to an Order of Operations error: not reading left-to-right for multiplication/division.
\item \( \text{None of the above} \)

 You may have gotten this by making an unanticipated error. If you got a value that is not any of the others, please let the coordinator know so they can help you figure out what happened.
\end{enumerate}

\textbf{General Comment:} While you may remember (or were taught) PEMDAS is done in order, it is actually done as P/E/MD/AS. When we are at MD or AS, we read left to right.
}
\litem{
Choose the \textbf{smallest} set of Complex numbers that the number below belongs to.
\[ \sqrt{\frac{36}{0}}+\sqrt{182} i \]

The solution is \( \text{Not a Complex Number} \), which is option A.\begin{enumerate}[label=\Alph*.]
\item \( \text{Not a Complex Number} \)

* This is the correct option!
\item \( \text{Rational} \)

These are numbers that can be written as fraction of Integers (e.g., -2/3 + 5)
\item \( \text{Nonreal Complex} \)

This is a Complex number $(a+bi)$ that is not Real (has $i$ as part of the number).
\item \( \text{Irrational} \)

These cannot be written as a fraction of Integers. Remember: $\pi$ is not an Integer!
\item \( \text{Pure Imaginary} \)

This is a Complex number $(a+bi)$ that \textbf{only} has an imaginary part like $2i$.
\end{enumerate}

\textbf{General Comment:} Be sure to simplify $i^2 = -1$. This may remove the imaginary portion for your number. If you are having trouble, you may want to look at the \textit{Subgroups of the Real Numbers} section.
}
\litem{
Choose the \textbf{smallest} set of Complex numbers that the number below belongs to.
\[ \frac{-18}{10}+36i^2 \]

The solution is \( \text{Rational} \), which is option B.\begin{enumerate}[label=\Alph*.]
\item \( \text{Irrational} \)

These cannot be written as a fraction of Integers. Remember: $\pi$ is not an Integer!
\item \( \text{Rational} \)

* This is the correct option!
\item \( \text{Pure Imaginary} \)

This is a Complex number $(a+bi)$ that \textbf{only} has an imaginary part like $2i$.
\item \( \text{Nonreal Complex} \)

This is a Complex number $(a+bi)$ that is not Real (has $i$ as part of the number).
\item \( \text{Not a Complex Number} \)

This is not a number. The only non-Complex number we know is dividing by 0 as this is not a number!
\end{enumerate}

\textbf{General Comment:} Be sure to simplify $i^2 = -1$. This may remove the imaginary portion for your number. If you are having trouble, you may want to look at the \textit{Subgroups of the Real Numbers} section.
}
\litem{
Simplify the expression below into the form $a+bi$. Then, choose the intervals that $a$ and $b$ belong to.
\[ (3 + 6 i)(-4 - 7 i) \]

The solution is \( 30 - 45 i \), which is option E.\begin{enumerate}[label=\Alph*.]
\item \( a \in [26, 31] \text{ and } b \in [44.7, 45.2] \)

 $30 + 45 i$, which corresponds to adding a minus sign in both terms.
\item \( a \in [-60, -47] \text{ and } b \in [0.2, 3.5] \)

 $-54 + 3 i$, which corresponds to adding a minus sign in the first term.
\item \( a \in [-16, -3] \text{ and } b \in [-42.6, -39.8] \)

 $-12 - 42 i$, which corresponds to just multiplying the real terms to get the real part of the solution and the coefficients in the complex terms to get the complex part.
\item \( a \in [-60, -47] \text{ and } b \in [-4, -1.7] \)

 $-54 - 3 i$, which corresponds to adding a minus sign in the second term.
\item \( a \in [26, 31] \text{ and } b \in [-47.3, -44.9] \)

* $30 - 45 i$, which is the correct option.
\end{enumerate}

\textbf{General Comment:} You can treat $i$ as a variable and distribute. Just remember that $i^2=-1$, so you can continue to reduce after you distribute.
}
\litem{
Choose the \textbf{smallest} set of Real numbers that the number below belongs to.
\[ -\sqrt{\frac{-605}{11}} \]

The solution is \( \text{Not a Real number} \), which is option E.\begin{enumerate}[label=\Alph*.]
\item \( \text{Integer} \)

These are the negative and positive counting numbers (..., -3, -2, -1, 0, 1, 2, 3, ...)
\item \( \text{Rational} \)

These are numbers that can be written as fraction of Integers (e.g., -2/3)
\item \( \text{Whole} \)

These are the counting numbers with 0 (0, 1, 2, 3, ...)
\item \( \text{Irrational} \)

These cannot be written as a fraction of Integers.
\item \( \text{Not a Real number} \)

* This is the correct option!
\end{enumerate}

\textbf{General Comment:} First, you \textbf{NEED} to simplify the expression. This question simplifies to $-\sqrt{55} i$. 
 
 Be sure you look at the simplified fraction and not just the decimal expansion. Numbers such as 13, 17, and 19 provide \textbf{long but repeating/terminating decimal expansions!} 
 
 The only ways to *not* be a Real number are: dividing by 0 or taking the square root of a negative number. 
 
 Irrational numbers are more than just square root of 3: adding or subtracting values from square root of 3 is also irrational.
}
\end{enumerate}

\end{document}