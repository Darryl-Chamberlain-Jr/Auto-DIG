\documentclass{extbook}[14pt]
\usepackage{multicol, enumerate, enumitem, hyperref, color, soul, setspace, parskip, fancyhdr, amssymb, amsthm, amsmath, latexsym, units, mathtools}
\everymath{\displaystyle}
\usepackage[headsep=0.5cm,headheight=0cm, left=1 in,right= 1 in,top= 1 in,bottom= 1 in]{geometry}
\usepackage{dashrule}  % Package to use the command below to create lines between items
\newcommand{\litem}[1]{\item #1

\rule{\textwidth}{0.4pt}}
\pagestyle{fancy}
\lhead{}
\chead{Answer Key for Progress Quiz 6 Version A}
\rhead{}
\lfoot{9689-6866}
\cfoot{}
\rfoot{Spring 2021}
\begin{document}
\textbf{This key should allow you to understand why you choose the option you did (beyond just getting a question right or wrong). \href{https://xronos.clas.ufl.edu/mac1105spring2020/courseDescriptionAndMisc/Exams/LearningFromResults}{More instructions on how to use this key can be found here}.}

\textbf{If you have a suggestion to make the keys better, \href{https://forms.gle/CZkbZmPbC9XALEE88}{please fill out the short survey here}.}

\textit{Note: This key is auto-generated and may contain issues and/or errors. The keys are reviewed after each exam to ensure grading is done accurately. If there are issues (like duplicate options), they are noted in the offline gradebook. The keys are a work-in-progress to give students as many resources to improve as possible.}

\rule{\textwidth}{0.4pt}

\begin{enumerate}\litem{
Simplify the expression below into the form $a+bi$. Then, choose the intervals that $a$ and $b$ belong to.
\[ \frac{18 - 44 i}{1 + 6 i} \]The solution is \( -6.65  - 4.11 i \), which is option C.\begin{enumerate}[label=\Alph*.]
\item \( a \in [-246.5, -245] \text{ and } b \in [-5.5, -3.5] \)

 $-246.00  - 4.11 i$, which corresponds to forgetting to multiply the conjugate by the numerator and using a plus instead of a minus in the denominator.
\item \( a \in [7, 8] \text{ and } b \in [1, 2.5] \)

 $7.62  + 1.73 i$, which corresponds to forgetting to multiply the conjugate by the numerator and not computing the conjugate correctly.
\item \( a \in [-7.5, -6] \text{ and } b \in [-5.5, -3.5] \)

* $-6.65  - 4.11 i$, which is the correct option.
\item \( a \in [17.5, 19] \text{ and } b \in [-8, -6.5] \)

 $18.00  - 7.33 i$, which corresponds to just dividing the first term by the first term and the second by the second.
\item \( a \in [-7.5, -6] \text{ and } b \in [-153, -151.5] \)

 $-6.65  - 152.00 i$, which corresponds to forgetting to multiply the conjugate by the numerator.
\end{enumerate}

\textbf{General Comment:} Multiply the numerator and denominator by the *conjugate* of the denominator, then simplify. For example, if we have $2+3i$, the conjugate is $2-3i$.
}
\litem{
Simplify the expression below and choose the interval the simplification is contained within.
\[ 4 - 1^2 + 8 \div 12 * 13 \div 11 \]The solution is \( 3.788 \), which is option A.\begin{enumerate}[label=\Alph*.]
\item \( [3.15, 4.29] \)

* 3.788, this is the correct option
\item \( [4.48, 5.05] \)

 5.005, which corresponds to two Order of Operations errors.
\item \( [5.36, 7.19] \)

 5.788, which corresponds to an Order of Operations error: multiplying by negative before squaring. For example: $(-3)^2 \neq -3^2$
\item \( [2.71, 3.17] \)

 3.005, which corresponds to an Order of Operations error: not reading left-to-right for multiplication/division.
\item \( \text{None of the above} \)

 You may have gotten this by making an unanticipated error. If you got a value that is not any of the others, please let the coordinator know so they can help you figure out what happened.
\end{enumerate}

\textbf{General Comment:} While you may remember (or were taught) PEMDAS is done in order, it is actually done as P/E/MD/AS. When we are at MD or AS, we read left to right.
}
\litem{
Simplify the expression below into the form $a+bi$. Then, choose the intervals that $a$ and $b$ belong to.
\[ (-10 + 6 i)(3 + 7 i) \]The solution is \( -72 - 52 i \), which is option E.\begin{enumerate}[label=\Alph*.]
\item \( a \in [-33, -29] \text{ and } b \in [40, 47] \)

 $-30 + 42 i$, which corresponds to just multiplying the real terms to get the real part of the solution and the coefficients in the complex terms to get the complex part.
\item \( a \in [12, 16] \text{ and } b \in [87, 94] \)

 $12 + 88 i$, which corresponds to adding a minus sign in the second term.
\item \( a \in [12, 16] \text{ and } b \in [-89, -86] \)

 $12 - 88 i$, which corresponds to adding a minus sign in the first term.
\item \( a \in [-74, -67] \text{ and } b \in [52, 53] \)

 $-72 + 52 i$, which corresponds to adding a minus sign in both terms.
\item \( a \in [-74, -67] \text{ and } b \in [-52, -51] \)

* $-72 - 52 i$, which is the correct option.
\end{enumerate}

\textbf{General Comment:} You can treat $i$ as a variable and distribute. Just remember that $i^2=-1$, so you can continue to reduce after you distribute.
}
\litem{
Choose the \textbf{smallest} set of Complex numbers that the number below belongs to.
\[ \sqrt{\frac{1001}{0}}+\sqrt{105} i \]The solution is \( \text{Not a Complex Number} \), which is option A.\begin{enumerate}[label=\Alph*.]
\item \( \text{Not a Complex Number} \)

* This is the correct option!
\item \( \text{Nonreal Complex} \)

This is a Complex number $(a+bi)$ that is not Real (has $i$ as part of the number).
\item \( \text{Pure Imaginary} \)

This is a Complex number $(a+bi)$ that \textbf{only} has an imaginary part like $2i$.
\item \( \text{Irrational} \)

These cannot be written as a fraction of Integers. Remember: $\pi$ is not an Integer!
\item \( \text{Rational} \)

These are numbers that can be written as fraction of Integers (e.g., -2/3 + 5)
\end{enumerate}

\textbf{General Comment:} Be sure to simplify $i^2 = -1$. This may remove the imaginary portion for your number. If you are having trouble, you may want to look at the \textit{Subgroups of the Real Numbers} section.
}
\litem{
Choose the \textbf{smallest} set of Real numbers that the number below belongs to.
\[ \sqrt{\frac{69696}{484}} \]The solution is \( \text{Whole} \), which is option B.\begin{enumerate}[label=\Alph*.]
\item \( \text{Integer} \)

These are the negative and positive counting numbers (..., -3, -2, -1, 0, 1, 2, 3, ...)
\item \( \text{Whole} \)

* This is the correct option!
\item \( \text{Irrational} \)

These cannot be written as a fraction of Integers.
\item \( \text{Rational} \)

These are numbers that can be written as fraction of Integers (e.g., -2/3)
\item \( \text{Not a Real number} \)

These are Nonreal Complex numbers \textbf{OR} things that are not numbers (e.g., dividing by 0).
\end{enumerate}

\textbf{General Comment:} First, you \textbf{NEED} to simplify the expression. This question simplifies to $264$. 
 
 Be sure you look at the simplified fraction and not just the decimal expansion. Numbers such as 13, 17, and 19 provide \textbf{long but repeating/terminating decimal expansions!} 
 
 The only ways to *not* be a Real number are: dividing by 0 or taking the square root of a negative number. 
 
 Irrational numbers are more than just square root of 3: adding or subtracting values from square root of 3 is also irrational.
}
\litem{
Simplify the expression below into the form $a+bi$. Then, choose the intervals that $a$ and $b$ belong to.
\[ \frac{45 + 77 i}{6 + i} \]The solution is \( 9.38  + 11.27 i \), which is option A.\begin{enumerate}[label=\Alph*.]
\item \( a \in [8.5, 10] \text{ and } b \in [11, 12] \)

* $9.38  + 11.27 i$, which is the correct option.
\item \( a \in [4, 6.5] \text{ and } b \in [13, 14.5] \)

 $5.22  + 13.70 i$, which corresponds to forgetting to multiply the conjugate by the numerator and not computing the conjugate correctly.
\item \( a \in [8.5, 10] \text{ and } b \in [415.5, 417.5] \)

 $9.38  + 417.00 i$, which corresponds to forgetting to multiply the conjugate by the numerator.
\item \( a \in [346, 348.5] \text{ and } b \in [11, 12] \)

 $347.00  + 11.27 i$, which corresponds to forgetting to multiply the conjugate by the numerator and using a plus instead of a minus in the denominator.
\item \( a \in [6.5, 8] \text{ and } b \in [75.5, 77.5] \)

 $7.50  + 77.00 i$, which corresponds to just dividing the first term by the first term and the second by the second.
\end{enumerate}

\textbf{General Comment:} Multiply the numerator and denominator by the *conjugate* of the denominator, then simplify. For example, if we have $2+3i$, the conjugate is $2-3i$.
}
\litem{
Choose the \textbf{smallest} set of Complex numbers that the number below belongs to.
\[ \sqrt{\frac{0}{14}}+\sqrt{9}i \]The solution is \( \text{Pure Imaginary} \), which is option E.\begin{enumerate}[label=\Alph*.]
\item \( \text{Not a Complex Number} \)

This is not a number. The only non-Complex number we know is dividing by 0 as this is not a number!
\item \( \text{Nonreal Complex} \)

This is a Complex number $(a+bi)$ that is not Real (has $i$ as part of the number).
\item \( \text{Irrational} \)

These cannot be written as a fraction of Integers. Remember: $\pi$ is not an Integer!
\item \( \text{Rational} \)

These are numbers that can be written as fraction of Integers (e.g., -2/3 + 5)
\item \( \text{Pure Imaginary} \)

* This is the correct option!
\end{enumerate}

\textbf{General Comment:} Be sure to simplify $i^2 = -1$. This may remove the imaginary portion for your number. If you are having trouble, you may want to look at the \textit{Subgroups of the Real Numbers} section.
}
\litem{
Choose the \textbf{smallest} set of Real numbers that the number below belongs to.
\[ \sqrt{\frac{13}{0}} \]The solution is \( \text{Not a Real number} \), which is option D.\begin{enumerate}[label=\Alph*.]
\item \( \text{Rational} \)

These are numbers that can be written as fraction of Integers (e.g., -2/3)
\item \( \text{Integer} \)

These are the negative and positive counting numbers (..., -3, -2, -1, 0, 1, 2, 3, ...)
\item \( \text{Whole} \)

These are the counting numbers with 0 (0, 1, 2, 3, ...)
\item \( \text{Not a Real number} \)

* This is the correct option!
\item \( \text{Irrational} \)

These cannot be written as a fraction of Integers.
\end{enumerate}

\textbf{General Comment:} First, you \textbf{NEED} to simplify the expression. This question simplifies to $\sqrt{\frac{13}{0}}$. 
 
 Be sure you look at the simplified fraction and not just the decimal expansion. Numbers such as 13, 17, and 19 provide \textbf{long but repeating/terminating decimal expansions!} 
 
 The only ways to *not* be a Real number are: dividing by 0 or taking the square root of a negative number. 
 
 Irrational numbers are more than just square root of 3: adding or subtracting values from square root of 3 is also irrational.
}
\litem{
Simplify the expression below and choose the interval the simplification is contained within.
\[ 18 - 16 \div 20 * 2 - (14 * 9) \]The solution is \( -109.600 \), which is option A.\begin{enumerate}[label=\Alph*.]
\item \( [-110.2, -108.8] \)

* -109.600, which is the correct option.
\item \( [-109.5, -107.3] \)

 -108.400, which corresponds to an Order of Operations error: not reading left-to-right for multiplication/division.
\item \( [142.5, 144.2] \)

 143.600, which corresponds to not distributing addition and subtraction correctly.
\item \( [20.2, 24] \)

 21.600, which corresponds to not distributing a negative correctly.
\item \( \text{None of the above} \)

 You may have gotten this by making an unanticipated error. If you got a value that is not any of the others, please let the coordinator know so they can help you figure out what happened.
\end{enumerate}

\textbf{General Comment:} While you may remember (or were taught) PEMDAS is done in order, it is actually done as P/E/MD/AS. When we are at MD or AS, we read left to right.
}
\litem{
Simplify the expression below into the form $a+bi$. Then, choose the intervals that $a$ and $b$ belong to.
\[ (6 + 2 i)(-4 - 9 i) \]The solution is \( -6 - 62 i \), which is option B.\begin{enumerate}[label=\Alph*.]
\item \( a \in [-8, 3] \text{ and } b \in [57, 64] \)

 $-6 + 62 i$, which corresponds to adding a minus sign in both terms.
\item \( a \in [-8, 3] \text{ and } b \in [-62, -57] \)

* $-6 - 62 i$, which is the correct option.
\item \( a \in [-46, -33] \text{ and } b \in [-50, -45] \)

 $-42 - 46 i$, which corresponds to adding a minus sign in the first term.
\item \( a \in [-29, -19] \text{ and } b \in [-19, -15] \)

 $-24 - 18 i$, which corresponds to just multiplying the real terms to get the real part of the solution and the coefficients in the complex terms to get the complex part.
\item \( a \in [-46, -33] \text{ and } b \in [45, 47] \)

 $-42 + 46 i$, which corresponds to adding a minus sign in the second term.
\end{enumerate}

\textbf{General Comment:} You can treat $i$ as a variable and distribute. Just remember that $i^2=-1$, so you can continue to reduce after you distribute.
}
\end{enumerate}

\end{document}