\documentclass{extbook}[14pt]
\usepackage{multicol, enumerate, enumitem, hyperref, color, soul, setspace, parskip, fancyhdr, amssymb, amsthm, amsmath, bbm, latexsym, units, mathtools}
\everymath{\displaystyle}
\usepackage[headsep=0.5cm,headheight=0cm, left=1 in,right= 1 in,top= 1 in,bottom= 1 in]{geometry}
\pagestyle{fancy}
\lhead{}
\chead{Answer Key for Module\,1\,-\,Real\,and\,Complex\,Numbers Version A}
\rhead{}
\lfoot{Summer\,C\,2020}
\cfoot{}
\rfoot{}
\begin{document}
\textbf{This key should allow you to understand why you choose the option you did (beyond just getting a question right or wrong). \href{https://xronos.clas.ufl.edu/mac1105spring2020/courseDescriptionAndMisc/Exams/LearningFromResults}{More instructions on how to use this key can be found here}.}

\textbf{If you have a suggestion to make the keys better, \href{https://forms.gle/CZkbZmPbC9XALEE88}{please fill out the short survey here}.}

\textit{Note: This key is auto-generated and may contain issues and/or errors. The keys are reviewed after each exam to ensure grading is done accurately. If there are issues (like duplicate options), they are noted in the offline gradebook. The keys are a work-in-progress to give students as many resources to improve as possible.}

\rule{\textwidth}{0.4pt}

1. Simplify the expression below into the form $a+bi$. Then, choose the intervals that $a$ and $b$ belong to.
\[ \frac{-36 - 88 i}{6 + i} \] 
The solution is $ -8.22  - 13.30 i $ 

\begin{enumerate}[label=\Alph*.] 
\item $ a \in [-3.87, -3.45] \text{ and } b \in [-16.4, -14.5] $ 

  $-3.46  - 15.24 i$, which corresponds to forgetting to multiply the conjugate by the numerator and not computing the conjugate correctly. 
\item $ a \in [-8.27, -7.2] \text{ and } b \in [-493.1, -490.7] $ 

  $-8.22  - 492.00 i$, which corresponds to forgetting to multiply the conjugate by the numerator. 
\item $ a \in [-8.27, -7.2] \text{ and } b \in [-13.8, -12.2] $ 

 * $-8.22  - 13.30 i$, which is the correct option. 
\item $ a \in [-304.04, -303.28] \text{ and } b \in [-13.8, -12.2] $ 

  $-304.00  - 13.30 i$, which corresponds to forgetting to multiply the conjugate by the numerator and using a plus instead of a minus in the denominator. 
\item $ a \in [-6.36, -5.23] \text{ and } b \in [-88.9, -86.1] $ 

  $-6.00  - 88.00 i$, which corresponds to just dividing the first term by the first term and the second by the second. 
\end{enumerate} 
 
\textbf{General Comment:} Multiply the numerator and denominator by the *conjugate* of the denominator, then simplify. For example, if we have $2+3i$, the conjugate is $2-3i$. 

-----------------------------------------------

2. Simplify the expression below into the form $a+bi$. Then, choose the intervals that $a$ and $b$ belong to.
\[ (7 + 9 i)(2 + 6 i) \] 
The solution is $ -40 + 60 i $ 

\begin{enumerate}[label=\Alph*.] 
\item $ a \in [-43, -34] \text{ and } b \in [57, 63] $ 

 * $-40 + 60 i$, which is the correct option. 
\item $ a \in [12, 17] \text{ and } b \in [51, 57] $ 

  $14 + 54 i$, which corresponds to just multiplying the real terms to get the real part of the solution and the coefficients in the complex terms to get the complex part. 
\item $ a \in [64, 69] \text{ and } b \in [19, 27] $ 

  $68 + 24 i$, which corresponds to adding a minus sign in the first term. 
\item $ a \in [-43, -34] \text{ and } b \in [-66, -54] $ 

  $-40 - 60 i$, which corresponds to adding a minus sign in both terms. 
\item $ a \in [64, 69] \text{ and } b \in [-27, -17] $ 

  $68 - 24 i$, which corresponds to adding a minus sign in the second term. 
\end{enumerate} 
 
\textbf{General Comment:} You can treat $i$ as a variable and distribute. Just remember that $i^2=-1$, so you can continue to reduce after you distribute. 

-----------------------------------------------

3. Simplify the expression below and choose the interval the simplification is contained within.
\[ 1 - 20 \div 17 * 13 - (12 * 16) \] 
The solution is $ -206.294 $ 

\begin{enumerate}[label=\Alph*.] 
\item $ [188, 195] $ 

  192.910, which corresponds to not distributing addition and subtraction correctly. 
\item $ [-196, -186] $ 

  -191.090, which corresponds to an Order of Operations error: not reading left-to-right for multiplication/division. 
\item $ [-421, -420] $ 

  -420.706, which corresponds to not distributing a negative correctly. 
\item $ [-218, -202] $ 

 * -206.294, which is the correct option. 
\item $ \text{None of the above} $ 

  You may have gotten this by making an unanticipated error. If you got a value that is not any of the others, please let the coordinator know so they can help you figure out what happened. 
\end{enumerate} 
 
\textbf{General Comment:} While you may remember (or were taught) PEMDAS is done in order, it is actually done as P/E/MD/AS. When we are at MD or AS, we read left to right. 

-----------------------------------------------

4. Choose the \textbf{smallest} set of Complex numbers that the number below belongs to.
\[ \frac{\sqrt{55}}{17}+\sqrt{-6}i \] 
The solution is $ \text{Irrational} $ 

\begin{enumerate}[label=\Alph*.] 
\item $ \text{Pure Imaginary} $ 

 This is a Complex number $(a+bi)$ that \textbf{only} has an imaginary part like $2i$. 
\item $ \text{Rational} $ 

 These are numbers that can be written as fraction of Integers (e.g., -2/3 + 5) 
\item $ \text{Not a Complex Number} $ 

 This is not a number. The only non-Complex number we know is dividing by 0 as this is not a number! 
\item $ \text{Nonreal Complex} $ 

 This is a Complex number $(a+bi)$ that is not Real (has $i$ as part of the number). 
\item $ \text{Irrational} $ 

 * This is the correct option! 
\end{enumerate} 
 
\textbf{General Comment:} Be sure to simplify $i^2 = -1$. This may remove the imaginary portion for your number. If you are having trouble, you may want to look at the \textit{Subgroups of the Real Numbers} section. 

-----------------------------------------------

0. Choose the \textbf{smallest} set of Real numbers that the number below belongs to.
\[ \sqrt{\frac{23104}{361}} \] 
The solution is $ \text{Whole} $ 

\begin{enumerate}[label=\Alph*.] 
\item $ \text{Whole} $ 

 * This is the correct option! 
\item $ \text{Integer} $ 

 These are the negative and positive counting numbers (..., -3, -2, -1, 0, 1, 2, 3, ...) 
\item $ \text{Rational} $ 

 These are numbers that can be written as fraction of Integers (e.g., -2/3) 
\item $ \text{Not a Real number} $ 

 These are Nonreal Complex numbers \textbf{OR} things that are not numbers (e.g., dividing by 0). 
\item $ \text{Irrational} $ 

 These cannot be written as a fraction of Integers. 
\end{enumerate} 
 
\textbf{General Comment:} First, you \textbf{NEED} to simplify the expression. This question simplifies to $152$. 
 
 Be sure you look at the simplified fraction and not just the decimal expansion. Numbers such as 13, 17, and 19 provide \textbf{long but repeating/terminating decimal expansions!} 
 
 The only ways to *not* be a Real number are: dividing by 0 or taking the square root of a negative number. 
 
 Irrational numbers are more than just square root of 3: adding or subtracting values from square root of 3 is also irrational. 

-----------------------------------------------


\end{document}

