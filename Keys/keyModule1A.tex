\documentclass{extbook}[14pt]
\usepackage{multicol, enumerate, enumitem, hyperref, color, soul, setspace, parskip, fancyhdr, amssymb, amsthm, amsmath, bbm, latexsym, units, mathtools}
\everymath{\displaystyle}
\usepackage[headsep=0.5cm,headheight=0cm, left=1 in,right= 1 in,top= 1 in,bottom= 1 in]{geometry}
\usepackage{dashrule}  % Package to use the command below to create lines between items
\newcommand{\litem}[1]{\item #1

\rule{\textwidth}{0.4pt}}
\pagestyle{fancy}
\lhead{}
\chead{Answer Key for Progress Quiz 4 Version A}
\rhead{}
\lfoot{6286-1986}
\cfoot{}
\rfoot{Fall 2020}
\begin{document}
\textbf{This key should allow you to understand why you choose the option you did (beyond just getting a question right or wrong). \href{https://xronos.clas.ufl.edu/mac1105spring2020/courseDescriptionAndMisc/Exams/LearningFromResults}{More instructions on how to use this key can be found here}.}

\textbf{If you have a suggestion to make the keys better, \href{https://forms.gle/CZkbZmPbC9XALEE88}{please fill out the short survey here}.}

\textit{Note: This key is auto-generated and may contain issues and/or errors. The keys are reviewed after each exam to ensure grading is done accurately. If there are issues (like duplicate options), they are noted in the offline gradebook. The keys are a work-in-progress to give students as many resources to improve as possible.}

\rule{\textwidth}{0.4pt}

\begin{enumerate}\litem{
Simplify the expression below into the form $a+bi$. Then, choose the intervals that $a$ and $b$ belong to.
\[ (-7 - 10 i)(4 - 2 i) \]
The solution is \( -48 - 26 i \), which is option D.\begin{enumerate}[label=\Alph*.]
\item \( a \in [-8, -3] \text{ and } b \in [-57, -53] \)

 $-8 - 54 i$, which corresponds to adding a minus sign in the second term.
\item \( a \in [-50, -41] \text{ and } b \in [21, 31] \)

 $-48 + 26 i$, which corresponds to adding a minus sign in both terms.
\item \( a \in [-32, -24] \text{ and } b \in [20, 21] \)

 $-28 + 20 i$, which corresponds to just multiplying the real terms to get the real part of the solution and the coefficients in the complex terms to get the complex part.
\item \( a \in [-50, -41] \text{ and } b \in [-26, -25] \)

* $-48 - 26 i$, which is the correct option.
\item \( a \in [-8, -3] \text{ and } b \in [47, 55] \)

 $-8 + 54 i$, which corresponds to adding a minus sign in the first term.
\end{enumerate}

\textbf{General Comment:} You can treat $i$ as a variable and distribute. Just remember that $i^2=-1$, so you can continue to reduce after you distribute.
}
\litem{
Choose the \textbf{smallest} set of Complex numbers that the number below belongs to.
\[ \frac{18}{-20}+\sqrt{-49}i \]
The solution is \( \text{Rational} \), which is option D.\begin{enumerate}[label=\Alph*.]
\item \( \text{Irrational} \)

These cannot be written as a fraction of Integers. Remember: $\pi$ is not an Integer!
\item \( \text{Nonreal Complex} \)

This is a Complex number $(a+bi)$ that is not Real (has $i$ as part of the number).
\item \( \text{Pure Imaginary} \)

This is a Complex number $(a+bi)$ that \textbf{only} has an imaginary part like $2i$.
\item \( \text{Rational} \)

* This is the correct option!
\item \( \text{Not a Complex Number} \)

This is not a number. The only non-Complex number we know is dividing by 0 as this is not a number!
\end{enumerate}

\textbf{General Comment:} Be sure to simplify $i^2 = -1$. This may remove the imaginary portion for your number. If you are having trouble, you may want to look at the \textit{Subgroups of the Real Numbers} section.
}
\litem{
Simplify the expression below into the form $a+bi$. Then, choose the intervals that $a$ and $b$ belong to.
\[ \frac{54 - 22 i}{-3 + 5 i} \]
The solution is \( -8.00  - 6.00 i \), which is option C.\begin{enumerate}[label=\Alph*.]
\item \( a \in [-272.5, -271.5] \text{ and } b \in [-6.5, -5.5] \)

 $-272.00  - 6.00 i$, which corresponds to forgetting to multiply the conjugate by the numerator and using a plus instead of a minus in the denominator.
\item \( a \in [-9, -7.5] \text{ and } b \in [-204.5, -203.5] \)

 $-8.00  - 204.00 i$, which corresponds to forgetting to multiply the conjugate by the numerator.
\item \( a \in [-9, -7.5] \text{ and } b \in [-6.5, -5.5] \)

* $-8.00  - 6.00 i$, which is the correct option.
\item \( a \in [-2.5, -1] \text{ and } b \in [8.5, 10.5] \)

 $-1.53  + 9.88 i$, which corresponds to forgetting to multiply the conjugate by the numerator and not computing the conjugate correctly.
\item \( a \in [-19, -16] \text{ and } b \in [-4.5, -2.5] \)

 $-18.00  - 4.40 i$, which corresponds to just dividing the first term by the first term and the second by the second.
\end{enumerate}

\textbf{General Comment:} Multiply the numerator and denominator by the *conjugate* of the denominator, then simplify. For example, if we have $2+3i$, the conjugate is $2-3i$.
}
\litem{
Choose the \textbf{smallest} set of Real numbers that the number below belongs to.
\[ \sqrt{\frac{2145}{13}} \]
The solution is \( \text{Irrational} \), which is option C.\begin{enumerate}[label=\Alph*.]
\item \( \text{Rational} \)

These are numbers that can be written as fraction of Integers (e.g., -2/3)
\item \( \text{Integer} \)

These are the negative and positive counting numbers (..., -3, -2, -1, 0, 1, 2, 3, ...)
\item \( \text{Irrational} \)

* This is the correct option!
\item \( \text{Not a Real number} \)

These are Nonreal Complex numbers \textbf{OR} things that are not numbers (e.g., dividing by 0).
\item \( \text{Whole} \)

These are the counting numbers with 0 (0, 1, 2, 3, ...)
\end{enumerate}

\textbf{General Comment:} First, you \textbf{NEED} to simplify the expression. This question simplifies to $\sqrt{165}$. 
 
 Be sure you look at the simplified fraction and not just the decimal expansion. Numbers such as 13, 17, and 19 provide \textbf{long but repeating/terminating decimal expansions!} 
 
 The only ways to *not* be a Real number are: dividing by 0 or taking the square root of a negative number. 
 
 Irrational numbers are more than just square root of 3: adding or subtracting values from square root of 3 is also irrational.
}
\litem{
Simplify the expression below into the form $a+bi$. Then, choose the intervals that $a$ and $b$ belong to.
\[ \frac{-27 + 44 i}{6 + i} \]
The solution is \( -3.19  + 7.86 i \), which is option D.\begin{enumerate}[label=\Alph*.]
\item \( a \in [-5.2, -4.05] \text{ and } b \in [42.5, 44.5] \)

 $-4.50  + 44.00 i$, which corresponds to just dividing the first term by the first term and the second by the second.
\item \( a \in [-6, -5.25] \text{ and } b \in [6, 6.5] \)

 $-5.57  + 6.41 i$, which corresponds to forgetting to multiply the conjugate by the numerator and not computing the conjugate correctly.
\item \( a \in [-3.75, -2.85] \text{ and } b \in [290.5, 292.5] \)

 $-3.19  + 291.00 i$, which corresponds to forgetting to multiply the conjugate by the numerator.
\item \( a \in [-3.75, -2.85] \text{ and } b \in [6.5, 8.5] \)

* $-3.19  + 7.86 i$, which is the correct option.
\item \( a \in [-118.15, -117.85] \text{ and } b \in [6.5, 8.5] \)

 $-118.00  + 7.86 i$, which corresponds to forgetting to multiply the conjugate by the numerator and using a plus instead of a minus in the denominator.
\end{enumerate}

\textbf{General Comment:} Multiply the numerator and denominator by the *conjugate* of the denominator, then simplify. For example, if we have $2+3i$, the conjugate is $2-3i$.
}
\litem{
Simplify the expression below and choose the interval the simplification is contained within.
\[ 10 - 19 \div 12 * 16 - (4 * 11) \]
The solution is \( -59.333 \), which is option B.\begin{enumerate}[label=\Alph*.]
\item \( [52.9, 56.9] \)

 53.901, which corresponds to not distributing addition and subtraction correctly.
\item \( [-61.33, -57.33] \)

* -59.333, which is the correct option.
\item \( [-214.67, -207.67] \)

 -212.667, which corresponds to not distributing a negative correctly.
\item \( [-42.1, -27.1] \)

 -34.099, which corresponds to an Order of Operations error: not reading left-to-right for multiplication/division.
\item \( \text{None of the above} \)

 You may have gotten this by making an unanticipated error. If you got a value that is not any of the others, please let the coordinator know so they can help you figure out what happened.
\end{enumerate}

\textbf{General Comment:} While you may remember (or were taught) PEMDAS is done in order, it is actually done as P/E/MD/AS. When we are at MD or AS, we read left to right.
}
\litem{
Choose the \textbf{smallest} set of Real numbers that the number below belongs to.
\[ \sqrt{\frac{24336}{144}} \]
The solution is \( \text{Whole} \), which is option E.\begin{enumerate}[label=\Alph*.]
\item \( \text{Integer} \)

These are the negative and positive counting numbers (..., -3, -2, -1, 0, 1, 2, 3, ...)
\item \( \text{Not a Real number} \)

These are Nonreal Complex numbers \textbf{OR} things that are not numbers (e.g., dividing by 0).
\item \( \text{Irrational} \)

These cannot be written as a fraction of Integers.
\item \( \text{Rational} \)

These are numbers that can be written as fraction of Integers (e.g., -2/3)
\item \( \text{Whole} \)

* This is the correct option!
\end{enumerate}

\textbf{General Comment:} First, you \textbf{NEED} to simplify the expression. This question simplifies to $156$. 
 
 Be sure you look at the simplified fraction and not just the decimal expansion. Numbers such as 13, 17, and 19 provide \textbf{long but repeating/terminating decimal expansions!} 
 
 The only ways to *not* be a Real number are: dividing by 0 or taking the square root of a negative number. 
 
 Irrational numbers are more than just square root of 3: adding or subtracting values from square root of 3 is also irrational.
}
\litem{
Simplify the expression below and choose the interval the simplification is contained within.
\[ 6 - 16^2 + 5 \div 18 * 2 \div 8 \]
The solution is \( -249.931 \), which is option D.\begin{enumerate}[label=\Alph*.]
\item \( [-250, -249.97] \)

 -249.983, which corresponds to an Order of Operations error: not reading left-to-right for multiplication/division.
\item \( [262.06, 262.09] \)

 262.069, which corresponds to an Order of Operations error: multiplying by negative before squaring. For example: $(-3)^2 \neq -3^2$
\item \( [261.98, 262.04] \)

 262.017, which corresponds to two Order of Operations errors.
\item \( [-249.95, -249.91] \)

* -249.931, this is the correct option
\item \( \text{None of the above} \)

 You may have gotten this by making an unanticipated error. If you got a value that is not any of the others, please let the coordinator know so they can help you figure out what happened.
\end{enumerate}

\textbf{General Comment:} While you may remember (or were taught) PEMDAS is done in order, it is actually done as P/E/MD/AS. When we are at MD or AS, we read left to right.
}
\litem{
Choose the \textbf{smallest} set of Complex numbers that the number below belongs to.
\[ \sqrt{\frac{484}{441}}+\sqrt{208} i \]
The solution is \( \text{Nonreal Complex} \), which is option E.\begin{enumerate}[label=\Alph*.]
\item \( \text{Not a Complex Number} \)

This is not a number. The only non-Complex number we know is dividing by 0 as this is not a number!
\item \( \text{Irrational} \)

These cannot be written as a fraction of Integers. Remember: $\pi$ is not an Integer!
\item \( \text{Pure Imaginary} \)

This is a Complex number $(a+bi)$ that \textbf{only} has an imaginary part like $2i$.
\item \( \text{Rational} \)

These are numbers that can be written as fraction of Integers (e.g., -2/3 + 5)
\item \( \text{Nonreal Complex} \)

* This is the correct option!
\end{enumerate}

\textbf{General Comment:} Be sure to simplify $i^2 = -1$. This may remove the imaginary portion for your number. If you are having trouble, you may want to look at the \textit{Subgroups of the Real Numbers} section.
}
\litem{
Simplify the expression below into the form $a+bi$. Then, choose the intervals that $a$ and $b$ belong to.
\[ (-8 - 9 i)(3 + 4 i) \]
The solution is \( 12 - 59 i \), which is option D.\begin{enumerate}[label=\Alph*.]
\item \( a \in [-30, -19] \text{ and } b \in [-36, -35] \)

 $-24 - 36 i$, which corresponds to just multiplying the real terms to get the real part of the solution and the coefficients in the complex terms to get the complex part.
\item \( a \in [12, 14] \text{ and } b \in [59, 69] \)

 $12 + 59 i$, which corresponds to adding a minus sign in both terms.
\item \( a \in [-64, -57] \text{ and } b \in [3, 6] \)

 $-60 + 5 i$, which corresponds to adding a minus sign in the second term.
\item \( a \in [12, 14] \text{ and } b \in [-62, -56] \)

* $12 - 59 i$, which is the correct option.
\item \( a \in [-64, -57] \text{ and } b \in [-6, -3] \)

 $-60 - 5 i$, which corresponds to adding a minus sign in the first term.
\end{enumerate}

\textbf{General Comment:} You can treat $i$ as a variable and distribute. Just remember that $i^2=-1$, so you can continue to reduce after you distribute.
}
\end{enumerate}

\end{document}