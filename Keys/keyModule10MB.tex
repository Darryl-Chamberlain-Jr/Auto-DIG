
\documentclass{article}[14pt]
\usepackage{multicol, enumerate, enumitem, hyperref, color, soul, setspace, parskip, fancyhdr, amssymb, amsthm, amsmath, bbm, latexsym, units, mathtools}
\everymath{\displaystyle}
\usepackage[headsep=0.5cm,headheight=0cm, left=1 in,right= 1 in,top= 1 in,bottom= 1 in]{geometry}
\pagestyle{fancy}
\lhead{}
\chead{Answer Key for Module\,10M\,-\,Modeling\,with\,Power\,Functions Version B}
\rhead{}
\lfoot{debug}
\cfoot{}
\rfoot{}
\begin{document}
\textbf{This key should allow you to understand why you choose the option you did (beyond just getting a question right or wrong). \href{https://xronos.clas.ufl.edu/mac1105spring2020/courseDescriptionAndMisc/Exams/LearningFromResults}{More instructions on how to use this key can be found here}.}

\textbf{If you have a suggestion to make the keys better, \href{https://forms.gle/CZkbZmPbC9XALEE88}{please fill out the short survey here}.}

\textit{Note: This key is auto-generated and may contain issues and/or errors. The keys are reviewed after each exam to ensure grading is done accurately. If there are issues (like duplicate options), they are noted in the offline gradebook. The keys are a work-in-progress to give students as many resources to improve as possible.}

\rule{\textwidth}{0.4pt}

46. A town has an initial population of 70000. The town's population for the next 10 years is provided below. Which type of function would be most appropriate to model the town's population?
Check for table in main PDF. 
The solution is $ \text{Exponential} $ 

\begin{enumerate}[label=\Alph*.] 
\item $ \text{Logarithmic} $ 

 This suggests the slowest of growths that we know. 
\item $ \text{Direct variation} $ 

 This suggests a growth faster than constant but slower than exponential. 
\item $ \text{Exponential} $ 

 This suggests the fastest of growths that we know. 
\item $ \text{Linear} $ 

 This suggests a constant growth. You would be able to add or subtract the same amount year-to-year if this is the correct answer. 
\end{enumerate} 
 
\textbf{General Comments:} We are trying to compare the growth rate of the population. Growth rates can be characterized from slowest to fastest as: logarithmic, indirect, linear, direct, exponential. The best way to approach this is to first compare it to linear (is it linear, faster than linear, or slower than linear)? If faster, is it as fast as exponential? If slower, is it as slow as logarithmic?

-----------------------------------------------

47. For the scenario below, model the rate of vibration (cm/s) of the string in terms of the length of the string. Then determine the variation constant $k$ of the model (if possible). The constant should be in terms of cm and s.
\begin{center} \textit{The rate of vibration of a string under constant tension varies based on the type of string and the length of the string. The rate of vibration of string $\omega$ decreases as the quartic length of the string decreases. For example, when string $\omega$ is 3 mm long, the rate of vibration is 25 cm/s.} \end{center} 
The solution is $ k = 0.20 $ 

\begin{enumerate}[label=\Alph*.] 
\item $ k = 0.20 $ 

 * This is the correct option, which corresponds to the model $R = \frac{k}{l^{4}}$ AND converts from mm to cm. 
\item $ k = 0.31 $ 

 This option uses the model $R = kl^{4}$ as if this is a direct variation AND does not convert from mm to cm so that the units match. 
\item $ k = 3086.42 $ 

 This option uses the model $R = kl^{4}$ as if this is a direct variation. 
\item $ k = 2025.00 $ 

 This option uses the correct model, $R = \frac{k}{l^{4}}$, but does not convert from mm to cm so that the units match. 
\item $ \text{None of the above.} $ 

 Talk with the coordinator if you chose this option. 
\end{enumerate} 
 
\textbf{General comments:} The most common mistake on this question is to not convert mm to cm! When modeling, you need to make sure all of the units for your variables are compatible.

-----------------------------------------------

48. For the scenario below, find the variation constant $k$ of the model (if possible).
\begin{center} \textit{In an alternative galaxy, the cube of the time, $T$ (Earth years), required for a planet to orbit Sun $\chi$ increases as the square of the distance, $d$ (AUs), that the planet is from Sun $\chi$ increases. For example, when Ea's average distance from Sun $\chi$ is 8, it takes 72 Earth days to complete an orbit.} \end{center} 
The solution is $ k = 5832.000 $ 

\begin{enumerate}[label=\Alph*.] 
\item $ k = 5832.000 $ 

 * This is the correct option corresponding to the model $T^{3} = k d^{2}$. 
\item $ k = 1.471 $ 

 This corresponds to the model $T^{1/3} = k d^{1/2}$. 
\item $ k = 23887872.000 $ 

 This corresponds to the model $T^{3} = \frac{k}{d^{2}}$. 
\item $ k = 4.028 $ 

 This copies the constant used in the homework. 
\item $ \text{Unable to compute the constant based on the information given.} $ 

 This corresponds to believing you cannot determine the type of model from the information given. 
\end{enumerate} 
 
\textbf{General comments:} since $T$ increases proportionally as $d$ increases, we know this is a direct variation model.

-----------------------------------------------

49. For the scenario below, use the model for the volume of a cylinder as $V = \pi r^2 h$ to find the coefficient for the model of the new volume $V = k r^2 h$.
\begin{center} \textit{Pepsi wants to increase the volume of soda in their cans. They've decided to increase the radius by 20 percent and increase the height by 12 percent. They want to model the new volume based on the radius and height of the original cans.} \end{center} 
The solution is $ k = 5.06676 $ 

\begin{enumerate}[label=\Alph*.] 
\item $ k = 0.01508 $ 

 This corresponds to the model: $V = \pi (0.20 r)^2 (0.12 h)$. 
\item $ k = 0.00480 $ 

 This corresponds to the model: $V = (0.20 r)^2 (0.12 h)$. 
\item $ k = 1.61280 $ 

 This corresponds to the model: $V = (1.20 r)^2 (1.12 h)$. 
\item $ k = 5.06676 $ 

 * This is the correct option and corresponds to the model: $V = \pi (1.20 r)^2 (1.12 h)$. 
\item $ \text{None of the above.} $ 

 If you chose this, please talk with the coordinator to discuss why you believe none of the options are correct. 
\end{enumerate} 
 
\textbf{General comments:} When calculating the new dimensions, you need to add/subtract from 100\%. For example, a 10\% increase in height would result in 110\% of the original height: $1.1h_{old} = h_{new}$.

-----------------------------------------------

50. Choose the model type that would best describe the scenario below.
\begin{center} \textit{Social distancing is a common tactic to counter potential epidemics. This is due to the exponential increase in number of people infected as the density of people living in an area increases.} \end{center} 
The solution is $ \text{None of the above} $ 

\begin{enumerate}[label=\Alph*.] 
\item $ \text{Direct variation} $ 

  
\item $ \text{Indirect variation} $ 

  
\item $ \text{Joint variation} $ 

  
\item $ \text{None of the above} $ 

  
\end{enumerate} 
 
This is an exponential variation, which grows significantly faster than any power function.

-----------------------------------------------


\end{document}

