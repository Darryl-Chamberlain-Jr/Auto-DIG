\documentclass{extbook}[14pt]
\usepackage{multicol, enumerate, enumitem, hyperref, color, soul, setspace, parskip, fancyhdr, amssymb, amsthm, amsmath, bbm, latexsym, units, mathtools}
\everymath{\displaystyle}
\usepackage[headsep=0.5cm,headheight=0cm, left=1 in,right= 1 in,top= 1 in,bottom= 1 in]{geometry}
\usepackage{dashrule}  % Package to use the command below to create lines between items
\newcommand{\litem}[1]{\item #1

\rule{\textwidth}{0.4pt}}
\pagestyle{fancy}
\lhead{}
\chead{Answer Key for Progress Quiz 9 Version A}
\rhead{}
\lfoot{8590-6105}
\cfoot{}
\rfoot{Fall 2020}
\begin{document}
\textbf{This key should allow you to understand why you choose the option you did (beyond just getting a question right or wrong). \href{https://xronos.clas.ufl.edu/mac1105spring2020/courseDescriptionAndMisc/Exams/LearningFromResults}{More instructions on how to use this key can be found here}.}

\textbf{If you have a suggestion to make the keys better, \href{https://forms.gle/CZkbZmPbC9XALEE88}{please fill out the short survey here}.}

\textit{Note: This key is auto-generated and may contain issues and/or errors. The keys are reviewed after each exam to ensure grading is done accurately. If there are issues (like duplicate options), they are noted in the offline gradebook. The keys are a work-in-progress to give students as many resources to improve as possible.}

\rule{\textwidth}{0.4pt}

\begin{enumerate}\litem{
Subtract the following functions, then choose the domain of the resulting function from the list below.
\[ f(x) = 6x + 3 \text{ and } g(x) = \frac{4}{4x+17} \]

The solution is \( \text{ The domain is all Real numbers except } x = -4.25 \), which is option A.\begin{enumerate}[label=\Alph*.]
\item \( \text{ The domain is all Real numbers except } x = a, \text{ where } a \in [-4.25, 0.75] \)


\item \( \text{ The domain is all Real numbers less than or equal to } x = a, \text{ where } a \in [-8.5, 5.5] \)


\item \( \text{ The domain is all Real numbers greater than or equal to } x = a, \text{ where } a \in [-8.75, -4.75] \)


\item \( \text{ The domain is all Real numbers except } x = a \text{ and } x = b, \text{ where } a \in [2.75, 8.75] \text{ and } b \in [-1.75, 7.25] \)


\item \( \text{ The domain is all Real numbers. } \)


\end{enumerate}

\textbf{General Comment:} The new domain is the intersection of the previous domains.
}
\litem{
Find the inverse of the function below (if it exists). Then, evaluate the inverse at $x = -12$ and choose the interval the $f^{-1}(-12)$ belongs to.
\[ f(x) = \sqrt[3]{4 x + 3} \]

The solution is \( -432.75 \), which is option A.\begin{enumerate}[label=\Alph*.]
\item \( f^{-1}(-12) \in [-433.9, -431.9] \)

* This is the correct solution.
\item \( f^{-1}(-12) \in [-431.6, -428.2] \)

 Distractor 1: This corresponds to 
\item \( f^{-1}(-12) \in [431.8, 433.7] \)

 This solution corresponds to distractor 2.
\item \( f^{-1}(-12) \in [429, 432.7] \)

 This solution corresponds to distractor 3.
\item \( \text{ The function is not invertible for all Real numbers. } \)

 This solution corresponds to distractor 4.
\end{enumerate}

\textbf{General Comment:} Be sure you check that the function is 1-1 before trying to find the inverse!
}
\litem{
Determine whether the function below is 1-1.
\[ f(x) = \sqrt{-4 x - 15} \]

The solution is \( \text{yes} \), which is option D.\begin{enumerate}[label=\Alph*.]
\item \( \text{No, because there is an $x$-value that goes to 2 different $y$-values.} \)

Corresponds to the Vertical Line test, which checks if an expression is a function.
\item \( \text{No, because there is a $y$-value that goes to 2 different $x$-values.} \)

Corresponds to the Horizontal Line test, which this function passes.
\item \( \text{No, because the range of the function is not $(-\infty, \infty)$.} \)

Corresponds to believing 1-1 means the range is all Real numbers.
\item \( \text{Yes, the function is 1-1.} \)

* This is the solution.
\item \( \text{No, because the domain of the function is not $(-\infty, \infty)$.} \)

Corresponds to believing 1-1 means the domain is all Real numbers.
\end{enumerate}

\textbf{General Comment:} There are only two valid options: The function is 1-1 OR No because there is a $y$-value that goes to 2 different $x$-values.
}
\litem{
Find the inverse of the function below (if it exists). Then, evaluate the inverse at $x = 11$ and choose the interval the $f^{-1}(11)$ belongs to.
\[ f(x) = \sqrt[3]{5 x + 3} \]

The solution is \( 265.6 \), which is option B.\begin{enumerate}[label=\Alph*.]
\item \( f^{-1}(11) \in [-265.87, -265.38] \)

 This solution corresponds to distractor 2.
\item \( f^{-1}(11) \in [265.17, 265.77] \)

* This is the correct solution.
\item \( f^{-1}(11) \in [-267.63, -266.79] \)

 This solution corresponds to distractor 3.
\item \( f^{-1}(11) \in [266.38, 267.55] \)

 Distractor 1: This corresponds to 
\item \( \text{ The function is not invertible for all Real numbers. } \)

 This solution corresponds to distractor 4.
\end{enumerate}

\textbf{General Comment:} Be sure you check that the function is 1-1 before trying to find the inverse!
}
\litem{
Find the inverse of the function below. Then, evaluate the inverse at $x = 5$ and choose the interval that $f^{-1}(5)$ belongs to.
\[ f(x) = \ln{(x+2)}-3 \]

The solution is \( f^{-1}(5) = 2978.958 \), which is option A.\begin{enumerate}[label=\Alph*.]
\item \( f^{-1}(5) \in [2975.96, 2980.96] \)

 This is the solution.
\item \( f^{-1}(5) \in [4.39, 9.39] \)

 This solution corresponds to distractor 1.
\item \( f^{-1}(5) \in [14.09, 20.09] \)

 This solution corresponds to distractor 2.
\item \( f^{-1}(5) \in [1088.63, 1100.63] \)

 This solution corresponds to distractor 4.
\item \( f^{-1}(5) \in [2979.96, 2983.96] \)

 This solution corresponds to distractor 3.
\end{enumerate}

\textbf{General Comment:} Natural log and exponential functions always have an inverse. Once you switch the $x$ and $y$, use the conversion $ e^y = x \leftrightarrow y=\ln(x)$.
}
\litem{
Choose the interval below that $f$ composed with $g$ at $x=-1$ is in.
\[ f(x) = 2x^{3} +4 x^{2} +x \text{ and } g(x) = -x^{3} +2 x^{2} +3 x \]

The solution is \( 0.0 \), which is option D.\begin{enumerate}[label=\Alph*.]
\item \( (f \circ g)(-1) \in [8.77, 9.31] \)

 Distractor 3: Corresponds to being slightly off from the solution.
\item \( (f \circ g)(-1) \in [9.79, 11.27] \)

 Distractor 2: Corresponds to being slightly off from the solution.
\item \( (f \circ g)(-1) \in [2.48, 5.86] \)

 Distractor 1: Corresponds to reversing the composition.
\item \( (f \circ g)(-1) \in [-0.31, 1.82] \)

* This is the correct solution
\item \( \text{It is not possible to compose the two functions.} \)


\end{enumerate}

\textbf{General Comment:} $f$ composed with $g$ at $x$ means $f(g(x))$. The order matters!
}
\litem{
Choose the interval below that $f$ composed with $g$ at $x=-1$ is in.
\[ f(x) = 3x^{3} +3 x^{2} +x \text{ and } g(x) = -3x^{3} -2 x^{2} +2 x \]

The solution is \( -1.0 \), which is option B.\begin{enumerate}[label=\Alph*.]
\item \( (f \circ g)(-1) \in [1.2, 7.3] \)

 Distractor 2: Corresponds to being slightly off from the solution.
\item \( (f \circ g)(-1) \in [-1.9, 0.2] \)

* This is the correct solution
\item \( (f \circ g)(-1) \in [7.5, 9.7] \)

 Distractor 3: Corresponds to being slightly off from the solution.
\item \( (f \circ g)(-1) \in [-1.9, 0.2] \)

 Distractor 1: Corresponds to reversing the composition.
\item \( \text{It is not possible to compose the two functions.} \)


\end{enumerate}

\textbf{General Comment:} $f$ composed with $g$ at $x$ means $f(g(x))$. The order matters!
}
\litem{
Find the inverse of the function below. Then, evaluate the inverse at $x = 9$ and choose the interval that $f^{-1}(9)$ belongs to.
\[ f(x) = \ln{(x-2)}+5 \]

The solution is \( f^{-1}(9) = 56.598 \), which is option E.\begin{enumerate}[label=\Alph*.]
\item \( f^{-1}(9) \in [1202606.28, 1202610.28] \)

 This solution corresponds to distractor 1.
\item \( f^{-1}(9) \in [59877.14, 59882.14] \)

 This solution corresponds to distractor 2.
\item \( f^{-1}(9) \in [1101.63, 1103.63] \)

 This solution corresponds to distractor 4.
\item \( f^{-1}(9) \in [49.6, 54.6] \)

 This solution corresponds to distractor 3.
\item \( f^{-1}(9) \in [54.6, 59.6] \)

 This is the solution.
\end{enumerate}

\textbf{General Comment:} Natural log and exponential functions always have an inverse. Once you switch the $x$ and $y$, use the conversion $ e^y = x \leftrightarrow y=\ln(x)$.
}
\litem{
Multiply the following functions, then choose the domain of the resulting function from the list below.
\[ f(x) = \frac{4}{3x+17} \text{ and } g(x) = \frac{4}{5x-28} \]

The solution is \( \text{ The domain is all Real numbers except } x = -5.666666666666667 \text{ and } x = 5.6 \), which is option D.\begin{enumerate}[label=\Alph*.]
\item \( \text{ The domain is all Real numbers except } x = a, \text{ where } a \in [3.67, 10.67] \)


\item \( \text{ The domain is all Real numbers greater than or equal to } x = a, \text{ where } a \in [0.33, 7.33] \)


\item \( \text{ The domain is all Real numbers less than or equal to } x = a, \text{ where } a \in [-5.33, -2.33] \)


\item \( \text{ The domain is all Real numbers except } x = a \text{ and } x = b, \text{ where } a \in [-12.67, -3.67] \text{ and } b \in [4.6, 13.6] \)


\item \( \text{ The domain is all Real numbers. } \)


\end{enumerate}

\textbf{General Comment:} The new domain is the intersection of the previous domains.
}
\litem{
Determine whether the function below is 1-1.
\[ f(x) = \sqrt{3 x - 20} \]

The solution is \( \text{yes} \), which is option C.\begin{enumerate}[label=\Alph*.]
\item \( \text{No, because the range of the function is not $(-\infty, \infty)$.} \)

Corresponds to believing 1-1 means the range is all Real numbers.
\item \( \text{No, because there is a $y$-value that goes to 2 different $x$-values.} \)

Corresponds to the Horizontal Line test, which this function passes.
\item \( \text{Yes, the function is 1-1.} \)

* This is the solution.
\item \( \text{No, because the domain of the function is not $(-\infty, \infty)$.} \)

Corresponds to believing 1-1 means the domain is all Real numbers.
\item \( \text{No, because there is an $x$-value that goes to 2 different $y$-values.} \)

Corresponds to the Vertical Line test, which checks if an expression is a function.
\end{enumerate}

\textbf{General Comment:} There are only two valid options: The function is 1-1 OR No because there is a $y$-value that goes to 2 different $x$-values.
}
\end{enumerate}

\end{document}