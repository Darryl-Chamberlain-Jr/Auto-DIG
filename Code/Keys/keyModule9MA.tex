41. A town has an initial population of 60. The town's population for the next 10 years is provided below. Which type of function would be most appropriate to model the town's population?
$$ 20, -20, -100, -260, -580, -1220, -2500, -5060, -10180, -20420 $$ 
The solution is $ \text{Exponential} $ 

\begin{enumerate}[label=\Alph*.] 
\item $ \text{Direct variation} $ 

 This suggests a growth faster than constant but slower than exponential. 
\item $ \text{Logarithmic} $ 

 This suggests the slowest of growths that we know. 
\item $ \text{Exponential} $ 

 This suggests the fastest of growths that we know. 
\item $ \text{Indirect variation} $ 

 This suggests a growth slower than constant but faster than logarithmic. 
\item $ \text{Linear} $ 

 This suggests a constant growth. You would be able to add or subtract the same amount year-to-year if this is the correct answer. 
\end{enumerate} 
 
General Comments: We are trying to compare the growth rate of the population. Growth rates can be characterized from slowest to fastest as: logarithmic, indirect, linear, direct, exponential. The best way to approach this is to first compare it to linear (is it linear, faster than linear, or slower than linear)? If faster, is it as fast as exponential? If slower, is it as slow as logarithmic?

-----------------------------------------------

