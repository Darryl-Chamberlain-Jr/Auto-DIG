\documentclass[14pt]{extbook}
\usepackage{multicol, enumerate, enumitem, hyperref, color, soul, setspace, parskip, fancyhdr} %General Packages
\usepackage{amssymb, amsthm, amsmath, bbm, latexsym, units, mathtools} %Math Packages
\everymath{\displaystyle} %All math in Display Style
% Packages with additional options
\usepackage[headsep=0.5cm,headheight=12pt, left=1 in,right= 1 in,top= 1 in,bottom= 1 in]{geometry}
\usepackage[usenames,dvipsnames]{xcolor}
\usepackage{dashrule}  % Package to use the command below to create lines between items
\newcommand{\litem}[1]{\item#1\hspace*{-1cm}\rule{\textwidth}{0.4pt}}
\pagestyle{fancy}
\lhead{Progress Quiz 8}
\chead{}
\rhead{Version A}
\lfoot{4553-3922}
\cfoot{}
\rfoot{Fall 2020}
\begin{document}

\begin{enumerate}
\litem{
Perform the division below. Then, find the intervals that correspond to the quotient in the form $ax^2+bx+c$ and remainder $r$.\[ \frac{25x^{3} +130 x^{2} +13 x -57}{x + 5} \]\begin{enumerate}[label=\Alph*.]
\item \( a \in [22, 26], \text{   } b \in [4, 6], \text{   } c \in [-12, -10], \text{   and   } r \in [-2, 10]. \)
\item \( a \in [-126, -120], \text{   } b \in [754, 761], \text{   } c \in [-3762, -3759], \text{   and   } r \in [18747, 18755]. \)
\item \( a \in [22, 26], \text{   } b \in [-26, -16], \text{   } c \in [133, 135], \text{   and   } r \in [-857, -852]. \)
\item \( a \in [-126, -120], \text{   } b \in [-499, -492], \text{   } c \in [-2462, -2455], \text{   and   } r \in [-12367, -12364]. \)
\item \( a \in [22, 26], \text{   } b \in [255, 262], \text{   } c \in [1283, 1293], \text{   and   } r \in [6381, 6384]. \)

\end{enumerate} }
\litem{
What are the \textit{possible Rational} roots of the polynomial below?\[ f(x) = 7x^{3} +2 x^{2} +4 x + 5 \]\begin{enumerate}[label=\Alph*.]
\item \( \text{ All combinations of: }\frac{\pm 1,\pm 5}{\pm 1,\pm 7} \)
\item \( \text{ All combinations of: }\frac{\pm 1,\pm 7}{\pm 1,\pm 5} \)
\item \( \pm 1,\pm 5 \)
\item \( \pm 1,\pm 7 \)
\item \( \text{ There is no formula or theorem that tells us all possible Rational roots.} \)

\end{enumerate} }
\litem{
Perform the division below. Then, find the intervals that correspond to the quotient in the form $ax^2+bx+c$ and remainder $r$.\[ \frac{10x^{3} -35 x^{2} + 49}{x -3} \]\begin{enumerate}[label=\Alph*.]
\item \( a \in [29, 42], b \in [51, 57], c \in [165, 167], \text{ and } r \in [544, 546]. \)
\item \( a \in [10, 16], b \in [-5, -4], c \in [-21, -9], \text{ and } r \in [2, 9]. \)
\item \( a \in [29, 42], b \in [-126, -123], c \in [374, 379], \text{ and } r \in [-1077, -1073]. \)
\item \( a \in [10, 16], b \in [-66, -59], c \in [194, 196], \text{ and } r \in [-536, -532]. \)
\item \( a \in [10, 16], b \in [-17, -9], c \in [-30, -26], \text{ and } r \in [-17, -9]. \)

\end{enumerate} }
\litem{
What are the \textit{possible Integer} roots of the polynomial below?\[ f(x) = 2x^{3} +3 x^{2} +2 x + 4 \]\begin{enumerate}[label=\Alph*.]
\item \( \text{ All combinations of: }\frac{\pm 1,\pm 2,\pm 4}{\pm 1,\pm 2} \)
\item \( \pm 1,\pm 2,\pm 4 \)
\item \( \pm 1,\pm 2 \)
\item \( \text{ All combinations of: }\frac{\pm 1,\pm 2}{\pm 1,\pm 2,\pm 4} \)
\item \( \text{There is no formula or theorem that tells us all possible Integer roots.} \)

\end{enumerate} }
\litem{
Perform the division below. Then, find the intervals that correspond to the quotient in the form $ax^2+bx+c$ and remainder $r$.\[ \frac{15x^{3} -65 x^{2} -75 x + 121}{x -5} \]\begin{enumerate}[label=\Alph*.]
\item \( a \in [13, 18], \text{   } b \in [-7, -4], \text{   } c \in [-99, -90], \text{   and   } r \in [-260, -254]. \)
\item \( a \in [13, 18], \text{   } b \in [7, 15], \text{   } c \in [-33, -23], \text{   and   } r \in [-4, 1]. \)
\item \( a \in [74, 80], \text{   } b \in [310, 316], \text{   } c \in [1470, 1476], \text{   and   } r \in [7496, 7497]. \)
\item \( a \in [13, 18], \text{   } b \in [-146, -139], \text{   } c \in [621, 627], \text{   and   } r \in [-3008, -2998]. \)
\item \( a \in [74, 80], \text{   } b \in [-444, -437], \text{   } c \in [2125, 2127], \text{   and   } r \in [-10506, -10500]. \)

\end{enumerate} }
\litem{
Perform the division below. Then, find the intervals that correspond to the quotient in the form $ax^2+bx+c$ and remainder $r$.\[ \frac{4x^{3} -49 x -56}{x -4} \]\begin{enumerate}[label=\Alph*.]
\item \( a \in [2, 7], b \in [14, 19], c \in [10, 19], \text{ and } r \in [-1, 7]. \)
\item \( a \in [2, 7], b \in [4, 14], c \in [-14, -10], \text{ and } r \in [-98, -92]. \)
\item \( a \in [12, 17], b \in [-66, -57], c \in [199, 210], \text{ and } r \in [-886, -880]. \)
\item \( a \in [2, 7], b \in [-17, -13], c \in [10, 19], \text{ and } r \in [-121, -114]. \)
\item \( a \in [12, 17], b \in [59, 67], c \in [199, 210], \text{ and } r \in [768, 773]. \)

\end{enumerate} }
\litem{
Factor the polynomial below completely. Then, choose the intervals the zeros of the polynomial belong to, where $z_1 \leq z_2 \leq z_3$. \textit{To make the problem easier, all zeros are between -5 and 5.}\[ f(x) = 20x^{3} -91 x^{2} -65 x + 100 \]\begin{enumerate}[label=\Alph*.]
\item \( z_1 \in [-1.7, -1], \text{   }  z_2 \in [0.77, 0.91], \text{   and   } z_3 \in [4.87, 5.26] \)
\item \( z_1 \in [-1.1, 0.3], \text{   }  z_2 \in [1.15, 1.8], \text{   and   } z_3 \in [4.87, 5.26] \)
\item \( z_1 \in [-5.1, -4.8], \text{   }  z_2 \in [-1.11, -0.79], \text{   and   } z_3 \in [1.18, 1.55] \)
\item \( z_1 \in [-5.1, -4.8], \text{   }  z_2 \in [-1.42, -0.83], \text{   and   } z_3 \in [0.64, 1.05] \)
\item \( z_1 \in [-5.1, -4.8], \text{   }  z_2 \in [-0.25, 0.24], \text{   and   } z_3 \in [4.87, 5.26] \)

\end{enumerate} }
\litem{
Factor the polynomial below completely. Then, choose the intervals the zeros of the polynomial belong to, where $z_1 \leq z_2 \leq z_3$. \textit{To make the problem easier, all zeros are between -5 and 5.}\[ f(x) = 6x^{3} -31 x^{2} +48 x -20 \]\begin{enumerate}[label=\Alph*.]
\item \( z_1 \in [-2.54, -2.1], \text{   }  z_2 \in [-2.51, -1.6], \text{   and   } z_3 \in [-0.7, -0.65] \)
\item \( z_1 \in [0.46, 0.88], \text{   }  z_2 \in [1.75, 2.24], \text{   and   } z_3 \in [2.47, 2.61] \)
\item \( z_1 \in [-2.08, -1.81], \text{   }  z_2 \in [-1.95, -1.48], \text{   and   } z_3 \in [-0.46, -0.34] \)
\item \( z_1 \in [-5.74, -4.33], \text{   }  z_2 \in [-2.51, -1.6], \text{   and   } z_3 \in [-0.37, -0.26] \)
\item \( z_1 \in [0.22, 0.57], \text{   }  z_2 \in [1.2, 1.91], \text{   and   } z_3 \in [1.84, 2.03] \)

\end{enumerate} }
\litem{
Factor the polynomial below completely, knowing that $x+4$ is a factor. Then, choose the intervals the zeros of the polynomial belong to, where $z_1 \leq z_2 \leq z_3 \leq z_4$. \textit{To make the problem easier, all zeros are between -5 and 5.}\[ f(x) = 25x^{4} +220 x^{3} +449 x^{2} -154 x -120 \]\begin{enumerate}[label=\Alph*.]
\item \( z_1 \in [-3.57, -2.48], \text{   }  z_2 \in [-0.56, 0.27], z_3 \in [3, 5], \text{   and   } z_4 \in [4.82, 5.61] \)
\item \( z_1 \in [-5.6, -4.25], \text{   }  z_2 \in [-4.15, -3.93], z_3 \in [-0.4, 0.6], \text{   and   } z_4 \in [0.43, 1.42] \)
\item \( z_1 \in [-2.04, -1.41], \text{   }  z_2 \in [2.41, 3.13], z_3 \in [3, 5], \text{   and   } z_4 \in [4.82, 5.61] \)
\item \( z_1 \in [-5.6, -4.25], \text{   }  z_2 \in [-4.15, -3.93], z_3 \in [-4.5, -0.5], \text{   and   } z_4 \in [1.58, 2.04] \)
\item \( z_1 \in [-0.97, -0.33], \text{   }  z_2 \in [0.15, 0.8], z_3 \in [3, 5], \text{   and   } z_4 \in [4.82, 5.61] \)

\end{enumerate} }
\litem{
Factor the polynomial below completely, knowing that $x+5$ is a factor. Then, choose the intervals the zeros of the polynomial belong to, where $z_1 \leq z_2 \leq z_3 \leq z_4$. \textit{To make the problem easier, all zeros are between -5 and 5.}\[ f(x) = 15x^{4} -16 x^{3} -321 x^{2} +630 x -200 \]\begin{enumerate}[label=\Alph*.]
\item \( z_1 \in [-6.4, -4.2], \text{   }  z_2 \in [0.58, 1.03], z_3 \in [2.2, 2.55], \text{   and   } z_4 \in [3.73, 4.83] \)
\item \( z_1 \in [-6.4, -4.2], \text{   }  z_2 \in [0.19, 0.4], z_3 \in [1.57, 1.71], \text{   and   } z_4 \in [3.73, 4.83] \)
\item \( z_1 \in [-4.8, -3.3], \text{   }  z_2 \in [-2.88, -2.32], z_3 \in [-0.76, -0.6], \text{   and   } z_4 \in [4.84, 5.6] \)
\item \( z_1 \in [-4.8, -3.3], \text{   }  z_2 \in [-1.76, -1.38], z_3 \in [-0.44, -0.4], \text{   and   } z_4 \in [4.84, 5.6] \)
\item \( z_1 \in [-6.4, -4.2], \text{   }  z_2 \in [-4.09, -3.84], z_3 \in [-0.31, -0.01], \text{   and   } z_4 \in [4.84, 5.6] \)

\end{enumerate} }
\end{enumerate}

\end{document}