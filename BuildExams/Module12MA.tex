\documentclass[14pt]{extbook}
\usepackage{multicol, enumerate, enumitem, hyperref, color, soul, setspace, parskip, fancyhdr} %General Packages
\usepackage{amssymb, amsthm, amsmath, bbm, latexsym, units, mathtools} %Math Packages
\everymath{\displaystyle} %All math in Display Style
% Packages with additional options
\usepackage[headsep=0.5cm,headheight=12pt, left=1 in,right= 1 in,top= 1 in,bottom= 1 in]{geometry}
\usepackage[usenames,dvipsnames]{xcolor}
\usepackage{dashrule}  % Package to use the command below to create lines between items
\newcommand{\litem}[1]{\item#1\hspace*{-1cm}\rule{\textwidth}{0.4pt}}
\pagestyle{fancy}
\lhead{Progress Quiz 8}
\chead{}
\rhead{Version A}
\lfoot{4553-3922}
\cfoot{}
\rfoot{Fall 2020}
\begin{document}

\begin{enumerate}
\litem{
Solve the modeling problem below, if possible.
\begin{center}
    \textit{ In CHM2045L, Brittany created a 16 liter 20 percent solution of chemical $\chi$ using two different solution percentages of chemical $\chi$. When she went to write her lab report, she realized she forgot to write the amount of each solution she used! If she remembers she used 19 percent and 43 percent solutions, what was the amount she used of the 43 percent solution? }
\end{center}
\begin{enumerate}[label=\Alph*.]
\item \( 0.67 \)
\item \( 5.90 \)
\item \( 8.00 \)
\item \( 15.33 \)
\item \( \text{There is not enough information to solve the problem.} \)

\end{enumerate} }
\litem{
Determine the appropriate model for the graph of points below.
\begin{center}
    \includegraphics[width=0.5\textwidth]{../Figures/identifyModelGraph12A.png}
\end{center}
\begin{enumerate}[label=\Alph*.]
\item \( \text{Linear model} \)
\item \( \text{Logarithmic model} \)
\item \( \text{Exponential model} \)
\item \( \text{Non-linear Power model} \)
\item \( \text{None of the above} \)

\end{enumerate} }
\litem{
For the information below, construct a linear model that describes the total time $T$ spent on the path in terms of the distance of a particular part of the path \textit{if we know that the time spent on each path was equal}.
\begin{center}
    \textit{ A bicyclist is training for a race on a hilly path. Their bike keeps track of their speed at any time, but not the distance traveled. Their speed traveling up a hill is 2 mph, 7 mph when traveling down a hill, and 5 mph when traveling along a flat portion. }
\end{center}
\begin{enumerate}[label=\Alph*.]
\item \( 14.000 D \)
\item \( 0.843 D \)
\item \( 70.000 D \)
\item \( \text{The model can be found with the information provided, but isn't options 1-3.} \)
\item \( \text{The model cannot be found with the information provided.} \)

\end{enumerate} }
\litem{
Determine the appropriate model for the graph of points below.
\begin{center}
    \includegraphics[width=0.5\textwidth]{../Figures/identifyModelGraph12CopyA.png}
\end{center}
\begin{enumerate}[label=\Alph*.]
\item \( \text{Exponential model} \)
\item \( \text{Logarithmic model} \)
\item \( \text{Linear model} \)
\item \( \text{Non-linear Power model} \)
\item \( \text{None of the above} \)

\end{enumerate} }
\litem{
For the scenario below, model the rate of vibration (cm/s) of the string in terms of the length of the string. Then determine the variation constant $k$ of the model (if possible). The constant should be in terms of cm and s.
\begin{center}
    \textit{ The rate of vibration of a string under constant tension varies based on the type of string and the length of the string. The rate of vibration of string $\omega$ increases as the square length of the string decreases. For example, when string $\omega$ is 5 mm long, the rate of vibration is 40 cm/s. }
\end{center}
\begin{enumerate}[label=\Alph*.]
\item \( k = 160.00 \)
\item \( k = 10.00 \)
\item \( k = 1000.00 \)
\item \( k = 1.60 \)
\item \( \text{None of the above.} \)

\end{enumerate} }
\litem{
For the scenario below, use the model for the volume of a cylinder as $V = \pi r^2 h$.
\begin{center}
    \textit{ Pringles wants to add 22 \text{percent} more chips to their cylinder cans and minimize the design change of their cans. They've decided that the best way to minimize the design change is to increase the radius and height by the same percentage. What should this increase be? }
\end{center}
\begin{enumerate}[label=\Alph*.]
\item \( \text{About } 7 \text{ percent} \)
\item \( \text{About } 3 \text{ percent} \)
\item \( \text{About } 10 \text{ percent} \)
\item \( \text{About } 11 \text{ percent} \)
\item \( \text{None of the above} \)

\end{enumerate} }
\litem{
Solve the modeling problem below, if possible.
\begin{center}
    \textit{ A new virus is spreading throughout the world. There were initially 6 many cases reported, but the number of confirmed cases has tripled every 4 days. How long will it be until there are at least 10000 confirmed cases? }
\end{center}
\begin{enumerate}[label=\Alph*.]
\item \( \text{About } 30 \text{ days} \)
\item \( \text{About } 28 \text{ days} \)
\item \( \text{About } 13 \text{ days} \)
\item \( \text{About } 14 \text{ days} \)
\item \( \text{There is not enough information to solve the problem.} \)

\end{enumerate} }
\litem{
Solve the modeling problem below, if possible.
\begin{center}
    \textit{ A new virus is spreading throughout the world. There were initially 4 many cases reported, but the number of confirmed cases has doubled every 5 days. How long will it be until there are at least 1000000 confirmed cases? }
\end{center}
\begin{enumerate}[label=\Alph*.]
\item \( \text{About } 63 \text{ days} \)
\item \( \text{About } 90 \text{ days} \)
\item \( \text{About } 34 \text{ days} \)
\item \( \text{About } 29 \text{ days} \)
\item \( \text{There is not enough information to solve the problem.} \)

\end{enumerate} }
\litem{
For the scenario below, use the model for the volume of a cylinder as $V = \pi r^2 h$.
\begin{center}
    \textit{ Pringles wants to add 44 \text{percent} more chips to their cylinder cans and minimize the design change of their cans. They've decided that the best way to minimize the design change is to increase the radius and height by the same percentage. What should this increase be? }
\end{center}
\begin{enumerate}[label=\Alph*.]
\item \( \text{About } 22 \text{ percent} \)
\item \( \text{About } 20 \text{ percent} \)
\item \( \text{About } 13 \text{ percent} \)
\item \( \text{About } 4 \text{ percent} \)
\item \( \text{None of the above} \)

\end{enumerate} }
\litem{
Solve the modeling problem below, if possible.
\begin{center}
    \textit{ In CHM2045L, Brittany created a 26 liter 14 percent solution of chemical $\chi$ using two different solution percentages of chemical $\chi$. When she went to write her lab report, she realized she forgot to write the amount of each solution she used! If she remembers she used 9 percent and 32 percent solutions, what was the amount she used of the 32 percent solution? }
\end{center}
\begin{enumerate}[label=\Alph*.]
\item \( 20.35 \)
\item \( 13.96 \)
\item \( 5.65 \)
\item \( 13.00 \)
\item \( \text{There is not enough information to solve the problem.} \)

\end{enumerate} }
\end{enumerate}

\end{document}