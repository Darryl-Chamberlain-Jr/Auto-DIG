\documentclass[14pt]{extbook}
\usepackage{multicol, enumerate, enumitem, hyperref, color, soul, setspace, parskip, fancyhdr} %General Packages
\usepackage{amssymb, amsthm, amsmath, latexsym, units, mathtools} %Math Packages
\everymath{\displaystyle} %All math in Display Style
% Packages with additional options
\usepackage[headsep=0.5cm,headheight=12pt, left=1 in,right= 1 in,top= 1 in,bottom= 1 in]{geometry}
\usepackage[usenames,dvipsnames]{xcolor}
\usepackage{dashrule}  % Package to use the command below to create lines between items
\newcommand{\litem}[1]{\item#1\hspace*{-1cm}\rule{\textwidth}{0.4pt}}
\pagestyle{fancy}
\lhead{Progress Quiz 6}
\chead{}
\rhead{Version A}
\lfoot{9689-6866}
\cfoot{}
\rfoot{Spring 2021}
\begin{document}

\begin{enumerate}
\litem{
Using the situation below, construct a linear model that describes the cost of the coffee beans $C(h)$ in terms of the weight of the low-quality coffee beans $h$.
\begin{center}
    \textit{ Veronica needs to prepare 90 of blended coffee beans selling for \$5.63 per pound. She has a high-quality bean that sells for \$6.81 a pound and a low-quality bean that sells for \$3.85 a pound. }
\end{center}
\begin{enumerate}[label=\Alph*.]
\item \( C(h) = -2.96 h + 612.90 \)
\item \( C(h) = 2.96 h + 346.50 \)
\item \( C(h) = 3.85 h \)
\item \( C(h) = 5.33 h \)
\item \( \text{None of the above.} \)

\end{enumerate} }
\litem{
A town has an initial population of 50000. The town's population for the next 9 years is provided below. Which type of function would be most appropriate to model the town's population?

\begin{tabular}{c|c|c|c|c|c|c|c|c|c}
\textbf{Year} &1 &2 &3 &4 &5 &6 &7 &8 &9\tabularnewline \hline
\textbf{Pop} &49977 &49965 &49937 &49925 &49897 &49885 &49857 &49845 &49817\end{tabular}\begin{enumerate}[label=\Alph*.]
\item \( \text{Linear} \)
\item \( \text{Non-Linear Power} \)
\item \( \text{Exponential} \)
\item \( \text{Logarithmic} \)
\item \( \text{None of the above} \)

\end{enumerate} }
\litem{
For the information provided below, construct a linear model that describes her total income, $I$, as a function of the number of months, $x$ she is at UF.
\begin{center}
    \textit{ Aubrey is a college student going into her first year at UF. She will receive Bright Futures, which covers her tuition plus a \$800 educational expense each year. Before college, Aubrey saved up \$9000. She knows she will need to pay \$1200 in rent a month, \$80 for food a week, and \$64 in other weekly expenses. }
\end{center}
\begin{enumerate}[label=\Alph*.]
\item \( I(x) = 1344 \)
\item \( I(x) = 1344 x \)
\item \( I(x) = 1776 x \)
\item \( I(x) = 1776 \)
\item \( \text{None of the above.} \)

\end{enumerate} }
\litem{
A town has an initial population of 70000. The town's population for the next 9 years is provided below. Which type of function would be most appropriate to model the town's population?

\begin{tabular}{c|c|c|c|c|c|c|c|c|c}
\textbf{Year} &1 &2 &3 &4 &5 &6 &7 &8 &9\tabularnewline \hline
\textbf{Pop} &70000 &69965 &69945 &69930 &69919 &69910 &69902 &69896 &69890\end{tabular}\begin{enumerate}[label=\Alph*.]
\item \( \text{Logarithmic} \)
\item \( \text{Exponential} \)
\item \( \text{Linear} \)
\item \( \text{Non-Linear Power} \)
\item \( \text{None of the above} \)

\end{enumerate} }
\litem{
What is the \textbf{best} way to describe the domain of the scenario below?
\begin{center}
    \textit{ Chemists commonly create a solution by mixing two products of differing concentrations together. A 10\% and 30\% solution can make an acid solution of some value between these, such as a 24\% acid solution. The chemist wants to make differing solution percentages of 7 liters each. }
\end{center}
\begin{enumerate}[label=\Alph*.]
\item \( \text{There is no restricted domain in this scenario} \)
\item \( \text{Proper subset of the Real numbers} \)
\item \( \text{Subset of the Integers} \)
\item \( \text{Subset of the Rational numbers} \)
\item \( \text{Subset of the Natural numbers} \)

\end{enumerate} }
\litem{
For the information provided below, construct a linear model that describes her total income, $I$, as a function of the number of months, $x$ she is at UF.
\begin{center}
    \textit{ Aubrey is a college student going into her first year at UF. She will receive Bright Futures, which covers her tuition plus a \$400 educational expense each year. Before college, Aubrey saved up \$6000. She knows she will need to pay \$1100 in rent a month, \$70 for food a week, and \$56 in other weekly expenses. }
\end{center}
\begin{enumerate}[label=\Alph*.]
\item \( I(x) = 6400 \)
\item \( I(x) = 400 x + 6000 \)
\item \( I(x) = 6000 x + 400 \)
\item \( I(x) = 6400 x \)
\item \( \text{None of the above.} \)

\end{enumerate} }
\litem{
For the information provided below, construct a linear model that describes the total distance of the path, $D$, in terms of the time spent on a particular path \textit{if we know that all parts of the path are equal length}.
\begin{center}
    \textit{ A bicyclist is training for a race on a hilly path. Their bike keeps track of their speed at any time, but not the distance traveled. Their speed traveling up a hill is 6 mph, 10 mph when traveling down a hill, and 8 mph when traveling along a flat portion. }
\end{center}
\begin{enumerate}[label=\Alph*.]
\item \( 24 t \)
\item \( 0.392 t \)
\item \( 480 t \)
\item \( \text{The model can be found with the information provided, but isn't options 1-3.} \)
\item \( \text{The model cannot be found with the information provided.} \)

\end{enumerate} }
\litem{
What is the \textbf{best} way to describe the domain of the scenario below?
\begin{center}
    \textit{ Hannah plans to pay off a no-interest loan from her parents. Her loan balance is \$1,000. She plans to pay \$35 at the end of every week until her balance is \$0. How many weeks will it be until she has paid off her loan? }
\end{center}
\begin{enumerate}[label=\Alph*.]
\item \( \text{There is no restricted domain in this scenario} \)
\item \( \text{Subset of the Natural numbers} \)
\item \( \text{Subset of the Rational numbers} \)
\item \( \text{Proper subset of the Real numbers} \)
\item \( \text{Subset of the Integers} \)

\end{enumerate} }
\litem{
For the information below, construct a linear model that describes the total time $T$ spent on the path in terms of the distance of a particular part of the path \textit{if we know that all parts of the path are equal length}.
\begin{center}
    \textit{ A bicyclist is training for a race on a hilly path. Their bike keeps track of their speed at any time, but not the distance traveled. Their speed traveling up a hill is 4 mph, 11 mph when traveling down a hill, and 7 mph when traveling along a flat portion. }
\end{center}
\begin{enumerate}[label=\Alph*.]
\item \( 0.484 D \)
\item \( 308.000 D \)
\item \( 22.000 D \)
\item \( \text{The model can be found with the information provided, but isn't options 1-3.} \)
\item \( \text{The model cannot be found with the information provided.} \)

\end{enumerate} }
\litem{
Using the situation below, construct a linear model that describes the cost of the coffee beans $C(h)$ in terms of the weight of the high-quality coffee beans $h$.
\begin{center}
    \textit{ Veronica needs to prepare 200 of blended coffee beans selling for \$3.66 per pound. She has a high-quality bean that sells for \$4.31 a pound and a low-quality bean that sells for \$2.32 a pound. }
\end{center}
\begin{enumerate}[label=\Alph*.]
\item \( C(h) = 4.31 h \)
\item \( C(h) = 3.31 h \)
\item \( C(h) = 1.99 h + 464.00 \)
\item \( C(h) = -1.99 h + 862.00 \)
\item \( \text{None of the above.} \)

\end{enumerate} }
\end{enumerate}

\end{document}