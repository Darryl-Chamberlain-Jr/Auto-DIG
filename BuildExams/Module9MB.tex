\documentclass[14pt]{extbook}
\usepackage{multicol, enumerate, enumitem, hyperref, color, soul, setspace, parskip, fancyhdr} %General Packages
\usepackage{amssymb, amsthm, amsmath, latexsym, units, mathtools} %Math Packages
\everymath{\displaystyle} %All math in Display Style
% Packages with additional options
\usepackage[headsep=0.5cm,headheight=12pt, left=1 in,right= 1 in,top= 1 in,bottom= 1 in]{geometry}
\usepackage[usenames,dvipsnames]{xcolor}
\usepackage{dashrule}  % Package to use the command below to create lines between items
\newcommand{\litem}[1]{\item#1\hspace*{-1cm}\rule{\textwidth}{0.4pt}}
\pagestyle{fancy}
\lhead{Makeup Progress Quiz 3}
\chead{}
\rhead{Version B}
\lfoot{1648-1753}
\cfoot{}
\rfoot{Summer C 2021}
\begin{document}

\begin{enumerate}
\litem{
For the information below, construct a linear model that describes the total time $T$ spent on the path in terms of the distance of a particular part of the path \textit{if we know that all parts of the path are equal length}.
\begin{center}
    \textit{ A bicyclist is training for a race on a hilly path. Their bike keeps track of their speed at any time, but not the distance traveled. Their speed traveling up a hill is 6 mph, 11 mph when traveling down a hill, and 8 mph when traveling along a flat portion. }
\end{center}
\begin{enumerate}[label=\Alph*.]
\item \( 528.000 D \)
\item \( 0.383 D \)
\item \( 25.000 D \)
\item \( \text{The model can be found with the information provided, but isn't options 1-3.} \)
\item \( \text{The model cannot be found with the information provided.} \)

\end{enumerate} }
\litem{
For the information provided below, construct a linear model that describes the total distance of the path, $D$, in terms of the time spent on a particular path \textit{if we know that all parts of the path are equal length}.
\begin{center}
    \textit{ A bicyclist is training for a race on a hilly path. Their bike keeps track of their speed at any time, but not the distance traveled. Their speed traveling up a hill is 2 mph, 7 mph when traveling down a hill, and 5 mph when traveling along a flat portion. }
\end{center}
\begin{enumerate}[label=\Alph*.]
\item \( 70 t \)
\item \( 0.843 t \)
\item \( 14 t \)
\item \( \text{The model can be found with the information provided, but isn't options 1-3.} \)
\item \( \text{The model cannot be found with the information provided.} \)

\end{enumerate} }
\litem{
A town has an initial population of 100000. The town's population for the next 9 years is provided below. Which type of function would be most appropriate to model the town's population?

\begin{tabular}{c|c|c|c|c|c|c|c|c|c}
\textbf{Year} &1 &2 &3 &4 &5 &6 &7 &8 &9\tabularnewline \hline
\textbf{Pop} &99960 &99920 &99880 &99840 &99800 &99760 &99720 &99680 &99640\end{tabular}\begin{enumerate}[label=\Alph*.]
\item \( \text{Exponential} \)
\item \( \text{Logarithmic} \)
\item \( \text{Non-Linear Power} \)
\item \( \text{Linear} \)
\item \( \text{None of the above} \)

\end{enumerate} }
\litem{
What is the \textbf{best} way to describe the domain of the scenario below?
\begin{center}
    \textit{ Bridges on highways often have expansion joints, which are small gaps in the roadway between one bridge section and the next. The gaps are put there so the bridge will have room to expand when the weather gets hot. Assume the gap width varies constantly with the temperature. Suppose a bridge has a gap of 1.3 cm when the temperature is 22 degrees C and that the gap narrows to 0.9 cm when the temperature warms to 30 degrees C. }
\end{center}
\begin{enumerate}[label=\Alph*.]
\item \( \text{Subset of the Integers} \)
\item \( \text{Subset of the Natural numbers} \)
\item \( \text{Subset of the Rational numbers} \)
\item \( \text{There is no restricted domain in this scenario} \)
\item \( \text{Proper subset of the Real numbers} \)

\end{enumerate} }
\litem{
Using the situation below, construct a linear model that describes the cost of the coffee beans $C(h)$ in terms of the weight of the low-quality coffee beans $h$.
\begin{center}
    \textit{ Veronica needs to prepare 150 of blended coffee beans selling for \$3.66 per pound. She has a high-quality bean that sells for \$4.20 a pound and a low-quality bean that sells for \$2.43 a pound. }
\end{center}
\begin{enumerate}[label=\Alph*.]
\item \( C(h) = 1.77 h + 364.50 \)
\item \( C(h) = 3.32 h \)
\item \( C(h) = 2.43 h \)
\item \( C(h) = -1.77 h + 630.00 \)
\item \( \text{None of the above.} \)

\end{enumerate} }
\litem{
What is the \textbf{best} way to describe the domain of the scenario below?
\begin{center}
    \textit{ Hannah plans to pay off a no-interest loan from her parents. Her loan balance is \$1,000. She plans to pay \$35 at the end of every week until her balance is \$0. How many weeks will it be until she has paid off her loan? }
\end{center}
\begin{enumerate}[label=\Alph*.]
\item \( \text{Subset of the Natural numbers} \)
\item \( \text{There is no restricted domain in this scenario} \)
\item \( \text{Subset of the Rational numbers} \)
\item \( \text{Proper subset of the Real numbers} \)
\item \( \text{Subset of the Integers} \)

\end{enumerate} }
\litem{
For the information provided below, construct a linear model that describes her total income, $I$, as a function of the number of months, $x$ she is at UF.
\begin{center}
    \textit{ Aubrey is a college student going into her first year at UF. She will receive Bright Futures, which covers her tuition plus a \$800 educational expense each year. Before college, Aubrey saved up \$8000. She knows she will need to pay \$900 in rent a month, \$80 for food a week, and \$40 in other weekly expenses. }
\end{center}
\begin{enumerate}[label=\Alph*.]
\item \( I(x) = 8800 x \)
\item \( I(x) = 800 x + 8000 \)
\item \( I(x) = 8800 \)
\item \( I(x) = 8000 x + 800 \)
\item \( \text{None of the above.} \)

\end{enumerate} }
\litem{
Using the situation below, construct a linear model that describes the cost of the coffee beans $C(h)$ in terms of the weight of the high-quality coffee beans $h$.
\begin{center}
    \textit{ Veronica needs to prepare 100 of blended coffee beans selling for \$3.27 per pound. She has a high-quality bean that sells for \$3.77 a pound and a low-quality bean that sells for \$2.73 a pound. }
\end{center}
\begin{enumerate}[label=\Alph*.]
\item \( C(h) = 3.77 h \)
\item \( C(h) = -1.04 h + 377.00 \)
\item \( C(h) = 3.25 h \)
\item \( C(h) = 1.04 h + 273.00 \)
\item \( \text{None of the above.} \)

\end{enumerate} }
\litem{
A town has an initial population of 70000. The town's population for the next 9 years is provided below. Which type of function would be most appropriate to model the town's population?

\begin{tabular}{c|c|c|c|c|c|c|c|c|c}
\textbf{Year} &1 &2 &3 &4 &5 &6 &7 &8 &9\tabularnewline \hline
\textbf{Pop} &69970 &69940 &69910 &69880 &69850 &69820 &69790 &69760 &69730\end{tabular}\begin{enumerate}[label=\Alph*.]
\item \( \text{Non-Linear Power} \)
\item \( \text{Linear} \)
\item \( \text{Logarithmic} \)
\item \( \text{Exponential} \)
\item \( \text{None of the above} \)

\end{enumerate} }
\litem{
For the information provided below, construct a linear model that describes her total income, $I$, as a function of the number of months, $x$ she is at UF.
\begin{center}
    \textit{ Aubrey is a college student going into her first year at UF. She will receive Bright Futures, which covers her tuition plus a \$1000 educational expense each year. Before college, Aubrey saved up \$9000. She knows she will need to pay \$800 in rent a month, \$40 for food a week, and \$56 in other weekly expenses. }
\end{center}
\begin{enumerate}[label=\Alph*.]
\item \( I(x) = 9000 x + 1000 \)
\item \( I(x) = 10000 \)
\item \( I(x) = 1000 x + 9000 \)
\item \( I(x) = 10000 x \)
\item \( \text{None of the above.} \)

\end{enumerate} }
\end{enumerate}

\end{document}