\documentclass[14pt]{extbook}
\usepackage{multicol, enumerate, enumitem, hyperref, color, soul, setspace, parskip, fancyhdr} %General Packages
\usepackage{amssymb, amsthm, amsmath, latexsym, units, mathtools} %Math Packages
\everymath{\displaystyle} %All math in Display Style
% Packages with additional options
\usepackage[headsep=0.5cm,headheight=12pt, left=1 in,right= 1 in,top= 1 in,bottom= 1 in]{geometry}
\usepackage[usenames,dvipsnames]{xcolor}
\usepackage{dashrule}  % Package to use the command below to create lines between items
\newcommand{\litem}[1]{\item#1\hspace*{-1cm}\rule{\textwidth}{0.4pt}}
\pagestyle{fancy}
\lhead{Progress Quiz 6}
\chead{}
\rhead{Version B}
\lfoot{9689-6866}
\cfoot{}
\rfoot{Spring 2021}
\begin{document}

\begin{enumerate}
\litem{
Choose the interval below that $f$ composed with $g$ at $x=-1$ is in.\[ f(x) = -x^{3} -3 x^{2} -3 x \text{ and } g(x) = -x^{3} +2 x^{2} +4 x \]\begin{enumerate}[label=\Alph*.]
\item \( (f \circ g)(-1) \in [4.87, 5.74] \)
\item \( (f \circ g)(-1) \in [5.45, 7.54] \)
\item \( (f \circ g)(-1) \in [0.7, 2.2] \)
\item \( (f \circ g)(-1) \in [14.76, 15.37] \)
\item \( \text{It is not possible to compose the two functions.} \)

\end{enumerate} }
\litem{
Choose the interval below that $f$ composed with $g$ at $x=1$ is in.\[ f(x) = -2x^{3} -1 x^{2} -2 x + 4 \text{ and } g(x) = -2x^{3} -3 x^{2} +3 x -1 \]\begin{enumerate}[label=\Alph*.]
\item \( (f \circ g)(1) \in [-15, -13] \)
\item \( (f \circ g)(1) \in [54, 58] \)
\item \( (f \circ g)(1) \in [-7, -3] \)
\item \( (f \circ g)(1) \in [49, 54] \)
\item \( \text{It is not possible to compose the two functions.} \)

\end{enumerate} }
\litem{
Add the following functions, then choose the domain of the resulting function from the list below.\[ f(x) = 2x^{4} +5 x^{3} +2 x^{2} +5 x + 2 \text{ and } g(x) = \sqrt{-3x+4}  \]\begin{enumerate}[label=\Alph*.]
\item \( \text{ The domain is all Real numbers except } x = a, \text{ where } a \in [-7.33, -1.33] \)
\item \( \text{ The domain is all Real numbers greater than or equal to } x = a, \text{ where } a \in [0.33, 12.33] \)
\item \( \text{ The domain is all Real numbers less than or equal to } x = a, \text{ where } a \in [-1.67, 2.33] \)
\item \( \text{ The domain is all Real numbers except } x = a \text{ and } x = b, \text{ where } a \in [-6.6, -2.6] \text{ and } b \in [-7.4, -2.4] \)
\item \( \text{ The domain is all Real numbers. } \)

\end{enumerate} }
\litem{
Determine whether the function below is 1-1.\[ f(x) = 36 x^2 - 312 x + 676 \]\begin{enumerate}[label=\Alph*.]
\item \( \text{No, because the range of the function is not $(-\infty, \infty)$.} \)
\item \( \text{No, because there is an $x$-value that goes to 2 different $y$-values.} \)
\item \( \text{Yes, the function is 1-1.} \)
\item \( \text{No, because the domain of the function is not $(-\infty, \infty)$.} \)
\item \( \text{No, because there is a $y$-value that goes to 2 different $x$-values.} \)

\end{enumerate} }
\litem{
Find the inverse of the function below (if it exists). Then, evaluate the inverse at $x = 11$ and choose the interval that $f^{-1}(11)$ belongs to.\[ f(x) = 3 x^2 - 2 \]\begin{enumerate}[label=\Alph*.]
\item \( f^{-1}(11) \in [4.88, 5.45] \)
\item \( f^{-1}(11) \in [8.06, 10.23] \)
\item \( f^{-1}(11) \in [1.93, 2.97] \)
\item \( f^{-1}(11) \in [1.57, 2.01] \)
\item \( \text{ The function is not invertible for all Real numbers. } \)

\end{enumerate} }
\litem{
Find the inverse of the function below (if it exists). Then, evaluate the inverse at $x = -10$ and choose the interval the $f^{-1}(-10)$ belongs to.\[ f(x) = \sqrt[3]{5 x + 4} \]\begin{enumerate}[label=\Alph*.]
\item \( f^{-1}(-10) \in [-199.57, -198.15] \)
\item \( f^{-1}(-10) \in [-201.01, -200.17] \)
\item \( f^{-1}(-10) \in [200.09, 202.09] \)
\item \( f^{-1}(-10) \in [198.75, 199.54] \)
\item \( \text{ The function is not invertible for all Real numbers. } \)

\end{enumerate} }
\litem{
Determine whether the function below is 1-1.\[ f(x) = 36 x^2 + 456 x + 1444 \]\begin{enumerate}[label=\Alph*.]
\item \( \text{No, because the domain of the function is not $(-\infty, \infty)$.} \)
\item \( \text{Yes, the function is 1-1.} \)
\item \( \text{No, because there is a $y$-value that goes to 2 different $x$-values.} \)
\item \( \text{No, because the range of the function is not $(-\infty, \infty)$.} \)
\item \( \text{No, because there is an $x$-value that goes to 2 different $y$-values.} \)

\end{enumerate} }
\litem{
Find the inverse of the function below. Then, evaluate the inverse at $x = 7$ and choose the interval that $f^{-1}(7)$ belongs to.\[ f(x) = e^{x+3}+5 \]\begin{enumerate}[label=\Alph*.]
\item \( f^{-1}(7) \in [6.2, 6.67] \)
\item \( f^{-1}(7) \in [-2.33, -2.09] \)
\item \( f^{-1}(7) \in [7.4, 7.53] \)
\item \( f^{-1}(7) \in [7.16, 7.41] \)
\item \( f^{-1}(7) \in [3.55, 3.84] \)

\end{enumerate} }
\litem{
Multiply the following functions, then choose the domain of the resulting function from the list below.\[ f(x) = \sqrt{4x-30}  \text{ and } g(x) = 5x^{2} +3 x + 7 \]\begin{enumerate}[label=\Alph*.]
\item \( \text{ The domain is all Real numbers greater than or equal to } x = a, \text{ where } a \in [6.5, 12.5] \)
\item \( \text{ The domain is all Real numbers less than or equal to } x = a, \text{ where } a \in [-9.75, -0.75] \)
\item \( \text{ The domain is all Real numbers except } x = a, \text{ where } a \in [-0.6, 8.4] \)
\item \( \text{ The domain is all Real numbers except } x = a \text{ and } x = b, \text{ where } a \in [4.33, 14.33] \text{ and } b \in [-8.67, -2.67] \)
\item \( \text{ The domain is all Real numbers. } \)

\end{enumerate} }
\litem{
Find the inverse of the function below. Then, evaluate the inverse at $x = 9$ and choose the interval that $f^{-1}(9)$ belongs to.\[ f(x) = e^{x-2}+4 \]\begin{enumerate}[label=\Alph*.]
\item \( f^{-1}(9) \in [5.76, 6.31] \)
\item \( f^{-1}(9) \in [6.42, 7.33] \)
\item \( f^{-1}(9) \in [-0.46, 0.18] \)
\item \( f^{-1}(9) \in [6.23, 6.49] \)
\item \( f^{-1}(9) \in [2.99, 4.53] \)

\end{enumerate} }
\end{enumerate}

\end{document}