\documentclass[14pt]{extbook}
\usepackage{multicol, enumerate, enumitem, hyperref, color, soul, setspace, parskip, fancyhdr} %General Packages
\usepackage{amssymb, amsthm, amsmath, latexsym, units, mathtools} %Math Packages
\everymath{\displaystyle} %All math in Display Style
% Packages with additional options
\usepackage[headsep=0.5cm,headheight=12pt, left=1 in,right= 1 in,top= 1 in,bottom= 1 in]{geometry}
\usepackage[usenames,dvipsnames]{xcolor}
\usepackage{dashrule}  % Package to use the command below to create lines between items
\newcommand{\litem}[1]{\item#1\hspace*{-1cm}\rule{\textwidth}{0.4pt}}
\pagestyle{fancy}
\lhead{Makeup Progress Quiz 3}
\chead{}
\rhead{Version B}
\lfoot{1648-1753}
\cfoot{}
\rfoot{Summer C 2021}
\begin{document}

\begin{enumerate}
\litem{
Determine whether the function below is 1-1.\[ f(x) = 36 x^2 - 252 x + 441 \]\begin{enumerate}[label=\Alph*.]
\item \( \text{No, because the domain of the function is not $(-\infty, \infty)$.} \)
\item \( \text{No, because there is an $x$-value that goes to 2 different $y$-values.} \)
\item \( \text{No, because there is a $y$-value that goes to 2 different $x$-values.} \)
\item \( \text{Yes, the function is 1-1.} \)
\item \( \text{No, because the range of the function is not $(-\infty, \infty)$.} \)

\end{enumerate} }
\litem{
Determine whether the function below is 1-1.\[ f(x) = (3 x + 19)^3 \]\begin{enumerate}[label=\Alph*.]
\item \( \text{No, because the range of the function is not $(-\infty, \infty)$.} \)
\item \( \text{Yes, the function is 1-1.} \)
\item \( \text{No, because there is a $y$-value that goes to 2 different $x$-values.} \)
\item \( \text{No, because there is an $x$-value that goes to 2 different $y$-values.} \)
\item \( \text{No, because the domain of the function is not $(-\infty, \infty)$.} \)

\end{enumerate} }
\litem{
Find the inverse of the function below (if it exists). Then, evaluate the inverse at $x = 13$ and choose the interval that $f^-1(13)$ belongs to.\[ f(x) = 3 x^2 - 2 \]\begin{enumerate}[label=\Alph*.]
\item \( f^{-1}(13) \in [2.17, 2.58] \)
\item \( f^{-1}(13) \in [4.98, 5.3] \)
\item \( f^{-1}(13) \in [1.51, 2.17] \)
\item \( f^{-1}(13) \in [3.23, 3.35] \)
\item \( \text{ The function is not invertible for all Real numbers. } \)

\end{enumerate} }
\litem{
Subtract the following functions, then choose the domain of the resulting function from the list below.\[ f(x) = x^{2} +5 x + 8 \text{ and } g(x) = \frac{3}{4x-21} \]\begin{enumerate}[label=\Alph*.]
\item \( \text{ The domain is all Real numbers less than or equal to } x = a, \text{ where } a \in [-0.25, 6.75] \)
\item \( \text{ The domain is all Real numbers greater than or equal to } x = a, \text{ where } a \in [4, 9] \)
\item \( \text{ The domain is all Real numbers except } x = a, \text{ where } a \in [4.25, 9.25] \)
\item \( \text{ The domain is all Real numbers except } x = a \text{ and } x = b, \text{ where } a \in [-5.4, -2.4] \text{ and } b \in [1.25, 9.25] \)
\item \( \text{ The domain is all Real numbers. } \)

\end{enumerate} }
\litem{
Choose the interval below that $f$ composed with $g$ at $x=-1$ is in.\[ f(x) = -3x^{3} -2 x^{2} -2 x -2 \text{ and } g(x) = -3x^{3} -2 x^{2} +4 x \]\begin{enumerate}[label=\Alph*.]
\item \( (f \circ g)(-1) \in [5, 14] \)
\item \( (f \circ g)(-1) \in [58, 65] \)
\item \( (f \circ g)(-1) \in [-3, 5] \)
\item \( (f \circ g)(-1) \in [65, 70] \)
\item \( \text{It is not possible to compose the two functions.} \)

\end{enumerate} }
\litem{
Find the inverse of the function below. Then, evaluate the inverse at $x = 8$ and choose the interval that $f^-1(8)$ belongs to.\[ f(x) = \ln{(x+3)}+3 \]\begin{enumerate}[label=\Alph*.]
\item \( f^{-1}(8) \in [151.41, 152.41] \)
\item \( f^{-1}(8) \in [59871.14, 59872.14] \)
\item \( f^{-1}(8) \in [151.41, 152.41] \)
\item \( f^{-1}(8) \in [141.41, 150.41] \)
\item \( f^{-1}(8) \in [59877.14, 59879.14] \)

\end{enumerate} }
\litem{
Choose the interval below that $f$ composed with $g$ at $x=-1$ is in.\[ f(x) = x^{3} -4 x^{2} -2 x + 1 \text{ and } g(x) = -3x^{3} -4 x^{2} +2 x \]\begin{enumerate}[label=\Alph*.]
\item \( (f \circ g)(-1) \in [-59, -51] \)
\item \( (f \circ g)(-1) \in [4, 10] \)
\item \( (f \circ g)(-1) \in [6, 16] \)
\item \( (f \circ g)(-1) \in [-66, -64] \)
\item \( \text{It is not possible to compose the two functions.} \)

\end{enumerate} }
\litem{
Multiply the following functions, then choose the domain of the resulting function from the list below.\[ f(x) = 4x^{4} +6 x^{3} +7 x^{2} +7 x + 8 \text{ and } g(x) = x^{2} +7 x + 1 \]\begin{enumerate}[label=\Alph*.]
\item \( \text{ The domain is all Real numbers greater than or equal to } x = a, \text{ where } a \in [2.25, 7.25] \)
\item \( \text{ The domain is all Real numbers less than or equal to } x = a, \text{ where } a \in [0.25, 8.25] \)
\item \( \text{ The domain is all Real numbers except } x = a, \text{ where } a \in [-5.75, -3.75] \)
\item \( \text{ The domain is all Real numbers except } x = a \text{ and } x = b, \text{ where } a \in [-10.33, -1.33] \text{ and } b \in [3.2, 6.2] \)
\item \( \text{ The domain is all Real numbers. } \)

\end{enumerate} }
\litem{
Find the inverse of the function below (if it exists). Then, evaluate the inverse at $x = -11$ and choose the interval that $f^-1(-11)$ belongs to.\[ f(x) = \sqrt[3]{2 x + 3} \]\begin{enumerate}[label=\Alph*.]
\item \( f^{-1}(-11) \in [-665, -662.1] \)
\item \( f^{-1}(-11) \in [-669.5, -665.9] \)
\item \( f^{-1}(-11) \in [663.5, 665.9] \)
\item \( f^{-1}(-11) \in [666.6, 667.4] \)
\item \( \text{ The function is not invertible for all Real numbers. } \)

\end{enumerate} }
\litem{
Find the inverse of the function below. Then, evaluate the inverse at $x = 9$ and choose the interval that $f^-1(9)$ belongs to.\[ f(x) = e^{x+3}-5 \]\begin{enumerate}[label=\Alph*.]
\item \( f^{-1}(9) \in [-3.85, -3.55] \)
\item \( f^{-1}(9) \in [-2.87, -2.39] \)
\item \( f^{-1}(9) \in [5.29, 5.97] \)
\item \( f^{-1}(9) \in [-0.68, -0.23] \)
\item \( f^{-1}(9) \in [-3.3, -3.08] \)

\end{enumerate} }
\end{enumerate}

\end{document}