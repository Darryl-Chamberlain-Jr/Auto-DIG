\documentclass[14pt]{extbook}
\usepackage{multicol, enumerate, enumitem, hyperref, color, soul, setspace, parskip, fancyhdr} %General Packages
\usepackage{amssymb, amsthm, amsmath, bbm, latexsym, units, mathtools} %Math Packages
\everymath{\displaystyle} %All math in Display Style
% Packages with additional options
\usepackage[headsep=0.5cm,headheight=12pt, left=1 in,right= 1 in,top= 1 in,bottom= 1 in]{geometry}
\usepackage[usenames,dvipsnames]{xcolor}
\usepackage{dashrule}  % Package to use the command below to create lines between items
\newcommand{\litem}[1]{\item#1\hspace*{-1cm}\rule{\textwidth}{0.4pt}}
\pagestyle{fancy}
\lhead{Progress Quiz 9}
\chead{}
\rhead{Version B}
\lfoot{8590-6105}
\cfoot{}
\rfoot{Fall 2020}
\begin{document}

\begin{enumerate}
\litem{
Subtract the following functions, then choose the domain of the resulting function from the list below.\[ f(x) = x^{4} +8 x^{3} +3 x^{2} +5 x + 1 \text{ and } g(x) = 7x^{4} +7 x^{3} +5 x^{2} + 3 \]\begin{enumerate}[label=\Alph*.]
\item \( \text{ The domain is all Real numbers greater than or equal to } x = a, \text{ where } a \in [-10, 1] \)
\item \( \text{ The domain is all Real numbers less than or equal to } x = a, \text{ where } a \in [-7.67, 1.33] \)
\item \( \text{ The domain is all Real numbers except } x = a, \text{ where } a \in [-11.2, -2.2] \)
\item \( \text{ The domain is all Real numbers except } x = a \text{ and } x = b, \text{ where } a \in [-7.6, -0.6] \text{ and } b \in [-10.67, 2.33] \)
\item \( \text{ The domain is all Real numbers. } \)

\end{enumerate} }
\litem{
Find the inverse of the function below (if it exists). Then, evaluate the inverse at $x = 12$ and choose the interval the $f^{-1}(12)$ belongs to.\[ f(x) = \sqrt[3]{4 x + 3} \]\begin{enumerate}[label=\Alph*.]
\item \( f^{-1}(12) \in [-431.95, -429.73] \)
\item \( f^{-1}(12) \in [431.69, 433.5] \)
\item \( f^{-1}(12) \in [-434.68, -431.75] \)
\item \( f^{-1}(12) \in [430.68, 431.77] \)
\item \( \text{ The function is not invertible for all Real numbers. } \)

\end{enumerate} }
\litem{
Determine whether the function below is 1-1.\[ f(x) = 36 x^2 - 252 x + 441 \]\begin{enumerate}[label=\Alph*.]
\item \( \text{No, because the domain of the function is not $(-\infty, \infty)$.} \)
\item \( \text{No, because the range of the function is not $(-\infty, \infty)$.} \)
\item \( \text{No, because there is a $y$-value that goes to 2 different $x$-values.} \)
\item \( \text{No, because there is an $x$-value that goes to 2 different $y$-values.} \)
\item \( \text{Yes, the function is 1-1.} \)

\end{enumerate} }
\litem{
Find the inverse of the function below (if it exists). Then, evaluate the inverse at $x = 14$ and choose the interval that $f^{-1}(14)$ belongs to.\[ f(x) = 4 x^2 + 2 \]\begin{enumerate}[label=\Alph*.]
\item \( f^{-1}(14) \in [1.54, 1.89] \)
\item \( f^{-1}(14) \in [1.9, 2.03] \)
\item \( f^{-1}(14) \in [5.68, 5.84] \)
\item \( f^{-1}(14) \in [3.34, 3.92] \)
\item \( \text{ The function is not invertible for all Real numbers. } \)

\end{enumerate} }
\litem{
Find the inverse of the function below. Then, evaluate the inverse at $x = 9$ and choose the interval that $f^{-1}(9)$ belongs to.\[ f(x) = e^{x+5}+5 \]\begin{enumerate}[label=\Alph*.]
\item \( f^{-1}(9) \in [5.8, 7] \)
\item \( f^{-1}(9) \in [6.9, 9.7] \)
\item \( f^{-1}(9) \in [5.8, 7] \)
\item \( f^{-1}(9) \in [6.9, 9.7] \)
\item \( f^{-1}(9) \in [-4.6, -3.5] \)

\end{enumerate} }
\litem{
Choose the interval below that $f$ composed with $g$ at $x=1$ is in.\[ f(x) = 3x^{3} +2 x^{2} -4 x + 1 \text{ and } g(x) = 3x^{3} -4 x^{2} +4 x -4 \]\begin{enumerate}[label=\Alph*.]
\item \( (f \circ g)(1) \in [14, 21] \)
\item \( (f \circ g)(1) \in [12, 13] \)
\item \( (f \circ g)(1) \in [4, 9] \)
\item \( (f \circ g)(1) \in [12, 13] \)
\item \( \text{It is not possible to compose the two functions.} \)

\end{enumerate} }
\litem{
Choose the interval below that $f$ composed with $g$ at $x=-1$ is in.\[ f(x) = -3x^{3} +4 x^{2} +4 x \text{ and } g(x) = -x^{3} -3 x^{2} -3 x -2 \]\begin{enumerate}[label=\Alph*.]
\item \( (f \circ g)(-1) \in [2, 5] \)
\item \( (f \circ g)(-1) \in [-67, -58] \)
\item \( (f \circ g)(-1) \in [-76, -70] \)
\item \( (f \circ g)(-1) \in [4, 13] \)
\item \( \text{It is not possible to compose the two functions.} \)

\end{enumerate} }
\litem{
Find the inverse of the function below. Then, evaluate the inverse at $x = 7$ and choose the interval that $f^{-1}(7)$ belongs to.\[ f(x) = \ln{(x-5)}-5 \]\begin{enumerate}[label=\Alph*.]
\item \( f^{-1}(7) \in [162747.79, 162753.79] \)
\item \( f^{-1}(7) \in [10.39, 14.39] \)
\item \( f^{-1}(7) \in [162757.79, 162762.79] \)
\item \( f^{-1}(7) \in [-1.61, 3.39] \)
\item \( f^{-1}(7) \in [162747.79, 162753.79] \)

\end{enumerate} }
\litem{
Subtract the following functions, then choose the domain of the resulting function from the list below.\[ f(x) = \frac{2}{5x+31} \text{ and } g(x) = 2x^{4} +8 x^{2} +8 x + 7 \]\begin{enumerate}[label=\Alph*.]
\item \( \text{ The domain is all Real numbers less than or equal to } x = a, \text{ where } a \in [-5.4, 0.6] \)
\item \( \text{ The domain is all Real numbers except } x = a, \text{ where } a \in [-9.2, -5.2] \)
\item \( \text{ The domain is all Real numbers greater than or equal to } x = a, \text{ where } a \in [2.5, 8.5] \)
\item \( \text{ The domain is all Real numbers except } x = a \text{ and } x = b, \text{ where } a \in [-10.2, -2.2] \text{ and } b \in [-5.8, -4.8] \)
\item \( \text{ The domain is all Real numbers. } \)

\end{enumerate} }
\litem{
Determine whether the function below is 1-1.\[ f(x) = \sqrt{-5 x + 19} \]\begin{enumerate}[label=\Alph*.]
\item \( \text{No, because the domain of the function is not $(-\infty, \infty)$.} \)
\item \( \text{No, because the range of the function is not $(-\infty, \infty)$.} \)
\item \( \text{No, because there is an $x$-value that goes to 2 different $y$-values.} \)
\item \( \text{Yes, the function is 1-1.} \)
\item \( \text{No, because there is a $y$-value that goes to 2 different $x$-values.} \)

\end{enumerate} }
\end{enumerate}

\end{document}