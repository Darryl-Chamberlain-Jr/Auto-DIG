\documentclass[14pt]{extbook}
\usepackage{multicol, enumerate, enumitem, hyperref, color, soul, setspace, parskip, fancyhdr} %General Packages
\usepackage{amssymb, amsthm, amsmath, bbm, latexsym, units, mathtools} %Math Packages
\everymath{\displaystyle} %All math in Display Style
% Packages with additional options
\usepackage[headsep=0.5cm,headheight=12pt, left=1 in,right= 1 in,top= 1 in,bottom= 1 in]{geometry}
\usepackage[usenames,dvipsnames]{xcolor}
\usepackage{dashrule}  % Package to use the command below to create lines between items
\newcommand{\litem}[1]{\item#1\hspace*{-1cm}\rule{\textwidth}{0.4pt}}
\pagestyle{fancy}
\lhead{Progress Quiz 8}
\chead{}
\rhead{Version C}
\lfoot{4553-3922}
\cfoot{}
\rfoot{Fall 2020}
\begin{document}

\begin{enumerate}
\litem{
Perform the division below. Then, find the intervals that correspond to the quotient in the form $ax^2+bx+c$ and remainder $r$.\[ \frac{10x^{3} +5 x^{2} -80 x -79}{x -3} \]\begin{enumerate}[label=\Alph*.]
\item \( a \in [8, 13], \text{   } b \in [-27, -24], \text{   } c \in [-7, -3], \text{   and   } r \in [-67, -59]. \)
\item \( a \in [8, 13], \text{   } b \in [33, 37], \text{   } c \in [18, 32], \text{   and   } r \in [-7, -3]. \)
\item \( a \in [26, 34], \text{   } b \in [-87, -79], \text{   } c \in [174, 178], \text{   and   } r \in [-608, -597]. \)
\item \( a \in [26, 34], \text{   } b \in [91, 99], \text{   } c \in [196, 207], \text{   and   } r \in [534, 543]. \)
\item \( a \in [8, 13], \text{   } b \in [24, 27], \text{   } c \in [-31, -27], \text{   and   } r \in [-143, -134]. \)

\end{enumerate} }
\litem{
What are the \textit{possible Integer} roots of the polynomial below?\[ f(x) = 3x^{4} +2 x^{3} +3 x^{2} +7 x + 5 \]\begin{enumerate}[label=\Alph*.]
\item \( \pm 1,\pm 3 \)
\item \( \text{ All combinations of: }\frac{\pm 1,\pm 3}{\pm 1,\pm 5} \)
\item \( \pm 1,\pm 5 \)
\item \( \text{ All combinations of: }\frac{\pm 1,\pm 5}{\pm 1,\pm 3} \)
\item \( \text{There is no formula or theorem that tells us all possible Integer roots.} \)

\end{enumerate} }
\litem{
Perform the division below. Then, find the intervals that correspond to the quotient in the form $ax^2+bx+c$ and remainder $r$.\[ \frac{16x^{3} -52 x^{2} + 33}{x -3} \]\begin{enumerate}[label=\Alph*.]
\item \( a \in [12, 25], b \in [-20, -18], c \in [-41, -36], \text{ and } r \in [-52, -45]. \)
\item \( a \in [47, 51], b \in [-199, -188], c \in [586, 591], \text{ and } r \in [-1736, -1727]. \)
\item \( a \in [47, 51], b \in [85, 98], c \in [275, 280], \text{ and } r \in [852, 865]. \)
\item \( a \in [12, 25], b \in [-9, -2], c \in [-14, -9], \text{ and } r \in [-6, -1]. \)
\item \( a \in [12, 25], b \in [-102, -93], c \in [296, 308], \text{ and } r \in [-868, -866]. \)

\end{enumerate} }
\litem{
What are the \textit{possible Integer} roots of the polynomial below?\[ f(x) = 3x^{2} +6 x + 2 \]\begin{enumerate}[label=\Alph*.]
\item \( \text{ All combinations of: }\frac{\pm 1,\pm 2}{\pm 1,\pm 3} \)
\item \( \text{ All combinations of: }\frac{\pm 1,\pm 3}{\pm 1,\pm 2} \)
\item \( \pm 1,\pm 3 \)
\item \( \pm 1,\pm 2 \)
\item \( \text{There is no formula or theorem that tells us all possible Integer roots.} \)

\end{enumerate} }
\litem{
Perform the division below. Then, find the intervals that correspond to the quotient in the form $ax^2+bx+c$ and remainder $r$.\[ \frac{8x^{3} +4 x^{2} -28 x -27}{x -2} \]\begin{enumerate}[label=\Alph*.]
\item \( a \in [3, 15], \text{   } b \in [-15, -10], \text{   } c \in [-4, 1], \text{   and   } r \in [-21, -18]. \)
\item \( a \in [3, 15], \text{   } b \in [8, 13], \text{   } c \in [-16, -15], \text{   and   } r \in [-51, -38]. \)
\item \( a \in [13, 20], \text{   } b \in [36, 39], \text{   } c \in [38, 48], \text{   and   } r \in [60, 67]. \)
\item \( a \in [3, 15], \text{   } b \in [17, 21], \text{   } c \in [9, 18], \text{   and   } r \in [-6, -2]. \)
\item \( a \in [13, 20], \text{   } b \in [-31, -24], \text{   } c \in [25, 36], \text{   and   } r \in [-86, -81]. \)

\end{enumerate} }
\litem{
Perform the division below. Then, find the intervals that correspond to the quotient in the form $ax^2+bx+c$ and remainder $r$.\[ \frac{8x^{3} -26 x^{2} + 13}{x -3} \]\begin{enumerate}[label=\Alph*.]
\item \( a \in [22, 30], b \in [45, 50], c \in [134, 140], \text{ and } r \in [425, 428]. \)
\item \( a \in [22, 30], b \in [-100, -94], c \in [293, 299], \text{ and } r \in [-870, -867]. \)
\item \( a \in [2, 11], b \in [-12, -5], c \in [-25, -15], \text{ and } r \in [-32, -26]. \)
\item \( a \in [2, 11], b \in [-52, -45], c \in [147, 152], \text{ and } r \in [-442, -429]. \)
\item \( a \in [2, 11], b \in [-7, 6], c \in [-6, -4], \text{ and } r \in [-8, 4]. \)

\end{enumerate} }
\litem{
Factor the polynomial below completely. Then, choose the intervals the zeros of the polynomial belong to, where $z_1 \leq z_2 \leq z_3$. \textit{To make the problem easier, all zeros are between -5 and 5.}\[ f(x) = 10x^{3} +49 x^{2} +68 x + 20 \]\begin{enumerate}[label=\Alph*.]
\item \( z_1 \in [-2.62, -2.31], \text{   }  z_2 \in [-3, -1], \text{   and   } z_3 \in [-0.4, 1.6] \)
\item \( z_1 \in [0.36, 0.41], \text{   }  z_2 \in [2, 3], \text{   and   } z_3 \in [2.5, 4.5] \)
\item \( z_1 \in [-2.62, -2.31], \text{   }  z_2 \in [-3, -1], \text{   and   } z_3 \in [-0.4, 1.6] \)
\item \( z_1 \in [0.36, 0.41], \text{   }  z_2 \in [2, 3], \text{   and   } z_3 \in [2.5, 4.5] \)
\item \( z_1 \in [0.18, 0.35], \text{   }  z_2 \in [2, 3], \text{   and   } z_3 \in [5, 8] \)

\end{enumerate} }
\litem{
Factor the polynomial below completely. Then, choose the intervals the zeros of the polynomial belong to, where $z_1 \leq z_2 \leq z_3$. \textit{To make the problem easier, all zeros are between -5 and 5.}\[ f(x) = 25x^{3} -50 x^{2} -9 x + 18 \]\begin{enumerate}[label=\Alph*.]
\item \( z_1 \in [-1.9, -1.2], \text{   }  z_2 \in [1.33, 2.15], \text{   and   } z_3 \in [1.9, 2.26] \)
\item \( z_1 \in [-2.3, -1.8], \text{   }  z_2 \in [-0.74, -0.57], \text{   and   } z_3 \in [0.47, 0.92] \)
\item \( z_1 \in [-1.1, 0.4], \text{   }  z_2 \in [0.5, 0.94], \text{   and   } z_3 \in [1.9, 2.26] \)
\item \( z_1 \in [-2.3, -1.8], \text{   }  z_2 \in [-0.25, 0.02], \text{   and   } z_3 \in [2.6, 3.82] \)
\item \( z_1 \in [-2.3, -1.8], \text{   }  z_2 \in [-1.74, -1.03], \text{   and   } z_3 \in [1.55, 1.94] \)

\end{enumerate} }
\litem{
Factor the polynomial below completely, knowing that $x-2$ is a factor. Then, choose the intervals the zeros of the polynomial belong to, where $z_1 \leq z_2 \leq z_3 \leq z_4$. \textit{To make the problem easier, all zeros are between -5 and 5.}\[ f(x) = 8x^{4} +18 x^{3} -75 x^{2} -46 x + 120 \]\begin{enumerate}[label=\Alph*.]
\item \( z_1 \in [-3, 2], \text{   }  z_2 \in [-1.26, -1.24], z_3 \in [1.34, 1.54], \text{   and   } z_4 \in [3.3, 5.1] \)
\item \( z_1 \in [-4, -3], \text{   }  z_2 \in [-1.5, -1.48], z_3 \in [1.16, 1.4], \text{   and   } z_4 \in [1.1, 3.9] \)
\item \( z_1 \in [-3, 2], \text{   }  z_2 \in [-0.95, -0.78], z_3 \in [0.62, 0.76], \text{   and   } z_4 \in [3.3, 5.1] \)
\item \( z_1 \in [-3, 2], \text{   }  z_2 \in [-0.63, -0.57], z_3 \in [2.75, 3.25], \text{   and   } z_4 \in [3.3, 5.1] \)
\item \( z_1 \in [-4, -3], \text{   }  z_2 \in [-0.74, -0.65], z_3 \in [0.68, 0.81], \text{   and   } z_4 \in [1.1, 3.9] \)

\end{enumerate} }
\litem{
Factor the polynomial below completely, knowing that $x-2$ is a factor. Then, choose the intervals the zeros of the polynomial belong to, where $z_1 \leq z_2 \leq z_3 \leq z_4$. \textit{To make the problem easier, all zeros are between -5 and 5.}\[ f(x) = 10x^{4} -71 x^{3} +174 x^{2} -171 x + 54 \]\begin{enumerate}[label=\Alph*.]
\item \( z_1 \in [0.65, 0.69], \text{   }  z_2 \in [1.66, 1.97], z_3 \in [1.59, 2.31], \text{   and   } z_4 \in [2.92, 3.18] \)
\item \( z_1 \in [0.53, 0.63], \text{   }  z_2 \in [1.16, 1.61], z_3 \in [1.59, 2.31], \text{   and   } z_4 \in [2.92, 3.18] \)
\item \( z_1 \in [-3.11, -3], \text{   }  z_2 \in [-2, -1.87], z_3 \in [-1.86, -1.65], \text{   and   } z_4 \in [-0.71, -0.66] \)
\item \( z_1 \in [-3.11, -3], \text{   }  z_2 \in [-2, -1.87], z_3 \in [-1.54, -1.11], \text{   and   } z_4 \in [-0.6, -0.49] \)
\item \( z_1 \in [-3.11, -3], \text{   }  z_2 \in [-3.1, -2.85], z_3 \in [-2.21, -1.79], \text{   and   } z_4 \in [-0.32, -0.21] \)

\end{enumerate} }
\end{enumerate}

\end{document}