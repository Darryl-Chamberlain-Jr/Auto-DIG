\documentclass[14pt]{extbook}
\usepackage{multicol, enumerate, enumitem, hyperref, color, soul, setspace, parskip, fancyhdr} %General Packages
\usepackage{amssymb, amsthm, amsmath, latexsym, units, mathtools} %Math Packages
\everymath{\displaystyle} %All math in Display Style
% Packages with additional options
\usepackage[headsep=0.5cm,headheight=12pt, left=1 in,right= 1 in,top= 1 in,bottom= 1 in]{geometry}
\usepackage[usenames,dvipsnames]{xcolor}
\usepackage{dashrule}  % Package to use the command below to create lines between items
\newcommand{\litem}[1]{\item#1\hspace*{-1cm}\rule{\textwidth}{0.4pt}}
\pagestyle{fancy}
\lhead{Progress Quiz 6}
\chead{}
\rhead{Version C}
\lfoot{9689-6866}
\cfoot{}
\rfoot{Spring 2021}
\begin{document}

\begin{enumerate}
\litem{
Perform the division below. Then, find the intervals that correspond to the quotient in the form $ax^2+bx+c$ and remainder $r$.\[ \frac{6x^{3} +12 x^{2} -78 x + 65}{x + 5} \]\begin{enumerate}[label=\Alph*.]
\item \( a \in [5, 8], \text{   } b \in [39, 48], \text{   } c \in [129, 136], \text{   and   } r \in [725, 729]. \)
\item \( a \in [5, 8], \text{   } b \in [-18, -15], \text{   } c \in [8, 14], \text{   and   } r \in [4, 8]. \)
\item \( a \in [-36, -29], \text{   } b \in [158, 163], \text{   } c \in [-890, -883], \text{   and   } r \in [4503, 4507]. \)
\item \( a \in [5, 8], \text{   } b \in [-28, -19], \text{   } c \in [61, 70], \text{   and   } r \in [-335, -324]. \)
\item \( a \in [-36, -29], \text{   } b \in [-140, -132], \text{   } c \in [-775, -767], \text{   and   } r \in [-3775, -3773]. \)

\end{enumerate} }
\litem{
Factor the polynomial below completely. Then, choose the intervals the zeros of the polynomial belong to, where $z_1 \leq z_2 \leq z_3$. \textit{To make the problem easier, all zeros are between -5 and 5.}\[ f(x) = 6x^{3} +19 x^{2} -65 x -50 \]\begin{enumerate}[label=\Alph*.]
\item \( z_1 \in [-2.8, -1.1], \text{   }  z_2 \in [0.01, 1.17], \text{   and   } z_3 \in [5, 7] \)
\item \( z_1 \in [-5.7, -4.5], \text{   }  z_2 \in [-0.94, -0.62], \text{   and   } z_3 \in [2.5, 4.5] \)
\item \( z_1 \in [-5.7, -4.5], \text{   }  z_2 \in [-1.56, -1.42], \text{   and   } z_3 \in [0.4, 1.4] \)
\item \( z_1 \in [-0.5, 0.4], \text{   }  z_2 \in [1.08, 1.78], \text{   and   } z_3 \in [5, 7] \)
\item \( z_1 \in [-1.4, -0.6], \text{   }  z_2 \in [1.6, 2.33], \text{   and   } z_3 \in [5, 7] \)

\end{enumerate} }
\litem{
Factor the polynomial below completely, knowing that $x+5$ is a factor. Then, choose the intervals the zeros of the polynomial belong to, where $z_1 \leq z_2 \leq z_3 \leq z_4$. \textit{To make the problem easier, all zeros are between -5 and 5.}\[ f(x) = 8x^{4} +54 x^{3} +15 x^{2} -350 x -375 \]\begin{enumerate}[label=\Alph*.]
\item \( z_1 \in [-3, -1.9], \text{   }  z_2 \in [1.09, 1.55], z_3 \in [2.85, 3.02], \text{   and   } z_4 \in [4.3, 5.3] \)
\item \( z_1 \in [-5.9, -3.4], \text{   }  z_2 \in [0.51, 0.66], z_3 \in [2.85, 3.02], \text{   and   } z_4 \in [4.3, 5.3] \)
\item \( z_1 \in [-5.9, -3.4], \text{   }  z_2 \in [-3.17, -2.87], z_3 \in [-1.05, -0.67], \text{   and   } z_4 \in [-0.7, 0.5] \)
\item \( z_1 \in [-5.9, -3.4], \text{   }  z_2 \in [-3.17, -2.87], z_3 \in [-1.67, -1.07], \text{   and   } z_4 \in [2.1, 4.2] \)
\item \( z_1 \in [-0.8, -0.3], \text{   }  z_2 \in [0.77, 0.92], z_3 \in [2.85, 3.02], \text{   and   } z_4 \in [4.3, 5.3] \)

\end{enumerate} }
\litem{
Factor the polynomial below completely, knowing that $x+5$ is a factor. Then, choose the intervals the zeros of the polynomial belong to, where $z_1 \leq z_2 \leq z_3 \leq z_4$. \textit{To make the problem easier, all zeros are between -5 and 5.}\[ f(x) = 6x^{4} +35 x^{3} -9 x^{2} -210 x -200 \]\begin{enumerate}[label=\Alph*.]
\item \( z_1 \in [-2.91, -2.04], \text{   }  z_2 \in [1.1, 1.43], z_3 \in [1.76, 2.49], \text{   and   } z_4 \in [4.7, 5.1] \)
\item \( z_1 \in [-5.06, -4.82], \text{   }  z_2 \in [-2.04, -1.88], z_3 \in [-1.2, -0.26], \text{   and   } z_4 \in [0, 1.4] \)
\item \( z_1 \in [-0.93, -0.54], \text{   }  z_2 \in [1.78, 2.01], z_3 \in [3.71, 4.91], \text{   and   } z_4 \in [4.7, 5.1] \)
\item \( z_1 \in [-0.41, 0], \text{   }  z_2 \in [0.58, 1.14], z_3 \in [1.76, 2.49], \text{   and   } z_4 \in [4.7, 5.1] \)
\item \( z_1 \in [-5.06, -4.82], \text{   }  z_2 \in [-2.04, -1.88], z_3 \in [-1.43, -0.99], \text{   and   } z_4 \in [2.2, 4.7] \)

\end{enumerate} }
\litem{
What are the \textit{possible Integer} roots of the polynomial below?\[ f(x) = 3x^{4} +7 x^{3} +2 x^{2} +5 x + 7 \]\begin{enumerate}[label=\Alph*.]
\item \( \text{ All combinations of: }\frac{\pm 1,\pm 7}{\pm 1,\pm 3} \)
\item \( \text{ All combinations of: }\frac{\pm 1,\pm 3}{\pm 1,\pm 7} \)
\item \( \pm 1,\pm 7 \)
\item \( \pm 1,\pm 3 \)
\item \( \text{There is no formula or theorem that tells us all possible Integer roots.} \)

\end{enumerate} }
\litem{
Factor the polynomial below completely. Then, choose the intervals the zeros of the polynomial belong to, where $z_1 \leq z_2 \leq z_3$. \textit{To make the problem easier, all zeros are between -5 and 5.}\[ f(x) = 4x^{3} +4 x^{2} -23 x -30 \]\begin{enumerate}[label=\Alph*.]
\item \( z_1 \in [-5.03, -4.83], \text{   }  z_2 \in [0.73, 0.83], \text{   and   } z_3 \in [1.36, 2.23] \)
\item \( z_1 \in [-2.6, -2.36], \text{   }  z_2 \in [1.48, 1.51], \text{   and   } z_3 \in [1.36, 2.23] \)
\item \( z_1 \in [-2.45, -1.97], \text{   }  z_2 \in [-1.58, -1.45], \text{   and   } z_3 \in [2.1, 2.57] \)
\item \( z_1 \in [-0.72, -0.31], \text{   }  z_2 \in [0.66, 0.68], \text{   and   } z_3 \in [1.36, 2.23] \)
\item \( z_1 \in [-2.45, -1.97], \text{   }  z_2 \in [-0.77, -0.56], \text{   and   } z_3 \in [0.32, 0.7] \)

\end{enumerate} }
\litem{
Perform the division below. Then, find the intervals that correspond to the quotient in the form $ax^2+bx+c$ and remainder $r$.\[ \frac{8x^{3} +44 x^{2} +16 x -25}{x + 5} \]\begin{enumerate}[label=\Alph*.]
\item \( a \in [-48, -39], \text{   } b \in [238, 247], \text{   } c \in [-1209, -1200], \text{   and   } r \in [5995, 6001]. \)
\item \( a \in [3, 13], \text{   } b \in [-8, 1], \text{   } c \in [32, 41], \text{   and   } r \in [-267, -261]. \)
\item \( a \in [3, 13], \text{   } b \in [81, 85], \text{   } c \in [435, 440], \text{   and   } r \in [2152, 2163]. \)
\item \( a \in [-48, -39], \text{   } b \in [-160, -153], \text{   } c \in [-767, -762], \text{   and   } r \in [-3847, -3841]. \)
\item \( a \in [3, 13], \text{   } b \in [4, 8], \text{   } c \in [-7, 1], \text{   and   } r \in [-7, -4]. \)

\end{enumerate} }
\litem{
What are the \textit{possible Rational} roots of the polynomial below?\[ f(x) = 4x^{3} +5 x^{2} +4 x + 7 \]\begin{enumerate}[label=\Alph*.]
\item \( \text{ All combinations of: }\frac{\pm 1,\pm 2,\pm 4}{\pm 1,\pm 7} \)
\item \( \text{ All combinations of: }\frac{\pm 1,\pm 7}{\pm 1,\pm 2,\pm 4} \)
\item \( \pm 1,\pm 2,\pm 4 \)
\item \( \pm 1,\pm 7 \)
\item \( \text{ There is no formula or theorem that tells us all possible Rational roots.} \)

\end{enumerate} }
\litem{
Perform the division below. Then, find the intervals that correspond to the quotient in the form $ax^2+bx+c$ and remainder $r$.\[ \frac{12x^{3} -28 x^{2} + 14}{x -2} \]\begin{enumerate}[label=\Alph*.]
\item \( a \in [8, 16], b \in [-7, -2], c \in [-9, -4], \text{ and } r \in [-7, 4]. \)
\item \( a \in [17, 30], b \in [-77, -73], c \in [152, 155], \text{ and } r \in [-296, -289]. \)
\item \( a \in [8, 16], b \in [-52, -50], c \in [102, 111], \text{ and } r \in [-197, -192]. \)
\item \( a \in [8, 16], b \in [-22, -11], c \in [-20, -15], \text{ and } r \in [-7, 4]. \)
\item \( a \in [17, 30], b \in [15, 23], c \in [38, 42], \text{ and } r \in [91, 99]. \)

\end{enumerate} }
\litem{
Perform the division below. Then, find the intervals that correspond to the quotient in the form $ax^2+bx+c$ and remainder $r$.\[ \frac{9x^{3} +21 x^{2} -8}{x + 2} \]\begin{enumerate}[label=\Alph*.]
\item \( a \in [6, 11], b \in [36, 46], c \in [75, 79], \text{ and } r \in [143, 151]. \)
\item \( a \in [-24, -11], b \in [54, 63], c \in [-115, -110], \text{ and } r \in [218, 226]. \)
\item \( a \in [6, 11], b \in [1, 5], c \in [-9, -2], \text{ and } r \in [0, 6]. \)
\item \( a \in [-24, -11], b \in [-17, -13], c \in [-32, -28], \text{ and } r \in [-70, -66]. \)
\item \( a \in [6, 11], b \in [-6, 1], c \in [14, 24], \text{ and } r \in [-63, -61]. \)

\end{enumerate} }
\end{enumerate}

\end{document}