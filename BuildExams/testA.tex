\documentclass[14pt]{extbook}
\usepackage{multicol, enumerate, enumitem, hyperref, color, soul, setspace, parskip, fancyhdr} %General Packages
\usepackage{amssymb, amsthm, amsmath, bbm, latexsym, units, mathtools} %Math Packages
\everymath{\displaystyle} %All math in Display Style
% Packages with additional options
\usepackage[headsep=0.5cm,headheight=12pt, left=1 in,right= 1 in,top= 1 in,bottom= 1 in]{geometry}
\usepackage[usenames,dvipsnames]{xcolor}
\usepackage{dashrule}  % Package to use the command below to create lines between items
\newcommand{\litem}[1]{\item#1\hspace*{-1cm}\rule{\textwidth}{0.4pt}}
\pagestyle{fancy}
\lhead{test}
\chead{}
\rhead{Version A}
\lfoot{1466-9383}
\cfoot{}
\rfoot{testing}
\begin{document}

\begin{enumerate}
\litem{
First, find the equation of the line containing the two points below. Then, write the equation as $ y=mx+b $ and choose the intervals that contain $m$ and $b$.\[ (-5, -10) \text{ and } (-8, 5) \]\begin{enumerate}[label=\Alph*.]
\item \( m \in [-8, -4] \hspace*{3mm} b \in [32, 37] \)
\item \( m \in [-8, -4] \hspace*{3mm} b \in [7, 20] \)
\item \( m \in [-8, -4] \hspace*{3mm} b \in [-6, -3] \)
\item \( m \in [2, 7] \hspace*{3mm} b \in [42, 46] \)
\item \( m \in [-8, -4] \hspace*{3mm} b \in [-35, -29] \)

\end{enumerate} }
\end{enumerate}

\end{document}